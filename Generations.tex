%\documentclass[10pt,french,oneside]{book}
\documentclass[11pt,french]{book}

\usepackage[a4paper,top=1in,inner=1in,ratio={2:3},includehead,nofoot]{geometry}
%\usepackage[a4paper,ignorehead,bottom=1.42857in]{geometry}
%\usepackage[pass]{geometry}

%\usepackage[utf8x]{inputenc}

\usepackage[T1]{fontenc}


%\usepackage{fontspec}
\usepackage{xltxtra}
\defaultfontfeatures{Numbers=OldStyle,Ligatures=Rare,Mapping=tex-text}
%\setmainfont{Adobe Garamond Pro}
\setmainfont{Garamond Premier Pro}


%%%%%%% Ce qui suit permet d'avoir le nom des parties en en-tête avec 
\usepackage{etoolbox}
%\makeatletter
%%%% patch \@part[#1]{#2} and \@spart (see the class file) to save the part name
%\pretocmd{\@part}{\gdef\parttitle{#1}}{}{}
%\pretocmd{\@spart}{\gdef\parttitle{#1}}{}{}
%\makeatother
%
\makeatletter
\patchcmd{\@part}% <cmd>
  {\markboth{}{}}% <search>
  {\partmark{#1}}% <replace>
  {}{}% <success><failure>
\makeatother



\makeatletter
%%% Numéroter les chapitres par partie %%% remplacé par package chngcntr ci-dessous
%\@addtoreset{chapter}{part}
%%% Vider le style des pages de parties et de chapitres
\let\ps@plain=\ps@empty
\makeatother


\usepackage{chngcntr}
\counterwithin*{chapter}{part}
%%% Le package hyperref a besoin de différencier les chapitres (dont le compteur est réinitialisé aux parties)
\newcommand{\theHchapter}{\thepart\arabic{chapter}}
%%% Pour l'instant, ma commande vrefnum (ci-dessous) ne fonctionne qu'avec des captions
%%% numérotées en format simple, donc un compteur de table indépendant pour tout le document
\counterwithout{table}{chapter}


\usepackage{varioref}
%%%%%% Ce qui suit ajoute la commande vrefnum au package varioref
\makeatletter
\def\vref@num#1#2{%
  \expandafter\expandafter\expandafter\vref@@num
  \csname r@#2\endcsname{}{}\@nil#1%
}
\def\vref@@num#1#2#3\@nil#4{%
  \def#4{#1}%
}
\def\vrefnum#1#2{%
  \vref@num{#1}{#2}%
}
\makeatother


\usepackage{array}

\usepackage{layout}

\usepackage[normalem]{ulem}

%\usepackage{slantsc}

\usepackage{enumitem}

%\usepackage[pagestyles]{titlesec}
%%%%%% titlesec pose des problèmes avec la bibliographie... En remplacement :
\usepackage{fancyhdr}
\pagestyle{fancy}

\newcommand{\partmark}[1]{\markboth{\scshape #1}{}}
\renewcommand{\chaptermark}[1]{\markright{\scshape #1}}
\renewcommand{\sectionmark}[1]{}

\fancyhf{}
\fancyhead[LE,RO]{\thepage}
\fancyhead[RE]{\leftmark}
\fancyhead[LO]{\rightmark}

%[\thepage][][\scshape\parttitle]{\scshape\chaptertitle}{}{\thepage}

\usepackage{numprint}

%\usepackage{quotchap}

\usepackage[latin,english,main=french]{babel}
\frenchbsetup{og=«,fg=»}

\usepackage[rightmargin=0pt,leftmargin=\parindent]{quoting}
\usepackage[autostyle]{csquotes}
\SetBlockEnvironment{quoting}

\usepackage{hyperref}
\hypersetup{colorlinks=true, linkcolor=blue, urlcolor=blue}

\usepackage[backend=biber, bibstyle=verbose, citestyle=verbose-trad1]{biblatex}
\renewcommand{\newunitpunct}[0]{\addcomma\addspace}
\bibliography{genehist.bib}



% Suppression de l'affichage des numéros de sectionnement en dessous des chapitres :
\setcounter{secnumdepth}{0}

%%% Un peu d'air autour des captions...
\addtolength{\abovecaptionskip}{5pt}
\addtolength{\belowcaptionskip}{10pt}

%%% Notes temporaires invisibles
%\newcommand{\tempnote}[1]{}
%%% Notes temporaires visibles
\makeatletter
\newcommand{\tempnote}[1]%
{\marginpar{\ifodd\c@page\raggedright\else\raggedleft\fi\tiny\fontspec[Ligatures=Rare,Variant=8]{Zapfino}\selectfont #1}}
\makeatother

%%% Activer le soulignement ondulé rouge
\makeatletter\newcommand\tempuwave{\bgroup\markoverwith{\textcolor{red}{\lower3.5\p@\hbox{\sixly \char58}}}\ULon}\makeatother
%%% Désactiver le soulignement ondulé rouge
%\newcommand\tempuwave{}

%%% Activer le passage à la page suivante (temporaire)
\newcommand\temppagebreak{\pagebreak}
%%% Désactiver le passage à la page suivante (temporaire)
%\newcommand\temppagebreak{}


%%% \fsc comme avec efrench : interdit la coupure des noms, même aux espaces, mais ici l'autorise aux tirets
%%% \bsc vient de babel-french (petites majuscules, et insécables : dans une boîte)... sinon :
% \newcommand*{\bsc}[1]{\leavevmode\hbox{\scshape #1}}
\makeatletter
%
\def\@fscespace[#1 #2]%
{%
 \MakeUppercase{\@car#1\@nil}%
 \MakeLowercase{\@cdr#1\@nil}%
 \if\relax\detokenize{#2}\relax%
 \else%
  ~\@fscespace[#2]%
 \fi%
}
%
\newcommand\fsc[1]{\@fsctiret[#1-]}
\def\@fsctiret[#1-#2]%
{%
 \bsc%
 {%
  \@fscespace[#1 ]%
  \if\relax\detokenize{#2}\relax%
  \else%
   --\@fsctiret[#2]%
  \fi%
 }%
}
%
\makeatother


% D'après l'Imprimerie Nationale, les siècles s'écrivent avec des chiffres romains
% en petites capitales...
\newcommand*{\crm}[1]{\textsc{\romannumeral #1}}

\providecommand{\ier}{\textsuperscript{\lowercase{er}}}
\providecommand{\ieme}{\textsuperscript{\lowercase{e}}}
\providecommand{\iemes}{\textsuperscript{\lowercase{es}}}

\newcommand*{\crmieme}[1]{%
\crm{#1}%
\ifnum#1=1%
    \ier%
\else%
    \ieme%
\fi%
}

\newcommand*{\siecle}[1]{%
\crmieme{#1}~siècle%
}

\newcommand*{\siecles}[2]{%
\crm{#1}--\crm{#2}\iemes\ siècles%
}


\newcommand\vrefbetterrange[2]%
{%
 \vrefnum\numrefun{#1}%
 \vrefnum\numrefdeux{#2}%
 \vrefpagenum\pagerefun{#1}%
 \vrefpagenum\pagerefdeux{#2}%
%\vref{#1} et \vref{#2}
 \ifnumcomp{\numrefun + 1}{=}{\numrefdeux}%
 {%
  \ifnumcomp{\pagerefun}{=}{\pagerefdeux}%
  {\ref{#1} et \ref{#2} \vpageref[]{#2}}%
  {\ref{#1} \vpageref[ci-dessus]{#1} et \ref{#2} \vpageref[ci-dessus]{#2}}%
 }%
 {%
  \ifnumcomp{\pagerefun}{=}{\pagerefdeux}%
  {\ref{#1} à \ref{#2} \vpageref[]{#2}}%
  {\ref{#1} \vpageref[ci-dessus]{#1} à \ref{#2} \vpageref[ci-dessus]{#2}}%
 }%
}


%%%%%% Créer commande \anglais{texte en anglais à mettre en emphase et règles de césure anglaises...}
%%%%%% Cf. en particulier Police_des_familles dans la marge de la troisième page
\newcommand\anglais[1]%
{%
\selectlanguage{english}\emph{#1}\selectlanguage{french}%
}

\newcommand\latin[1]%
{%
\selectlanguage{latin}\emph{#1}\selectlanguage{french}%
}


%\renewcommand{\theenumii}{\alph{enumii}}
\renewcommand{\labelenumii}{\theenumii.}

\title{PROSPECTIVE}
\author{Hervé \fsc{Tigréat}}


\begin{document}

%%%%%% Si titlesec
%%%%%% définition du style "livre", merci etoolbox pour \parttitle
%\newpagestyle{livre}{\sethead[\thepage][][\scshape\parttitle]{\scshape\chaptertitle}{}{\thepage}}
%%%\newpagestyle{livre}{\sethead[\thepage][][\scshape\chaptertitle]{\scshape\sectiontitle}{}{\thepage}}
%\pagestyle{livre}

\maketitle


\frontmatter



%\sloppy



”\chapter{Présentation}









Depuis les années soixante du vingtième siècle la reproduction des humains est en Europe soumise à de tels bouleversements qu'il ne s'agit plus d'une évolution, mais bien d'une révolution. Pour le moment les changements en cours dans le droit et dans les mœurs ne sont pas parvenus à leurs termes et le présent est instable, déroutant et difficile à penser. C'est là qu'un point de vue extérieur est utile pour se décentrer et comprendre un peu mieux où l'on en est. Un tel point de vue peut être fourni par les observations des ethnologues et sociologues\footnote{Cf. \emph{les Métamorphoses de la parenté} de l'ethnologue Maurice GODELIER (2004), ou bien \emph{L'origine des systèmes familiaux, T. I} (2011) ou encore \emph{Où en sommes-nous ? Une esquisse de l'histoire humaine} (2017) d'Emmanuel TODD, qui parcourt à la fois l'espace et le temps.} mais il peut aussi être trouvé dans l'histoire. La situation présente de la reproduction ne prend en effet tout son sens que par ses écarts avec les pratiques des siècles antérieurs.

La famille traditionnelle occidentale a fonctionné en Europe du haut moyen-âge au milieu du XXème siècle. Elle est née d'une synthèse entre les pratiques des romains de l'Empire, celles des juifs et celles des chrétiens de l'Antiquité (ces pratiques et les représentations qui les sous-tendaient étaient elles-mêmes le point d'aboutissement d'autant d'évolutions particulières). Les bases juridiques de la famille « traditionnelle » européenne ont été promulguées sous le règne de l'empereur Constantin et celui de ses successeurs directs mais elles ont mis de nombreux siècles à s'imposer, non sans résistances ni déformations multiples par rapport aux desseins initiaux. La trajectoire de cette forme de famille n'a atteint son apogée qu'aux derniers siècles de l'Ancien Régime. Depuis lors elle a fonctionné de manière presque hégémonique, demeurant le modèle de référence jusqu'au baby-boom, en dépit de quelques modifications significatives. Si à partir des années soixante du vingtième siècle ses fondations juridiques ont été presque totalement dynamitées, elle ne s'efface pourtant pas sans résistances, et pour l'instant elle n'est tout à fait morte ni dans les têtes ni dans les comportements. Mais dans le même temps de nouvelles formes d'union et de parentalité ont fait leur apparition et de très anciennes problématiques que l'on croyait définitivement obsolètes reviennent au devant des préoccupations. Il n'est pas question pour moi de prétendre concurrencer les historiens professionnels sur leur terrain et cet essai est fondé sur leurs écrits.  Si ce texte parvient à exposer clairement la situation où en est aujourd'hui la reproduction humaine et à la problématiser, alors il aura atteint son but.  
 
 La famille occidentale est si "traditionnelle" qu'elle nous paraissait parfaitement \emph{naturelle}, c'est pourquoi pour introduire mon sujet je vais commencer par me tourner vers les sociétés primitives. Dans son \emph{Anthropologie de l’esclavage}, (1986) Quentin MEILLASSOUX montre que les membres des sociétés primitives se sentent\footnote{…pour ceux d'aujourd'hui que la civilisation n'a pas encore arrachés à leur monde, mais les ethnologues du passé ont fait les mêmes observations.} liés par une continuité organique avec leur territoire et avec l'univers matériel dans son ensemble, avec les esprits dans (ou de) la nature, avec leurs ancêtres, avec le monde du ou des dieux. Souvent ils se pensent comme les seuls humains dignes de ce nom : chacun dans sa langue, ils se désignent eux-mêmes comme « les humains par excellence », ce qui implique que les autres, ceux qui leur sont étrangers, ne sont pas humains, ou pas vraiment humains, ou pas au même degré qu’eux. Pour eux la bonne vie, la seule vie vivable, n'est possible que sur le territoire dont ils ont hérité (même quand ils pratiquent le nomadisme c'est dans des bornes relativement limitées). Partout ailleurs c'est l'inconnu, l'étrange, l'étranger, le non humain, l'inhumain.

Le plus souvent ces sociétés ne connaissent aucune forme d'écriture. Elles se caractérisent d'abord par l'absence d'échanges marchands et de moyens de paiement, comme par la faiblesse ou l'absence de leurs structures étatiques. Elles peuvent se choisir plutôt démocratiquement un chef, mais son autorité est limitée. 

Dans ces sociétés la famille est le cadre essentiel, et parfois le cadre unique des rapports entre individus. Le chef de famille, presque toujours un homme, a pour première tâche de veiller à la pérennité de son groupe familial. La vie de chacun appartient au groupe, et il n'est pas question d'opposer à celui-ci les droits d'un individu particulier ni de mettre l'ensemble en danger pour un seul de ses membres. Si trop de naissances mettent en danger l'intérêt collectif le don des nouveaux nés excédentaires, leur abandon ou leur infanticide sont des pratiques ordinaires. En cas de disette il arrive que des vieillards se laissent mourir pour que les jeunes survivent, ou qu'on les y pousse.

La parenté assigne à chacun une fonction précise : des obligations mais aussi des droits sur les ressources du groupe. Celui-ci vit d'une économie de subsistance tournée vers l'auto-consommation. L'accumulation des biens n'est pas pensée comme la constitution d'un capital susceptible d'être réinvesti dans des opérations économiques nouvelles. Il s'agit plutôt d'acquérir des objets à haute valeur symbolique (religieuse, esthétique, magique, etc.), ou de constituer des réserves destinées à être consommées de manière festive ou/et ostentatoire. La nature lui \emph{donne} maternellement ses fruits et c'est la fécondité de son territoire qui limite la récolte et non le nombre de bras ou celui des heures de travail disponibles. Aucun membre de la famille n'est exclu des redistributions, mais les parts peuvent être très inégalement distribuées, sans tenir compte de la contribution de chaque membre du groupe au volume des biens à répartir, ni de ses besoins réels, mais plutôt de son rang et de sa place symbolique. Il est fréquent qu’à ce compte les femmes soient mal loties, mais ce n’est pas systématique. La règle de base est que les adultes travaillent pour nourrir les plus jeunes et les plus vieux. Les plus jeunes reçoivent plus qu'ils ne donnent, jusqu'à ce qu'ils soient à leur tour capables de nourrir tous ceux qui les ont nourris. Les plus vieux sont directement ou indirectement nourris par ceux qu'ils ont élevés : chacun investit dans une descendance pour préparer ses vieux jours. Quand tout se passe harmonieusement c'est au fil d'une vie entière que les tâches et les droits s'équilibrent pour chaque individu. 

Dans un tel système aucun garçon ne possède rien en propre : ni terres, ni troupeaux, etc. Si son clan refuse de lui procurer une femme, ou de lui donner les moyens d'en acquérir une, il reste bloqué dans un statut de dépendance juvénile. Condamné à travailler toute sa vie pour les enfants des autres, il n'accèdera jamais au statut avantageux et respecté de ceux qui ont de grands enfants productifs.

Ce devenir concerne moins les filles. Étant donné le taux de mortalité qui frappe les femmes enceintes, les parturientes et les enfants, les sociétés primitives ont le plus souvent besoin que chaque femme ait tous les enfants qu'elle peut porter. Les familles ont trop besoin d'enfants pour ne pas marier leurs filles dès lors qu'elles sont nubiles, éventuellement à des hommes bien plus âgés qu'elles et même déjà dotés d'autres femmes. Cette valeur qu'on leur accorde ne conduit pas à leur donner le pouvoir. Il est bien entendu qu'elles ne font pas des enfants pour elles seules, mais pour les partager ou les donner. 

Dans ces sociétés il n'y a pas de sens à faire une place à un étranger : à quel titre, au nom de quoi ? Et quel rôle lui donner ? Comment l'accueillir sans déséquilibrer le réseau compliqué et tendu des échanges et des obligations réciproques ? Du point de vue d'un guerrier l'étranger qu’il a capturé vivant est une preuve de sa valeur, mais il ne peut être un moyen d'entretenir et d'accroître sa puissance, un moyen de production de richesses (un esclave). Il n'est bon qu'à être rapidement consommé d'une façon ostentatoire : il n'est bon qu'à être sacrifié. Accepter de ses proches une rançon serait déjà entrer dans le monde marchand où une vie humaine a un coût et peut donc s'acheter ou s'échanger contre des biens réels, ce qui ne fait pas partie de leurs représentations.

Par contre s'il y a une place vacante dans une famille, celle-ci peut adopter un étranger ou une étrangère pour occuper cette place, afin que la vie continue, afin que les prestations masculines et féminines continuent d'être procurées, afin que les enfants continuent de naître, que les vieillards ne soient pas à l'abandon, que les ancêtres continuent d'être honorés, et que le monde continue sa course, etc. S'il n'y a pas assez d'épouses pour tous les garçons, on peut enlever des filles dans un autre groupe, ou leur en acheter. Un ennemi prisonnier peut d'autant plus facilement remplacer un mari ou un fils mort, que c'est ordinairement à ses voisins, à ceux que l'on pourrait épouser, qu'on fait la guerre.

En cas de conflit, de délit ou de crime, la mise au ban du groupe est d'autant plus fréquemment choisie qu'elle présente sur la mise à mort l'avantage d'éviter la souillure du territoire familial par un meurtre, ainsi que le ressentiment des ancêtres ou des dieux contre le ou les exécuteurs éventuels. Celui qui est condamné à l'exil est comme mort pour son groupe d'origine. S'il tombe aux mains d'un autre groupe, s'il est asservi (et a fortiori s'il leur était vendu) sa famille ne cherchera ni à le racheter ni à le délivrer\footnote{Ainsi le livre de la Genèse raconte comment Joseph, benjamin de Jacob, a été vendu par ses frères parce qu'ils étaient jaloux de voir qu'il était le préféré de son père. Leur première intention était de le tuer, mais comme une caravane de marchands passait par là cela leur a évité d'avoir à assumer la culpabilité de sa mort, et par dessus le marché la vente leur a rapporté de l'argent : l'Asie Mineure où se passe cette histoire était en partie entrée dans le monde marchand au moment où ce récit a été écrit.}. Si un individu qui a été banni est tué ses parents ne chercheront pas à le venger. Le jour où il mourra, les rites et sacrifices funéraires nécessaires au repos de son esprit ne seront pas exécutés. Il ne pourra pas rejoindre le monde de ses ancêtres et il ne sera pas rituellement nourri par les vivants. Son souvenir ne sera pas honoré. Cela l'exclura de son clan une deuxième fois. A ses yeux l'errance et l'exil dans un monde hostile \emph{cf. le sortde Cain après l'assassinat d'Abel} valent-ils mieux qu'une mort immédiate au milieu des siens, dans son pays ?

 A cette description Emmanuel TODD ajoute l'idée que l'organisation des familles primitives (l'organisation primitive des familles) est la même d'une extrémité de la terre à l'autre et se caractérise par des couples stables (même si les divorces sont possibles), de parents élevant eux-mêmes les enfants qu'ils ont conçus (même si les avortements, les infanticides et les abandons sont pratiqués à l'occasion) et respectant l'interdit de l'inceste. Selon lui le conjoint est choisi au sein du groupe de vie (au sens large) mais en dehors de la famille nucléaire. Les relations entre familles apparentées (frères et soeurs, beaux-frères et belles-soeurs) ont de l'importance étant donné le soutien mutuel qu'elles peuvent se fournir. Elles sont donc entretenues et les familles du père et de la mère ont autant d'importance l'une que l'autre. Les règles de succession sont souples et il n'y a pas de souci d'égalité stricte ni de principe de primogéniture. Il n'existe chez les primitifs aucune société réellement matriarcale mais le statut des femmes n'y est pas dévalorisé, même s'il peut y avoir à l'occasion de la polygamie. Selon TODD c'est ce type de famille qui prévalait en Europe de l'Ouest (Germanie, Gaule, Iles britanniques, Scandinavie...) avant l'entrée en scène des romains et les bouleversements de tous ordres qu'ils ont apportés, dont notamment une vision assez radicale du patriarcat. 


Des sociétés primitives à celles d'aujourd’hui l'histoire de la reproduction humaine est indissociable de celle de la prise en charge des personnes faibles, malades, âgées, infirmes ou démunies. En effet la famille, quelle que soit sa composition et son organisation, a toujours été la première institution d'assistance, quand elle n'était pas la seule\footnote{Pour aller plus loin on pourra se reporter à l'\emph{Histoire des enfants, des familles et des institutions d'assistance, La protection de l'enfance de l'antiquité à nos jours}, Hervé Tigréat, Pascale Planche et Jean-Luc Goascoz, préface de Pascal David, L'Harmattan, 2018.}.





Cet essai est à la disposition de tous pour un usage privé ou dans le cadre d'un enseignement. 

Usage commercial non autorisé. 

Tous droits de représentation et de reproduction réservés.

Copyright : libre de droits, mentionner l'auteur












\mainmatter

%\layout



%E1 Le tournant Constantinien 
%E2 Constantin et le droit des personnes 
%F1 entrée en scène des barbares 
%F2 Les sociétés du Bas-Empire et du Haut Moyen Âge 
%F3 l'esclavage chez les chrétiens de l'antiquité tardive et du haut moyen-âge 
%F4 clercs et religieux 
%F5 Le "mariage constantinien" 
%F6 Familles de chair 
%F7 Familles spirituelles 
%G1 Les familles de l'Ancien Régime entre autorités civiles et religieuses 
%G2 Les devoirs des pères de l'Ancien Régime 
%G3 Création d'une police des familles (XIVème-XVIIIème siècles)

\part[Les familles de l'ancien régime]{Les familles de l'ancien régime}

\chapter[Entre autorités civiles et religieuses]{Entre autorités civiles et religieuses}



\section{Le mariage constantinien}



Le « mariage constantinien » c'est le mariage des romains de l'antiquité tardive tel qu'il a été modifié par Constantin et ses successeurs pour l'accommoder aux conceptions chrétiennes. Il télescope plusieurs fonctions distinctes sur une seule personne : l'époux est à la fois le détenteur des droits juridiques de son épouse (son curateur), son amant (à qui on recommande la mesure), le géniteur de ses enfants, le détenteur des droits de ces enfants mineurs, et le responsable de leur éducation, c'est-à-dire leur père légal. Symétriquement, une épouse est la seule femme capable de donner à son époux des enfants légitimes, des héritiers, quel que soit le nombre de ses concubines. Pour être légitime chaque enfant doit être l'enfant biologique de ses parents légaux (leur enfant « naturel » au sens antique du terme). Et par définition seuls les enfants légitimes ont droit à une part d'héritage et à succéder à leurs parents.

Si le mariage constantinien, c'est le mariage tel que le conçoit l'église, à cela s'ajoute un élément essentiel, qui est que les autorités civiles traitent l'Eglise  comme l'une des sources du droit, ce qui donne à ses doctrines force de loi. 

Au terme d'une longue évolution commencée par Constantin,  à la fin du Moyen-Âge il n'y a plus de droit civil du mariage dans une Europe totalement christianisée où les seuls dissidents religieux reconnus sont les juifs. L'église a gagné suffisamment d'ascendant moral et de prestige intellectuel pour faire prévaloir une grande partie de sa doctrine. A partir de la réforme Grégorienne (XIème siècle) le mariage et la famille sont de son ressort exclusif. Les autorités civiles se soumettent, bon gré mal gré, à ses décisions : c'est elle qui reconnaît les mariages, qui ordonne les séparations, qui reconnaît les nullités de mariage, qui décide des interdits de mariage, et qui accorde des dispenses, etc...


Les religieux sont voués au célibat par choix : leurs voeux de pauvreté, d'obéissance à un supérieur et de chasteté sont incompatibles avec une vie de famille. Pour les pretres et les éveques, qui ne prononcent pas de voeux de chasteté, s'ils sont malgré tout interdits de mariage c'est pour les empecher d'avoir des fils légitimes, de leur transmettre leurs bénéfices (cure, éveché, etc...), et de créer des dynasties cléricales. 
 
En dehors de ces exceptions le célibat est licite, mais chez les jeunes gens sans enfants, en bonne santé et disposant de moyens matériels suffisants, il est suspect d'égoïsme, de libertinage ou de désirs « contraires à la nature » (homosexualité dont la mise en acte a toujours été condamnée moralement). Quels que soient les préférences individuelles la copulation n'est légitime que dans l'état de mariage monogame, qui est le seul moyen acceptable de répondre au \emph{"croissez et multipliez"} de la Genèse).  Comme la fin première du mariage est la procréation d'enfants légitimes les remariages sont déconseillés (quoique autorisés) quand cette fin n'est pas atteignable étant donné l'âge ou l'état de santé des conjoints. 
 
 

 \emph{C'est le mariage qui fonde la famille, et non la présence d'enfants}, même si l'Eglise met l'accueil des enfants au premier rang des « fins du mariage ». De son point de vue, le mariage crée  une parenté nouvelle, \emph{une seule chair}, entre les époux, \emph{qu'ils soient féconds ou non}. Cette parenté « par alliance » a des effets directs et immédiats sur les membres des parentèles des époux (frères, sœurs,~etc.) : elle étend le cercle des partenaires qui leur sont désormais définitivement interdits, même si l'un des époux décède.

 Selon la doctrine chrétienne, identique sur ce point au droit romain, ce sont les époux qui s'unissent l'un à l'autre : cela implique qu'ils soient capables de discernement (âge suffisant, santé mentale) et libres de leur personne : célibataires ou veufs, non esclaves, non engagés par contrat dans une entreprise qui empêcherait la vie commune, à l'abri de toute pression, libres de tout vœux religieux, sexuellement aptes au mariage.Contrairement au droit romain elle en est progressivement venue à ne reconnaître la réalité juridique d'un mariage que lorsqu'il a été consommé. 
 
 L'Église a toujours soutenu contre les parents que les jeunes gens ont le pouvoir de se marier validement sans leur accord, même si elle admettait qu'en leur désobéissant ces jeunes gens les déliaient de leur devoir de les établir dans la vie. 
 
 Contrairement au droit romain l'Eglise pose depuis sa fondation que le "oui" que se donnent les époux est irrévocable et elle enseigne que le mariage est indissoluble. Le divorce est interdit\footnote{...pour les chrétiens, mais pas pour les juifs soumis aux tribunaux rabbiniques.} quel que soit le motif. Seules sont possibles les actions en nullité de mariage, ou les actions en séparation (\emph{divortium}) avec interdiction de se remarier du vivant du conjoint. C'est sans doute l'un des points où les frictions entre clercs et laics ont été les plus vives (cf. Henri VIII, etc...).



 Les deux époux se doivent réciproquement fidélité. Même si les infidélités du mari ne sont pas sanctionnées par la loi, au contraire de celles de l'épouse, tout est fait pour qu'il n'ait aucun intérêt à entretenir des maîtresses. Cela ne lui interdit pas d'avoir des rapports avec des prostitué(e)s, rapports qui par nature ne s'inscrivent pas dans la durée et sont moins menaçants pour l'épouse. 
  
  Chacun des deux époux reconnaît à l'autre un droit sur son propre corps et a l'obligation de satisfaire ses désirs sexuels autant qu'il est en son pouvoir , ce qui veut dire que l'épouse doit accepter les étreintes de son mari, quoi qu'elle puisse penser des risques de grossesse et de santé à quoi cela l'expose, et quels que soient ses propres désirs. Ceci dit la modération est prêchée aux maris, qui se voient prescrire la continence de nombreux jours par an. Le \emph{devoir conjugal} n'est exempt de faute morale que si aucun obstacle n'est mis à la fécondation (éjaculation ailleurs que dans le vagin, pessaire, douche intime, ~etc.). 

Il n'est pas permis de se débarrasser des enfants non voulus par l'avortement ou par l'infanticide. Sauf indigence extrême il n'est pas non plus permis de s'en débarrasser par l'exposition ni la vente. Une femme n'a donc pas à craindre qu'on l'oblige contre son gré à avorter ou à abandonner son nouveau-né. Mais sa fécondité ne lui appartient pas plus qu'elle n'appartient à son mari, et pas plus que son mari elle n'a droit de vie ou de mort sur l'enfant qu'elle porte.


 Il n'est pas possible de répudier une épouse présumée stérile (en cas de stérilité dans un couple, c'est celle de la femme qui est toujours suspectée en premier) au motif de sa stérilité supposée. 
Les couples stériles, dont le nombre n'est pas du tout négligeable jusqu'à l'avènement de la médecine moderne, 20 % environ, et ceux dont aucun enfant n'ont atteint vivant l'âge adulte, sont invités à consacrer aux bonnes œuvres, aux pauvres et à l'Église les ressources qu'ils auraient transmises à leurs héritiers s'ils en avaient eus.

En cas de décès du père, c'est la mère qui, chez les Romains et à partir de 390, exerce la tutelle de ses enfants mineurs (et d'eux seuls) si elle a cinquante ans et plus, et du moins tant qu'elle ne se remarie pas, ce qui est le cas général des veuves dotées d'enfants\footnote{les femmes chargées d'enfants trouvent un compagnon beaucoup moins facilement que les autres. C'est encore vrai aujourd'hui.}. À partir de 390 une femme n'est donc plus considérée comme incapable par nature de représenter juridiquement une autre personne qu'elle-même. 

Les épouses savent qu'il est, sinon impossible, du moins difficile de les chasser de leur maison ou de leur imposer de cohabiter avec une concubine\footnote{... même si pour les hommes dont la puissance excède de beaucoup celle du commun des mortels, aristocrates, rois, la question peut se présenter différemment, et si les amours ancillaires sont de tous les temps.}% 
. Elles sont à peu près assurées que les infidélités de leur époux n'entraîneront de conséquences graves ni pour elles, ni pour leurs enfants, ni pour le futur héritage de ceux-ci. Tout au plus des « aliments » devront-ils être versés aux enfants nés des maîtresses de leurs maris, mais jusqu'au XXème siècle cela ne portera que sur d'assez petites sommes et seulement jusqu'à ce qu'ils soient mis au travail : 8-10 ans. Il n'est pas question de financer leur établissement dans la vie. 

 Tous les enfants nés hors mariage sont pénalisés. Même si la prise en charge d'\latin{alumnii} et leur installation dans l'existence est une bonne œuvre, il n'est guère possible pour un homme de se faire des héritiers sans se marier. La \emph{légitimation par mariage subséquent} est la seule exception de plein droit\footnote{... jusqu'au \siecle{20}. Les enfants irréguliers légitimés par les empereurs, les rois ou les papes, ne l'étaient pas de plein droit mais à la faveur d'une grâce, qui pouvait être refusée sans justification, et qui n'allait pas toujours sans contreparties coûteuses.} 
à la pénalisation des enfants nés hors mariage, et ses conditions sont strictes. Chacun, quelque puissant qu'il soit, doit savoir que s'il a l'imprudence de faire un enfant hors mariage ou dans un mariage contesté par son curé, par son évêque, par son seigneur, par le roi ou par sa propre parentèle, il ne pourra pas le faire reconnaître comme un de ses héritiers sans combat ou sans procès. Cet enfant ne pourra probablement pas lui succéder. L'exhérédation totale ou partielle des enfants illégitimes est restée jusqu'à la fin du \siecle{20} le premier frein apporté au désir des hommes de se procurer une descendance autrement qu'avec une femme épousée en bonne et due forme.

 Contrairement au droit romain de l'antiquité le concubinage n'est pas traité comme une union de second ordre, propre aux gens qui n'ont pas d'héritage à transmettre, mais comme un état de fornication durable. Il n'est possible à un homme de légitimer les enfants qu'il a obtenus d'une concubine qu'en l'épousant (\emph{mariage subséquent}), et à la condition de ne pas etre déjà marié. Et la perspective de se retrouver avec un enfant à charge, seule, sans aucun espoir d'une légitimation (ni même d'une aide significative venant du père de l'enfant lorsqu'il était déjà marié puisque aucune donation au-delà des frais d'éducation n'était autorisée depuis Constantin) a été un obstacle majeur à l'exercice d'une sexualité féminine en dehors du mariage ou avant le mariage. 


La première des méthode de limitation des naissances autorisées est la continence. Elle est fiable et donne des résultats immédiats. A plus long terme on peut y ajouter  le retard de l'âge au mariage, notamment celui des filles : si au lieu de se marier à 16 ans comme beaucoup de celles qui sont bien dotées elles attendent 24 ans en travaillant et en économisant sou à sou pour se constituer la petite dot qui leur permettra de se marier elles "gagnent" huit années sans grossesses\footnote{...soit 4 enfants environ en moyenne, sans contraception et avec allaitement.}. 

Pour diminuer le nombre de leurs descendants les parents peuvent aussi utiliser l'entrée en religion de certains de leurs enfants. Cela leur permet de concentrer leurs ressources sur l'établissement des autres. Meme s'il faut ordinairement une "dot" pour entrer dans les couvents et monastères son montant dépend de la notoriété de l'établissement, et il suffit de ne pas viser trop haut pour faire des économies sérieuses et mieux doter les filles qu'il convient de marier. 
Les voeux des religieux et religieuses sont reconnus par les autorités civiles. Le voeu de pauvreté les enlève définitivement tout droit à tout héritage. Le voeu d'obéissance à leur supérieur les délie de leur devoir d'obéissance à leurs parents. Il est possible d'etre relevé de ces voeux, mais cela demande une procédure lourde et aléatoire. Une fois un enfant placé au couvent il lui sera difficile de brouiller les stratégies familiales.  
   



 



Les laïcs n'ont jamais totalement épousé les points de vue des clercs et jusqu'à la Réforme Grégorienne (\siecle{11}), l'Église n'avait pas le monopole du droit familial. Les écarts entre le droit religieux (droit \emph{canon}) et les lois civiles n'ont jamais été nuls, pour ne pas parler de \emph{l'à peu près} avec lequel ces lois étaient respectées. Ce que j'appelle le mariage constantinien est donc un modèle qui n'a jamais été pleinement réalisé, et surtout pas sous Constantin. Pourtant il s'est peu à peu incarné dans les pratiques et les représentations, et il n'a peut-être jamais été aussi bien respecté que durant les derniers siècles de notre ancien régime.

Malgré sa rigueur, et même si elle n'a souvent été suivie que de manière plutôt lâche la morale familiale et sexuelle enseignée par l'Église est devenue la norme au fil des siècles . Si du \crmieme{11} au \siecle{16} des mouvements de contestation religieuse se succèdent, qui culmineront avec la réforme protestante, rares sont ceux qui à cette période mettent vraiment en question la morale familiale et sexuelle enseignée par l'Église. Au contraire les opposants s'appuient sur elle pour critiquer les écarts des clercs avec leurs propres principes. Cette morale a été formulée dès les premiers temps de l'Église, mais les règles de droit qui en découlent ont mis au moins un millénaire à s'imposer comme le droit commun.  Certaines régions s'y sont conformées avec beaucoup de rigueur tandis que d'autres ont été beaucoup plus tolérantes avec les irrégularités.  Le christianisme a certes contribué à façonner les sociétés d'ancien régime, mais il en a été lui-même fortement influencé. Il lui a été demandé de bénir, et même de sacraliser, leurs mécanismes et leurs logiques, et de les conforter dans leur fonctionnement, et c'est ce qu'il a souvent fait. 


 
 
 \section{traitement des naissances illégitimes}

 Les contraintes et limites imposées à la reproduction par le roi et par l'Église n'ont jamais été acceptées totalement ni par tous. C'était notamment le cas dans la noblesse. Depuis l'Antiquité tardive et durant tout le Moyen Âge elle était tenue, par elle-même et par les autres ordres de la société, pour une \emph{race} supérieure qui transmettait ses vertus par son sang. Cette antique croyance n'accordait pas d'importance au statut juridique ou religieux de l'union des parents. Elle coexistait sereinement dans les têtes avec le modèle canonique judéo-chrétien. 

 Les membres les plus puissants de la noblesse, et d'abord les rois eux-mêmes, n'ont jamais cessé de pratiquer une polygamie de fait qui leur donnait de nombreux enfants, de second rang certes, mais parfois bien utiles à défaut ou en complément d'enfants légitimes et valeureux. Jusqu'au \siecle{11} les différences faites dans les familles puissantes entre enfants légitimes et bâtards nés du chef de famille étaient faibles (capacité d'hériter, de succéder,~etc.). En effet le sang du père ennoblissait. Cette conception était un héritage des mœurs d'inspiration germanique du haut Moyen Âge. Le nombre des bâtards nobles semble même avoir crû au \siecle{15}. Par comparaison les bourgeois reconnaissaient beaucoup moins d'enfants illégitimes. De 1400 à 1649 les rois de France ont reconnu 24~\% d'enfants naturels tandis que les grands officiers, employés roturiers de la maison du roi, n'en avouaient que 10,3~\%. 

 Alors que le mariage des grands seigneurs ne répondait habituellement qu'à des critères politiques, les enfants qu'ils avaient conçus avec leurs maîtresses, \emph{enfants de l'amour}, étaient fréquemment \emph{réussis}. S'ils étaient légitimés, ces enfants pouvaient leur permettre des alliances profitables. Or les rois d'Europe, héritiers en cela aussi de l'empereur de Rome, pouvaient légitimer les « bâtards » par \emph{lettres royaux}. Les bâtards des familles aristocratiques ont donc souvent été légitimés par le roi ou par mariage subséquent. Même s'ils ne l'étaient pas, cela n'a pas fait problème pendant longtemps. Souvent ils n'ont été légitimés qu'après la mort de leur père, \latin{ad honores}, c'est-à-dire pour accéder aux honneurs qu'ils détenaient, c'est-à-dire pour leur succéder dans les emplois d'intérêt public, les charges qu'ils exerçaient. Par contre aucune reine, princesse du sang ou femme d'officier n'a pu légitimer d'enfant naturel autrement que par un mariage subséquent : c'est l'homme qui ennoblissait, c'est lui qui légitimait. 

 Les cas d'illégitimité susceptibles de bénéficier d'une légitimation par mariage subséquent avaient été étendus par les rois au-delà des critères de Constantin, de façon à inclure les enfants nés d'une relation passagère, et ceux conçus dans le cadre d'un enlèvement suivi d'un viol (enfants dont le géniteur n'était pas forcément le futur mari). Par ce biais la fiction retrouvait une place dans la filiation : l'adoption par l'époux de la mère redevenait possible. 

 Et pourtant les déclarations officielles de l'Église stigmatisaient tous les enfants illégitimes. Le Concile de Bourges (1031) confirmait les jugements des conciles antérieurs (« semence maudite »). Et l'Église continuait d'écarter de la prêtrise les fils de prêtres, sauf dispense (à vrai dire facilement accordée). Et elle était en cela d'accord avec le reste de la société. À partir du \siecle{11}, le mot « bâtard » devient un terme de mépris, une injure. Dès le \siecle{12}, avec la renaissance du droit romain, le bâtard n'appartient plus à aucune famille même dans les pays de droit coutumier\footnote{En gros les pays situés au nord de la Loire, opposés aux pays de droit (romain) écrit, situés au sud de la Loire.} 
: pas même celle de sa mère. 

 
 L'Hôpital du Saint Esprit de Paris était initialement un hospice destiné à toutes les personnes démunies. À la fin du Moyen Âge il était devenu l'orphelinat de Paris. Vers le milieu du \siecle{15} le roi lui a demandé de prendre en charge les enfants abandonnés de Paris. Tout roi qu'il fût ses demandes réitérées ont été récusées par les maîtres de l'un des hôpitaux de sa propre ville : \emph{"...en faisant prévaloir les statuts de fondation et la bonne réputation dont jouissent les enfants qu'il entretient et éduque déjà."} Si tous les orphelins d'origine inconnue lui étaient conduits, les gens de métier qui viennent chercher un apprenti, ou les jeunes compagnons qui y prennent femme ne seraient plus assurés de la légitimité, et partant de la moralité de l'adolescent (\fsc{SAUNIER}, \emph{Le « pauvre malade » dans le cadre hospitalier médiéval, France du Nord}, vers 1300-1500, 1993, p. 53).
 Réellement convaincu par ces arguments, ou bien de guerre lasse, le roi a confirmé les maîtres de l'Hôpital du Saint-Esprit dans l'idée que leurs statuts et eux-mêmes se faisaient de leur mission : en 1445 il a donc accepté qu'on n'y admette que les \emph{orphelins et orphelines nés en loyal mariage} et à qui on ne peut reprocher \emph{la tache de bâtardise}, car, selon un autre argument fourni par les maîtres de l'hôpital, \emph{"...ſy on y admettoit des baſtards, il ſeroit à craindre que la division ne ſe miſt bientôt dans cette maiſon par les reproches continuels que les enfants légitimes feroient aux baſtards".}
(\emph{Lettres patentes de Charles~VII du 7 août 1445, A.A.P. Saint-Esprit, L, II,} p. 32 ; cité par \fsc{SAUNIER}, id. p. 212).
 
 
 À cette époque les maîtres du Saint-Esprit faisaient remettre tous les enfants exposés qu'on leur présentait aux paroisses sur le territoire desquelles on les avait trouvés, alors qu'ils acceptaient de prendre en charge les adolescents légitimes (orphelins pauvres) qui sortaient convalescents de l'Hôtel-Dieu. Même les léproseries excluaient les bâtards \emph{"...parce que les gardes des maladreries diſaient que les bâtards n'avaient pas de lignage, ni n'étaient à hériter de nul droit par quoy ils ne ſe pouvaient aider de leur maiſon, pas plus qu'un étranger qui ſerait venu d'Eſpagne"} \fsc{SAUNIER}, id. p. 213.
Mais ces mêmes léproseries admettaient sans réserves les lépreux sans ressources s'ils étaient de naissance légitime : l'indigence leur faisait encore moins peur que l'illégitimité.



Être un « bâtard » était une tare, et semble avoir été de plus en plus pénalisant du Moyen Âge au \siecle{18} : est-ce pour des raisons religieuses ? ou plutôt parce que la société reposait sur l'alliance des familles, alliance que protégeait la mise hors jeu des enfants nés en dehors de ce cadre ? Dans une société où l'on n'était fils ou fille de quelqu'un que si l'on était né de deux parents légitimement mariés, un jeune de naissance illégitime ou né de parents inconnus (traité lui aussi de bâtard) portait une tare indélébile. Il était considéré comme fils de personne, hors parenté, même s'il vivait au foyer de l'un de ses deux parents (la mère en général). Un enfant non légitime ne pouvait ni succéder à un membre de sa parentèle dans un office, ni en hériter, sauf si aucun autre héritier légitime n'y trouvait ombrage. Personne ne se portait caution pour lui. Sans père il ne pouvait pas apporter sa contrepartie dans un système d'alliance. Il était condamné à une position marginale, du moins par rapport à celle de ses éventuels demi-frères et sœurs. En contrepartie de ces exclusions il n'était pas non plus contraint de se porter caution pour un parent. Il n'avait aucune autorisation parentale à demander pour convoler : ses géniteurs comme ses tuteurs ne pouvaient pas le lui interdire.

 

 Jusqu'à la fin de l'ancien régime les « bâtards » étaient exclus de nombreux métiers. En règle générale les corporations les refusaient, de la même façon et pour les mêmes raisons que la prêtrise leur était interdite. Maître Jacques \fsc{Ducros}, avocat au Parlement de Bordeaux, et premier Consul d'Agen en 1659, écrit dans ses \href{http://www.babordnum.fr/files/original/859d36685f2d7b2f871c648ea08bd103.pdf}{\emph{Réflexions singulières sur l'ancienne coutume d'Agen}}  :
%
\begin{displayquote}

{[...] \emph{les batards n'ont pas le bonheur de paſſer pour des domeſtiques%
% [11] 
\footnote{Ici « domestiques » signifie « appartenant à une maison », pas forcément comme salarié au service du maître. Cela inclut aussi tous les enfants et parents vivant dans la maison.} 
ny d'auoir rien en propre dans les maiſons. Ils ſont les productions du vice \& les enfans d'iniquité. Les peres les forment dans les tenebres \emph{[et]} les meres en cachent la conception. A meſme qu'ils sont nez , ces infames producteurs les deſauoüent. Les enfans legitimes cherchent le iour \& la lumiere, les illegitimes la nuit \& l'obſcurité. A proprement parler ce ſont des excremens, deſquels à meſme que la nature les chaſſe \& les pouſſe dehors, on couure d'ordure \& de ſaleté : ils n'ont ny nom ny race ny famille , c'eſt pourquoy ils ne peuuent eſtre admis au nombre des proches de ſang de conſanguinité ny d'allience}}%
%[12]
\footnote{\fsc{CAPUL}, Thèse, tome II, p. 111.%
\label{notecapul111}}%
.

\end{displayquote}


 \fsc{CAPUL} rapporte que lors des États généraux de 1614, le Tiers-état d'Agenais demande au roi : \enquote{\emph{que toutes lettres de légitimation ſeront deſnyees a tous enfens nez d'inceſte, d'adultère ou filz de prebſtres, et qu'on n'y aura aucun eſgard, ſoit pour ſucceſion, dignites, offices, bennefices (eccléſiaſtiques) et tous autres droitz}}%
% [13]
\footnote{%\fsc{CAPUL}, Thèse, tome II, p. 111.}%
Voir note \ref{notecapul111}.}%
. Les places désirables étaient trop peu nombreuses pour se montrer généreux. 
Les bourgeois prospères qui représentaient leurs concitoyens de l'Agenais, et qui exprimaient probablement l'opinion publique de leur époque, ne s'identifiaient en aucune façon aux enfants nés des unions sexuelles illicites, pourtant innocents des actes de leurs pères et mères. Ils ne toléraient pas que les « bâtards » soient confondus avec les enfants légitimes, et surtout pas avec les orphelins. Ces députés tenaient fermement à ce qu'aucun passe-droit ne puisse désavantager leurs propres fils dans la course aux honneurs, et leurs filles dans la chasse aux maris. C'était la défense la plus intransigeante de la morale conjugale qui servait leurs intérêts, puisqu'elle leur permettait d'écarter une partie des concurrents nobles ou bourgeois de leurs propres enfants.  

Ceci dit leur démarche auprès du roi  conforte l'idée que les interdits qui pesaient sur les « fruits du péché » pouvaient assez aisément être tournés avec de l'argent et/ou de l'entregent, mais cela ne concernait que les rares enfants illégitimes qui étaient investis par des parents puissants ou fortunés. Ainsi Erasme, (1469-1536), « prince des humanistes », âme de la « république des lettres » de son temps, était-il fils de prêtre. Fils d'un père cultivé il reçut une instruction soignée dans les écoles monastiques de son temps et entra en 1688 chez les chanoines de Saint-Augustin, où il fut ordonné prêtre en 1492. S'il ne fit pas une brillante carrière dans les allées du pouvoir temporel, comme évêque ou cardinal, c'est parce qu'il refusa les offres qu'on lui en fit au profit de la recherche intellectuelle et théologique, où il réussit il est vrai de manière exceptionnelle. Sa bâtardise et le statut ecclésiastique de son père ne semblent avoir fait problème à personne.

 
 
 

\section{Conflits de juridictions}

 Le droit romain n'a jamais totalement disparu dans les pays de droit écrit, au Sud de la Loire ou en Italie, mais à partir de sa redécouverte au \siecle{12} il a connu une nouvelle faveur en tant que modèle et outil de réflexion. La Renaissance a vu le triomphe du droit tel que les empereurs chrétiens l'ont mis en forme%
% [4]
\footnote{Justinien~I\ier{} (483 - 565) ou Justinien le Grand, empereur de Byzance de 527 à 565, essaya de restaurer l'unité de l'empire romain. Il a ordonné et dirigé une compilation du droit romain, le \latin{Corpus iuris civilis}, qui est l'une des bases du droit civil de divers pays européens.}% 
. Si bien qu'au bout de plus d'un millénaire, c'étaient encore les choix de Constantin et de ses successeurs immédiats qui modelaient en profondeur les mœurs familiales européennes : celles-ci n'ont jamais été aussi conformes à ses décrets qu'aux \crmieme{16} et \siecle{17}. À partir de la Renaissance et jusqu'au \siecle{20} les femmes \emph{mariées} ont été pratiquement réduites à un statut de mineures. Par rapport au Moyen Âge le retour en faveur du droit romain a appesanti l'autorité des pères sur les enfants, et contribué à enlever aux femmes, et surtout aux épouses, une part significative des libertés économiques et juridiques que le Moyen Âge leur avait reconnues.

 Du \crmieme{13} au \siecle{18} les autorités civiles reprennent peu à peu une grande partie du terrain qu'elles avaient concédé aux autorités religieuses au fil du haut Moyen Âge. Le \emph{Concordat de Bologne} (1516) accorde au roi de France le droit de nommer les titulaires des principaux bénéfices (évêques et abbés et abbesses), en dépit la règle qui depuis l'Antiquité voulait qu'ils soient désignés (élus) par leur communauté. 

 À partir du \siecle{14} un petit nombre de curés ont commencé de tenir des \emph{registres de catholicité} où ils enregistraient les baptêmes (autant dire les naissances dans un monde où tous sont baptisés) et parfois aussi les décès et les mariages. L'intérêt de ces registres était de faire foi dans les procès éventuels, même si manquaient les témoins capables de répondre à des questions concernant par exemple l'âge des personnes, ou leurs liens de parenté,~etc. En raison de cet intérêt quelques évêques ont ordonné à tous leurs curés d'en faire autant. \emph{L'Ordonnance de Villers-Cotterêts} (1539) a généralisé à tous les curés du royaume de France l'obligation d'enregistrer par écrit tous les baptêmes. \emph{L'Ordonnance de Blois} (1579) la complète en ordonnant que soient également notés sur ces registres tous les mariages et tous les décès, ce qui permettait de lutter contre les bigames. À cela s'ajoute l'obligation légale d'une publication des bans préalable au mariage, préconisée depuis longtemps par les conciles, mais appliquée de manière irrégulière, afin que ceux qui connaîtraient un empêchement au mariage projeté puissent le déclarer en temps utile. Par ces diverses initiatives et par d'autres les autorités civiles ont repris pied dans le domaine matrimonial. Le contrôle des unions importait en effet au moins autant aux rois et aux parents qu'à l'Église, et les mariages avaient un effet déterminant sur le bon fonctionnement de la société civile, sur la paix des familles et sur l'organisation économique. Les juges royaux ont cherché et trouvé, ou reçu du roi, des moyens de contester certaines des décisions des juges ecclésiastiques, si bien qu'on a pu observer un retour progressif du contentieux des mariages devant les tribunaux civils. 

 Le conflit le plus rude entre les autorités civiles et les autorités religieuses a porté sur la place à donner à l'autorité des parents sur les unions. Selon l'Église catholique les conjoints s'unissaient irrévocablement l'un à l'autre par leur « oui », devant le prêtre qui n'était qu'un témoin représentant l'Église, un témoin privilégié à partir du moment où il tenait les registres d'état civil (le curé de la paroisse de l'un ou l'autre des époux sauf dispense). La position traditionnelle de l'Église était que l'autorisation parentale n'était pas nécessaire pour la validité du mariage, même si elle déconseillait aux jeunes gens de s'en passer et si elle ne contestait pas aux parents le droit de déshériter les contrevenants. Les autorités civiles et les familles pensaient au contraire qu'un mariage, alliance entre deux familles et contrat civil, ne pouvait pas être valide, quel que soit l'âge des conjoints, sans l'accord formel de leurs parents. Du point de vue de ces derniers (et des clercs eux-mêmes lorsqu'ils ne parlaient pas en tant que représentants de l'Église, mais en tant que membres d'une famille particulière) l'absence de cet accord était une preuve du manque de bon sens, de l'immaturité des deux jeunes concernés, ou de la perversité de l'un d'eux (cf. la réaction du chanoine Fulbert, oncle d'Héloïse, face au mariage secret, et néanmoins canoniquement valide, de sa nièce avec Abélard). 

 Malgré la pression du roi de France, et le besoin qu'ils avaient de son appui et de ses armées, les évêques rassemblés en concile à Trente ont refusé de modifier la doctrine traditionnelle. Le roi a donc promulgué \emph{l'édit de 1556} qui ne contestait pas la validité religieuse des mariages célébrés sans l'accord des parents \emph{(mariages clandestins)} mais qui les déclarait civilement illégaux. Il confirmait le droit traditionnellement accordé aux parents de déshériter les enfants qui se rendaient coupables de tels mariages. Il décidait surtout que l'instigateur ou l'instigatrice d'un tel mariage (c'est-à-dire celui qui avait à y gagner, en principe le plus pauvre) pouvait être condamné à la peine de mort pour \emph{rapt}, ce qui réglait radicalement la question de l'indissolubilité du mariage. Cet édit a été en vigueur jusqu'à la Révolution, et semble avoir réglé le problème à la satisfaction des pères et des mères de familles. 
 
 
 La position des protestants était très proche de celle du roi de France. Pour eux le mariage n'était pas un sacrement mais seulement un contrat entre deux personnes, par nature révocable, et du ressort des seules autorités civiles. Selon Luther (Traubüchlein, 1529) : « \emph{il faut laisser à chaque ville et à chaque pays ses us et coutumes tels qu'ils sont pratiqués \emph{[... tous ces usages]} c'est aux princes et aux magistrats qu'il appartient de les établir et de les régler} ». Le roi et les pères et mères étaient d'accord sur l'idée que le choix d'un conjoint était trop important pour être laissé à la discrétion des futurs époux. Mais le fait de dénier au mariage le poids d'un sacrement et de n'y voir qu'un contrat ne lui enlevait pas une certaine forme d'indissolubilité. Même Henri~VIII n'avait pas rompu le lien entre l'Église d'Angleterre et Rome pour divorcer, mais pour faire reconnaître par les évêques de son royaume la nullité de son premier mariage. 
 
 À partir du moment où le mariage était invalide lorsque les parents des futurs conjoints ne lui apportaient pas leur approbation, comme pour tout autre contrat, il était du devoir de ceux-ci de se soumettre à la volonté de ceux-là dans ce domaine comme dans tous les autres. Une fois leurs parents d'accords, au terme de négociations plus ou moins âpres, et après que le père de la mariée ait remis celle-ci en mains propres à son futur gendre, les époux se devaient de respecter la volonté de leurs auteurs et de rester ensemble en dépit des difficultés éventuelles. C'est pour cela que tout en reconnaissant aux époux le droit de divorcer, les protestants leur imposaient tant de conditions (en Angleterre il y fallait entre autres un acte du parlement) que leurs divorces étaient en réalité difficiles à obtenir et coûteux (800 livres au \siecle{19} en Angleterre) et donc rares : en Angleterre 184 divorces entre 1715 et 1852, pour 9 millions d'habitants environ ; au Massachusetts, état américain bien plus libéral, 143 divorces entre 1692 et 1786 pour \nombre{300000} habitants environ (un et demi par an !). 
 

\chapter{Les devoirs des parents}


 \section{Des pères et des rois}


À partir du \siecle{4} dans l'empire romain, ce n'est plus d'abord et avant tout par la relation de pouvoir qu'il exerce sur les membres de sa maison que le père est juridiquement défini. En effet, il est soumis au devoir de \emph{piété} 
\footnote{La piété était l'affection réciproque et le respect mutuel entre les divers membres de la famille nucléaire, y compris le devoir d'assistance.} 
à l'égard de ses enfants au même titre qu'ils le sont à son égard, et autant qu'eux. L'accent se déplace sur sa responsabilité envers eux. 

La paternité est exaltée en liaison avec la paternité divine, mais la valorisation des parentalités spirituelles à travers le culte de Saint Joseph (père adoptif de Jésus selon les évangiles) affirme la prééminence de la relation éducative sur la reproduction biologique. Au même moment la maternité est très fortement idéalisée à travers le culte de la Vierge Marie. Au total c'est la famille nucléaire, le couple et ses enfants légitimes, qui est sacralisée. Cela s'exprime entre autres dans le culte de la « sainte famille » qui prend son essor au \siecle{17} : c'est en 1665 qu'est fondée la \emph{confrérie de la Sainte Famille} (la fête religieuse de la sainte famille n'a été instaurée qu'en 1893). 


 A la fin du Moyen Âge le fonctionnement des familles semble avoir eu tendance à se rigidifier dans un patriarcat de plus en plus rigoureux, en même temps que les états montaient en puissance et que les doctrines absolutistes gagnaient de l'audience. Vécu comme un père par ses sujets, le roi s'identifiait à son tour à tous les pères de famille. Eux et lui étaient autant de représentants de Dieu « le Père » et ils se confortaient les uns les autres. 

Qu'ils soient protestants ou catholiques les philosophes, les théoriciens du droit et les chantres de l'absolutisme (Jean Bodin, Omer Talon, Bossuet, Thomas Hobbes,~etc.) soutenaient la nécessité d'un pouvoir fort, incarné par un souverain absolu, c'est-à-dire sans contre-pouvoirs significatifs. On peut supposer que cela découlait pour une part de l'expérience des guerres, civiles ou entre états, dans lesquelles les européens se sont laissés entraîner par les divergences entre options religieuses coexistantes. Cette expérience a révélé la violence mortelle que peuvent provoquer les identités, notamment religieuses. Elle a aussi révélé les limites de la capacité du pape et des évêques à réguler pacifiquement les conflits d'interprétation, ce qui les a délogés de leur millénaire position d'autorité et de partenaires autonomes et incontournables des pouvoirs civils. Bon gré mal gré les européens s'en sonr donc remis à "\emph{César}", quitte à s'accomoder de son despotisme ("\emph{cujus regio ejus religio}") et des injustices de la raison d'état. Tout valait mieux à leurs yeux que les désordres d'un monde où chacun serait un loup pour l'autre. 

Les ecclésiastiques eux-mêmes se sont ralliés à cette position : les protestants bien entendu, qui déniaient toute autorité particulière à l'éveque de Rome et confortaient les princes dans leurs désirs de contrôler les cultes, mais aussi les catholiques. Ainsi Pierre de Bérulle (fondateur de l'Ordre de l'Oratoire) écrivait en 1623, dans un discours\footnote{\emph{Discours de l'État et des grandeurs de Jésus}.} au Roi (Louis~XIII)  :
    \enquote{\emph{un monarque est un Dieu selon le langage de l'écriture : un Dieu non par essence mais par puissance ; un Dieu non par nature mais par grâce ; un Dieu non pour toujours mais pour un temps. Un Dieu non pour le Ciel mais pour la Terre. Un Dieu non subsistant, mais dépendant de celui qui est le subsistant par soi-même ; qui étant le Dieu des Dieux, fait les rois Dieux en ressemblance, en puissance et en qualité, Dieux visibles, images du Dieu invisible}}. Jusqu'au milieu du \siecle{18} l'image de l'autorité était globalement positive, et l'exercice que les rois et les pères (et avec eux les « pères spirituels » de tous ordres) faisaient de leurs pouvoirs était regardé comme légitime et bénéfique. Dans ce cadre de pensée s'opposer au souverain comme aux pères c'était faire preuve de présomption et peut-être s'opposer à Dieu lui-même. 
    
    
    Dans quelle mesure cette vision du pouvoir et de la paternité a-t-elle rejailli sur l'image que les gens d'alors se faisaient de Dieu ? Ils prêtaient en effet à celui-ci une dureté ou même une cruauté impitoyable : exigences morales inflexibles, poids de la culpabilisation, arbitraire de la grâce, prédestination, terrorisme de la damnation, croyance en la rareté des élus,~etc. Mais on peut aussi bien se demander si ce ne sont pas les thèses des théologiens de la fin du Moyen Age qui ont renforcé l'absolutisme des pères et des rois ? Ils valorisaient en effet sans limites la toute-puissance divine. Selon Jean-Claude Monod\footnote{ in \emph{La querelle de la sécularisation : théologie politique et philosophie de l'histoire de hegel à Blumenberg}, Jean-Claude Monod, Paris, Vrin, 2002.} : "\emph{L'importance du nominalisme à la fin du Moyen Âge tient à ce que ce courant de pensée a mis en crise le système scolastique en voulant pousser l'homme à une capitulation sans condition dans l'acte de foi, et a retiré à la théologie toute tâche de médiation entre la connaissance et la foi. Ainsi en est-il de la souveraineté absolue de Dieu : volonté insaisissable et opaque "potentia absoluta", le Dieu du nominalisme et ses "décrets" se situent au-delà de toute tentative de compréhension par l'esprit humain. Tout ce qui est fait peut être défait, toute loi peut être suspendue, nulle garantie ne doit être attendue de Dieu, dont l'entendement est incommensurable au nôtre et dont dépend pourtant entièrement notre salut."} Comment penser la liberté des individus si Dieu connaît à l'avance tout leur avenir ? Comment peuvent-ils être responsables de leurs actes ? Comment imaginer un Dieu bon s'il n'est lié par aucune exigence de justice ? etc.  
    
    
    L'époque était au renforcement de toutes les autorités laïques puisque les autorités religieuses avaient failli. Les pères se voyaient donc rappeler leur devoir de maintenir leur maison en bon ordre, dans le respect des lois civiles et religieuses. On attendait d'eux qu'ils le fassent sans faiblesse, et pour y parvenir il leur était reconnu une grande part de la puissance paternelle des romains. Dans les pays de Droit écrit, revenus avant la fin du Moyen Âge à une application stricte du Droit romain, leur puissance ne cessait qu'avec leur mort. Leur mission éducative impliquait le \emph{droit de correction}. On considérait que c'était pour eux un devoir moral et social que de corriger les enfants \emph{et les épouses} indisciplinés. Jusqu'au \siecle{18} (au moins) il était admis qu'une tendresse excessive était plus dommageable, et donc plus coupable, qu'une sévérité excessive : « {qui aime bien châtie bien} ». Montaigne nous dit qu'il fut placé de sa naissance à l'âge de quatre ans chez des bûcherons, puis mis en pension en collège à partir de six ans. Il dit s'être trouvé mieux de cette enfance loin de sa famille... parce qu'il lui semblait que son père était « trop tendre%
% [1] 
\footnote{En justifiant la décision de son père par son « excès de tendresse » Montaigne nous fournit un bel exemple de ce qu'on désigne aujourd'hui sous le nom de « fidélité » ou de « loyauté » des enfants, et des trésors de compréhension dont ils sont prêts à faire preuve face à toutes les décisions, quelles qu'elles soient, que leurs parents ont pu prendre.} 
» !

 Le roi soutenait l'autorité des époux sur leur épouse et leurs enfants, et leur prêtait main-forte s'ils le demandaient, entre autres moyens par les \emph{lettres de cachet} ordonnant sans jugement\footnote{...ancêtre des placements administratifs actuels, dont il faut reconnaître qu'ils sont mieux contrôlés qu'alors par les autorités judiciaires. Il est infiniment plus aisé aujourd'hui de mettre en question leur pertinence parce que nous n'idéalisons plus la parole des pères ni des autres autorités. Au contraire nous les tenons en suspicion.} l'incarcération de l'enfant récalcitrant, mineur ou majeur, ou de l'épouse indigne, volage ou frivole ou de mauvais caractère,~etc. S'il le jugeait nécessaire, il pouvait se substituer de sa propre initiative%
% [2] 
\footnote{De même que lorsqu'il s'agit de ses enfants un père n'attend pas d'être saisi : par définition il parle en leur nom et à leur place (et le Droit romain lui donne ce droit même quand ils sont adultes).} 
aux pères défaillants dans leur fonction de faire régner l'ordre dans leurs familles. 
 Mais il se devait aussi de contrôler qu'ils n'abusaient pas de leurs pouvoirs : leur droit de correction n'était pas un droit de vie ou de mort. Jamais les parents n'ont été autorisés à estropier leurs enfants, et l'appui donné par la force publique à leurs décisions n'était pas automatique.

 

\section{Les transmissions}

    
   Le devoir des pères est d'établir ses enfants dans la vie. Il doit donc transmettre : transmettre ses biens, ses charges et fonctions, transmettre ses connaissances. Il peut se faire aider par plus compétent que lui dès que c'est nécessaire.
   
   Il n'est pas question ici de faire une histoire de l'enseignement, mais seulement d'en esquisser les traits qui ont directement rapport à notre sujet\footnote{\\\fsc{FURET} et \fsc{OZOUF}, \emph{Lire et écrire, l'alphabétisation des français de Calvin à Jules Ferry}, 1977.
\\Maurice \fsc{CAPUL}, \emph{Internat et internement sous l'ancien régime, contribution à l'histoire de l'éducation spéciale}, Thèse d'état, CTNERHI-PUF, Paris, 1983-1984.
\\Martine \fsc{SONNET}, {« Une fille à éduquer », in \emph{Histoire des femmes en Occident}, III, \siecles{16}{18}}, Collectif, sous la direction de Georges \fsc{DUBY} et Michelle \fsc{PERROT}, 2002, Chapitre 4, p. 131 à 168.
\\Sous la direction de Marie-Madeleine \fsc{COMPERE} et Philippe\fsc{SAVOIE}, \emph{L’établissement scolaire. Des collèges d'humanités à l'enseignement secondaire, XVIe-XXe siècles}, numéro spécial 90 de la revue \emph{Histoire de l’éducation}, mai 2001
\\ Marie-Madeleine \fsc{COMPERE}, \emph{Du collège au lycée. Généalogie de l'enseignement secondaire français (1500-1850)}
Collection Archives (n° 96), Gallimard, 1985.
\\sous la dir. de Marie-Madeleine \fsc{COMPERE} et d'André \fsc{CHERVEL}, \emph{Les Humanités classiques}, Paris : Institut national de la recherche pédagogique, 1997.
\\Marie-Madeleine \fsc{COMPERE},	\emph{L'histoire de l'éducation en Europe : essai comparatif sur la façon dont elle s'écrit} Paris : Institut national de recherche pédagogique ; Bern : P. Lang, 1995. }.

 Pour l'énorme majorité (99\%) des jeunes la durée de l'enseignement se réduisait à quelques années et à l'apprentissage des rudiments de la lecture et de l'écriture, mais il ne faut pas oublier l'apprentissage d'un métier dont bénéficiaient beaucoup de jeunes (la plupart ?) de façon informelle et gratuite auprès de leur père ou d'un oncle (dont tous les paysans, les pécheurs, les fils d'artisans ou de commerçants, etc...), ou à titre onéreux auprès d'un maitre d'apprentissage. 
 
L'ensemble du domaine scolaire était sous le contrôle et à la charge des évêques. Des décisions royales répétées au fil des siècles (à commencer par Charlemagne) confirmaient ceux-ci dans leurs droits et aussi dans leurs obligations : en accord avec les autorités civiles ils remplissaient d'autant plus naturellement cette mission de service public que jusqu'à la fin du moyen-âge la plupart de ceux qui avaient suivi un enseignement poussé (les "humanités") faisaient partie du clergé. En principe il était gratuit. Le financement venait de subventions, de dons, ou de taxes affectées, ou du produit de fondations qui faisaient partie des biens ecclésiastiques. 

 
En dehors des monastères qui ont toujours formé leurs propres novices, le réseau des écoles s'est développé depuis le début du Moyen Âge (cf. troisième \emph{Concile de Vaison}, 529) à partir des \emph{écoles cathédrales}, d'une part vers l'enseignement élémentaire avec les \emph{petites écoles} (écoles primaires), d'autre part vers l'enseignement supérieur (incluant à cette époque ce qu'à partir du \siecle{19} on appellera l'enseignement \emph{secondaire}) avec les \emph{universités} et leurs \emph{collèges}.  Elles étaient placées sous le contrôle du Chapitre de la Cathédrale. L'un des chanoines exerçait cette responsabilité : l'\emph{écolâtre},  le \emph{chantre} ou le \emph{chancelier}... C'est lui qui jusqu'à la fin de l'ancien régime agréera tous les candidats à l'enseignement, agrément sans lequel nul n'avait le droit d'enseigner sur le territoire sous sa juridiction. 





À partir des derniers siècles du Moyen Âge des \emph{petites écoles} paroissiales existaient dans toutes les villes importantes, fondées par les curés, ou par les municipalités, et ordinairement par les deux à la fois. A côté des connaissances profanes (d'abord la lecture, le calcul, souvent l'écriture, mais pas toujours) on enseignait aussi la religion, les disciplines du corps et de l'esprit, les bonnes manières de se conduire. Leur mission était en effet d'éduquer autant que d'enseigner. L'instruction, une fois entendu qu'elle se devait d'inclure la religion, était considérée comme la meilleure défense contre l'envie de mal faire. Curés ou pasteurs protestants, parents et autorités locales étaient d'accord sur ce point. D'autre part les citadins voyaient aussi en elle la meilleure arme pour trouver et pour garder un travail, ce qui avait à la fois un intérêt économique et un intérêt social. À partir de la Réforme et du Concile de Trente cette foi en l'enseignement s'est exprimée en un véritable apostolat (c'est pourquoi divers ordres enseignants ont été créés au fil des siècles).


 Les petites écoles s'adressaient aux « enfants des pauvres », c'est-à-dire, dans le langage d'alors, à tous ceux dont les ressources étaient précaires, ceux qui n'avaient pas de rentes, de quelque nature qu'elles soient, et qui devaient gagner leur vie en travaillant. Il s'agissait donc de l'essentiel de la population des villes. Mais les petites écoles ne pouvaient pas toujours être complètement gratuites (pas plus que les universités). Elles étaient donc à la portée des bourgeois aisés, des commerçants et artisans, mais pas toujours à celle des autres. Lorsque les paroisses ne pouvaient pas exempter ces derniers des frais de scolarité, ce qui était le cas lorsque l'ensemble de leurs paroissiens étaient réellement pauvres, seuls de généreux donateurs et surtout des ordres religieux pouvaient les prendre en charge (cf.  les « écoles de charité »). Les religieux bénéficiaient en effet d'une sécurité financière, d'une surface sociale et d'un entregent que ne pouvaient avoir des particuliers ou des communes pauvres. Certains ordres avaient d'ailleurs explicitement été créés pour assurer gratuitement l'enseignement des indigents.

Beaucoup d'enfants n'étaient pourtant pas scolarisés, même dans les villes où l'enseignement était gratuit : leurs parents avaient trop besoin du produit de leur travail, ou bien ils ne voyaient aucune utilité à un apprentissage scolaire. Même aux yeux de ceux qui envoyaient leurs enfants à l'école il n'était pas toujours évident qu'il faille que ceux-ci soient scolarisés avec assiduité pendant plusieurs années. Beaucoup, et peut-être même la plupart, se contentaient des quelques mois ou années nécessaires pour apprendre à lire et/ou à écrire. Jusqu'à la fin de l'ancien régime l'instruction des paysans (plus de 90~\% de la population) n'était pas jugée nécessaire. Étant donné le niveau des techniques alors en usage l'illettrisme n'avait pas d'incidence sur leur productivité. D'autre part leurs maîtres et seigneurs craignaient qu'une instruction même minime ne les rende « raisonneurs » et « arrogants ». En l'absence d'école les plus brillants pouvaient être distingués par leur curé (qui avait depuis l'antiquité l'obligation de faire à tous le catéchisme et de repérer les plus alertes d'esprit) et recevoir de lui les bases nécessaires pour aller au collège. Quant aux enfants de famille aisée, bourgeois et aristocrates, leur première scolarité se faisait traditionnellement auprès d'un précepteur recruté par leurs parents.  



Les écoles cathédrales et les écoles monastiques ont été créées pour fournir l'Église en clercs, mais elles ont toujours reçu un petit contingent d'élèves promis à la vie civile. Charlemagne leur en a fait une obligation. À partir du \siecle{10} la croissance des villes a provoqué la demande d'une instruction de niveau universitaire (à l'époque le secondaire et le supérieur n'étaient pas encore distingués). À partir du \siecle{12} les universités se sont créées comme des corporations autogérées de professeurs indépendants, avec l'appui des autorités civiles. Elles étaient sous le contrôle de l'évêque du lieu et leur personnel comme leurs étudiants bénéficiaient des avantages et exemptions attachés aux clercs (à cette époque la majorité d'entre eux étaient religieux ou prêtres ou destinés à le devenir). En cas de conflits entre l'évêque et l'Université le pape arbitrait. Dans le cadre des universités ont été créés des collèges caractérisés d'abord par la vie en internat à l'intention des étudiants pauvres, sur le modèle des écoles monastique (p. ex. le collège qui deviendra la Sorbonne). Leur mission était de permettre aux jeunes gens sans fortune d'entrer dans le clergé (puisqu'il y fallait un titre universitaire) mais peu à peu ces collèges d'universités sont devenus des lieux d'enseignement en complément des cours publics, et ils ont de ce fait été recherchés par d'autres candidats. L'enseignement que nous nommons « secondaire » est donc sorti de l'enseignement « universitaire ».

Même si la gestion d'un internat était lourde et source de tracas beaucoup de pédagogues en avaient une image positive, mais entre l'externat des collèges, gratuit ou presque, et la pension des internats l'écart des coûts était énorme
\footnote{Selon Martine \fsc{SONNET}, la pension d'un seul enfant, garçon ou fille, représentait presque la totalité du salaire d'un ouvrier (« une fille à éduquer », Chapitre 4 de \emph{L'Histoire des femmes en Occident}, III, \siecles{16}{18}, p. 146). C'est pourquoi en 1760 les internats parisiens n'accueillaient que 13~\% de la population scolarisée de la ville, et il semble qu'il en était de même ailleurs.}% 
. Aux familles qui ne pouvaient payer les frais d'une pension, c'est-à-dire la plupart, seul l'externat était accessible, en vivant en ville chez ses parents\footnote{... d'où la pression des municipalités pour créer un collège ou un « petit collège », et au minimum une classe de latin, une « régence latine », pour gagner quelques années de scolarité sans recourir à la pension.} ou chez un parent, ou chez un logeur peu exigeant. 

C'est en se transformant profondément que la formule du collège s'est généralisée à partir du \siecle{16} avec à cette époque une majorité d'élèves externes promis à la vie laïque, avec une réduction forte de la dispersion des âges, et un classement des élèves par niveaux, etc.. De nombreux collèges ont été créés à la demande des municipalités et/ou des évêques. Les initiatives étaient très décentralisées et les créations partaient le plus souvent des besoins et des demandes locales, et d'abord des demandes des parents d'élèves potentiels. Les enseignants étaient recrutés au sein du clergé diocésain local en fonction des compétences et des titres universitaires. Au fil du temps la gestion de beaucoup de ces nouveaux collèges a été confiée par leurs fondateurs à des ordres religieux spécialisés : surtout les jésuites et les oratoriens, à côté desquels d'autres ordres comme les bénédictins ou les dominicains ont aussi joué un certain rôle. Jusqu'à leur expulsion (1765) les jésuites ont eu un rôle prépondérant avec leurs immenses collèges de \nombre{2000} élèves, tous externes, qui recevaient une scolarité gratuite valorisant le latin comme une langue vivante et utilisant des méthodes d'enseignement actives. 

Mais tous les collèges n'étaient pas de plein exercice, avec des classes de tous les niveaux jusqu'à la classe de philosophie comprise. Beaucoup d'entre eux (les "petits collèges") se contentaient des quelques niveaux de base, quand la commune ne se bornait pas à entretenir un seul professeur de latin (une \emph{régence latine}) pour les quelques élèves concernés. Les élèves désireux d'aller jusqu'au bout du cursus secondaire allaient le terminer dans un plus gros établissement,  Beaucoup de parents se contentaient d'une scolarité réduite à quelques années de collège. 

Au fil des générations la scolarité secondaire a eu tendance à se décentraliser sans que pour autant le nombre d'élèves concernés n'ait augmenté. Ce dernier est resté étonnement stable pendant très longtemps : moins de un pour cent de la population des jeunes garçons d'âge scolaire\footnote{Selon un rapport établi en 1843 par A. F. Villemain  et cité par Antoine LEON (\emph{Histoire de l'enseignement en France}, Que Sais-je ?, PUF, Paris, 1967) il existait à la veille de la Révolution 562 collèges avec \nombre{73000} élèves, dont \nombre{40000} boursiers : 178 collèges congréganistes et 384 collèges dépendant des universités ou gérés par des communes ou des particuliers. En 1812 il y avait 36 lycées et 337 collèges publics avec \nombre{44000} élèves, et \nombre{1000} autres institutions et pensionnats privés pour \nombre{27000} élèves, soit \nombre{71000} élèves au total. En 1880 il y avait environ \nombre{150000} élèves dans les lycées et collèges, et \nombre{500000} en 1940.}. 

Pour les parents des collégiens l'éducation était un investissement familial, même quand ils se destinaient à devenir des clercs\footnote{Les membres du clergé, qui n'étaient pas astreints au voeu de pauvreté, pouvaient faire des carrières très lucratives et ainsi aider matériellement leurs frères et soeurs. Même les religieux et religieuses pouvaient aider à caser l'un ou l'autre de leurs neveux et nièces, ce qui pouvait conduire au "népotisme", c'est-à-dire à l'art d'avantager abusivement ses neveux.}, ce qui était le cas d'une proportion significative jusqu'au \siecle{18}. Cela justifiait qu'ils soient improductifs pendant leurs années de scolarité. Ceux qui étaient sans ressources mais dont les dons intellectuels étaient évidents pouvaient bénéficier de bourses, surtout ceux qui se destinaient à entrer dans les ordres.



Les collèges proposaient aux jeunes garçons de s'investir dans la découverte du savoir, et celui-ci était ressenti par leurs enseignants et leurs parents comme quelque chose qui en valait la peine. Ils entraient dans une aristocratie de l'esprit. À l'époque dans toute l'Europe l'enseignement secondaire et supérieur se faisait en latin. Sans lui on savait peut-être lire, mais on n'en demeurait pas moins un \emph{illettré} qui ne connaissait pas les \emph{belles lettres}
\footnote{C'est en latin qu'Héloïse et Abélard se sont écrit toute leur vie. C'est en latin que la République des Lettres de la Renaissance correspondait d'un bout de l'Europe à l'autre. Dans toute l'Europe les thèses de doctorat seront encore soutenues en latin durant la plus grande partie du \siecle{19}.}. C'était la langue vivante, la langue de communication des communautés intellectuelles du temps. 

Mais depuis l'\emph{ordonnance de Villers-Cotterêts} (1539) qui imposait le français comme langue administrative du Royaume, il n'était plus possible de tenir un \emph{office} public si on ne le maîtrisait pas suffisamment. La langue française n'était encore que le patois de l'Île-de-France, domaine du roi. Partout ailleurs c'était une langue étrangère qui allait mettre très longtemps à déloger les langues locales des places et des marchés. Même si dans les collèges l'accent était mis sur le latin l'enseignement du français était donc incontournable pour exercer un office au service du roi. 

Quant aux jeunes filles de famille aisée, la clôture des couvents leur interdisait toute rencontre avec les jeunes gens de leur âge et protégeait leur « vertu » et leur réputation en attendant que leurs parents les marient. Sauf quand elles se destinaient à être religieuses la durée de leur séjour au couvent était très inférieure à celle de leurs frères dans leurs collèges : un an pour préparer leur communion solennelle, deux ou trois au plus. Les jeunes filles bien dotées étaient mariées bien plus tôt que les autres. Le savoir qui leur était dispensé était nettement moins poussé que celui que recevaient leurs frères, même si les novices bénéficiaient d'un enseignement qui en faisait des lettrées, des sœurs « de chœur », capables au minimum de chanter les offices en comprenant le latin qu'elles chantaient, et d'enseigner aux jeunes pensionnaires.

\section{La correction paternelle}

 Les jeunes étaient sous l'autorité de leurs parents jusqu'à leur majorité. Qu'ils soient orientés vers des études plus ou moins longues, ou mis au travail dès qu'ils pouvaient gagner leur pain, c'est leur père (à défaut leur mère) qui en décidait. Mais  tous n'entraient pas docilement dans les projets parentaux : jeunes en fugue de leur classe d'école ou de leur atelier, fréquentations suspectes, beuveries, insultes et de voies de faits, exclusion pour indiscipline de leur collège ou de leur apprentissage, vols domestiques, inconduite sexuelle,  jeunes « libertins », c'est-à-dire rétifs à toute mesure éducative,~etc. 

 À la demande de leurs parents, ces jeunes pouvaient être traités comme des délinquants. Pour les enfants difficiles des familles aisées il y avait des solutions payantes dans les sections des collèges et internats contemporains affectés à la « correction ». Ceux qui n'en avaient pas les moyens étaient internés avec les délinquants condamnés. Leurs parents payaient une pension qui tenait compte de leurs ressources. 

 À partir de la fin du \siecle{17} et de plus en plus souvent au fil du \crmieme{18}, les \emph{enfants de famille}, garçons et filles mineurs \emph{et majeurs}, qui avaient commis de vrais actes de délinquance, mais aussi ceux qui donnaient simplement du mécontentement à leurs parents par leurs fréquentations, leur mauvaise conduite, leur indocilité, leur violence aveugle ou leur absence de sens commun (« insensés »), leurs dépenses inconsidérées, ou leurs dettes de jeu, pouvaient, sur la demande de ces derniers,  faire l'objet d'une \emph{lettre de cachet}, c'est-à-dire d'une \emph{décision administrative d'internement} dans un hôpital, une prison, une forteresse, un couvent, un collège, ou même leur déportation aux colonies. Les lettres de cachet, qui ont une origine très ancienne, bien antérieure au \siecle{17}, pouvaient aussi être accordées à l'encontre de conjoints aux comportements répréhensibles : cette mesure a beaucoup plus souvent frappé les épouses que les époux. 

 L'autorité publique n'était pas obligée d'accorder satisfaction aux demandes qui lui était faites, et restait seule juge de l'opportunité de la mesure. Elle était surtout sollicitée à Paris, notamment par les couches populaires, contrairement aux provinces où l'internement administratif était moins facile à obtenir et où les couches populaires n'y avaient guère recours. Même si au fil du temps les lettres de cachet ont fait l'objet de critiques de plus en plus virulentes et si les autorités publiques y répugnaient de plus en plus, les demandes se sont faites de plus en plus nombreuses au fil du \siecle{18}. 

 En effet les familles sollicitaient ces lettres comme une grâce : cela leur évitait la honte causée par la publicité du recours à la justice, le coût d'un procès, et la publicité de la mesure d'enfermement. La réputation du jeune (ou de l'adulte) ainsi placé pouvait s'en relever plus facilement. Cela évitait le contrôle par la justice de la nature exacte des faits incriminés et de la proportionnalité des sanctions aux dommages et délits constatés. Cela permettait à l'occasion à d'authentiques délinquants bien nés d'échapper à peu de frais aux conséquences normales de leurs actes. 

Mais cela permettait aussi aux parents abusifs d'exercer des pressions sur leurs enfants rétifs à leurs projets (ce qui expliquait les critiques de plus en plus virulentes des lettres de cachet au fil du \siecle{18}), à une époque où le consentement des parents était exigé à tout âge et pour tout mariage sous peine d'exhérédation, et où bien des entrées en religion étaient imposées par eux sans tenir compte des désirs du ou de la jeune concerné. Et cela confortait aussi l'autorité des épouses sur leurs épouses

\section{Des enfants « adoptifs » ?}

 On a vu que dans le but de défendre le mariage monogame et indissoluble, l'Église a tout fait depuis l'Antiquité pour que les enfants illégitimes ne puissent pas devenir des héritiers de plein exercice. C'est pour cette raison que l'adoption était interdite, et pourtant... De l'Antiquité à la fin de l'ancien régime, on peut observer en nombre non négligeable des situations plus ou moins proches d'une adoption, où une personne, souvent un ecclésiastique (cf. \hbox{Villon}, adopté par un chanoine), souvent aussi un couple sans enfants, exerçaient la puissance paternelle sur un enfant qui n'était pas né d'eux et qu'ils élevaient jusqu'à sa majorité. C'était par exemple le cas à Lyon, où les recteurs de l'Hôtel-Dieu « adoptaient » ainsi des orphelins. 

 Ces situations d'\latin{alumnii} (adoptions simples) étaient parfois sanctionnées par des actes juridiques où les nourriciers faisaient un legs à l'enfant devant un procureur fiscal, et où ils s'engageaient à l'élever, instruire et établir matériellement à leurs frais comme leur propre enfant. Pour autant cela ne faisait pas de lui un membre de leur famille ni un héritier. 

 En principe seul un enfant légitime sans parents pouvait bénéficier de ce dispositif. Souvent, probablement le plus souvent, il était orphelin, mais des enfants légitimes pouvaient aussi être abandonnés solennellement par leurs parents, qui reconnaissaient par écrit qu'ils renonçaient à leur puissance paternelle, et à l'héritage de leur enfant s'il décédait. Pour autant ce dernier ne changeait ni de parenté ni de nom. Quand il possédait des biens, l'adoptant, tel un tuteur, les gérait jusqu'à sa majorité et il était responsable sur ses propres biens de sa gestion. 

 Les enfants abandonnés, nés de parents inconnus, ont très longtemps été exclus de ce genre de prise en charge, parce qu'ils étaient suspects d'être illégitimes. Ainsi en était-il à Lyon jusqu'en 1765. Ensuite ils y ont été traités comme les autres. Ce n'est que dans la deuxième moitié du \siecle{18} que les idées ont changé sur ce point : à partir des années 1760-1770.}. Pourtant il était courant que des personnes accueillent pour l'élever un enfant abandonné à eux confié par un hôpital ou par une paroisse, qu'elles refusent d'être rémunérées pour l'élever, qu'elles le gardent jusqu'à sa majorité et qu'elles l'établissent dans la vie, ce qui en fait ressemblait beaucoup à la situation des enfants nés légitimes et juridiquement « adoptés ». Si aucun de leurs héritiers légitimes ne s'y opposait, elles faisaient de lui l'un de leurs héritiers. Mais il n'était pas question pour cet enfant d'hériter d'une fonction impliquant l'exercice public du pouvoir. 
Derrière les mots employés il n'est pas toujours facile de reconnaître les situations réelles : adoption simple ? tutelle ? parrainage ?
\footnote{Cf. Jean-Pierre \fsc{GUTTON}, \emph{Histoire de l'adoption en France}, 1993.} 

\section{Les enfants illégitimes}

 Alors que les grossesses n'avaient pas à être déclarées, à partir de 1556 obligation est faite par Henri~II de déclarer toutes les grossesses illégitimes (et elles seules), sous peine pour les filles non mariées et les femmes veuves depuis plus d'un an qui seraient enceintes d'être accusées d'infanticide si leur enfant décédait avant son baptême\footnote{... qui avait valeur officielle de déclaration de naissance puisque les curés avaient reçu peu de temps auparavant l'obligation de tenir les \emph{registres de catholicité}, ou registres de baptême, ancêtres directs des registres d'état civil.} 
(crime en principe puni de mort). Dans la déclaration devait figurer le nom du père allégué par la mère, sauf refus de celle-ci. Cette déclaration renforçait la position de la mère face à l'homme qui l'avait engrossée, et celle de son enfant, et permettait les actions en justice. Cette décision royale a été rappelée par Henri~III en 1585, et renforcée par Louis~XIV. C'est ainsi qu'en 1708 ce dernier ordonnait encore aux curés de la rappeler en chaire tous les trois mois. 

 Si elle l'a été si souvent, c'est qu'elle n'a jamais été observée de manière rigoureuse. Il semble même que la majorité des grossesses illégitimes n'aient jamais été déclarées. En dépit de la sévérité des peines annoncées les mères préféraient oublier de se signaler à l'attention des autorités lorsqu'elles pensaient pouvoir mieux défendre leurs intérêts et leur réputation (et ceux de leur enfant) par un arrangement discret avec le géniteur (ex. : mariage, pension alimentaire, octroi d'une dot,~etc.) ou par un abandon discret. Combien parmi les veuves et filles dont l'enfant est décédé sans baptême ont-elles effectivement subi les peines prévues ? Il ne semble pas que les autorités aient poursuivi ce genre d'infraction avec beaucoup d'énergie : le plus souvent les tribunaux accordaient de larges circonstances atténuantes aux « coupables » déférées devant elles
\footnote{Frédéric \fsc{Chauvaud}, Jacques-Guy \fsc{Petit}, Jean-Jacques \fsc{Yvorel}, \emph{Histoire de la justice de la Révolution à nos jours}, Presses universitaires de Rennes, 2007.}% 
. 

 Tout enfant, même illégitime, avait le droit d'exiger de ses auteurs des « aliments » c'est-à-dire des moyens de vivre. Un vieil adage juridique, toujours cité, disait en effet que \emph{qui fait l'enfant doit le nourrir}. Le représentant naturel de l'enfant né hors mariage est sa mère, et \emph{protéger celle-ci était aussi protéger l'enfant}. Les actions de la mère%
% [14] 
\footnote{Nommée « fille-mère », et n'ayant droit qu'au titre de « mademoiselle » jusqu'au milieu du \siecle{20}. Ce n'est pas un enfant qui pouvait faire d'elle une femme, une « dame », mais un époux en règle.} 
contre le géniteur qui l'avait délaissée étaient encouragées et soutenues, notamment par les hôpitaux, qui en cas d'abandon de l'enfant devaient en assumer seuls la charge. Elle pouvait entreprendre une \latin{actio provisionis} : demande de provisions pour frais de grossesse ou d'accouchement. Si plusieurs hommes avaient partagé à la même période son intimité ils pouvaient être solidairement responsables de l'enfant. Elle pouvait aussi tenter une \latin{actio susceptionnis partus} ou \latin{actio captionis} : action qui demandait de condamner le géniteur à assumer les frais de l'éducation de l'enfant, sur lequel il ne recevait pour autant aucune autorité. 

 \latin{L'actio dotis} prévoyait que le coupable d'un viol épouse la célibataire qu'il avait déflorée, surtout s'il l'avait engrossée. S'il refusait de l'épouser, ce qui était son droit, il devait payer une dot à la mère et financer l'entretien de l'enfant. Il en était de même si le géniteur était déjà engagé ailleurs (mariage, vœux religieux, ordination sacerdotale). Quel que soit son statut (célibataire, marié, clerc, moine, noble, roturier ou serf) il était et demeurait responsable de la vie de l'enfant et devait donc le nourrir. Même si le géniteur n'était pas père légal il restait \latin{nutritor}.

 Par contre les enfants adultérins étaient toujours traités comme des enfants abandonnés, comme s'ils n'avaient ni père, ni mère, ni \latin{nutritor}. Ils n'avaient aucun droit vis-à-vis de leurs deux géniteurs, dans la famille desquels ils n'entraient pas et auxquels ils ne pouvaient pas réclamer des aliments%
% [15]
\footnote{Ceci dit la loi n'interdisait pas à leurs auteurs de prendre librement l'initiative de pourvoir à leur éducation.}% 
. Ils étaient exclus de toute possibilité de légitimation, même par mariage, puisque leurs géniteurs ne pourraient pas se marier, même après la mort de l'époux qui faisait obstacle à leur mariage. 

 Étaient encore plus rigoureusement exclus de toute légitimation les enfants nés d'une relation incestueuse.


\chapter{Organisation d'une police des pauvres}

 À la fin du Moyen Âge il était courant que les mendiants représentent 10~\% de la population 
\footnote{José \fsc{CUBERO}, \emph{Histoire du vagabondage}, 1998, p. 8.}% 
. Les solutions en vigueur depuis la fin de l'antiquité pour traiter l'indigence et les malheurs individuels, pensées pour de petites communautés rurales où tous se connaissent
\footnote{José \fsc{CUBERO}, 1998, p. 42 et suivantes.}% 
, n'étaient plus à l'échelle des problèmes en un temps où les villes débordaient de leurs murailles anciennes, et où les États modernes se constituaient, imposant plus d'ordre, de rigueur et de contrôles, et rognant peu à peu les larges marges jusque là consenties entre les principes et les pratiques réelles. 

 Face à la pauvreté dès 1350 apparaissent les signes avant-coureurs d'un changement des mentalités et des pratiques. Les vagabonds ne sont plus assimilés aux pèlerins du moyen-âge mais sont de plus en plus souvent considérés comme des fauteurs de trouble. On commence à parler de « {bons} » et de « {mauvais pauvres} ». Les « {bons pauvres} » ou « {pauvres honteux} » ont le droit moral de mendier parce qu'il leur est impossible de travailler, et parce qu'ils restent rattachés à leur cadre villageois, à leur paroisse d'origine : enfants, infirmes, malades, vieillards... Ils ne se soustraient pas au contrôle de leur communauté. Les \emph{mauvais pauvres} sont ceux qui ont force et santé mais qui fuient le travail par paresse ou par goût de l'errance
\footnote{... ou par refus de conditions de travail par trop inacceptables (mais cela c'est notre point de vue du \siecle{21}, ce n'était pas celui des décideurs d'alors).} 
loin de tous les cadres sociaux, sans aveu. On soupçonne les vagabonds de vivre dans la débauche et de commettre nombre de délits (notamment des vols). On a peur de leur nombre qui favorise la mendicité agressive et qui intimide les personnes sans défense (enfants, jeunes filles, femmes, vieillards). On les accuse de contrefaire maladies ou infirmités, d'enlever des enfants pour exciter la pitié des passants
\footnote{José \fsc{CUBERO}, 1998, p. 70.}% 
, et même de mutiler ces derniers pour obtenir plus d'aumônes%
%[5]
\footnote{Bronislaw \fsc{GEREMEK} fait état de procès tenus dans la région parisienne en 1449 où ont été condamnés des criminels qui avaient successivement enlevé plusieurs enfants à leurs parents, enfants auxquels ils avaient crevé les yeux et coupé bras ou jambes, pour en tirer profit en mendiant. (in \emph{Les marginaux parisiens aux \crmieme{14} et \crmieme{15} siècles}, Paris, 1976).}% 
. 

 Dès le \siecle{14} les hôpitaux refusent de plus en plus souvent les vagabonds%
% [6]
\footnote{José \fsc{CUBERO}, 1998, p. 68.}% 
, tandis que de nombreuses mesures de police tentent de les contrôler et surtout de les chasser. À l'intention des petits délinquants, des vagabonds et autres chômeurs sans ressources avouables on fait des expériences multiples de travaux « forcés », travaux d'utilité publique, ou même galères du roi%
%[7]
\footnote{En 1456 les États du Languedoc prévoient cette peine pour les vagabonds invétérés. \emph{En 1486, Charles~VIII étend cette mesure à l'ensemble du royaume.} La condamnation aux galères, résurgence de la condamnation antique aux mines, \latin{ad metallas}, se substitue alors dans la plupart des cas à la peine de mort, jusque là appliquée largement en l'absence de peines plus adaptées : \emph{avec la peine des galères... le Moyen Âge renoue avec la notion antique de l'esclavage... Seul le travail rédempteur peut éviter les galères[7] à ces mendiants valides et vagabonds qui menacent la paix}, José \fsc{CUBERO}, p. 78 idem.}% 
. Le fait que ces décisions d'expulsion aient été périodiquement reformulées montre et leur relative inefficacité, et la persistance des représentations qui les sous-tendent.

 Les cités, en expansion, sont dirigées par leurs bourgeois, commerçants, artisans, juristes et autres détenteurs d'offices. Leur expérience personnelle les porte à tenir pour synonymes les vertus familiales et « bourgeoises » : fidélité, économie, sens de l'effort, contrôle de soi et prévoyance. Pour eux un sou est un sou : contrairement aux aristocrates ils ne valorisent ni le panache, ni le faste, ni la prodigalité. Les anathèmes des religieux contre la richesse, traditionnels, ne les impressionnent plus, sauf lorsqu'ils sont à l'article de la mort, et ils sont fiers de leur fortune. Ils ont la tranquille assurance de ceux qui ont réussi. À leurs yeux les autres n'ont qu'à en faire autant, et ils se font forts de le leur enseigner.
 
 
 Cela s'accompagne  de l'exercice d'un contrôle rigoureux sur les comportements, les siens et ceux des autres. Dans la plus grande partie des sociétés européennes c'est à la suite des réformes protestante et catholique que les écarts seront les plus faibles entre la morale sexuelle et conjugale officielle et les pratiques réelles. Ce sera le moment où tous les laïcs ou presque se marieront et feront des enfants. Ce sera le moment où les taux de naissances illégitimes et de conceptions pré conjugales seront au plus bas de toute l'histoire européenne : de 1650 à 1750, Normandie : 2 à 3~\% d'enfants illégitimes ; bassin parisien : 1~\% ; Languedoc et Bretagne : 1 à 2~\%. Angleterre sous Cromwell : moins de 1~\% ; en 1600 : 3,2~\%. Ces taux impliquent un haut degré de contrôle social, exercé conjointement par les familles, par les autorités civiles et par les autorités religieuses.

 Au moment où les peuples d'Europe sont en train de se cliver entre catholiques et protestants apparaissent simultanément dans tous les grands États européens des mesures très semblables pour contrôler pauvres et vagabonds. Pas un seul instant la marche vers la rationalisation du contrôle des pauvres et l'organisation de leur mise au travail, forcé si nécessaire, n'a été entravée ou modifiée par les guerres de religion, et tous les États concernés connaissent des évolutions à peu de choses près superposables : mêmes représentations, mêmes solutions, mêmes réussites et mêmes impuissances. 

\subsection{Création du placement d'office des mineurs de famille}
 Le 22 Avril 1532 le Parlement de Paris ordonne une fois de plus que tous ceux qui dans cette ville peuvent travailler et n'ont ni emploi ni revenus avouables seront contraints à entrer dans les ateliers publics qu'il organise pour eux. Ils travailleront enchaînés deux à deux, gardés rigoureusement et employés aux travaux d'utilité publique les plus rudes. On reconnaît là les pratiques des bagnes%
% [8]
\footnote{... décrites par exemple par Philippe \fsc{HENWOOD} dans \emph{Bagnards à Brest} : « l'accouplement » des bagnards enchaînés, deux par deux, p. 40, 41 et 42,~etc.}% 
. \emph{Mais l'Ordonnance royale du 22 avril 1532 est fondamentale en ceci qu'elle ordonne le placement d'office des enfants des vagabonds arrêtés.} L'autorité parentale peut désormais être disqualifiée en l'absence de tout autre délit que le vagabondage. Ce n'était pas la première fois que des essais de ce genre étaient tentés (exemple : Reims, 1454) mais cette fois il s'agit de le faire à Paris, où se trouve la plus grande concentration de vagabonds du royaume (environ un tiers) et l'Ordonnance est signée par le roi. Elle donne aux \emph{bureaux des pauvres}, où siègent des représentants des autorités ecclésiastiques et judiciaires, une part de l'autorité de l'État. Ils exercent une fonction d'autorité sur tous les pauvres, dont ils peuvent et doivent contrôler non seulement l'incapacité de travailler, mais aussi la correction des pratiques conjugales, éducatives et religieuses. Contrôle et assistance sont désormais liés, et les assujettis ont peu de recours judiciaires possibles : ils subissent une justice d'exception. 

 L'Ordonnance royale de 1566 étend l'interdiction de la mendicité à tout le royaume de France et met les pauvres à la charge de leur paroisse d'origine (\emph{domicile de secours} : seul lieu où l'indigent a droit aux secours) ce qui leur interdit de vagabonder. Elle prévoit que les \emph{bureaux des pauvres} et autres \emph{aumônes générales} doivent si nécessaire organiser et financer des ateliers pour donner du travail aux indigents valides. Entre 1550 et 1600, des forces de police spéciales placées sous l'autorité directe des {bureaux des pauvres} (souvent appelées \emph{archers de l'Hôpital}) sont chargées de traquer la mendicité, de poursuivre hors de l'hôpital et d'arrêter les vagabonds, de récupérer les enfants placés par les bureaux des pauvres lorsqu'ils ont fugué de leur lieu de placement, et de faire régner l'ordre dans les hospices et hôpitaux. 

 Au \siecle{17} les expériences réalisées et les réflexions entretenues par les divers acteurs de l'assistance et du contrôle social confluent dans l'idée qu'il convient de regrouper en une seule administration centralisée les hôpitaux et les hospices, et d'y renfermer tous les indigents qui ne peuvent se prendre en charge seuls, en raison de leur immaturité, de leurs infirmités ou maladies, ou bien en raison de leurs comportements%
% [9]
\footnote{Sources principales :
\\Collectif sous la direction de Jean \fsc{IMBERT}, \emph{L'histoire des hôpitaux en France}, 1982.
\\Maurice \fsc{CAPUL}, \emph{Internat et internement sous l'ancien régime, contribution à l'histoire de l'éducation spéciale}, Thèse d'État, 4 tomes, Tomes 1 et 2, \emph{Les enfants placés}, Tome 3 et 4, \emph{La pédagogie des maisons d'assistance}, 1983-1984.
\\Michel \fsc{FOUCAULT}, \emph{Folie et déraison : histoire de la folie à l'âge classique}, 1961.
\\Michel \fsc{FOUCAULT}, \emph{Surveiller et punir, naissance de la prison}, 1975.
\\Bronislaw \fsc{GEREMEK}, \emph{La potence ou la pitié, l'Europe et les pauvres du Moyen Âge à nos jours}, 1987.
\\Jean \fsc{IMBERT}, \emph{Le droit hospitalier de l'ancien régime}, 1993.
\\Jacques \fsc{TENON}, \emph{Mémoires sur les hôpitaux de Paris}, 1788.}. 

\subsection{Création de l'Hôpital général}

 Louis~XIV ordonne en 1656 la création d'un \emph{Hôpital Général} dans toutes les grandes villes du royaume, et le 14 juin 1662 l'établissement d'un hôpital général dans \emph{toutes les villes et gros bourgs}. Les directeurs, nommés à vie, reçoivent des pouvoirs administratifs et de police pour accomplir leurs missions : \emph{tout pouvoir d'autorité, de direction, d'administration, commerce, police, juridiction, corrections et châtiments sur tous les pauvres de Paris, tant en dehors qu'au-dedans de l'hôpital général \emph{[...]} sans que l'appel puisse être reçu des ordonnances qui seront par eux rendues} [...] Les administrateurs de l'hôpital jugent sans appel, à charge pour eux \emph{si lesdits pauvres méritent peine afflictive plus grande que le fouet, de le mettre es mains du juge ordinaire pour à la requête du procureur d'office leur procez estre fait et parfait}. 





 Que le mouvement de création des Hôpitaux généraux se soit poursuivi à la demande des autorités locales, et pas seulement en France, jusqu'à la fin du \siecle{18} montre que cette formule de l'institution fermée et à l'écart du monde correspondait%
% [10] 
\footnote{Maurice \fsc{CAPUL}, idem, T III, p 301.} 
bien aux conceptions de l'époque : partout en Europe on observait à cette période le même mouvement. Les \anglais{Poor Laws} anglaises ordonnaient en 1661 ou 1662 l'enfermement des pauvres dans des \anglais{Workhouses} qui sont l'exact pendant (en plus dur ?) des hôpitaux généraux. Il en était de même à Berlin,~etc.

 Les contemporains essayaient de ne pas avoir personnellement affaire à ces institutions dont le régime n'était pas fait pour être désirable. Par contre ils approuvaient leur utilisation pour mettre à l'écart les indésirables et pour éviter les catastrophes en cas de disette ou de crise de l'emploi. 

 Et pourtant il y avait des listes d'attente pour entrer à l'hôpital et il fallait souvent patienter avant d'y être admis. Une recommandation était ordinairement nécessaire (très souvent celle de son curé). Un certain nombre de personnes, pauvres mais non indigentes, acceptaient même de payer pension pour y entrer, ce qui laisse à penser que même si les conditions de vie y étaient rudes (mais ces personnes-là n'étaient pas astreintes au travail forcé) il y avait encore pire ailleurs. Pour elles l'Hôpital Général fonctionnait comme une maison de retraite (cf. les « petites maisons » dans le cadre de celui de Paris), et assumait une forme de prise en charge qui existait déjà avant sa propre création.

 Quant à ceux des mendiants et vagabonds qui troublaient l'ordre public par leurs débordements, ils ne venaient pas à l'hôpital de leur plein gré et leurs comportements le traduisaient, aussi les employés des hôpitaux généraux ne faisaient-ils aucun effort pour les garder. Au bout d'un siècle d'expériences cela conduira les Intendants du roi à créer à partir de 1768 à l'intention de cette population les \emph{dépôts de mendicité}, dépôts qui seront à l'origine des futures \emph{prisons départementales}%
% [11]
\footnote{Leur histoire est complexe et s'étend sur une bonne part du \siecle{19}. Voir entre autres : \emph{Lieux d'hospitalité : hospices, hôpital, hostellerie}, ouvrage collectif sous la direction d'Alain \fsc{MONTANDON}, P.U. Blaise Pascal, 2001.}% 
.






 En ce qui concerne les enfants les plus jeunes la croyance en la vertu éducatrice et rééducatrice de l'internat est à cette époque à son apogée. Les décideurs n'ont pas encore compris l'importance des relations interpersonnelle (corps à corps et cœur à cœur) dans la construction d'une personnalité d'enfant. Ils n'ont pas plus compris combien est déterminante, pour l'investissement de quelque enseignement que ce soit, la différence entre le placement en internat scolaire choisi par les parents, et l'internement d'office ordonné contre leur gré par une instance administrative ou judiciaire. Ils n'ont pas compris non plus la différence qui existe entre la prise en charge des enfants sans famille (orphelins ou abandonnés) qui ni les uns ni les autres n'ont plus de parents, et celle des enfants qui connaissent leurs parents mais à qui on prétend interdire de s'identifier à eux. 


\subsection{Enfants trouvés et abandonnés}

Les enfants abandonnés pris en charge par les institutions d'assistance pouvaient avoir été déposés dans un lieu public ou dans le « tour » d'un hôpital, ou confiés par leur père ou leur mère, ou volontairement « perdus » par eux dans un lieu inconnu%
%[18]
\footnote{L'histoire du \emph{Petit Poucet}, racontée par \fsc{Perrault} dans les \emph{Contes de ma mère l'oye} (1697) a parfois correspondu à une réalité, pour des enfants très jeunes incapables de dire de quelle commune ils venaient ni comment s'appelaient leurs parents.}% 
. Beaucoup de nouveaux-nés étaient abandonnés par leurs mères dans les services d'accouchement des hôpitaux, que seules fréquentaient les indigentes qui ne pouvaient accoucher à leur propre domicile ni chez une sage-femme. D'autres tout-petits n'étaient pas abandonnés à proprement parler. Il s'agissait par exemple d'enfants dont les pères ou/et mères étaient incarcérés dans les « \emph{lieux de force} » (dont la prison pour femmes de \emph{La Force} qui faisait partie de l'hôpital de la Salpêtrière) pour vagabondage, prostitution ou autres actes de délinquance, et qui ne pouvaient donc pour un temps s'occuper d'eux. Dès que l'incarcération durait un temps significatif (un an ?) la restauration des droits parentaux devenait impossible. 

 À part ce cas les enfants abandonnés pouvaient être repris par leurs parents. Il fallait évidemment que leur abandon n'ait pas été anonyme pour que ce retour soit possible. En fait ces \emph{retours en famille}étaient rares, les causes de l'abandon, et d'abord la misère, persistant dans la plupart des cas.
 De nombreux enfants entraient à l'Hôpital bien après leur petite enfance : « \emph{... dans la généralité de Lyon, le plus grand nombre d'enfants présentés aux hôpitaux par leurs parents ont une dizaine d'années…} » À cet âge la plupart des enfants « de famille » travaillaient déjà. Ceux qui étaient confiés à l'hôpital étaient donc souvent ceux qui étaient jugés inaptes au travail. Certains d'entre eux se présentaient d'eux-mêmes à l'hôpital. 

 Au-dessous de 4 à 5 ans les enfants de l'Hôpital sont placés en nourrice. Une fois finie la petite enfance, le placement en institution est préféré. Les administrateurs croient que leurs Hôpitaux offrent des possibilités d'éducation nettement supérieures à une famille nourricière, pour des raisons variées, dont la modestie du niveau culturel des nourrices et de leur maris, qui sont le plus souvent paysans ou ouvriers agricoles, et parce que l'hôpital fournit une scolarité qu'on ne trouve pas à la campagne. Ils estiment aussi que les possibilités de trouver un emploi sont plus grandes en ville. Peut-être ne se sentent-ils pas non plus le droit de déraciner pour toujours des jeunes citadins en les laissant vivre à la campagne, surtout s'ils ont de la parenté dans la ville ? Mais il faut aussi tenir compte du fait que le prix de journée de l'hôpital est à l'époque nettement inférieur au salaire d'une nourrice.

 Tous les enfants de 6 ans et plus, non placés chez un maître artisan ou un nourricier, vivent dans les murs de l'hôpital. Même quand ils ont une famille, les enfants placés en sont plus ou moins radicalement coupés, \emph{même quand leurs parents sont placés dans le même établissement}. Les clôtures internes de l'hôpital sont aussi hautes que son mur d'enceinte%
% [19]
\footnote{Il n'est pour en être persuadé que de visiter la chapelle de l'Hôpital de La Salpêtrière.} 
. Pour nombre d'enfants cette coupure est définitive. 

 En dépit d'un souci éducatif certain%
% [20] 
\footnote{... manifesté à Paris par 5 heures 30 d'enseignement par jour, durant six jours par semaine, ce qui n'a rien à envier aux écoles primaires d'aujourd'hui... mais aussi un nombre d'élèves très élevé pour un seul maitre.} 
l'encadrement humain des jeunes placés est extrêmement réduit (d'où la modestie du prix de journée), ce qui contraint les relations entre les jeunes et les adultes à être formelles, distantes et souvent impersonnelles%
%[21]
\footnote{Selon l'expression de Maurice \fsc{CAPUL} : \emph{pour les pauvres, les moyens de la pédagogie étaient pauvres}.}% 
. Contrairement aux jeunes « de famille » inscrits par leurs parents dans les collèges contemporains, il ne s'agit pas d'intégrer ces jeunes à la « grande » culture ni de leur donner les moyens de penser plus ou moins librement : il s'agit seulement, comme dans les petites écoles, de leur donner les rudiments de la lecture et de l'écriture, et d'en faire de bons pauvres.

\subsection{« Correctionnaires »}

Les mineurs « correctionnaires » sont les jeunes qu'il faut « corriger », ceux dont les comportements font problème, c'est-à-dire les délinquants, rebelles et opposants : mineurs condamnés par décision de justice, faux saulniers de moins de 14 ans, vagabonds, mendiants, prostitué(e)s, « enfants de bohême ». Les enfants au dessus de 6 ans sont soumis aux mêmes règles de droit que les adultes. Dès l'âge de 8 ou 10 ans la peine de mort peut leur être appliquée si une « malignité » exceptionnelle justifie de les exclure du bénéfice de l'excuse de minorité. Les jeunes délinquants sont ordinairement condamnés à un temps d'incarcération déterminé : de quelques mois à 20 ans et plus. Mais ils peuvent aussi être enfermés pour une durée indéterminée : aussi longtemps que l'administration estimera qu'ils ne seront pas suffisamment amendés, jusqu'à leurs 25 ans et plus. Les jeunes garçons condamnés aux galères pour des délits commis sans l'excuse de minorité ne peuvent y être envoyés avant leurs 15 ou 16 ans. Ils attendent donc à l'hôpital d'avoir atteint l'âge d'aller au bagne, soumis au régime des autres correctionnaires, mais le temps qu'ils passent à l'hôpital ne compte pas comme temps d'exécution de la peine

\subsection{« Religionnaires »}

À partir de la \emph{Révocation de l'Édit de Nantes} (1685) ce terme désigne les enfants des protestants rebelles à la conversion au catholicisme qu'on exige d'eux%
%[22]
\footnote{L'Angleterre avait précédé la France dans la persécution des dissidents religieux et leur exclusion de toutes les charges et fonctions officielles. La légitimité des mariages des protestants n'est plus reconnue, ce qui fait de leurs enfants des bâtards incapables d'hériter. Ils se voient retirer leurs droits parentaux. Pour cette raison dès l'âge de sept ans leurs enfants leur sont enlevés. C'était l'application stricte du principe \latin{cujus regio, cujus religio}, « {un roi, une foi, une loi} ». Il faudra attendre la fin du \siecle{18} pour que la tolérance apparaisse comme une vertu et non comme une faiblesse.}

 À partir de cette date il est demandé aux hôpitaux généraux d'enfermer et rééduquer les membres de la « \emph{religion prétendue réformée} » (RPR) si aucune autre solution n'est possible. Les enfants de ceux qui ne peuvent payer sont placés en hôpital général, avec les correctionnaires. Les autres sont placés aux frais de leurs parents dans une section de correction d'un collège (catholique comme tous les collèges du royaume à partir de la Révocation), avec les enfants indisciplinés ou récalcitrants des mêmes milieux sociaux qu'eux. Ils y sont soumis à une pression morale ouverte ou insidieuse, brutale ou habile, pour les pousser à abjurer la religion de leurs parents et à se convertir au catholicisme. Leur sortie de l'hôpital ou du collège dépend en grande partie de leur « conversion ». 

 Selon Maurice \fsc{CAPUL}, cette politique a été poursuivie activement de 1685 au milieu du \siecle{18}, en dépit du fait qu'elle ne donnait que des résultats insatisfaisants : selon les observateurs du temps elle produisait des adultes peu consistants, qui ne savaient plus à quoi ils croyaient, ou des sceptiques qui ne croyaient plus à rien. D'autre part elle jetait la discorde au sein des familles et la brouille entre les parents et les enfants. Elle va se déliter peu à peu après le milieu du \siecle{18}, mais ce n'est qu'en 1787 que \emph{l'Édit de Versailles} y met un terme en créant un état-civil laïque, qui rend aux enfants protestants leur légitimité. Le roi prend officiellement acte de la tolérance dont le culte protestant avait fini par bénéficier à cette date%
% [23] 
\footnote{Depuis l'affaire Calas (condamné par le parlement de Toulouse à être roué, exécuté en 1762) et l'intervention de Voltaire (qui avait entrainé sa réhabilitation en 1765) la répression du protestantisme s'était adoucie : dans l'opinion publique la légitimité avait changé de camp.} 
dans la réalité quotidienne. Les dispositions de cet édit concernent aussi les français de confession juive.





%H1 Contestation des familles par les Lumières
%H2 La révolution française et les familles
%I1 La famille du Code Napoléon
%I2 La police des familles au XIXème siècle
%J1 3ème et 4ème républiques, principales décisions dans le domaine des familles
%J2 Séparation des Eglises et de l'Etat
%J3 Contestation de la famille du Code Napoléon
%J4 L'Etat, providence des familles ? 

\part{Des Lumières au baby-boom}

% Le 28.02.2015 :
% Antiquité
% Moyen Âge
% _, --> ,
% Le 24.02.2015 :
% ~etc.
% Moyen-Âge
% ~\%


\chapter[Contestation des autorités établies par le mouvement des Lumières]{Contestation des autorités établies par le mouvement des Lumières}


 Les penseurs qui se réclamaient des « lumières » de la raison, et qui entendaient tout leur soumettre se sont attaqués à l'argument d'autorité et à tous les dogmes, à « l'obscurantisme ». Leur audience est allée croissant au fur et à mesure qu'avançait leur siècle, et surtout à partir de 1760-1765, moment charnière d'une grande importance. À partir de cette date, et du moins dans la population « éclairée », un certain nombre de faits se mettent à poser insupportablement problème, et des solutions jusque là inenvisageables deviennent évidentes. En lien avec ces phénomènes on assiste à partir du milieu du \siecle{18} au décollage économique d'une France qui se développe à grands pas%
% [1]
\footnote{Sources principales :
\\José \fsc{CUBERO}, \emph{Histoire du vagabondage du Moyen Âge à nos jours},1998.
\\Jacques \fsc{DONZELOT}, \emph{La police des familles}, 1977.
\\Bronislaw \fsc{GEREMEK}, \emph{La potence ou la pitié, l'Europe et les pauvres du Moyen Âge à nos jours}, 1987.
\\Jack \fsc{GOODY}, \emph{L'évolution de la famille et du mariage en Europe}, 1985.
\\Jean-Philippe \fsc{LÉVY} et André \fsc{CASTALDO}, \emph{Histoire du droit civil}, 2002.}% 
.

 
\section{Contestation de l'autonomie de l'Église}

 Outre le service du culte dans les quarante mille paroisses du pays, l'Église subvenait aux besoins de l'assistance (hôpitaux, hospices, enfants des hôpitaux placés en nourrice, et une part notable de l'assistance au domicile) et de l'enseignement (petites écoles, collèges et universités), dont une assez grande part était gratuite. Depuis la fin de l'Antiquité, les prêtres, religieux et religieuses fournissaient la majeure partie du personnel des hôpitaux, des collèges et des universités. Les hôpitaux étaient fondés comme les couvents et les collèges, dans la plupart de cas sur des initiatives individuelles. 

 Sauf exception leurs ressources étaient similaires : ils vivaient des revenus de biens en capital reçus de leurs bienfaiteurs (le plus souvent des personnes privées) et ressortissant du régime juridique et fiscal des biens ecclésiastiques (biens de \emph{mainmorte}). L'Église possédait%
% [2] 
\footnote{Cf. François \fsc{BLUCHE}, \emph{L'ancien régime, institutions et sociétés}, 1993, p. 71-72.} 
les églises, les cures, et les bâtiments nécessaires à l'activité des hôpitaux, collèges et universités. Elle possédait aussi des biens de toute nature (7~\% des terres du royaume, maisons de rapport,~etc.) dont les revenus subvenaient aux besoins de fonctionnement de ces diverses institutions. 

 Au fil du \siecle{18} ce modèle a été de plus en plus sévèrement critiqué. L'opinion publique considérait que la gestion des biens ecclésiastiques était négligente et entachée d'amateurisme. Il existait de grandes disparités de revenus entre communautés religieuses : certaines étaient plus riches que nécessaire pendant que d'autres vivaient dans une gêne extrême. Elle dénonçait le tribut prélevé sur les revenus de ces biens par les rentes de situation, les emplois fictifs, et d'abord le plus criant, c'est-à-dire le système de la \emph{commende}.

 Dans ce système mis en place à la Renaissance, le roi de France avait obtenu du Pape que les revenus d'une abbaye ou d'un couvent (rarement ceux d'un hôpital), soient attribués à un \emph{bénéficier} de son choix au même titre que les autres \emph{bénéfices} ecclésiastique (évêchés et cures%
% [3]
\footnote{C'était une façon pour le roi de récupérer les revenus des biens qui avaient été donnés par les autorités civiles aux ordres religieux, une façon déguisée de soumettre les religieux à une imposition, alors qu'ils étaient théoriquement non imposables. Cf. \fsc{MINOIS}, 1989.}% 
). Il n'était pas nécessaire que le \emph{bénéficier} appartienne à la maison concernée, ni qu'il y réside, et il n'y exerçait aucune autorité spirituelle. Il fallait et il suffisait qu'il soit homme et tonsuré, ce qui ne l'engageait à rien en termes de vie religieuse, à part le port de la tenue ecclésiastique et l'interdiction de se marier (ce qui ne voulait pas dire faire voeu de chasteté). Par définition le bénéficiaire de cette nomination n'avait pas non plus fait vœu de pauvreté. Il suffisait qu'il laisse aux moines de quoi vivre, après quoi il pouvait consommer tout le reste. L'abbé commendataire avait financièrement intérêt à ce qu'il y ait le moins de moines ou de religieuses possible, et à minimiser les dépenses d'entretien et toutes les aumônes aux pauvres et autres dépenses improductives, tandis que de leur côté les religieux avaient intérêt à tirer de leurs biens le maximum de revenus pour qu'il leur en reste assez après le prélèvement du commendataire, ce qui les poussait à être exigeants face à leurs fermiers et locataires. 

 Par ailleurs on reprochait aux fondations religieuses d'être trop nombreuses et d'accaparer sans cesse plus de biens puisqu'elles n'avaient pas d'héritiers, ce qui donnait à leurs gestionnaires un pouvoir d'influence excessif sur la société, au détriment parfois des objectifs des autorités civiles et de l'intérêt commun, en stérilisant une part excessive de la richesse nationale. Les économistes de cette époque pensaient que le total des richesses existantes était fixe, inextensible. La croissance d'une famille nouvelle (charnelle ou spirituelle) impliquait donc à leurs yeux l'appauvrissement de toutes les autres : \emph{"un des principaux objets de notre attention, ce sont les inconvénients de la multiplication des établissements des gens de mainmorte et la facilité qu'ils trouvent à acquérir des fonds naturellement destinés à la subsistance et à la conservation des familles, \emph{[...]} qui ont souvent le déplaisir de s'en voir privées \emph{[...]} en sorte qu'une très grande part des fonds de notre moyenne se trouve actuellement possédés par eux ..."} (Édit du 25 Août 1749). 

 Mais ces problèmes patrimoniaux n'avaient rien de nouveau. S'ils ont été mis en avant avec détermination durant la seconde moitié du \siecle{18}, c'est que le monopole de l'Église sur les fonctions d'assistance et d'enseignement n'allait plus de soi : son autorité morale était contestée avec vigueur. 


\section{Contrôle des autorités civiles sur les congrégations religieuses}

 Il semble, sans qu'on puisse en faire une règle générale, que beaucoup de monastères aient été dans un état pitoyable à partir du milieu du \siecle{18} : effectifs squelettiques, ferveur discutable ou absente, indiscipline, non respect des règles de l'ordre... Les causes sont sans doute nombreuses et l'état général des esprits à l'époque des lumières n'inclinait peut-être pas à la vie contemplative. D'autre part l'utilisation traditionnelle des monastères au service de la régulation des familles était en train de tomber en désuétude. Si les critiques contre les vocations forcées, aussi anciennes que le phénomène lui-même, pouvaient enfin être entendues c'est peut-être que les pères de famille n'avaient plus autant besoin qu'auparavant de l'Église pour caser leurs enfants surnuméraires, soit que les pratiques de limitation du nombre d'héritiers qui se répandaient rapidement à cette époque%
% [5] 
\footnote{Banalisation des abandons. Recours aux méthodes de prévention des naissances disponibles alors : \emph{coïtus interruptus} certainement, douches vaginales,~etc.} 
aient diminué le nombre des enfants à établir, soit que le développement économique de la fin de l'Ancien Régime ait offert aux cadets de famille des perspectives plus alléchantes que l'entrée en religion ?

 Est-ce pour cela que commençait de paraître scandaleuse%
% [6] 
\footnote{Le roman \emph{La religieuse} de \fsc{DIDEROT} paraît en 1782 : il exprime cet état d'esprit. Cette \emph{effrayante satire des couvents}, selon l'auteur lui-même, raconte l'histoire d'une fille contrainte par sa famille à prendre le voile et à vivre dans un couvent aux modes de vie et de penser terrifiants. Une de ses propres sœurs était religieuse.} 
l'idée qu'on puisse à vingt ans aliéner sa liberté de manière définitive en prononçant des vœux perpétuels, sans tenir compte des évolutions psychologiques et intellectuelles qu'une vie peut entraîner ? Les \emph{philosophes} n'épargnaient d'ailleurs pas davantage l'indissolubilité du mariage, qui leur paraissait une oppression du même ordre. Mais dans le cas des religieux cela leur paraissait d'autant plus monstrueux, que depuis la fin de l'Antiquité le droit privait ceux-ci de tous leurs droits familiaux et en faisait des morts civils. 

 D'autre part il est symptomatique que le vœu d'obéissance à une autorité étrangère (le Pape) ait été l'un des griefs principaux formulés contre les jésuites. À une période où il s'est peut-être créé plus d'internats éducatifs qu'à aucune autre, la société civile (par l'intermédiaire du Parlement de Paris) entendait exercer un contrôle sur les contenus de l'enseignement, et ne plus laisser les mains libres à des corps de spécialistes comme les jésuites, qui se situeraient au-dessus de la nation ou de l'État (ou du roi), même au nom d'une légitimité religieuse supranationale. On reprochait aussi au mode de vie des religieux d'être inadapté aux institutions d'éducation ou d'assistance. À partir de leur expulsion en 1761, les nombreux collèges des jésuites ont été repris en main par les représentants du pouvoir civil, qui les ont confié aux ecclésiastiques de leur choix. Le mandat qu'ils entendaient donner aux nouveaux professeurs des collèges était de préparer les jeunes « de famille » à vivre dans le siècle, non à l'ombre des cloîtres. Il ne s'agissait plus d'en faire des clercs. Ils entendaient que le monde contemporain, avec ses réalités matérielles et ses techniques profanes, soit introduit dans les internats éducatifs%
% [7]
\footnote{C'est à cette époque que sont créées les premières grandes écoles, puisque l'université refusait de développer en son sein les enseignements techniques de niveau supérieur dont la société d'alors commençait à avoir un besoin impérieux (exemples : Chirurgie, Ponts et Chaussées, Mines,~etc.).}%
.

 L'aspiration des auteurs de la fin du \siecle{18} à la maîtrise de soi, comme leur amour de l'ordre, étaient aussi grands que ceux de leurs prédécesseurs : ce qui changeait, c'est qu'ils fondaient leurs projets sur une représentation idéalisée des républiques antiques et non plus sur saint Augustin. Ce qui changeait, c'était la valorisation du modèle militaire%
% [8] 
\footnote{Les premières écoles militaires datent aussi de cette période. Les casernes deviennent des lieux de dressage rationnel (cf. le « \emph{drill} » prussien) qui transforme une « piétaille » indisciplinée et timorée en une machine de guerre efficace.} 
aux dépens du modèle monastique. Il ne s'agissait plus de former des âmes pour le service de Dieu mais de former des corps et des caractères pour la Cité, l'État, la Nation. Ils parlaient de vertu « spartiate », « romaine », « républicaine » ou « citoyenne » ...

 Le premier juin 1739, sur requête du Parlement de Metz, Louis~XV publiait un édit spécial à destination de la Lorraine {[...] \emph{pour empêcher que par des voies indirectes on ne fasse de nouveaux établissements sans autorisation, soit pour empêcher les communautés autorisées de faire sans permission de nouveaux acquêts}}. Il interdisait qu'aucune donation, qu'aucun legs et qu'aucune rente ne soit plus acceptés à l'avenir par une communauté religieuse sans permission, et qu'aucune acquisition ne soit faite par une communauté, qu'aucune communauté nouvelle ne soit créée sans une enquête préalable \emph{de commodo et incommodo}. Il défendait à quiconque de se faire prête-nom pour des religieux. Il donnait droit aux gens lésés par des dons ou des ventes non autorisés de réclamer leurs biens aux communautés concernées,~etc. En 1749, Louis~XV étendait par édit ces règles à l'ensemble du royaume : \emph{il ne sera fait aucun nouvel établissement, chapitre, séminaire, communauté religieuse quelconque même sous prétexte d'hospice, de quelque qualité que ce soit, sans permission expresse par lettres patentes enregistrées}. Il s'agissait de distinguer entre les fondations d'intérêt public (dont les hôpitaux et les hospices, mais seulement au cas par cas, et s'ils étaient approuvés par les autorités civiles) et les autres (couvents et monastères divers). 

 Les autorités civiles contestaient à l'Église le droit de s'opposer à l'autorité de l'État, garant de \emph{l'intérêt général} et de la tranquillité publique, ainsi que le montre l'arrêt du conseil du 24 mai 1766 : [considérant que] \emph{s'il appartient à l'autorité spirituelle d'examiner et d'approuver les instituts religieux dans l'ordre de la religion ; si elle seule peut consacrer les vœux, en dispenser ou en relever dans le for intérieur, la puissance temporelle a le droit de déclarer abusifs et non véritablement émis les vœux qui n'auraient pas été formés suivant les règles canoniques et civiles, comme aussi d'admettre ou de ne pas admettre les ordres religieux suivant qu'ils peuvent être utiles ou dangereux dans l'État, même d'exclure ceux qui s'y seraient établis contre lesdites règles ou qui deviendraient nuisibles à la tranquillité publique,~etc.}

 Dans le même esprit, de 1766 à 1784 la \emph{Commission royale des Réguliers} supprimait 458 monastères et couvents%
% [9] 
\footnote{François \fsc{BLUCHE}, idem, p. 68.} 
sans en référer à Rome et en dépit de l'opposition d'une grande partie du clergé français. Sur son instigation, un édit royal de 1773 supprimait \emph{l'exemption} qui depuis le haut Moyen Âge interdisait aux évêques (nommés par le Roi, contrôlés par lui, et qui donc le représentaient) d'exercer leur autorité sur tous les couvents et monastères de leurs diocèses. Il leur confiait la mission de les contrôler. 

 En 1768 la Commission royale des Réguliers repoussait à 21 ans (18 ans pour les filles) l'âge à partir duquel les postulants avaient le droit de prononcer des vœux solennels. L'objectif était de protéger la liberté des jeunes gens contre toutes les formes d'oppression : celle des pères était sans doute visée au moins autant que celle des couvents qui n'étaient généralement pas à un ou deux ans près, et qui avaient peu à gagner à s'encombrer de membres malheureux, aigris ou révoltés.

 Par ailleurs à la fin de l'ancien régime la part d'héritage donnée aux postulants religieux a été limitée par les autorités civiles pour empêcher des surenchères, des luttes de prestige, et pour ne pas immobiliser trop d'argent dans des institutions en principe vouées à la pauvreté. Les familles pouvaient se borner à payer une pension viagère pour leur fils ou fille, sans qu'il soit plus question d'accroître définitivement le capital du couvent.

 Cela étant dit les relations entre les autorités civiles et le monde religieux ne se résumaient pas à ces antagonismes. C'était bien plus complexe, et les religieux ne venaient pas d'un autre monde, au contraire ils étaient en majorité issus des couches aisées et bien insérées de la population. En 1761, c'est à des ecclésiastiques séculiers que les parlementaires confient les collèges dont ils venaient d'expulser les jésuites. Ils partageaient la même sympathie pour les thèses jansénistes et gallicanes. Ces clercs, soumis aux évêques, soumis eux-mêmes au roi, avaient toute leur confiance. De la même façon {Tenon} exprimait en 1788 toute son estime pour les religieuses hospitalières que sa carrière l'amenait à côtoyer jour après jour dans les hôpitaux de Paris. Alors qu'il participait de sa place au mouvement de réflexion, de rationalisation et de modernisation des Lumières, il n'imaginait pas un instant se passer de leurs services. 


\section{Contestation de la puissance des pères}

 En conséquence du retour au droit romain à partir du \siecle{12}, la puissance paternelle avait été restaurée dans toute sa force à partir de la fin du Moyen Âge, dans les pays de droit écrit surtout, mais aussi dans le reste de la France%
% [10]
\footnote{Cf. \emph{Histoire des pères et de la paternité}, Collectif, 1990, édition 2000.}% 
. Aussi longtemps que le père vivait il conservait sa puissance de décision dans les domaines essentiels de la vie de ses enfants, même devenus adultes (mariage, achats et ventes de pièces du patrimoine...). 

 Mais le roi soleil, qui avait donné un éclat incomparable à la monarchie de droit divin, avait également révoqué l'édit de Nantes, et c'est lui qui avait ordonné les persécutions qui s'en étaient suivies pendant des générations contre les membres de la « {religion prétendue réformée} » (RPR). C'est donc lui qui avait attaqué la fonction paternelle dans la personne de ceux qu'il avait disqualifiés aux yeux de leurs propres enfants en ne reconnaissant pas leurs unions conjugales comme légitimes. Cela faisait de ces enfants des bâtards et les empêchait d'hériter des biens de leurs parents, ce qui gênait beaucoup leur établissement dans la vie. C'est lui qui avait enlevé aux parents réformés le droit d'élever leurs enfants en émancipant ces derniers dès l'âge de raison (7 ans). En faisant tout cela il avait placé les représentations idéologiques et (surtout ?) le pouvoir de l'État au-dessus des pères qu'il prétendait pourtant défendre : il s'était conduit comme un père abusif%
% [11]
\footnote{Maurice \fsc{CAPUL}, \emph{Infirmités et hérésies, les enfants placés sous l'ancien régime} (tome II), 1989, 1990.}% 
.

 Au contraire les auteurs des Lumières voulaient que les pères soient au service de l'épanouissement de leurs enfants, et que leur autorité ne s'exerce que durant le temps où ces derniers étaient incapables de se conduire seuls. Ils voulaient qu'ils appuient leur autorité sur l'affection plutôt que sur la crainte. En 1762 paraissait \emph{L'Émile} et son succès était immédiat, ce qui prouve combien ce roman était en accord avec l'air de son temps. Jean-Jacques \fsc{ROUSSEAU}%
% [12] 
\footnote{Compte tenu de leurs expériences personnelles, ni Rousseau ni Voltaire ni D'Alembert ne pouvaient supporter que puisse exister quelque chose comme un droit supérieur, divin, des pères. De même plusieurs des personnages emblématiques de la Révolution Française ont eu maille à partir avec leur père et avec le droit de correction paternelle tel qu'il pouvait s'exercer sous l'Ancien Régime : Mirabeau, Sade...} 
y proposait une nouvelle image de l'enfance et des rapports parents--enfants. 

 Le regard que les « {philosophes} » portaient sur l'enfance avait-il pour autant radicalement changé ? Même s'ils ne parlaient plus en termes de péché, les pédagogues et philosophes du \siecle{18} ne montraient pas beaucoup plus de vraie confiance en la bonté \emph{naturelle} des enfants que ceux des siècles précédents. Pour \fsc{ROUSSEAU} l'enfant nouveau-né ne portait plus la marque d'un quelconque péché originel, mais il le jugeait sans défenses face aux tentations et trop aisément corruptible par la société, c'est-à-dire d'abord par son entourage immédiat. Cela se traduisait dans \emph{l'Émile} par une pédagogie aussi peu spontanée et naturelle que l'internat le plus contrôlé. Et l'incroyable phobie de la masturbation masculine et féminine qui a régné à partir du \siecle{18} et jusqu'aux années trente du \crmieme{20}, phobie qui a fait déraisonner tant de médecins et de spécialistes de l'éducation, ne se fondait pas sur des raisons religieuses, que ces autorités qui se voulaient scientifiques récusaient et n'auraient jamais reconnues. Par contre elle fournissait aux parents et aux éducateurs tous les arguments légitimes, fondés en raison, pour exercer sur les enfants et adolescents et sur leurs premiers pas dans la découverte de leurs corps et leur sexualité une surveillance intrusive. Il faudra attendre \fsc{FREUD} pour que changent les représentations.


\section{Banalisation des abandons}

 Si l'on décomptait \nombre{312} abandons à Paris en 1670, du temps de Monsieur Vincent de Paul, on en dénombrait \nombre{5842} en 1790, pour environ \nombre{600000} habitants. Ils représentaient 40~\% des naissances parisiennes en 1772, et 33 à 34~\% à la veille de la révolution%
% [13]
\footnote{Pour l'ensemble de la France de 2010 de tels taux donneraient un nombre d'abandon de l'ordre de \nombre{600000} enfants (six cent mille), soit trois fois plus que le nombre actuel d'IVG, et au bas mot 600 fois plus que le nombre d'abandons actuels.}%
. À la fin du \siecle{18} les mœurs ont donc beaucoup changé.

 Pour être juste il faut dire aussi que le nombre des abandons dans les villes était artificiellement gonflé et cela d'autant plus qu'elles étaient grandes. À Paris c'était clairement le cas. On y envoyait des enfants de plusieurs \emph{centaines} de kilomètres à la ronde. Néanmoins la croissance du nombre et du pourcentage des abandons depuis l'époque de Monsieur Vincent était indiscutable et massive. La grande période de l'abandon d'enfant, la période où il a été utilisé de la façon la plus massive et la moins contestée, se situe entre 1760 et 1860%
% [14]
\footnote{... en 1810 : \nombre{55700} abandons sur toute la France ; en 1833 : \nombre{164000} abandons.}%
.

 À cette époque, le poids des interdits religieux avait diminué, au moins dans les villes et dans certaines campagnes, dont celles du bassin parisien, ce qui facilitait à la fois les relations sexuelles hors mariage et le refus des géniteurs masculins de « réparer » en cas de grossesse, comme c'était l'usage jusque là quand un mariage, des vœux religieux ou l'inégalité des conditions ne s'y opposaient pas. La croissance des villes et des fabriques, ateliers, mines et autres industries nouvelles concourait au relâchement de la pression sociale sur les comportements individuels. Le nombre des naissances hors mariage non légitimées par mariage subséquent avait donc augmenté. 

 Mais même dans les couples stables, concubins ou mariés, le recours à l'abandon s'était généralisé. Les scrupules religieux avaient cessé de le freiner. On s'était mis à pratiquer l'abandon des nouveaux-nés dans tous les milieux et de plus en plus souvent à visage découvert. L'abandon était devenu un droit pour tous, exercé sans honte, sans questions, sans enquête, sans formalités, sans poursuites, même en dehors des cas de nécessité vitale, et c'est ce qui était nouveau. C'était devenu un moyen comme un autre de régulation des familles. 

 Le grand public culpabilisait d'autant moins l'abandon qu'il croyait en \emph{la bonté de l'éducation donnée par les hôpitaux}. \fsc{Rousseau} explique dans ses \emph{Confessions} que s'il a abandonné ses cinq enfants, c'est parce que \emph{tout pesé, je choisis pour mes enfants le mieux ou ce que je crus l'être. J'aurais voulu, je voudrais encore avoir été élevé et nourri comme ils l'ont été}. Il croyait que les nouveaux-nés abandonnés avaient de réelles chances de survie. Cette croyance semble à l'époque avoir été très largement partagée. Les administrateurs des hôpitaux étaient presque les seuls à savoir combien la réalité était loin de cet idéal. 

 Les enfants pouvaient être abandonnés à tout âge, et le lien était évident entre le nombre des abandons et les crises économiques. Le manque de ressources des parents, leur maladie, le chômage, ou encore le veuvage, expliquent que des enfants n'étaient pas abandonnés à la naissance, mais après un certain nombre de mois ou d'années. Les mères seules étaient dans une situation économique particulièrement fragile : à travail égal les femmes étaient \emph{beaucoup} moins bien payées que les hommes. 

 Parmi les nouveaux-nés des villes placés en nourrice%
% [15] 
\footnote{La grande majorité des enfants des villes, même non abandonnés, vivaient alors leurs premières années en placement nourricier rural : \emph{1780 : Le lieutenant de police Lenoir constate, non sans amertume, que sur les \nombre{21000} enfants qui naissent annuellement à Paris, \nombre{1000} à peine sont nourris par leur mère. \nombre{1000} autres, des privilégiés, sont allaités par des nourrices à demeure. Tous les autres quittent le sein maternel pour le domicile plus ou moins lointain d'une nourrice mercenaire. Nombreux sont les enfants qui mourront sans avoir jamais connu le regard de leur mère. Ceux qui reviendront quelques années plus tard sous le toit familial découvriront une étrangère : celle qui leur a donné le jour.} (cité par Élisabeth \fsc{BADINTER}). Les propos du lieutenant de police montrent aussi qu'en 1780, les bébés et l'allaitement maternel font désormais partie des sujets de préoccupation légitimes d'un haut fonctionnaire.} 
à la campagne par leurs parents, un certain nombre entraient dans la catégorie des enfants abandonnés si leurs parents ne payaient plus les gages de la nourrice. Lorsque celle-ci n'obtenait pas de réponse à ses réclamations, il allait de soi qu'elle remettait l'enfant à l'hôpital le plus proche : elle n'avait pas reçu de mandat pour faire autre chose, et elle avait besoin de son salaire. 


\section{Valorisation de l'éducation familiale et maternelle}

 Traditionnellement les enfants placés en nourrice par les hôpitaux y revenaient quand leur petite enfance était achevée. Mais dès 1696 le \emph{bureau de l'Hôpital} observait que : [...] \emph{les enfants qu'on ramène à 4 ans à Paris s'accoutument mal à l'air de la capitale et qu'il en meurt beaucoup. On pense qu'il serait bon de les laisser un an de plus à la campagne...} C'est pourquoi au fil du \siecle{18} leur séjour à la campagne s'est prolongé.

 De nouveaux règlements sont édictés en 1761 par l'Hôpital des Enfants Trouvés de Paris%
% [16]
\footnote{Comme celui-ci exerce un rôle de modèle national puisqu'il reçoit le tiers des indigents du royaume, et que le roi suit de très près ce qui se passe dans la ville dont il est le seigneur, ces règlements vont avoir une postérité importante.}% 
. En ce qui concerne ces enfants-là, le placement dans des familles nourricières est désormais mis sur le même pied que l'internat de l'hôpital. On lui reconnaît une valeur au moins égale, au nom de la vie qu'il permet de sauver. 

 À cette époque le sort ordinaire des enfants ordinaires était de commencer très tôt à travailler chez leurs parents ou chez le maître où ceux-ci les avaient placés : souvent dès l'âge de 6 ans. Dans les villes l'entrée au travail attendait dans les meilleurs cas l'issue de la scolarité dans une petite école, scolarité qui durait fort peu de temps. Si le jeune était placé chez un maître, celui-ci avait une large délégation de l'autorité parentale. Il en était ainsi depuis l'Antiquité pour la majorité de la population, pour tous les humbles. Le sort des enfants placés en nourrice paraissait donc naturel et normal, à défaut d'être désirable.

 Il s'agissait d'insérer l'enfant dans un milieu naturel rural ou artisanal, et si possible de lui donner une famille, même si l'adoption demeurait impensable et impossible. Comme l'observera en 1790 \fsc{LA ROCHEFOUCAULT-LIANCOURT}, du Comité de Salut Public, une génération après la prise de décision de ne pas ramener ces jeunes à l'hôpital : \emph{presque tous les enfants conservés par les nourrices sont gardés dans leurs maisons jusqu'à ce qu'ils se marient, y sont traités comme leurs propres enfants, le plus grand nombre tourne bien et ils deviennent de bons habitants des campagnes}. Les enfants placés en nourrice restaient les enfants de l'hôpital, employeur des nourrices, qui avait pleine autorité sur eux, et exerçait l'autorité paternelle jusqu'à leur majorité (25 ans) ou leur mariage (pour les filles). 

 D'emblée ce système a été jugé satisfaisant, mis à part le fait que beaucoup de garçons avaient tendance à s'en aller avant d'avoir eu 25 ans, pour gagner de l'argent. D'autre part un certain nombre de garçons étaient \emph{... renvoyés par le nourricier}. Comme toujours les filles posaient nettement moins de problèmes de discipline que les garçons. 

 Désormais l'objectif était de faire grandir un futur sujet pour le service du roi et de l'État. Il était encore moins question qu'auparavant d'enlever systématiquement leurs enfants aux indigents pour les placer dans un Hôpital coûteux et à la valeur éducative douteuse. À un moment de crise économique (1770) le ministre Turgot a ordonné qu'on mette en place dans chaque paroisse un \emph{bureau d'aumône}, ou \emph{bureau de charité}, à l'intention des pauvres domiciliés, et d'eux seuls, avec pour mission de redistribuer des taxes levées sur les propriétaires aisés de la paroisse%
% [17]
\footnote{Ces institutions n'existaient alors que dans certaines paroisses, même si selon des décisions vieilles de plusieurs siècles, et jamais abrogées, elles auraient dû exister partout. Inutiles durant les périodes de bonne santé économique, elles étaient de celles qu'il fallait refonder constamment.}% 
. Pour secourir les pauvres dociles, les bons pauvres, les femmes seules chargées de famille, les journaliers au chômage, pour protéger les jeunes filles pauvres et en danger de « se perdre » dans la prostitution, sans pour autant les héberger ni les prendre en charge totalement, il a fait ouvrir, ou plutôt rouvrir, des \emph{ateliers de charité}. Les pauvres y travaillaient comme ils l'auraient fait à l'hôpital, et ils continuaient à vivre à leur domicile, dans leur communauté. Ils n'étaient pas déracinés, ni désocialisés, et cela coûtait moins cher. 

 Pour éviter les abandons, les hôpitaux (ou du moins certains hôpitaux) aidaient financièrement les mères indigentes à nourrir chez elles leur propre enfant. Cela ne leur coûtait pas plus cher que de mettre un enfant abandonné en nourrice, mais en termes de survie c'était bien plus efficace : ainsi la mortalité des bébés de Rouen vivant avec leur mère, alors que celles-ci étaient secourues à domicile par l'Hôpital Général (c'étaient donc des indigentes) ne dépassait pas 18,7~\% entre 1777 et 1789%
% [18]
\footnote{Selon Élisabeth \fsc{BADINTER}.}% 
. À la même période ceux des enfants qui étaient mis en nourrice par leurs parents, avec l'aide matérielle du même Hôpital (c'étaient donc des indigents eux aussi), subissaient une mortalité de 38,1~\%. Quant à ceux qui étaient abandonnés à ce même hôpital, il en mourait plus de 90~\%. À Lyon il en était de même à la même période : les bébés nourris par les mères qui ont été secourues à domicile par le bureau de bienfaisance maternelle n'ont subi de 1785 à 1788 qu'un taux de mortalité de 16~\% avant l'âge d'un an. Ces taux étaient très bons pour l'époque, même comparés à ceux des familles non indigentes dont les mères nourrissaient elles-mêmes : il est vrai qu'il s'agissait d'enfants uniques (sans quoi leurs mères n'étaient plus jugées dignes, « méritantes », de bénéficier d'une telle mesure), qu'elles avaient voulu les garder et les élever, et qu'elles avaient le temps de s'en occuper.

 Des recherches véritablement scientifiques ont été menées afin de diminuer la mortalité infantile en collectivité. À partir de 1784 une expérience a été conduite dans l'une des salles de la Couche de Paris sous l'impulsion et le contrôle du corps médical. Elle avait pour principe d'augmenter le taux d'encadrement et l'intensité des relations des bébés avec les soignants. Cette expérience s'est poursuivie pendant 4 ans (1784-1788) sous le contrôle de l'Académie de Médecine. Elle a conduit à une baisse significative du taux de la mortalité. Aussi avec l'approbation de la même faculté de médecine (1788) ces pratiques ont-elles connu un début de généralisation timide : un tel dispositif était en effet fort coûteux, et la Révolution a suspendu sa mise en œuvre.

 Voici ce qu'écrivait en l'an XI \fsc{CAMUS}, membre du Conseil qui avait dans ses attributions les maisons d'enfants trouvés, dans son \emph{Rapport au Conseil général des hospices sur les hôpitaux et hospices, les secours à domicile, la direction des nourrices} : \emph{Peut-être est-il beaucoup plus difficile de suppléer aux soins de la mère et de la nourrice qu'à leur lait. On est assez avancé dans les connaissances chimiques pour composer une boisson qui ait la qualité du lait de femme, même avec les variations que le lait éprouve pendant la durée de l'allaitement%
% [19] 
\footnote{En réalité à cette date aucun essai n'avait réussi. Tout au plus savait-on à peu près compléter un allaitement insuffisant par des bouillies, et cela ressortait de l'art des mères et des nourrices plus que de l'expertise des médecins.} 
; mais ces tendres soins d'une femme pour l'enfant auquel elle donne une partie de sa substance, cette gestation entre les bras, ces embrassements continus, ces baisers fréquents : en un mot, cette espèce d'incubation qui doit suivre la sortie du sein de la mère, voilà ce qu'on n'obtient ni avec des combinaisons chimiques, ni avec des règlements, ni avec des gages.}%
%[20] 
\footnote{Cité par \fsc{DUPOUX}, idem, p. 136 et 181, 1958.} 




% 28.02.2015 :
% haut Moyen Âge
% _, --> ,
% ~etc.
% Antiquité


\chapter{La Révolution française et les familles}


 La première démarche des représentants de la nation réunis en 1789 a été de rédiger une \emph{Déclaration des droits de l'homme et du citoyen}. Ils ont commencé par refuser les privilèges \emph{et les désavantages} fondés sur les circonstances de la conception, de la naissance, sur le statut des parents, sur la religion ou l'absence de religion. Selon l'article~1 de la Déclaration de 1789, \emph{les hommes naissent et demeurent libres et égaux en droit. Les distinctions sociales ne peuvent être fondées que sur l'utilité commune}. L'article~6 de la même Déclaration précise que \emph{tous les citoyens étant égaux aux yeux de la loi sont également admissibles à toutes les dignités, places et emplois publics, selon leur capacité et sans autres distinctions que celles de leurs vertus et de leurs talents}. En conséquence, personne ne naît plus esclave ni serf, et aucun nouveau-né ne doit être traité différemment des autres, quoi qu'aient pu commettre ses parents et quelles qu'aient pu être les circonstances de sa naissance : conception hors mariage, adultère, inceste,~etc. 


\section{Limitation de la puissance paternelle}

 La législation révolutionnaire sur la famille a une histoire complexe mais ses acteurs étaient d'accord sur l'essentiel. Ils avaient d'abord en ligne de mire la puissance paternelle\footnote{Cf. \emph{l'Histoire des pères et de la paternité}, voir en particulier le chapitre XI (p. 289 à 328) : « La volonté d'un homme » écrit par Jacques \fsc{MULLIEZ}.}. 
Certains d'entre eux allaient jusqu'à affirmer que les enfants appartenaient à l'état avant d'appartenir à leurs parents. Dans le même ordre d'idée les plus radicaux auraient voulu que tous les jeunes soient pris en charge en internat dès l'âge de 5 ans, pour les préserver de l'influence néfaste de leurs parents, suspects d'être « contre-révolutionnaires » ou « obscurantistes », et pour en faire des citoyens conformes à leurs désirs : répéter en somme pour tous les français ce que Louis~XIV avait cherché en vain à faire avec les protestants et autres « déviants ». En fait ces extrémistes étaient peu nombreux. La majorité tenait à ce que la nation contrôle l'éducation de sa jeunesse, mais elle était ouverte à une large liberté de l'enseignement, et en dépit des péripéties plus ou moins chaotiques vécues par certains l'essentiel du corps enseignant en place à la fin de l'ancien régime (en grande partie constitué d'ecclésiastiques) a formé l'armature des écoles privées ou publiques et des collèges de la Révolution.  

 En 1790 l'Assemblée constituante avait aboli les \emph{lettres de cachet}, dont la plus grande part était octroyée par les autorités civiles dans l'intérêt des chefs de famille. Ceci dit, le père, ou la mère si elle était seule, ou le tuteur (et eux seuls) pouvaient demander à un juge d'emprisonner pour un temps un enfant qui leur créait des "sujets de mécontentement". Mais cet internement n'était renouvelable qu'une seule fois pour un jeune de moins de 16 ans, et un jeune récalcitrant ne pouvait être interné plus d'une année entre 16 et 21 ans pour ce seul motif. D'autre part les parents devaient d'abord obtenir l'accord des \emph{tribunaux de la famille}, qui délibéraient \emph{sous l'autorité d'un juge professionnel}, même si leurs membres étaient recrutés au sein de la famille élargie (et à défaut dans le voisinage immédiat). Ces tribunaux étaient par ailleurs chargés de rétablir la concorde dans les foyers en conflit. 


\section{Privatisation des vœux perpétuels et ouverture du droit au divorce}

 La Constitution de 1791 refusait de reconnaître une valeur juridique aux vœux prononcés par les religieux et fermait tous les couvents. Dans la même logique, les révolutionnaires refusaient de reconnaître tout aspect religieux au mariage et d'y voir autre chose qu'un contrat civil, révocable comme tout autre contrat. En conséquence, la loi du 20 septembre 1792 supprimait la \emph{séparation de corps}, qui sentait trop le catholicisme ...

 ... tandis qu'elle autorisait le divorce par \emph{consentement mutuel} et le divorce \emph{sur demande d'un seul époux}, demande qu'elle autorisait de manière très large et d'abord pour \emph{convenance personnelle}. 

 Le divorce est à ce moment-là devenu aussi facile et plus rapide qu'aujourd'hui (2018). Jusqu'à l'an VII on observe \emph{en ville} un divorce pour 5 mariages ; ensuite un divorce pour 3 mariages. L'inflation du nombre des divorces, non anticipée par la plupart de ceux qui les avaient facilités, a choqué bien des sensibilités. Si les citadins ont recouru très largement du nouveau droit, les habitants des campagnes ne l'ont guère utilisé, d'où l'on conclura sans risque de se tromper qu'on ne divorce pas d'une terre obtenue par mariage aussi facilement que du conjoint qui l'a procurée. C'est une constante de l'histoire : comme le dit le Talmud : \emph{" malheur à celui qui est mal marié et ne peut rembourser la dot de son épouse "}. Ceci dit les campagnes n'avaient pas été gagnées au même degré que les villes par les critiques des philosophes contre l'indissolubilité du mariage, que ce soit pour des raisons religieuses ou parce que celle-ci allait en réalité dans le sens de l'intérêt des familles. 

 


\section{Autonomisation des enfants majeurs}

 La Révolution affirmait l'égalité entre les héritiers et elle la défendait contre tout droit d'aînesse. Pour ce motif et pour empêcher les parents d'exercer une pression indirecte sur les actes de leurs enfants majeurs, la liberté des testateurs était très limitée.
 
 En 1792 l'âge de la majorité a été abaissé de 25 à 21 ans, et surtout les enfants majeurs ont été totalement déliés de la puissance paternelle. Leur capacité juridique a été reconnue comme pleine et entière, qu'il s'agisse d'aliéner leurs biens ou de s'engager dans n'importe quel contrat. Ils pouvaient notamment se marier ou divorcer librement, si nécessaire en passant outre à l'opposition de leurs parents, sans risquer d'être déshérités pour autant. 

 


\section{Nul ne peut être parent contre son gré}

 Les révolutionnaires assimilaient les enfants illégitimes aux enfants légitimes, qu'ils soient adultérins, incestueux ou nés hors mariage de personnes libres de tout engagement ou empêchement, \emph{à la condition expresse qu'ils aient été reconnus par au moins l'un de leurs deux géniteurs}. 
 Mais ils affirmaient aussi que \emph{nul ne peut être parent contre son gré}. Nul, ni femme ni homme, ne devait être contraint à reconnaître un enfant pour sien. L'enfant ne devait être reconnu que volontairement et librement. En cas de naissance hors mariage, une décision libre de chacun des géniteurs était nécessaire pour qu'il devienne parent. La seule exception était le viol avec enlèvement, auquel cas le coupable perdait son droit de ne pas reconnaître l'enfant et de ne pas assumer de responsabilité financière vis à vis de lui. 
De là découlait que dans le temps même où les révolutionnaires accordaient aux enfants illégitimes le droit à entrer dans la famille du parent qui les reconnaissait, et d'hériter de lui à égalité avec un enfant légitime, ils écartaient toute possibilité de \emph{recherche de paternité naturelle}, même pour l'allocation de simples \emph{aliments}. 
 
 On peut s'étonner de cette rigueur à l'encontre des enfants nés hors mariage, comparée à la propension des tribunaux d'ancien régime à traiter favorablement toutes les accusations des mères célibataires portées contre leurs amants réels ou supposés : ils ne reconnaissaient pas la liberté de \emph{ne pas reconnaître} l'enfant dont on ne veut pas. Si les tribunaux "d'avant" écoutaient d'une oreille complaisante les accusations des mères célibataires, et si celles-ci avaient objectivement intérêt à accuser des hommes riches, ceux-ci s'en tiraient ordinairement sans trop de dommages. En effet ils ne risquaient pas de voir entrer les enfants concernés dans leur famille, ni de devoir les compter au nombre de leurs héritiers. Même s'ils l'avaient voulu les lois de l'ancien régime le leur interdisaient. Jusqu'à la Révolution une recherche en paternité se soldait dans le pire des cas, que l’homme condamné soit vraiment le géniteur de l’enfant concerné ou qu'il ait échoué à prouver le contraire, par l’obligation de verser des frais de \emph{gésine} puis \emph{d'aliments} proportionnés au statut social de la mère, jusqu'à ce que l'enfant puisse gagner son pain, à douze ans au plus tard. Par contre à partir du moment où les lois nouvelles ne faisaient plus de différence entre les enfants naturels et les enfants légitimes une recherche en paternité naturelle entraînait de tout autres conséquences sur les familles. 

 Dans le même esprit, l'adultère féminin ne posait pas problème aux révolutionnaires tant que la légitimité de l'enfant conçu n'était pas dénoncée par le mari, suivant le vieux principe du droit romain qui voulait que l'époux de la mère était le père de tous les enfants de celle-ci nés pendant leur union. Le mari d'une femme adultère avait le pouvoir de dénoncer sa paternité et il était le seul dans ce cas : ni l'épouse, ni son amant ne pouvaient le faire, même s'ils le voulaient. 

 Dans l'esprit des hommes de la Révolution, la contrepartie du droit de ne pas être parent si et de refuser de reconnaître un enfant né de ses œuvres, était l'ouverture aux enfants non reconnus d'un large droit à être adoptés dès leur plus jeune âge. En donnant aux personnes sans enfants le droit de se faire ainsi des successeurs et des héritiers ils espéraient résoudre le problème posé par le grand nombre d'enfants abandonnés de cette période. 

 


\section{Démembrement de l'Hôpital Général}

 Le 19 avril 1801 (an IX), le \emph{Conseil général des hospices} réorganisait administrativement les établissements hospitaliers. Les différentes fonctions assurées indistinctement jusque là étaient administrativement démembrées et réparties entre des institutions indépendantes et spécialisées%
% [4]
\footnote{En fait toutes les populations contrôlées par l'ancien Hôpital Général (hôpitaux, hospices, prisons, nourrices et même services d'assistance au domicile) et toutes les institutions nées de son éclatement, sont restées jusqu'au début du \siecle{20} sous la tutelle du Ministère de l'Intérieur, chargé par ailleurs de la police et des cultes. Les personnes incarcérées, quel que soit leur âge, dépendront du ministère de l'intérieur jusqu'en 1911, date où elles passeront sous l'autorité du ministère de la Justice.}%
, dont les ressources allaient être de plus en plus exclusivement assurées par l'impôt. Le classement des établissements établi en 1801 est à l'origine de celui qui a cours aujourd'hui, même si à l'intérieur de chacune de ses catégories de profondes évolutions ont depuis lors transformé le traitement des problèmes des personnes prises en charge :
%\begin{enumerate}[leftmargin=*,itemsep=0pt]
\begin{itemize}
%1) 
\item pour les prévenus, pour les hommes condamnés à de courtes peines, et pour toutes les femmes condamnées : les prisons ; 
\item pour les hommes condamnés à de longues peines, les bagnes ;
\item pour les malades mentaux (c'est-à-dire ceux désignés comme tels par leurs familles ou les autorités civiles, après confirmation du diagnostic par les médecins aliénistes) : les hôpitaux psychiatriques (dont l'architecture, l'organisation interne et le personnel présentaient une grande proximité avec ceux des prisons) ;
\item pour les malades pauvres : les hôpitaux\footnote{Le même jour, le 16 avril 1801, le Conseil général des hospices supprimait les lits de plus d'une personne. C'était la survivance d'un archaïsme que Tenon tenait dès 1788 pour une aberration nuisible à la cure des malades. Mais à Paris cette situation perdurait partout, et surtout dans les sections des indigents des hôpitaux généraux.}, réservés aux malades pauvres ou sans famille, aux personnes en voyage loin de chez elles, et aux militaires éloignés de toute infirmerie de garnison. Les riches préféraient se faire soigner à domicile ou dans des "cliniques" privées payantes, comme toujours ;
\item pour les personnes incapables de gagner leur vie : vieillards sans ressources, infirmes et enfants non abandonnés dont on connaît les parents (qu'ils soient ou non vivants) : les hospices ;
\item pour tous les nourrissons, d'une part, et pour les enfants abandonnés (pupilles) jusqu'à leur majorité d'autre part : placements en nourrice sous l'autorité des hospices mentionnés ci-dessus ;
\item les mineurs vagabonds ou délinquants étaient emprisonnés en tant que délinquants comme les adultes, et avec les adultes. 
 \end{itemize}

 




% Le 18.03.2015 :
% Moyen Âge
% Antiquité
% droit-Droit
% Le 24.02.2015 :
% ~etc.
% Moyen-Âge
%~\%



\chapter{La famille du Code Napoléon}

\section{Suppression du divorce}
 En 1802 Napoléon signait avec le Pape un Concordat qui reconnaissait la religion catholique comme la \emph{religion de la majorité des Français}. À ce titre, il reconnaissait à cette religion une vocation à être l'une des sources du droit et à l'État le droit de nommer les évêques. Il prenait acte de l'expropriation par les révolutionnaires des propriétés de l'Église. En contrepartie l'État s'engageait à salarier et loger les ministres du culte (avant la révolution c'était l'une des revendications du bas clergé). Au nom de l'égalité, l'État reconnaissait aussi les églises protestantes et le judaïsme, et salariait également les pasteurs protestants et les rabbins.
 
 Napoléon a fait réorganiser le droit civil par les professeurs de Droit les plus réputés en faisant une synthèse de la législation révolutionnaire et du droit coutumier de l'ancien régime.  Le \emph{Code Civil} (ou \emph{Code Napoléon}) paraît en 1804. Que ce soit en souci de conformité avec le droit Canon, dans le cadre du Concordat, ou en réaction aux innovations révolutionnaires, qui n'avaient eu guère de succès en dehors de la population des villes, très minoritaire, il restaure presque intégralement la \emph{famille constantinienne}, fondée sur l'union monogame et (quasi) indissoluble d'un homme et d'une femme, et qui exclut de l'héritage tous les enfants adultérins. Il la défend contre tous les courants centrifuges qui pourraient menacer son unité et donc la fragiliser. 
 
 La société, ou du moins une grande partie de celle-ci, se sent attaquée lorsqu'un mariage est menacé, comme du temps d'Auguste, et elle réprouve le divorce, même quand elle le permet. En 1804 le Code Civil supprime le divorce pour \emph{incompatibilité d'humeur} et pose tant de conditions au divorce par \emph{consentement mutuel} qu'il devient très rare, environ cinquante par an, alors qu'en l'an~VII de la Révolution le nombre des divorces dans les villes était le tiers de celui des mariages. Il comprend le divorce comme la sanction d'une faute : adultère du partenaire, condamnation à une peine infamante, excès, sévices ou injures graves... Il restaure la \emph{séparation de corps}, qui interdit le remariage. D'autre part les époux divorcés n'ont plus le droit de se remarier l'un avec l'autre (comme dans le droit juif). Enfin l'époux condamné pour adultère se voit interdire à vie d'épouser son ou sa complice. On reconnaît là une règle de droit instituée par Constantin et ses successeurs immédiats, et jamais abrogée ensuite jusqu'à la Révolution. 



 La Restauration poursuit dans le même sens et supprime le droit au remariage après divorce dès 1816
\footnote{Sous l'ancien régime chacun était soumis dans le domaine familial au droit de sa propre religion, ce qui permettait aux protestants français (à partir du moment où leur religion était tolérée), aux protestants étrangers résidents permanents (en tout temps), et aux juifs, de rompre une union selon leurs propres règles et d'en contracter légalement une nouvelle : les "sans-religion" n'avaient pas de place dans ce modèle. En raison du principe de l'universalité de la loi institué par la Révolution, et donc de l'impossibilité de reconnaître des droits particuliers à certains citoyens, le divorce a été interdit à tous les français par le Code Napoléon quelle que soit leur religion ou leur absence de religion.}. Aux époux mal mariés il ne restait plus que la séparation, comme avant la Révolution. Pour l'obtenir, le demandeur\footnote{qui était le plus souvent une demanderesse : ce n'est pas d'aujourd'hui que les femmes demandent le divorce plus souvent que les hommes.} devait invoquer la faute de son conjoint. L'accord des deux partenaires ne suffisait pas. Les femmes accusaient ordinairement leurs maris de les maltraiter, physiquement ou moralement. Les hommes invoquaient le plus souvent l'adultère de leur épouse. Aux yeux de la loi les infidélités masculines n'étaient des injures graves que s'ils introduisaient leurs maîtresses sous le toit conjugal. 

 Selon une tradition française ancienne, la garde des enfants était ordinairement remise quel que soit leur âge à celui des parents qui était jugé non coupable : pour ce motif elle était le plus souvent confiée aux mères (en Angleterre au contraire les enfants ont été assez systématiquement remis à leur père jusqu'au milieu du \siecle{19}, comme sous l'empire romain). Dans tous les cas de figure, c'est le père qui devait subvenir aux besoins des enfants. Comme toujours depuis l'Antiquité la condamnation d'un conjoint à une peine infamante permettait au conjoint innocent d'obtenir la séparation et la garde des enfants.

\section{Restauration de l'autorité des pères}

 Les familles du Code Napoléon étaient presque aussi patriarcales que celles de l'ancien régime. Certes les points de friction les plus irritants de l'Ancien Régime avaient disparu : les jeunes gens étaient libres de leurs choix professionnels à partir de 21 ans, et il n'était guère possible de les déshériter... 
 
 ...mais beaucoup d'entre eux travaillaient sous l'autorité de leur père dans son entreprise, son atelier, sa boutique ou son exploitation agricole. Ils devaient attendre son décès ou son retrait volontaire, ce qui les maintenait dans sa dépendance jusque dans le choix de leur conjoint, choix d'autant plus contrôlé qu'il restait souvent l'une des clés de leur établissement professionnel. Ils avaient besoin de l'accord de leurs parents pour se marier, quel que soit leur âge, et ne pouvaient passer outre à leur refus qu'à certaines conditions. 

 Dans le principe, les droits parentaux (ce qu'on appelait la \emph{puissance paternelle}) étaient reconnus à chacun des deux parents mais au nom de l'unité du commandement jugée nécessaire à toute institution la cellule familiale était confiée à la direction du mari, et les épouses étaient sous la tutelle de leurs maris. Seuls les hommes participaient à la vie publique et pouvaient exercer le pouvoir politique. Tant qu'ils étaient vivants et non déchus de leurs droits pour condamnation infamante ou pour maltraitance grave de leurs enfants, ou pour démence, ou pour absence, c'étaient eux qui exerçaient la puissance paternelle. Ce n'est qu'en cas d'absence, de séparation à leurs torts, de condamnation à une peine infamante, ou de décès, que les mères pouvaient les remplacer, et encore devaient-elles dans certaines circonstances être assistées dans l'exercice de ce droit par un ou plusieurs membres mâles de la famille de leur époux. 
 
 
 Le Code Civil de 1804 donnait au père le droit de faire appel au juge s'il estimait que son autorité n'était pas respectée par son enfant mineur\footnote{Cf. Pascale \fsc{QUINCY-LEFEBVRE}, « Une autorité sous tutelle. La justice et le droit de correction des pères sous la Troisième République », in \emph{Lien social et politiques, Politiques du père,} RIAC, 37, Printemps 1997, p. 99-109.}%
. Il pouvait faire enfermer un de ses enfants de moins de 16 ans pendant un mois (renouvelable s'il le jugeait nécessaire). Le mineur « de famille » interné pour ce motif était traité comme les délinquants du même âge. S'il avait 16 ans et plus (majorité pénale), ou s'il possédait des biens, ou si son père était remarié, il bénéficiait de plus de garanties : le magistrat pouvait accepter, réduire ou refuser la demande d'incarcération. Mais à partir de seize ans celle-ci pouvait durer six mois renouvelables. Même si la loi mettait des limites au droit de correction, le juge n'avait qu'une assez faible liberté d'appréciation : \emph{il se devait} d'apporter son aide au père qui la sollicitait. En cas de décès du père et si la mère ne s'était pas remariée, c'est elle qui exerçait le droit de correction paternelle \emph{avec l'accord des deux plus proches parents du défunt}. 

 Les lettres de cachet avaient certes disparu, mais la Justice restait \emph{tenue} de fournir son aide aux parents qui la lui demandaient pour contenir et corriger les mineurs dont la conduite préoccupait ces derniers. Durant la majeure partie du \siecle{19} elle l'a fait sans trop se poser de questions. Rapportées au nombre de jeunes français, le nombre des mesures administratives de \emph{correction paternelle} était d'ailleurs limité : quelques milliers par an tout au plus. Et il y avait de grandes disparités dans le nombre des recours au juge suivant les régions et suivant les milieux sociaux. Ils étaient beaucoup plus fréquents dans les familles populaires de Paris que partout ailleurs : plus de la moitié des mesures%
%[4]
\footnote{Cf. Pascale \fsc{QUINCY-LEFEBVRE}, article cité, p. 99.}%
. Ailleurs on se débrouillait autrement avec les jeunes « récalcitrants », fugueurs, « paresseux », « libertins », ou « vicieux » (c'était le langage de l'époque). Il était peut-être plus facile d'élever un adolescent à la campagne ou dans des villes beaucoup plus petites, plus paisibles et moins bouillonnantes de sollicitations que Paris ? Et surtout nulle part ailleurs qu'à Paris n'existait la même tradition de proximité, et même de familiarité, avec la personne du souverain, ce qui facilitait les recours. Quant aux bourgeois, de Paris ou d'ailleurs, ils disposaient toujours de toute une gamme d'internats pour mettre un peu de distance entre eux-mêmes et leurs adolescents trop difficiles à élever, et pour offrir à ceux-ci une rencontre avec des éducateurs professionnels en principe plus sereins et moins impliqués.

 Jusqu'à 1882 l'école n'était pas obligatoire et les enfants des classes populaires qui n'y allaient pas commençaient à travailler très tôt. Comme toujours, s'ils ne les employaient pas eux-mêmes leurs pères les plaçaient chez un patron et touchaient leurs gains jusqu'à leur majorité. Les enfants qui avaient été scolarisés étaient mis au travail dès qu'ils avaient fini leur temps d'école (dix/douze ans). C'est dans ce cadre que doivent être interprétés les reproches formulés par les pères. Les mineurs fugueurs ou vagabonds fuyaient parfois moins l'autorité paternelle que l'atelier, la boutique, l'usine ou la maison bourgeoise où ils (elles) avaient été placés.

 De même qu'en 1801 on avait séparé les aliénés des délinquants, sous la Restauration on a séparé autant que faire se pouvait les mineurs, délinquants et vagabonds, des majeurs, pour éviter qu'ils ne soient maltraités ou « pervertis » par eux. On a donc créé des établissements de correction (ou de redressement) spécialisés dans la prise en charge et la rééducation des délinquants et vagabonds mineurs : prisons spéciales vers 1820 (quartiers spécialisés au sein des prisons, la Petite Roquette,~etc.) puis en 1830 pénitenciers pour mineurs, puis à partir de 1840 les \emph{Colonies agricoles et pénitentiaires} privées. Comme aux siècles précédents, depuis le début du \crmieme{19} les fugueurs, les vagabonds, les prostitués et les mendiants de moins de seize ans (mineurs pénaux) étaient arrêtés par la force publique (du moins s'ils causaient du trouble à l'ordre public). À Paris ils étaient conduits à la Préfecture de police. Ceux qui étaient condamnés allaient en prison. Ceux qui étaient acquittés mais que leurs parents ne réclamaient pas étaient déférés à l'autorité judiciaire. Ils allaient en Colonie Pénitentiaire.

 Les jeunes de la correction paternelle étaient placés à la Petite Roquette pour les garçons, au couvent des dames de Saint-Michel pour les filles. En province ils étaient placés en maison d'arrêt avec les détenus de tous les âges (d'où le moindre recours des parents à cette mesure ?). 

 Face à leurs jeunes « indisciplinés », les familles plus aisées recouraient à des internats scolaires comme aux siècles précédents, sans faire appel à la Justice. Ainsi à partir de 1850 à côté de la Colonie agricole et pénitentiaire de Mettray existait une \emph{Maison Paternelle} réputée, créée par le même fondateur que la Colonie Pénitentiaire, et qui fonctionnait toujours vers 1910, jusqu'à ce que le suicide d'un pensionnaire la fasse fermer. Elle était vouée à la correction des fils des familles suffisamment aisées pour en payer la pension.

\section{Interdiction des recherches en paternité}

 Le Code Napoléon (1804) ramenait les « bâtards » non reconnus par mariage subséquent à leur situation antérieure à la Révolution. Il interdisait la reconnaissance des enfants adultérins et incestueux par leurs géniteurs. Même lorsqu'ils avaient été reconnus par ceux-ci il les excluait de leur succession, donc de leur famille, et ne leur reconnaissait que leur droit traditionnel à des legs « alimentaires ». 

 L'adoption était autorisée par le Code Napoléon, mais il s'agissait uniquement de \emph{l'adoption d'adultes majeurs} par des personnes de 50 ans et plus, et non d'enfants mineurs ni de nouveaux-nés. Contrairement aux vœux des révolutionnaires, il ne s'agissait pas en principe de donner une famille à un enfant sans parents, mais de répondre au besoin d'enfant d'une famille en mal d'héritier. Ces adoptions seront rares durant tout le \siecle{19} et jusqu'en 1923 : 114,4 par an en moyenne de 1840 à 1886 pour toute la France, dont 49,4 enfants naturels, reconnus ou non, et 17,76 neveux, nièces et autres alliés
\footnote{« Statistiques des adoptions au \siecle{19} d'après les comptes généraux de l'administration de la justice civile », tableau cité dans \emph{l'avis présenté au nom de la Commission des affaires sociales sur la proposition de loi, adoptée par l'Assemblée Nationale, relative à l'adoption}, \no~298, session ordinaire du Sénat de 1995-1996, annexe au procès-verbal de la séance du 28 mars 1996.}
. L'objectif de ces adoptions était d'abord de transmettre un patrimoine
\footnote{Plus de la moitié des adoptants étaient des rentiers, terme qui désignait notamment les personnes qui s'étaient retirées des affaires après avoir vendu leur entreprise ou leur commerce et qui vivaient des rentes produites par leur capital : c'étaient en somme des retraités.}
. 

 Le nouveau Code durcissait encore l'interdiction révolutionnaire des recherches en paternité naturelle. Pourtant les recherches en maternité naturelle restaient autorisées. Il n'acceptait les recherches en paternité qu'en cas d'enlèvement, mais il les excluait en cas de viol sans enlèvement,~etc. Comme preuves de la paternité il n'acceptait que les aveux formels écrits par le père, ou bien la cohabitation prolongée du père avec la mère, ou encore la \emph{possession d'état}%
%[7]
\footnote{Situation où le mineur est élevé comme son enfant par le père supposé, même s'il ne l'a pas formellement reconnu.}%
,~etc. Dans ces conditions aucun homme ou presque ne pouvait être contraint contre son gré à reconnaître un enfant naturel, ni condamné à verser une pension alimentaire, même quand tout le monde savait parfaitement à quoi s'en tenir sur ses responsabilités. Il ne restait aux enfants naturels qu'à espérer que leur géniteur veuille bien prendre librement l'initiative de les reconnaître. 

 Cela ne pouvait que pousser les mères célibataires à ne pas garder leur enfant et à l'abandonner anonymement, et cette conséquence était acceptée sans état d'âme. Le placement à la campagne des enfants abandonnés était jugé satisfaisant par tout le monde, et le "bâtard" était un obstacle presque insurmontable à la « rédemption » de sa mère par le mariage, sauf s'il était légitimé par le mariage de celle-ci avec son géniteur (à la rigueur avec un autre homme), ce qui restait la solution préférée. 


\section{le mariage des intérêts}
 
 Depuis les temps les plus reculés le premier objectif des jeunes gens raisonnables n'était pas tant de vivre mieux que leurs parents et de s'enrichir, que de se maintenir au même niveau de fortune qu'eux, et de ne pas tomber dans l'indigence, de ne pas être un \emph{déclassé}. Au \siecle{19} (et probablement en était-il de même auparavant) un homme dépensait plus s'il était célibataire que s'il était marié, sauf à employer une bonne "à tout faire" (souvent nommée "gouvernante"). Il était plus rentable pour un homme employé à plein temps d'entretenir une « ménagère » à domicile que de manger tous les jours au restaurant, de faire blanchir son linge ~etc. En dehors de sa dot, très mince ou inexistante dans les milieux populaires, une épouse pouvait fournir beaucoup de services qu'il était coûteux de se procurer sur le marché. Il était donc avantageux pour les jeunes gens dotés d'un emploi et pour les jeunes filles sans fortune de se mettre en ménage, même sans parler de l'échange de prestations sexuelles ou de désir d'enfant. Mais s'il pouvait être préférable de se marier que de ne pas le faire, il fallait aussi éviter de compromettre, par enthousiasme naïf, par imprudence ou par sottise, les bases économiques d'un futur couple et le statut social des enfants à venir. 
 
 Si un trop grand nombre de ces derniers pouvait dégrader irrémédiablement la situation économique d'une famille, le recul de l'âge au mariage permettait aux filles sans fortune de le limiter tout en se constituant une dot par leur travail. C'est pourquoi leur âge moyen au mariage était bien plus élevé que celui des riches héritières. D'autre part à partir d'un certain âge (peu à peu reculé par la loi) et jusqu'à leur majorité les enfants contribuaient à leur tour aux revenus du ménage, et en l'absence de retraite (sauf pour les fonctionnaires) ils étaient une garantie pour les vieux jours de leurs parents (leurs "bâtons de vieillesse"). 

 Le choix du mariage d'inclination, fondé sur l'amour passion et non sur la raison (c'est-à-dire l'intérêt) était la marque des imprévoyant(e)s. Entre mariage d'inclination et concubinage les liens paraissaient évidents. C'est ainsi que s'unissaient ceux qui ne possédaient que leurs bras, les ouvriers, les manœuvres, les valets, les ouvrières et les servantes, etc. Ceux qui se mettaient en ménage avant d'avoir « assis » leur « situation » se condamnaient à « tirer le diable par la queue ». Selon les moralistes, avec lesquels faisaient chorus tous les parents angoissés, la soumission des jeunes imprévoyants à leurs appétits charnels et à leurs affects leur faisait courir le risque de gâcher leur vie, de connaître la misère et de perdre un jour la main sur leurs propres enfants, ainsi qu'il en avait toujours été depuis le début du monde. Ils risquaient en effet de ne pas pouvoir les élever et de devoir les abandonner aux institutions d'assistance. Ils ne pourraient pas les « établir » en leur donnant un capital matériel, ou en finançant leur apprentissage professionnel auprès d'un maître qualifié, ou en les mettant à l'école, même gratuite, puisqu'ils seraient contraints de les placer chez un maître dès que leur âge le permettrait. En cas de chômage et de disette, ils seraient contraints de les envoyer mendier. Ils ne pourraient pas compter sur ces enfants, condamnés à être pauvres à leur tour, pour soutenir leur propre vieillesse. Ils risquaient de finir leurs jours dans la solitude et la misère, affective et matérielle, des hospices.

 Au contraire les parents prévoyants établissaient leurs enfants dans un mariage profitable grâce à leurs économies, à leurs relations et à des stratégies complexes : échanges simultanés et réciproques d'enfants, de terres, de droits d'exploitation, d'entreprises, de commerces, de gérances, d'offices (ministériels), etc\footnote{... sans compter jusqu'à la Révolution l'entrée en religion de ceux qu'ils ne pouvaient ou ne voulaient pas marier de manière conforme à leur milieu social.}... Voilà pourquoi l'accord des parents était demandé depuis la Renaissance pour tout mariage : selon le Code civil de 1804 l'âge à partir duquel le mariage était autorisé était de 15 ans pour les filles et 18 ans pour les garçons, mais l'âge où l'accord des parents cessait d'être exigé était bien plus tardif : 21 ans pour les filles et 25 ans pour les garçons.
 
 Des stratégies familiales si complexes ne pouvaient pas toujours tenir compte des préférences sexuelles ou amoureuses de chacun, et on n'en faisait pas grief aux parents : les femmes s'en consoleraient avec leurs enfants ou la religion, les hommes avec le travail, l'argent, le pouvoir, les prostituées ou les maîtresses (le recours à celles-là étant toujours préférable, du point de vue des épouses, au choix de celles-ci). Les patrimoines étaient verrouillés contre les effets des infidélités des uns et des autres. Jusqu'au début du XXème siècle une épouse ne pouvait introduire d'enfant adultérin dans sa famille que si son mari le voulait bien, mais en ce cas sa paternité sur cet enfant devenait absolument inattaquable : le géniteur n'avait aucun recours. Quant aux enfants illégitimes du mari, ils ne pouvaient pas être légitimés et menacer l'héritage des enfants de l'épouse. 

 Sauf emploi salarié stable et suffisamment rémunérateur (au service de l'état si possible) la pérennité des couples raisonnables était favorisée par la synergie des ressources que leurs familles respectives avaient sagement et laborieusement conjointes (même les militaires de carrière étaient invités à épouser des filles bien dotées). Leurs parents étaient les premiers à tenir fermement à ce qu'ils, et elles plus encore, ne mettent pas ces arrangements en danger par des comportements imprudents ou des passions irréfléchies, d'où leur accord profond sur ce point avec les autorités morales et religieuses de l'époque. Même en cas de dissensions l'intérêt matériel des époux était le plus souvent de rester unis, quitte à accepter des renoncements ou des compromis sur les vrais désirs de chacun, et à cultiver comme un des fondements du vivre-ensemble une dose convenable d'hypocrisie : d'ailleurs, depuis l'antiquité païenne, il était inconvenant d'afficher publiquement une affection trop vive entre les conjoints. 

 Certes, l'impossibilité de placer les préférences individuelles avant tout autre critère pouvait faire souffrir, et l'amour passion comme la liberté de choix du conjoint faisaient rêver. Les œuvres littéraires du passé reflètent la prégnance de ces représentations. Ainsi, pour ne prendre qu'un seul exemple, la plupart des intrigues de Molière reposent sur le refus d'un mariage arrangé. Les romans de Jane Austen sont des archétypes parmi les milliers d'autres fondés sur les "problèmes de coeur" de jeunes gens et de jeunes filles, apparemment libres de leurs choix et en réalité exctrêmement contraints. Les contraintes économiques étaient indépassables, en dépit des souffrances et des renoncements qu'elles entraînaient. Cela n'empêchait pas la société de continuer siècle après siècle à fonctionner sur le même mode. 
 


\chapter{La police des familles au \siecle{19}}


\section{Les enfants trouvés et abandonnés}

 Le Décret Impérial du 19 janvier 1811 concernant les enfants trouvés ou abandonnés et les orphelins pauvres a créé le Service des enfants trouvés et abandonnés, qui sera nommé Service des enfants assistés à partir de 1866. Il ordonnait qu'il y ait \emph{dans chaque arrondissement un hospice où les enfants trouvés pourront être reçus}. Ont donc été désignés des hospices dépositaires où les nouveaux-nés étaient abandonnés et des dépôts départementaux accueillant au sein de ces mêmes hospices les enfants plus grands. 

 L'abandon anonyme, par l'intermédiaire d'un tour, restait la norme, d'autant plus que \emph{nul ne peut être parent contre son gré} et que les recherches en paternité étaient désormais totalement interdites. L'administration ne faisait pas de distinction entre enfants trouvés (ceux dont on ne connaît pas les parents) et enfants abandonnés (ceux dont on les connaît), ni en fonction de leur légitimité réelle ou supposée.

 Le service prenait en charge les enfants sans famille (trouvés ou abandonnés) et les orphelins pauvres. Il assumait également les enfants dont les deux parents étaient prévenus ou condamnés à une peine de prison. Lorsque ceux-ci étaient incarcérés pour plus d'un an leurs enfants étaient définitivement classés parmi les enfants abandonnés, et traités comme tels%
% [1]
\footnote{Leurs parents ne pouvaient plus les reprendre à leur sortie de prison. Ils étaient considérés comme « infâmes » (la prison est une peine infamante), comme de mauvais exemple, et à ce titre incapables de prendre en main l'éducation d'enfants. On a vu que c'est un accessoire (!) de peine hérité de l'antiquité via l'ancien régime : à Rome le condamné à une peine de travail forcé telle que celle des mines devenait en effet \emph{ipso facto} un esclave \emph{(esclave de la peine)} et il perdait de ce fait tous ses droits civiques et parentaux. Il cessait d'être le père légal de ses enfants et le mari de son épouse. Au \siecle{19} si une personne était condamnée à une peine infamante, le divorce était de droit pour son conjoint innocent : il n'avait pas besoin d'autre motif.}% 
. Il ne s'occupait des enfants âgés de plus de douze ans, abandonnés, orphelins, enfants d'indigents ou de vagabonds, que s'il les avait déjà pris en charge avant leurs douze ans. Si un mineur pauvre de plus de douze ans était incapable de travailler, quel qu'en soit le motif, il pouvait être admis à l'hospice mais en ce cas c'était à cause de son incapacité à travailler, et de son manque de ressources, et non parce qu'il était mineur. Le décret de 1811 prévoyait que les enfants incapables d'être placés chez un maître, parce que malades chroniques, estropiés ou infirmes, malades mentaux, ou retardés intellectuels, etc. resteraient à l'hospice [...] \emph{où on les fera travailler autant que faire se pourra}.

 Le Service confiait les pupilles délinquants à la Justice, exerçant ainsi son droit de correction paternelle comme n'importe \emph{bon père de famille} d'alors était tenu de le faire. Le jeune ainsi remis à l'administration compétente allait en prison (en prison pour mineurs s'il en existait une dans le département) ou en \emph{maison de redressement}, ou en \emph{colonie agricole}, etc. 

 Ses règles de fonctionnement reprenaient, pour l'essentiel, le \emph{règlement concernant les enfans-trouvés} promulgué en 1761, un demi-siècle plus tôt, par le \emph{bureau de l'hôpital} de Paris. Alors que le \emph{Code Napoléon} (1804) interdisait les adoptions de mineurs, l'objectif suivant a été inscrit dans tous les règlements du service dès l'an XIII (1805) : \emph{créer une famille nouvelle à la place de celle qui l'abandonne}%
% [2]
\footnote{In l'A.P. en 1900, p. 349.}% 
. Pour les administrateurs et autres responsables du service, un des objectifs du placement était la greffe du pupille dans un nouveau milieu : \emph{on pourrait citer de nombreux exemples d'enfants déjà grands qui, réclamés par leurs parents, refusent absolument de se séparer de leur famille d'adoption. Il n'est pas rare que des nourriciers dotent un enfant assisté, lui réservent une part ou la totalité de leur héritage. Ainsi, l'enfant abandonné, favorisé au point de vue des soins et des précautions matériels, trouve ordinairement chez ses nourriciers une affection, un attachement qui lui rendent véritablement une famille.}%
\footnote{Idem, p. 340.} 

 Il est troublant de voir qu'au \siecle{19} et une partie de \crmieme{20}, les mots \emph{adoption, père, mère} et \emph{parents} sont constamment employés par les professionnels des placements pour parler des nourriciers, là où ces mots ne sont évidemment pas appropriés. Cela a pourtant été une pratique courante et presque une règle pendant au moins un siècle et demi%
% [4]
\footnote{Les textes écrits par \fsc{Soulé} et \fsc{Noël} entre 1955 et 1965 montrent que cet usage extensif du vocabulaire de la parenté a persisté sans aucun changement jusqu'à eux, jusqu'à ce que depuis une génération se généralise l'idée que tout placement est fait pour préparer le retour vers les parents et que le non-retour est un échec.}% 
. Il est vrai que la plupart des parents des enfants placés étaient fermement tenus à l'écart. Cela contribuait au sentiment des pupilles de constituer une famille avec leurs nourriciers, sentiment d'autant plus facile à verbaliser que la loi interdisait leur adoption. Ils étaient placés dans un entre-deux insatisfaisant, mais qui dans une certaine mesure pouvait être protecteur. Pour les nourriciers cet entre-deux était pratique et confortable. Ils n'avaient à se confronter ni à une adoption en bonne et due forme, ni à des parents réels et vivants. 

 C'est en raison de cet état d'esprit que les enfants qui avaient une famille connue, même constituée de parents décédés (orphelins pauvres), n'étaient pas placés en nourrice au-delà de leur petite enfance. Ils étaient voués à vivre en collectivité dans les hospices dépositaires jusqu'à leur placement professionnel. Au fil du siècle de nombreuses voix, laïques comme religieuses, se sont élevées pour réclamer qu'ils soient distingués des enfants trouvés et abandonnés, et pour qu'il leur soit fourni des conditions d'éducation plus soignées. Dès la première moitié du siècle ces protestations vont se traduire par la création de nombreux \emph{orphelinats} privés qui fonctionneront jusqu'au \siecle{20} à la manière des internats scolaires. 

 Durant tout le \siecle{19} les placements en nourrice ont fonctionné à la satisfaction générale. Les enfants placés à la campagne grandissaient et s'adaptaient \emph{comme on l'attendait d'eux} aussi bien sur le plan professionnel que sur le plan social ou scolaire%
% [7]
\footnote{In l'A.P. en 1900, p. 346. Selon DUPOUX, p. 193, en 1898 68,5~\% des enfants de l'Assistance Publique présentés au certificat d'études l'avaient obtenu, à une époque où plus de la moitié d'une classe d'âge sortait du primaire sans ce diplôme, et où il était donc probablement aussi discriminant que le baccalauréat d'aujourd'hui : aujourd'hui 60~\% des jeunes obtiennent un baccalauréat... mais seulement quelques pour cent de ceux que place l'ASE. Pourtant la moitié de ces derniers obtient aux épreuves psychométriques standardisées des résultats qui les situent dans la zone normale et parfois au-dessus.}% 
. Beaucoup s'intégraient définitivement dans les milieux où l'administration les avait transplantés. L'assistance aux enfants de cette période donnait le spectacle d'une espèce d'équilibre : chacun croyait qu'il savait ce qu'il faisait. Les enfants n'étaient là que parce que leurs parents les avaient confiés à l'institution ou parce qu'ils avaient été dans les formes légales qualifiés d'incompétents ou de dangereux. L'institution apportait un secours indispensable à des enfants qui sans cela seraient ou morts ou dans une grande misère. Elle était sûre d'elle et avait une excellente image dans le public, d'autant plus que le taux de survie des enfants abandonnés a augmenté au fil du siècle de manière extrêmement spectaculaire.

 Pour les mères qui abandonnaient leur enfant la question de son adoption par d'autres ne se posait pas. Leur acte était donc loin d'avoir le sens de \emph{consentement à l'adoption} qui lui serait donné aujourd'hui. Même au moment où elles mettaient leur enfant au tour beaucoup de mères ne croyaient pas que la séparation était définitive, et glissaient par exemple des signes de reconnaissance dans ses langes%
% [8]
\footnote{\fsc{DUPOUX}, idem, p. 200. Le procès-verbal d'abandon rédigé lorsqu'il était découvert décrivait l'enfant et toute sa vêture, en notant soigneusement tous les signes distinctifs.}% 
. Même si le lieu du placement leur était caché, même si les contacts directs et les correspondances ont été interdits jusqu'à la fin du \siecle{19}, au fil des années de plus en plus de parents ont cherché à reprendre l'enfant qu'ils avaient abandonné, du moins tant qu'il avait moins de quatre ou cinq ans. Ceci étant dit il ne faut pas oublier que cette démarche demeurait très minoritaire.


\section{La prévention des abandons}

 Le nombre des abandons a cru rapidement au début du \siecle{19}, \nombre{55700} enfants trouvés en 1810, \nombre{164000} en 1833. À l'époque cette augmentation a été imputée à l'anonymat de l'abandon, permis par les tours, plutôt qu'à l'accroissement du nombre des femmes réduites à survivre misérablement dans les conditions du travail salarié d'alors (elles étaient payées \emph{beaucoup} moins que les hommes pour le même travail), ou à l'impossibilité où elles se trouvaient de recourir aux recherches en paternité pour obtenir l'aide des géniteurs de leurs enfants.

 Face à l'augmentation du nombre des abandons, deux réactions ont été opposées : la diminution puis la fermeture des tours, d'une part, et d'autre part l'aide aux mères. Certains départements ont créé des \emph{Secours préventifs contre l'abandon}, organisés sur le modèle de ce qui s'était fait dans quelques villes avant la Révolution, à l'intention des mères seules et sans ressources, célibataires pour la plupart (« filles mères », comme les nommait alors l'administration). Mis en place en 1837 à Paris, ils ont été généralisés à tous les départements à partir de 1850 (arrêté du 23 décembre). Un certain nombre des jeunes enfants ainsi \emph{secourus}, une minorité, étaient placés chez une nourrice choisie par leur mère elle-même, conformément aux pratiques des populations citadines de l'époque. Le but était en ce dernier cas que ces mères puissent exercer une activité professionnelle sans abandonner leur enfant pour autant.

 Les derniers tours encore en fonction ont été fermés en 1861. Désormais les abandons devaient se faire \emph{à bureau ouvert}. L'anonymat de l'abandon restait possible si la personne qui déposait l'enfant refusait de donner l'état-civil de celui-ci et éventuellement le sien, mais dans le cas contraire on notait l'identité des parents, le plus souvent celle de la mère seule. Cette identité est restée secrète, même pour l'enfant devenu adulte, jusqu'aux années 1980. L'abandon à bureau ouvert n'a donc guère changé la situation des pupilles (ce n'était pas son objectif) par contre il a eu un effet visible sur le nombre des abandons, qui a rapidement et fortement décru%
% [9]
\footnote{Les avortements (clandestins) ont-ils augmenté dans les mêmes proportions ? Selon Stanislas \fsc{DU MORIEZ} (\emph{L'avortement}, 1912) et Edmond \fsc{PIERSON} (\emph{La dépopulation de la France}, 1913) de 1826 à 1880 les tribunaux français ont traité \nombre{9300} affaires d'avortements, dont \nombre{1020} ont donné lieu à des sanctions ; de 1881 à 1909 ils ont traité \nombre{14731} affaires d'avortement, dont \nombre{715} ont donné lieu à des peines diverses. Le faible nombre d'affaires par rapport à ce qu'on suppose être le nombre des avortements, et surtout la faiblesse du pourcentage des condamnations effectives (12 et 5~\%) est à noter : les journaux de la Belle Époque sont remplis de petites annonces de sages-femmes proposant leurs services de manière à peine voilée pour supprimer les grossesses indésirables : en dépit de la stigmatisation morale qui frappait les avorteurs et avorteuses, il existait en fait une relative tolérance qui disparaîtra après la première guerre mondiale. Quant au nombre effectif d'avortements provoqués, il était à cette époque estimé par les auteurs ci-dessus au minimum à \nombre{200000} par an, et au maximum à \nombre{1000000} (un million) et plus. Autrement dit, ils n'en savaient presque rien, ce qui n'est pas étonnant pour une pratique clandestine et qui demande peu de moyens techniques.}% 


\section{Des clivages idéologiques durables autour des familles}

 Les secours préventifs contre l'abandon ont provoqué de très virulents débats parlementaires. Ceux qui les critiquaient pensaient que rien ne vaut un couple conjugal légitime (rural si possible). Ils pensaient que l'aide destinée à l'enfant était le plus souvent détournée de son objet pour \emph{ alimenter la débauche} des mères et de leurs amants, qu'elle n'assurait pas l'avenir d'enfants \emph{sans pères} et donc \emph{sans repères}, et qu'au contraire elle entretenait \emph{une masse d'enfants vagabonds indisciplinés, qui encombrent les cités, constituent un péril social, et dont il faut à grands frais punir les méfaits ou réprimer l'audace toujours croissante}%
% [10]
\footnote{de~\fsc{GERANDO}, cité par \fsc{BIANCO} et \fsc{LAMY}, 1980.}% 
. Avant d'encourager les \emph{filles mères} à garder leurs enfants illégitimes il fallait donc moraliser leur vie et leur donner un « tuteur » en les mariant à un homme travailleur, sobre et économe. Il convenait de faire passer les couples de concubins devant monsieur le maire, et si possible devant monsieur le curé. Lorsque cela n'était pas possible il était préférable pour l'ordre public et pour l'État de confier les bébés sans père aux placements nourriciers ruraux, qui aux yeux des participants de cette sensibilité étaient parfaitement au point : [...] \emph{le service des enfants assistés fournit au contraire \emph{[...]} une race honnête, vigoureuse, fixée à la campagne, fournissant un contingent peu élevé de criminalité}%
%[11]
\footnote{Idem.}%
.

 Ceux qui défendaient les secours préventifs contre l'abandon étaient plus sensibles à la détresse des mères et aux risques pour l'enfant qu'entraîne la coupure d'avec sa famille, même réduite à une seule personne. Ils ne pensaient pas que l'absence d'un époux rendait les mères incompétentes. Ils ne pensaient pas que le soutien d'un père soit irremplaçable. Ils ne pensaient pas qu'un enfant sans père était condamné à l'inadaptation et à la délinquance. Ils croyaient que la société pouvait fournir une aide suffisante aux mères et aux enfants pour que cela n'arrive pas. Ce n'était pas un débat nouveau, puisqu'on l'observait dès le \siecle{18}. Ceux qui n'attachent pas une importance déterminante au mariage et à la naissance légitime et qui regardent la paternité avec une certaine distance ont tendance à sympathiser avec les idées d'égalité et d'autonomie individuelle promues par les Lumières et la Révolution. Même sans aller jusqu'à désigner l'État comme la seule instance qui ait une autorité légitime sur les enfants, ils sont plus ouverts que les autres à l'idée qu'il puisse légitimement exercer un contrôle sur tous les parents. Ceux qui au contraire tiennent pour essentiel que l'enfant grandisse dans une famille fortement structurée autour d'un couple mixte, avec des rôles différenciés (père, mère, enfants), ont plus tendance à n'être que peu ou pas du tout séduits par les discours révolutionnaires et considèrent plus facilement comme abusif que l'État cherche à s'immiscer dans la relation entre les parents et les enfants. Ils sont aussi plus enclins que les autres à supporter que la loi fasse des différences entre les enfants légitimes et les autres au nom de la défense de l'institution familiale. Il y a là une ligne de partage que l'on retrouve aujourd'hui encore.
 



% Le 18 mars 2015 :
% Antiquité
% Moyen Âge
% ~etc.


\chapter{III\ieme{} et IV\ieme{} républiques}


 Depuis la fin du \siecle{19} il s'est produit beaucoup d'évènements qui ont eu un impact décisif, directement ou indirectement, sur les familles et pour commencer voici les décisions essentielles :
\footnote{Sources :
\\Ouvrage collectif de l'Administration générale de l'Assistance Publique, \emph{L'Assistance Publique en 1900}, écrit à l'occasion de l'Exposition Universelle de Paris de 1900, composé et imprimé par les pupilles de la Seine de l'école d'Alembert à Montévrain, Paris, 1900. Consultable au Musée Social, Paris, VIIème.
\\Collectif, sous la direction de Michel \fsc{CHAUVIERE}, Pierre \fsc{LENOËL}, Eric \fsc{PIERRE}, \emph{Protéger l'enfant, Raison juridique et pratiques socio-judiciaires (\crmieme{19} et \siecle{20}{}s)}, Presses Universitaires de Rennes, 1996.
\\Collectif, sous la direction de Jean \fsc{DELUMEAU} et Daniel \fsc{ROCHE}, \emph{Histoire des pères et de la paternité}, Larousse, 1990, édition 2000.
\\Collectif, sous la direction de Jean \fsc{IMBERT}, \emph{Histoire des hôpitaux en France}, Privat, 1982, 559 p.
\\Collectif, sous la direction d'Alain \fsc{BURGUIERE}, Christine \fsc{KLAPISH-ZUBER}, Martine \fsc{SEGALEN}, Françoise \fsc{ZONABEND}, \emph{Histoire de la famille, 3, Le choc des modernités}, Armand Colin Éditeur, Paris, 1986.
\\\fsc{BROUSOLLE} Paul, \emph{Délinquance et déviance, brève histoire de leurs approches psychiatriques}, Privat, Toulouse, 1978.
\\\fsc{CUBERO} José, \emph{Histoire du vagabondage du Moyen Âge à nos jours}, Imago, Paris, 1998.
\\\fsc{DONZELOT} Jacques, \emph{L'invention du social, essai sur le déclin des passions politiques}, Seuil, Paris, 1994.
\\\fsc{DUPOUX} Albert, \emph{Sur les pas de Monsieur Vincent, 300 ans d'histoire parisienne de l'enfance abandonnée}, Édité par la Revue de l'Assistance Publique, Paris, 1958.
\\Patrice \fsc{PINELL}, Markos \fsc{ZAFIROPOULOS}, \emph{Un siècle d'échecs scolaires (1882-1982}), Les éditions ouvrières, Paris, 1983.}% 
 :

\begin{description}
\item[1880] Création d'un enseignement secondaire public pour les filles, calqué sur le modèle de celui des garçons. 

\item[1881] Obligation pour chaque commune de mettre à la disposition de ses administrés une école gratuite et laïque.

%1881 : 
Création du \emph{Service des enfants moralement abandonnés}. Il a pour objet de recevoir les jeunes de 12 à 16 ans sans support familial, pénalement mineurs, non secourus puisque le service ne recevait pas de nouveaux entrants après l'âge de douze ans, et n'ayant que la mendicité et le vol pour subsister : mineurs arrêtés pour {\emph{vagabondage et autres menus délits, et aussi ceux que leurs parents se montraient incapables de diriger}}.

\item[1882] Obligation scolaire pour les garçons et filles de 6 à 12 ans. Avant leurs 12 ans (11 ans s'ils ont le certificat d'études) il est interdit aux pères de placer leurs enfants chez un employeur, ou de les employer eux-mêmes à plein-temps.

\item[1884] Réouverture du droit au divorce (uniquement pour faute, comme en 1804).

\item[1886] Laïcisation du personnel enseignant des écoles publiques. 

\item[1889] La loi du 24 juillet {\emph{sur la protection judiciaire des enfants maltraités et moralement abandonnés}} précise les conditions de la déchéance de la puissance paternelle. Cette déchéance totale est prononcée par un Juge :
\begin{enumerate}[leftmargin=*,itemsep=0pt]
% 1°)
\item facultativement pour inconduite des parents,
% 2°)
\item facultativement en cas de mauvais traitements ou de délaissement de l'enfant,
% 3°)
\item et de plein droit dans le cas de certaines condamnations infamantes (ce qui était le cas depuis l'Antiquité).
\end{enumerate}

% 1889 : 
Cette loi du 24 juillet 1889 %{\emph{sur la protection judiciaire des enfants maltraités et moralement abandonnés}} 
confie à l'administration (c'est-à-dire à l'Assistance Publique) la tutelle des enfants maltraités, victimes de crimes, ou de délits, ou délaissés. Le service les prend en charge même s'ils sont âgés de plus de 12 ans à leur entrée. Ces enfants sont retirés autoritairement à leurs parents et deviennent des pupilles comme les autres. Ils sont traités à l'instar des autres enfants du service. Quel que soit leur âge, autant que faire se peut ils seront placés en nourrice, pour de longues durées, et dans tous les cas ils seront coupés de leurs parents déchus.

\item[1893] Les femmes séparées de corps ont la pleine capacité civile : elles récupèrent les droits qu'elles avaient quand elles étaient célibataires.

%\item[$\!\!\!$] À partir de \textbf{1896} 
\item[1896] {\emph{Les familles indigentes mises devant la nécessité d'abandonner \emph{[sont]} autorisées après enquête à correspondre directement avec enfants et nourriciers}}. D'autre part les {\emph{enfants de parents internés \emph{[sont désormais]} considérés comme n'ayant pas été abandonnés volontairement}}, et les correspondances directes entre parents et enfants sont autorisées.

\item[1897] Les femmes mariées peuvent être témoins dans les actes civils et notariés.

\item[1901] Loi sur les associations à but non lucratif. Leur fondation est libre, basée sur la notion de contrat entre personnes. Elles ne peuvent recevoir ni dons ni legs.

\item[1904] Dénonciation unilatérale du Concordat de 1802.

% 1904 : 
Autorisation donnée aux amants condamnés pour adultère de s'épouser après leur(s) divorce(s) ou le décès du conjoint trompé.

% 1904 : 
 Loi du 12 avril : majorité pénale à 18 ans au lieu de 16, élargissement de l'excuse de minorité, affirmation de la nécessité de faire passer l'éducatif avant le répressif pour les mineurs pénaux.

\item[1907] La loi du 13 juillet permet aux femmes mariées de toucher et de gérer elles-mêmes leur propre salaire, au lieu qu'il soit remis à leur mari comme c'était la règle durant tout le \siecle{19}. 

\item[1912] Autorisation des recherches en paternité naturelle. Les enfants naturels peuvent demander des aliments à chacun de leurs géniteurs : ce texte vise essentiellement les pères, et les mères peuvent agir au nom de leurs enfants. 

% 1912 : 
La loi du 22 juillet créée des tribunaux spéciaux pour enfants et adolescents. Elle pose les premiers jalons de la liberté surveillée

\item[1913] Mesures d'assistance en faveur des femmes en couche nécessiteuses, et des familles nombreuses nécessiteuses.

\item[1917] Une femme peut être nommée tutrice et siéger au conseil de famille

\item[1920] Une femme mariée peut adhérer à un syndicat sans l'autorisation de son mari.

% 1920 : 
Toute forme de propagande anticonceptionnelle ou de publicité pour des instruments de lutte anticonceptionnelle est interdite (préservatifs, pessaires, diaphragmes,~etc. qui restent néanmoins disponibles en pharmacie).

\item[1921] La loi ouvre la possibilité de prononcer une déchéance partielle de l'autorité paternelle. 

\item[1923] L'adoption des enfants abandonnés (sans limite d'âge inférieure) est ouverte aux couples mariés. Nommée légitimation adoptive, elle n'annule pas le passé de l'enfant.

\item[1924] Identité complète des programmes d'études dans le secondaire féminin et masculin.

\item[1925] L'A.P. commence à placer en nourrice les jeunes enfants (âgés de moins de quatre ans, dans un premier temps) placés \emph{en dépôt} par leurs parents et elle les y laisse grandir. 

\item[1932] \emph{Allocations familiales} (pour tous les enfants).

\item[1931] Les femmes peuvent être nommées (élues ?) juges.

\item[1935] Le décret-loi du 30 octobre sur {\emph{la correction paternelle et l'assistance éducative}} institue l'assistance éducative à domicile. 

% 1935 : 
Le \emph{vagabondage} des mineurs cesse d'être un délit, (contrairement à la mendicité et au racolage qui demeurent des délits). 

\item[1938] La femme mariée acquiert certains des droits des femmes célibataires : droit à une carte d'identité, à un passeport, à ouvrir un compte en banque sans l'autorisation de son époux.

\item[1941] Allocation de salaire unique \emph{(parmi les textes promulgués sous l'occupation, ne comptent que ceux qui ont été confirmés à la Libération : plusieurs d'entre ces derniers avaient été préparés bien avant la guerre)}. 

% 1941 : 
Ouverture des hôpitaux à tous, quels que soient leurs revenus : depuis longtemps déjà les hôpitaux recevaient des malades qui payaient leur séjour et les soins qui leur étaient dispensés. Ils payaient eux-mêmes ou c'est un tiers qui le faisait : militaires (dès l'ancien régime), accidentés du travail (1897), assurés sociaux (1928),~etc. En 1900 cela contribuait pour 20~\% aux recettes des hôpitaux. En 1940 40~\% des hospitalisés donnaient lieu à un remboursement. Le 21 décembre 1941 il est décidé d'étendre cette possibilité à tout le monde, sans maintenir d'exclusive. Comme bien des décisions de cette époque, ce n'était que la mise en œuvre de décisions préparées dés 1938, c'est pourquoi cette orientation n'a pas été remise en question à la libération.

\item[1943] La loi du 15 avril 1943 donne un droit aux secours aux enfants \emph{qui ont un père, même quand celui-ci est valide}. Le droit des parents au « dépôt » volontaire de leurs enfants à l'Assistance Publique est élargi.

\item[1944] Octroi du \emph{droit de vote} aux femmes. 

\item[1945] L'ordonnance du 2 février crée le corps des juges pour enfants, pour les jeunes de moins de 18 ans. Elle crée l'éducation surveillée à l'intention des mineurs délinquants.

\item[1946] Création des \emph{allocations prénatales}.

% 1946 : 
La constitution déclare \emph{égaux} les droits des hommes et des femmes.
\end{description}

\section{Séparation de l'église catholique et de l'État}


 La France n'était jusque là jamais sortie du cadre intellectuel et moral du catholicisme dans lequel elle s'était constituée, sauf durant quelques années pendant la Révolution française. Les mouvements de laïcisation de la fin du Moyen Âge et de la Renaissance, comme ceux du \siecle{18}, avaient travaillé sur les limites entre ce qui revenait aux pouvoirs civils et ce qui revenait au personnel ecclésiastique, mais ils n'avaient pas fondamentalement mis en question la place de la religion catholique comme source du Droit. Le Concordat de \hbox{Napoléon} avait remanié cette situation sans la modifier radicalement. Les autres confessions et les « sans religion » ne représentaient en 1804 qu'un très faible pourcentage de la population. Même si le degré d'identification des français à l'Église fluctuait beaucoup suivant les régions et les milieux sociaux, la population française était très majoritairement catholique. A la fin du \siecle{19} les religieux étaient bien plus nombreux qu'à la fin de l'ancien régime : {\emph{\nombre{81000} religieux en 1789, \nombre{13000} en 1808, \nombre{160000} en 1878}%
% [1]
\footnote{Christian \fsc{SORREL}, \emph{La République contre les congrégations – Histoire d'une passion française 1899-1904}, éd. du Cerf 2003, p. 12. Dans \emph{L'ancien régime, institutions et sociétés} (Le livre de poche, 1993, p.68) François \fsc{BLUCHE} donne des chiffres différents, mais du même ordre de grandeur pour les religieux : \emph{le monde ecclésiastique comprenait, à l'extrême fin de l'ancien régime, un peu moins de \nombre{140000} membres. Le clergé régulier (religieux et religieuses, moines et moniales) regroupait quelques \nombre{59000} âmes (dont \nombre{28000} femmes)... Le clergé séculier représentait quelque \nombre{80000} hommes d'Église (\nombre{139} prélats, environ \nombre{10000} chanoines et les \nombre{70000} prêtres assurant le culte des \nombre{40000} paroisses).}}%
}... La Révolution avait supprimé les monastères et les couvents, et confisqué tous leurs biens, et le Concordat n'avait prévu aucun cadre juridique pour les congrégations religieuses. Et pourtant d'innombrables congrégations nouvelles avaient été créées durant tout le siècle, tandis que beaucoup parmi les anciennes s'étaient relevées de leur état de langueur du \siecle{18}. Les « congréganistes » s'investissaient d'abord et avant tout dans l'enseignement, alors en plein essor, notamment dans le primaire, et aussi et comme toujours dans les services hospitaliers, eux aussi en expansion : en 1847 il y avait en France plus de sept mille religieuses hospitalières, à la fin du Second Empire plus de dix mille, en 1905 plus de douze mille.

 C'est justement là que le bât blessait : le programme des républicains qui avaient conquis le pouvoir en 1879 faisait de la solidarité et de l'enseignement des outils essentiels de gouvernement%
% [2]
\footnote{Cf. \emph{L'invention du social, essai sur le déclin des passions politiques}, Jacques \fsc{DONZELOT}, 1994.}% 
, et il n'était pas question pour eux de les laisser aux mains des employés permanents de l'Église. Ils voulaient retirer à celle-ci les points d'appui institutionnels sur lesquels elle avait assis son influence depuis Constantin. À partir du moment où la gauche radicale l'a emporté dans les urnes, l'histoire des familles comme celle du traitement de la pauvreté a changé de direction. La nouvelle majorité s'est donnée pour mission des tâches traditionnellement dévolues à la Providence%
%[3]
\footnote{Le terme « Providence » est l'un des noms de Dieu.}%
. Au lieu de déplorer les malheurs et les injustices de la \emph{vallée de larmes} où vivraient les hommes, tout en comptant sur un \emph{au-delà} paradisiaque ou infernal pour régler à chacun son compte, ses membres ont estimé du devoir de l'État de s'attaquer lui-même aux sources des malheurs individuels, et d'abord aux injustices sociales, sans se reposer sur les initiatives privées, expressions de la Providence, et de procurer aux citoyens sinon le bonheur du moins un droit effectif à une aide efficace, afin de prévenir le malheur quand c'est possible et de soulager les souffrances quand cela ne l'est pas. 
 Ils s'attaquaient résolument et en pleine connaissance de cause à son autorité sur les esprits.

 À côté de mesures de portée limitée ou relativement symbolique%
% [4]
\footnote{Exemples : suppression de l'obligation du repos dominical (rétabli dès 1906 sous la pression des associations ouvrières) ; sécularisation des cimetières ; suppression des prières publiques constitutionnelles et de tous les signes religieux présents dans les lieux publics ; imposition du service militaire aux religieux et aux séminaristes ; exclusion des membres du clergé des commissions d'enseignement des hôpitaux en tant que membres de droit : curés chargés d'une paroisse, évêques,~etc.}% 
, les républicains ont d'emblée exclu les facultés de théologie des universités publiques, et le personnel religieux du corps enseignant universitaire. L'institution de l'obligation scolaire jusqu'à 12 ans était devenue inéluctable à cette époque%
%[5]
\footnote{Selon \fsc{FURET} et \fsc{OZOUF}, dès le milieu du \siecle{19} près des trois quarts des enfants français sont scolarisés. Chaque commune était depuis Guizot astreinte à l'obligation de fournir une école primaire publique à ses habitants, mais pas à en garantir la laïcité, d'ailleurs souvent refusée par la majorité de la population, comme la suite de l'histoire l'a montré. Le nombre d'enfants scolarisés en 1850 dans les écoles primaires (héritières des petites écoles des siècles précédents) représentait 73~\% du nombre des enfants de la tranche d'âge des 6-13 ans. Il en représentait même 105~\% en 1876-1877 : plus de 100~\%, ce qui s'explique par les enfants scolarisés avant 6 ans et après 13 ans (\fsc{FURET} et \fsc{OZOUF}, 1977, p. 173). Par conséquent en 1880 les enfants d'âge scolaire non scolarisés ne représentaient plus qu'une petite minorité. Mais ces chiffres moyens couvraient des disparités extrêmement grandes :
%\begin{itemize}
\begin{enumerate}[label=\alph*.,itemsep=0pt]
%A)
\item entre régions (le nord et l'est étaient très scolarisés depuis des siècles, au contraire du sud et de l'ouest, très peu scolarisés),
% B)
\item entre villes et campagnes,
% C)
\item parmi les régions rurales elles-mêmes, entre celles de civilisation exclusivement orale comme la Bretagne (valorisant la parole « vivante », et se défiant de la parole « morte », c'est-à-dire écrite), le Pays Basque, la Catalogne,~etc. et celles (de langue française) largement pénétrées par l'écrit,
% D)
\item et au moins autant entre classes sociales.
\end{enumerate}
%\end{itemize}

 Rien ne permettait de penser que ces petites minorités réfractaires à l'école d'alors étaient prêtes à rejoindre spontanément et rapidement le mouvement général, ce qui justifiait d'obliger par la loi les parents à scolariser leurs enfants.}% 
, mais il n'en était pas de même de la laïcité de l'enseignement. Celle-ci était évidemment une arme contre l'Église et son influence dans le domaine scolaire. Alors qu'à cette époque les femmes étaient les plus fidèles soutiens de l'Église, les républicains ont créé pour elles un enseignement secondaire public et laïque similaire en (presque) tout point à celui des garçons. Il s'agissait à la fois de lutter contre l'influence des congrégations en leur interdisant tout enseignement avant de les expulser, et contre la vision traditionnelle d'une femme soumise au contrôle masculin pour l'accès au savoir et à la culture. 

 En légalisant le divorce en 1884, les républicains affranchissaient le mariage civil des règles du Droit Canon. En 1904 l'adultère cesse d'être une faute contre la société dans son ensemble et n'est plus qu'une affaire privée : une fois libérés de leurs unions antérieures, les amants adultères ont le droit de s'unir légalement, ce qui leur était interdit à vie depuis l'empereur Justinien --- interdit qui les empêchait de légitimer leurs enfants déjà nés (adultérins) et leurs enfants encore à naître. 

 Le titre III de la loi 1901 sur les associations a refusé aux congrégations religieuses la liberté d'association. Par conséquent à partir de sa promulgation toutes les congrégations existantes ont été dans l'obligation d'obtenir une autorisation législative, ce dont elles s'étaient le plus souvent passées depuis le début du \siecle{19}. Cette autorisation a été refusée à la grande majorité d'entre elles, \emph{ipso facto} dissoutes. Les congrégations enseignantes, dont les effectifs étaient de beaucoup supérieurs à celui des religieuses hospitalières, ont toutes été interdites%
% [6]
\footnote{Sur un nombre de plus de \nombre{1300} congrégations, \nombre{140} congrégations masculines et \nombre{888} congrégations féminines ont été dissoutes. Cela a concerné plus de cent cinquante mille personnes dont 80~\% de femmes...}% 
. Seules ont été épargnées les congrégations hospitalières ... et toutes les congrégations implantées aux colonies. 

 La laïcisation des hôpitaux a commencé dès 1879, mais elle ne pouvait se faire qu'au rythme de la formation du personnel laïc d'encadrement et infirmier, ce qui demandait d'abord de créer les écoles d'infirmières et de surveillantes nécessaires, puisque les noviciats des congrégations hospitalières en avaient jusque là tenu le rôle. On observe quelques créations vers 1880, puis en 1899 est prise la décision de créer une école d'infirmière dans toutes les villes de faculté. Ceci étant dit la laïcisation de chaque institution dépendait d'abord et surtout de la couleur politique du conseil municipal dont elle dépendait.

 En 1900 les religieuses formaient la plus grande part du personnel soignant, par contre en 1975 l'ensemble du personnel soignant (sans compter les autres employés des hôpitaux) se comptait à près de \nombre{300000} personnes. Les religieuses ne représentaient plus à cette date qu'une toute petite minorité vieillissante. C'est qu'il ne s'agissait plus des anciens hôpitaux et hospices voués essentiellement aux indigents. Désormais il s'agissait d'établissements industriels, la réalisation concrète des « machines à guérir » dont les penseurs de la fin du \siecle{18} avaient rêvé. Les clients avaient complètement changé, et l'échelle aussi : en 1975 on trouvait dans les hôpitaux publics plus de \nombre{10000} (dix mille) médecins des hôpitaux à plein temps et \nombre{26900} internes, beaucoup plus qu'il n'y avait de personnels soignants, tous statuts confondus, dans tous les hôpitaux du \siecle{19}%
% [8]
\footnote{Cf. Jean \fsc{IMBERT}, \emph{Histoire des hôpitaux en France}, 1982.}% 
.
\section{Critiques de gauche et anarchistes de la famille}

 

Selon Jacques \fsc{Donzelot} depuis la Belle Époque des « militants », qu'il classe parmi les anarchistes ou à côté d'eux, ont mis {\emph{"...en place les petites machines de guerre contre la famille \emph{[... que sont]} la célébration de l'union libre, \emph{[...]} la distribution des produits anticonceptionnels et \emph{[...]} la propagande pour la grève des ventres}\footnote{Idem p. 163.}.}\footnote{Jacques \fsc{DONZELOT}, \emph{La police des familles}, 1977, 220 pages. Chapitre 5, « La régulation des images », p. 154 à 211.}" 
Parmi eux on trouvait, à côté des militants de base, des médecins comme Adolphe \fsc{Pinard}, des écrivains comme Octave \fsc{Mirbeau}, des hommes politiques de gauche comme Léon \fsc{Blum}, des savants comme Paul \fsc{Langevin}, soucieux {\emph{"...d'incorporer l'hygiène et donc le contrôle des naissances dans le fonctionnement des institutions."}} On trouvait également en première ligne la \emph{Ligue des droits de l'homme} et la \emph{Société de prophylaxie sanitaire et morale}, dirigées toutes deux par le docteur \fsc{Sicard~de Plauzolles}. Ils s'exprimaient dans divers ouvrages tels que \emph{La fonction sexuelle} (1908) du même docteur, ou \emph{Du mariage} de Léon \fsc{BLUM} (1908). 

Leur discours \emph{"...est à peu près celui-ci : puisque la famille est détruite par les nécessités économiques de l'ordre social actuel, il faut que la collectivité remplace le père pour assurer la subsistance de la mère et des enfants. Au père se substituera ainsi la mère comme chef de la famille ; puisqu'elle en est le centre fixe, la matrice et le cœur, elle en sera la tête. Les enfants seront sous sa tutelle, centralisée par l'autorité publique. Tous porteront le nom de leur mère ; ainsi les enfants nés d'une même femme mais de pères différents auront le même nom ; aucune différence n'existera plus entre légitimes et bâtards. L'influence de l'homme sur la femme et sur les enfants sera en rapport avec l'amour et l'estime qu'il inspirera ; il n'aura d'autorité que par sa valeur morale : il n'aura de place au foyer que celle qu'il méritera..\footnote{Idem, p. 164.}.} 

 Pour les plus radicaux de ces théoriciens, tels l'avocat Ernest \fsc{TARBOURIECH} (in \emph{La cité future, essai d'une utopie scientifique}, 1902), socialiste marxiste et collectiviste : \emph{"La puissance paternelle aura disparu... Le père et la mère n'auront sur leur progéniture aucun droit d'aucune sorte, mais seulement des devoirs qui peuvent ainsi se formuler : aider l'état dans la tâche qui lui incombe vis à vis des jeunes générations. L'éducation et l'instruction, affaires d'état, seront réglées souverainement par l'état au mieux. Les médecins représentant la communauté confieront chaque enfant à la personne qui donnera les soins les plus tendres et les plus éclairés. La loi présumera que cette personne est la mère mais cette présomption si naturelle... ne sera pas... de Jure... mais... susceptible de preuve contraire.
 ...l'autorité médico-judiciaire pourra intervenir}..." à tout instant jusqu'à la majorité du mineur (p. 309). 
 
 En résumé  c'est l'État, détenteur des moyens de production et pourvoyant aux besoins de la totalité de la population, active comme inactive, qui gère les effectifs de ses employés. Il dirige donc la reproduction et l'éducation. Il ordonne l'euthanasie des nouveaux-nés jugés par une commission scientifique (le "juge médical") mal conformés, "vicieux", "tarés", ou voués au crime ou à l'impuissance économique \emph{"...pendant cette période où ils ne sont pas encore une personnalité."} (p. 397). Le même "juge médical" accorde ou refuse aux individus le droit à une vie sexuelle. L'état prescrit de déclarer toutes les grossesses et il les surveille. C'est lui qui décide si la génitrice est apte à collaborer avec l'état dans la mission d'élever le futur citoyen. Il peut à tout moment la remplacer au profit d'un éleveur ou d'un éducateur offrant plus de garanties qu'elle (éventuellemet le géniteur de l'enfant). Il s'agit pour \fsc{tarbouriech} d'étendre à toute la société le régime de la tutelle, et à toutes les mères l'attribution des secours éducatifs et du contrôle sanitaire, afin qu'elles soient payées comme nourrices de leurs propres enfants et qu'elles les élèvent non pour elles mais pour l'État et sous son controle. Il étend à tous les enfants le régime des "enfants assistés" (pupilles)  de l'Assistance Publique : \emph{"Bref,} selon \fsc{Donzelot}  \emph{...une gestion médicale de la sexualité libérera la femme et les enfants de la tutelle patriarcale, cassera le jeu familial des alliances et des filiations au profit d'une emprise plus grande de la collectivité sur la reproduction et d'une prééminence de la mère. Soit un féminisme d'état\footnote{Une telle utopie serait-elle vraiment un "féminisme (d'Etat)" ? Il faut n'avoir aucune idée de la dépendance des "nourrices" de l'Assistance face à l'administration pour le croire. La responsabilité de l'éducation reviendrait entièrement à l'Etat, et la puissance enlevée aux pères ne serait pas donnée aux mères, qui ne détiendraient qu'une délégation d'autorité parentale, révocable à tout instant. Face à leurs enfants elles auraient moins de garanties juridiques qu'avec le Code Napoléon, tout patriarcal que soit celui-ci.}."\footnote{Idem, p. 164.}}
 



 De la Belle Epoque à la fin du baby-boom les néo-malthusiens se sont opposés aux « populationnistes ». Ceux-ci se recrutaient dans la bourgeoisie traditionnelle, attachée pour de multiples raisons à la transmission de son patrimoine, mais aussi parmi {"[...] \emph{les ligues de pères de famille, la Ligue des mères de familles nombreuses, l'Association des parents d'élèves des lycées et collèges, l'École des parents, l'Union des assistantes sociales, les organisation scoutes, les ligues d'hygiène morale, d'assainissement des kiosques de journaux, des abords des lycées,~etc}\footnote{Idem, p. 162.}".} Les membres de ces groupes de pression défendaient la répartition traditionnelle des rôles sexués et des pouvoirs au sein de la famille. Ils pensaient en effet que plus la structure familiale était forte, plus elle avait de chances d'être prolifique, et de bien réaliser sa mission éducative. Ils luttaient {\emph{"...contre tout ce qui peut fragiliser la famille : le divorce, les pratiques anticonceptionnelles, l'avortement."}\footnote{Idem.}}




 Bien des mesures décidées à cette époque par la Gauche au pouvoir allaient dans le sens des néo-malthusiens et contre les populationnistes. Le divorce%
% [8] 
\footnote{... pour faute seulement, parce que l'opinion d'alors n'acceptait pas d'autre motif, mais divorce tout de même : d'où jusqu'aux années 1970 tout un folklore de manœuvres vaudevillesque pour fabriquer en commun une « faute » légalisable (lettres d'injures...) même quand les conjoints étaient d'accord sur l'objectif.} 
permettait aux épouses maltraitées, délaissées ou bafouées, de sortir de la prison où le mariage les retenait jusque là. 

 Le premier objectif de l'obligation scolaire était certes de répandre le savoir, la culture commune, et de ne laisser personne à l'écart de cette richesse, mais un effet pleinement assumé, et même désiré, de cette obligation était aussi de contraindre toutes les familles à accepter l'entrée en leur sein de points de vue extérieurs. Elles ne pouvaient plus élever leur enfant à l'écart du monde. 

 La création d'un enseignement secondaire public pour filles calqué sur celui des garçons promouvait l'égalité complète des filles et des garçons, même si en 1880 on en était loin. C'était un choix historique, une rupture dans la répartition sexuée traditionnelle des tâches et compétences. 

 L'obligation scolaire interdisait aux parents de placer leurs enfants chez un employeur avant leurs 12 ans (11 ans s'ils avaient obtenu le certificat d'études) ou de les employer eux-mêmes à plein-temps%
% [9]
\footnote{Après le rapport de \fsc{VILLERMÉ} sur le travail des enfants, la loi du 22 mars 1841 avait fixé pour la première fois une limite d'âge au-dessous de laquelle il était interdit aux employeurs, et donc (indirectement) aux parents, de mettre les enfants au travail. La première borne avait été posée à l'âge de 8 ans. Elle avait été plus ou moins respectée mais ce n'en était pas moins le début d'une lente progression. La loi sur la scolarité obligatoire s'inscrivait comme une nouvelle étape dans cette progression, et l'exploitation du travail de l'enfant par ses parents commençait d'apparaître comme une forme de maltraitance.}% 
. 

 Quant au droit des femmes mariées à gérer leur propre salaire, c'était une part de souveraineté symbolique en moins pour les maris. En fait dans bien des ménages populaires c'étaient les femmes qui tenaient les cordons de la bourse, d'un commun accord entre conjoints (en a-t-il toujours été ainsi ? Les épouses semblent être presque toujours chargées de gérer les réserves, les resserres et les greniers, ce que symbolise le fait qu'on leur confiait les clés).

 À partir de 1912 les enfants sans père reçoivent le droit de demander des aliments à leur géniteur \emph{(recherche en paternité naturelle)}. Une mère célibataire n'est plus sans recours devant celui qui l'a laissée seule avec son enfant, qu'elle représente devant la justice. Cela entraîne pour corollaire qu'une femme mariée n'est plus aussi à l'abri qu'avant des conséquences matérielles et sociales des frasques pré ou extra conjugales de son conjoint. Pour autant un enfant illégitime ne peut toujours pas hériter de son père. 


\section{Mise en question du droit de correction}

 À aucun moment de l'histoire les parents n'ont été autorisés à faire subir à leurs enfants \emph{tout} ce qu'ils pouvaient imaginer. La tolérance à leurs abus de pouvoir a pu varier au fil des siècles, mais ils n'ont jamais eu le droit de les estropier, pas plus qu'ils n'avaient le droit d'estropier les enfants des autres, ni d'en faire leurs partenaires sexuels. Mais la Justice a toujours beaucoup de difficultés à les poursuivre lorsque les traces des sévices ne se voient pas, ou dans les cas de négligence simple, d'abandon moral. En faisant du délaissement et de la maltraitance des délits, la loi {\emph{sur la protection judiciaire des enfants maltraités et moralement abandonnés}} a permis de prononcer la déchéance des droits parentaux pour ces seuls motifs. 

 Le fait de ne pas juger les mineurs et les majeurs selon les mêmes critères est sans doute aussi vieux que la justice elle-même. Refuser de traiter les fautes des mineurs autrement que celles des adultes serait manquer de bon sens. Ce qui fait la différence, c'est l'âge de la coupure entre l'irresponsabilité complète, l'atténuation de la responsabilité \emph{(l'excuse de minorité)} et la responsabilité pleine et entière. Ce sont aussi les peines encourues : nature des peines, durée... La minorité pénale était fixée à 16 ans depuis l'ancien régime. La loi du 12 avril 1904 la repousse de 16 à 18 ans, et elle affirme la prééminence de l'éducatif sur le répressif.

 La loi du 28 juin 1904 s'inscrit dans le courant d'idées qui a confié les enfants \emph{moralement abandonnés} à l'Assistance Publique. Elle ordonne que les pupilles \emph{difficiles} soient confiés non plus à des prisons, mais à des écoles professionnelles publiques ou privées. Ce texte confirme à l'administration du service des enfants assistés, détentrice de la puissance paternelle sur les pupilles, le droit de désigner ceux qu'elle garderait et ceux qu'elle refuserait d'assumer et pour lesquels elle solliciterait l'aide de la Justice. Au même moment c'étaient encore en principe les pères qui définissaient ce qui sous leur toit était indiscipline et insoumission à leur autorité.

 La loi du 24 juillet 1889 {\emph{sur la protection judiciaire des enfants maltraités et moralement abandonnés}} donne aux juges la possibilité de prononcer la déchéance totale de la puissance paternelle pour inconduite des parents, en cas de mauvais traitements ou de délaissement de l'enfant (et de plein droit dans le cas de certaines condamnations infamantes). La même loi confie à l'administration (c'est-à-dire à l'Assistance Publique) la tutelle des enfants maltraités, victimes de crimes ou de délits ou délaissés. Le service les prend en charge même s'ils sont âgés de plus de 12 ans à leur entrée. Ces enfants deviennent des pupilles comme les autres. Ils sont traités à l'instar des autres enfants du service. Quel que soit leur âge, autant que faire se peut ils seront placés en nourrice, pour de longues durées, et dans tous les cas ils seront totalement coupés de leurs parents déchus.

 Une nouveauté majeure est introduite en 1912, avec la création des tribunaux pour enfants (sans magistrats spécialisés) et la création de la liberté surveillée%
% [10]
\footnote{Cf. l'ouvrage collectif \emph{Protéger l'enfant} (1996), qui aborde les problèmes de la jeunesse sous l'angle de la \emph{protection judiciaire}. Il présente un résumé de l'histoire de celle-ci, et des débats d'idée et des conflits de pouvoir qui ont présidé à sa naissance et qui la traversent encore...}%
. Cette nouveauté avait été précédée depuis les années 80 par tout un mouvement d'idées, notamment chez les magistrats chargés de l'application du droit de correction paternelle. Il y a en effet un lien direct entre la dénonciation de l'indignité des pères (cf. la loi de 1889 sur la déchéance paternelle), et la mise en cause du droit de correction%
%[11]
\footnote{Pascale \fsc{QUINCY-LEFEBVRE}, « Une autorité sous tutelle. La justice et le droit de correction des pères sous la troisième république », in \emph{Lien social et politiques-RIAC}, 37, Printemps 1997, p. 99 à 109.}% 
. Ceux qui s'intéressaient à ce problème ne contestaient en aucune façon l'existence d'enfants \emph{insoumis}, difficiles à élever et qui provoquaient le \emph{légitime} mécontentement de leurs parents. Ils estimaient par contre que c'était un problème qui débordait le cadre familial, parce qu'on pensait qu'en règle générale ceux qui étaient insoumis à leurs parents ne faisaient pas de bons citoyens, et risquaient de devenir délinquants, c'est pourquoi l'état ne pouvait s'en désintéresser. Ils estimaient surtout qu'il n'était pas possible de s'en tenir à la parole du parent, et qu'il fallait s'assurer par une enquête approfondie de la réalité et de la nature des problèmes. 

 D'autre part ils estimaient que la prison n'était pas un outil de correction efficace, et qu'il fallait fournir aux jeunes insoumis une prestation éducative de durée suffisante pour obtenir d'eux un amendement réel. Ils pensaient que cette prestation devait être fournie par un internat sous le contrôle de la Justice et non sous celui des pères. Ils accusaient en effet ceux-ci (ceux du moins qui réclamaient à la justice son aide, c'est-à-dire ceux des pères, tous de milieu populaire, qui ne pouvaient supporter les frais d'une pension dans l'un des internats privés dont c'était la spécialité) d'être trop prompts à retirer leurs enfants (comme ils en avaient le droit) dès que ceux-ci semblaient suffisamment \emph{intimidés} par l'incarcération. Ils les suspectaient de n'avoir qu'un seul but, celui de mettre le plus vite possible leurs enfants au travail pour toucher leur salaire. Aux yeux des réformateurs, les droits des pères (éducatifs ou financiers) importaient moins que l'intérêt des enfants, qui était de recevoir une bonne éducation durant le temps nécessaire et avec la sévérité qui convenait, et que l'intérêt de la société, qui était de voir conduire à son terme la \emph{correction des insoumis}. 

 C'est donc du fait des juges et non à la demande de la société que la correction paternelle est peu à peu tombée en désuétude. Ils ont pris l'habitude dès les années 1890 de demander systématiquement une enquête pour vérifier si le parent demandeur avait vraiment des \emph{sujets de mécontentement très graves}, et s'il n'était pas plutôt un parent \emph{indigne}. Ils ont ainsi retiré aux parents leur droit de qualifier eux-mêmes de fautifs les comportements de leurs enfants. Puis la loi de 1889 leur a donné la possibilité non seulement de refuser aux parents indignes une demande de correction paternelle, mais encore de leur retirer la garde de l'enfant. Enfin la loi de 1904 les a explicitement autorisés à mettre les pupilles indisciplinés en maison de correction pendant plusieurs années (c'était déjà le sort des pupilles indisciplinés ou récalcitrants du \siecle{19}). Ces pupilles pouvaient être les enfants de parents déclarés \emph{indignes} une fois que leur déchéance était prononcée. Les parents étaient souvent qualifiés d'indignes parce qu'ils laissaient la bride sur le cou de leur enfant en ne le contrôlant pas d'assez près, ou parce qu'ils entravaient les efforts des éducateurs qui tentaient de les amender. Il semble qu'à cette époque les juges et les premiers travailleurs sociaux déploraient plus leur laxisme que leur autoritarisme. 

 En 1921 une loi ouvre la possibilité de prononcer une \emph{déchéance partielle} de l'autorité paternelle. Une déchéance totale des droits parentaux était une mesure aux effets quasi irréversibles. À partir de 1921 les magistrats n'ont plus été réduits au tout ou rien d'une telle mesure face aux parents qu'ils jugeaient incompétents, délinquants ou négligents. Au contraire ils pouvaient prononcer une déchéance partielle et provisoire, non seulement là où la déchéance totale aurait été injustifiée, mais même là où ils y auraient recouru par nécessité en l'absence d'une mesure plus souple. Le nombre de ces décisions a donc crû rapidement. Cela ne s'est pas traduit par un accroissement important du nombre de jeunes placés, mais par un changement du statut de beaucoup d'enfants placés : le nombre des pupilles a décru au fur et à mesure qu'augmentait celui des enfants en garde, sans que l'effectif total ne se modifie sensiblement. Pendant ce temps le nombre des abandons ne cessait de diminuer.

 On a vu que dès la fin du \siecle{19} des juges avaient commencé d'ordonner des enquêtes pour évaluer la pertinence des demandes de correction paternelle. En 1923 le succès de cette pratique a conduit à la création à Paris, où étaient traitées les deux tiers des demandes de correction paternelle faites en France, d'un service social réalisant pour le tribunal des enquêtes débordant largement la matérialité des faits reprochés par les parents à leur enfant. Désormais la demande d'intervention des parents était entendue comme l'expression d'un dysfonctionnement dans la famille, qui dépassait largement le mineur concerné. Cela entraînait une enquête sociale, c'est-à-dire l'introduction au sein de la famille, d'un observateur extérieur mandaté par les juges. À partir de cette base ces derniers se sont donné le droit de conseiller les parents face aux problèmes que leur posaient leurs enfants : dans la plupart des cas cela les conduisait à mettre en œuvre une action non judiciaire, confiée sous leur contrôle à des institutions privées. Il s'agissait très souvent d'une \emph{action éducative en milieu ouvert}, mais ils pouvaient aussi prendre l'initiative de placer en établissement de correction les mineurs qui leur semblaient en avoir besoin, entre autres au titre des lois de 1889 et de 1921 sur la déchéance paternelle, et de 1904 sur les pupilles difficiles ou vicieux. Tous ces placements écartaient le contrôle paternel.

 Les magistrats n'accédaient plus à la demande de correction paternelle que dans un nombre de cas de plus en plus petit : environ un cas sur quatre ou cinq en 1917, un sur dix dans les années trente. Ils ont ainsi vidé de sa substance le droit de correction paternelle. Le nombre des mineurs placés à ce titre n'a donc cessé de baisser jusqu'à devenir marginal, comme le montre la table \vref{ord-corr-pat}.
 
\begin{table}[h]
\centering
\caption{Ordonnances de correction paternelle}
\label{ord-corr-pat}
\begin{tabular}{ccc}
 & France & part \\
Année & entière & Seine \\
\hline
1881 & 1192 & 63,7~\% \\
1891 & 737 & 59,6~\% \\
1901 & 731 & 50,0~\% \\
1911 & 644 & 66,9~\% \\
1921 & 270 & 83,3~\% \\
1931 & 60 & 66,7~\%
\end{tabular}
\end{table}
 
 Le décret-loi du 30 octobre 1935 sur {\emph{la correction paternelle et l'assistance éducative}} institue l'assistance éducative à domicile. Il entérine les changements qui travaillaient depuis deux générations l'exercice du droit de correction paternelle. Il a également dépénalisé le vagabondage des mineurs (les fugues simples, sans délits caractérisés) ce qui suggère que ces deux ordres de faits se recouvraient. Désormais les jeunes vagabonds ne ressortissaient plus de la Justice, mais mais d'une assistance placée sous le contrôle des juges. 

\section{Vers une assistance non punitive ?}

Les mesures d'assistance en faveur des femmes en couche et des familles nécessiteuses, mais aussi les allocations ouvertes à toutes les familles (allocations familiales, salaire unique, allocations prénatales, etc.), les consultations de nourrissons, les crèches, et les améliorations progressives des conditions de travail, tout cela a facilité la vie des familles. Depuis le début du siècle le nombre des abandons a décru régulièrement et massivement.


Depuis le milieu du \siecle{19}, l'administration pouvait verser aux mères seules une aide afin qu'elles placent elles-mêmes leur enfant chez une nourrice de leur choix. Dans le même esprit, le placement chez une nourrice directement salariée par l'assistance publique a de plus en plus souvent été perçu comme une forme de secours à la famille, et non plus comme le remplacement d'une famille par une autre. On aidait la mère en lui fournissant une nourrice, là où les familles citadines non indigentes de l'époque se la procuraient elles-mêmes. À partir de 1924, l'assistance publique de Paris a commencé de placer en nourrice des enfants non abandonnés, dont les parents n'étaient pas déchus de leurs droits, des enfants qui n'étaient pas des pupilles. 

 Pour cela, il avait fallu franchir une barrière psychologique et oser placer en nourrice pour une durée indéterminée, l'enfant qu'une femme pourrait reprendre un jour. Cela allait contre les pratiques antérieures de l'assistance publique, mais (et ce n'est sans doute pas un hasard) c'est également à partir de l'année 1924 que la loi a permis d'adopter les enfants mineurs. C'est à partir de cette date que le service a la possibilité de procéder à des adoptions d'enfants abandonnés. Face à la réalité d'adoptions authentiques (quel que soit leur nombre réel, quelques centaines par ans semble-t-il), l'illusion que le placement dans une famille nourricière salariée était une espèce d'adoption ne pouvait plus tenir. Il ne pouvait plus être question pour un « nourricier » de prendre la place d'un parent dans le cœur de l'enfant, mais seulement de fournir à ce dernier une assistance pendant un temps plus ou moins long. 

 Au motif que les liens avec leurs parents n'étaient pas coupés, l'administration pouvait laisser les petits enfants concernés en collectivité (on a vu que c'était sa position traditionnelle face aux enfants qui avaient des parents), mais :
\begin{enumerate}
%a)
\item elle priverait alors autant de nourrices de leur emploi alors que les régions pauvres où elles vivaient avaient besoin de ces emplois et que le nombre des enfants abandonnés avait déjà beaucoup baissé depuis le début du siècle,
% b)
\item les nourrices coûtaient moins cher que les internats,
% c)
\item d'autre part, et surtout, on savait qu'en collectivité l'état de santé des petits enfants (0~à 4 ans) se dégrade inexorablement et rapidement au fil du temps, comme l'expérience l'avait régulièrement démontré depuis plusieurs siècles. Dès que le placement courait le risque d'être durable (plus de quelques semaines), il fallait donc autant que possible éviter aux plus jeunes les « dépôts » des enfants de l'Assistance Publique et les « orphelinats ».
\end{enumerate}
 Les jeunes concernés ont donc assez systématiquement été placés en famille d'accueil. On a rapidement observé, comme il était prévisible au vu de l'histoire antérieure de l'assistance, que le placement en famille d'accueil des tout-petits donnait de très bons résultats en ce qui concerne la santé physique et psychologique (indissociables à cet âge). Cette observation a assuré le succès de cette formule. Peu à peu les âges d'admission ont été assouplis. Les enfants qui ont des parents et qui gardent un lien avec eux ont pu entrer en famille d'accueil à un âge de plus en plus avancé, et tous ont fini par bénéficier de cette formule. 

 Le droit des parents au « dépôt » de leurs enfants à l'A.P. a été élargi par la loi du 15 avril 1943. Cette loi ouvrait un droit aux secours (dont le placement est l'une des formes possibles) aux enfants \emph{qui ont un père}, même quand celui-ci est valide et donc capable de travailler. Elle impliquait qu'un homme qui ne peut subvenir financièrement aux besoins de sa famille n'était pas pour autant disqualifié comme époux et comme père. Il n'avait pas pour autant à être sanctionné comme un débiteur insolvable à écarter de sa partenaire et de ses enfants. Il convenait plutôt de l'assister.

 On peut supposer que les séparations familiales et les privations de cette période de rationnement avaient facilité cette décision. L'un de ses objectifs a pu être de fournir une aide aux femmes et aux enfants des prisonniers de guerre retenus en Allemagne (valides certes, mais enfermés au loin et travaillant sans rémunération dans la tradition plurimillénaire de l'esclavage des vaincus). Mais il s'agissait aussi de reconnaître l'évolution des pratiques réelles des services. Preuve en est que cette réglementation n'a pas été abrogée après la guerre. 

 C'était aussi un corollaire du fait qu'on se libérait un peu de la représentation patriarcale du monde qui dominait les siècles précédents : si les hommes n'étaient plus les patriarches tout-puissants qu'on avait pensé qu'ils étaient (ou voulu qu'ils soient), leur impuissance financière ne justifiait plus leur éviction.
 
 \section{Construction de l'État-providence}
 
 À côté des drames qui en ont fait aussi une période tragique (deux guerres mondiales, diverses guerres locales, les guerres de décolonisation et au moins une grande crise économique) le demi-siècle qui va de 1910 à 1960 a vu la fin silencieuse d'une civilisation rurale millénaire (ce qui a représenté un drame d'une autre nature pour bien des gens) et révolutionné la vie quotidienne : création des média de masse, construction de banlieues concentrationnaires, progrès fulgurants de la lutte contre les maladies infectieuses, début de la croissance explosive qui a caractérisé les « trente glorieuses »,~etc. Il a vu l'essor du salariat, celui de multiples caisses d'assurances sociales et de retraite (déjà initié à petite échelle dès la fin du \siecle{19}) puis leur extension à l'ensemble de la population. Il a vu la création des allocations familiales. Il a vu le début de la démocratisation des enseignements secondaire et supérieur, qui étaient à l'époque des outils indiscutables d'ascension sociale. 

 Ces années étaient marquées par la conviction qu'il était possible d'aller vers un monde meilleur, lorsque les forces du mal seraient vaincues (guerres mondiales, guerres coloniales, capitalisme, fascisme, communisme, etc.) et ce monde semblait alors à portée de main. Apparue après la seconde guerre mondiale, l'expression \emph{État-providence} (en Angleterre le « \anglais{Welfare State} », état de bien-être par opposition à l'état de guerre) exprimait un aspect de ce projet : à défaut de faire descendre {\emph{ici-bas}} le paradis, procurer à tous au moins un solide filet de sécurité contre l'indigence et l'abandon social. 

 Cet effort de longue haleine, commencé par endroits dès le \siecle{19}, a obtenu des résultats très significatifs, grâce auxquels à partir de 1945 l'ensemble de la population européenne (entre autres), et en particulier les travailleurs les plus pauvres, a bénéficié d'assurances sociales et de retraites par répartition qui mutualisaient les risques. Ces systèmes imposés par les États rendaient en principe inutile le recours à l'assistance et à la bienfaisance comme la prise en charge des indigents par leur propre parentèle. 

 Le montant des aides financières accordées à tous les parents pour la prise en charge de leurs enfants a connu son apogée entre 1945 et 1965. Elles protégeaient de l'indigence les enfants des pauvres mieux qu'on ne l'avait jamais fait jusqu'alors.  
 
 Jusqu'aux années soixante du vingtième siècle, la législation, les mœurs, les discours dominants et le niveau élevé des aides matérielles à la famille et à la procréation, étaient conformes aux vœux des « populationnistes ». Les allocations familiales n'ont jamais été aussi fortes qu'alors par rapport aux salaires de base, ce qui a contribué à permettre à beaucoup de femmes de rester chez elles élever plus d'enfants qu'elles n'en auraient eu sans cela. Le marché de l'emploi aurait probablement permis à beaucoup d'entre elles de travailler au dehors de leur famille. Si les allocations étaient si substantielles, c'est bien parce que l'atmosphère nataliste d'alors était favorable à cette représentation des familles. Il s'agissait pour l'État de promouvoir les naissances, ce qui justifiait d'aider les familles prolifiques et de ne pas heurter de front les idées de ceux qui les représentaient. C'est dans cette atmosphère idéologique très favorable aux couples conjugaux, aux familles et aux associations qui les représentaient que les enfants du « \anglais{baby-boom} » ont été conçus et élevés. 
 





\chapter{L'État, providence des familles ?}


 À côté des drames qui en ont fait aussi une période noire (deux guerres mondiales, diverses guerres locales, les guerres de décolonisation et au moins une grande crise économique) le demi-siècle qui va de 1910 à 1960 a vu la fin silencieuse d'une civilisation rurale millénaire (ce qui a représenté un drame d'une autre nature pour bien des gens) et révolutionné la vie quotidienne : création des média de masse, construction de banlieues concentrationnaires, progrès fulgurants de la lutte contre les maladies infectieuses, début de la croissance explosive qui a caractérisé les « trente glorieuses »,~etc. Il a vu l'essor du salariat, celui de multiples caisses d'assurances sociales et de retraite (déjà initié à petite échelle dès la fin du \siecle{19}) puis leur extension à l'ensemble de la population. Il a vu la création des allocations familiales. Il a vu le début de la démocratisation des enseignements secondaire et supérieur, qui étaient à l'époque des outils indiscutables d'ascension sociale. 

 Ces années étaient marquées par la conviction qu'il était possible d'aller vers un monde meilleur, lorsque les forces du mal seraient vaincues (guerres mondiales, guerres coloniales, capitalisme, communisme, etc.) et ce monde semblait alors à portée de main. Apparue après la seconde guerre mondiale, l'expression \emph{État Providence} (en Angleterre le « \emph{welfare state} », état de bien-être par opposition à l'état de guerre) exprimait un aspect de ce projet : à défaut de faire descendre {\emph{ici-bas}} le paradis, procurer à tous au moins un solide filet de sécurité contre l'indigence et l'abandon social. 

 Cet effort de longue haleine, commencé dès le \siecle{19}, a obtenu des résultats très significatifs, grâce auxquels à partir de 1945 l'ensemble de la population, et en particulier les travailleurs les plus pauvres, a bénéficié d'assurances sociales et de retraites par répartition qui mutualisaient les risques. Ces systèmes imposés par les états rendaient en principe inutile le recours à l'assistance et à la bienfaisance comme la prise en charge des indigents par leur propre parentèle : il s'agissait de l'immense majorité de la population, et d'abord des salariés. 

 Le montant des aides financières accordées à tous les parents pour la prise en charge de leurs enfants a connu son apogée entre 1945 et 1965. Elles protégeaient de l'indigence les enfants des pauvres mieux qu'on ne l'avait jamais fait jusqu'alors. 

 Les mesures d'assistance en faveur des femmes en couche et des familles nécessiteuses, mais aussi les allocations ouvertes à toutes les familles (allocations familiales, salaire unique, allocations prénatales, etc.), les consultations de nourrissons, les crèches, et les améliorations progressives des conditions de travail, tout cela a facilité la vie des familles et permis que depuis le début du siècle le nombre des abandons décroisse régulièrement et massivement.

 Depuis le milieu du \siecle{19}, l'administration pouvait verser aux mères seules une aide afin qu'elles placent elles-mêmes leur enfant chez une nourrice de leur choix. Dans le même esprit, le placement chez une nourrice directement salariée par l'assistance publique a de plus en plus souvent été perçu comme une forme de secours, et non plus comme le remplacement d'une famille par une autre. On aidait la mère en lui fournissant une nourrice, là où les familles citadines non indigentes de l'époque se la procuraient elles-mêmes. À partir de 1924, l'assistance publique de Paris a commencé de placer en nourrice des enfants non abandonnés, dont les parents n'étaient pas déchus de leurs droits, des enfants qui n'étaient pas des pupilles. 

 Pour cela, il avait fallu franchir une barrière psychologique et oser placer en nourrice pour une durée indéterminée, l'enfant qu'une femme pourrait reprendre un jour. Cela allait contre les pratiques antérieures de l'assistance publique, mais (et ce n'est sans doute pas un hasard) c'est également à partir de l'année 1924 que la loi a permis d'adopter les enfants mineurs. C'est à partir de cette date que le service a la possibilité de procéder à des adoptions d'enfants abandonnés. Face à la réalité d'adoptions authentiques (quel que soit leur nombre réel, quelques centaines par ans semble-t-il), l'illusion que le placement dans une famille nourricière salariée était une espèce d'adoption ne pouvait plus tenir. Il ne pouvait plus être question pour un « nourricier » de prendre la place d'un parent dans le cœur de l'enfant, mais seulement de fournir à ce dernier une assistance pendant un temps plus ou moins long. 

 Au motif que les liens avec leurs parents n'étaient pas coupés, l'administration pouvait laisser les petits enfants concernés en collectivité (on a vu que c'était sa position traditionnelle face aux enfants qui avaient des parents), mais :
\begin{enumerate}
%a)
\item elle priverait alors autant de nourrices de leur emploi alors que les régions pauvres où elles vivaient avaient besoin de ces emplois et que le nombre des enfants abandonnés avait déjà beaucoup baissé depuis le début du siècle,
% b)
\item les nourrices coûtaient moins cher que les internats,
% c)
\item d'autre part, et surtout, on savait qu'en collectivité l'état de santé des petits enfants (0~à 4 ans) se dégrade inexorablement et rapidement au fil du temps, comme l'expérience l'avait régulièrement démontré depuis plusieurs siècles. Dès que le placement courait le risque d'être durable (plus de quelques semaines), il fallait autant que possible éviter aux plus jeunes les « dépôts » des enfants de l'Assistance Publique et les « orphelinats ».
\end{enumerate}
 Les jeunes concernés ont donc assez systématiquement été placés en famille d'accueil. On a rapidement observé, comme il était prévisible au vu de l'histoire antérieure de l'assistance, que le placement en famille d'accueil des tout-petits donnait de très bons résultats en ce qui concerne la santé physique et psychologique (indissociables à cet âge). Cette observation a assuré le succès de cette formule. Peu à peu les âges d'admission ont été assouplis. Les enfants qui ont des parents et qui gardent un lien avec eux ont pu entrer en famille d'accueil à un âge de plus en plus avancé, et tous ont fini par bénéficier de cette formule. 

 Le droit des parents au « dépôt » de leurs enfants à l'A.P. a été élargi par la loi du 15 avril 1943. Cette loi ouvrait un droit aux secours (dont le placement est l'une des formes possibles) aux enfants \emph{qui ont un père}, même quand celui-ci est valide et donc capable de travailler. Elle impliquait qu'un homme qui ne peut subvenir financièrement aux besoins de sa famille n'était pas pour autant disqualifié comme époux et comme père. Il n'avait pas pour autant à être sanctionné comme un débiteur insolvable à écarter de sa partenaire et de ses enfants. Il convenait plutôt de l'assister.

 On peut supposer que les séparations familiales et les privations de cette période de rationnement avaient facilité cette décision. L'un de ses objectifs a pu être de fournir une aide aux femmes et aux enfants des prisonniers de guerre retenus en Allemagne (valides certes, mais enfermés au loin et travaillant sans rémunération dans la tradition plurimillénaire de l'esclavage des vaincus). Mais il s'agissait aussi de reconnaître l'évolution des pratiques réelles des services. Preuve en est que cette réglementation n'a pas été abrogée après la guerre. 

 C'était un corollaire (paradoxal) du fait qu'on sortait peu à peu de la représentation patriarcale du monde qui dominait les siècles précédents : si les hommes n'étaient plus les patriarches tout-puissants qu'on avait pensé qu'ils étaient (ou voulu qu'ils soient), leur impuissance financière ne justifiait plus leur éviction.
 
 



%K1 démantèlement de la famille traditionnelle
%K2 Victoire du mariage d'inclination
%K3 "Le corps des femmes est à elles"
%K4 Incestes et paradoxes
%K5 Perplexités éducatives
%K6 Désarrois masculins
%K7 Inertie des pratiques
%L1 Un enfant pour quoi ? pour qui ?
%L2 Qui pour accueillir l'enfant ?
%L3 Droit à l'enfant ?
%L4 Progrès ou régressions ? 

\part{Depuis 1960, le temps des expériences}

% Le 19 mars 2015 :
% ~etc.
% Moyen Âge
% Antiquité
% " --> « ou » ou \enquote{}
% même


\chapter[Démantèlement de la famille traditionnelle]{Démantèlement\\de la famille traditionnelle}


 

\section{Révolution dans le droit}

\begin{description}

\item[1961] Une mesure administrative qui sur le moment n'a pas frappé beaucoup d'esprits, mais dont l'importance symbolique n'en est pas moins significative (les grandes fractures commencent souvent par une fissure imperceptible à l'œil nu) : le ministère de l'éducation nationale supprime le caractère obligatoire de l'enseignement du droit romain dans le programme de la licence de droit, obligation qui datait de la création des études de droit dans les universités aux \crmieme{11} et \siecle{12}s. Ce corpus n'est plus qu'une option facultative parmi d'autres. 

\item[1965] La loi du 13 juillet lève les derniers obstacles à l'exercice d'une activité commerciale par les femmes mariées sans la tutelle de leurs maris. Ceux-ci ne gèrent plus de droit les biens propres de leurs épouses (dot,~etc.). Elles n'ont plus à obtenir leur autorisation pour exercer une profession séparée, quelle qu'elle soit.

\item[1966] La loi du 11 juillet sur l'adoption assimile les enfants adoptés aux enfants légitimes non adoptés (adoption plénière). 

%\item[1966] la (même)
Cette
loi du 11 juillet ouvre le droit à l'adoption plénière à une personne seule, qu'elle soit célibataire ou non et quelles que soient ses préférences sexuelles, d'au moins 28 ans.

\item[1967] La loi \fsc{NEUWIRTH} dépénalise la prévention des naissances : elle autorise la publicité concernant les méthodes anticonceptionnelles (interdite depuis les années 20), et elle autorise leur mise à disposition du public :
%la première visée et la principale était la « pilule » anticonceptionnelle 
la « pilule » anticonceptionnelle était principalement visée,
qui venait d'être mise au point. L'accord du mari n'est pas nécessaire, son refus n'a pas d'effet.

\item[1972] La loi du 3 janvier fait entrer les enfants naturels dans la famille du ou des parents qui les ont reconnus. À quelques restrictions près -- enfants adultérins, elle leur ouvre un droit à l'héritage égal à celui des enfants légitimes.

%\item[1972] 
La puissance paternelle est abolie au profit de l'autorité parentale. En cas de séparation, cette autorité est conservée à égalité par chacun des deux parents. 

%\item[1972] 
Une loi ordonne l'égalité des salaires féminins avec les salaires masculins.

\item[1974] L'âge de la majorité légale est abaissé de 21 à 18 ans. 

\item[1975] Loi du 30 juin relative aux institutions sociales et médicosociale : les usagers et les familles doivent être associés au fonctionnement de l'établissement qui les prend en charge (il doit les « prendre en compte » ).  Le pa(ma)ternalisme des institutions d'aide sociale est un peu affaibli.

%\item[1975] 
À côté du divorce pour faute, la loi du 11 juillet ouvre la possibilité de divorcer par consentement mutuel ou pour rupture de la vie commune. Par ailleurs, cette loi met les deux époux à égalité en matière de choix résidentiel et en matière de contribution aux charges du mariage.

%\item[1975] 
La loi \fsc{WEILL} dépénalise l'avortement (\emph{interruption volontaire de grossesse} ou IVG). La loi ne demande pas l'avis des maris éventuels.

%\item[1975] 
Les épouses ne sont plus tenues de faire usage du nom de leur mari dans la vie quotidienne et les relations avec l'administration.

%\item[1975] 
L'adultère féminin est dépénalisé. Ce n'est plus un délit qui concerne la société, ce n'est qu'un affront privé qui ne concerne que le mari.

\item[1976] Loi du 22 décembre relative aux conditions d'adoption : la présence d'enfants légitimes n'est plus un obstacle à l'adoption, même si leur avis est entendu. 

\item[1978] La loi du 6 janvier donne à tout individu majeur le droit de connaître le contenu de tout dossier administratif le concernant. Cela concerne notamment tous les enfants abandonnés.

\item[1983] Un arrêt de la cour de cassation du 21 mars 1983 admet la légalité de la garde conjointe de l'enfant après divorce.

%\item[1983] 
Loi sur l'égalité professionnelle entre femmes et hommes.

\item[1984] La loi du 6 juin relative aux \emph{droits des familles dans leurs rapports avec les services chargés de la protection de la famille et de l'enfance}, et au statut des pupilles de l'État, donne aux parents des droits plus étendus face à l'administration. L'autorité des parents sur leurs enfants placés à l'ASE est confortée dans tous les domaines (sauf limites définies expressément par un juge).

\item[1985] La loi du 23 décembre 1985 met les deux parents à égalité dans la gestion des biens de l'enfant : ils exercent cette tâche conjointement quand ils exercent en commun l'autorité parentale. Sinon l'un des deux l'exerce sous le contrôle du juge.

\item[1987] L'autorité parentale est redéfinie par la loi du 22 juillet (loi \fsc{MALHURET}) en termes de \emph{responsabilité parentale ordonnée à l'intérêt de l'enfant}. Elle est à égalité assumée par chacun des deux parents, qu'ils cohabitent ou pas.

\item[1989] \emph{Convention Internationale des Droits de l'Enfant} promulguée dans le cadre de l'ONU le 20 novembre : reconnaît le droit de tout mineur à une famille, et ses droits face à sa propre famille.

%\item[1989] 
Loi du 10 juillet \emph{relative à la prévention des mauvais traitements à l'égard des mineurs et à la protection de l'enfance}. Elle prévoit que le délai de prescription ne court qu'à partir de la majorité pour les mineurs victimes de violences.

\item[1993] L'autorité parentale conjointe devient la règle pour les couples de concubins comme pour les couples mariés.

\item[1996] Convention européenne du 25 janvier sur l'exercice des droits de l'enfant. Elle donne le droit aux enfants mineurs de donner leur avis sur les mesures qui les concernent lors du divorce de leurs parents.

\item[1999] Création du PACS : pacte civil de solidarité, ouvert aux couples hétérosexuels et aux couples homosexuels.

\item[2000] La pilule {\emph{du lendemain}} est en vente libre dans les pharmacies, et distribuée gratuitement aux mineures par les infirmières scolaires sur simple demande de la mineure, sans demander l'avis de ses parents, et sans qu'ils en soient informés.

\item[2002] Sur décision de la Cour Européenne de Justice les dernières discriminations juridiques que subissaient en matière d'héritage les enfants adultérins et incestueux sont effacées. Seuls sont distingués les enfants nés des incestes parent--enfant, qui ne peuvent être reconnus que par un seul de leurs deux parents. Ils doivent néanmoins être traités absolument en tout le reste, et d'abord en ce qui concerne l'héritage, comme leurs éventuels demi-frères ou sœurs. 

%\item[2002] 
Loi du 4 mars : {[...] \emph{les parents associent l'enfant aux décisions qui le concernent, selon son âge et son degré de maturité}}. L'administration de la famille par les deux parents doit être démocratique.

\item[2005] La loi autorise les femmes à donner à leurs enfants leur propre nom à égalité avec leur mari à compter de janvier 2005. 

%\item[2005] 
Interdiction du mariage des filles avant dix-huit ans (traditionnel âge au mariage des garçons).

\item[2013] Loi \fsc{Taubira} : ouverture du mariage aux couples de même sexe.

\item[2013] Remboursement à 100~\% de l'IVG.

\item[2014] Suppression de l'exigence d'une « détresse » pour reconnaître à une femme enceinte son droit à un avortement. 
\end{description}
 
 \section{Le corps des femmes est à elles}


 La pilule anticonceptionnelle ("la pilule") a été autorisée en France quelques années seulement après sa mise au point : en 1967. Et elle l'a été par une assemblée de députés dans laquelle il n'y avait pratiquement que des pères de famille, qui ont apparemment plus pensé aux intérêts de leurs épouses et de leurs filles qu'à la défense du patriarcat. Dès ce moment la pilule a été largement utilisée par toutes les femmes majeures, célibataires ou mariées, et par bien des mineures. Grâce à elle, les femmes pouvaient prendre l'initiative d'une rencontre sexuelle sans obérer leur avenir. Cela a permis de constater que si c'était sans risque de grossesse, bien des parents ne refusaient pas que leurs filles aient une vie sexuelle hors mariage. Il n'était plus nécessaire de donner une valeur à la virginité ou à la chasteté des femmes non mariées et les filles n'étaient plus contraintes par le risque de grossesse de fuir les garçons ni de nier, de réprimer ou de refouler leurs propres désirs sexuels
\footnote{En Janvier ou février 1968, les garçons de la cité universitaire de Nanterre réclament bruyamment le droit d'entrer dans les chambres des filles de la cité, jusque là sanctuaires (en principe) inviolés. Ce fait divers, grand-guignolesque à nos yeux d'aujourd'hui, n'en a pas moins servi de détonateur à la chaîne d'évènements mémorables qui ont culminé au mois de Mai de cette année-là. Il se trouve que la disponibilité (alors toute nouvelle, mais déjà répandue comme une trainée de poudre) de la pilule anticonceptionnelle, venait d'enlever à cette revendication une part du caractère scandaleux (à tout le moins angoissant pour les pères et mères des jeunes filles) qu'elle aurait eu peu de temps auparavant. Le sexe librement recherché pour lui-même devenait un jeu sans enjeu dramatique. Ce n'est que bien après ce printemps-là que le SIDA est venu lui rendre une gravité nouvelle.}% 
. Elles ont reçu une liberté égale à celle de leurs frères, et l'âge moyen de leurs premiers rapports sexuels (autant qu'on puisse le connaître) est rapidement passé de 21 à 17 ans (comme eux).

 La « pilule » a-t-elle été inventée dès que son emploi est apparu comme acceptable ? Ou bien est-ce plutôt le contraire ? On peut en effet se demander pourquoi les préservatifs, disponibles en vente libre en pharmacie depuis le début du \siecle{20} (officiellement en tant qu'outil de prévention contre les maladies vénériennes) n'ont pas été employés en France comme un outil de prévention des naissances, alors qu'ils l'étaient dans d'autres pays ? Et pourquoi la pilule ne s'est pas heurtée à la même réticence ? 

 C'est que ce sont les femmes qui ont la maîtrise de cet outil-là : les députés leur ont en effet accordé le droit de prendre la pilule anticonceptionnelle même en cas de désaccord avec leurs maris. Elles \emph{peuvent} la prendre sans le dire à leurs partenaires. Elles \emph{peuvent} aussi cesser de la prendre sans les prévenir. Et\emph{ ils n'y peuvent rien}. Avec l'appui de la législation et des pouvoirs publics, le Planning Familial, héritier des néo-malthusiens, s'emploie à rendre effective cette liberté pour toutes les femmes, mineures comme majeures. 

 C'est dans la foulée de cette première mesure que l'avortement a été autorisé par la loi. Il ne s'agissait plus de l'avortement à la romaine : celui de l'épouse ou de l'esclave sur l'ordre du \emph{pater familias}, ou avec son accord exprès. Désormais une femme peut prendre seule l'initiative d'un avortement, en dépit du refus de son compagnon, comme elle peut garder leur enfant même s'il lui demande d'avorter\footnote{Certes en accouchant « sous X",  mode d'accouchement aux origines très anciennes et ouvert aux femmes mariées autant qu'aux autres, puisqu'elles n'ont pas à donner leur identité, les femmes ont toujours pu priver de paternité, en toute légalité, le géniteur de l'enfant qu'elles portaient, mais il est peu vraisemblable que cela ait été leur objectif losqu'elles recouraient à cette procédure et l'immense majorité des accouchements sous X a concerné et concerne encore des femmes seules et sans soutien.}.
 
  \section{Personne n'est illégitime}
 
 Aujourd'hui tout se passe comme s'il n'existait plus que des enfants légitimes : tout enfant a vocation à faire partie de la famille de chacun de ses deux géniteurs quelles que soient les circonstances de sa conception. Tout enfant a vocation à hériter de ses deux parents à égalité avec ses éventuels demi-frères et demi-sœurs. Tout enfant est un « enfant de famille ». On peut aussi bien dire que tous les enfants sont devenus « naturels » et que la notion même de légitimité s'est évaporée, réduite à un mot sans épaisseur puisqu'il n'a plus de prise sur rien, puisque les effets concrets de la légitimité sont les mêmes que ceux de l'illégitimité, et inversement.
 \begin{displayquote}
\emph{"Cette distinction légitimité-illégitimité était totalement structurante de la société. Aujourd'hui il n'est pas un pays qui n'ait soit complètement aboli cette distinction, soit s'apprête à l'abolir. C'est un changement majeur des rapports entre famille et société qui montre que nous sommes face à des changements de la structure sociale elle-même"}.
\footnote{Irène \fsc{THERY}, \enquote{Peut-on parler d'une crise de la famille ? Un point de vue sociologique}, \emph{Neuropsychiatrie de l'enfance et de l'adolescence}, 2001, 49, 492-501, p. 403.}% } 
\end{displayquote}

La cheville qui depuis \nombre{1600} ans tenait ensemble tout le système de la famille constantinienne a été retirée, et cela se passe apparemment à la satisfaction de tous. Puisque ni les parents ni les enfants ne risquent plus aucun désagrément du fait d'une naissance illégitime, à quoi bon le mariage, surtout quand on est convaincu que le seul couple légitime c'est celui qui repose sur l'accord quotidien de deux volontés. Depuis plus d'une génération, le nombre d'enfants nés hors mariage a donc progressé en même temps que croissait leur assimilation aux enfants nés dans le mariage. Aujourd'hui ils représentent la moitié des naissances. 

 Dans la France d'aujourd'hui, l'illégitimité a cessé d'être honteuse et il n'est plus socialement utile que le nom que l'on porte atteste qu'on a été reconnu par un homme. Une loi de 2001 autorise (donc ?) les couples mariés à donner à leurs enfants le patronyme de la mère à la place de celui du père, ou bien à côté de lui (texte complété par la loi \fsc{Taubira} de 2013). Ce changement est significatif, puisque la pratique traditionnelle n'était pas celle-là (contrairement à l'Espagne, par exemple) tout comme est significatif le moment où il a été institué. 


 
 \section{Victoire du mariage d'amour}


 
 Aujourd'hui, à la condition de posséder une vraie qualification professionnelle (capital intellectuel), il n'est plus besoin de capitaux pour s'établir. Rien ne vaut un « bon » métier : un métier qui implique beaucoup de savoirs et de savoir-faire, dans un secteur d'activité porteur. Il n'est plus honteux de « servir ». Au fil du \siecle{20}, le salariat s'est souvent révélé plus sûr que la possession de capitaux ou d'outils de production, surtout au service de l'État. D'autre part la scolarité est désormais gratuite ou presque jusqu'aux niveaux les plus élevés (même si ce n'est souvent pas vrai aux niveaux les plus élevés). Les parents ont intégré cette logique : depuis la Libération, le taux de scolarisation n'a cessé de s'accroître bien au-delà de la fin de l'obligation scolaire, qui elle-même est passée de 14 (1936) à 16 ans (1959). Le nombre des diplômés de l'enseignement supérieur a explosé. Le nombre de bacheliers se situe actuellement entre 60~\% et 80~\% d'une classe d'âge, contre 5~\% à la Libération et 8~\% en 1960. Même si ce diplôme s'est largement dévalué et ne peut plus depuis longtemps procurer un emploi à lui seul, le niveau de culture moyen a indiscutablement progressé. 

 Aujourd'hui la rentabilité du travail domestique a été fortement réduite par les innovations techniques, commerciales et sociales du \siecle{20} : infrastructures collectives (électricité, tout à l'égout, eau courante) ; machines qui économisent le temps de travail (chauffage central, machine à laver le linge, la vaisselle, cuisines équipées électriques ou au gaz, aspirateur,~etc.) ; grande distribution qui rend non-compétitive l'auto production en couture, en jardinage vivrier, en préparation des aliments,~etc. Sans oublier les écoles maternelles et les garderies d'enfants. Si l'on peut dire que les ménagères ont été libérées d'une grande partie du poids des tâches domestiques, cela signifie aussi qu'elles été réduites au chômage technique, ce que traduit le fait que c'est au même moment qu'on observe la fin des « bonnes ». Pour contribuer significativement aux ressources de leur ménage les épouses doivent désormais travailler au dehors de leur foyer. Cela leur a ouvert la possibilité de se trouver un autre emploi que celui d'être la « maîtresse de maison » d'un homme mais on peut aussi bien dire que cela les  y a contraintes. 
 Elles y ont d'autant plus été contraintes que depuis la libéralisation du divorce elles ne peuvent plus être assurées, comme l'étaient leurs grand-mères, de leur position d'épouse titulaire d'un homme nommément désigné. Les filles de la bourgeoisie ont compris que leur avenir serait mieux assuré par un « bon » métier que par une « belle » dot, un « beau » parti, et par le « grand » mariage qui était jusqu'alors le point de focalisation de tous les désirs familiaux, le signe et le sommet de la réussite féminine (celle des filles et celle des mères). Toutes ont compris que grâce aux savoirs et aux diplômes elles seraient libres : indépendantes des désirs d'un homme et de sa bonne volonté, à l'abri des effets matériels des répudiations, en mesure de prendre l'initiative et de sortir des situations affectives ou familiales dans lesquelles elles ne trouveraient pas leur compte. Dans la course au diplôme les filles se sont (donc ?) montrées significativement plus déterminées que les garçons (quant aux décrochages scolaires de ces derniers, leurs causes n'ont probablement rien à voir avec les motifs de la détermination des filles). 

 S'il n'est plus nécessaire pour « s'établir » d'avoir l'appui financier ni des relations de ses parents, alors le mariage ne scelle plus l'alliance (économique surtout) de deux familles : alors rien n'exige plus que les jeunes gens subissent un mariage arrangé, un mariage d'argent et d'entregent. Ils peuvent sans risque \emph{matériel} s'offrir le luxe de n'être pas raisonnables et de baser leur couple sur la seule passion amoureuse : aujourd'hui les autres stratégies ne sont pas plus raisonnables que celle-là.

 C'est en tout cas tellement devenu notre logique que cela révolutionne notre compréhension du mariage lui-même. Si celui-ci se définit d'abord comme l'union de deux personnes qui s'aiment, alors la question de la durée perd de son sens. L'authenticité des désirs inscrits dans les actes posés ici et maintenant a plus d'importance que la fidélité à une promesse ancienne. L'infidélité conjugale n'est plus une offense à un ordre public qui ne se donne plus pour but de sanctuariser les familles. Ce n'est plus qu'une offense privée, le signe d'un désaccord entre deux associés. La séduction devient une obligation permanente. L'accord du conjoint à une relation charnelle ne peut plus être tenu pour acquis d'avance, par contrat. La notion de \emph{devoir conjugal} s'est vidée de son sens, et la loi ne le reconnaît plus. La notion de viol entre époux prend du sens, et comme tout viol c'est un délit punissable par la loi. 

 Si c'est l'amour mutuel qui fonde le couple, alors sa fécondité potentielle perd de son importance. Que le couple soit constitué d'un homme et d'une femme ne va plus sans le dire. La reconnaissance publique d'un couple de deux hommes ou de deux femmes n'est plus impossible à penser. 

 Mais si c'est l'enfant qui fait la famille, et s'il héritera de ses deux parents quoi qu'il arrive, alors à quoi bon se marier ?

 
 \section{Nouveaux jugements sur les violences sexuelles}

L'abondance actuelle, depuis les années 1985-1990, des discours sur les \emph{abus} sexuels intra familiaux 
\footnote{... comme s'il y avait un usage correct du sexe entre les générations différentes au sein des familles ?} 
signifie-t-elle qu'il s'en commet plus qu'autrefois ? Si l'on en croit le témoignage de Jeannine \fsc{NOEL} (1965) il est permis d'en douter : selon elle entre le quart et le tiers des adolescentes placées à l'Hôpital Hospice Saint Vincent de Paul
\footnote{À cette époque c'était encore le Foyer de l'Enfance de Paris (anciennement « dépôt de l'Assistance Publique ») recevant (souvent avant une orientation ailleurs) tous les jeunes dont les parents ne pouvaient pas s'occuper ou de l'autorité desquels ils avaient été soustraits par décision de justice. On plaçait et place toujours dans les foyers de l'enfance les jeunes qui n'ont pas d'autre lieu où aller, quelle que soit la raison qui les a mis dans cette situation.} 
 au cours des années cinquante du \siecle{20} avaient été confrontées à des problèmes de ce genre : la situation ne semble pas être pire aujourd'hui. 
 
Par contre depuis un demi-siècle toutes les formes de violences sexuelles, qu'elles soient extra ou intra familiales, ont été regardées avec un oeil nouveau. Meme si ce n'est que très progressivement que l'on a pris conscience de la gravité de leurs effets sur leurs victimes il nous est devenu moins difficile de nous identifier aux souffrances de celles-ci. Cela s'est traduit par la requalification de certaines actes délictueux, et surtout par une nouvelle façon d'écouter les plaignants et plaignantes et d'accorder crédit à leur parole.  
 

 Alors pourquoi n'est-ce qu'aujourd'hui que le caractère absolu des secrets professionnels imposé aux professionnels susceptibles de découvrir des violences sexuelles sur mineurs a été mis en question ? Pourquoi n'est-ce qu'aujourd'hui que l'évocation des sévices intra familiaux obtient un tel effet ? Avait-on peur d'ébranler l'autorité et la représentation d'une institution familiale sacralisée, et préférait-on lui sacrifier ses victimes ? Pensait-on que ces délits et ces crimes, aussi condamnables qu'ils étaient, ne pouvaient être traités pénalement, et qu'il était préférable de les recouvrir du \emph{manteau de Noé} ? 
 
 
 \section{Désarroi masculin}


 
S'il veut une femme et/ou des enfants un homme ne peut plus s'y prendre aujourd'hui comme naguère. Il ne lui sert plus à rien de demander à un futur beau-père la main de sa fille, de lui demander un transfert d'autorité, puisque ce dernier ne la détient plus et ne peut donc plus la donner. D'ailleurs lui-même n'a plus besoin d'un gendre pour légitimer les petits enfants que sa fille lui donnera et pour en faire des héritiers, puisqu'il n'y a plus de fonctions interdites aux enfants illégitimes et donc plus d'enfants illégitimes. Il n'y a donc plus d'intérêt commun entre beau-père et gendre, et le soupirant doit négocier seul et sans intermédiaire avec la femme dont il recherche les faveurs. Il n'aura d'elle des enfants que si elle le veut bien. Et elle pourra d'autant plus facilement le quitter en emmenant leurs enfants communs (ou le pousser hors du domicile familial) que l'absence d'un homme à côté d'une mère ne fait plus problème, tandis que la présence de celle-ci semble encore indispensable\footnote{Cela changera peut-être si on constate des aptitudes au "maternage" chez les pères célibataires ou les couples homosexuels masculins ?}. 
 
 Les ressources dont disposent les hommes (puissance économique, compétences culturelles et professionnelles, pouvoir politique,~etc.) ont la vertu de les rendre désirables. Ils se doivent comme toujours d'être « ceux qui peuvent », ceux qui ne sont pas marqués par le manque ou la défaillance. Plus ils sont intellectuellement et professionnellement qualifiés, plus ils ont de probabilités d'être mariés. C'est le contraire pour les femmes, ce qui suggère que dès qu'elles ont les moyens de leur indépendance elles n'ont plus intérêt à être mariées. Cela confirme la solidité de la répartition traditionnelle des rôles masculins et féminins.
 
 On a vu que « l'obligation de résultat », l'obligation de fécondité, qui pesait sur les seules femmes mariées a été supprimée par Constantin, qui a exclu la stérilité des motifs de divorce. La loi impériale romaine a ensuite confirmé l'interdit fait aux chrétiens de se remarier après divorce. À l'obligation de fécondité des épouses s'est substituée une obligation de moyens pour chacun des deux époux : ne pas mettre d'obstacle aux fécondations autre que l'abstinence 
\footnote{En France les relevés démographiques montrent l'érosion progressive du respect de cette obligation, et l'extension depuis trois siècles des pratiques anticonceptionnelles : ce que les anciens moralistes nommaient les « \emph{funestes secrets} ».}. Si les femmes mariées ont ainsi été protégées contre la répudiation et contre la privation de leurs enfants, par contre la loi ne les autorisait pas plus qu'avant à se dérober au « devoir conjugal » lorsque leur mari l'exigeait, ni aux grossesses qui en découleraient, et à leurs risques, sauf à demander une séparation. 

 Depuis 1967, même si leurs maris le désirent, les femmes mariées ne sont plus tenues par la loi de laisser libre cours à leur fécondité. Aujourd'hui le corps des femmes est à elles, \emph{y compris l'embryon ou le fœtus, qui juridiquement en fait partie depuis 1975}, comme c'était le cas dans le droit romain antique. La loi ne se soucie plus de soutenir le désir masculin en ce domaine. Même si elles sont leurs épouses, même s'ils sont les géniteurs de l'enfant qu'elles portent, même si elles avaient été d'accord pour le concevoir avec eux, les hommes n'ont plus le droit d'exiger des femmes qu'elles donnent naissance à cet enfant. Elles peuvent choisir d'avorter ou de l'abandonner à la naissance en dépit du désir du géniteur de l'enfant. On est au plus loin du droit du \emph{pater familias} romain de faire surveiller la grossesse et l'accouchement de son épouse (ou ex épouse), pour qu'elle ne puisse pas lui dérober un enfant né de ses œuvres.

 Dans le même temps ont été supprimées toutes les limites légales qui pouvaient interdire le rattachement d'un enfant naturel à un homme. Comme sous l'ancien régime une mère qui le demande recevra toujours l'appui de la justice pour rechercher le géniteur de son enfant (sauf insémination avec donneur), quelle que soit la situation personnelle de cet homme, mais désormais cela se fera avec une efficacité  imparable. Aucun père n'est plus « \emph{incertus} ». Vivant ou mort son ADN le désignera, sauf lorsque la mère veut cacher son identité à son enfant ou aux tiers. Si la mère le veut, le géniteur sera contraint d'assumer financièrement un enfant qui héritera de lui à part entière, contrairement à ce qui se passait jusqu'au \siecle{19}. Mais cela ne lui donnera pas forcément le moindre droit sur l'éducation de l'enfant : en ce sens cela n'en fera pas un père. Si une femme qui accouche « sous X » refuse de laisser à son enfant des renseignements sur sa propre identité, elle en a le droit. 

 Pour l'essentiel, on peut donc dire que la maîtrise de la génération est passée du côté des femmes. La famille monoparentale d'aujourd'hui, c'est le plus souvent la famille \emph{moins} le père. Dans la grande majorité des séparations (85~\%) ce sont les mères qui gardent les enfants. Est-ce pour ces raisons que l'initiative des divorces vient des femmes beaucoup plus souvent (trois fois sur quatre) que des hommes  ? 




  \section{Police des familles ?}
 
 
Selon Jacques \fsc{Donzelot} (\emph{La police des familles}, 1977), nous sommes passés du gouvernement « des » familles au gouvernement « par les » familles.  Aujourd'hui le pouvoir royal des pères sur leurs enfants est mort, et celui des mères en même temps et ils ne leur reste plus que celui que l'état leur concède, à la condition qu'ils se conforment aux modèles promus par ce dernier.

\begin{displayquote}
{\emph{Ce qui caractérise la loi de 1970 (qui substitue l'autorité parentale à la puissance paternelle) ce sont trois concepts au centre de la réforme, celui « d'égalité » des époux et parents, celui « d'intérêt de l'enfant » et enfin celui de « contrôle judiciaire » devenu nécessaire pour arbitrer d'éventuels conflits entre les parents, entre parents et enfants. On assiste à un recentrage des positions de chacun des membres de la famille. Au centre l'enfant, en face de lui, responsables de lui, ses parents. Entre les deux des médiateurs, les spécialistes judiciaires}%
\footnote{Françoise \fsc{HURSTEL}, \emph{La déchirure paternelle}, p. 117.}%. 
}.
\end{displayquote}

Nous avons assisté à la délégitimation de la justice domestique, du droit des deux parents à régler eux-mêmes sans tiers extérieur tous les conflits intra familiaux. Lorsqu'ils ne réussissent pas à se mettre d'accord entre eux ou avec leurs enfants, ils sont désormais contraints (par leur égalité elle-même) à recourir à un tiers extérieur pour arbitrer leur différend.  

De nouveaux personnages se sont imposés au sein des familles. Sous l'autorité des juges les travailleurs sociaux et les experts (psychologues, psychiatres, médiateurs,~etc.) sont entrés dans le champ, jusque là bien clos, des familles ordinaires, des familles non stigmatisées au préalable comme défaillantes (en ce qui concerne les familles reconnues officiellement comme incompétentes ou délinquantes, c'est depuis toujours que les représentants de la société y avaient leurs entrées). Ils font régner la bonne parole et les bonnes pratiques et vérifient que les familles adoptent les bonnes pratiques dans la prise en charge de leurs enfants, pratiques définies par les mêmes personnages. Quand ils l'estiment nécessaire ils ont l'appui des autorités pour faire passer le message.


 

 Assiste-t-on à la disparition de la sphère privée, cette sphère de la vie de chacun qui se définit par le fait que tant qu'il n'enfreint aucune loi, il n'a aucun compte à rendre sur ce qui s'y passe, et surtout pas à l'État et à ses représentants ? 

 Tout ce qui concerne les enfants est-il entré dans le domaine public, alignant le traitement de l'ensemble des familles sur celui qui était autrefois réservé aux seuls « cas sociaux », et mettant implicitement en cause l'aptitude des parents à défendre suffisamment bien (en \enquote{\emph{bons pères de famille}}) l'intérêt de leurs propres enfants ?



\section{Inertie des pratiques}


 Dans la réalité les changements ne sont pas (encore ?) aussi importants que dans l'idée que l'on s'en fait. Depuis une génération le nombre de mariage diminue innexorablement : en 1990, 90~\% des couples existants étaient mariés, en 1999, année où le Pacs est entré dans les pratiques ils n'étaient plus que 83~\% (\emph{Histoires de familles, histoires familiales}, INSEE, 1999). 
 Le Pacs, à l'origine pensé pour les couples homosexuels, est en réalité le plus souvent choisi par des couples mixtes (dix-neuf pacs sur vingt sont contractés par eux), dont près de la moitié finit par se marier, et la somme des Pacs et des mariages est plus élevée que le nombre des seuls mariages avant la création du Pacs. Le lien entre naissances et mariage semble solide : à la naissance du deuxième enfant 86~\% des couples sont mariés, et 93~\% au troisième. On est donc, avec le Pacs, tout près d'un mariage à l'essai.

 Le nombre des divorces se situe aujourd'hui entre le tiers et la moitié de celui des mariages. Ce nombre est élevé ou bas suivant le point de vue. Sur 29~millions d'adultes vivant en couple, mariés ou non, 26~millions (90~\%) en sont \emph{toujours} à leur première expérience de couple, et pour l'instant les recompositions de familles concernent \emph{seulement} 3~millions de personnes sur 29. C'est que le nombre de couples mariés de tous âges (le « stock ») est si important que les divorces n'en représentent pas plus de 1~\% par an : 99~\% des gens qui étaient mariés au premier janvier le sont encore au 31 décembre qui suit (mais qu'en sera-t-il de ces chiffres dans une génération ?). 

 En 2006, 1,2~millions de mineurs vivaient en famille recomposée, soit 9~\% de l'ensemble des mineurs. Parmi ces mineurs, \nombre{400000} sont nés des deux membres du nouveau couple. Ceux-là vivaient donc avec leurs deux parents, bien que dans une famille "recomposée". À la même date, 2,2~millions de mineurs vivaient au sein d'une famille monoparentale (six fois sur sept avec leur mère), tandis que 10,25~millions de mineurs
vivaient avec leur père et leur mère (dont les \nombre{400000} enfants vivant au sein de familles recomposées et nés du couple nouveau). 
 


\newlength{\lcol}
\setlength{\lcol}{0.16666667\textwidth}
\addtolength{\lcol}{-2\tabcolsep}


\begin{table}[!ht]% [!htb]
%\centering
\begin{minipage}{\textwidth} 
\caption[Cadre de vie des jeunes en 1999]%
{Cadre de vie des jeunes en 1999%
\footnote{Sources :
« Histoires de familles, histoires familiales », \emph{Les cahiers de l'INED}, \no 156 ;
\emph{Recensement de la population}, INSEE, 1999, p. 281.} }
\label{tableau-cadre-vie-1999}
\begin{tabular}{*{6}{>{\hspace{0pt}\centering\arraybackslash}b{\lcol}}}
Âge des jeunes (années) & Vivant avec les deux parents de naissance & Avec un parent seul%
\footnote{Familles monoparentales.}
 & Avec un parent et un beau-parent%
\footnote{Familles recomposées.}
 & Autres situations%
\footnote{En internat, en appartement, en chambre, chez un logeur, en placement ASE, en prison, en hôpital,~etc.}
 & Total\\
\hline
 0-4     & 85,0 & 11,1 & 1,8 & 2,2  & 100~\% \\
 5-9     & 77,7 & 15,6 & 5,2 & 1,5  & 100~\% \\
 10-14 & 72,7 & 17,5 & 8,4 & 1,5  & 100~\% \\
 15-19 & 68,5 & 18,7 & 8,6 & 4,1  & 100~\% \\
 20-24 & 43,5 & 11,5 & 4,3 & 40,6 & 100~\% \\
\hline
 0-17  & 76,5 & 15,7 & 6,0 & 1,8  & 100~\%
\end{tabular}
\end{minipage}
\end{table}

%CADRE DE VIE DES JEUNES EN 1999[6]
% 
%\emph{Age des jeunes}
%\emph{ (années)}
%\emph{Vivant avec ses deux parents de naissance}
%\emph{Avec un parent seul[7]}
%\emph{Avec un parent et un beau-parent[8]}
%\emph{Autres situations [9]}
%\emph{Total}
%\emph{0-4}

  

\makeatletter
\if@twoside
\begin{table}[t]% [!htb]
\else
\begin{table}[!t]% [!htb]
\fi
\makeatother
%\centering




\begin{minipage}{\textwidth} 
\caption[Cadre de vie des jeunes en 2004-2007]%
{Cadre de vie des jeunes en 2004-2007%
\footnote{Source : \emph{Moyenne annuelle des enquêtes emploi de 2004 à 2007}, INSEE.} }



\label{tableau-cadre-vie-2004-2007}

\begin{tabular}{*{6}{>{\hspace{0pt}\centering\arraybackslash}b{\lcol}}}
Âge des jeunes (années) & Vivant avec les deux parents de naissance & Avec un parent seul & Avec un parent et un beau-parent & Autres situations & Total\\
\hline
 0-6     & 82,2 & 10,1 & 7,2 & 0,5  & 100~\% \\
 7-13   & 72,8 & 16,6 & 9,9 & 0,7  & 100~\% \\
 14-17 & 66,9 & 19,0 & 9,8 & 4,4  & 100~\%
\end{tabular}

\end{minipage}

\end{table}

%CADRE DE VIE DES JEUNES EN 2004/2007[11]
% 
%\emph{Age des jeunes (années)}
%\emph{Vivant avec ses deux parents de naissance}
%\emph{Avec un parent seul[12]}
%\emph{Avec un parent et un beau-parent}
%\emph{Autres situations[13]}
%\emph{Total}
%\emph{0-6}
 
 
 
 
 L'évolution des comportements n'a rien de fulgurant. Vivre séparé de l'un de ses deux géniteurs reste une situation minoritaire : pour l'instant les trois quarts des mineurs vivent sous le même toit que leurs \emph{deux} parents \emph{de naissance} (dont les deux tiers des mineurs de 15 ans à 18 ans).
 
 Mais les évolutions actuelles sont aussi des évolutions symboliques : il n'y a peut-être (à vérifier) jamais eu autant d'enfants qu'aujourd'hui à vivre jusqu'à leur majorité avec leur père et leur mère de naissance et pourtant les familles ne sont plus pensées comme l'alliance irréversible de deux lignées, ni comme des institutions aux limites intangibles, mais comme des associations d'individus à géométrie variable. Les enfants d'aujourd'hui apprennent très tôt que les couples mixtes sont fragiles, qu'on rencontre aussi des couples mariés de même sexe, qu'amour ne rime pas avec toujours, que les princes et les princesses n'ont pas forcément beaucoup d'enfants, et qu'ils se séparent souvent avant la fin de leur histoire. Ils apprennent à dissocier parentalité et conjugalité, ou plutôt ils n'apprennent plus à les associer de manière indéfectible. À côté des scénarii traditionnels de leurs jeux d'imagination (le gendarme et le voleur, le client et la marchande, le malade et le docteur, l'indien et le cow-boy,~etc.) ils disposent maintenant du jeu du mariage et du divorce.

 Sous l'Ancien Régime c'était le contraire : en droit civil comme en droit canon, les mariages étaient indissolubles. Par contre, la mortalité d'alors, très élevée par rapport à celle d'aujourd'hui, faisait que plus de la moitié des époux étaient séparés par la mort avant même que leurs enfants n'aient atteint leurs vingt ans, et à cet âge il était normal d'être orphelin d'au moins un de ses deux parents. La durée moyenne effective des couples conjugaux était faible, environ quinze ans, comparée à celle d ceux des couples d'aujourd'hui qui n'ont pas divorcé, autour de cinquante ans. 

 La Révolution avait autorisé et facilité le divorce \emph{par consentement mutuel}, et à la suite de cette décision le taux de divorces observé dans les villes (mais \emph{seulement dans les villes}) avait rapidement atteint le niveau actuel. Mais contrairement à ce qui s'était passé dès l'an~III, aujourd'hui personne ne semble s'en inquiéter. Personne ne se donne plus pour objectif d'enrayer ce phénomène comme ce fut le cas avec le Code Napoléon, pendant la plus grande partie du \siecle{19} et sous le régime de Vichy (1940-45). Il ne s'agit plus de punir un coupable, ou deux, ni de chercher à prouver aux conjoints qu'ils peuvent respecter leurs engagements conjugaux au prix de quelques accommodements. Au contraire, les lois accompagnent ce mouvement de « \emph{démariage}
\footnote{Cf. Irène \fsc{THERY}, et son livre du même nom.}, et le divorce par consentement mutuel est devenu le modèle du bon divorce. 

 C'est en majeure partie du fait des divorces que les personnes seules avec enfants ont crû en nombre et en visibilité depuis 1970. En effet, le pourcentage de veufs et de veuves en leur sein a beaucoup baissé, au contraire de celui des divorcés : 9 fois sur 10 il s'agit de femmes seules avec enfants. 
 


 


 
\section{Problèmes de transmission}


 Les décideurs du passé n'avaient guère de difficultés à se mettre d'accord sur l'éducation des enfants et adolescents : jusqu'au milieu du \siecle{20} a régné dans le domaine éducatif un assez grand consensus autour de règles communes et de limites peu ou pas discutées. Cela se traduisait par exemple par le fait qu'au même moment les établissements éducatifs du \crmieme{19} ou du \siecle{20} présentaient partout à des nuances près les mêmes modes de fonctionnement, les mêmes limites, la même séparation des sexes, les mêmes styles de communication, qu'ils se réfèrent à un corps de doctrine religieux ou à une laïcité stricte. Les décideurs ne doutaient pas de leur droit à imposer leurs analyses aux parents qu'ils jugeaient négligents, délinquants, ou mal pensants. Ils valorisaient l'éducation au sein de la famille, mais au nom même de celle-ci ils plaçaient sans hésiter les enfants loin de leurs parents lorsque les conditions de leur éducation leur paraissaient compromises. C'est pourquoi l'époque actuelle est atypique en ce que depuis un bon demi-siècle elle hésite dans l'idée qu'elle se fait de l'intérêt de l'enfant, dans le choix de ce qu'elle veut lui transmettre, et dans les modalités de la transmission. 

 Il n'existe pas de savoir scientifique sur ce qu'est l'intérêt de l'enfant : il existe certes des connaissances scientifiques de plus en plus affinées sur les liens entre telle mesure éducative et tel résultat, telle performance, tel taux de morbidité,~etc. mais l'intérêt de l'enfant c'est bien autre chose. Il est lié aux fins que les hommes se donnent, qui ne sont pas scientifiques, mais politiques, philosophiques, religieuses, éthiques... 

 En l'absence de croyance partagée sur ce qu'est l'intérêt de l'enfant, l'accord minimal se fait sur l'idée qu'il faut avant tout ne pas lui nuire. Le reproche majeur qu'encourt un éducateur ce n'est plus de le « gâter » par sa complaisance et par son manque de fermeté. Le principal des risques actuels de son métier, c'est d'être accusé de le maltraiter par des exigences excessives. Parfois les enfants semblent assimilés à une population à libérer (par le droit) de l'arbitraire oppressif que les parents, les enseignants et les institutions d'assistance et de rééducation exerceraient sur eux%
% [1]
\footnote{Références : 
\\Gérard \fsc{MENDEL}, \emph{Pour décoloniser l'enfant, socio psychanalyse de l'autorité}, 1971.
\\Alain \fsc{RENAUT}, \emph{La libération des enfants, contribution philosophique à une histoire de l'enfance}, 2002.}% 
.

 Les « \emph{événements de mai 1968} » ont rendu visible et accéléré une remise en question des institutions, des hiérarchies et de l'argument d'autorité, mise en question qui avait commencé bien avant : avec les « maîtres du soupçon » ? Dès 1889 et la possibilité de déchoir les pères indignes ? Dès la Révolution française et l'exécution de Louis~XVI ? Avec les Lumières et Rousseau ? Avec la Renaissance et Rabelais,~etc. ? C'est aussi que bien des personnes et des institutions revêtues d'autorité avaient montré leurs propres limites, au nom de l'ordre et de l'obéissance, au cours des guerres et sous les régimes totalitaires dont le \siecle{20} a connu quelques beaux exemples, tandis que la psychanalyse soulignait la place centrale du désir du sujet dans son propre développement. D'autre part différentes recherches et expériences scientifiques avaient montré que les méthodes autoritaires d'éducation pouvaient être nocives, et que les méthodes non autoritaires de direction des groupes (Moren, Rogers) comme les pédagogies basées sur la découverte et l'initiative (Freinet, Montessori,~etc.) pouvaient être plus efficaces que les autres.

 Lorsqu'un anonyme inspiré a écrit sur un mur : « \emph{il est interdit d'interdire} », cette proposition jaillie d'on ne sait où a donc été reprise comme une évidence. Pour quelle raison ce paradoxe a-t-il à ce point fait vibrer la génération née au sortir de l'occupation ? ... sans doute parce qu'il était le corollaire d'un autre slogan aussi fameux de la même période : « \emph{jouissons sans entraves} ». 

 Les enfants se sont vus reconnaître le droit à la parole sur tout ce qui les concerne, notamment leurs orientations, mais aussi le droit à une vie sexuelle active dès l'âge de quinze ans (y compris le droit au secret médical, y compris pour les filles le droit si elles le désirent de mener à bien une grossesse ou de choisir une IVG,~etc.) tandis que leurs parents ont été sommés par la loi de les conduire démocratiquement vers une indépendance aussi précoce que possible (sauf dans le diomaine financier). 

 Depuis 1882, la durée de l'obligation scolaire s'est allongée et l'âge minimum de la mise au travail est passé de douze à seize ans afin de donner aux jeunes le maximum de chances d'insertion, de les protéger de toute exploitation au travail, et d'écrêter les différences entre milieux scolaires et sociaux différents. Depuis la création du Collège unique (\emph{Réforme \fsc{Haby}}, 1975), tous les enfants bénéficient de ce qui était un privilège jusqu'aux années soixante du \siecle{20}. Les études longues sont plus que jamais la voie royale vers la réussite personnelle. Grâce à leur quasi gratuité, tous ceux qui en ont les moyens intellectuels et le désir (et aussi des parents suffisamment aisés pour subvenir à leurs besoins matériels jusqu'à la fin de leurs études) ont des chances sérieuses de pouvoir en faire. Quant à savoir s'ils sont contents de l'extension de l'âge de 12 ans à 14 ans, puis à 16 ans, de leurs 5 ou 7 heures journalières de fréquentation des enseignants, il est assez évident que nombre d'entre eux, et notamment de garçons, ne la vivent pas bien et le font bruyamment savoir.

 Dans un contexte de concurrence scolaire généralisée, les richesses financières et culturelles des parents ne peuvent plus suppléer aussi massivement qu'autrefois à l'incompétence d'un jeune ou à son absence d'implication personnelle (même si elles jouent beaucoup). C'est pourquoi même s'ils sont toujours soucieux de l'avenir de leurs enfants, la pression qu'ils exercent a changé de lieu d'application : du contrôle rigoureux de leur sexualité pré conjugale, autrefois impératif pour leur futur établissement, et désormais sans importance, à l'exigence de performances scolaires aussi brillantes que possible, désormais sans alternative. C'est que rien n'a changé, bien au contraire, dans les règles du jeu qui permettent d'accéder aux meilleures sections des grands lycées et aux plus réputées des grandes écoles françaises et par là aux emplois les mieux payés, les plus attrayants ou les plus influents. Les jeunes n'ont donc plus guère à réprimer leurs désirs sexuels ni à supporter la culpabilité qui s'y attachait, devant un dieu ou devant leurs parents (sauf sans doute les jeunes aux tendances homosexuelles). Par contre il leur faut satisfaire à des normes exigeantes d'autonomie, de productivité intellectuelle et de compétitivité. Ceux qui n'y parviennent pas vivent une « honte » qui peut être au moins aussi insupportable que les anciennes culpabilités. Il n'y a sans doute pas moins de pression parentale aujourd'hui qu'autrefois, et il n'est peut-être pas plus agréable d'y être soumis, ni plus facile d'y satisfaire... 

 Sans parler de la responsabilité qui repose sur les épaules des enfants sur qui l'on compte, à défaut d'autre liens, pour donner sens à la vie de leurs parents :

\begin{displayquote}
\emph{Encore plus importante, naturellement, cette question : qu'est-ce qu'un enfant ? Le paradoxe est ici encore plus important car on n'a jamais autant prêté attention à l'enfant, on ne s'est jamais autant soucié de lui et on n'a jamais autant désenfantisé l'enfant.}
 
\emph{Désenfantiser l'enfant, comme s'il n'était possible de le concevoir comme notre égal qu'en le concevant comme notre semblable}[...]

\emph{L'enfant soutien de famille : ceci évoque un renversement tout à fait fondamental. \emph{[...]} la parentification des enfants dans les familles recomposées, c'est-à-dire un mouvement nouveau où, de façon tout à fait inattendue, la prise en compte de l'enfance aboutit à un déni d'enfance et où l'infantilisation du monde des adultes aboutit à une parentification du monde des enfants.}

 [... cette question encore paradoxale :] \emph{est-ce à l'enfant de dire qui appartient ou qui n'appartient pas à sa propre famille ?}%
% [2]
\footnote{Irène \fsc{THERY}, « Peut-on parler d'une crise de la famille ? un point de vue sociologique », \emph{Neuropsychiatrie de l'enfance et de l'adolescence}, 2001, 49, 492-501.} 
\end{displayquote}



 
 
 

 
 


\chapter{Victoire du mariage d'amour}


 Jusqu'au \siecle{19}, le premier objectif des jeunes gens raisonnables n'était pas tant de vivre mieux que leurs parents et de s'enrichir, que de réussir au moins à reproduire le même mode de vie qu'eux, et de ne pas tomber dans l'indigence. Pour cela, ils n'avaient le choix qu'entre un mariage arrangé par leurs parents, ou pas de mariage du tout, et si le plus souvent (pas toujours) il valait mieux à tous points de vue se marier que ne pas le faire, il leur fallait aussi éviter de compromettre, par enthousiasme naïf, par imprudence ou par sottise, les bases économiques de leur futur couple et le statut social de leurs enfants à venir. 

 Selon les moralistes d'alors le choix du mariage d'inclination, fondé sur l'amour passion et non sur la raison (c'est-à-dire l'intérêt) était la marque des imprévoyant(e)s. Entre mariage d'inclination et concubinage les liens paraissaient évidents. C'est ainsi que s'unissaient ceux qui ne possédaient que leurs bras, les ouvriers, les manœuvres, les valets, les ouvrières et les servantes, etc. Ceux qui se mettaient en ménage avant d'avoir « assis » leur « situation » se condamnaient à « tirer le diable par la queue ». Selon les mêmes moralistes, avec lesquels faisaient chorus tous les parents angoissés (et dans beaucoup de moralistes, même célibataires, il y a un parent angoissé), la soumission des jeunes imprévoyants à leurs appétits charnels leur faisait courir le risque de gâcher leur vie, de connaître la misère et de perdre un jour la main sur leurs propres enfants, ainsi qu'il en avait toujours été depuis le début du monde. 

 Ils risquaient en effet de ne pas pouvoir les élever et de devoir les abandonner aux institutions d'assistance. Ils ne pourraient les « établir », ni en leur donnant un capital matériel, ni en finançant leur apprentissage professionnel auprès d'un maître qualifié, ni en les mettant à l'école, même gratuite, puisqu'ils seraient contraints de les placer chez un maître dès que leur âge le permettrait. En cas de chômage et de disette, ils les enverraient mendier. Ils ne pourraient pas compter sur ces enfants, condamnés à être pauvres à leur tour, pour soutenir leur propre vieillesse. Ils risquaient de finir leurs jours dans la solitude et la misère, affective et matérielle, des hospices.

 Au contraire les parents prévoyants établissaient leurs enfants dans un mariage profitable grâce à leurs économies, à leurs relations et à des stratégies complexes : échanges simultanés et réciproques d'enfants, de terres, de droits d'exploitation, d'entreprises, de gérances, d'offices (ministériels), etc. sans compter jusqu'au \siecle{18} l'entrée en religion plus ou moins volontaire de ceux qu'ils ne pouvaient ou ne voulaient pas marier de manière conforme à leur milieu social. 

 Ces stratégies complexes ne pouvaient pas toujours tenir compte des préférences sexuelles ou amoureuses de chacun, et on n'en faisait pas grief aux parents. Les femmes s'en consoleraient avec leurs enfants ou la religion, les hommes avec le travail, le pouvoir, les prostituées ou les maîtresses (le recours à celles-là et aux "maisons closes" étant toujours préférable, du point de vue des épouses, au choix de celles-ci). Les patrimoines étaient verrouillés contre les effets des infidélités des uns et des autres. Une épouse ne pouvait introduire d'enfant adultérin dans sa famille que si son mari le voulait bien, mais en ce cas la paternité de celui-ci devenait absolument inattaquable : le géniteur n'avait aucun recours. Quant aux enfants illégitimes du mari, ils ne pouvaient en aucun cas être légitimés ni menacer l'héritage des enfants légitimes. Les épouses pouvaient dormir tranquilles (sur ce point en tout cas) même quand leurs maris découchaient. 

 La pérennité des couples raisonnables était favorisée par la synergie des ressources que leurs familles respectives avaient sagement et laborieusement conjointes. Leurs parents étaient les premiers à tenir fermement à ce qu'ils, et elles plus encore, ne mettent pas ces arrangements en danger par des comportements imprudents ou des passions irréfléchies, d'où leur accord profond avec les autorités morales et religieuses de l'époque. L'intérêt matériel des époux était le plus souvent de rester ensemble, quitte à accepter des renoncements ou des compromis sur les vrais désirs de chacun, et à promouvoir comme un des fondements du savoir-vivre une dose convenable d'hypocrisie : d'ailleurs, dès l'antiquité païenne, il était très inconvenant d'afficher publiquement une affection trop vive entre conjoints. 

 Certes, l'impossibilité de placer les préférences individuelles avant tout autre critère pouvait faire souffrir, et l'amour passion comme la liberté de choix du conjoint faisaient rêver. Les œuvres littéraires du passé reflètent la prégnance de ces représentations. Ainsi, pour ne prendre qu'un seul exemple, la plupart des intrigues de Molière reposent sur le refus d'un mariage arrangé. les romans de Jane Austen surnagent comme des modèles parmi des milliers d'autres fondés sur les "problèmes de coeur" de jeunes gens et surtout de jeunes filles, apparemment libres de leurs choix, plus libres en Angleterre qu'en France au moins au premier regard, et en réalité excessivement contraints. Les gens raisonnables savaient que le choix du conjoint n'était habituellement pas libre. Ce n'était que du rêve. Les contraintes économiques étaient indépassables, en dépit des souffrances et des renoncements qu'elles entraînaient. Cela n'empêchait pas la société de continuer siècle après siècle à fonctionner sur le même mode. 

 Aujourd'hui, à la condition de posséder une vraie qualification professionnelle (capital intellectuel), il n'est plus besoin de capitaux pour s'établir. Rien ne vaut un « bon » métier : un métier qui implique beaucoup de savoirs et de savoir-faire, dans un secteur d'activité porteur. Il n'est plus honteux de « servir ». Au fil du \siecle{20}, le salariat s'est souvent révélé plus sûr que la possession de capitaux ou d'outils de production, surtout au service de l'État. D'autre part la scolarité est désormais gratuite ou presque jusqu'aux niveaux les plus élevés (même si ce n'est souvent pas vrai aux niveaux les plus élevés). Les parents ont intégré cette logique : depuis la Libération, le taux de scolarisation n'a cessé de s'accroître bien au-delà de la fin de l'obligation scolaire, qui elle-même est passée de 14 (1936) à 16 ans (1959). Le nombre des diplômés de l'enseignement supérieur a explosé. Le nombre de bacheliers se situe actuellement entre 60~\% et 80~\% d'une classe d'âge, contre 5~\% à la Libération et 8~\% en 1960. Même si ce diplôme s'est largement dévalué et ne peut plus depuis longtemps procurer un emploi à lui seul, le niveau de culture moyen a indiscutablement progressé. 

 Au \siecle{19}, un homme dépensait plus s'il était célibataire que s'il était marié, sauf à employer une « bonne à tout faire ». Il était plus rentable d'entretenir une « ménagère » à domicile que de manger tous les jours au restaurant, de faire blanchir son linge,~etc. En dehors de sa dot (très mince ou inexistante dans les milieux populaires), une épouse fournissait gratuitement une somme de prestations qu'il eût été coûteux de se procurer sur le marché. Aujourd'hui la rentabilité du travail domestique a été fortement réduite par les innovations techniques, commerciales et sociales du \siecle{20} : infrastructures collectives (électricité, tout à l'égout, eau courante) ; machines qui économisent le temps de travail (chauffage central, machine à laver le linge, la vaisselle, cuisines équipées électriques ou au gaz, aspirateur,~etc.) ; grande distribution qui rend non-compétitive l'auto production en couture, en jardinage vivrier, en préparation des aliments,~etc. Sans oublier les écoles maternelles et les garderies d'enfants. 

 Si les ménagères ont été libérées d'une grande partie du poids des tâches domestiques, elles ont en même temps été réduites au chômage technique : c'est au même moment qu'on observe la fin des « bonnes ». Pour contribuer significativement aux ressources de leur ménage elles doivent désormais travailler au dehors de leur foyer. Cela leur a ouvert la possibilité de -- mais on peut tout aussi bien dire que cela les a contraintes à -- se trouver d'autres emplois que d'être la « maîtresse de maison » titulaire d'un homme. 

 Elles y ont d'autant plus été poussées, que depuis la libéralisation du divorce elles ne peuvent plus être assurées, comme l'étaient leurs grand-mères, de leur position d'épouse en titre d'un homme nommément désigné. Les filles de la bourgeoisie ont compris que leur avenir serait mieux assuré par un « bon » métier que par une « belle » dot, un « beau » parti, et par le « grand » mariage qui était jusqu'alors le point de focalisation de tous les désirs familiaux, le signe et le sommet de la réussite féminine. Toutes ont compris que grâce aux savoirs et aux diplômes elles seraient libres : indépendantes des désirs d'un homme et de sa bonne volonté, à l'abri des effets matériels des répudiations, en mesure de prendre l'initiative et de sortir des situations affectives ou familiales dans lesquelles elles ne trouveraient pas leur compte. Dans la course au diplôme les filles se sont (donc ?) montrées significativement plus déterminées que les garçons (quant aux décrochages scolaires de ces derniers, ils ont bien entendu d'autres causes). 

 S'il n'est plus nécessaire pour « s'établir » d'avoir l'appui financier de ses parents ni de leurs relations, alors le mariage ne scelle plus l'alliance (économique et éventuellement politique) de deux familles : alors rien n'exige plus que les jeunes gens subissent un mariage arrangé, un mariage d'argent et d'entregent. Ils peuvent sans risque \emph{matériel} excessif s'offrir le luxe de n'être pas raisonnables et de baser leur couple sur la seule passion amoureuse.

 C'est en tout cas tellement devenu notre logique que cela révolutionne notre compréhension du mariage lui-même. Si celui-ci se définit d'abord comme l'union de deux personnes qui s'aiment, alors la question de la durée perd de son sens. L'authenticité des désirs inscrits dans les actes posés ici et maintenant a plus d'importance que la fidélité à une promesse ancienne. L'infidélité conjugale n'est plus une offense à un ordre public qui ne se donne plus pour but de sanctuariser les familles. Ce n'est plus qu'une offense privée, le signe d'un désaccord entre deux associés. La séduction devient une obligation permanente. L'accord du conjoint à une relation charnelle ne peut plus être tenu pour acquis d'avance, par contrat. La notion de \emph{devoir conjugal} s'est vidée de son sens, et la loi ne le reconnaît plus. La notion de viol entre époux prend du sens, et comme tout viol c'est un délit punissable par la loi. 

 Si c'est l'amour mutuel qui fonde le couple, alors sa fécondité potentielle perd de son importance. Que le couple soit constitué d'un homme et d'une femme ne va plus sans le dire. La reconnaissance publique d'un couple de deux hommes ou de deux femmes n'est plus impossible à penser. 

 Mais si c'est l'enfant qui fait la famille, à quoi bon se marier ?


% 28.02.2015 :
% ~etc.
% Moyen Âge
% _, --> ,
% Antiquité




 









\chapter{Perplexités éducatives}


 Les décideurs du passé n'avaient guère de difficultés à se mettre d'accord sur l'éducation des enfants et adolescents : jusqu'au milieu du \siecle{20} a régné dans le domaine éducatif un assez grand consensus autour de règles communes et de limites peu ou pas discutées. Cela se traduisait par exemple par le fait qu'au même moment les établissements éducatifs du \crmieme{19} ou du \siecle{20} présentaient partout à des nuances près les mêmes modes de fonctionnement, les mêmes limites, la même séparation des sexes, les mêmes styles de communication, qu'ils se réfèrent à un corps de doctrine religieux ou à une laïcité stricte. Les décideurs ne doutaient pas de leur droit à imposer leurs analyses aux parents qu'ils jugeaient négligents, délinquants, ou mal pensants. Ils valorisaient l'éducation au sein de la famille, mais au nom même de celle-ci ils plaçaient sans hésiter les enfants loin de leurs parents lorsque les conditions de leur éducation leur paraissaient compromises. C'est pourquoi l'époque actuelle est atypique en ce que depuis un bon demi-siècle elle hésite dans l'idée qu'elle se fait de l'intérêt de l'enfant, dans le choix de ce qu'elle veut lui transmettre, et dans les modalités de la transmission. 

 Il n'existe pas de savoir scientifique sur ce qu'est l'intérêt de l'enfant : il existe certes des connaissances scientifiques de plus en plus affinées sur les liens entre telle mesure éducative et tel résultat, telle performance, tel taux de morbidité,~etc. mais l'intérêt de l'enfant dépend de bien autre chose. Il est lié aux fins que les hommes se donnent, qui ne sont pas scientifiques, mais politiques, philosophiques, religieuses, éthiques... 

 En l'absence de croyance partagée sur ce qu'est l'intérêt de l'enfant, l'accord minimal se fait sur l'idée qu'il faut avant tout ne pas lui nuire. Le reproche majeur qu'encourt un éducateur ce n'est plus de le « gâter » par sa complaisance et par son manque de fermeté. Le principal des risques actuels de son métier, c'est d'être accusé de le maltraiter par des exigences excessives. Parfois les enfants semblent assimilés à une population à libérer (par le droit) de l'arbitraire oppressif que les parents, les enseignants et les institutions d'assistance et de rééducation exerceraient sur eux%
% [1]
\footnote{Références : 
\\Gérard \fsc{MENDEL}, \emph{Pour décoloniser l'enfant, socio psychanalyse de l'autorité}, 1971.
\\Alain \fsc{RENAUT}, \emph{La libération des enfants, contribution philosophique à une histoire de l'enfance}, 2002.}% 
.

 Les « \emph{événements de mai 1968} » ont rendu visible et accéléré une remise en question des institutions, des hiérarchies et de l'argument d'autorité, mise en question qui avait commencé bien avant : avec les « maîtres du soupçon » ? Dès 1889 et la possibilité de déchoir les pères indignes ? Dès la Révolution française et l'exécution de Louis~XVI ? Avec les Lumières et Rousseau ? Avec la Renaissance et Rabelais,~etc. ? C'est aussi que bien des personnes et des institutions revêtues d'autorité avaient montré leurs propres limites, au nom de l'ordre et de l'obéissance, au cours des guerres et sous les régimes totalitaires dont le \siecle{20} a connu quelques beaux exemples, tandis que la psychanalyse soulignait la place centrale du désir du sujet dans son propre développement. D'autre part différentes recherches et expériences scientifiques avaient montré que les méthodes autoritaires d'éducation pouvaient être nocives, et que les méthodes non autoritaires de direction des groupes (Moren, Rogers) comme les pédagogies basées sur la découverte et l'initiative (Freinet, Montessori,~etc.) pouvaient être plus efficaces que les autres.

 Lorsqu'un anonyme inspiré a écrit sur un mur : « \emph{il est interdit d'interdire} », cette proposition jaillie d'on ne sait où a donc été reprise comme une évidence. Pour quelle raison ce paradoxe a-t-il à ce point fait vibrer la génération née au sortir de l'occupation ? ... sans doute parce qu'il était le corollaire d'un autre slogan aussi fameux de la même période : « \emph{jouissons sans entraves} ». 

 Les enfants se sont vus reconnaître le droit à la parole sur tout ce qui les concerne, notamment leurs orientations, mais aussi le droit à une vie sexuelle active dès l'âge de quinze ans (y compris récemment le droit au secret médical, et y compris pour les filles le droit de mener à bien une grossesse ou de choisir une IVG,~etc.) tandis que leurs parents ont été sommés par la loi de les conduire démocratiquement vers une indépendance aussi précoce que possible. 

 Depuis 1882, la durée de l'obligation scolaire s'est allongée et l'âge minimum de la mise au travail est passé de douze à seize ans afin de donner aux jeunes le maximum de chances d'insertion, de les protéger de toute exploitation au travail, et d'écrêter les différences entre milieux scolaires et sociaux différents. Depuis la création du Collège unique (\emph{Réforme \fsc{Haby}}, 1975), tous les enfants bénéficient de ce qui était un privilège jusqu'aux années soixante du \siecle{20}. Les études longues sont plus que jamais la voie royale vers la réussite personnelle. Grâce à leur quasi gratuité, tous ceux qui en ont les moyens intellectuels et le désir (et aussi des parents suffisamment aisés pour subvenir à leurs besoins matériels jusqu'à la fin de leurs études) ont des chances sérieuses de pouvoir en faire. Quant à savoir s'ils sont contents de l'extension de l'âge de 12 ans à 14 ans, puis à 16 ans, de leurs 5 ou 7 heures journalières de fréquentation des enseignants, il est assez évident que nombre d'entre eux, et notamment de garçons, ne la vivent pas bien et le font bruyamment savoir.

 Dans un contexte de concurrence scolaire généralisée, les richesses financières et culturelles des parents ne peuvent plus suppléer aussi massivement qu'autrefois à l'incompétence d'un jeune ou à son absence d'implication personnelle. C'est pourquoi même s'ils sont toujours soucieux de l'avenir de leurs enfants, la pression qu'ils exercent a changé de lieu d'application : du contrôle rigoureux de leur sexualité pré conjugale, autrefois impératif pour leur futur établissement, et désormais sans importance, à l'exigence de performances scolaires aussi brillantes que possible, désormais sans alternative. C'est que rien n'a changé, bien au contraire, dans les règles du jeu qui permettent d'accéder aux meilleures sections des grands lycées et aux plus réputées des grandes écoles françaises et par là aux emplois les mieux payés, les plus attrayants ou les plus influents. Les jeunes n'ont donc plus guère à réprimer leurs désirs sexuels ni à supporter la culpabilité qui s'y attachait, devant un dieu ou devant leurs parents. Par contre il leur faut satisfaire à des normes exigeantes d'autonomie, de productivité et de compétitivité. Ceux qui n'y parviennent pas vivent une « honte » qui peut être au moins aussi insupportable que les anciennes culpabilités. Il n'y a sans doute pas moins de pression parentale aujourd'hui qu'autrefois, et il n'est peut-être pas plus agréable d'y être soumis, ni plus facile d'y satisfaire... 

 Sans parler de la responsabilité qui repose sur les épaules des enfants sur qui l'on compte, à défaut d'autre liens, pour donner sens à la vie de leurs parents :

\begin{displayquote}
\emph{Encore plus importante, naturellement, cette question : qu'est-ce qu'un enfant ? Le paradoxe est ici encore plus important car on n'a jamais autant prêté attention à l'enfant, on ne s'est jamais autant soucié de lui et on n'a jamais autant désenfantisé l'enfant.}
 
\emph{Désenfantiser l'enfant, comme s'il n'était possible de le concevoir comme notre égal qu'en le concevant comme notre semblable}[...]

\emph{L'enfant soutien de famille : ceci évoque un renversement tout à fait fondamental. \emph{[...]} la parentification des enfants dans les familles recomposées, c'est-à-dire un mouvement nouveau où, de façon tout à fait inattendue, la prise en compte de l'enfance aboutit à un déni d'enfance et où l'infantilisation du monde des adultes aboutit à une parentification du monde des enfants.}

 [... cette question encore paradoxale :] \emph{est-ce à l'enfant de dire qui appartient ou qui n'appartient pas à sa propre famille ?}%
% [2]
\footnote{Irène \fsc{THERY}, « Peut-on parler d'une crise de la famille ? un point de vue sociologique », \emph{Neuropsychiatrie de l'enfance et de l'adolescence}, 2001, 49, 492-501.} 
\end{displayquote}




\chapter{Désarrois masculins}


 Notre retour sur l'histoire montre à quel point la situation actuelle est révolutionnaire. Le mariage avait pour but, essentiel sinon unique, de fabriquer des pères et de leur donner des enfants. L'ancien Droit romain semblait réduire les femmes à n'être que des ventres au service des hommes. Ceux-ci répudiaient celles qui ne leur avaient pas donné les héritiers qu'ils voulaient, ou prenaient des concubines. En cas de séparation, le Droit leur attribuait systématiquement tous leurs enfants%
% [1]
\footnote{... ce qu'il a fait jusqu'au milieu du \siecle{19} dans les pays anglo-saxons.}% 
. Ils pouvaient assumer eux-mêmes leur éducation, ou les confier à leur propres parents. Ils pouvaient même les confier à leurs ex épouses, mais toujours sous leur propre autorité et à leurs frais.

 Si jusqu'à ces dernières décennies le mariage alliait deux lignées en associant un homme et une femme dans le cadre d'une division sexuelle du travail indiscutée, la première de ses fonctions, ressentie comme incontournable et justifiée par la survie des individus aussi bien que celle de l'espèce, était de donner des enfants aux hommes%
% [6]
\footnote{{\emph{La fonction principale du mariage était d'ailleurs de fabriquer du père} [...]}, Irène \fsc{THERY}, idem.}% 
. Ils ne pouvaient en effet donner le jour à des semblables, mais ils n'en avaient pas moins un impérieux besoin pour s'occuper d'eux jusqu'à leur mort même quand ils ne pourraient plus subvenir à leurs propres besoins et pour leur succéder. Cela impliquait de donner de la valeur au fait que les enfants aient un père et non un géniteur anonyme. 
 
 Tout était (donc ?) fait pour décourager les femmes de concevoir des enfants sans en passer par un homme publiquement désigné (jusqu'à l'infériorité des salaires féminins à travail égal ?). « À cause des enfants », (grâce aux enfants ?) dont l'avenir, le statut et l'installation dans l'existence dépendaient plus d'eux que d'elles, les hommes tenaient les femmes en leur « main ». Le mariage permettait à presque tous ceux qui le désiraient (c'est-à-dire la plupart des hommes) d'avoir des enfants bien à eux et qui ne leur seraient contestés par personne et d'abord par leur mère. Il leur permettait aussi de s'attacher une femme et les services de tous ordres que seule une femme pouvait alors fournir. 
 
 Mais la réciproque était vraie aussi : le mariage permettait aux femmes d'avoir des enfants sans être obligées de les élever seules, dans la pauvreté et l'illégitimité. Quant à celles qui y attachaient du prix, il leur permettait de s'attacher solidement un homme%
% [7]
\footnote{... ce que symbolisaient depuis l'antiquité les anneaux que s'échangeaient les conjoints, et ce qu'exprimait sur le mode burlesque des expressions comme {\emph{se laisser mettre le grappin dessus}}, ou {\emph{se passer la corde au cou}}. Il est symptomatique que c'étaient les hommes qui employaient ces expressions : dans les représentations d'alors ce sont les femmes qui cherchaient le plus activement et anxieusement à se marier, et le jour de leur mariage était en quelque sorte celui de leur triomphe.}
 
 Un homme qui désire des enfants ne peut plus s'y prendre aujourd'hui comme naguère. Il ne lui sert plus à rien de demander à un futur beau-père la main de sa fille, de lui demander un transfert d'autorité, puisque ce dernier ne la détient plus et ne peut donc plus la donner. D'ailleurs lui-même n'a plus besoin d'un gendre pour légitimer les petits enfants que sa fille lui donnera et pour en faire des héritiers, puisqu'il n'y a plus de fonctions interdites aux enfants illégitimes et donc plus d'enfants illégitimes. Il n'y a donc plus d'intérêt commun entre beau-père et gendre, et le soupirant doit négocier seul et sans intermédiaire avec la femme dont il recherche les faveurs. Il n'aura d'elle des enfants que si elle le veut bien. Et elle pourra d'autant plus facilement le quitter en emmenant leurs enfants communs (ou le pousser hors du domicile familial) que l'absence d'un homme à côté d'une mère ne fait plus problème, tandis que la présence de celle-ci semble encore presque indispensable (mais cela changera peut-être si on constate des compétences maternantes au sein des couples masculins ?). 
 
 En ce qui concerne les hommes, les ressources dont ils disposent (puissance économique, puissance militaire, compétences professionnelles,~etc.) ont la vertu de les rendre désirables autant que leurs qualités physiques et psychologiques, et on l'a toujours su. Plus ils sont intellectuellement et professionnellement qualifiés, plus ils ont de chances d'être mariés. C'est le contraire pour les femmes. C'est peut-être une preuve que celles-ci n'ont pas intérêt au mariage dès qu'elles ont les moyens de leur indépendance ? au contraire des hommes ? Le fait que dans certains pays d'Europe un nombre significatif de femmes aux ressources au dessus de la moyenne semble choisir aujourd'hui de ne pas avoir d'enfants montre que pour elles en tout cas la famille et le mariage n'ont pas d'attraits. 
 
 Pour le moment, ces évolutions n'infirment pas la répartition traditionnelle des rôles érotiques masculins et féminins : les hommes se doivent encore d'être « ceux qui peuvent », ceux qui ne sont pas marqués par le manque ou la défaillance (pouvoir politique, financier, intellectuel, militaire, puissance sexuelle...) à défaut de quoi ils n'exercent guère d'attrait sur la plupart des femmes, tandis que lorsque celles-ci ne s'éprouvent pas, au moins un peu, comme « celles qui n'ont pas » (pas tout), comme celles qui « ne peuvent pas » (pas toutes seules), elles n'ont pas besoin des hommes (mais peut-être n'exercent-elles pas non plus d'attrait sur eux ?).
 
 On a vu que « l'obligation de résultat », l'obligation de fécondité, qui pesait sur les seules femmes mariées a été supprimée par Constantin, qui a exclu la stérilité des motifs de divorce. La loi impériale romaine a ensuite confirmé l'interdit fait aux chrétiens de se remarier après divorce. À l'obligation de fécondité des épouses s'est substituée une obligation (religieuse) de moyens pour chacun des deux époux de ne pas mettre d'obstacle autre que l'abstinence%
% [2] 
\footnote{En France les relevés démographiques montrent l'érosion progressive du respect de cette obligation, et l'extension depuis trois siècles des pratiques anticonceptionnelles : ce que les anciens moralistes nommaient les « \emph{funestes secrets} ».} 
à une conception et à une naissance. Si les femmes mariées ont ainsi été protégées contre la répudiation et contre la privation de leurs enfants, par contre la loi ne les autorisait pas plus qu'avant à se dérober au « devoir conjugal » lorsque leur mari l'exigeait, ni aux grossesses qui en découleraient, et à leurs risques, sauf à demander une séparation. 

 Depuis 1967, même si leurs maris le désirent, les femmes mariées ne sont plus tenues par la loi de laisser libre cours à leur fécondité. Aujourd'hui le corps des femmes est à elles, y compris l'embryon ou le fœtus, qui juridiquement en fait partie depuis 1975, comme c'était le cas dans le droit romain antique. 

 La loi ne se soucie plus de soutenir le désir masculin en ce domaine. Même si elles sont leurs épouses, même s'ils sont les géniteurs de l'enfant qu'elles portent, même si elles avaient été d'accord pour le concevoir avec eux, les hommes n'ont plus le droit d'exiger des femmes qu'elles leur donnent cet enfant. Elles peuvent choisir d'avorter ou de l'abandonner à la naissance contre le gré du père de l'enfant. On est au plus loin du droit du \emph{pater familias} romain de faire surveiller la grossesse et l'accouchement de son épouse (ou ex épouse), pour qu'elle ne puisse pas lui dérober un enfant né de ses œuvres.

 Dans le même temps ont été supprimées toutes les limites légales qui pouvaient interdire le rattachement d'un enfant naturel à son géniteur, à l'exception des inséminations artificielles avec donneur, ou IAD. Une mère qui le demande recevra toujours l'appui de la justice pour rechercher le géniteur de son enfant, quelle que soit la situation personnelle de cet homme, comme sous l'ancien régime, mais désormais cela se fera avec une efficacité  imparable. Aucun père n'est plus « \emph{incertus} ». Vivant ou mort son ADN le désignera, sauf lorsque la mère veut cacher son identité à l'enfant ou aux tiers (mais si une mère qui accouche « sous X » refuse de laisser à son enfant des renseignements sur sa propre identité, elle en a le droit). Si la mère le veut, le géniteur sera contraint d'assumer financièrement un enfant qui héritera de lui à part entière, contrairement à ce qui se passait jusqu'au \siecle{19}. Mais cela ne lui donnera pas forcément le moindre droit sur l'éducation de l'enfant : en ce sens cela n'en fera pas un père.

 Pour l'essentiel, on peut donc dire que la maîtrise de la génération est passée du côté des femmes. La famille monoparentale d'aujourd'hui, c'est assez ordinairement la famille \emph{moins} le père. Dans la majorité des séparations (85~\%) ce sont les mères qui gardent les enfants. Est-ce pour ces raisons que l'initiative des divorces vient beaucoup plus souvent des épouses que des maris ? Beaucoup d'hommes ont plus à perdre que leurs femmes au divorce, et surtout les plus pauvres. 

 Les mères ont toujours eu une place de choix dans les représentations : elles sont traditionnellement du côté de l'accueil de la vie et de son entretien, de l'intime, de la tendresse, du cœur. Mais aujourd'hui cette idéalisation n'est plus contrebalancée par l'idéalisation symétrique des pères des siècles classiques. Aujourd'hui la déploration des déficiences des pères, de leurs fragilités et de leur irresponsabilité, est un passage obligé de tout discours sur la famille, tandis que l'idée qu'ils puissent exercer une force ou une puissance dans leur relation à des enfants renvoie automatiquement à des représentations de violence et de maltraitance. Quand on parle sans les spécifier des violences conjugales ou intra familiales, il va de soi qu'il s'agit des violences masculines, alors que l'observation montre que les femmes sont très capables de concurrencer les hommes dans ce domaine aussi. 

 D'ailleurs maintenant que le capital le plus utile c'est le capital intellectuel, maintenant que l'avenir des enfants se prépare à coup d'études longues, financées en grande partie par la collectivité, sous la houlette de professionnels de l'enseignement et sous le contrôle de l'État, qu'est-ce qu'un père pourrait bien transmettre à ses enfants (à part ses biens) sans menacer leur autonomie ?

 Dans l'effritement de l'autorité des pères, Françoise \fsc{HURSTEL} pointe trois moments clé : la loi de 1889 contre les « \emph{parents indignes} », la loi de 1935 abolissant le droit de « \emph{correction paternelle} » et la loi de 1938 abolissant la « \emph{puissance maritale} ». Ont été abolies toutes les dispositions juridiques sur lesquelles était fondé dans le passé l'exercice masculin d'un rôle patriarcal. Le résultat est que « [...] \emph{nous ne savons plus ce qu'est la place d'un père et ce que sont ses fonctions} », et que « \emph{ce ne sont pas des petits bouts de la paternité qui ont changé, mais l'ensemble du système a muté avec la mort du \emph{pater familias}.} »%
% [3]
\footnote{Françoise \fsc{HURSTEL}, « Penser la paternité contemporaine dans le monde occidental : quelles places et quelles fonctions du père pour le devenir humain, sujet et citoyen des enfants ? », in \emph{Neuropsychiatrie de l'enfance et de l'adolescence}, 53 (2005) 224-230.} 

 Autrefois (jusqu'aux années 60 du siècle dernier ?) c'est l'excès de présence et de poids des pères qui faisait problème. Aujourd'hui on déplore qu'ils ne soient jamais assez présents, ou jamais là où il faut. Françoise \fsc{HURSTEL} soutient que cela est l'effet de ces changements, et non leur cause. Si les lois suivaient l'évolution des mœurs, alors la promulgation d'une loi serait le signe que les esprits sont prêts à l'accueillir. Dans cette hypothèse, pendant les années précédant la promulgation de chacune des lois ci-dessus, on aurait dû observer un mouvement de l'opinion publique stigmatisant les parents indignes, le recours abusif au droit de correction paternelle, ou le scandale que constitue l'existence d'une puissance maritale. Selon elle ce n'est pas ainsi que cela s'est passé, au contraire. Ce n'est qu'à partir de la promulgation de la loi de 1889 que la presse aurait commencé de dénoncer les carences des pères « indignes%
% [4]
\footnote{« \emph{alcoolique, pauvre, inculte et violent} », Françoise \fsc{HURSTEL}, \emph{la déchirure paternelle}, p. 113.} 
 ».

 Et de même ce n'est que vers 1942 que les spécialistes de l'éducation auraient commencé de dénoncer les pères sans autorité, tandis que la notion de carence n'aurait envahi les écrits qu'à partir de 1950 :
 
\begin{displayquote}
\emph{C'est donc quelques années après la promulgation de ces lois faisant disparaître des textes juridiques les termes de puissance (maritale) et ceux de correction paternelle tout en maintenant ceux de chef et d'autorité (paternelle), qu'est décrite cette figure d'un père manquant d'autorité et de sévérité ; et que les spécialistes admonestent les pères d'une position qui est bien celle de chef de famille.}
\end{displayquote}

 Selon elle, l'opinion publique n'aurait donc appelé aucune de ces lois de ses vœux. Ces réformes n'auraient été imaginées, réclamées, et parfois discrètement expérimentées que par les seuls experts, médecins, administrateurs, juges et travailleurs sociaux directement intéressés à leur mise en œuvre. Pour Françoise \fsc{HURSTEL}, tous les discours sur les déficiences des pères actuels ne sont que des productions imaginaires qui coexistent avec des réalités qui n'ont pas grand-chose à voir avec eux. En effet, les enquêtes sur le terrain ne montrent rien qui permette de croire que les pères d'aujourd'hui seraient dans l'ensemble moins attentifs et moins présents que ne l'étaient ceux du passé%
% [5]
\footnote{... mais cela exige d'éviter les biais méthodologiques. Il faut notamment que ces enquêtes ne se placent pas consciemment ou inconsciemment du seul point de vue des mères. Cf. Germain \fsc{DULAC}, « La configuration du champ de la paternité : politiques, acteurs et enjeux », in \emph{Politiques du père, numéro spécial de Lien social et politiques}, (n° 37) 1997, p. 133-142.}%
. Certes il y a des pères qui sont incompétents, irresponsables ou délinquants, mais cela n'a rien de nouveau, et rien ne permet d'affirmer qu'il y en ait plus qu'autrefois. Les discours ne portent pas tant sur ce que font réellement les pères que sur ce qu'ils devraient faire dans l'idéal pour être de bons pères. 

 Pour elle, il s'agit, à l'aide de ces discours, d'asseoir l'autorité de ceux qui prétendent savoir ce qu'est un bon père et qui sont les bons pères :
 
\begin{displayquote} 
\emph{Du point de vue de la paternité les hommes de la période contemporaine n'auront pas été gâtés. Je propose une image pour illustrer ce que peut être la notion de carence : lorsqu'un homme devient père, il endosse un pardessus plein de trous et de soupçons..., plus précisément une image de plus en plus dévalorisée, et cela quelle que soit la valeur personnelle de l'homme qui assume une telle fonction. Et ce qui les caractérise est un discours dévalorisant des spécialistes ; tellement dévalorisant qu'il apparaît, en fait, comme un discours de l'exclusion des pères... au profit du super père spécialiste. Si les pères peuvent être dits carents \emph{[en Droit, le père « carent » est celui qui ne laisse rien à ses enfants, qui ne leur laisse aucun héritage]}, c'est parce qu'ils sont relégués à cette place par ceux-là mêmes qui normalisent les pratiques autour de l'enfant. Nous dirons que ces pères carents sont en fait d'abord des pères exclus par les théoriciens de l'éducation.}%
% [6]
\footnote{Françoise \fsc{HURSTEL}, \emph{la déchirure paternelle}, p. 112-113.} 

[...] \emph{Ainsi les signifiants inscrits dans la loi produisent des effets imaginaires qui se repèrent dans les représentations collectives, les modèles normatifs du père et les pratiques sociales.} 

\emph{Je ferai ici un pas de plus et avancerai ceci : non seulement les signifiants des lois produisent des effets imaginaires, mais encore les lois elles-mêmes ne sont connues que par le biais de ces productions...}

\emph{Les figures du père carent semblent bien avoir une fonction sociale et idéologique importante, celle d'être l'une de ces fonctions sociales qui rendent compte et qu'il y a du père dans notre société (au sens du père symbolique et de la fonction paternelle) et qu'il y a du changement dans les montages qui instituent le père... bref, elles seraient un mode d'historicisation d'une structure.}

\emph{Mais en retour cet imaginaire du père marquera chaque homme ayant à assumer la fonction paternelle, chaque mère appelée à reconnaître qu'il y a du père pour son enfant.}%
% [7]
\footnote{Idem, p. 113-115.}
\end{displayquote}

 Il est ordinaire et au fond assez normal que les adolescents, garçons et filles, soient en état d'incertitude identitaire, avec tous les malaises que cela implique, mais ils supportent encore moins bien les incertitudes identitaires de leurs adultes de référence que les leurs propres. Ce n'est pas un hasard si ce sont les garçons qui expriment aujourd'hui le plus durement leur désarroi : violences contre les personnes et les biens, prises de risques inconsidérées, désinvestissement scolaire, etc. Ils ont besoin que les adultes (hommes et femmes) reconnaissent que c'est une puissance valeureuse qui croît en eux et non une violence erratique, brutale et destructrice, juste bonne à être périodiquement sacrifiée en holocauste aux dieux de la guerre.

 Puisque le patriarcat est mort et que les femmes ne retourneront plus dans des gynécées, sinon contraintes et forcées, et puisque dans le domaine familial aussi le droit à l'égalité s'impose comme le principe de base indiscutable, il faudra inventer (ou découvrir, ou redécouvrir) pour les hommes une place qui soit aussi désirable que celle des femmes : des points de vue et des désirs spécifiquement masculins sur les enfants sont-ils acceptables ? Mais dans un environnement allergique à tout ce qui ressemble à du paternalisme, qu'est-ce qu'un homme est autorisé à désirer concernant des enfants ? Les hommes sont-ils fondés à dire quelque chose sur les enfants ? Sont-ils fondés à dire quelque chose aux enfants ? Il faudra sans doute commencer par admettre qu'il existe des valeurs masculines, ou une manière masculine de faire vivre les valeurs universelles.

 Derrière le problème de la paternité se profile la question « à qui appartient l'enfant ? ». Il ne s'appartient pas à lui-même, sauf à supprimer le statut de mineur. On ne peut pas plus dire qu'il n'appartient à personne. Du point de vue des enfants, n'appartenir à personne (ou appartenir à une institution) c'est être abandonné. Depuis très longtemps les enfants n'appartiennent plus aux seuls pères. Est-ce qu'ils appartiennent désormais aux seules mères ? ... ou bien aux deux parents, comme le dit la loi ? ... ou bien à l'ensemble de ceux qui les élèvent en leur donnant leur argent et leur temps, dont les beaux-pères et belles-mères ? ... ou bien encore à l'État ?




% 28.02.2015 :
% haut Moyen Âge
% _, --> ,
% ~etc.
% Antiquité
% ~\%


\chapter{Inertie des pratiques}


 Dans la réalité les changements ne sont pas (encore ?) aussi importants que dans l'idée que l'on s'en fait. Si depuis une génération le nombre de mariage diminue indiscutablement%
% [4]
\footnote{En 1990, 90~\% des couples existants étaient mariés, en 1999, année où le Pacs est entré dans les pratiques ils n'étaient plus que 83~\%. \emph{Histoires de familles, histoires familiales}, INSEE, 1999.}% 
, le Pacs, à l'origine pensé pour les couples homosexuels, est le plus souvent choisi par des couples mixtes (dix-neuf pacs sur vingt sont contractés par des couples mixtes), dont la moitié environ finit par se marier, et au total la somme des Pacs et des mariages est plus élevée que le nombre des seuls mariages avant la création du Pacs. Le lien entre naissances et mariage semble solide : à la naissance du deuxième enfant 86~\% des couples sont mariés, et 93~\% au troisième. On n'est pas loin, avec le Pacs, d'un mariage à l'essai.

 Si par ailleurs le nombre des divorces se situe aujourd'hui entre le tiers et la moitié de celui des mariages, il faut considérer que ce nombre est à la fois élevé et bas. Sur 29~millions d'adultes vivant en couple, mariés ou non, 26~millions (90~\%) en sont \emph{encore} à leur première expérience de couple, et pour l'instant les recompositions de familles concernent \emph{seulement} 3~millions de personnes sur 29. C'est que le nombre de couples mariés de tous âges (le « stock ») est si important que les divorces n'en représentent pas plus de 1~\% par an : 99~\% des gens qui étaient mariés au premier janvier le sont encore au 31 décembre qui suit (mais qu'en sera-t-il de ces chiffres dans une génération ?).

 En 2006, 1,2~millions de mineurs vivent en famille recomposée, soit 9~\% de l'ensemble des mineurs. Parmi ces mineurs, \nombre{400000} sont nés du couple qui s'est « recomposé ». Ceux-là vivent donc avec leurs deux parents, bien que dans une famille "recomposée". À la même date, 2,2~millions de mineurs vivent au sein d'une famille monoparentale (six fois sur sept avec leur mère), tandis que 10,25~millions de mineurs%
% [5] 
\footnote{... dont les \nombre{400000} enfants vivant au sein de familles recomposées et nés du couple nouveau.} 
vivent avec leur père et leur mère (mariés ou non). 
 


\newlength{\lcol}
\setlength{\lcol}{0.16666667\textwidth}
\addtolength{\lcol}{-2\tabcolsep}


\begin{table}[!ht]% [!htb]
%\centering
\begin{minipage}{\textwidth} 
\caption[Cadre de vie des jeunes en 1999]%
{Cadre de vie des jeunes en 1999%
\footnote{Sources :
\\« Histoires de familles, histoires familiales », \emph{Les cahiers de l'INED}, \no 156 ;
\\\emph{Recensement de la population}, INSEE, 1999, p. 281.} 
}
\label{tableau-cadre-vie-1999}
\begin{tabular}{*{6}{>{\hspace{0pt}\centering\arraybackslash}b{\lcol}}}
Âge des jeunes (années) & Vivant avec les deux parents de naissance & Avec un parent seul%
\footnote{Familles monoparentales.}
 & Avec un parent et un beau-parent%
\footnote{Familles recomposées.}
 & Autres situations%
\footnote{En internat, en appartement, en chambre, chez un logeur, en placement ASE, en prison, en hôpital,~etc.}
 & Total\\
\hline
 0-4     & 85,0 & 11,1 & 1,8 & 2,2  & 100~\% \\
 5-9     & 77,7 & 15,6 & 5,2 & 1,5  & 100~\% \\
 10-14 & 72,7 & 17,5 & 8,4 & 1,5  & 100~\% \\
 15-19 & 68,5 & 18,7 & 8,6 & 4,1  & 100~\% \\
 20-24 & 43,5 & 11,5 & 4,3 & 40,6 & 100~\% \\
\hline
 0-17  & 76,5 & 15,7 & 6,0 & 1,8  & 100~\%
\end{tabular}
\end{minipage}
\end{table}

%CADRE DE VIE DES JEUNES EN 1999[6]
% 
%\emph{Age des jeunes}
%\emph{ (années)}
%\emph{Vivant avec ses deux parents de naissance}
%\emph{Avec un parent seul[7]}
%\emph{Avec un parent et un beau-parent[8]}
%\emph{Autres situations [9]}
%\emph{Total}
%\emph{0-4}

 La comparaison %de ce tableau avec le suivant 
des tables \vrefbetterrange{tableau-cadre-vie-1999}{tableau-cadre-vie-2004-2007} 
montre que l'évolution des familles et de leurs comportements n'a rien de fulgurant. Vivre séparé de l'un de ses deux géniteurs reste une situation minoritaire : pour l'instant les trois quarts des mineurs vivent sous le même toit que leurs \emph{deux} parents \emph{de naissance} (dont les deux tiers des mineurs de 15 ans à 18 ans).

\makeatletter
\if@twoside
\begin{table}[t]% [!htb]
\else
\begin{table}[!t]% [!htb]
\fi
\makeatother
%\centering

\begin{minipage}{\textwidth} 
\caption[Cadre de vie des jeunes en 2004-2007]%
{Cadre de vie des jeunes en 2004-2007%
\footnote{Source : \emph{Moyenne annuelle des enquêtes emploi de 2004 à 2007}, INSEE.} 
}
\label{tableau-cadre-vie-2004-2007}

\begin{tabular}{*{6}{>{\hspace{0pt}\centering\arraybackslash}b{\lcol}}}
Âge des jeunes (années) & Vivant avec les deux parents de naissance & Avec un parent seul & Avec un parent et un beau-parent & Autres situations & Total\\
\hline
 0-6     & 82,2 & 10,1 & 7,2 & 0,5  & 100~\% \\
 7-13   & 72,8 & 16,6 & 9,9 & 0,7  & 100~\% \\
 14-17 & 66,9 & 19,0 & 9,8 & 4,4  & 100~\%
\end{tabular}

\end{minipage}

\end{table}

%CADRE DE VIE DES JEUNES EN 2004/2007[11]
% 
%\emph{Age des jeunes (années)}
%\emph{Vivant avec ses deux parents de naissance}
%\emph{Avec un parent seul[12]}
%\emph{Avec un parent et un beau-parent}
%\emph{Autres situations[13]}
%\emph{Total}
%\emph{0-6}
 
 Mais les évolutions actuelles sont aussi (sont d'abord ?) symboliques : peut-être n'y a-t-il jamais eu autant d'enfants qu'aujourd'hui à vivre jusqu'à leur majorité avec leur père et leur mère de naissance \tempuwave{(?)}, et pourtant les familles ne sont plus pensées comme l'alliance irréversible de deux lignées, ni comme des institutions aux limites intangibles, mais comme des associations d'individus à géométrie variable. Les enfants d'aujourd'hui apprennent très tôt que les couples mixtes sont fragiles, qu'on rencontre aussi des couples mariés de même sexe, qu'amour ne rime pas avec toujours, que les princes et les princesses n'ont pas forcément beaucoup d'enfants, et qu'ils se séparent souvent avant la fin de leur histoire. Ils apprennent à dissocier parentalité et conjugalité, ou plutôt ils n'apprennent plus à les associer de manière indéfectible. À côté des scénarii traditionnels de leurs jeux d'imagination (le gendarme et le voleur, le client et la marchande, le malade et le docteur,~etc.) ils disposent maintenant du jeu du mariage et du divorce.

 Sous l'Ancien Régime c'était le contraire : en droit civil comme en droit canon, les mariages étaient indissolubles. Par contre, la mortalité d'alors, très élevée par rapport à celle d'aujourd'hui, faisait que plus de la moitié des époux étaient séparés par la mort avant même que leurs enfants n'aient atteint leurs vingt ans, et à cet âge il était normal d'être orphelin d'au moins un de ses deux parents. La durée moyenne effective des couples conjugaux était faible, environ quinze ans, comparée à celle des couples d'aujourd'hui qui n'ont pas divorcé, autour de cinquante ans. 

 La Révolution avait autorisé et facilité le divorce \emph{par consentement mutuel}, et à la suite de cette décision le taux de divorces observé dans les villes (mais \emph{seulement dans les villes}) avait rapidement atteint le niveau actuel. Mais contrairement à ce qui s'était passé dès l'an~III, aujourd'hui personne ne semble s'en inquiéter. Personne ne se donne plus pour objectif d'enrayer ce phénomène comme ce fut le cas avec le Code Napoléon, pendant la plus grande partie du \siecle{19} et sous le régime de Vichy (1940-45). Il ne s'agit plus de punir un coupable, ou deux, ni de chercher à prouver aux conjoints qu'ils peuvent respecter leurs engagements conjugaux au prix de quelques accommodements. Au contraire, les lois accompagnent ce mouvement de « \emph{démariage}%
% [14]
\footnote{Cf. Irène \fsc{THERY}, et son livre du même nom.}
» et le divorce par consentement mutuel est devenu le modèle. 

 C'est en majeure partie du fait des divorces que les personnes seules avec enfants ont crû en nombre et en visibilité depuis 1970. En effet, le pourcentage de veufs et de veuves en leur sein a beaucoup baissé, au contraire de celui des divorcés : 9 fois sur 10 il s'agit de femmes seules avec enfants. Les appeler « familles monoparentales » comme on le fait depuis une génération, eut semblé absurde du temps tout proche où c'était le mariage et non l'enfant qui fondait les familles. 
 











\input{chapitres/et_apres}


\backmatter

% Le 02.03.2015 :
% Antiquité
% Moyen Âge
% ~etc.
% ~\%

\part*{Conclusion}

\chapter[Conclusion]{}


 Celui qui embrasse d'un seul regard le passé des familles et de la reproduction humaine ne peut pas ne pas observer que le mouvement de l'histoire semble par moments « repasser les plats » et reproduire des configurations déjà observées durant des périodes antérieures. Pourtant à bien y regarder c'est à chaque fois sous une forme originale et imprévue.  
 
 Ainsi, nous avons pu remarquer : 
\begin{itemize}

\item qu'une grande partie du Moyen Âge européen a accordé plus d'autonomie aux femmes et aux enfants que ne l'avait jamais fait une Antiquité grecque et romaine au patriarcat exemplaire ;

\item qu'au rebours de cette tendance, les \crmieme{17} et \siecle{18}s ont atteint un sommet dans le renforcement du pouvoir des pères sur toute leur famille, épouse comprise, sur le modèle du \latin{pater familias} de l'Antiquité tardive, renforcement initié par les professeurs de droit civil et religieux du onzième et du douzième siècle. Sous la pression de la compétition des deux Réformes ennemies et jumelles pour diriger les consciences et les comportements, et avec l'appui des États modernes (à moins que ce ne soit le contraire ?), les sociétés européennes ont atteint alors dans le domaine de la reproduction un niveau de conformité avec le droit canon inégalé jusque là ;

\item que la suite de cet apogée des maîtrises patriarcales au sein des familles a été la réaction anti pères, anti familles et anticléricale des Lumières et de la Révolution française : \frquote{\emph{divorce facile, autorité paternelle partagée avec la mère et contrôlée par la justice, égalité de l'enfant naturel avec l'enfant légitime, plénitude de l'adoption, administration commune des époux, diminution des incapacités dues à l'âge, les exemples sont nombreux, dans le droit de la famille, à témoigner du surprenant modernisme du législateur révolutionnaire%
% [2]
\footnote{Marcel \fsc{GARAUD}, \emph{La révolution française et la famille}, 1978, p. 191.}%
}} ;

\item qu'à cette réaction, le code civil de Napoléon a réagi en restaurant l'essentiel du droit familial d'inspiration romaine de l'ancien régime ;
 
\item qu'en fait c'est la troisième République qui a inscrit durablement dans le droit français de la famille les changements que voulaient faire les révolutionnaires ;
 
\item que c'est pourtant dans le cadre de l'État providence initié par cette même troisième république, et mis en œuvre par la quatrième, que se sont le mieux épanouies les familles traditionnelles : \enquote{\emph{Les années 1945-1965, qu'on pourrait appeler les « vingt glorieuses » de la famille (ce moment où s'impose un modèle familial, quasiment unique, où presque tout le monde se marie, où la famille est relativement féconde, où elle est stable avec un taux de divorces de moins d'un mariage sur dix, et où elle est organisée selon un principe assez strict de partage des rôles sexués, masculin et féminin) : Cette période de 1945-1965 qui nous sert souvent de référence pour penser la famille « traditionnelle » et lui opposer la famille actuelle, a été à bien des égards un moment historique exceptionnel%
%[1]
\footnote{Irène \fsc{THERY}, « Peut-on parler d'une crise de la famille ? Un point de vue sociologique », \emph{Neuropsychiatrie de l'enfance et de l'adolescence}, 2001, 49, p. 492-501.}%
}} \mbox{-- }moment où la réalité vécue par les familles a été plus proche du modèle traditionnel qu'à toute autre période ;

\item que ce sont justement ceux qui sont nés à ce moment-là, les enfants du baby-boom, qui ont joyeusement et férocement poussé la critique des familles si exemplairement traditionnelles dont ils sortaient ;
 
 Ils ont plébiscité sans réserves les conceptions des révolutionnaires français dans le domaine familial.
 
\end{itemize}
 
 Nous avons vu en quoi la famille constantinienne était une synthèse originale des pratiques matrimoniales des grecs, des juifs, des romains et des chrétiens. Cet amalgame avait pris en une seule génération, au milieu du \siecle{4}. Il avait « pris » comme une mayonnaise ou un mortier peuvent se coaguler en une forme stable et résistante aux déformations.  Si l'on fait abstraction des inflexions apportées par le Code Napoléon, de la création en 1884 du divorce pour faute et des adoucissements apportés à la belle époque au sort des enfants illégitimes, il a fonctionné jusqu'aux années cinquante du \siecle{20}, soit près de \nombre{1600} ans. 
 
 Cette antiquité vénérable donnait en quelque sorte à la famille constantinienne l'air d'être « naturelle », mais cela n'a pas empêché toutes les règles de droit sur lesquelles elle était fondée d'être abrogées entre 1965 et 1985, c'est-à-dire en un temps encore plus bref que celui qui avait été nécessaire pour les mettre en place. 
 
 La famille constantinienne n'était que l'une des modalités des familles patriarcales, et pas la plus typique. Mais elle est en train d'entraîner toutes les autres dans sa chute, même celles des sociétés qui ne l'ont pas pratiquée. Elles sont toutes travaillées par une même lame de fond partie de l'occident et qui déferle sur la planète. Le domaine de la reproduction humaine est aujourd'hui en pleine révolution.
  
 Lorsque ce remue-ménages sera fini tout redeviendra-t-il comme avant, à la façon dont le Code civil n'a gardé du droit révolutionnaire que ce que Napoléon et ses juristes ont choisi d'en préserver, tandis que pour l'essentiel ils restauraient le droit (romain) de l'ancien régime, parfois en le durcissant ? 
 
 Ou bien n'en sommes-nous qu'au début d'un processus dont l'issue est à proprement parler inimaginable pour ceux qui ont grandi dans le monde révolu des familles traditionnelles ?

 Sommes-nous parvenus au terminus indépassable de l'histoire des mœurs et de la reproduction humaine ? ... ou seulement à l'étape actuelle d'une course indéfinie, d'une histoire encore à écrire ? 
 
 La seule chose qui soit certaine c'est qu'un retour pur et simple aux pratiques du passé n'est pas possible. C'est le moment de se souvenir que selon Michel \fsc{FOUCAULT}, l'histoire avance par crises entre lesquelles règnent des périodes de stabilité, définies par un cadre de pensée (\emph{épistémè}, ou paradigmes) à chaque fois différent (cf. \emph{Les mots et les choses}). Il faut certes que les règles ainsi adoptées soient tenables, qu'elles ne provoquent pas d'effets trop pervers ni de dysfonctionnements trop insupportables ... mais les humains sont rusés et inventifs et capables d'interpréter de façon créative n'importe quel système.

 En même temps que pan à pan s'effondre dans la loi, dans les têtes et dans les comportements ce qui reste de la synthèse constantinienne, les traits de ce qui va la remplacer commencent sans aucun doute à se dessiner sous nos yeux, même lorsque nous ne savons pas les reconnaître.

On peut penser que bien des historiens, des sociologues et des moralistes du passé auraient donné cher pour être à notre place devant l'expérience d'écologie humaine en grandeur réelle dans laquelle le mouvement de l'histoire nous entraîne inexorablement ? Il sera en tout cas passionnant d'observer comment les enfants et les petits enfants des « baby-boomers » reprendront à leur propre compte toutes les questions actuellement en crise.



\nocite{*}
\printbibliography

\listoftables
\tableofcontents



\end{document}