%\documentclass[10pt,french,oneside]{book}
\documentclass[11pt,french]{book}

\usepackage[a4paper,top=1in,inner=1in,ratio={2:3},includehead,nofoot]{geometry}
%\usepackage[a4paper,ignorehead,bottom=1.42857in]{geometry}
%\usepackage[pass]{geometry}

%\usepackage[utf8x]{inputenc}
\usepackage[T1]{fontenc}


\usepackage{fontspec}
\defaultfontfeatures{Numbers=OldStyle,Ligatures=Rare,Mapping=tex-text}
%\setmainfont{Adobe Garamond Pro}
\setmainfont{Garamond Premier Pro}


%%%%%% Ce qui suit permet d'avoir le nom des parties en en-tête avec 
\usepackage{etoolbox}
\makeatletter
%%% patch \@part[#1]{#2} and \@spart (see the class file) to save the part name
\pretocmd{\@part}{\gdef\parttitle{#1}}{}{}
\pretocmd{\@spart}{\gdef\parttitle{#1}}{}{}
\makeatother

\makeatletter
%%% Numéroter les chapitres par partie %%% remplacé par package chngcntr ci-dessous
%\@addtoreset{chapter}{part}
%%% Vider le style des pages de parties et de chapitres
\let\ps@plain=\ps@empty
\makeatother


\usepackage{chngcntr}
\counterwithin*{chapter}{part}
%%% Le package hyperref a besoin de différencier les chapitres (dont le compteur est réinitialisé aux parties)
\newcommand{\theHchapter}{\thepart\arabic{chapter}}
%%% Pour l'instant, ma commande vrefnum (ci-dessous) ne fonctionne qu'avec des captions
%%% numérotées en format simple, donc un compteur de table indépendant pour tout le document
\counterwithout{table}{chapter}


\usepackage{varioref}
%%%%%% Ce qui suit ajoute la commande vrefnum au package varioref
\makeatletter
\def\vref@num#1#2{%
  \expandafter\expandafter\expandafter\vref@@num
  \csname r@#2\endcsname{}{}\@nil#1%
}
\def\vref@@num#1#2#3\@nil#4{%
  \def#4{#1}%
}
\def\vrefnum#1#2{%
  \vref@num{#1}{#2}%
}
\makeatother


\usepackage{array}

\usepackage{layout}

\usepackage[normalem]{ulem}

%\usepackage{slantsc}

\usepackage{enumitem}

\usepackage[pagestyles]{titlesec}

\usepackage{numprint}

%\usepackage{quotchap}

\usepackage[latin,english,main=french]{babel}
\frenchbsetup{og=«,fg=»}

\usepackage{hyperref}
\hypersetup{colorlinks=true, linkcolor=blue, urlcolor=blue}



% Suppression de l'affichage des numéros de sectionnement en dessous des chapitres :
\setcounter{secnumdepth}{0}

%%% Un peu d'air autour des captions...
\addtolength{\abovecaptionskip}{5pt}
\addtolength{\belowcaptionskip}{10pt}

%%% Notes temporaires invisibles
%\newcommand{\tempnote}[1]{}
%%% Notes temporaires visibles
\makeatletter
\newcommand{\tempnote}[1]%
{\marginpar{\ifodd\c@page\raggedright\else\raggedleft\fi\tiny\fontspec[Ligatures=Rare,Variant=8]{Zapfino}\selectfont #1}}
\makeatother

%%% Activer le soulignement ondulé rouge
\makeatletter\newcommand\tempuwave{\bgroup\markoverwith{\textcolor{red}{\lower3.5\p@\hbox{\sixly \char58}}}\ULon}\makeatother
%%% Désactiver le soulignement ondulé rouge
%\newcommand\tempuwave{}

%%% Activer le passage à la page suivante (temporaire)
\newcommand\temppagebreak{\pagebreak}
%%% Désactiver le passage à la page suivante (temporaire)
%\newcommand\temppagebreak{}


%%% \fsc comme avec efrench : interdit la coupure des noms, même aux espaces, mais ici l'autorise aux tirets
%%% \bsc vient de babel-french (petites majuscules, et insécables : dans une boîte)... sinon :
% \newcommand*{\bsc}[1]{\leavevmode\hbox{\scshape #1}}
\makeatletter
%
\def\@fscespace[#1 #2]%
{%
 \MakeUppercase{\@car#1\@nil}%
 \MakeLowercase{\@cdr#1\@nil}%
 \if\relax\detokenize{#2}\relax%
 \else%
  ~\@fscespace[#2]%
 \fi%
}
%
\newcommand\fsc[1]{\@fsctiret[#1-]}
\def\@fsctiret[#1-#2]%
{%
 \bsc%
 {%
  \@fscespace[#1 ]%
  \if\relax\detokenize{#2}\relax%
  \else%
   --\@fsctiret[#2]%
  \fi%
 }%
}
%
\makeatother


% D'après l'Imprimerie Nationale, les siècles s'écrivent avec des chiffres romains
% en petites capitales...
\newcommand*{\crm}[1]{\textsc{\romannumeral #1}}

\newcommand*{\crmieme}[1]{%
\crm{#1}%
\ifnum#1=1%
  \ifdefined\ier%
    \ier%
  \else%
    \textsuperscript{\lowercase{er}}%
  \fi%
\else%
  \ifdefined\ieme%
    \ieme%
  \else%
    \textsuperscript{\lowercase{e}}%
  \fi%
\fi%
}

\newcommand*{\siecle}[1]{%
\crmieme{#1}\ siècle%
}

\newcommand*{\siecles}[2]{%
\crm{#1}--\crm{#2}%
\ifdefined\iemes%
  \iemes%
\else%
  \textsuperscript{\lowercase{es}}%
\fi%
\ siècles%
}


\newcommand\vrefbetterrange[2]%
{%
 \vrefnum\numrefun{#1}%
 \vrefnum\numrefdeux{#2}%
 \vrefpagenum\pagerefun{#1}%
 \vrefpagenum\pagerefdeux{#2}%
%\vref{#1} et \vref{#2}
 \ifnumcomp{\numrefun + 1}{=}{\numrefdeux}%
 {%
  \ifnumcomp{\pagerefun}{=}{\pagerefdeux}%
  {\ref{#1} et \ref{#2} \vpageref[]{#2}}%
  {\ref{#1} \vpageref[ci-dessus]{#1} et \ref{#2} \vpageref[ci-dessus]{#2}}%
 }%
 {%
  \ifnumcomp{\pagerefun}{=}{\pagerefdeux}%
  {\ref{#1} à \ref{#2} \vpageref[]{#2}}%
  {\ref{#1} \vpageref[ci-dessus]{#1} à \ref{#2} \vpageref[ci-dessus]{#2}}%
 }%
}


%%%%%% Créer commande \anglais{texte en anglais à mettre en emphase et règles de césure anglaises...}
%%%%%% Cf. en particulier Police_des_familles dans la marge de la troisième page
\newcommand\anglais[1]%
{%
\selectlanguage{english}\emph{#1}\selectlanguage{french}%
}

\newcommand\latin[1]%
{%
\selectlanguage{latin}\emph{#1}\selectlanguage{french}%
}

\newcommand\vieuxfranc[1]%
{%
%\emph{#1}%
{\addfontfeature{Style = Historic, Ligatures=Historical}\emph{#1}}
%{\addfontfeature{Style = Alternate}\emph{#1}}
}


%\renewcommand{\theenumii}{\alph{enumii}}
\renewcommand{\labelenumii}{\theenumii.}

\title{Générations et Histoire}
\author{Hervé \fsc{Tigréat}}


\begin{document}

%%%%%% définition du style "livre", merci etoolbox pour \parttitle
\newpagestyle{livre}{\sethead[\thepage][][\scshape\parttitle]{\scshape\chaptertitle}{}{\thepage}}
%\newpagestyle{livre}{\sethead[\thepage][][\scshape\chaptertitle]{\scshape\sectiontitle}{}{\thepage}}
\pagestyle{livre}

\maketitle


\frontmatter

%\begin{savequote}%[45mm]
%Ceux qui ont tout oublié n'ont pas d'avenir
%\qauthor{Régis~\fsc{Debray}, France Culture le 4.1.2013 à 8 h 35}
%\end{savequote}

%\sloppy



”\chapter{Présentation}









Depuis les années soixante du vingtième siècle la reproduction des humains est en Europe soumise à de tels bouleversements qu'il ne s'agit plus d'une évolution, mais bien d'une révolution. Pour le moment les changements en cours dans le droit et dans les mœurs ne sont pas parvenus à leurs termes et le présent est instable, déroutant et difficile à penser. C'est là qu'un point de vue extérieur est utile pour se décentrer et comprendre un peu mieux où l'on en est. Un tel point de vue peut être fourni par les observations des ethnologues et sociologues\footnote{Cf. \emph{les Métamorphoses de la parenté} de l'ethnologue Maurice GODELIER (2004), ou bien \emph{L'origine des systèmes familiaux, T. I} (2011) ou encore \emph{Où en sommes-nous ? Une esquisse de l'histoire humaine} (2017) d'Emmanuel TODD, qui parcourt à la fois l'espace et le temps.} mais il peut aussi être trouvé dans l'histoire. La situation présente de la reproduction ne prend en effet tout son sens que par ses écarts avec les pratiques des siècles antérieurs.

La famille traditionnelle occidentale a fonctionné en Europe du haut moyen-âge au milieu du XXème siècle. Elle est née d'une synthèse entre les pratiques des romains de l'Empire, celles des juifs et celles des chrétiens de l'Antiquité (ces pratiques et les représentations qui les sous-tendaient étaient elles-mêmes le point d'aboutissement d'autant d'évolutions particulières). Les bases juridiques de la famille « traditionnelle » européenne ont été promulguées sous le règne de l'empereur Constantin et celui de ses successeurs directs mais elles ont mis de nombreux siècles à s'imposer, non sans résistances ni déformations multiples par rapport aux desseins initiaux. La trajectoire de cette forme de famille n'a atteint son apogée qu'aux derniers siècles de l'Ancien Régime. Depuis lors elle a fonctionné de manière presque hégémonique, demeurant le modèle de référence jusqu'au baby-boom, en dépit de quelques modifications significatives. Si à partir des années soixante du vingtième siècle ses fondations juridiques ont été presque totalement dynamitées, elle ne s'efface pourtant pas sans résistances, et pour l'instant elle n'est tout à fait morte ni dans les têtes ni dans les comportements. Mais dans le même temps de nouvelles formes d'union et de parentalité ont fait leur apparition et de très anciennes problématiques que l'on croyait définitivement obsolètes reviennent au devant des préoccupations. Il n'est pas question pour moi de prétendre concurrencer les historiens professionnels sur leur terrain et cet essai est fondé sur leurs écrits.  Si ce texte parvient à exposer clairement la situation où en est aujourd'hui la reproduction humaine et à la problématiser, alors il aura atteint son but.  
 
 La famille occidentale est si "traditionnelle" qu'elle nous paraissait parfaitement \emph{naturelle}, c'est pourquoi pour introduire mon sujet je vais commencer par me tourner vers les sociétés primitives. Dans son \emph{Anthropologie de l’esclavage}, (1986) Quentin MEILLASSOUX montre que les membres des sociétés primitives se sentent\footnote{…pour ceux d'aujourd'hui que la civilisation n'a pas encore arrachés à leur monde, mais les ethnologues du passé ont fait les mêmes observations.} liés par une continuité organique avec leur territoire et avec l'univers matériel dans son ensemble, avec les esprits dans (ou de) la nature, avec leurs ancêtres, avec le monde du ou des dieux. Souvent ils se pensent comme les seuls humains dignes de ce nom : chacun dans sa langue, ils se désignent eux-mêmes comme « les humains par excellence », ce qui implique que les autres, ceux qui leur sont étrangers, ne sont pas humains, ou pas vraiment humains, ou pas au même degré qu’eux. Pour eux la bonne vie, la seule vie vivable, n'est possible que sur le territoire dont ils ont hérité (même quand ils pratiquent le nomadisme c'est dans des bornes relativement limitées). Partout ailleurs c'est l'inconnu, l'étrange, l'étranger, le non humain, l'inhumain.

Le plus souvent ces sociétés ne connaissent aucune forme d'écriture. Elles se caractérisent d'abord par l'absence d'échanges marchands et de moyens de paiement, comme par la faiblesse ou l'absence de leurs structures étatiques. Elles peuvent se choisir plutôt démocratiquement un chef, mais son autorité est limitée. 

Dans ces sociétés la famille est le cadre essentiel, et parfois le cadre unique des rapports entre individus. Le chef de famille, presque toujours un homme, a pour première tâche de veiller à la pérennité de son groupe familial. La vie de chacun appartient au groupe, et il n'est pas question d'opposer à celui-ci les droits d'un individu particulier ni de mettre l'ensemble en danger pour un seul de ses membres. Si trop de naissances mettent en danger l'intérêt collectif le don des nouveaux nés excédentaires, leur abandon ou leur infanticide sont des pratiques ordinaires. En cas de disette il arrive que des vieillards se laissent mourir pour que les jeunes survivent, ou qu'on les y pousse.

La parenté assigne à chacun une fonction précise : des obligations mais aussi des droits sur les ressources du groupe. Celui-ci vit d'une économie de subsistance tournée vers l'auto-consommation. L'accumulation des biens n'est pas pensée comme la constitution d'un capital susceptible d'être réinvesti dans des opérations économiques nouvelles. Il s'agit plutôt d'acquérir des objets à haute valeur symbolique (religieuse, esthétique, magique, etc.), ou de constituer des réserves destinées à être consommées de manière festive ou/et ostentatoire. La nature lui \emph{donne} maternellement ses fruits et c'est la fécondité de son territoire qui limite la récolte et non le nombre de bras ou celui des heures de travail disponibles. Aucun membre de la famille n'est exclu des redistributions, mais les parts peuvent être très inégalement distribuées, sans tenir compte de la contribution de chaque membre du groupe au volume des biens à répartir, ni de ses besoins réels, mais plutôt de son rang et de sa place symbolique. Il est fréquent qu’à ce compte les femmes soient mal loties, mais ce n’est pas systématique. La règle de base est que les adultes travaillent pour nourrir les plus jeunes et les plus vieux. Les plus jeunes reçoivent plus qu'ils ne donnent, jusqu'à ce qu'ils soient à leur tour capables de nourrir tous ceux qui les ont nourris. Les plus vieux sont directement ou indirectement nourris par ceux qu'ils ont élevés : chacun investit dans une descendance pour préparer ses vieux jours. Quand tout se passe harmonieusement c'est au fil d'une vie entière que les tâches et les droits s'équilibrent pour chaque individu. 

Dans un tel système aucun garçon ne possède rien en propre : ni terres, ni troupeaux, etc. Si son clan refuse de lui procurer une femme, ou de lui donner les moyens d'en acquérir une, il reste bloqué dans un statut de dépendance juvénile. Condamné à travailler toute sa vie pour les enfants des autres, il n'accèdera jamais au statut avantageux et respecté de ceux qui ont de grands enfants productifs.

Ce devenir concerne moins les filles. Étant donné le taux de mortalité qui frappe les femmes enceintes, les parturientes et les enfants, les sociétés primitives ont le plus souvent besoin que chaque femme ait tous les enfants qu'elle peut porter. Les familles ont trop besoin d'enfants pour ne pas marier leurs filles dès lors qu'elles sont nubiles, éventuellement à des hommes bien plus âgés qu'elles et même déjà dotés d'autres femmes. Cette valeur qu'on leur accorde ne conduit pas à leur donner le pouvoir. Il est bien entendu qu'elles ne font pas des enfants pour elles seules, mais pour les partager ou les donner. 

Dans ces sociétés il n'y a pas de sens à faire une place à un étranger : à quel titre, au nom de quoi ? Et quel rôle lui donner ? Comment l'accueillir sans déséquilibrer le réseau compliqué et tendu des échanges et des obligations réciproques ? Du point de vue d'un guerrier l'étranger qu’il a capturé vivant est une preuve de sa valeur, mais il ne peut être un moyen d'entretenir et d'accroître sa puissance, un moyen de production de richesses (un esclave). Il n'est bon qu'à être rapidement consommé d'une façon ostentatoire : il n'est bon qu'à être sacrifié. Accepter de ses proches une rançon serait déjà entrer dans le monde marchand où une vie humaine a un coût et peut donc s'acheter ou s'échanger contre des biens réels, ce qui ne fait pas partie de leurs représentations.

Par contre s'il y a une place vacante dans une famille, celle-ci peut adopter un étranger ou une étrangère pour occuper cette place, afin que la vie continue, afin que les prestations masculines et féminines continuent d'être procurées, afin que les enfants continuent de naître, que les vieillards ne soient pas à l'abandon, que les ancêtres continuent d'être honorés, et que le monde continue sa course, etc. S'il n'y a pas assez d'épouses pour tous les garçons, on peut enlever des filles dans un autre groupe, ou leur en acheter. Un ennemi prisonnier peut d'autant plus facilement remplacer un mari ou un fils mort, que c'est ordinairement à ses voisins, à ceux que l'on pourrait épouser, qu'on fait la guerre.

En cas de conflit, de délit ou de crime, la mise au ban du groupe est d'autant plus fréquemment choisie qu'elle présente sur la mise à mort l'avantage d'éviter la souillure du territoire familial par un meurtre, ainsi que le ressentiment des ancêtres ou des dieux contre le ou les exécuteurs éventuels. Celui qui est condamné à l'exil est comme mort pour son groupe d'origine. S'il tombe aux mains d'un autre groupe, s'il est asservi (et a fortiori s'il leur était vendu) sa famille ne cherchera ni à le racheter ni à le délivrer\footnote{Ainsi le livre de la Genèse raconte comment Joseph, benjamin de Jacob, a été vendu par ses frères parce qu'ils étaient jaloux de voir qu'il était le préféré de son père. Leur première intention était de le tuer, mais comme une caravane de marchands passait par là cela leur a évité d'avoir à assumer la culpabilité de sa mort, et par dessus le marché la vente leur a rapporté de l'argent : l'Asie Mineure où se passe cette histoire était en partie entrée dans le monde marchand au moment où ce récit a été écrit.}. Si un individu qui a été banni est tué ses parents ne chercheront pas à le venger. Le jour où il mourra, les rites et sacrifices funéraires nécessaires au repos de son esprit ne seront pas exécutés. Il ne pourra pas rejoindre le monde de ses ancêtres et il ne sera pas rituellement nourri par les vivants. Son souvenir ne sera pas honoré. Cela l'exclura de son clan une deuxième fois. A ses yeux l'errance et l'exil dans un monde hostile \emph{cf. le sortde Cain après l'assassinat d'Abel} valent-ils mieux qu'une mort immédiate au milieu des siens, dans son pays ?

 A cette description Emmanuel TODD ajoute l'idée que l'organisation des familles primitives (l'organisation primitive des familles) est la même d'une extrémité de la terre à l'autre et se caractérise par des couples stables (même si les divorces sont possibles), de parents élevant eux-mêmes les enfants qu'ils ont conçus (même si les avortements, les infanticides et les abandons sont pratiqués à l'occasion) et respectant l'interdit de l'inceste. Selon lui le conjoint est choisi au sein du groupe de vie (au sens large) mais en dehors de la famille nucléaire. Les relations entre familles apparentées (frères et soeurs, beaux-frères et belles-soeurs) ont de l'importance étant donné le soutien mutuel qu'elles peuvent se fournir. Elles sont donc entretenues et les familles du père et de la mère ont autant d'importance l'une que l'autre. Les règles de succession sont souples et il n'y a pas de souci d'égalité stricte ni de principe de primogéniture. Il n'existe chez les primitifs aucune société réellement matriarcale mais le statut des femmes n'y est pas dévalorisé, même s'il peut y avoir à l'occasion de la polygamie. Selon TODD c'est ce type de famille qui prévalait en Europe de l'Ouest (Germanie, Gaule, Iles britanniques, Scandinavie...) avant l'entrée en scène des romains et les bouleversements de tous ordres qu'ils ont apportés, dont notamment une vision assez radicale du patriarcat. 


Des sociétés primitives à celles d'aujourd’hui l'histoire de la reproduction humaine est indissociable de celle de la prise en charge des personnes faibles, malades, âgées, infirmes ou démunies. En effet la famille, quelle que soit sa composition et son organisation, a toujours été la première institution d'assistance, quand elle n'était pas la seule\footnote{Pour aller plus loin on pourra se reporter à l'\emph{Histoire des enfants, des familles et des institutions d'assistance, La protection de l'enfance de l'antiquité à nos jours}, Hervé Tigréat, Pascale Planche et Jean-Luc Goascoz, préface de Pascal David, L'Harmattan, 2018.}.





Cet essai est à la disposition de tous pour un usage privé ou dans le cadre d'un enseignement. 

Usage commercial non autorisé. 

Tous droits de représentation et de reproduction réservés.

Copyright : libre de droits, mentionner l'auteur











\chapter{Prologue}


Les membres des sociétés primitives%
\footnote{Pour rédiger cette présentation, je me suis particulièrement appuyé sur
Quentin \fsc{Meillassoux} : \emph{Anthropologie de l'esclavage}, 1986.}
se sentent liées par une 
continuité organique avec leur territoire et avec l'univers matériel dans 
son ensemble, avec les esprits dans (ou de) la nature, avec leurs ancêtres, 
avec le monde des esprits et du ou des dieux. Souvent ils se pensent 
comme les seuls vraiment humains : chacun dans sa langue, ils se 
désignent alors eux-mêmes comme « les humains par excellence », ce
qui implique que les autres, ceux qui leur sont étrangers, ne sont pas humains, 
ou pas vraiment humains, ou pas au même degré qu'eux. Pour eux la 
bonne vie, la seule vie vivable, n'est possible que sur le territoire dont ils 
ont hérité. Partout ailleurs c'est l'inconnu, l'étrange, l'étranger, le
non-humain, l'inhumain.

Le plus souvent ces sociétés ne connaissent aucune forme
d'écriture. Elles se caractérisent d'abord par l'absence d'échanges marchands et 
de moyens de paiement, comme par la faiblesse ou l'absence de leurs 
structures étatiques. Elles vivent d'une économie de subsistance tournée 
vers l'auto consommation. L'accumulation des biens n'est pas pensée par 
elles comme la constitution d'un capital susceptible d'être réinvesti dans 
des opérations économiques nouvelles. Il s'agit plutôt d'acquérir des
objets à haute valeur symbolique (religieuse, esthétique, magique,~etc.) ou 
de constituer des réserves destinées à être consommées de manière
festive ou/et ostentatoire. La nature leur donne ses fruits (maternellement). 
C'est la fécondité de leur territoire qui limite la récolte et non le nombre
de bras ou celui des heures de travail disponibles.

Dans ces sociétés la famille est le cadre essentiel, et parfois le
cadre unique des rapports entre individus, même si le groupe de vie est constitué de plusieurs familles. Le chef du groupe, du clan, de la tribu.., presque toujours un 
homme, a pour première tâche de veiller à la pérennité de son groupe 
familial. La vie de chacun appartient au groupe, et il n'est pas question 
d'opposer à celui-ci les droits d'un individu particulier ni de mettre
l'ensemble du groupe en danger pour un seul de ses membres. Si trop de 
naissances mettent en danger l'intérêt collectif le don des nouveaux nés 
excédentaires, leur abandon ou leur infanticide sont des pratiques
ordinaires. En cas de disette il arrive que des vieillards se laissent mourir 
pour que les jeunes survivent.

La parenté assigne à chacun une fonction précise : des obligations 
mais aussi des droits sur les ressources du groupe. Aucun membre de la 
famille n'est exclu des redistributions, mais les parts peuvent être très 
inégalement distribuées, sans tenir compte de la contribution de chaque 
membre du groupe au volume des biens à répartir, ni de ses besoins
réels, mais plutôt de son rang et de sa place symbolique. Il est fréquent 
qu'à ce compte les femmes soient mal loties, mais ce n'est pas
systématique. La règle de base est que les adultes travaillent pour nourrir les plus 
jeunes et les plus vieux. Les plus jeunes reçoivent plus qu'ils ne donnent, 
jusqu'à ce qu'ils soient à leur tour capables de nourrir tous ceux qui les 
ont nourris. Les plus vieux sont directement ou indirectement nourris 
par ceux qu'ils ont élevés : chacun investit dans une descendance pour 
préparer ses vieux jours. Quand tout se passe normalement c'est au fil 
d'une vie entière que les tâches et les droits s'équilibrent pour chaque
individu.

Dans un tel système aucun garçon (et a fortiori aucune fille) ne 
possède rien en propre : ni terres, ni troupeaux,~etc. Si son groupe familial refuse de 
lui procurer une femme, ou de lui donner les moyens d'en acquérir une, 
il reste bloqué dans un statut de dépendance (juvénile). Condamné à
travailler toute sa vie pour les enfants des autres, il n'accèdera jamais au
statut avantageux et respecté de ceux qui ont de grands enfants productifs.

Dans ces sociétés il n'y a pas de sens à faire une place à un
étranger : à quel titre, au nom de quoi ? Et quel rôle lui donner ? Comment 
l'accueillir sans déséquilibrer le réseau compliqué et tendu des échanges 
et des obligations réciproques ? Pour un guerrier l'étranger ou l'ennemi qu'il a
capturé vivant est une preuve de sa valeur, mais il ne peut être un moyen 
d'entretenir et d'accroître sa puissance, un moyen de production de
richesses. Il n'est bon qu'à être rapidement consommé d'une façon
ostentatoire : il n'est bon qu'à être sacrifié : accepter de ses proches une
rançon serait déjà entrer dans le monde marchand, ce monde où une vie humaine a un 
coût et peut se monnayer, ce qui ne fait pas partie de leurs représentations.

Par contre s'il y a une place vacante dans une famille, celle-ci peut 
adopter un étranger ou une étrangère pour occuper cette place, afin que 
la vie continue, afin que les prestations masculines et féminines
continuent d'être procurées, afin que les enfants continuent de naître, que les 
vieillards ne soient pas à l'abandon, que les ancêtres continuent d'être 
honorés, et que le monde continue sa course,~etc. S'il n'y a pas assez 
d'épouses pour tous les garçons, on peut enlever des filles dans un autre 
groupe, ou leur en acheter. Un ennemi prisonnier peut d'autant plus 
facilement remplacer un mari ou un fils mort, que c'est ordinairement à ses 
voisins, à ceux que l'on pourrait épouser, qu'on fait la guerre.

En cas de conflit, de délit ou de crime, la mise au ban du groupe 
est d'autant plus fréquemment choisie qu'elle présente sur la mise à mort 
l'avantage d'éviter la souillure du territoire familial par un meurtre, ainsi 
que le ressentiment des ancêtres ou des dieux contre le ou les exécuteurs 
éventuels. Celui qui est condamné à l'exil est comme mort pour son 
groupe d'origine. S'il tombe aux mains d'un autre groupe, s'il est asservi 
(et a fortiori s'il leur était vendu) sa famille ne cherchera ni à le racheter 
ni à le délivrer. Ainsi le livre de la Genèse raconte comment Joseph,
benjamin de Jacob, a été vendu par ses frères parce qu'ils étaient jaloux de 
voir qu'il était le préféré de son père. Leur première intention était de le 
tuer, mais comme une caravane de marchands passait par là cela leur a 
évité d'avoir à assumer la culpabilité de sa mort, et par dessus le marché 
la vente leur a rapporté de l'argent (l'Asie Mineure où cela se passait était 
déjà en partie entrée dans le monde des marchands).

Si un individu banni est tué ses parents ne chercheront pas à le 
venger. Le jour où il mourra, les rites et sacrifices funéraires nécessaires 
au repos de son esprit ne seront pas exécutés dans les règles. Il ne pourra 
pas rejoindre le monde de ses ancêtres et il ne sera pas rituellement
nourri par les vivants. Son souvenir ne sera pas honoré. Cela l'exclura de son 
clan une deuxième fois. Aux yeux des intéressés l'errance et l'exil
valent-ils toujours mieux qu'une mort immédiate, mais au milieu des siens ?

Aucune société primitive ne faisait le poids militairement face 
aux sociétés plus développées, telles que les cités antiques. Elles leur 
ont servi de réserves de futurs esclaves à capturer périodiquement (dès 
qu'ils avaient surmonté la crise démographique et morale créée par le précédent prélèvement). Ce 
système a perduré longtemps : au \siecle{19} on l'observait encore
en bien des endroits écartés des centres où se concentraient les hommes 
et où s'inventait l'avenir.

%NB. Pour rédiger cette présentation je me suis particulièrement appuyé sur Quentin \fsc{Meillassoux} :
%\emph{Anthropologie de l'esclavage}, 1986.
%    %Passé en note au début du chapitre





\mainmatter

\layout

%A1 Cités antiques, inégalitaires et patriarcales 
%A2 Cités esclavagistes 
%A3 Les représentations des hommes de l'antiquité 
%B1 L'exception juive 
%B2 moeurs juives 
%B3 Travailleurs et esclaves juifs 
%C1 les lois d'Auguste 
%C2 Evolutions du droit civil romain sous l'Empire 
%D1 les chrétiens et la génération 
%D2 Une contre-société chrétienne ?

\part{Nos Antiquités}

% Le 10.03.2015 :
% \emph{pater familias}
% Grec, Romain
% 28.02.2015 :
% haut Moyen Âge
% _, --> ,
% Antiquité
% ~etc.
%~\%


\chapter[Cités antiques, inégalitaires et patriarcales]{Cités antiques,\\inégalitaires et patriarcales}


Les cités grecques et romaines n'étaient des démocraties que pour
leurs propres citoyens, et ceux-ci ne représentaient qu'une faible fraction de leur
population. Côte à côte vivaient des personnes aux statuts personnels
différents et hiérarchisés, sans parler des citoyennes, traitées comme des
mineures, réduites au silence sur les affaires publiques, et plus ou moins
cloîtrées chez elles, même à Rome où les femmes honnêtes ne sortaient
de chez elles que voilées de la tête aux pieds.

À côté des citoyens vivaient les étrangers domiciliés et les étrangers
de passage. On les nommait \emph{métèques} chez les Grecs, \emph{pérégrins} chez les
Romains, c'est-à-dire voyageurs, étrangers en déplacement (d'où est venu
notre \emph{pèlerin}). Leur situation était précaire. Ils devaient apporter des avantages
à la cité et ne lui faire souffrir ni dépense ni désagrément. En Grèce
ils devaient payer des taxes de séjour. Ils n'avaient pas le droit de vote. Ils
n'avaient pas le droit d'épouser des citoyens ou des citoyennes. Ils ne
pouvaient hériter d'un citoyen ni posséder terres ou maisons. Ils ne pouvaient
porter d'armes, ni faire partie des soldats, et en cas de guerre on
les tenait en suspicion. Ils étaient tenus à l'écart des cérémonies du culte
des dieux de la cité. Ils ne pouvaient porter plainte ni se défendre eux-mêmes.
Ils ne pouvaient être entendus directement par les tribunaux. Ils
devaient trouver parmi les citoyens honorables un « patron » qui les représente
et les défende en justice, mais ce « patronage » n'était pas gratuit.
Les peines qu'ils subissaient étaient beaucoup plus sévères que celles
que subissaient les citoyens. Ils ne pouvaient devenir citoyens qu'avec
beaucoup de difficultés, surtout dans l'aire grecque où ils pouvaient demeurer
métèques de père en fils dans la même cité pendant plusieurs siècles%
\footnote{Ainsi
les juifs d'Alexandrie, exclus pendant trois siècles de la citoyenneté de cette cité grecque et bloqués dans le statut de métèques. Ils ont fini par être totalement éliminés par un dernier pogrom, à la suite d'une dernière révolte.}.

À côté de l'ensemble des personnes libres, citoyens et étrangers,
s'activait la masse des esclaves, parfois plus nombreuse que celle des citoyens :
 trente à quarante pour cent de la population, peut-être plus, dans
les périodes où l'esclavage était le plus prospère. L'esclavage colorait la
totalité du fonctionnement et des représentations des sociétés antiques,
et notamment du fonctionnement des familles.

À divers signes vestimentaires chacun pouvait repérer si une
femme était célibataire ou si elle vivait en couple, si une jeune fille était
libre ou esclave. De la même façon le corps des jeunes garçons de famille
comme celui des autres hommes libres étaient protégés par des signes visibles.
Le vêtement permettait de savoir quel(le) esclave était prostitué(e)
au public, quelle autre était réservée par son maître à son service exclusif.
Chacun savait ainsi où il était autorisé à désirer et où il ne l'était pas, où il
pouvait porter la main et où il lui fallait absolument s'abstenir.

Seuls les hommes libres avaient la plénitude des droits civiques. Ils
étaient les chefs de leur famille, qui étaient toutes « patriarcales ».
Ils étaient de droit les prêtres du culte des ancêtres et de celui des dieux
de la maison sur l'autel familial. Ces cérémonies étaient si importantes
que leur bonne exécution était surveillée par les cités. C'est eux qui exerçaient
la justice dans leur maison. Ni leurs concitoyens ni les lois
n'étaient fondés à s'immiscer dans les relations qu'ils entretenaient avec
leurs dépendants (épouse, enfants, esclaves, clients), sauf s'ils excédaient
les limites de manière scandaleuse ou sacrilège : si leur ivresse de pouvoir
ou de colère dépassait la mesure \latin{(hubris)} ou si leurs pratiques incestueuses
risquaient d'attirer les foudres du ciel et d'atteindre indirectement leurs
voisins.

Les hommes adultes et libres n'étaient jamais astreints à la continence,
et les maris n'étaient pas astreints à la fidélité (à Rome ils pouvaient
même entretenir des maîtresses ou des amants sous le toit conjugal).
Au contraire, la vie sexuelle des femmes était toujours contrôlée par
un homme (père, frère, mari, concubin,~etc.), et les épouses se devaient
d'être fidèles à leurs époux quoi qu'ils fassent avec d'autres (filles ou garçons)
qu'elles.

Parmi les relations sexuelles hors mariage on distinguait le concubinage (un homme, marié ou non, et une
femme non mariée, une liaison stable avec à Rome obligation de fidélité pour la concubine), le stupre (un homme 
marié, une femme non mariée, pas de stabilité, liaison sans lendemain), la fornication (un homme et une femme 
non mariés, pas de stabilité, liaison sans lendemain), ou l'adultère (un homme, marié ou non, une femme mariée, 
et une liaison sans lendemain).

Pour Aristote, entre maître et esclaves, entre père et enfants, il n'y
a pas de rapports de justice à strictement parler, mais seulement une
justice \emph{domestique}, exercée de façon monarchique par le père de famille \emph{et
son épouse}. Si cette dernière est en position seconde par rapport à son
époux, elle ne lui est pas pour autant assujettie%
\footnote{{\emph{Il est clair que les grands textes classiques de la Grèce conçoivent le mariage comme une association établie en vue de la bonne gestion du patrimoine et de la procréation des enfants pour la continuation de la famille et le peuplement de la cité. Toutefois, l'idée d'une ségrégation des femmes dans le gynécée, la mainmise de l'homme libre sur toutes les formes de gouvernement risqueraient d'occulter les exigences éthiques que Xénophon, Platon, Aristote imposaient aux époux. La femme, objet d'échange entre un père et celui qui devient son gendre, acquiert, une fois mariée, des droits et des privilèges. Les rapports des époux relèvent de la vertu de justice. Et, si la femme commet une injustice en refusant de se soumettre à son mari, celui-ci a le devoir de la former, afin qu'elle partage son pouvoir sur la maison. L'épouse ne saurait se confondre ni avec une esclave ni avec une enfant. Le rapport que l'époux exerce sur elle n'en est pas moins de domination.}}
Marie-Odile \fsc{Métral-Stiker} : \emph{Mariage et couple} © Encyclopædia Universalis, 2004.}.
Elle a droit
d'émettre des objections, elle a droit à des explications, elle peut
réclamer d'être convaincue et chercher elle-même à convaincre : elle a
droit à la parole. Selon le philosophe les deux époux sont donc dans
une relation \emph{politique}. 

Les citoyennes du temps des républiques antiques vivaient sous la
tutelle des hommes, en position de perpétuelles mineures. C'étaient les pères qui seuls avaient pleine autorité
sur leurs enfants. En cas de divorce ou de répudiation ils en avaient la
garde, et ils les confiaient à qui ils voulaient, éventuellement à leur ex
épouse, mais sous leur propre contrôle. Elles pouvaient
être répudiées au moyen d'une simple lettre, sans que leur mari ait
à se justifier, tandis qu'elles-mêmes ne pouvaient pas demander le divorce,
puisque leur statut de mineures légales leur déniait la capacité de poser des actes juridiques. Pour le même motif elles ne pouvaient
pas intenter d'action en justice ni gérer leurs biens propres
(leur dot et/ou leurs autres biens personnels). Le travail salarié
n'existait pas pour elles, ce qui ne veut pas dire qu'elles ne faisaient
rien, mais qu'elles travaillaient quasi exclusivement
à domicile sous l'autorité de leur père ou de leur mari.  Elles ne pouvaient
diriger une entreprise ni un commerce. Veuves, elles ne pouvaient
exercer la tutelle de leurs propres enfants. 





Sous la République romaine et selon la tradition la plus ancienne,
les filles quittaient leur propre famille en se mariant et entraient dans celle
de leurs maris. Elles étaient placées par leurs pères sous l'autorité de
ceux-ci, comme s'ils les adoptaient. Juridiquement, leur statut ressemblait
à celui de leurs propres fils et filles : \emph{comme si} elles étaient les sœurs de
leurs enfants. En cas de décès de leur mari, elles étaient d'ailleurs comptées
au nombre de ses héritiers, à égalité avec chacun de leurs enfants
communs. Mais bien avant la fin de la république romaine, elles ont le
plus souvent été mariées \emph{sine manu}, sans « la main », c'est-à-dire
que leurs pères conservaient leur autorité sur elles \latin{(manu)} et
qu'elles restaient juridiquement étrangères à la famille de leurs enfants.
Leurs pères pouvaient de leur propre initiative rompre les mariages de
leurs filles s'ils y trouvaient intérêt ou (plus souvent ?) si elles le leur demandaient
et s'ils y consentaient. Ces dispositions juridiques n'empêchaient
pas les jeunes Romains de savoir qu'ils étaient nés de l'union de
deux familles, ni d'être fiers de la famille de leur mère et de l'aimer autant
que celle de leur père.

Les femmes mariées de Rome avaient droit au \emph{manteau des matrones},
qui enveloppait tout leur corps et ne laissait voir que leur visage à l'exclusion
des cheveux. Les femmes du monde grec étaient plus strictement
recluses dans leurs gynécées et exclues de l'espace public que les Romaines.
Quant aux femmes d'Asie (Asie Mineure), certaines d'entre elles
étaient depuis toujours entièrement voilées, visage y compris. Le
« manteau des matrones », le voile, protégeait celles qu'il recouvrait
contre les regards et surtout les gestes masculins. C'était en effet un signal :
les agressions sexuelles contre une femme étaient punissables
comme des crimes si (et seulement si) la victime portait le manteau des
matrones ou la tunique des jeunes filles libres.

Les enfants légitimes sont ceux qui peuvent hériter de la fortune
de leur père même sans testament, qui peuvent de droit lui succéder et le
remplacer (les garçons au moins) dans toutes ses charges et prérogatives.
Pour les Grecs et les Romains n'étaient légitimes que les enfants nés d'une
mère mariée \emph{en justes noces} avec leur père au moment de leur conception.

Les noces n'étaient \emph{justes}, un mariage n'était valide, que si les règles
suivantes étaient respectées :
\begin{enumerate}
%a) 
\item Il fallait que les liens de parenté ne l'interdisent pas%
\footnote{Cf. p. 92, \emph{Histoire du droit civil}, Jean-Philippe \fsc{Levy}, André \fsc{Castaldo}.}.
Le mariage
était toujours interdit en ligne directe jusqu'à l'infini (parents,
grands-parents, enfants, petits-enfants,~etc.). Par contre le monde hellénistique
autorisait le mariage entre frères et sœurs, ce qui était surtout
pratiqué en Égypte. Les Grecs autorisaient le mariage entre demi-frères et
sœurs. À Athènes le mariage était possible entre demi-frères et sœurs de
même père (consanguins). À Sparte il était autorisé entre demi-frères et
sœurs de même mère (utérins). 

Par contre les romains ont toujours éprouvé une grande horreur pour tout ce qui se rapproche de l'inceste. Ils ont toujours interdit
le mariage entre frères et sœurs. Jusqu'au \siecle{3} avant notre
ère, le droit romain interdisait le mariage jusqu'au sixième degré, aussi
bien du côté des \emph{agnats} (famille du père) que des \emph{cognats} (famille de la
mère). A partir du \siecle{3} av. J.-C.  la rigueur des interdictions s'est relâchée. Le mariage a continue d'être interdit entre bru et beau-père,
belle-fille et beau-père, gendre et belle-mère, ou beau-fils et belle-mère,
mais il est devenu possible entre beaux-frères et belles-sœurs. La \latin{sobrina}
(cousine issue de germains) puis la \latin{consobrina} (cousine germaine) ont été
autorisées. Le mariage entre tante et neveu a toujours été interdit,
entre l'oncle paternel et la nièce il a fini par être autorisé sous le règne de Claude (empereur de l'an 41 à l'an 54) qui a épousé sa nièce Agrippine, mère de Néron. A partir du début du deuxième siècle les anciens interdits vont à nouveau être progressivement remis en vigueur. 
 
\item Il fallait que le statut personnel des deux conjoints autorise un
mariage. L'union d'un citoyen avec une métèque ou une pérégrine (et inversement)
ou une esclave (idem) n'était pas valide, ce qui ne veut pas dire
qu'il n'y en avait pas, mais que les enfants qui en naissaient étaient illégitimes,
et avaient ordinairement le statut de leur mère, même quand ils
étaient reconnus par leur père. À Rome les patriciens, nobles, ne pouvaient
se marier en dehors de leur caste.
%c)
\item La fille ne devait avoir été ni enlevée ni séduite contre l'accord
de son père (ou de son tuteur). Il fallait qu'il l'ait donnée à son époux, et
de bon cœur. Il ne devait pas l'avoir vendue contre de l'argent, ni échangée
avec des cadeaux ou des avantages personnels, comme cela se passait
pour les concubines.
\end{enumerate}
Les hommes ne supportaient pas l'idée de mourir sans héritier, ne
serait-ce que pour assurer le culte des morts familial : leur propre avenir
\latin{post mortem} était en jeu. Ils accordaient donc un soin jaloux au choix de
leur héritier, et il n'était pas question qu'on leur en impose contre leur
gré. Il s'agissait aussi et en même temps d'assurer la prospérité de leur
maison et la sécurité de leurs vieux jours. Pour y parvenir, ils disposaient
d'un certain nombre d'outils :

\begin{enumerate}
%a) 
\item La fidélité des épouses : La plus grande des fautes que celles-ci
pouvaient commettre était d'introduire \emph{subrepticement} dans la famille de
leur mari un héritier qui ne serait pas né de ses œuvres%
\footnote{En l'absence des connaissances sur la physiologie de la reproduction qui sont les nôtres aujourd'hui les représentations étaient incertaines et infiltrées de projections. On se demandait si le corps des femmes coopérait à la 
fécondation à égalité avec celui des hommes, s'il existait en quelque sorte un « sperme féminin » dont la présence 
était nécessaire, ou s'il n'était qu'un réceptacle passif pour le sperme masculin, seul actif (cf. Aline \fsc{Rousselle}, 
1998). Les modèles en présence oscillaient de celui où le corps de l'homme et celui de la femme coopèrent à la fécondation à celui où seul l'homme compte, mais ils n'écartaient pas absolument celui où les précédents amants de 
la femme comptaient aussi (mais pas les amantes de l'homme). On supposait parfois que le corps d'une femme 
était marqué par la semence de chacun de ses partenaires (en se mêlant à son sang), et que cette imprégnation jouait
un rôle obscur mais réel
dans la fécondation de ses
enfants légitimes,
conçus des œuvres
de son mari. Cette 
croyance redoublait la faute des femmes adultères et augmentait l'importance de la virginité des jeunes filles. Elle 
aggravait encore la gravité des viols qu'elles pouvaient subir, les rendant irréparables.}.
À Rome (mais pas en Grèce) s'il était consentant nul n'avait plus rien à y redire.
%b) 
\item La répudiation : la stérilité des couples (20~\% des unions jusqu'au
\siecle{19} ?) était presque toujours imputée à la femme, c'est
pourquoi le renvoi de celle-ci, suivi d'un remariage, était la solution normale,
attendue.
%c) 
\item Compte tenu du caractère pré scientifique des idées du temps
sur la grossesse il y avait une relative indistinction entre les méthodes anticonceptionnelles
et les méthodes abortives. On estimait que la vie ne
commençait qu'à partir de la coagulation de la semence (masculine seule,
ou masculine et féminine, suivant les auteurs) coagulation dont la durée
pouvait être longue puisque c'est le mouvement du fœtus qui en fournissait
la preuve. Aristote (384-322 avant notre ère) pensait que les garçons
étaient animés au bout de quarante jours de grossesse, les filles au bout
de quatre-vingt (cette opinion sera tenue pour la plus vraisemblable jusqu'à
la fin du Moyen Âge en raison de l'immense autorité de son auteur).
Les mesures anticonceptionnelles et abortives (entendues à partir du
moment où les mouvements du fœtus étaient perceptibles) n'étaient
frappées d'aucun interdit. Plus que des médecins c'était d'abord le domaine
des femmes d'expérience, dans le secret des gynécées. Dans ce
domaine la faute pour une femme c'était d'aller contre le désir de son
époux ou de son maître, qui avaient le droit de la contraindre à avorter,
comme celui de lui interdire de le faire.
%d) 
\item L'infanticide : ceux des enfants qui naissaient mal formés ou
trop chétifs semblent avoir été aussitôt noyés ou étouffés par les sages-femmes,
avec l'assentiment de tous. Selon Sénèque : {\emph{les enfants, s'ils sont
débiles ou difformes, nous les noyons}%
\footnote{\fsc{Seneque}, \latin{De ira}, Livre I, chapitre VI}%
}.
Pour \hbox{Soranos}, médecin grec du
\siecle{2} de notre ère, la puériculture est l'art de décider {\emph{quels sont les
nouveaux-nés qui méritent qu'on les élève}}. L'infanticide était aussi le sort ordinaire
des esclaves nouveaux-nés et de tous les enfants nés de conceptions
irrégulières, embarrassantes ou scandaleuses (inceste, adultère...).
%e) 
\item L'abandon : le père de famille faisait abandonner {\emph{au bon cœur
des inconnus}%
\footnote{John \fsc{Boswell}, 1993.}%
} les nou\-veaux-nés dont il ne voulait pas. Tant qu'il le faisait
dans les premiers jours, avant de leur avoir donné un nom, personne
n'avait rien à y redire. Lorsqu'il souhaitait qu'ils vivent il les faisait exposer
en un lieu public, connu de tous, et où se rendaient régulièrement
ceux qui étaient à la recherche d'un nourrisson sans parents (à tel endroit
du forum, entre telle et telle colonne de tel temple,~etc.). Selon un auteur
de comédie du \siecle{3} avant notre ère c'étaient d'abord les filles
qu'on exposait : {\emph{un garçon on l'élève même si on est pauvre, une fille, on l'expose
même si on est riche%
\footnote{Cité par Jean-Nicolas \fsc{Corvisier} et Wieslaw \fsc{Suder}, \emph{la population de l'Antiquité classique}, Que sais-je, PUF, 2000.}%
}} (boutade qui ne nous dit malheureusement pas
quels étaient les pourcentages d'abandons effectifs).
%f) 
\item La vente : longtemps les pères ont eu le droit de vendre leurs
enfants à tout âge, dès leur naissance. Ils pouvaient a fortiori les mettre
en gage chez un prêteur. Assez tôt les cités méditerranéennes ont pourtant
interdit la réduction en esclavage de leurs propres citoyens (mais pas
celle de leurs métèques et pérégrins) autrement que par un jugement public
en bonne et due forme : une sanction pénale. Mais comment punir
un père d'avoir vendu son enfant si c'était le seul moyen qu'on lui avait
laissé pour que cet enfant et lui-même puissent survivre (cf. chapitre~\vref{vente-parent}) ?
%g) 
\item Le don d'un fils à un autre citoyen : un citoyen suffisamment
pourvu en garçons pouvait donner l'un d'eux à un concitoyen dépourvu
de fils. Le fils donné perdait tout droit à l'héritage de son père, mais devenait
l'héritier de son père adoptif. Quant à son père de naissance il
pouvait ainsi donner une part d'héritage plus importante aux héritiers qui
lui restaient.
%h) 
\item Le choix des gendres : les filles ne choisissaient pas leur mari.
Les pères (et à défaut les tuteurs) donnaient leurs filles aux hommes de
leur choix, et à certaines conditions ils pouvaient aussi rompre leur mariage
à leur gré si une autre alliance paraissait préférable.
%i) 
\item L'adoption d'un garçon pour en faire son héritier : qu'il soit célibataire
ou marié un homme pouvait adopter des enfants ou des adultes
pour en faire ses héritiers et successeurs. Seul un citoyen pouvait hériter
d'un autre citoyen, afin que ses biens (notamment la terre de la cité, toujours
ressentie comme sacrée) ne tombent pas aux mains d'étrangers :
seuls des citoyens étaient donc adoptables par les citoyens. C'étaient les
seuls dont les successions avaient une importance politique.

En Grèce un homme ne pouvait adopter qu'en l'absence complète
d'héritier mâle, et il fallait que l'adopté soit choisi au plus près possible
dans sa parentèle mâle. Seuls pouvaient être adoptés des enfants ou des
jeunes gens légitimes (nés en justes noces de deux parents eux-mêmes
nés libres, citoyens et légitimes). Si l'adoptant n'avait qu'une fille et aucun
fils légitime (fille \emph{épiclère}), le garçon qu'il adoptait se devait d'épouser cette
fille (sa cousine germaine dans beaucoup de cas) sauf à perdre la jouissance
de l'héritage de celle-ci au profit de ses enfants dès qu'ils seraient
en âge de le revendiquer, à leur majorité.

Ni à Rome ni en Grèce une femme ne pouvait adopter. Elle pouvait
prendre en charge un enfant jusqu'à l'âge adulte et l'établir dans la
vie, si elle obtenait de son mari ou de son tuteur le droit de le faire. Mais
elle ne pouvait pas pour autant en faire son successeur, et elle n'avait jamais
l'autorité d'un père sur lui. De même aux yeux de la loi une épouse
n'était-elle pas la mère légale des enfants \emph{adoptés} par son mari.
%j) 
\item La légitimation des enfants des concubines : \emph{dans l'aire grecque}
l'enfant d'une concubine, même née libre, et d'un citoyen était un citoyen
illégitime et ne pouvait hériter. Quant à l'enfant né d'un citoyen et d'une
esclave, il ne pourrait même pas devenir un citoyen, ni lui ni sa postérité
après lui. Les enfants des concubines, libres ou esclaves, ne pouvaient
donc jamais succéder à leurs pères. Les laisser vivre ne présentait pour
ces derniers que des inconvénients.

\emph{À Rome par contre} les esclaves devenaient des citoyens adoptables à
la condition d'être affranchis dans les règles : seul un citoyen romain
pouvait transmettre ce statut, non une citoyenne ni un pérégrin. Il pouvait
déclarer dès leur naissance comme libres les enfants de ses esclaves,
ce qui faisait d'eux des citoyens. Il pouvait alors les reconnaître pour
siens et en faire ses héritiers. Si une concubine lui donnait un garçon
c'était une assurance contre le risque de se retrouver un jour sans descendance,
aussi n'avait-il aucune raison de s'interdire d'avoir une ou plusieurs
concubines, libres ou esclaves. Les concubines, affranchies ou libres,
étaient astreintes à la fidélité au même titre que les épouses, et elles
avaient droit elles aussi au manteau des matrones. Le citoyen romain
pouvait aussi laisser dans le statut d'esclave un enfant né de ses relations
avec une de ses esclaves. Il pouvait ainsi en obtenir une obéissance plus
exacte, et se donner le temps de vérifier s'il était digne de lui succéder. À
la différence de la déclaration faite dès la naissance l'affranchissement ultérieur
ne pouvait pas faire de l'affranchi un fils légitime, ni effacer la
marque servile qui limitait ses droits civiques et qui lui interdisait la plupart
des « honneurs », mais elle lui permettait quand même d'hériter. Il
suffisait pour cela que le père affranchisse son enfant par testament et en
fasse son légataire universel.
%k) 
\item À côté de l'adoption les anciens connaissaient la prise en charge
d'un \latin{alumnus}, d'un enfant nourricier. Cela bénéficiait à des garçons ou des
filles sans parents. Celui qui avait accueilli un enfant sans connaître son
statut de naissance (ce qui était la règle avec les nouveaux-nés exposés)
l'élevait à son gré comme un esclave ou comme un libre. C'était une espèce
d'adoption incomplète. \latin{L'alumnus} n'entrait pas dans la famille de
l'adoptant et n'avait aucun droit à hériter. Les Romains pouvaient néanmoins
choisir leur héritier parmi leurs \latin{alumnii}. Cette pratique ne reposait
que sur la bonne volonté de l'adoptant et sur l'affection mutuelle, parfois
teintée d'une dose de pédophilie. Puisqu'il ne s'agissait pas de faire des
\latin{alumnii} des successeurs les femmes étaient sur ce terrain à égalité avec les
hommes, et il n'y avait pas de limites au nombre d'enfants qu'ils ou elles
pouvaient ainsi accueillir. Ils ou elles se sentaient généralement le devoir
moral d'établir dans la vie ceux de ces jeunes qu'ils avaient élevés comme
des personnes libres.
\end{enumerate}


\section{Le \emph{pater familias} romain}

Il ne faut pas confondre le patriarcat et la prééminence des hommes
sur les femmes. Le patriarcat des anthropologues c'est la prééminence du principe paternel
dans l'ordre de la parenté, ce n'est pas seulement la prééminence du masculin : celle-ci
se retrouve régulièrement à peu près partout, à un degré plus ou moins accusé, même là où les familles sont matriarcales dans l'ordre de la parenté. 
Le cas le plus significatif c'est celui des sociétés où l'enfant appartient
entièrement à la famille de son père, et où il n'a rien à attendre en
terme de protection ou d'héritage de la famille de sa mère. 

Il va de soi que la prééminence masculine au quotidien est confortée lorsque la famille paternelle l'emporte dans l'odre de la parenté.
C'était le cas des familles de la Rome républicaine qui présentaient
des traits patriarcaux très purs, au moins sur le plan du Droit.

Le \latin{pater familias} de Rome n'était pas un père banal : en fait il n'avait
même pas besoin d'être père. Un père pouvait n'être pas le \latin{pater familias}
de ses enfants, tandis que quelqu'un qui n'avait aucun enfant pouvait être
\latin{pater familias}.

Les dépendants d'un \latin{pater familias} s'adressaient à lui en l'appelant
\latin{dominus} : « maître », « seigneur ». Et une \latin{familia} n'était pas ce que nous
appelons une famille, mais l'ensemble des personnes dépendantes d'un même
\latin{dominus} : d'abord l'ensemble de ses esclaves, et accessoirement tous ses
autres \emph{familiers}, enfants y compris.

Par contre son épouse n'en faisait partie que si son propre père
avait donné à ce dernier sa puissance sur elle : c'était le mariage \latin{cum manu}
(« avec la main »), en voie de désuétude à la fin de la République.

L'autorité d'un \latin{pater familias} s'éteignait à sa mort, ou bien quand il
était condamné à une peine infamante, ou bien quand il devenait esclave,
quel qu'en soit le motif, ou bien quand il renonçait de lui-même à sa
propre puissance sur un enfant en l'émancipant, ou bien en le vendant,
ou bien en le donnant à un autre \latin{pater familias} pour que celui-ci l'adopte.
Pour posséder le titre de \latin{pater familias} il fallait être homme \latin{(vir)}, et citoyen
romain. Il fallait avoir vécu soi-même sous l'autorité d'un autre \latin{pater familias}.
Il fallait ne plus être sous sa puissance (\latin{sub manu}, « sous sa main »).
Cette autorité se transmettait donc comme un témoin qu'on se passe
d'homme à homme, à la façon d'un adoubement ou d'une onction, à la
suite d'une initiation, ou encore comme une espèce de grade militaire.

Un homme ne devenait \latin{pater familias} qu'à la mort de son propre
père, même s'il était chargé d'enfants depuis longtemps : à Rome il n'y
avait de majorité civile (à partir de 25 ans) que pour les orphelins de père.
Seuls ces derniers étaient autonomes : \latin{sui juris}.

Tant qu'un \latin{pater familias} était vivant, aucun de ses enfants ne possédait
rien en propre et ne pouvait prendre aucune initiative susceptible
de diminuer sa fortune (gérer des affaires personnelles, emprunter, se
marier, affranchir des esclaves,~etc.). Rien ne pouvait le contraindre de
couvrir les dettes contractées par un de ses enfants : nul n'était donc assez
imprudent pour prêter de l'argent à celui qui n'était pas encore orphelin
de père, quel que soit son âge. Le statut juridique d'un homme en
puissance de père était donc en partie celui d'un mineur, même s'il avait
femme et enfants, même s'il était d'âge mûr, même s'il était magistrat ou
s'il dirigeait des campagnes militaires,~etc. Ceci dit l'âge au mariage des
hommes, (autour de 25 ans, et souvent plus) et les taux de mortalité antiques
faisaient qu'à 20 ans la plupart des fils étaient orphelins de père.

La consommation sexuelle n'était pas nécessaire à la validité du
mariage romain. De droit le \latin{pater familias} était le père de tous les enfants
de son épouse, sauf de ceux dont il contestait la paternité, ce qu'il était le
seul à pouvoir faire. Du moment que les choses se passaient avec son accord
ou sur son ordre, un eunuque était légalement le père des enfants de
son épouse, et nul n'avait rien à y redire.

On a vu que les femmes commettaient une faute très grave si elles
introduisaient l'enfant d'un autre homme dans la famille de leur mari à
l'insu de celui-ci. Elles en commettaient une tout aussi grave si elles le
privaient volontairement d'un héritier, par avortement, abandon, infanticide,
etc.

Le \latin{pater familias} de la république romaine pouvait abandonner ses
enfants, les vendre, les donner, les faire mourir à leur naissance, ou les
chasser (= premier sens de l'émancipation). Sans aller jusque là, il pouvait
les déshériter et léguer ses biens à un enfant adopté%
\footnote{Il fallait tout de même qu'il ait un motif de mécontentement suffisant pour choisir cette solution sinon
la jurisprudence voulait que la justice casse le testament.}%
.

Celui qui avait vendu un de ses enfants n'en récupérait pas moins
sa puissance \latin{(potestas)} sur cet enfant si celui-ci était affranchi par son maître
(encore fallait-il qu'il le sache). Selon la loi romaine des XII tables, table
IV, (vers 450 avant notre ère), il pouvait néanmoins perdre définitivement
sa puissance. {\latin{Si pater filium ter venum duit, filius a patre liber esto}}:
si un père vend trois fois son fils, ce fils sera émancipé de la puissance de
ce père. Ce cas de figure implique un père qui utiliserait son fils (qui
pouvait parfaitement être un homme adulte) comme une source de revenus,
au lieu de le laisser libre de trouver des revenus lui-même, à défaut
de pouvoir subvenir lui-même à ses besoins.

Du moment qu'il pouvait prouver sa paternité, le père qui avait
abandonné un nourrisson ou un enfant (abandonné et non vendu, ce qui
aurait éteint ses droits tant que l'enfant n'était pas affranchi) pouvait
n'importe quand le reprendre à celui qui l'avait recueilli, sans avoir à lui
verser aucune indemnité. Sa puissance n'avait pas été éteinte par l'abandon.

À Rome quel que soit son âge, un citoyen qui n'avait pas de \latin{pater
familias} pouvait se mettre volontairement sous l'autorité d'un autre citoyen :
c'était \emph{l'adrogation}. Le résultat était le même qu'une adoption. Dans
les deux cas il fallait un écart d'âge minimum entre les partenaires, pour
mimer la nature. Une fille aussi pouvait bénéficier de l'adrogation. Pas
plus qu'un adopté un adrogé n'était \latin{ipso facto} héritier principal et successeur
de son \latin{pater familias}. Il fallait pour cela un testament en bonne et
due forme.


\section{Les incertitudes du patriarcat ?}

Les sociétés méditerranéennes des derniers siècles avant notre ère
(romaine, grecque, juive...) étaient patriarcales. La famille romaine
semble être le modèle même de la famille patriarcale, si l'on s'en tient à
l'image que nous en donne le Droit du temps de la République. Peut-on
dire pour autant que l'organisation patriarcale est la forme la plus ancienne
de la famille ? À cette question Emmanuel \fsc{Todd}%
\footnote{Emmanuel \fsc{Todd}, \emph{L'origine des systèmes familiaux}, Tome I, L'Eurasie, NRF,
essais, Gallimard, Paris, 2011.}
répond par
la négative (il récuse d'ailleurs tout autant l'idée qu'il ait existé des familles
authentiquement matriarcales).

Selon lui le patriarcat aurait été inventé en Mésopotamie durant le
troisième millénaire, et en Chine à la fin du deuxième millénaire avant
notre ère. Il se serait ensuite très lentement diffusé de proche en proche
dans le monde ancien. Le patriarcat donne aux hommes la prééminence
et l'autorité et met les femmes en une position inférieure, assujettie. Là
où il a réussi il aurait selon \fsc{Todd} supplanté la famille nucléaire qui
donne la même importance aux parentés maternelles et paternelles et où
femmes et hommes ont une valeur et une autonomie comparables.

Le patriarcat n'aurait atteint la cité romaine qu'au cours du dernier
millénaire avant notre ère, bien après son arrivée en Grèce, et il n'aurait
pas réussi à se propager plus loin vers le nord de l'Italie, vers la Gaule et
la Germanie... Le droit romain de la famille serait donc le témoin de la
pointe la plus extrême de la marche vers l'ouest du patriarcat durant
l'Antiquité.

Mais toujours selon \fsc{Todd}, les signes que la situation réelle vécue
au sein des familles romaines ne reflétait pas la pureté du modèle patriarcal
sont nombreux. Ce modèle aurait été adopté par les Romains avec
l'enthousiasme des néophytes, témoin le Droit romain des deuxième au quatrième siècle avant notre éré, mais ils auraient
continué de ressentir et de penser en termes de famille nucléaire,
d'égalité des familles maternelle et paternelle et d'égalité des époux en dignité,
et ils auraient valorisé l'amour conjugal de manière bien peu patriarcale.
C'est d'abord les familles les plus riches en patrimoine,
l'aristocratie, qui aurait intériorisé les normes patriarcales, tandis que celles
qui avaient peu à transmettre y auraient été peu sensibles. La Grèce
aurait au contraire effectué une conversion plus complète.

À partir de la fin de la république et pour diverses raisons, les Romains
(c'est-à-dire toutes les personnes libres de l'Empire à partir du troisième
siècle de notre ére) seraient revenus à une situation moins déséquilibrée, reconnaissant
de plus en plus à égalité la double parenté paternelle et maternelle, meme dans les lois.



% Le 20 mars 2015 :
% ~etc.
% Antiquité
% ~\%
% Grec
% 


\chapter{Cités esclavagistes}

Un esclave, c'est quelqu'un qu'on achète et qu'on vend comme un
objet ou un animal : un {\emph{instrument animé}}, un {\emph{animal
domestique qui parle}}. Il n'a pas droit à un nom propre. Il n'est compté ni au nombre des citoyens,
ni au nombre des étrangers. Il n'est enregistré nulle part et sa
mort ne compte que pour son propriétaire. Son corps n'est pas protégé
par la loi. Jusqu'à la fin de la République romaine, un maître a droit de vie
et de mort sur ses esclaves. A fortiori peut-il légitimement user de leur
corps comme il l'entend : le battre, le blesser, le torturer, le violer, le
prostituer, le castrer, l'avorter,~etc. Dans la mesure où un esclave n'est
pas considéré comme une personne, mais comme un simple prolongement
de la personne de son maître (une « prothèse »), ses rapports
sexuels avec celui-ci n'ont pas le sens d'une relation interpersonnelle.
Un esclave ne peut pas porter plainte. S'il est à bout de souffrances
il n'a comme recours que de chercher asile dans un temple, en espérant
être revendu à un maître moins dur. C'est le maître qui est responsable
de tout ce que l'esclave fait, dit ou subit, qui porte plainte et qui
touche les dommages et intérêts s'il est blessé ou tué, qui supporte les
frais et les amendes s'il cause un accident, et c'est encore lui qui le punit,
sauf à l'abandonner à la partie adverse comme compensation \latin{(nexus)}.

Un esclave n'a pas droit à la parole (publique) : on ne peut faire
crédit à ses paroles que s'il éprouve une crainte plus forte que celle que
son maître lui inspire, ce qui implique de le torturer. En son nom propre
il ne peut ni signer un contrat, ni s'engager par serment, ni se marier, ni
exercer une autorité sur personne. S'il engendre un enfant il n'est pas son
parent légal, même si son maître lui en confie l'éducation. Si une esclave
donne naissance à un enfant celui-ci appartient à son maître : s'il le laisse
vivre il peut le lui enlever pour le vendre ou le confier à quelqu'un d'autre.
Par contre il peut exercer par délégation un pouvoir sans limites autres
que le désir de son maître. Il n'est qu'un instrument, mais si son maître
est un grand personnage sa puissance effective peut être très grande.

Seul le maître subvient aux besoins de l'esclave. Il est admis qu'il
en a le devoir même quand ce dernier est malade, estropié, trop vieux
pour servir,~etc. mais jusqu'à la fin de la République romaine ce n'est
qu'une élégance morale. L'affranchir parce qu'il est devenu incapable de
fournir un travail (ce qui le réduit à mendier et/ou à mourir plus ou
moins vite de faim et de misère) n'est pas un délit, tout juste une inélégance
morale.

Tout se passe donc comme si l'esclave ne faisait pas partie du genre
humain, mais personne n'est dupe de cette convention juridique. Sans
cela on ne lui ferait pas la morale. Sans cela on se contenterait de le menacer
des pires supplices s'il venait à porter la main contre son maître (ce
qu'on ne manque pas de faire aussi). Sans cela on ne traiterait pas comme
un assassin le meurtrier d'un esclave (sauf si c'est son maître). Mais la
meilleure preuve qu'un esclave n'est pas une chose, c'est qu'il peut acquérir
ou retrouver les droits des personnes libres. À partir de ce moment il
reçoit un nom propre (jusque là il n'en avait pas), bâti (à Rome) sur celui
de son maître et non sur celui qu'il possédait éventuellement avant de
devenir esclave. C'est une espèce de renaissance sociale. À partir de ce
moment l'ancien esclave est compté dans les recensements. Il peut porter
plainte lui-même si on lui fait tort. En dépit de quelques limites à ses
droits, qui constituent la \emph{marque servile}, il peut prêter serment, se marier,
devenir enfin le père légal de ses enfants (s'il a pu les racheter) et pas seulement
leur géniteur, comme de ceux qu'il aura à l'avenir.

Le maître peut donner à son esclave les droits d'une personne libre,
ou les lui vendre : très souvent l'affranchi doit en effet verser à son
maître une somme au moins égale à sa propre valeur vénale, prélevée sur
le « \emph{pécule} » qui lui est laissé pour encourager ses efforts. Si le maître peut
donner ces droits, cela signifie qu'il les détenait. Un esclave n'est donc
pas exactement quelqu'un qui n'a aucun droit, c'est quelqu'un dont tous
les droits appartiennent à un autre que lui-même, ce qui est le statut des
personnes mineures. D'ailleurs le latin désigne les esclaves et les enfants
du même mot, \latin{puer}.

Les mineurs ne peuvent parler pour eux-mêmes \latin{(in fans)}. Ce sont
leurs parents qui détiennent tous leurs droits. Ils les commandent en
tout, et les sanctionnent s'ils leur désobéissent. Ce sont aussi les parents
qui portent plainte si leurs enfants sont lésés par un tiers. Et c'est encore
eux qui reçoivent l'argent des dommages et intérêts. Mais au cours de
leur croissance les enfants prennent progressivement possession de leurs
droits en apprenant à les exercer dans le cadre et les limites fixés par la
loi et par leurs parents. Leur soumission n'a qu'un temps, contrairement
à ce qui se passe pour l'esclave, et ils n'ont pas à racheter leur statut
d'homme libre.

%Lorsqu'ils sont affranchis les esclaves restent dans la mouvance de
Affranchis, les esclaves restent dans la mouvance de
leur maître, à qui comme des enfants ils doivent leur liberté (même
quand ils la lui ont achetée, puisqu'il aurait pu refuser de la leur vendre),
qui devient leur « patron » : \latin{patronus} est dérivé de \latin{pater}. Ils portent son
nom. Ils lui doivent respect et assistance. Ils renforcent son prestige et
son influence avec plus de docilité et de gratitude obligée que des fils selon
la chair, des gendres ou des beaux-frères.


\section{Un esclave, pour quoi faire ?}

\subsection{Une main-d'œuvre à bon marché}

On peut appeler \emph{servile} n'importe quel travail, s'il est fait par un esclave
sur l'ordre d'un maître. Son statut ne lui interdit en aucune façon
d'être compétent ou cultivé, et parfois bien plus que son maître. Il peut
exercer la médecine ou enseigner une langue étrangère ou la philosophie,
etc. C'est ainsi que Rome a été intellectuellement conquise par les Grecs
qu'elle avait réduits en esclavage. Ils lui ont fourni ses précepteurs, ses
professeurs, ses artistes, et ses médecins. Un maître avisé faisait en sorte
que ceux de ses jeunes esclaves qui étaient doués d'un esprit vif ou d'un
talent particulier soient bien instruits. Cela lui permettait de louer plus
cher leurs compétences, de les établir à son compte dans un atelier, une
boutique, un cabinet médical, une exploitation agricole, à charge pour
eux de lui verser une contribution régulière.

Mais à l'exception de ces relatifs privilégiés c'est de manière brutale
qu'on exploitait la force de travail des esclaves sur les domaines agricoles,
dans des ateliers artisanaux ou industriels, sur les galères,~etc. D'autres
travaillaient dans les mines ou les chantiers de travaux publics : au
cinquième siècle avant notre ère plusieurs dizaines de milliers d'esclaves
travaillaient dans les mines publiques du \emph{Laurion} (mines d'argent), et
remplissaient ainsi les caisses d'Athènes, qui grâce à eux a pu à diverses
périodes se passer de lever des impôts tout en équipant ses armées et sa
puissante flotte de guerre. À côté d'autres ressources matérielles et intellectuelles,
l'esclavage a ainsi puissamment concouru au « Miracle Grec ».
Parfois loués à la journée par leur propriétaire (comme dans les entreprises
d'intérim actuelles), ces esclaves vivaient dans des brigades à l'organisation
militaire, encasernés comme les bagnards français, les déportés du
Troisième Reich ou ceux de tous les Goulag(s).

L'objectif n'était ni leur rééducation ni leur mort lente, pourtant la
vie de ces esclaves-là n'en était pas moins une épreuve, parfois pire que la
mort. Ils travaillaient d'ailleurs à côté des délinquants condamnés à l'esclavage
(ex : les condamnations romaines \latin{ad minas}, aux mines, substituts
à la peine de mort). On attendait des esclaves un rendement productif.
Cela impliquait qu'ils demeurent en bonne santé : les premiers lieux de
soins collectifs, les premières infirmeries, pour ne pas dire les premiers
hôpitaux, semblent être nés dans les exploitations industrielles ou agricoles
qui employaient des esclaves par centaines ou par milliers. Les premiers
professionnels de la médecine ont peut-être été des esclaves ou des
affranchis grecs formés dans ces hôpitaux-là ? Une clientèle captive et
peu considérée socialement est en effet une aubaine pour l'entraînement
à la pratique de la médecine et de la chirurgie.

\subsection{Un corps sans défenses}

Pour celui qui est devenu un esclave, ni ses parents, ni son milieu
d'origine, ni la ville à laquelle il a peut-être été arraché n'existent plus. Ils
n'ont pas su, pas voulu ou pas pu le défendre contre sa dégradation. Ils
l'ont peut-être même chassé. Si un individu se vend lui-même c'est qu'il
accepte l'idée que sa famille ne peut plus lui apporter de soutien. Il
l'abandonne autant qu'il s'en voit abandonné. Son passé est disqualifié,
réduit à rien, et son nom avec. Il est désormais seul, hors parenté, coupé
de ses racines et disponible pour toutes les nouvelles impressions.

L'esclave participe au culte des ancêtres du maître, c'est-à-dire qu'il
adopte pour ancêtres ceux de son maître, avec ordre d'oublier les siens,
qui n'ont pas su lui porter chance. Au dehors de la maison du maître il
est désigné par le nom collectif de la \latin{familia}. En quelque sorte il porte sa
livrée, son « logo ».

En lui refusant un nom propre on lui dénie le droit à une identité
personnelle. Il est la chose de son maître, aux intérêts et désirs duquel il
faut qu'il s'identifie s'il veut survivre. C'est cette dépersonnalisation (cette
aliénation) de l'un des deux et cette exaltation de l'autre qui est au cœur
de l'intérêt que présente l'esclave pour son maître. Tout se passe comme
si seul le maître avait un narcissisme.

Particulièrement typique du travail servile est le travail au contact
et au service du corps d'autrui (l'aider à s'habiller, lui laver les pieds, l'accompagner
aux thermes et porter ses affaires de bain, le raser, se soumettre
à ses désirs sexuels,~etc.). C'était une position traditionnellement désignée
comme « féminine » ou (et) infantile. L'esclave mâle était dévirilisé
par son statut. Les esclaves \emph{domestiques} vivaient sous le toit du maître :
portiers, femmes de chambre, concubines, cuisiniers, cochers et palefreniers,
valets, intendants, secrétaires, précepteurs... Ils le connaissaient
personnellement et pouvaient en être remarqués et favorisés. Ils étaient
\emph{comparativement} privilégiés et vivaient plutôt plus confortablement que les
autres esclaves. Mais il allait de soi \emph{dans ce cadre de pensée} que le maître ne
pouvait pas abuser de ses esclaves, puisque ceux-ci n'avaient le droit de
se refuser à aucune de ses exigences. On peut aussi bien dire que le maître
avait le droit d'abuser de tout esclave et que celui-ci ne pouvait s'en
plaindre à personne. Les esclaves pouvaient être mis au service du public,
et on en trouvait dans tous les types de services à la personne. Le plus
constant de ces services était la prostitution. Elle reposait sur des bataillons
d'esclaves des deux sexes, jeunes filles et jeunes garçons (parfois castrés
pour conserver les attributs de la juvénilité). Il n'y avait pas d'âge
pour commencer. C'est le marché qui commandait. Le statut d'esclave
permettait de satisfaire en toute légalité la demande des pédophiles et autres
amateurs de pratiques de toutes espèces.

Pour les ambitieux l'esclave était un outil qui offrait de grands
avantages. Il n'obéissait qu'à son maître et il était de par la loi dans sa «
main » plus qu'aucun collaborateur libre : c'était un outil docile. Il pouvait
être corrigé sans ménagements, sans qu'aucune famille ne puisse le
défendre ou venger l'affront subi. D'autre part il ne faisait pas partie des
citoyens, il ne pouvait donc pas briguer la position sociale de son maître
ni lui porter ombrage. S'il était promu par ce dernier à un poste enviable
il continuait de tout lui devoir, et pouvait toujours être remis à la place
où il était avant sa promotion. Si son maître faisait de mauvaises affaires
ou tombait en disgrâce, lui-même se retrouvait sur le marché aux esclaves
et tout était pour lui à recommencer. Même si les esclaves n'étaient
pas attachés à leurs maîtres par l'affection, ils leur étaient liés solidement
par la juste appréhension de leurs intérêts réciproques. Chaque puissant
personnage avait intérêt à posséder beaucoup d'esclaves. Il lui était même
possible de les organiser en milices privées pour combattre contre celles
de ses concurrents, les armes à la main. Le latin nommait les esclaves et
les enfants du même mot, \latin{puer} : en effet l'esclave était bloqué comme un
éternel mineur dans une position infantile. Souvent il s'agissait d'ailleurs
d'un enfant, plus ou moins promis à être affranchi une fois adulte. L'esclavage
réalisait alors une espèce de transfert du père au maître : une espèce
d'adoption « au petit pied ». Lorsque les esclaves étaient affranchis
ils restaient dans la mouvance de leur maître, qui devenait leur patron. Ils
portaient son nom. Ils pouvaient même (à Rome) être adoptés. Ils venaient
ainsi renforcer leur \latin{dominus} avec plus de docilité et de gratitude
obligée que des fils selon la chair, des gendres ou des beaux-frères.

Mais c'est justement pour cela que l'intérêt d'un dirigeant qui était
parvenu au sommet du pouvoir, du Prince, n'était jamais de voir se multiplier
les esclaves de ses concurrents potentiels. D'autre part l'esclavage «
interne », celui de ses propres concitoyens, celui des individus issus de
son propre peuple, lui retirait des sujets mobilisables et imposables, et
renforçait le pouvoir de ses rivaux potentiels. C'est pourquoi selon Alain
\fsc{Testart} lorsque ceux qui parvenaient au pouvoir suprême étaient avisés
et qu'ils avaient suffisamment de force ou d'autorité :
\begin{enumerate}
% A)
\item leur première
démarche était d'interdire de réduire en esclavage aucun de leurs
propres sujets, quel qu'en soit le motif. Cette interdiction leur valait la reconnaissance
du petit peuple, et obligeait leurs rivaux potentiels à recourir
exclusivement à des étrangers s'ils voulaient des esclaves ;
% B)
\item leur seconde
initiative était de se poser en protecteur de tous les esclaves, s'immisçant
ainsi en tiers au sein de la relation maître -- esclave pour affaiblir
le lien de soumission absolue qui reliait le second au premier. L'objectif
ultime du Prince était de se faire des esclaves des sujets dévoués ;
% C)
\item une
troisième solution était qu'il se réserve le monopole de la possession des
esclaves, qu'il garde donc pour lui et ses vassaux l'exclusivité de ceux que
la guerre lui procurait. En constituant ses propres esclaves comme une
armée à sa dévotion, et en se réservant l'exclusivité de ce type d'outil il se
donnait le moyen de désarmer les citoyens libres (et d'abord ses concurrents
directs) et de faire régner son ordre.
\end{enumerate}

On trouvait donc des esclaves publics dans l'entourage de bien des
dirigeants. Parfois une partie ou la totalité de l'administration était constituée
d'esclaves, jusqu'au sommet y compris. Les républiques antiques
n'ont jamais eu recours à des armées d'esclaves pour lutter contre leurs
ennemis extérieurs, mais seulement à des citoyens libres, égaux (en principe)
et dotés du droit à la parole. Par contre dès l'apogée des cités grecques
des municipalités possédaient des esclaves (publics) qui assumaient
les tâches des employés municipaux d'aujourd'hui. Ils étaient ouvriers
communaux, employés à la bibliothèque, au tribunal, à l'assemblée, au lycée,
ou chargés de la police de la ville,~etc.%
%[1]
\footnote{À Athènes dès le \siecle{5} avant J.-C., ils peuvent loger où ils veulent et vivre en famille avec une concubine. Ces esclaves n'ont pas le souci du lendemain : c'est à la ville, à l'État (leur propriétaire), de s'en charger, sauf à les libérer ou les vendre à quelqu'un d'autre. De vrais fonctionnaires titulaires de leur poste ? Il se peut que cela ait été vécu ainsi, au moins par endroits et par moments. Cela dit le système répressif qui leur était réservé ne leur permettait pas de prendre à la légère leurs attributions. Quinze ou vingt siècles plus tard en Turquie le corps des Janissaires était constitué de soldats esclaves du sultan. Ils avaient été « prélevés » (enlevés) encore enfants, au titre du tribut, sur les populations non musulmanes de l'empire (populations soumises au statut de dhimmi. À partir de ce jour ces enfants n'entretenaient plus de relations avec leur famille. Convertis d'autorité à la religion de leur maître ils n'avaient plus d'autre parent que lui.}

En conclusion le sort concret des esclaves n'était pas homogène,
et les gens de l'Antiquité en avaient bien conscience, au point peut-être de
moins percevoir les ressemblances que les différences ? Chaque cas
concret devait être repéré sur les axes suivants : du travail manuel au travail
intellectuel ; du travail d'exécution le moins qualifié aux tâches de
commandement d'équipes de travail, à la gestion d'une entreprise, ou à
l'expertise (artisans, artistes, enseignants, médecins) ; de l'intimité charnelle
la plus étroite avec le maître à la distance la plus impersonnelle. Le
seul point commun à tous était le statut des esclaves. C'est ce qui fait dire
à Paul \fsc{Veyne} que {\emph{l'esclavage est, tantôt un lien juridique archaïque qui s'appliquait
au rapport de domesticité, tantôt esclavage de plantation, comme dans le sud
des États-Unis avant 1865. Dans l'Antiquité, la première forme est de très loin la
plus répandue. L'esclavage de plantation, qui seul concerne les forces et rapports de
production, est une exception propre à l'Italie et à la Sicile de la basse période hellénistique,
de même que l'esclavage de plantation était une exception dans le monde du \siecle{19} :
la règle en matière agraire pour l'Antiquité était... la paysannerie libre
ou le servage. Spartacus, après avoir détruit le système de l'économie de plantation,
aurait évidemment admis, comme toute son époque, l'esclavage domestique.}} (in Paul
\fsc{Veyne}, \emph{Comment on écrit l'histoire}, chapitre VIII, Éditions du Seuil, Paris,
1971).


\section{La fabrique des esclaves}

Les esclaves que consommaient les sociétés antiques provenaient de
plusieurs sources. La violence armée en fournissait un nombre qui variait
selon les lieux et les époques, mais il existait aussi d'autres filières auxquelles
on pense moins, imbriquées étroitement dans le fonctionnement
ordinaire des familles antiques, et qui suivant les périodes ont pu être au
moins aussi « productives ».

\subsection{La violence}

Selon les lois de la guerre unanimement acceptées le vaincu appartenait
corps et âme au vainqueur, lui, sa ou ses femmes, ses parents, ses
enfants, ses trésors, ses esclaves et ses terres. Le vainqueur pouvait à son
gré le tuer ou l'épargner. Il pouvait lui faire payer une rançon, exiger de
lui un tribut régulier en signe de soumission, ou encore en faire un esclave.
Il était ordinaire que les combattants soient tués tandis que les jeunes,
les femmes et les esclaves capturés étaient traités comme un butin, distribués
ou vendus. De là à faire la guerre pour se procurer des esclaves il
n'y avait qu'un pas, souvent franchi.

Aristote l'admettait d'ailleurs sans détours : si ceux qui sont faits
pour obéir, c'est-à-dire les barbares, acceptaient de se soumettre sans faire
d'histoire à ceux qui sont faits pour commander, c'est-à-dire les Grecs,
il ne serait pas nécessaire de leur faire la guerre, mais comme ils ne sont
pas raisonnables il faut bien aller les chercher les armes à la main, dans
leur propre intérêt (leur intérêt, de son point de vue, étant qu'ils laissent
les plus raisonnables diriger leurs vies).

Les pirates et les bandits de grand chemin savaient eux aussi faire
des esclaves en s'en prenant aux voyageurs solitaires, aux maisons isolées,
aux bateaux de pêche et de commerce, en enlevant les enfants mal
gardés,~etc. Dès que les autorités baissaient leur garde ce mode de production
des esclaves reprenait de l'importance, et cela durera jusqu'à la
fermeture du dernier marché aux esclaves.

\subsection{La sanction pénale}

Les prisons antiques n'étaient pas des lieux d'exécution des peines.
Avant jugement elles servaient à mettre les prévenus à la disposition de la
justice et à l'abri des vengeances privées, et après jugement à attendre
l'exécution de la peine. Par contre la grande majorité des maîtres possédaient
une prison privée, l'\emph{ergastule}%
%[2]
\footnote{Cela est si vrai qu'à différentes reprises les autorités romaines ont inspecté tous les ergastules pour délivrer les personnes libres qui y étaient injustement retenues, ou pour en débusquer les citoyens qui se soustrayaient ainsi à l'incorporation dans l'armée.}%
, sans laquelle ils ne pouvaient
conserver leur personnel ni dormir en sécurité. Les esclaves y demeuraient
enfermés, entravés ou non, en dehors de leur temps de travail.
Dans ce cadre la sanction des délits commis par des hommes libres, citoyens
ou métèques, pouvait être leur réduction au statut d'esclave. Un
condamné pouvait être vendu à un marchand d'esclaves, à un organisateur
de combats de gladiateurs, ou condamné aux mines et carrières de
l'État. Une condamnée pouvait être vendue à une maison de prostitution,
au même titre que n'importe quel(le) autre esclave, ou employée au
service du personnel masculin des mines, des carrières,~etc. Dans ces bagnes
vivaient enfermés des milliers d'esclaves, dont un bon nombre avait
été condamné par un tribunal. Les condamnés à mort avaient eux aussi le
statut d'esclaves, {\emph{esclaves de la peine}}, durant l'intervalle entre leur
condamnation et leur exécution.

\subsection{Le sur-endettement}

Mais à côté de ces portes d'entrée dans l'esclavage il existait
d'autres mécanismes qui conduisaient à l'esclavage sans faire appel à la
violence armée, ni à une condamnation légale. Le plus efficace était
l'usure : le crédit à la consommation était à très court terme (ex : une semaine),
les taux d'intérêt étaient exorbitants (ex : 20 ou 30~\%), et les intérêts
produisaient eux-mêmes des intérêts. Si le remboursement n'était pas
effectué sans délai, la dette accrue des intérêts accumulés atteignait en
quelques mois des niveaux vertigineux. Cela mettait très vite ceux qui
étaient contraints d'emprunter, paysans pauvres et travailleurs libres
payés à la tâche, les \emph{mercenaires}, dans l'impossibilité de rembourser.

Même si l'entretien des esclaves ne coûtait que la nourriture la
plus grossière et la plus économique, comme ils mangeaient même s'ils
ne travaillaient pas, il y avait avantage à les maintenir occupés tant qu'il y
avait de l'ouvrage, quelle que soit la rémunération de cet ouvrage. De ce
fait les travailleurs libres ne trouvaient d'embauche que lorsqu'il y avait
trop à faire pour les esclaves et c'étaient eux qui absorbaient les irrégularités
du marché du travail. Dès qu'il était possible de s'en passer les mercenaires
ne pouvaient plus entretenir élever leur progéniture. Une fois
épuisées leurs économies ils étaient contraints de mendier pour acheter à
manger, ce qui était hasardeux, ou d'emprunter, sauf à être les clients
d'un patron \latin{(patronus)} d'autant plus généreux qu'ils avaient peu de chances
de pouvoir lui rendre un service à la hauteur de ses secours.

Rome connaissait la \emph{prison pour dettes}. La prison en question était
une prison privée (l'ergastule du créancier). À la condition d'être prêt à
prouver la réalité de sa créance devant un magistrat s'il lui en faisait la
demande le créancier s'assurait lui-même, avec l'aide de ses clients et de
ses esclaves, de la personne physique du débiteur. Celui qui ne pouvait
pas rembourser était remis en tant que \latin{nexus} (mot dont vient {\emph{annexion}})
à son créancier, qui pouvait faire de lui ce qu'il voulait, le faire travailler à
son profit, en abuser de différentes façons, le vendre (à Rome) comme
esclave au delà du Tibre, ou le tuer (aux temps anciens).

Les prêteurs exigeaient des gages. Un jour venait où l'emprunteur
n'avait plus d'autre gage à donner que l'un de ses dépendants ou lui-même.
Celui qui était \emph{gagé} vivait chez le prêteur, incarcéré avec les esclaves
de la maison, et aussi rigoureusement qu'eux. Il travaillait comme eux
et avec eux. Si l'emprunt était remboursé, il récupérait de plein droit sa
liberté. Dans le cas contraire il était inévitable que vienne un jour où il
serait abandonné au prêteur pour éteindre la dette \latin{(nexus)}. Si l'on admettait
que son travail remboursait jour après jour une fraction de la dette,
alors échapper à la servitude demeurait encore imaginable mais si son
travail n'était considéré que comme la contrepartie de sa nourriture, alors
le capital emprunté devait être remboursé par un tiers. Les usages et les
lois ont varié sur ce point.

\subsection{La naissance}

Sauf initiative de leur maître les enfants des esclaves avaient le statut
de leur mère. Ceci dit elles étaient le plus souvent achetées pour travailler
durement et non pour faire des enfants. Les grossesses et les allaitements
les épuisaient d'autant plus qu'on refusait de tenir compte de
leur état. Les accouchements n'étaient pas sans danger, surtout pour celles
qui avaient été mal nourries et pour celles qui étaient mal conformées,
ce qui pouvait être le cas de celles d'entre elles qui avaient subi une enfance
de misère. D'autre part l'élevage des nouveaux-nés était risqué :
beaucoup mouraient bien avant d'avoir rendu le moindre service, et une
fois grands tous n'étaient pas employables avec profit, surtout si on avait
négligé leur formation physique, intellectuelle et professionnelle. Au lieu
de cela les esclaves vendus au marché étaient inspectés par un médecin,
garantis par le vendeur, et immédiatement rentables. Il était donc économiquement
préférable d'acquérir de grands enfants déjà élevés par
d'autres plutôt que de les produire soi-même. Les maîtres avaient donc
en général intérêt à empêcher les esclaves d'avoir des relations sexuelles :
les quartiers des femmes étaient séparés de ceux des hommes et soigneusement
verrouillés pour qu'ils ne puissent pas se rencontrer. Quant aux
prostituées, les maîtres les faisaient avorter lorsqu'elles étaient enceintes.

Si un maître donnait à l'un de ses esclaves l'exclusivité sexuelle de
l'une de ses esclaves et s'il lui permettait d'élever les enfants qui naissaient
de leurs rapports, c'est qu'il voulait le récompenser et le motiver.
C'est que c'était un esclave de confiance. Pourtant ce dernier ne possédait
aucun des enfants qu'il avait engendrés. S'il était affranchi il lui faudrait
encore les racheter à son maître.

\subsection{L'abandon}

L'abandon d'un enfant à sa naissance lui confère plusieurs des
traits de l'esclave : il n'est enregistré nulle part au nombre des citoyens ; il
n'a ni parents ni famille ; il est totalement dépendant de celui qui veut
bien le prendre en charge, qui n'a de comptes à rendre à personne s'il le
maltraite, et qui a le droit d'en faire son esclave. Son corps n'est pas protégé
par la loi : les auteurs de l'Antiquité tardive tiennent pour assuré que
tous les garçons et filles bien conformés abandonnés à la naissance trouvent
preneur et sont élevés pour être prostitués dès qu'ils trouveront des
amateurs, bien avant leur puberté. Dans le même ordre d'idées rien dans
les lois de la République de Rome n'interdit aux mendiants professionnels
de mutiler un enfant abandonné, non citoyen, pour exciter la
pitié des passants : c'est mal vu mais la loi ne s'en mêle pas.

\subsection{La vente par un parent}

\label{vente-parent}

En période de chômage ou de disette, les plus pauvres sont acculés
à vendre leurs enfants ou à mourir de faim avec eux. Certes la vente suspend
(mais ne fait pas disparaître) leurs droits parentaux, mais son produit
procure au parent de quoi manger pendant un certain temps. Quant
aux enfants ainsi vendus, leurs maîtres ont l'obligation de les nourrir, ou
de les revendre, ou de leur rendre leur liberté (ce qui les rend à l'autorité
de leur parent, s'il ne les a pas perdus de vue entre temps). Il semble vraisemblable
que sous la pression de la nécessité, les parents soient souvent
contraints de vendre leurs enfants bien en dessous du montant qu'ils ont
déjà investi dans leur éducation.

Ceci étant dit les ventes d'enfants par leurs parents n'ont pas forcément
une signification unique. Il existe en effet des ventes \emph{temporaires},
des ventes du seul travail de l'enfant, et non de son corps, pour une durée
déterminée, parfois très longue, 15 ans, et même 25 ans. Dans ces cas
la vente a vraisemblablement le sens d'un contrat de travail archaïque, ou
d'un louage de service, ou d'un apprentissage ?

C'étaient évidemment les enfants mal investis, mal protégés, mal
tenus, mal contenus, par des parents pauvres, malades ou psychologiquement
défaillants, qui étaient particulièrement prédestinés par leur histoire
au statut de \emph{mineur à vie sans famille} qu'est le statut d'esclave. C'étaient les
petites filles qui couraient le plus de risque d'être vendues (celles qui
n'avaient pas été abandonnées dès leur naissance). C'étaient celles dont
les pères et mères se séparaient les premières.

\fsc{Testart} soutient%
%[3] 
\footnote{Alain \fsc{Testart} : 2002, p. 176 à 193.} 
que ce sont les mêmes sociétés et les mêmes
groupes sociaux qui acceptent qu'on puisse être réduit en esclavage
pour dettes et que le père de la fiancée s'enrichisse en mariant sa fille.
Quand le \emph{prix de la fiancée}, (versé un peu partout, sous des formes variables,
par le fiancé à la famille de celle-ci) n'est pas contrebalancé par une
dot (versée au couple par le père de la fiancée) d'un montant équivalent
ou supérieur, le père de la fiancée a financièrement intérêt à marier sa fille :

\begin{displayquote}
\emph{Pour admettre que l'on puisse vendre sa fille en esclavage, il faut d'abord admettre
que l'on puisse la vendre en quelque sorte à son mari. Pour admettre que l'on puisse
vendre son fils, il faut d'abord que l'on admette que l'on puisse vendre sa fille. Pour
admettre qu'un père puisse se vendre en esclavage, il faut déjà admettre qu'il puisse
vendre ses enfants. Et pour admettre que l'on puisse réduire en esclavage un membre
libre de sa communauté et l'exploiter comme esclave, il faut d'abord qu'on admette que
l'on puisse le faire pour le plus démuni et le plus fragile d'entre eux}%
%[4]
\footnote{Alain \fsc{Testart} : 2002, p. 193.}%
.

[...] \emph{cette condition -- tenir pour légitime pour un père de tirer profit de ses filles --
nous apparaît comme une condition nécessaire (mais nullement suffisante) pour que
l'on tienne plus généralement pour légitime de réduire en esclavage un des membres de
la société pour des raisons uniquement financières (esclavage pour dettes ou vente)}%
%[5]
\footnote{Alain \fsc{Testart} : 2002, p. 176 à 193.}%
.
\end{displayquote}

\subsection{La vente par soi-même}

Une fois vendus leurs enfants il ne reste aux indigents qu'à se vendre
eux-mêmes. S'il est interdit à Rome comme à Athènes d'asservir un
citoyen, sauf condamnation pénale, sa mise en vente est juridiquement
inattaquable s'il a un âge suffisant pour être considéré comme responsable
de ses actes (vingt ans) et s'il a encaissé au moins une partie du
produit de la vente, ce qui prouve qu'il a été complice de son propre asservissement.
Celui qui se vend désavoue sa propre famille, dont il reconnaît
publiquement l'incapacité à lui apporter du secours. Il coupe
tous ses liens juridiques avec ses parents, son épouse, ses enfants,~etc. Il
démissionne de sa liberté et de ses responsabilités de citoyen. Il ne peut
plus hériter (en général il n'avait guère d'espoir de ce côté-là). S'il meurt
sous les mauvais traitements, ou d'un accident du travail (travaux publics,
mines, gladiateurs,~etc.) sa famille ne peut demander de compensation. Il
ne peut plus se prévaloir de son état de liberté antérieur. Même s'il est un
jour affranchi il gardera à vie la marque servile et ne retrouvera jamais la totalité
de ses droits antérieurs.
Des jeunes gens bien formés et ambitieux mais désargentés sont
amenés à se vendre à un employeur%
%[6] 
\footnote{Cf. P. \fsc{Veyne}, \emph{La société romaine}.}
pour tenir des emplois de confiance.

Quant à ceux qui ne sont pas citoyens, ils font ce qu'ils veulent.
Personne ne se formalise qu'ils soient asservis, de leur plein gré ou non,
et ils peuvent vendre tous leurs enfants si cela est leur intérêt. Dans le
monde régenté par Rome, à cette époque en expansion continue, les
peuples soumis constituent un gisement d'esclaves achetés de manière
tout à fait légale à leurs parents.




% Le 03.03.2015 :
% Antiquité
% Moyen Âge
% ~etc.
% ~\%


\chapter{Les représentations des hommes de l'Antiquité}

\section{Des hommes et des dieux}

À part quelques esprits forts, les grecs et les romains sont convaincus de l'existence d'un monde surnaturel%
%[1]
\footnote{Paul \fsc{VEYNE}, \emph{L'empire gréco-romain}, 2005, notamment le chapitre 8, pages 419 à 544.}%
. Ils l'imaginent sous la forme d'un monde invisible où des dieux innombrables vivent à côté des hommes. À leurs yeux la nature est enchantée, pleine d'entités spirituelles de toutes espèces et imprégnée de sacré, et la communication avec les dieux est chose aisée et banale. Il suffit d'observer avec un minimum d'attention les signes qu'ils fournissent en abondance et de les interpréter avec compétence. Des spécialistes, (augures et haruspices à Rome, devins ailleurs) sont à la disposition des cités et des particuliers pour poser aux dieux des questions et pour interpréter leurs réponses.

 Les relations des dieux de l'Antiquité avec les hommes sont des relations de voisinage et non de parenté. Même si les hommes se doivent de les honorer, les dieux s'intéressent d'abord à leurs propres aventures et à leur propre monde. Ils aiment d'ailleurs les mêmes choses que les hommes. On les séduit donc comme on séduit ces derniers : par l'offrande de fêtes, de spectacles et de banquets. D'où les décorations des temples, les beaux vêtement (le blanc est la couleur liturgique principale) et les processions, les chants et les danses des jeunes filles, les concours de force et d'adresse des jeunes gens, l'offrande de bêtes sans défauts pour les sacrifices. 

 Les principaux cultes sont organisés et financés par les cités, les temples sont construits et entretenus par elles, même si le rôle de l'initiative privée est important (avec les omniprésents \emph{évergètes}%
 %[2]
\footnote{Cf. Paul \fsc{VEYNE}, \emph{Le pain et le cirque}, 1976.}%
 ). L'action politique implique le service des dieux. Même si une pureté rituelle est ordinairement requise pour exercer les fonctions cultuelles (prêtrise, haruspice,~etc.), leurs exigences sont ordinairement faciles à satisfaire. Sauf dans les cultes exotiques importés d'Égypte ou d'Asie (mineure) il n'est pas question de mettre à part des officiants pour en faire un clergé professionnel permanent. Au cours de sa carrière politique (de son \latin{cursus honorum}) chaque notable est appelé à tenir un jour ou l'autre la place du célébrant d'un des cultes autorisés, mais seulement pour un temps, comme pour toutes les autres fonctions civiques. Au même moment chaque père de famille est le prêtre et le sacrificateur de son culte familial. 

 Les festins sont indissociables de la religion civique%
 %[3]
\footnote{... en Grèce du moins. Les viandes non brûlées et non consommées par les prêtres étaient vendues comme viandes de boucherie. Il semble que les anciens, dont le niveau de vie moyen était celui des habitants des pays sous-développés d'aujourd'hui, aient consommé très peu de viande, si bien qu'il se peut que les banquets religieux civiques aient été pour les plus humbles des grecs les seules occasions où ils en mangeaient.}%
. Le plus souvent les non citoyens en sont exclus, et encore plus strictement les esclaves. Lors des fêtes civiques les citoyens célèbrent leur unité en communiant à la chair des victimes pendant que la part des dieux (la graisse) monte vers ces derniers avec la fumée. C'est aussi par des sacrifices, sanglants ou non, et des conduites de repentance qu'on cherche à obtenir des dieux le pardon des fautes commises à leur égard. 

 Mais il ne faut s'abuser ni sur leur puissance, ni sur leur sollicitude, ni sur la constance de leurs intérêts pour les humains. Les dieux ont à peu près autant de respect et de sympathie que les hommes pour les conduites vertueuses, c'est-à-dire qu'il leur arrive de penser à autre chose, ou d'être à l'occasion injustes, menteurs, ou infidèles. On ne peut donc pas compter sur leurs interventions en faveur du droit et de la justice, même s'il n'est pas interdit de garder espoir. Il faut s'attacher leur bienveillance comme on cultive celle d'un voisin puissant. Il faut éviter de les avoir contre soi. Le vrai problème c'est donc de les connaître, de les hiérarchiser et de vivre en bonne entente à côté d'eux sans en vexer aucun, d'où les autels « \emph{à l'ensemble des dieux} », ou bien « \emph{au dieu inconnu} ». 

 Si bien des dieux sont attachés à un lieu, comme la déesse Athéna à Athènes, chaque cité a la liberté de rendre un culte à tous les dieux de son choix, et chacun peut rendre un culte à ses dieux préférés. On pense d'ailleurs souvent que sous des noms différents il s'agit des mêmes dieux. L'idée de liberté religieuse est pourtant incompréhensible puisque chacun croit à tous les dieux ... même à ceux de ses ennemis, et que le monde des dieux ne peut pas être dissocié du monde civique. Chacun sait que la négligence de ses devoirs religieux fait courir des risques à la cité, et qu'il n'est donc pas question de l'accepter. En ce sens-là on pourrait dire qu'il n'existe qu'une seule religion, à l'intérieur de laquelle toutes les dévotions individuelles sont possibles et compatibles. Si les dieux des vainqueurs sont les plus forts, ce que leur victoire démontre, il n'en est pas moins absurde et imprudent de prétendre que ceux des vaincus n'ont pas ou plus d'importance. C'est pourquoi les dieux d'une cité vaincue sont respectueusement invités à déménager chez les vainqueurs. 

 D'où viennent les dieux ? Ils n'ont pas créé le monde puisqu'ils ont eux-mêmes été créés. Ils en font partie au même titre que les humains, même s'ils sont plus puissants, et immortels. Quant à l'origine du monde le champ des spéculations est ouvert aux philosophes. Beaucoup comme Platon ou Aristote postulent qu'il existe un être créateur de tout ce qui existe \emph{(premier moteur)}, ce qui ne dit rien de ses autres attributs. D'autres pensent que la matière est éternelle et incréée. Mais en dehors du cercle des philosophes ces idées ont peu d'influence. 

 Les anciens grecs et romains ne doutent pas de l'existence d'une âme individuelle (au moins une par personne, souvent plusieurs). Ils croient en une survie, au moins partielle, de cette âme. Ils croient en l'existence d'un autre monde pour les morts, ordinairement localisé sous terre : les Enfers, l'Hadès,~etc. Le destin des âmes des morts est misérable quelle qu'ait été leur valeur humaine : ce ne sont plus que des ombres qui se traînent mélancoliquement dans les Enfers, leur séjour lugubre où n'entre pas le soleil, et où ils peuvent finir par se dissoudre. S'il existe aussi une forte tendance populaire à vouloir que les méchants subissent après leur mort des punitions appropriées à leurs forfaits, cette tendance n'emporte pas la conviction : c'est peut-être prendre ses désirs pour la réalité.

 Les anciens croient que les défunts peuvent au même titre que les dieux exercer une certaine influence sur les vivants et leur destin. Ils croient aussi que la qualité de la survie des âmes des morts dépend des soins que prennent leurs descendants à les honorer, d'où l'importance du culte qui leur est rendu. Quand ils leur font des libations et leur offrent de la nourriture ou des sacrifices cela les ranime un peu, et diminue d'autant leur envie de nuire à ceux qui jouissent encore de la vie. Toute famille pieuse garde donc la mémoire de ses ancêtres sur plusieurs générations. Pour éviter leur courroux et une éventuelle punition individuelle qui retentirait sur la collectivité il convient que chacun prenne soin de ses morts et des dieux de sa maison (dieux \latin{lares} attachés au foyer, et plus précisément à l'âtre où se faisait la cuisine). La cité exerce donc un droit de regard sur l'exercice du culte domestique et prend elle-même en charge celui des familles disparues sans héritier. La négligence de ces devoirs familiaux et religieux (qui constitue \emph{l'impiété}) est sanctionnée par la loi%
%[4]
\footnote{À côté de la religion traditionnelle, civique et donc obligatoire, existaient des \emph{cultes à mystère} dans lesquels l'investissement individuel et l'implication affective étaient au premier plan. Concernant l'au-delà de la mort ils promouvaient des perspectives plus exaltantes que les enfers lugubres de la religion traditionnelle. Ils n'étaient pas vécus comme incompatibles avec cette dernière et n'exigeaient pas de profession de foi exclusive.}%
. 


\section{La vie bonne}

 À l'exception des stoïciens et des juifs, les hommes de l'Antiquité sont unanimes dans leur mépris pour le travail d'exécution, et d'abord pour le travail des mains, celui qui « sent la sueur ». L'opinion des plus grands philosophes grecs, au premier rang desquels Platon et Aristote, est que l'on ne peut pas être un homme de bien, c'est-à-dire cultiver la vertu, si l'on travaille de ses propres mains. Ce travail est digne des seuls esclaves, et disqualifie ceux qui le font. Même les artistes les plus admirés ne sont que des artisans, des gens sans dignité, à peine plus estimables que les tâcherons sans qualification. N'importe quel propriétaire est plus noble, du fait qu'il dirige le travail des autres.

 Les sociétés antiques apprécient la richesse sans mauvaise conscience. La fortune matérielle est la preuve qu'on est béni des dieux. De là à penser que la richesse prouve la valeur personnelle il n'y a qu'un pas. L'idéal de l'homme antique est de vivre de ses rentes. Cet idéal n'interdit d'ailleurs aucunement de faire des affaires, de combiner des coups financiers, de prêter de l'argent ou de spéculer. Mais seule la possession de la terre donne droit à l'exercice plénier des droits du citoyen. Elle est le signe du lien charnel à la patrie, à la terre des pères, à la terre mère de la patrie, d'où l'interdiction faite aux étrangers et aux non citoyens de l'acheter ou d'en hériter. Celui qui possède un domaine peut à volonté se couper des circuits de l'échange et vivre en autarcie des seuls produits de sa terre. Il est le plus libre des hommes et donc le plus digne de confiance. Seuls les riches accèdent aux honneurs civiques, en fonction du niveau de leur fortune : cela ne suffit pas, mais c'est une condition nécessaire%
% [5]
\footnote{Cf. \fsc{VEYNE}, \emph{Le pain et le cirque}.}%
.

 Tout citoyen se doit de soutenir son rang noblement, en particulier s'il veut jouer un rôle politique. L'exercice d'une autre profession que celle de propriétaire terrien fait déroger, sauf le métier d'avocat, de médecin (dont la promotion a été lente et progressive, à Rome les premiers d'entre eux étaient des esclaves grecs), de professeur (enseignement secondaire et supérieur), d'artiste (ceux qui ont réussi : peintre, sculpteur, architecte, mais pas acteur ou chanteur), et les activités de banquier, armateur ou commerçant (à la condition qu'ils soient riches). Celui qui exerce une profession \emph{libérale}, une profession qui ne fait pas déroger, par opposition aux métiers serviles, est censé fournir gratuitement ses services à ses amis et relations, qui sont censés lui témoigner leur reconnaissance par des cadeaux « spontanés » qu'on appelle \emph{honoraires}. Cela dit s'il atteint à la reconnaissance sociale c'est moins pour ses compétences techniques que pour les richesses qu'elles lui procurent. 

 Le statut de travailleur libre, \emph{mercenaire}, c'est-à-dire salarié, existe depuis la plus haute Antiquité. Bien que non infâme, ce statut n'est pas plus honorable ni plus désirable que celui d'un manœuvre d'aujourd'hui. Celui à qui son manque de fortune interdit de vivre du travail des autres et qui ne dispose pas de compétences particulières, doit se louer pour gagner son pain. Son activité concrète n'est pas différente de celle des esclaves. C'est donc un métier \emph{servile} qui le dévalorise. Le fait qu'il choisisse son maître et ne lui soit pas lié au-delà de son contrat est la preuve de sa liberté, mais il n'en reste pas moins un pauvre parmi les pauvres. 


\section{Piété et solidarité familiale}

 Dans un monde sans systèmes d'assurance, sans assistance mutuelle et sans retraites, l'individu est solidaire du groupe auquel il appartient. Seule la famille est tenue de fournir de l'aide à ses membres : tous les membres d'une famille se doivent mutuellement secours et assistance. Cela fait partie de la \emph{piété filiale}, dont nul ne peut s'estimer quitte sauf les enfants émancipés, abandonnés ou vendus comme esclaves. Cela explique la soumission des enfants à leurs parents : ils ne peuvent compter que sur ces derniers pour défendre leurs droits personnels, pour subvenir à leurs besoins et pour les établir dans la vie. Chez les romains les gains de tout enfant, quel que soit son âge, reviennent à son \latin{pater familias} jusqu'à la mort de ce dernier. Les plus pauvres ne possèdent aucun capital et ne vivent que du travail de leurs propres mains. Leurs enfants (\latin{proles} : d'où leur nom de \emph{prolétaires}) sont leur seule richesse. Ils ont le droit d'exiger (si nécessaire par voie de justice) que les enfants ou petits-enfants \emph{qu'ils ont élevés et établis} les prennent matériellement en charge dès qu'ils sont assez grands pour travailler, et surtout quand ils sont eux-mêmes trop âgés et trop faibles. Ce n'est qu'à cette condition qu'ils peuvent vieillir dignement. 

 Dans le monde des républiques gréco-romaines on ne peut pas parler d'assistance au sens où nous l'entendons aujourd'hui. L'hospitalité est un devoir impérieux, mais il va sans dire que c'est à charge de revanche. Quant à ceux qui ne peuvent \emph{rendre} aucun service (malades, infirmes, vieillards,~etc.) ils ne peuvent compter que sur la pitié du public, dans un monde où celle-ci n'est pas plus valorisée que dans le nôtre.

 Personne ne soutient sérieusement qu'un pauvre sans ressources et sans travail est fondé à revendiquer des secours au nom du seul fait qu'il est un homme. S'il est citoyen, et seulement dans ce cas, on admet qu'il a un droit moral à demander des secours, mais à sa seule cité. Celle-ci a en effet intérêt à conserver ses citoyens pauvres, eux et leurs enfants légitimes qui sont mobilisables comme soldats, plutôt qu'à les voir soustraits à sa juridiction s'ils sont asservis. De ce fait les cités se sentent concernées par les difficultés de leurs membres et en cas de disette tentent de fournir des aliments à leurs concitoyens affamés. Les pauvres peuvent aussi rechercher la protection et le soutien des riches, que ceux-ci soient motivés par leur bon cœur ou par la crainte du scandale que provoquerait leur indifférence. Ce soutien est institutionnalisé dans la pratique du \emph{patronage}. Par ailleurs \emph{l'évergétisme} désamorce l'envie des pauvres en obligeant les citoyens riches à financer les budgets des cités et à offrir de somptueux cadeaux à leurs concitoyens moins fortunés pour obtenir leurs votes.

 Par contre les étrangers domiciliés n'ont aucun recours de ce genre à espérer. Au mieux ils peuvent compter sur ceux de leurs propres concitoyens qui vivent dans la même cité, d'où l'importance des associations d'étrangers. S'ils ne payent pas une mensualité de leur taxe de séjour les métèques d'Athènes sont vendus comme esclaves, même s'ils l'ont acquittée scrupuleusement pendant des années.

 En bien des cités, dont Rome, des médecins et des enseignants sont recrutés et payés par la cité et soignent ou enseignent : mais si l'état de besoin donne droit à ces services, il est réservé aux citoyens. 

 Quand les légionnaires romains ont achevé leur engagement ils reçoivent de l'argent ou des terres. Il s'agit là de rémunérer les services rendus par un capital retraite. Si les enfants d'un citoyen mort au combat sont élevés aux frais de la collectivité, c'est parce que l'ensemble des citoyens survivants a contracté une dette envers leur père, dette que la cité honore ainsi, ne serait-ce que pour que les futurs pères de famille continuent d'accepter de mourir pour elle. Dans un état organisé qui se respecte les militaires en retraite et les orphelins de guerre ne peuvent pas être confondus avec les indigents, même s'ils vivent dans une certaine austérité : ils personnalisent en effet la vertu, le courage et le dévouement à la cité. Ils sont donc particulièrement honorables, au contraire des métèques, des indigents et des vagabonds, et il ne s'agit pas d'assistance.

 Il serait pourtant injuste d'oublier les institutions vouées aux malades, établies %constituées 
autour d'un temple aux dieux de la médecine (\latin{aesculapium} en grec ou \latin{valetudinarium} en latin) qui sont accessibles à tous, esclaves compris, mais qui tiennent plus du lieu de pélerinage que de l'hôpital. 

 Quant aux infirmeries et hôpitaux fonctionnant au sein des unités militaires ou des entreprises utilisant de nombreux esclaves (jusqu'à \nombre{500} lits), ce ne sont pas des institutions d'assistance, mais des outils au service d'une bonne gestion d'un capital humain précieux. Ils n'en fournissent pas moins à leurs médecins (qui ont très longtemps été des esclaves formés sur le tas, comme les autres artisans) un cadre adapté à la réalisation d'observations systématiques et à l'acquisition de compétences professionnelles.


\section{Morale d'esclave}

 Du point de vue des citoyens des cités antiques l'esclavage est regrettable pour ceux qui y sont assujettis, pour ceux que les dieux n'ont pas protégé d'un destin si funeste, mais ils voient de leurs propres yeux les bénéfices très concrets qu'ils en tirent. Le miracle économique et culturel grec reposait d'abord sur l'efficacité avec laquelle les libres citoyens des cités grecques avaient su durablement asservir leurs voisins et les mettre au travail forcé dans les industries lourdes de l'époque. Cela leur avait permis sans effort financier exorbitant de s'équiper des matériels militaires les plus coûteux et les plus performants, et leurs guerres leur avaient fourni de nouveaux esclaves : le travail des esclaves payait l'achat de nouveaux esclaves. 

 Un monde sans esclaves n'est alors pas pensable : si l'on veut éviter de travailler soi-même, ce qui selon les philosophes est tout de même la moindre des choses, il faut bien se résoudre à asservir d'autres hommes. Pour Aristote il est nécessaire, il est inévitable, et donc conforme à la nature, qu'il y ait des personnes faites pour commander et d'autres faites pour obéir. Il convient seulement de savoir qui peut légitimement être soumis par la violence, et qui ne doit pas l'être. Il s'agit de désigner correctement ceux à qui une valeur égale à soi-même peut à bon droit être déniée. 

 Quels sont ceux qui ont vocation à être esclaves ? Les anciens pensent qu'il n'est finalement pas injuste, qu'il est même légitime de traiter comme des esclaves ceux qui ont assez de mépris pour eux-mêmes pour se reconnaître comme tels. Ils considèrent que l'asservissement ne peut être regardé comme moins grave que la mort que par ceux qui tiennent d'abord à la vie, c'est-à-dire à peu près tout le monde sauf les aristocrates. Les autres, les aristocrates et ceux qui méritent de l'être, ne se laissent pas réduire en esclavage : ils meurent au combat, ils s'évadent, ils se suicident, ils se laissent mourir de faim. Ceux qui acceptent de survivre à leur asservissement signent par là leur renoncement au statut d'homme libre, au droit à la parole. 

 L'existence de l'esclavage n'est pas perçue comme un scandale. D'ailleurs les esclaves peuvent acheter des esclaves, même si légalement ces derniers appartiennent à leur propre maître, comme tout ce dont celui-ci leur laisse la jouissance. Même les esclaves révoltés n'espèrent rien de mieux que de posséder des esclaves à leur tour. C'est qu'il y a pire : vaincus égorgés sur le champ de bataille ; cités entièrement passées au fil de l'épée, femmes et enfants compris ; chômeurs qui meurent réellement de faim ; infirmes et malades (esclaves ou libres) abandonnés dans les temples aux bons soins des dieux, c'est-à-dire à la pitié du public ; enfants chétifs ou mal formés abandonnés sans aucune chance d'être accueillis par quiconque ; enfants non désirés tués à la naissance,~etc.

 En se soumettant à leur maître les esclaves montrent qu'ils tiennent plus à la vie qu'à la liberté et à l'honneur, ce qui implique qu'ils n'ont pas de volonté propre ni de courage. C'est pourquoi leur parole compte pour rien devant la justice. La morale qu'on leur accorde se résume en peu de mots : puisque leur maître est tout-puissant, leur devoir se borne à faire tout ce qu'il veut : \emph{il n'y a pas de honte à faire ce que le maître commande}. C'est une dénégation puisqu'ils sont justement affectés aux prestations qu'il est honteux à tout autre de fournir. Ils doivent se laisser utiliser par autrui comme un moyen, et surtout ne jamais montrer de fierté : chez eux c'est de l'insolence, une arrogance insupportable, et ils reçoivent des coups s'ils s'y risquent. C'est de la docilité qu'on attend d'eux, ou plus exactement de la \emph{servilité}. C'est pour elle qu'on les a achetés ou élevés. 

 Le paradoxe c'est évidemment qu'on les méprise justement à cause de cette servilité. On exige d'elles et d'eux une soumission qui est infamante en elle-même%
% [9]
\footnote{\emph{Les Bas-Fonds de l'Antiquité},
Catherine \fsc{SALLES}, 1982.
 
  \emph{La société romaine}, Paul \fsc{VEYNE}, 1991. 
  
  \emph{La contamination spirituelle, science, droit et religion dans l'Antiquité}, Aline \fsc{ROUSSELLE}, 1998.}% 
. Ils sont donc dans une position morale impossible : ils ne peuvent survivre que s'ils acceptent de faire ce qui est une honte pour tous leurs contemporains. Ce n'est que s'ils mouraient héroïquement qu'on leur reconnaîtrait avec étonnement, mais un peu tard, le courage et la dignité d'hommes (de femmes) véritables : \emph{un homme libre dans un corps d'esclave}. 


\section{Vertu virile}

 \latin{Virtus} vient de \latin{vir} qui désigne l'homme par opposition à la femme, le mâle. Ce mot signifie « vertu, courage, force ». La vertu antique s'identifie à la force, au courage physique et à la maîtrise de soi. C'est le contraire de la faiblesse, de la passivité, de la mollesse et de la lascivité, tous traits de caractère prêtés aux femmes, et par extension aux homosexuels passifs ou aux esclaves. 

 Le maître d'une \latin{familia}, qui vit constamment au milieu d'esclaves plus ou moins dociles, et parfois très nombreux, ne doit pas craindre de s'imposer, de faire peur et de faire mal quand il le juge nécessaire, ni de faire couler le sang. En un mot il doit être ce que l'argot des entreprises d'aujourd'hui appelle un « tueur ». Dans ce monde c'est la force des forts et leur mépris de la mort (celle des autres et la leur) qui légitiment l'asservissement des moins forts, d'où l'importance « pédagogique » des jeux du cirque, de la mise à mort de l'autre comme spectacle%
% [6]
\footnote{\emph{À Pompéi, sur une fresque illustrant l'accomplissement de tâches domestiques et artisanales, l'artiste a substitué aux vilains esclaves qui les remplissent habituellement de doux et souriants angelots. Les Pompéiens vivaient un mythe où tout leur paraissait venir du ciel en gratification naturelle et méritée de leur raffinement obtus. Comme toutes les exploitations, l'esclavage ne conduit pas qu'à l'aliénation des exploités, mais aussi à celle des exploiteurs. Il conduit à la négation de l'humanité des hommes et des femmes, à leur mépris et à la haine. Il incite au racisme, à l'arbitraire, aux sévices et aux meurtres purificateurs, armes caractéristiques des guerres de classe les plus cruelles. Si tant est que l'esclavage ait contribué à un quelconque progrès matériel, il nous a aussi légué comme maîtres à penser des philosophes et des politiques dont la conscience était le produit de cet aveuglement et de ces préjugés. N'est-ce pas parce qu'elle s'est communiquée jusqu'à nous, portée par une culture indiscutée et ininterrompue d'exploiteurs, que leur aliénation nous demeure toujours imperceptible et nous donne encore pour humanistes des sociétés construites sur le saccage de l'homme.} \fsc{MEILLASSOUX}, \emph{Anthropologie de l'esclavage} (1986, 1998, p. 321).}%
.

 La dureté et l'insensibilité sont les qualités les plus nécessaires aux chasseurs d'esclaves. Il ne faut pas qu'ils lésinent sur les forces à mettre en œuvre s'ils veulent capturer leur « gibier » sans l'abîmer ni le déprécier, sans qu'il ait le temps et la possibilité de se blesser lui-même pendant la capture, ni celle de se suicider dans les moments de désespoir qui suivent : il leur faut livrer leurs proies aux « consommateurs » en suffisamment bon état pour les vendre avec profit. Et il faut « assouplir » leurs capacités de résistance psychologique, ce qui ne se fait pas au moyen de caresses. Afin que la victime consente à s'identifier \emph{servilement} aux désirs d'un maître il convient d'induire en elle un niveau suffisant de dépersonnalisation et de sidération des défenses, à coup d'humiliations et de sévices terrorisants. 

 La chasse aux esclaves est un sport qui demande de ne pas avoir froid aux yeux, c'est un « travail d'homme » qui permet à ceux qui s'y livrent de faire la preuve de leur virilité, telle qu'on la conçoit alors. Elle leur permet de faire la preuve de leur fécondité. En effet ceux qui savent comment fabriquer des esclaves augmentent le nombre des individus qui dépendent d'eux, et donc leur propre poids social, sans avoir à en passer pour ce faire par la volonté d'un beau-père et le sexe d'une épouse. 

 En Grèce et à Rome face aux femmes, aux enfants et aux esclaves des deux sexes, définis par leur soumission et leur passivité, l'homme se définit par son activité. Dans ce système de représentations la fidélité du mari à son épouse n'a pas de sens. Elle ne regarde que lui. Il est en droit de compter sur l'obéissance sans limite de ses esclaves et sur la complaisance de ses affranchis et de leurs épouses. Rien ne lui interdit sexuellement les impubères, du moment qu'ils lui appartiennent. Seuls lui sont interdits ses propres enfants et les membres trop proches de sa lignée, ainsi que toutes les personnes qui dépendent d'un autre citoyen : leurs enfants, leurs épouses ou leurs esclaves.

 Ce que l'esprit du temps trouve le plus répugnant, le plus avilissant, c'est qu'un homme libre de tout lien de dépendance -- ni esclave, ni affranchi -- se mette au service du plaisir sexuel de quelqu'un d'autre, homme ou femme, soit qu'il pratique le cunnilingus, soit qu'il subisse de son plein gré « \emph{ce que l'on fait aux femmes} » : des rapports homosexuels passifs%
% [7]
\footnote{\fsc{VEYNE}, 1991; \fsc{ROUSSELLE}, 1998. En Grèce tout homme fait qui avait été victime d'un viol, par exemple pendant qu'il était prisonnier de guerre, perdait à tout jamais ses droits civiques. La passivité sexuelle, même subie et non assumée, n'était chez eux admise que jusqu'à ce que la barbe ait poussé. À Rome par contre une fois libéré celui qui avait subi un viol pendant sa captivité retrouvait l'intégralité de ses droits civiques.}%
. Le soupçon qu'un citoyen ait du goût pour ces pratiques est gravissime. Si la preuve en est faite il perd son droit à la parole publique et devient infâme avec toutes les conséquences légales liées à ce statut. Les armées romaines prohibent vigoureusement ces comportements chez leurs membres. Une telle accusation peut pousser au suicide. Par contre les pratiques homosexuelles masculines \emph{actives} ne soulèvent aucune objection morale, pas plus d'ailleurs que les pratiques homosexuelles des femmes. En ce qui concerne ces dernières c'est parce que ce qu'elles peuvent faire n'a aucune importance sociale tant qu'aucun homme n'y est impliqué. Il s'agit donc d'une morale \emph{machiste} sans nuances ni aménagements.

 L'infamie est une peine accessoire entraînée par certaines condamnations judiciaires \emph{(peines infamantes)}. C'est aussi la sanction automatique de certains actes et de certaines activités : le fait de coucher avec la femme d'autrui, de choisir volontairement de subir « ce que l'on fait aux femmes », de travailler dans le monde du spectacle, des jeux du cirque, ou dans le monde de la prostitution. Le service d'autrui en tant que tel fait problème. Se mettre au service du plaisir d'autrui est honteux, particulièrement pour les hommes. Sans même parler des proxénètes et des organisateurs de spectacles \latin{(leno)}, regardés avec un mépris sans nuances, les acteurs, même célèbres et adulés, sont aussi infâmes que les prostitué(e)s libres et les gladiateurs libres.

 L'infamie retire à l'intéressé(e) ses droits civiques et parentaux, invalide (rompt) son mariage, lui interdit de passer contrat et de se marier à l'avenir. Aux hommes elle interdit de prendre la parole dans une assemblée politique, de porter plainte comme de plaider pour un autre qu'eux-mêmes (comme avocat), d'être magistrats, d'exercer les prérogatives d'un \latin{pater familias}, d'un patron, ou la tutelle d'un mineur. De ce fait les esclaves affranchi(e)s par un infâme sont libres des obligations légales des clients, auxquelles tous les autres affranchis sont assujettis.

 Les affranchies qui ont été prostituées par leur maître pendant leur servitude mais qui ne se prostituent plus depuis leur affranchissement échappent à l'infamie, et elles n'ont pas d'obligation face à leur ancien proxénète. Ce n'est pas le cas des hommes affranchis qui ont été gladiateurs ou prostitués (ou au service d'un proxénète comme tenancier ou employé de maison close) avant leur affranchissement : ceux-là demeurent personnellement infâmes quoi qu'ils fassent, même retirés des jeux sous les applaudissements du public (pourquoi sont-ils marqués à vie au contraire de celles-là ?). 

 L'infamie est héréditaire et se transmet aux enfants de l'intéressé(e) nés après l'acte qui l'a entrainée, ou après la condamnation à une peine infamante.

 Le suicide n'est pas l'objet d'un jugement moral de condamnation. Si la vie devient sans intérêt, si le déshonneur est imminent et inévitable, si l'on se retrouve dans une impasse existentielle, quel qu'en soit le motif, alors le suicide est une porte de sortie honorable. Dans ce cas les médecins se font un devoir de fournir une aide compétente. 


\section{Pudeur féminine}

 Pour les philosophes et les médecins gréco-romains les femmes sont des hommes incomplets, des hommes ratés%
% [8]
\footnote{Ce point de vue a été universel et indiscuté jusqu'au \siecle{18}. Georges \fsc{DUBY}, Michelle \fsc{PERROT}, \emph{Histoire des femmes en occident, I, l'Antiquité}, 2002, chapitre 2 : \emph{philosophies du genre}, Giulia \fsc{SISSA}, p. 83-124.}%
 : elles sont définies par le manque. Elles sont \emph{celles qui n'ont pas} (d'où leur désir pour les hommes et ce dont ils sont pourvus).

 Les veuves sans famille, sans fortune et chargées de petits enfants (orphelins de père), les épouses répudiées, les concubines abandonnées sont l'archétype de la faiblesse, de l'impuissance et de la pauvreté quand elles sont âgées et sans enfants. 

 Infâmes sont les femmes \emph{de mauvaise vie} : les prostituées libres, les comédiennes, les danseuses, chanteuses, musiciennes,~etc. Les femmes convaincues d'adultère sont elles aussi condamnées à l'infamie. À Rome aucune de ces femmes n'a droit à la protection légale de leur corps qu'offre le \emph{manteau des matrones}. Elles portent une toge comme les hommes, ou tout autre vêtement de leur choix, et elles sont punies si elles osent porter le manteau des matrones. Qu'elles le veuillent ou non leur vêtement affiche la disponibilité de leur corps. Les femmes esclaves non plus n'ont pas droit au manteau des matrones, ni au vêtement des enfants et adolescents \emph{de famille}.

 Même citoyennes, les femmes n'ont pas de liberté de choix : que les hommes dont elles dépendent les marient ou les démarient, que leurs époux les répudient ou leur refusent le divorce, de toute manière elles font ce qu'on leur dit de faire. Qu'elles aient une vie sexuelle ou que cela leur soit refusé ne dépend pas d'elles. Il en est de même pour le fait de conserver leurs enfants ou de les abandonner, d'avorter ou de ne pas avorter. Leur corps n'est pas à elle. Elles se doivent d'être dociles, soumises et dévouées aux objectifs de leur maître du moment. La notion de viol marital n'a aucun sens : elles ne peuvent être violées par leurs maris puisqu'ils sont maîtres de leurs corps.

 Si elles subissent un viol on s'attend à ce qu'elles se suicident. Ce n'est pas une obligation légale, mais c'est une sorte d'obligation morale. L'histoire (ou la légende) de Lucrèce (vers 500 avant J.-C.) donne l'exemple d'une conduite digne d'être admirée chez une femme mariée qui a été violée :

 \textsl{... Ils trouvèrent Lucrèce assise dans sa chambre, accablée de chagrin. Elle se mit à pleurer en voyant arriver les siens. Son mari lui demanda si elle était souffrante : « Oui, répondit-elle ; comment une femme qui a perdu son honneur pourrait-elle bien se porter ? Un homme (...) a souillé ta couche ; on m'a fait violence, mais mon cœur est resté pur : ma mort en fournira la preuve. Prenez ma main et jurez de punir mon déshonneur. Sextus Tarquin m'a fait violence ; il est venu la nuit dernière avec une arme, non comme un hôte mais comme un ennemi et il est reparti après avoir pris un plaisir dont je meurs et dont il mourra aussi si vous êtes des hommes ». Ils promirent tous, l'un après l'autre. Ils cherchèrent à apaiser son tourment, affirmant que le coupable n'était pas la victime mais l'auteur de l'attentat ; c'était l'intention et non l'acte qui constituait la faute. « Fixez vous-mêmes le prix qu'il doit payer ; pour moi, bien qu'innocente, je ne m'estime pas quitte de la mort. Jamais une femme ne s'autorisera de l'exemple de Lucrèce pour survivre à son déshonneur. » Elle plongea dans son cœur un couteau qu'elle tenait caché sous son vêtement et tomba sous le coup, mourante ...} (\fsc{Tite-Live}, \emph{Histoire romaine}, Livre I, trad. Annette \fsc{Flobert}).

 Ce récit définit l'honneur comme le fait de n'avoir subi dans son corps aucune effraction non autorisée par une autorité légitime (son \latin{dominus} pour les femmes). Les victimes féminines d'un viol qui, au contraire de Lucrèce, choisissent de ne pas se suicider (les plus nombreuses ?) courent le risque d'être considérées comme adultères (si elles n'ont pas crié et ne se sont pas débattues avec assez de force) ou comme accessibles à tous les hommes comme le sont les filles publiques. Au contraire celles qui se suicident restaurent activement leur honneur perdu, au prix d'un sacrifice dont elles sont le sacrificateur et l'offrande. En protestant par leur acte de leur refus sans concession de l'offense subie {\emph{(plutôt mourir que de vivre dans un corps ainsi souillé)}} elles se mettent définitivement à l'abri des récriminations et autres représailles futures de leurs proches, et les mettent en demeure de les venger, alors que les proxénètes font tout pour que leurs protégées ne se plaignent pas des sévices subis dans l'exercice de leur métier, et acceptent de continuer de se prostituer. 
 
En ce qui concerne les hommes adultes victimes de viol le suicide ne restaurerait pas l'image perdue. Seule la vengeance à l'encontre du violeur, autrement dit sa mort, serait suffisante. 




% Le 9 mars 2015 :
% ~etc.
% ~\%
% Antiquité
% Moyen Âge
% _, --> ,
% Grec, Romain


\chapter{L'exception juive}

 Les Romains nommaient \emph{juifs} (originaires de \emph{Judée}) ceux que les Grecs nommaient \emph{hébreux}%
\footnote{Ils seront présentés ici tels qu'ils étaient après leur intégration à l'Empire romain, mais alors que le Temple de Jérusalem était encore en fonction (avant l'an 70 de notre ère).
Sources : André \fsc{CHOURAQUI}, \emph{La vie quotidienne des hommes de la Bible}, 1978. A.~\fsc{COHEN}, \emph{Le Talmud, exposé synthétique du Talmud et de l'enseignement des rabbins sur l'éthique, la religion les coutumes et la jurisprudence}, 1980. Alain \fsc{STEINSALTZ}, \emph{Introduction au Talmud}, 1976, Paris 2002. Collectif, \emph{Aux origines du christianisme}, 2000. André-Marie \fsc{GERARD}, \emph{Dictionnaire de la Bible}, 1989. Alan \fsc{UNTERMAN}, \emph{Dictionnaire du judaïsme, Histoires, mythes, traditions}, 1997.}%
.

 Aux alentours du début de notre ère, et jusqu'aux deux « guerres juives » (\hbox{66-70} et \hbox{132-135}), il y avait un fort noyau de juifs en Judée et en Galilée : ils représentaient semble-t-il la majorité des habitants (2 millions ?) de ces territoires. Mais la plupart d'entre eux étaient dispersés sur toute la terre alors connue : c'était la \latin{diaspora} (la « dispersion »). Toutes les grandes villes antiques abritaient des communautés juives, parfois plus nombreuses que la population de Jérusalem (Alexandrie, Antioche...)%
\footnote{Il semble qu'au moins un habitant sur dix de l'Empire romain était juif : de 6 à 8 millions pour un nombre total compris entre 50 et 60 millions ? Dans la partie orientale de la méditerranée c'était un habitant sur cinq ?}%
. Et il existait aussi des communautés juives en dehors du monde romain (Mésopotamie, Arabie, Éthiopie,~etc.).

 C'est souvent comme esclaves que les juifs avaient voyagé vers la terre de leur exil ou de leur déportation. Cela avait commencé dès 722 avant J.-C., avec la fin du Royaume du nord, celui des 10 tribus d'israel écrasées par les Assyriens. Cela avait continué en 586 avant notre ère, avec la destruction du Royaume du sud, celui de Juda et du premier Temple de Jérusalem, et la déportation de ses élites vers Babylone. Cela s'était poursuivi avec la guerre des Macchabées (deuxième siècle avant notre ère). 
 
 Ces tribulations culmineront avec la grande révolte de l'an 70 de notre ère, avec la destruction du Temple d'Hérode et la déportation des survivants de Jérusalem. En l'an 135 la diaspora deviendra l'unique lieu de vie des juifs, qu'ils soient descendants d'immigrés ou de déportés, originaires de Judée ou de Galilée, ou bien qu'eux ou leurs pères se soient convertis. 


\section{Un dieu exclusif}

 Les juifs étaient un cas particulier dans le paysage religieux de l'empire romain. Ils se désignaient comme le peuple de YHWH, leur dieu. Par respect pour sa transcendance ils évitaient de prononcer son nom, et le désignaient par un autre mot (\emph{le Nom, le Seigneur, le Saint,} etc...). Ils se définissaient aussi comme le peuple de la \emph{Tora}. Au sens large celle-ci est une collection d'écrits (\emph{ta biblia} en Grec : les livres), de récits de plus ou moins grande ancienneté, et au sens strict ce sont les cinq livres du \emph{Pentateuque}, attribués à Moïse\footnote{La \emph{Tora} recouvre pour l'essentiel ce que les chrétiens appellent \emph{l'Ancien Testament} («l'ancien témoignage »). La plupart de ces textes sont très largement antérieurs à notre ère, mais la date de mise par écrit de chacun d'eux fait encore l'objet de débats entre spécialistes : du huitième siècle avant notre ère au dernier ? La \emph{Septante} est la version grecque de la Tora, traduite pour les juifs de la diaspora grecque aux troisième et second siècles avant notre ère.}%. 
 
 
 
 La déportation de l'élite du peuple à Babylone à partir de 586 avant notre ère, et l'interruption forcée des sacrifices à Jérusalem, avaient provoqué une révolution dans leur pensée et dans leurs pratiques religieuses. S'ils s'étaient crus autorisés à se construire des temples à Babylone, comme le faisaient les autres déportés qu'ils y côtoyaient, peut-être auraient-ils oublié qu'ils étaient des exilés, peut-être auraient-ils fini par s'y sentir chez eux, satisfaits d'être l'une des minorités influentes d'un riche et puissant Empire ? Mais à leurs yeux les sacrifices à leur dieu ne pouvaient être valides qu'à Jérusalem. Face à leur situation de déportés ils avaient refusé l'interprétation qui pour tous leurs contemporains était évidente. Ils avaient refusé de voir dans la ruine du Temple de Jérusalem et leur exil la défaite et l'assujettissement de leur dieu national par les dieux de Babylone. Contre le sens commun (antique) ils avaient choisi de comprendre ces tribulations comme une punition à eux infligée en raison de leurs infidélités. Ils avaient dit que le Seigneur s'était servi des étrangers pour punir Israël, ce qui prouvait une fois de plus qu'il était plus puissant que tous leurs dieux. Ils avaient choisi de croire que leur retour vers YHWH se traduirait par la fin de leur déportation, et l'histoire avait conforté leur croyance. 
 
 La particularité des juifs n'était pas qu'ils affirmaient avoir un dieu particulier, ni qu'ils prétendaient que ce dieu était le plus puissant ou le meilleur de tous. Cela, c'est ce que pensaient \emph{toutes} les cités de l'Antiquité. Ce que croyaient les juifs de la fin de la République romaine c'est que leur dieu était \emph{le seul} dieu existant. Ils ne croyaient pas à l'existence des dieux des autres : pour ce motif ils étaient qualifiés d'athées, («{sans dieux}». Depuis le retour de leur exil à Babylone les dieux des autres peuples n'étaient plus pour eux que des images, des simulacres, des \emph{idoles} impuissantes, derrière lesquelles se cachaient parfois des démons malfaisants. 
 
 Les juifs (et eux seuls) avaient réussi à faire admettre qu'en termes de religion \emph{civique} ils se borneraient à intercéder auprès de YHWH en faveur de l'état romain. Celui-ci leur avait accordé ce privilège quand ils lui avaient fait appel pendant les guerres des Macchabées, parce que la Judée et de la Galilée valaient bien une entorse aux croyances et coutume de Rome. En contrepartie les juifs de la diaspora versaient un impôt spécial, le \latin{fiscus judaïcus}, au Temple de Jérusalem pour payer les sacrifices offerts pour l'État romain, et celui-ci protégeait la collecte et le transport de cet impôt comme une œuvre d'intérêt public. Etre inscrit sur les rôles de cet impôt libérait de l'obligation de participer aux sacrifices publics. 
 
 Le fait que les Romains acceptaient le refus des juifs de sacrifier aux dieux de Rome ne veut pas dire qu'ils appréciaient leur religion ni leurs manières de vivre. Au contraire ils les trouvaient détestables et ne se privaient pas de le dire. Si les citoyens romains qui étaient juifs étaient exemptés du service militaire (c'étaient les seuls), cette exemption n'était pas un privilège mais une exclusion. Leur mode de vie était trop contraire à celui des militaires romains et interdisait leur intégration dans des légions à la vie rythmée par les cérémonies de la religion civique, et leurs règles de pureté interdisaient leur intégration dans une chambrée de \emph{gentils} (c'est-à-dire de membres des \latin{gens}, des autres peuples que le peuple juif). 

 

 




\section{Une approche originale du problème du mal}

 A supposer que le dieu des juifs ait été le seul vrai dieu, il n'était pourtant que trop visible qu'eux-mêmes étaient loin d'être puissants : jamais riches, rarement souverains, et maintes fois écrasés par leurs puissants voisins et déportés. Ceux qui suivaient les prescriptions de la loi de Moïse (les \emph{dix commandements}) n'étaient pas plus épargnés que les autres par les soucis et les malheurs. Des méchants prospéraient avec insolence, tandis que des justes étaient accablés. Comment l'idée d'un dieu unique, créateur, omniscient et bon était-elle crédible face à tout ce mal ? Si vraiment il savait et pouvait tout, pourquoi permettait-il l'injustice ? Pouvait-il exiger des faibles hommes une justice impeccable et ne pas la pratiquer Lui-même ? Quel était l'intérêt de passer alliance avec Lui ? 

 Pour répondre à cette question, peu à peu l'hypothèse s'est faite jour chez les prophètes que si le juste (ou Israël) est accablé de maux et de souffrances, c'est peut-être que le Seigneur le met à l'épreuve, pour tester sa foi et sa détermination. D'autre part (avec le \emph{Second Isaïe} surtout) est apparue l'idée que le juste peut souffrir non seulement du fait des pécheurs, mais \emph{à leur place} :

\begin{displayquote}
{\emph{Le rôle du serviteur de YHWH dans le procès du salut est souligné ici pour la première fois : on sait quel destin eut cette idée dans la littérature religieuse des Juifs et des Chrétiens. La souffrance a une valeur expiatrice, rédemptrice et salvatrice. Telle est la nouvelle réponse que le prophète apporte au problème du mal. Israël, serviteur de YHWH, souffre non pas à cause de ses fautes mais pour expier celles des peuples qui le haïssent, le persécutent et le pillent. Sa souffrance est féconde puisqu'elle fera prendre conscience aux nations des crimes qui les souillent. La grandeur du serviteur se situe ainsi dans son rejet, sa déchéance et sa souffrance ; il accepte de les subir afin d'amener la rédemption du monde.}%
\footnote{\fsc{CHOURAQUI}, 1978, p. 281.}%
}
\end{displayquote}

Le Juste souffrirait donc pour racheter la faute des autres \footnote{...pour la \emph{rédimer}, c'est-à-dire pour payer à leur place leur dette, leur dîme, d'où la notion de \emph{rédemption}.}%
? C'était une interprétation nouvelle des très anciens rites hébraïques du « \emph{bouc émissaire} » et de la « \emph{victime propitiatoire} ». En effet c'était \emph{parce que} la victime, le bouc émissaire, était innocente, qu'elle était en mesure de prendre sur elle et de racheter les fautes des méchants. 

 L'idée que ceux qui souffraient payaient pour les autres faisait appel à une économie de la douleur et des mérites humainement incompréhensible sinon scandaleuse : comment cela pouvait-il se faire ? Selon quelle comptabilité sinistre ou obscène ? Se pouvait-il que le Seigneur jouisse du spectacle de la souffrance humaine ? Ces questions sont au cœur du \emph{Livre de Job}. La réponse de l'auteur de ce drame est que Dieu est si transcendant, si au-dessus et au-delà des capacités de l'homme, qu'il n'est pas possible à celui-ci de comprendre ses intentions. Il est donc vain de lui demander des comptes et de lui faire des reproches. Malgré l'épreuve du mal injustement, absurdement subie, il convient au contraire de lui faire confiance, de continuer de croire en sa bonté. 

 On ne pouvait commencer à trouver humainement cohérente une telle doctrine, ce qui n'est pas la même chose que la comprendre, que si l'on croyait que le Seigneur répondrait par une rétribution \latin{post mortem} à l'injustice subie et à toutes les peines endurées par l'innocent. La plupart des courants du judaïsme des derniers siècles avant J.-C. en étaient venus à croire en la survie des morts, avec un jugement individuel portant sur la totalité des actes de chaque individu, et en conséquence un paradis éternel de jouissance pour les justes, notamment ceux qui mouraient \emph{à cause de} leur fidélité au Seigneur (cf. les \emph{martyrs} du \emph{Deuxième livre des Macchabées}) et une éternité de tourments pour les méchants%
\footnote{La notion d'une survie éternelle des morts ou de leur principe vital (leur « âme »), avec un jugement rétrospectif de la vie du mort, avait été élaborée par les égyptiens bien avant que les juifs n'y adhèrent. Durant les derniers siècles avant notre ère c'était une notion presque universelle dans le monde antique, mais la qualité de vie promise par la plupart des « enfers », ou lieux de survie des « ombres » des morts, laissait encore beaucoup à désirer. Kant soutient que la morale implique logiquement un jugement et une sanction \latin{post mortem} parce que les hommes sont ainsi faits et ne peuvent penser autrement. C'est une exigence interne de la pensée qui en tant que telle ne peut prouver ni qu'il existe un dieu bon et juste  ni qu'il n'existe pas.}.

 Si la solution biblique au problème du mal n'avait rien de raisonnable elle entrainait néanmoins des conséquences positives pour ceux qui souffraient :
\begin{enumerate}

\item étant donné que ni la maladie, le malheur, la souffrance physique, les deuils, les persécutions, ni l'exil, ni la mort n'étaient la preuve que le sujet qui les subissait ou ses ascendants avaient fauté, rien ne permettait de dire qu'ils les avaient mérités (cf. le \emph{Livre de Job}) ;
\item pour protéger la perfection ou la bonté de Dieu il n'était pas nécessaire de dénier l'existence de ces maux en minimisant leur poids ou en les traitant comme des illusions  (le mal restait un mal),
\item ni une santé prospère, ni une vie amoureuse et conjugale réussie, ni des affaires florissantes, ni une nombreuse progéniture, ni le pouvoir conquis, ni les victoires sur l'ennemi n'étaient des preuves de vertu ou de la faveur de Dieu.
\end{enumerate}

 De cela il découlait que :
\begin{enumerate}
\item celui qui souffrait n'avait pas à croire que le Seigneur lui en voulait ni qu'il le punissait ;
\item rien n'autorisait les autres à le mépriser ;
\item il était même possible qu'il soit en train de payer à leur place leurs propres dettes morales ;
\item leur devoir le plus élémentaire était donc de prendre leur part de son fardeau en lui apportant aide et assistance;
\item sinon c'est eux qui seraient un jour dans le malheur après leur mort, tandis qu'il serait glorifié (comme Job), s'il endurait ses maux sans perdre confiance dans la justice du Seigneur%
\footnote{Cf. dans l'évangile de Luc la parabole de Lazare et du « mauvais riche » : Luc 16, 19-31.}%
.
\end{enumerate}

 Dans cette perspective la preuve de la sincérité de l'attachement à Dieu, c'était le service des pauvres, des malades, des malheureux de toutes sortes, c'était de tout mettre en œuvre pour soulager celui qui était dans la peine :

\begin{verse}
 « \emph{Quel est le jeûne que je veux ? \\
 C'est briser les chaînes du crime, \\
 Délier le harnais et le joug, \\
 Renvoyer libre l'opprimé \\
 Et déposer le joug. \\
 Partager ton pain avec l'affamé, \\
 Ramener chez toi le pauvre des rues, \\
 Couvrir celui que tu vois nu, \\
 C'est ta propre chair que tu ne fuis plus.} » \\
 (Isaïe, chap. 58, 6--7)
\end{verse}



 La justice juive était inséparable de la religion. À côté des fonctions cultuelles ou de direction spirituelle la fonction de prêtre ou de sage (\emph{rabbi}) impliquait de dire le droit et d'arbitrer les conflits soumis par les fidèles. Même en diaspora les tribunaux des synagogues jugeaient les affaires que leurs membres voulaient bien leur soumettre. Cela posait problème si l'un des plaignants refusait le jugement. Les communautés de la diaspora n'avaient en effet aucun intérêt à ce que les autorités civiles se mêlent de leurs affaires. L'impossibilité où elles se trouvaient d'exercer une contrainte physique sur leurs membres devait donc être compensée par l'autorité morale de leurs juges. 
 
 Il fallait que leur équité soit indiscutable. Lorsqu'il s'agissait de mesurer la gravité d'un acte l'intention du sujet était déterminante. Chacun ne répondait que pour lui-même. Il n'y avait pas de responsabilité héréditaire ou collective. Les châtiments s'appliquaient à la seule personne des criminels et à leurs biens. Parmi les délits on trouvait l'inceste, le meurtre, le vol, mais aussi la profanation du sabbat, les jurons contre le Seigneur, l'idolâtrie, la sorcellerie, l'adultère... Le meurtre était la faute la plus grave. Les sacrifices humains étaient prohibés depuis Abraham. Tout ce qui s'en rapprochait de près ou de loin l'était aussi. Les spectacles de gladiateurs étaient interdits à un double titre : d'une part comme meurtres, d'autre part comme sacrifices idolâtres, puisque leur origine se situait dans le cadre du culte des ancêtres. Même la chasse et les spectacles sanglants où étaient abattus des animaux étaient prohibés : en effet ils étaient impurs puisqu'on y versait le sang. 

 Maltraiter autrui était une \emph{faute contre YHWH}, et cette faute était aussi grave que de ne pas rendre un culte à celui-ci ou de se prosterner devant les idoles. La Tora revenait sans cesse sur le devoir d'aider les pauvres, et d'abord les veuves et les orphelins. Il était pieux et méritoire d'entretenir les orphelins de les recueillir et de les élever, de doter les orphelines et de les marier (le mot grec \emph{orphanos} pouvait englober les enfants abandonnés sans parents). C'était le prototype de l'œuvre vraiment bonne (\fsc{COHEN}, 1980, p. 225). Ce faisant il ne s'agissait pas de les adopter, mais de les élever comme des \latin{alumnii}, comme des enfants choisis, comme des enfants nourriciers ou spirituels. 

 Les pauvres avaient droit à ce dont ils avaient besoin pour vivre. L'aumône (\emph{tzedaka} = justice) rétablissait l'équité, puisque selon la Tora les richesses des riches ne leur avaient été confiées par Dieu qu'en gérance et non en pleine propriété (la punition de celui qui s'en dispensait était laissée à la discrétion du Seigneur puisque ce n'était qu'une obligation morale). Certaines dîmes spéciales étaient affectées aux pauvres, ce qui impliquait une administration collective de l'assistance (caisses de secours,~etc.). Il était interdit de saisir ce qui était nécessaire aux débiteurs pour vivre. 

 Quand aux étrangers, résidents ou de passage, il était prescrit de les traiter avec équité et sans discrimination : selon la Tora les ancêtres des juifs avaient été eux aussi des étrangers quand ils vivaient en Égypte. 

 Comme chez les autres peuples de l'Antiquité l'hospitalité était un devoir pour tous et à l'égard de tous, à charge de revanche. Pas plus qu'ailleurs elle n'était illimitée. Il s'agissait d'un droit moral à un hébergement ponctuel (trois nuits, sauf maladie ou blessure). Au-delà le voyageur était invité à pourvoir lui-même à ses besoins en travaillant. 

 En diaspora la judéité se superposait à la citoyenneté locale. C'était une citoyenneté comme une autre, puisqu'elle avait des effets au regard de la loi civile, même romaine, mais celui qui le désirait pouvait entrer dans le peuple juif, par la circoncision ou par le mariage avec un juif. De même le juif qui était aussi citoyen romain pouvait désavouer sa judéité en sacrifiant aux dieux. Il rejoignait ainsi les citoyens non-juifs dans toutes les manifestations civiques.
 
\section{Un Dieu jaloux}

 À Babylone les exilés avaient procédé à une relecture de leurs traditions, en vue de les adapter leur vie au milieu de peuples étrangers. Le point clé de leur révolution cultuelle et culturelle, c'est qu'ils avaient mis la pratique des "bonnes oeuvres" à égalité avec les sacrifices du Temple. D'autre part, pour remplacer ces derniers et les fêtes grâce auxquels les autres peuplent renouvelaient leur communion, ils avaient institué l'obligation de l'étude personnelle et collective des textes sacrés. C'est à ce moment-là qu'aurait commencé le processus de mise par écrit des livres de la \emph{Tora}. Cette bibliothèque devait en quelque sorte remplacer le temple et le pays perdus. À l'encontre des autres religions antiques, la pratique religieuse des simples fidèles incluait désormais l'étude et la réflexion. 

 Pour un homme (vir) l'étude de la Tora était à la fois un devoir et une prière. Les parents avaient pour premier devoir d'initier leurs fils à la Tora et de tenir fils et filles à l'écart des séductions du monde païen pour en faire des adultes fidèles à YHWH. L'enseignement devait veiller à ne laisser à l'écart ni les orphelins ni les indigents. Dès l'âge de six ou sept ans on apprenait à lire et écrire dans le texte de la Tora%
\footnote{Chez les Grecs et les Romains non plus personne n'aurait imaginé à cette époque-là un enseignement « primaire » non imbibé de religion : les textes des premiers exercices scolaires grecs et romains étaient les vies des divers dieux.}%
. L'enseignement de la langue grecque était accepté mais la littérature et surtout la philosophie grecques étaient récusées. 
Les synagogues étaient les instruments les plus visibles de la nouvelle façon de vivre. Chacune d'elles était à la fois école primaire, maison d'étude, maison de prière, centre communautaire, lieu d'assemblée, restaurant et lieu de réception%
\footnote{De la même façon les temples grecs et romains louaient des salles fermées ou des salles à manger en plein air pour ceux qui voulaient recevoir plus de personnes qu'ils ne pouvaient loger dignement chez eux.}%
, hôtellerie pour les coreligionnaires de passage, et tribunal pour les mariages, répudiations, et autres conflits de tous ordres entre coreligionnaires. Aucune autorité centrale ne les créait ni ne les contrôlait. Tout juif pouvait en créer une. Tout adulte mâle pouvait en diriger le culte.



 Les catégories du pur et de l'impur étaient investies par tous les peuples de l'Antiquité, mais c'était tout particulièrementle cas pour les juifs. Diverses choses rendaient impur, « souillaient » le corps, et exigeaient donc un rite ou un délai de purification : ne pas respecter le Shabbat ; manger sans avoir fait préalablement les ablutions rituelles ; manger des aliments pour lesquels la dîme (un impôt du dixième de chaque production) n'avait pas été payée ; approcher les non circoncis (impurs par excellence) et surtout manger avec eux, comme eux,~etc. Rendaient impur l'exercice de certains métiers sales ou malodorants (tanneur, vidangeur,~etc.) ; le fait de s'approcher d'un cadavre d'homme ou d'animal ; de faire couler le sang ; de consommer du sang, de consommer des animaux impurs (porcs, poissons sans écailles,~etc.), des animaux abattus de façon irrégulière, ou morts d'accident (non abattus rituellement), ou abattus depuis trop longtemps, ou cuisinés de manière irrégulière,~etc. Étaient impurs tous les écoulement issus des organes génitaux (règles, écoulement séminal spontané ou non, maladies vénériennes,~etc.), et toutes les maladies de peau (« lèpre »,~etc.). L'impureté c'était aussi « connaître » charnellement une femme ou un homme, même son conjoint légitime. Toucher l'autel du Temple sans être dans l'état de pureté convenable souillait l'offrande et invalidait le sacrifice. Toucher à mains nues les objets consacrés au culte du Temple « souillait les mains », toucher à mains nues les rouleaux de la Thora « souillait les mains »,~etc.

 

 On voit que les règles de pureté contenaient de nombreuses contradictions : c'est que la notion de pureté est complexe : à un premier niveau, le plus archaïque sans doute, la pureté n'était comme partout ailleurs qu'un problème de frontières. Le divin et le démoniaque, le surnaturel faste et le surnaturel néfaste étaient opposés au même titre au monde sans danger, banal et familier des hommes ordinaires. En ce sens-là le contact à mains nues des rouleaux de la Tora était impur, puisqu'ils étaient sacrés. Dans le coït l'homme et la femme participaient de manière directe à l'œuvre du Créateur. C'est pourquoi le coït le plus conforme au {\emph{croissez et multipliez}} de la Genèse, le plus légitime, le plus innocent, le plus sain, pour ne pas dire le plus saint, rendait \emph{impur jusqu'au soir}. Le sang est signe de vie mais le voir rapproche de la mort. Dans ce cadre on pouvait résumer ainsi : pur = inoffensif, profane, normal, quotidien, vivant, conforme à la norme (ex. : poisson avec écailles, ruminant au sabot fendu,~etc.). Impur = sacré, divin, démoniaque, prodigieux, sang, sperme, coït, naissance, mort, animal réunissant les caractéristiques de plusieurs classes d'animaux, par exemple le porc au sabot fendu mais non ruminant, les poissons sans écailles,~etc. 

 Selon une deuxième perspective la pureté et l'impureté étaient des propriétés des choses et des corps. Dans ce cadre de pensée on pouvait faire les oppositions suivantes : pur = vivant, propre, net, limpide, clair, beau, harmonieux, gracieux, habile, droit, sain, intègre, jeune, neuf, vierge, intact ; impur = mort, sale, trouble, louche, sombre, laid, difforme, gauche, maladroit, malade, lépreux, infirme, mutilé, vieux, usagé, usé, abîmé, cassé, défloré, marqué d'un sceau, d'une marque de propriété.

 Ces deux acceptions du sacré et du profane n'étaient pas propres aux juifs. Elles étaient également présentes dans toutes les religions de l'Antiquité. Même si chacune d'entre celles-ci variait dans le détail de ses classifications elles définissaient toutes de la même façon les impuretés essentielles : le coït, le sang, le sperme, la naissance, la mort,~etc. 
 
 A un troisième niveau les hébreux enrichissaient la notion de souillure de connotations morales avec plus d'insistance que ne le faisaient les autres peuples de l'Antiquité : Pur = bon, bien, vrai, juste, droit, sincère, bienveillant, honnête, intègre, innocent, juste, équitable, désintéressé. Impur = mauvais, mal, faute, faux, pervers, menteur, malveillant, méchant, injuste, inéquitable, intéressé, coupable. Dans cet approfondissement les prophètes d'Israël avaient joué un rôle déterminant. En effet ils avaient interprété les relations de leur peuple avec leur dieu comme une relation amoureuse entre deux personnes. Dans leur bouche il parlait comme un époux épris, blessé par l'infidélité de son épouse plus que comme un père ou un maître tout-puissant. 
Ils suivaient souvent le canevas suivant :
\begin{enumerate}
\item Il n'est ni facile ni drôle d'être le peuple élu,
 \begin{enumerate}
 \item parce que les règles de pureté auxquelles il s'est engagé le séparent des autres, et le rendent infréquentable, alors qu'il voudrait qu'on l'aime, comme tout le monde,
 \item parce que ces règles sont un joug et un carcan et qu'il lui est impossible de les observer toutes,
 \item parce qu'elles lui interdisent les jouissances faciles et sans dangers des autres peuples,
 \item parce que ses prétentions à l'exclusivité de l'élection divine le rendent odieux au reste du monde. 
 \end{enumerate}
\item Voilà pourquoi Israël finit toujours par regarder ailleurs. Il oublie constamment les promesses faites par ses pères et se montre infidèle. 
\item Mais le Seigneur est un dieu jaloux. Il laisse sa colère s'abattre sur son peuple. Il le châtie, c'est-à-dire qu'Il permet à ses ennemis de l'accabler. 
\item Face à ces maux Israël se repent et revient vers Lui, parce qu'il n'y a pas d'autre dieu.
\item YHWH est fidèle et tient ses promesses. Il ne sait pas résister à la prière de son peuple. Il le délivre et disperse ceux qui le tourmentaient.
\end{enumerate}

 À partir du moment où les rapports du Seigneur et de son peuple devenaient une histoire d'amour, les problèmes de pureté et d'impureté ne pouvaient plus être traités seulement de manière rituelle. Ce qui comptait désormais le plus c'était la question : \frquote{\emph{est-ce que tu m'aimes ?}}. 
La véritable impureté c'était désormais l'infidélité et le refus de reconnaître la faute commise. Pour les prophètes la véritable offrande de réparation, la seule qui ne serait jamais refusée, ce n'était plus la bête de choix, c'était un « cœur brisé », c'est-à-dire un repentir sincère. 

Le paradoxe est que ce nouveau point de vue n'abolissait aucune des exigences rituelles. La propension des amoureux est au contraire d'en rajouter, d'en faire plus que ce que l'usage ne prescrit. Les exigences rituelles devenaient une façon de parler, une façon de prier. L'amour pour YHwh poussait des personnes des deux sexes à se faire \emph{Nazir}, « consacré » par un voeu. Ils (elles) rasaient alors leur chevelure et durant un temps plus ou moins long ils s'abstenaient de certaines nourritures et boissons comme de toute relation sexuelle. Il a existé pendant au moins un siècle et demi des communautés dont le style de vie était quasi monastique, les Esséniens qui pour se rapprocher de Dieu fuyaient toute impureté rituelle.
 
 Afin d'achever de se différencier définitivement des autres peuples auxquels ils étaient mêlés les « sages » avaient voulu que pour chaque activité humaine il y ait une manière juive de procéder. Selon le Talmud%
\footnote{\fsc{Cohen}, 1980, p. 227. Le Talmud est la deuxième grande œuvre des juifs, après la Tora. Il a fixé par écrit à partir du deuxième siècle de notre ère la « Loi orale » transmise de sage en sage, de rabbin en rabbin, à côté de la Tora. En ce qui concerne la question de l'existence d'écoles \emph{pour tous} au tout début de notre ère, l'information donnée par le Talmud est pourtant d'autant plus vraisemblable qu'à l'époque où nous nous situons les cités grecques et romaines finançaient elles aussi des institutions scolaires pour leurs jeunes citoyens.}
ils avaient planté autour du peuple la \emph{haie} des prescriptions de la Tora, dressée comme un rempart qui le séparait des autres peuples et le gardait \emph{pur} de toute contamination. D'une certaine manière ils avaient ainsi  transformé leur vie entière en liturgie. 

La réaction de rejet des observateurs de l'Antiquité devant la « superstition » des juifs s'expliquait en grande partie par la rigidité du cadre dans lequel ces derniers s'étaient corsetés. Ils trouvaient que les juifs étaient infréquentables, et « ennemis du genre humain ». De toutes leurs bizarreries les plus discourtoises étaient en effet leur refus d'assister à tout sacrifice aux dieux des gentils et de manger avec aucun incirconcis, en un temps où il n'y avait pas de vraie cérémonie publique sans sacrifice aux dieux ni de sacrifice sans repas en commun. 

 Et pourtant, malgré la coupure avec le monde ordinaire qu'impliquait le mode d'existence juif, les communautés de la diaspora exerçaient une attraction certaine sur leur environnement, et le nombre des « prosélytes » était relativement important. Beaucoup se satisfaisaient d'être des \emph{craignant Dieu}%
\footnote{La \emph{Crainte de Dieu} désignait l'attitude d'adoration respectueuse du Seigneur et la volonté de respecter ses commandements : moins une attitude de peur (encore qu'elle n'en soit pas exempte) que de révérence.}%
, non circoncis%
\footnote{La circoncision était douloureuse et non sans risques, et surtout très mal vue chez les Grecs et les Romains, quand elle n'était pas interdite par ces derniers à tous les hommes libres comme toute autre mutilation.}%
. Les plus courageux ou les plus convaincus se faisaient circoncire, ce qui en faisait de nouveaux juifs. Si la judéité découlait normalement de la naissance, elle pouvait aussi être le fruit d'un choix délibéré par amour pour le Seigneur, un amour qui pouvait aller jusqu'à la mort si les autorités civiles exigeaient quelque chose de contraire à la Tora. Comme le fait remarquer Paul \fsc{VEYNE}, c'était radicalement différent des cités contemporaines qu'on ne choisissait jamais par amour pour leurs dieux.

 
 
 
\section{Règles de mariage}

 
 
 Selon la Genèse \emph{Dieu créa l'homme à son image, à l'image de Dieu il le créa, homme et femme il le créa. Dieu les bénit et leur dit : Soyez féconds, multipliez, emplissez la terre et soumettez-la.} (Gn 1, 27-28). \emph{C'est pourquoi l'homme quitte son père et sa mère et s'attache à sa femme, et ils deviennent une seule chair.} (Gn 2, 24). Selon ce texte fondateur le mariage crée une parenté aussi proche que celle entre parents et enfants (une seule chair). Les interdits de mariage découlaient de cette parenté nouvelle, qui ne privilégiait ni le père ni la mère : ils portaient donc de manière presque symétrique du côté paternel et du côté maternel, par le sang ou par l'alliance.
 
 Comme conjoint(e) étaient interdits : le père, la mère, les femmes du père, les frère, sœurs, demi-frères, demi-sœurs, les époux du frère ou de la sœur, du demi-frère ou de la demi-sœur, les fils et filles et leurs conjoints, les petits-enfants et leurs conjoints. Les sœurs de l'épouse étaient interdites du vivant de l'épouse. Le mariage était autorisé entre oncle et nièce, par contre les tantes étaient interdites aux neveux. Il était interdit d'avoir en même temps des relations avec une mère et sa fille ou sa petite-fille. Les mariages entre cousins étaient autorisés et parfois préférés. 
 
 Il était interdit d'entretenir un culte des ancêtres : cela aurait été faire preuve de paganisme. Le Seigneur seul connaissait le destin des ancêtres \emph{(dans le sein d'Abraham)}, et il n'y avait pas à s'en occuper. Ce n'était pas l'existence d'un culte familial qui définissait la famille, c'est pourquoi était interdite l'adoption d'un étranger (l'adoption à cette époque avait pour objectif d'assurer la permanence du culte des ancetre en cas d'absence de fils). 



La polygynie était permise. A côté des épouses légitimes, celles qui avaient reçu une dot de leur père et dont le mariage avait fait l'objet d'un contrat écrit, on admettait des concubines : femmes libres sans dot ou bien esclaves acquises à prix d'argent ou reçues comme butin de guerre\footnote{Selon le Talmud l'homme qui voulait avoir plusieurs épouses ou concubines devait obtenir l'accord de sa première épouse, pouvoir entretenir matériellement deux ou plusieurs foyers, et être en mesure de remplir son devoir conjugal avec chacune de ses femmes de façon à les satisfaire. Sinon elles étaient en droit de se plaindre à la justice, et d'obtenir le divorce à leur avantage.}%
. Une femme qui n'avait pas donné d'enfant à son mari pouvait lui suggérer de prendre une deuxième épouse ou une concubine\footnote{ainsi avait fait Sara, stérile, qui avait offert Agar sa propre esclave à son époux Abraham, afin d'avoir ainsi une descendance.}. La monogamie n'en était pas moins le modèle comme chez les Grecs et les Romains. Le Grand Prêtre, qui se devait d'etre un modèle de pureté rituelle, n'avait droit qu'à une seule épouse, et à aucune concubine : la polygamie impliquait donc une certaine imperfection et une moindre pureté\footnote{Les œuvres les plus tardives de la littérature hébraïque ne mettent en scène que des couples monogames. Aucun rabbin du passé, même dans l'Antiquité, n'est connu pour avoir eu plus d'une femme : à vrai dire les sociétés où la polygamie est autorisée ne peuvent jamais compter beaucoup de foyers polygames, ne serait-ce que parce qu'il est fort coûteux d'entretenir plus d'un foyer, surtout si on élève tous les enfants qui y naissent. À Rome, il était usuel que les grossesses des concubines soient interrompues par un avortement, contrairement aux grossesses des épouses légitimes tant que celles-ci n'avaient pas donné le jour au nombre d'enfants légitimes souhaitable ou prescrit par la loi : là aussi l'usage le plus fréquent était donc de n'entretenir qu'un seul foyer.}  . Comme à Rome, les enfants des épouses et des concubines étaient légitimes s'ils étaient reconnus par leur père et si la fidélité de la mère n'était pas mise en doute. 

Par ailleurs si deux frères demeuraient ensemble et si l'un d'eux  mourait sans laisser d'enfant mâle, le survivant avait\footnote{...comme chez dans bien d'autres peuples d'Asie ou d'Afrique de toutes époques, notamment les Égyptiens, les Babyloniens et les Phéniciens.} le devoir d'épouser la veuve du défunt (\emph{lévirat}, Dt 25, 5-10), toute autre alliance étant interdite à celle-ci. En cas de refus du survivant de donner une descendance au défunt, ce qui impliquait de prendre en charge matériellement la veuve et ses enfants à venir, celle-ci pouvait se remarier avec un autre conjoint une fois constaté publiquement son refus de « \emph{relever la maison de son frère} ». 

La famille juive était aussi patriarcale que les autres familles de l'Antiquité méditerranéenne, et ses règles de fonctionnement internes ne différaient guère. Le père avait tout pouvoir sur les siens : femme, enfants%
\footnote{Selon la Tora et le Talmud la rébellion d'un enfant contre son père serait punie de mort (comme à Rome) \emph{si ce dernier le demandait}, mais il faudrait qu'un jugement en bonne et due forme approuve sa demande.}%
, esclaves. Les filles étaient exclues de l'héritage du père et l'ainé des garçons était privilégié par rapport aux autres. 

Comme partout ailleurs jusqu'au XXème siècle l'exclusion des filles de l'héritage était un signe de l'incapacité où on les tenait de succéder à leur père dans ses emplois. C'était aussi (en toute inconscience ?) un moyen de les tenir en sujétion puisque meme pour seulement subsister elles devaient en passer par la dépendance à un homme (père, frère, époux, beau-frère, fils, maître). Elles ne pouvaient hériter qu'en l'absence de garçons, mais en ce cas elles devaient se marier dans leur tribu pour que leur patrimoine ne sorte pas de celle-ci (cela ne diffère pas du statut de la fille \emph{épiclère} dans le monde grec). Dans ce cas elles n'en étaient pas moins soumises à leur mari. 



La virginité des femmes avant le mariage était exigée. L'épouse surprise en flagrant délit d'adultère était passible de la lapidation avec son complice sauf si son mari préférait la répudier et tout le monde s'attendait à ce qu'il le fasse, comme à Rome. C'était une cause de répudiation indiscutable. Par ailleurs une épouse ne pouvait pas prendre l'initiative de divorcer. Comme à Rome au même moment il fallait que son mari lui accorde le droit de le faire%
\footnote{Selon le Talmud en cas de conflit conjugal insoluble un homme pouvait néanmoins être conduit à divorcer par la pression des membres influents de sa communauté : cela était-il une pratique courante avant notre ère ?}%
. Le viol d'une femme libre était puni de mort. Le fautif n'échappait au châtiment que s'il épousait sa victime (pour cela il devait être accepté comme gendre par le père de celle-ci), mais il perdait en ce cas le droit de la répudier. Une femme répudiée puis épousée par un autre homme ne pouvait plus être épousée à nouveau par son premier mari. La démence de l'épouse interdisait à l'époux de la répudier, mais non de prendre une concubine. 



 Aux hommes les femmes de tous les autres hommes étaient interdites (adultère), par contre toutes les femmes non mariées leur étaient permises, comme dans les autres sociétés antiques. La prostitution était interdite par la Tora, mais forniquer avec une prostituée n'était qu'une faute morale sans gravité et non une infraction légale. Seuls étaient strictement interdits les prostituées et prostitués sacrés des temples païens, avec qui coïter équivalait à sacrifier aux idoles%
\footnote{La fidélité masculine est néanmoins présentée par la Tora ou le Talmud comme le modèle à atteindre. Proverbes 5 (15 à 20) "\emph{
15 Bois des eaux de ta citerne, et des ruisseaux de ton puits.
16 Tes fontaines doivent-elles se répandre dehors, et tes ruisseaux d'eau sur les places publiques ?
17 Qu'ils soient à toi seul, et non aux étrangers avec toi.
18 Que ta source soit bénie; et réjouis-toi de la femme de ta jeunesse,
19 Comme d'une biche aimable et d'une chèvre gracieuse; que ses caresses te réjouissent en tout temps, et sois continuellement épris de son amour.
20 Et pourquoi, mon fils, t'égarerais-tu après une autre, et embrasserais-tu le sein d'une étrangère ?"} L'infidélité de l'époux favorise et entraîne celle de l'épouse : \emph{"lui parmi les fruits mûrs, elle parmi les plantes croissantes"}. 
}%
. 

 



 
 
 



Si l'appartenance à Israël était inscrite dans la chair des hommes par la circoncision,  elle l'était d'abord par la filiation. Le \emph{Deutéronome} interdisait d'épouser un étranger ou une étrangère (non juif-ve), ou un eunuque (incapable d'engendrer). Selon la Bible au retour de Babylone (cinquième siècle avant J.-C.), Esdras aurait chassé du peuple toutes les femmes étrangères, tous leurs enfants, tous les hommes qui ne voulaient pas se séparer d'elles, et aussi tous ceux dont les origines familiales étaient discutables ou qui ne pouvaient apporter la preuve du contraire (\emph{Esdras}, 10, 17). Que ce récit soit fondé sur un fait historique ou non, il témoigne de la volonté de construire une nation pure, à une période où les cités grecques renforçaient elles aussi le lien entre citoyenneté et descendance légitime. 

\section{naissances impures}
 
 
 L'honorabilité de l'ascendance était un brevet de valeur religieuse et sociale. Connaître ses ancêtres sur de nombreuses générations et pouvoir prouver la correction de leurs unions successives était un signe d'excellence. Comme les quartiers de noblesse de l'ancien régime français, ou la « pureté du sang » des siècles classiques d'Espagne et du Portugal, cette pureté-là se capitalisait au fil des générations. Elle permettait de s'allier avec d'autres familles à la pureté aussi préservée, aux alliances aussi bien choisies que les siennes. Elle s'accroissait par la gestion intelligente des unions, et se perdait si on se laissait aller aux rencontres de hasard. 

 

 
 Le \emph{Deutéronome} (Deut. 23, 3-4 ; 24, 4) interdisait d'épouser un \emph{mamzer}. On appelait \emph{mamzerim} les \emph{impurs de naissance}  c'est-à-dire tous les enfants issus d'unions interdites entre deux parents juifs. En effet cela ne concernait en principe que les enfants nés de deux parents juifs\footnote{in \emph{histoire de la famille, I}, p. 377-379. }, mais en pratique ce statut pouvait par extension englober d'autres personnes (le Talmud rend compte d'une certaine diversité dans les interprétations des interdits) : 
 \begin{enumerate}
 \item \emph{d'abord et avant tout} les enfants nés d'unions incestueuses (les unions interdites semblent avoir été les mêmes que les unions interdites par la loi romaine au même moment\footnote{Lévitique, chapitre 18 : \emph{1 Et l'Éternel parla à Moïse en disant :
2 Parle aux fils d'Israël et dis-leur : Je suis l'Éternel, votre Dieu.
3 Vous ne ferez pas comme on fait au pays d'Égypte où vous avez habité, et vous ne ferez pas comme on fait au pays de Canaan où je vous conduis ; vous ne marcherez pas selon leurs statuts ;
4 vous écouterez mes ordonnances et vous observerez mes statuts pour y marcher. Je suis l'Éternel, votre Dieu.
5 Vous observerez mes statuts et mes ordonnances l'homme qui les pratiquera vivra par elles : je suis l'Éternel.
6 Nul de vous ne s'approchera de sa proche parente pour découvrir sa nudité : je suis l'Éternel.\emph{ ("Découvrir la nudité" est un euphémisme  souvent employé dans la Bible pour désigner une relation sexuelle. La nudité n'était pas valorisée par les juifs comme elle pouvait l'etre au meme moment chez les grecs, mais elle était réservée exclusivement à l'intimité sexuelle.)}
7 Tu ne découvriras point la nudité de ton père et la nudité de ta mère ; c'est ta mère, tu ne découvriras pas sa nudité.
8 Tu ne découvriras point la nudité de la femme de ton père ; c'est la nudité de ton père.
9 Tu ne découvriras point la nudité de ta sœur, fille de ton père ou fille de ta mère ; qu'elle soit née dans la maison ou qu'elle soit née au dehors, tu ne découvriras point leur nudité.
10 Tu ne découvriras point la nudité de la fille de ton fils ou de la fille de ta fille, car c'est ta nudité.
11 Tu ne découvriras pas la nudité de la fille de la femme de ton père, née de ton père ; c'est ta sœur.
12 Tu ne découvriras pas la nudité de la sœur de ton père ; elle est du sang de ton père.
13 Tu ne découvriras pas la nudité de la sœur de ta mère ; elle est du sang de ta mère.
14 Tu ne découvriras pas la nudité du frère de ton père, tu ne t'approcheras point de sa femme ; c'est ta tante.
15 Tu ne découvriras pas la nudité de ta belle-fille ; c'est la femme de ton fils, tu ne découvriras point sa nudité.
16 Tu ne découvriras pas la nudité de la femme de ton frère ; c'est la nudité de ton frère.
17 Tu ne découvriras pas la nudité d'une femme et de sa fille ; tu ne prendras pas la fille de son fils, ni la fille de sa fille pour découvrir leur nudité ; elles sont proches parentes, c'est un crime.
18 Tu ne prendras pas la sœur de ta femme de manière à créer une rivalité, en découvrant la nudité de l'une avec celle de l'autre de son vivant.
19 Tu ne t'approcheras point d'une femme pendant son impureté périodique pour découvrir sa nudité.
20 Tu n'auras point commerce avec la femme de ton prochain pour te souiller avec elle.
21 Tu ne donneras point de tes enfants pour les sacrifier à Moloch et tu ne profaneras pas le nom de ton Dieu. Je suis l'Éternel.
22 Tu ne coucheras point avec un homme comme on couche avec une femme ; c'est une abomination.
23 Tu ne coucheras point avec aucune bête pour te souiller avec elle. La femme ne s'approchera point d'une bête pour se prostituer à elle ; c'est une chose monstrueuse.
24 Ne vous souillez par aucune de ces choses ; car c'est par toutes ces, choses que se sont souillées les nations que je chasse devant vous.
25 Le pays en a été souillé, j'ai puni son iniquité et la terre a vomi ses habitants.
26 Mais vous, vous garderez mes statuts et mes ordonnances, et vous ne commettrez aucune de ces abominations, ni l'indigène, ni l'étranger qui séjourne au milieu de vous.
27 Car toutes ces abominations, les hommes du pays, qui y ont été avant vous, les ont commises, et la terre en a été souillée.
28 Et la terre ne vous vomira pas pour l'avoir souillée, comme elle a vomi la nation qui y a été avant vous.
29 Car tous ceux qui auront commis quelqu'une de ces abominations, ceux qui auront fait cela seront retranchés du milieu de leur peuple.
30 Vous garderez mes observances afin de ne pratiquer aucune des coutumes abominables qui ont été pratiquées avant vous ; vous ne vous souillerez point par elles. Je suis l'Éternel, votre Dieu. }}) ;
 \item \emph{et au même titre} les enfants nés de l'union entre un homme juif, marié ou non, et une femme juive mariée (adultère féminin) ;
 \item les enfants d'une femme adultère ou prostituée dont le père ne pouvait pas etre connu avec certitude ;
 \item les enfants nés de père inconnu (mère juive non mariée) et ceux nés de père et mère inconnus (c'est-à-dire tous les enfants trouvés) ;
 \item \emph{mais aussi} l'enfant né de l'union d'un juif non mamzer avec un juif mamzer (ce statut se transmettait à toute la descendance des mamzerim);
\item les enfants nés du remariage d'un homme avec une femme dont il avait d'abord divorcé dans les formes légitimes, et qui s'était remariée entre temps avec un autre homme;
\item les enfants nés du mariage ou du concubinage d'un(e) juif(ve) avec un(e) non juif(ve), légalement nul pour les juifs, mais légitime pour la loi romaine : les enfants étaient juifs si leur mère était juive, non juifs dans le cas contraire.
\end{enumerate}

 Les mamzerim étaient \emph{exclus de l'assemblée du peuple jusqu'à la dixième génération}, ce qui veut entre autre dire qu'ils ne pouvaient épouser ceux qui n'étaient \emph{pas} nés impurs, en effet aucune famille bien née ne pouvait s'allier à eux. \emph{C'était même un devoir religieux que de ne pas leur donner un de ses enfants comme époux.} Quand il y avait des failles, des lacunes ou des faux-pas dans la chaîne des générations, la sanction c'était donc de ne plus pouvoir se marier qu'entre familles d'impureté équivalente. Un mamzer ne pouvait se marier qu'avec d'autres mamzerim, ou des païens, impurs par définition, ou des convertis, nés impurs. Il n'en avait pas moins droit à l'héritage à égalité avec les autres enfants de ses parents et il n'était exclu ni des synagogues ni des études religieuses.
 
 L'enfant d'une juive non mariée et d'un juif dont on connaissait l'identité, qu'il soit marié ou non, n'était un mamzer que si ses deux géniteurs étaient interdits de mariage. Il n'était donc ni nécessaire ni suffisant de naître en légitime mariage pour échapper au statut de mamzer, mais la reconnaissance par un homme non interdit de mariage avec la mère était toujours indispensable. Les enfants trouvés étaient souvent présumés mamzerim parce que les enfants nés d'unions interdites avaient plus de risques que les autres d'être abandonnés\footnote{...surtout lorsque les circonstances matérielles de l'exposition laissaient à penser que les parents ne voulaient pas vraiment que l'enfant soit trouvé vivant : lieu écarté loin des chemins, à la merci des bêtes sauvages, etc...}. Même dans ce cas il n'était pas possible de les rejeter puisqu'ils étaient présumés descendants d'Abraham comme les autres, et puisque toute vie humaine était sacrée. Si nécessaire, la communauté les prenait donc en charge, mais elle les maintenait au dernier rang.


 On note les ressemblances entre le statut des mamzerim juifs et celui des Romains atteints par \emph{l'infamie}, ou des Grecs frappés par \emph{l'atimie}. Il s'agissait là de l'expression d'une vision du monde commune à toutes les sociétés antiques. Même si chacune délimitait à sa façon la \emph{mauvaise réputation}, le noyau des hontes était commun. 

Les mariages mixtes étaient en principe interdits, mais un \emph{gentil} (non-juif) pouvait se convertir au judaïsme, et la Bible en fournissait de nombreux exemples. Comme c'est par la mère que se transmettait l'appartenance au peuple hébreu, celui ou celle qui se convertissait ne pouvait en faire pleinement partie, puisque sa mère n'était pas juive. Il (elle) n'était au sens strict qu'un allié du peuple saint. La conversion ne pouvait annuler le fait qu'une femme née non juive soit « impure ». Seules les femmes juives nées de mère juive pouvaient donner le jour à des enfants à la légitimité incontestable (= pureté religieuse). Ce n'est qu'au niveau de ses enfants que la famille d'un(e) converti(e) serait juive de plein droit. Cela ressemblait à l'intégration d'un affranchi dans le peuple Romain, qui lui non plus n'était jamais pleinement assimilé aux citoyens, au contraire de ses enfants. Cela décourageait les hommes de chercher leurs épouses et leurs concubines à l'extérieur de leur communauté, sauf quand ils ne pouvaient faire autrement, ce qui était souvent le cas des mamzerim.



\section{Sacralisation de la sexualité conjugale}

 


Selon \fsc{CHOURAQUI}, aux yeux des hommes
 de la Bible%
\footnote{A.~\fsc{CHOURAQUI}, \emph{La vie quotidienne des hommes de la bible}, 1978, p.153-155.}%
 :

\begin{displayquote}
{%
\emph{L'activité sexuelle normale et licite est un bien. Elle constitue même l'objet du premier commandement que Dieu donne à l'homme dans la Genèse au terme de la création du monde : « fructifiez, multipliez, remplissez la terre... ». La vie sexuelle n'est d'ailleurs pas dissociée du couple et aucun mot n'existe en hébreu pour la désigner comme telle...}

 \emph{Celle-ci est étalée au grand jour et fait l'objet d'une législation très stricte, très abondante et très détaillée qui prouve moins la vertu du peuple de la Bible que l'importance pour lui de ces problèmes. Les documents, faits divers ou lois, que la Bible nous lègue sur ce thème n'ont sans doute aucun parallèle dans aucune civilisation de l'Antiquité...}

 \emph{La femme mariée est consacrée, sanctifiée, mise à part pour son époux. De ce fait, la copulation provoque une impureté comme tout contact avec le sacré. Après le coït, le couple doit faire ses ablutions et se purifier : il restera impur jusqu'au soir. La loi est la même pour l'homme après toute copulation ou toute perte séminale. L'activité sexuelle, de quelle nature qu'elle soit, introduit l'homme dans l'univers du sacré. Il doit être purifié pour retrouver la plénitude des fonctions profanes. Aussi les mœurs tendent-elles à une ségrégation des sexes.}

 \emph{Toute activité sexuelle est prohibée avec une femme qui a ses règles, et tout contact direct ou indirect avec elle est également interdit. La perte du sang provoque l'impureté de la femme \emph{[...]} Car le sang, c'est la vie, et la perte du sang menstruel, comme les suites d'un accouchement, placent la femme dans la zone redoutable et mystérieuse qui se situe entre la vie et la mort, entre les pôles du pur et de l'immonde, qui définissent les termes majeurs de la dialectique biblique. En fait l'acte sexuel n'est ainsi permis qu'aux époques de fécondité de la femme et interdit quand elle est stérile.}

 \emph{Les interdits sexuels pleuvent dans la législation et les peines sont d'une redoutable sévérité : la mort par lapidation ou par « tranchement du peuple ». À l'opposé de la licence qui règne dans ce domaine dans toute l'Antiquité \emph{[...]} on constate dans la Bible un effort quasi désespéré qui tend à discipliner et à orienter l'activité sexuelle du couple.}

 \emph{L'homosexualité \emph{[...]} est qualifiée « d'abomination » \emph{[...]} Les prophètes et les législateurs hébreux sont sans doute les premiers à la prohiber avec une implacable sévérité.}

 \emph{La bestialité \emph{[...]} est, elle aussi, punie de mort \emph{[...]} la prostitution est condamnée par la loi, mais elle subsiste en fait \emph{[...]}}

 \emph{Les précautions prises pour définir l'activité sexuelle illicite mettent en relief le caractère profondément original de l'amour et de la vie du sexe selon les hébreux. Leur souci majeur a été de provoquer une démythisation, une démystification, une libération et une sacralisation de l'activité sexuelle du couple.}
}%
\end{displayquote}

 La Tora glorifiait l'amour entre l'homme et la femme (cf. \emph{le Cantique des cantiques}) et l'amour des parents pour les enfants. Et si la répudiation de l'épouse par le mari était autorisée (et non l'inverse), elle n'était pourtant pas bien vue ; \emph{"...YHWH est témoin entre toi et la femme de ta jeunesse envers qui tu te montras perfide, bien qu'elle fut ta compagne et la femme de ton alliance... Car Je hais la répudiation, dit le YHWH Dieu d'Israël"} (Malachie, 2, 14 et 16). 
 
 Le célibataire n'était pas considéré comme un homme complet et le célibat définitif n'était accepté qu'en cas d'incapacité complète de procréer. En effet chacun se devait d'avoir une descendance, et la stérilité était un malheur. Les épouses de ceux qui étaient décédés sans enfants devaient leur en donner dans le cadre du \emph{Lévirat}, en se mariant avec un de leurs frères. La répudiation d'une épouse avait souvent ce motif. 

 Les pratiques sexuelles qui ne peuvent aboutir à une conception étaient interdites. Les relations homosexuelles masculines étaient passibles de mort quel que soit le statut ou l'âge des deux partenaires. 

 L'exposition des nouveaux-nés était interdite, sauf en situation d'absolue détresse. De même il était interdit de mettre à mort un enfant quel qu'il soit. C'était la conséquence directe du \frquote{\emph{tu ne tueras point}} du Décalogue. Les avortements étaient considérés jusqu'à un certain point comme des assassinats, sauf risque pour la vie de la femme, préférée à celle du fœtus en cas de danger mortel. Ceci étant dit les juifs adhéraient comme l'ensemble des gens de l'Antiquité à l'idée que le fœtus ne devenait humain qu'au bout d'une certaine durée, avant laquelle il n'était pas animé, ce qui autorisait l'avortement. Sur cette durée les opinions variaient : quarante jour pour les garçons ? Quatre vingt ou quatre vingt dix pour les filles \footnote{On se posait d'ailleurs la meme question dans toute l'aire grecque.} ?

 Ces prescriptions favorisaient les naissances et l'équilibre du ratio garçons/filles. Les hébreux en ont-ils eu conscience ? Bien sûr, même si ce n'était pas leur objectif premier. Leurs contemporains étaient parfaitement conscients des effets à long terme de ces comportements natalistes et nous en ont laissé des témoignages%
\footnote{Cf. les commentaires de Tacite, dans Aline \fsc{ROUSSELLE}, 2001.}%
.

 



 



\section{Les prêtres du Temple}

 Il n'y avait qu'un seul temple pour tous les juifs, celui de Jérusalem. Les charges et dignités religieuses étaient héréditaires. Les prêtres et les lévites étaient choisis exclusivement dans la \emph{tribu de Lévi}. Les prêtres devaient descendre en ligne directe d'Aaron, frère de Moïse. La tribu de Lévi vivait du produit de la Dîme versée par les 11 autres tribus. Les communautés de la Diaspora contribuaient elles aussi par l'impôt spécial \latin{(fiscus judaïticus)} à l'entretien du Temple et de ses desservants. 

 Les prêtres se succédaient au Temple de semaine en semaine selon un tour de service. Le reste du temps ils vivaient chez eux, pas toujours à Jérusalem. Durant leur semaine de service ils présidaient aux cérémonies. Sacrificateurs ils abattaient rituellement, ils « immolaient » les bêtes offertes au nom du peuple ou des particuliers. Ils les découpaient et les préparaient comme le rituel le prescrivait. 
  Les lévites, qui ne descendaient pas d'Aaron en ligne directe, étaient chargés des tâches cultuelles autres que les sacrifices. Ils n'approchaient pas de l'autel et ne devaient pas le toucher. Ils jouaient de la musique chantaient les psaumes, faisaient la police du temple, géraient ses biens, recevaient les dimes, etc....

 De la conception à la mort les prêtres et les lévites vivaient dans l'obsession de la pureté rituelle. Ils respectaient les mêmes interdits que les autres hébreux mais en outre ils n'avaient pas le droit de porter d'armes, puisque celles-ci versaient le sang et étaient impures par nature. Ils avaient encore moins le droit de s'en servir. Verser le sang humain leur interdisait définitivement de remplir leurs fonctions cultuelles. Dans le même sens ils devaient rester éloignés de tout blessé susceptible de les souiller par son sang, et se tenir à l'écart de tout cadavre humain et animal, sauf celui de leurs parents les plus proches. C'est d'abord durant leur semaine de service au Temple que les prêtres devaient éviter toute impureté. Les actes sexuels leur étaient interdits durant les jours de purification préalable et durant tous leurs jours de présence à l'autel. Puisque seuls les hommes de la Tribu de Lévi pouvaient servir le Temple, ils devaient \emph{se procurer une descendance} en dehors de leur temps de service. \footnote{Plusieurs dizaines de siècles auparavant il en était de même en Égypte (Serge \fsc{SAUNERON}, \emph{les prêtres de l'ancienne Égypte}, Editions du Seuil, 1998 (première édition 1957); cf. page 47 et suivantes) : obligation de la circoncision, du rasage intégral de la tête et du corps, d'ablutions répétées à heures fixes, du respect de divers interdits alimentaires, de jeûnes, abstinence précédant de plusieurs jours et accompagnant les périodes de service au temple, interdiction de la polygamie, interdiction de porter certains tissus,~etc.} 
 
 Pour officier ils devaient être parfaitement sains de corps, sans aucune maladie, sans difformité physique et sans infirmité, et dans la force de l'âge (25 à 50 ans). Ils ne devaient pas avoir la voix embarrassée. Il ne devait pas leur manquer une seule phalange : une mutilation même minime (oreille ou phalange coupée, entorse mal réparée...) leur faisait d'autant plus sûrement perdre leur emploi qu'il y avait plus de candidats que de postes à pourvoir. 

 Les généalogies des membres de la tribu de Lévi, et surtout celles des prêtres, étaient d'autant plus impeccables que s'ils voulaient officier au Temple (et en vivre) ils ne pouvaient épouser ni une juive divorcée, ni la fille de deux parents convertis, ni une convertie, ni la fille d'un homme non juif et d'une mère juive, ni une veuve refusée par son beau-frère \emph{(halizah)} dans le cadre du \emph{lévirat}, ni une mamzer,~etc. 

 Il leur était interdit de vivre dans l'adultère, eux et tous ceux et celles qui vivaient sous leur toit. Ils devaient épouser une femme vierge, et surtout qui n'ait pas été prostituée. Si leur épouse était infidèle, elle était lapidée jusqu'à la mort. Leurs filles devaient rester vierges tant qu'elles vivaient auprès d'eux et qu'elles mangeaient donc les \emph{choses sacrées} provenant du Temple, les parts réservées au clergé sur les offrandes. 

 Le grand prêtre était sacré car il avait été consacré \emph{(oint)}. Il était enserré dans le réseau de règles le plus contraignant. Au contraire de ses concitoyens il n'avait droit ni à la polygamie, ni à une concubine : il n'avait droit qu'à une seule épouse, de pure origine juive, et épousée vierge. Ces contraintes qu'il subissait plus durement qu'aucun autre dans sa vie privée soulignent le cœur des conceptions juives en matière de morale matrimoniale.
 
 À partir de la destruction du Temple de Jérusalem, en l'an 70 de notre ère, c'est entre autre aux règles de pureté qui jusque là n'étaient imposées qu'à la caste sacerdotale que les rabbins vont plus ou moins soumettre l'ensemble du peuple pour en faire d'une certaine façon un peuple de pretre. Cela s'est révélé une manière particulièrement efficace de faire perdurer le souvenir du temple détruit et de perpétuer son oeuvre de sanctification.
 
 \section{Travailleurs et esclaves juifs}

 La Genèse fait du travail un devoir qui, à égalité avec la génération, construit l'homme : {\emph{Soyez féconds, multipliez, emplissez la terre et soumettez-la}} (Gn 1, 28), même si ce devoir est pénible à cause de la faute d'Adam : {\emph{à la sueur de ton visage tu mangeras ton pain}} (Genèse, 3, 19). Contrairement aux Grecs et aux Romains, les juifs ne ressentaient pas d'équivalence entre le loisir, {\latin{l'otium}}, et la vertu, pas plus qu'entre le labeur ({\latin{nec-otium}}, d'où vient le négoce) et l'incapacité à la vertu. Ils pensaient que celui qui vit dans l'oisiveté court le risque de se livrer à l'immoralité. Il n'y avait donc pas de métiers réellement et inconditionnellement impurs, en dehors (comme partout à l'époque) de ceux qui touchaient de près ou de loin à la prostitution et aussi, selon le Talmud, de ceux d'organisateurs de spectacles, acteurs, gladiateurs et cochers de cirque. Comme le souligne Aline \fsc{ROUSSELLE} (\emph{La contamination spirituelle}, Science, 1998), ce sont exactement les mêmes métiers que ceux que les Romains jugeaient infâmes. L'impureté attachée à certains autres métiers n'était pas une tare morale mais découlait (mécaniquement) du contact avec les ordures ou avec les sources de la vie et de la mort. Elle compliquait la vie du travailleur à cause des purifications incessantes qu'elle entraînait, mais si le métier exercé était lucratif il n'était pas méprisé. Cette impureté fonctionnelle n'avait en tous cas rien d'une tare indélébile, transmissible à la descendance, contrairement à celle du mamzer.
 
  Entre juifs seul le prêt sans intérêt était autorisé. Le métier d'usurier était l'un des plus vigoureusement blâmé : {\emph{les usuriers sont comparables à ceux qui répandent le sang}} (\fsc{Cohen}, p. 250) : les intérêts exigés par les prêteurs antiques étaient en effet exorbitants. 

 Comme les peuples contemporains le peuple hébreu comprenait des travailleurs libres et des esclaves. Face au travailleur libre, au mercenaire, face à celui qui n'avait que ses mains pour vivre, et dont aucun patron ne garantissait la subsistance, la première exigence était la justice : justice dans le contrat de travail, dans les horaires, dans le salaire, sans retenir indûment ce salaire, en le payant au contraire tous les jours.

 Comme partout à cette époque un hébreu pouvait se vendre lui-même (et en avait le droit moral) s'il manquait de ressources ou s'il était écrasé de dettes. S'il ne pouvait plus nourrir ses enfants il en était pour lui comme pour les autres pères de l'Antiquité placés dans la même situation. Il se devait si possible de les vendre à un coreligionnaire. De même des nouveaux-nés ou des petits enfants étaient exposés à la porte des synagogues afin d'être pris en charge par un juif et non par un païen. 

 Les juifs pouvaient posséder des esclaves juifs ou gentils. Ils ne déniaient à aucun d'eux son appartenance à l'humanité commune. Leurs propres prières leur rappelaient qu'eux aussi avaient été esclaves en Égypte. Ils n'acceptaient pas l'idée que les esclaves soient d'une autre nature que les hommes libres%
\footnote{Pourtant selon le Talmud (A.~\fsc{COHEN}, 1980, p. 259) les juifs n'avaient pas une grande opinion des esclaves. Ils accusaient l'esclavage d'être une source de démoralisation pour toute la maison: par le vol (esclave mâles), ou par la lubricité (esclaves femelles). Ils disaient que l'esclave est paresseux et ne gagne pas sa nourriture, qu'il est indolent, infidèle et libertin. Ils estimaient que le travailleur libre produisait deux fois plus que l'esclave : il n'y avait pas besoin de le nourrir quand il ne travaillait pas, d'autre part son ardeur au travail était stimulée par le fait qu'il ne pouvait compter sur un autre que lui-même pour le vivre et le couvert. En fait on retrouvait là tous les jugements classiques des Romains et des Grecs de l'époque sur les esclaves.}%
 : {\emph{As-tu un serviteur ? Considère-le comme un frère et ne jalouse pas ton propre sang}} (\emph{L'ecclésiaste}, XXXIII, 32). {\emph{Si j'ai privé de droits l'esclave, la servante qui se rebelle, que faire à l'heure où Dieu se dresse, Que répondre à son examen ? Car dans un ventre il les a faits comme il m'a fait, dans le sein il nous a unis}} (Job, XXXI, 13-15). 

 Cela étant dit il n'en reste pas moins qu'un esclave juif ne pouvait pas parler pour lui-même, mais seulement sous le contrôle de son maître, comme dans les autres sociétés antiques. Religieusement et civilement il était dépendant. Il était soumis à peu près aux mêmes obligations cultuelles que les femmes. À la synagogue, l'esclave circoncis mâle adulte ne comptait pas dans le quorum nécessaire pour certaines prières ou célébrations, sauf s'il n'y avait que des esclaves présents (et donc si les maitres étaient absents).

 Quant à l'esclave mâle incirconcis, il était aussi impur que tous les autres incirconcis. Cela rendait impossible sa cohabitation avec une famille juive. Il devait donc absolument être circoncis%
\footnote{Selon le Talmud, il avait droit à un délai d'un an pour accepter l'opération (ce qui peut évoquer un temps d'initiation religieuse) : s'il refusait l'opération, il était revendu à des non juifs. C'était la législation de l'Empire romain, hostile à tout ce qui évoquait une mutilation. Elle ne tolérait de circoncire que les seuls esclaves ayant consenti de manière expresse et par écrit à l'opération. Dans le cas contraire l'esclave circoncis contre son gré était libéré de droit, comme tous ceux que leur maître avait mutilés ou blessés gravement.}%
. Une fois circoncis il devait participer au culte dans tous les actes qui avaient lieu à la maison du maître, observer le sabbat, célébrer la pâque : on ne lui demandait pas d'acte de foi (orthodoxie) on lui demandait seulement le respect formel des rites (orthopraxie). S'il servait un membre du clergé, il pouvait désormais sans sacrilège manger les nourritures consacrées, la part des offrandes faites au temple qui revenait aux prêtres pour leur subsistance. Il était compté au nombre des juifs potentiels : ainsi il pouvait désormais épouser la fille de son maître, ou hériter de ce dernier. Dans ces deux dernières éventualités il était affranchi de droit, sans autre forme de procès, comme partout ailleurs.

 Selon la Tora, tout esclave devait être traité comme un hôte. Selon le Talmud, la loi punissait sévèrement le maître qui mettait à mort son esclave (c'était la même règle qu'à Rome \emph{sous l'Empire}). Celui qui avait été blessé ou mutilé, ou maltraité sévèrement, devait être libéré sans attendre : s'il avait des obligations ou des dettes, elles étaient annulées. S'il était mécontent l'esclave pouvait s'enfuir, il n'était pas poursuivi et il était interdit de le livrer à son maître. Des arrangements devaient être trouvés pour dédommager ce dernier (revente). C'était la pratique traditionnelle dans l'aire grecque, et elle s'était à cette époque répandue chez les Romains \emph{de l'Empire} où les esclaves pouvaient trouver refuge dans certains temples ou au pied des statues de l'empereur sans être poursuivis par la force publique comme esclaves fugitifs.

 Un juif ne pouvait pas être retenu comme esclave par un autre juif plus de six années : il devait alors être libéré, avec sa femme et ses enfants s'ils étaient mariés avant qu'il ne devienne esclave. Il ne devait même pas partir les mains vides. Plus que le statut d'un esclave, cela suggèrerait le statut d'un gagé pour dettes que par convention juridique un forfait de six années de travail servile libérerait de toutes ses obligations ?

 Mais l'esclave pouvait aussi refuser sa libération. Il pouvait préférer la sécurité de son emploi à la vie hasardeuse d'un pauvre, d'un journalier sans outil de travail. S'il ne pouvait pas subvenir seul à ses besoins, le libérer n'était pas lui rendre service. Il pouvait encore refuser de quitter la femme (esclave) dont son maître lui avait donné l'exclusivité, et les enfants qui leur étaient nés. Il choisissait alors de demeurer à vie dans la maison de son maître, dépendance consacrée par le percement de son oreille contre la porte (ou le pilier central) de la maison de ce dernier. Rivé à cette maison --- à cette famille, il en faisait désormais partie définitivement.

 Le Décalogue donnait aux esclaves un jour de repos par semaine (le \emph{Shabbat}). Selon le Talmud un esclave ne devait pas travailler plus longtemps qu'un travailleur libre, ni la nuit, ni à des tâches humiliantes. Il ne devait pas être soumis au travail forcé. Il (elle) devait dans tous les cas être traité avec les égards dus à un mercenaire libre qui habiterait dans la maison. Il ne devait pas être mis à la disposition du public (prostitué, acteur,~etc.) \emph{sauf si c'était son métier auparavant}. Il ou elle ne pouvait pas être prostitué sans son accord, ni contraint à des tâches d'esclave : \enquote{[...] \emph{à laver les pieds de son maître, à lui mettre ses sandales, à porter des vases pour lui dans la maison des bains, à lui prêter appui pour monter un escalier, ou à le transporter dans une litière, un fauteuil ou une chaise à porteurs, toutes choses que les esclaves font pour leur maître.}} (\fsc{COHEN}, p. 254.) Par contre il pouvait choisir de poser les mêmes actes de son plein gré : s'il se reconnaissait esclave, il n'y avait pas de faute à lui demander des actes d'esclave.

 Mais même dans ce cas, aucun rapport sexuel avec un ou une esclave n'était considéré comme insignifiant, comme une affaire entre soi (le maître) et soi (l'esclave). Si un juif voulait prendre pour concubine une captive, une prisonnière de guerre (et même une esclave achetée au marché ?) il devait lui laisser un mois de répit pour s'habituer à sa situation et apaiser sa douleur d'être loin des siens et sans moyens de résistance. Dès que le maître d'une esclave, ou l'un de ses fils, avait usé d'elle charnellement, il ne pouvait plus la revendre, quelle que soit son origine ou sa religion. Elle ne pouvait plus être libérée contre son gré, ce qui aurait signifié la jeter à la rue, puisque par définition elle était sans dot et sans famille. Le maître devait la garder et l'entretenir tant qu'il continuait d'avoir des relations sexuelles avec elle, c'est-à-dire qu'il devait la traiter comme une concubine non esclave. Elle devait être laissée libre de s'en aller à sa guise et partir en femme libre.

 Du fait de ces multiples contraintes, un esclave juif n'avait qu'une valeur médiocre pour un de ses coreligionnaires, d'où le dicton du Talmud : {\emph{quiconque acquiert un esclave hébreu se donne un maître à lui-même}}. Il était plus avantageux de le vendre à des non juifs qu'aucune règle n'obligeait à le libérer au bout de six ans, ni à ménager son corps. Mais en ce cas le devoir des siens était de le racheter parce qu'il serait condamné à vivre dans l'impureté qu'impliquait le commerce continuel avec les incirconcis, et ne pourrait plus suivre la Loi. Par contre, il n'y avait pas à libérer un esclave non hébreu au bout de 6 ans. L'obligation de libérer tous les esclaves quels qu'ils soient ne revenait que les années jubilaires, tous les 50 ans.

 






% Le 9 mars 2015 :
% ~etc.
% ~\%
% Anquité
% Moyen Âge
% Romain
% Empire
% Droit


\chapter{Les lois d'Auguste}

 Au moment où s'installait l'Empire de Rome, durant le dernier siècle avant notre ère%
%[1]
\footnote{Sources :\\
\fsc{CARRIE} Jean-Michel, \fsc{ROUSSELLE} Aline, \emph{L'Empire romain en mutation, des Sévères à Constantin}, 192-337, 1999.\\
\fsc{DELACAMPAGNE} Christian, \emph{Une histoire de l'esclavage, de l'Antiquité à nos jours}, 2002.\\
Collectif, \emph{Religions de l'Antiquité}, 1999, 592 p.\\
\fsc{LE ROUX} Patrick, \emph{Le Haut-Empire romain en Occident, d'Auguste aux Sévères}, 1998.\\
\fsc{PETIT} Paul, \emph{Histoire générale de l'Empire romain, 3, le Bas-Empire} (284-395), 1974.\\
\fsc{PUCCINI-DELBEY} Géraldine, \emph{La vie sexuelle à Rome}, 2007.\\
\fsc{ROUSSELLE} Aline, \emph{La contamination spirituelle, science, droit et religion dans l'Antiquité}, 1998.\\
\fsc{SARTRE} Maurice, \emph{Le Haut-Empire romain, Les provinces de Méditerranée orientale d'Auguste aux Sévères}, 1991.\\
\fsc{VEYNE} Paul, \emph{La société romaine}, 1991.\\
\fsc{VEYNE} Paul, \emph{Le pain et le cirque, sociologie historique d'un pluralisme politique}, 1976.}
les citoyens romains avaient tendance à délaisser le mariage avec les citoyennes et à lui préférer d'autres manières de se procurer des héritiers. Cet état de fait était favorisé par bien des facteurs dont les flots d'esclaves que les conquêtes militaires déversaient alors. 

 Même si le mariage romain reposait comme le nôtre (son héritier direct) sur le consentement des époux, et même si les maris conservaient une autorité sans partage sur leurs enfants en cas de divorce, à cette période ils n'avaient plus qu'une autorité limitée sur leurs épouses. Celles-ci restaient sous l'autorité de leur père ou de leur tuteur, même quand elles étaient mariées (mariage « \latin{sine manu} », sans \emph{la main}, l'autorité). Et elles avaient le droit de divorcer de leur propre initiative même si dans ce cas elles devaient abandonner une part ou la totalité de leur dot à leur mari --- comme dédommagement ? Comme contribution aux frais d'élevage de leurs enfants communs ?
 
 Le droit au divorce allait de soi pour les romains au tournant de notre ère, étant donnée leur conception du mariage qui exigeait une \emph{affectio maritalis} continuelle, c'est-à-dire une volonté continuelle de vivre ensemble chez deux personnes libres. L'essence du mariage ne reposait pas sur un acte initial, une promesse à tenir ensuite, mais sur l'existence d'une volonté continuée de vivre ensemble comme mari et femme. Dès que cette volonté disparaissait le mariage était dissous de plein droit, quelles que soient les formes que prenait l'affirmation publique de la volonté de l'un ou de l'autre de rompre l'union, et quelles que soient les conséquences. 

 Par contre, les esclaves n'avaient juridiquement plus de parents, plus de lien juridique avec une famille, donc aucun recours face à leur maître, qui disposait à leur encontre d'un puissant arsenal de punitions et de récompenses. Les esclaves concubines ne pouvaient donc se permettre de faire mauvaise figure à leur maître, ni de le quitter contre son gré. S'il se lassait d'elles il n'avait aucune dot à rendre à un beau-père mécontent : il les vendait au marché ou les donnait à un de ses esclaves ou de ses clients. S'il les affranchissait, il était de droit leur patron et leur tuteur jusqu'à sa mort. Elles ne pouvaient donc jamais fuir son autorité. Si elles passaient d'un amant à un autre, il avait tous les moyens de les sanctionner.

 Mais il est possible aussi que le nombre de citoyennes nubiles soit devenu insuffisant pour que chaque fils de famille puisse trouver une épouse parmi les citoyennes de son rang. Les citoyens aban\-don\-naient-ils un nombre de fillettes excessif ? C'est qu'une fille légitime devait être dotée si son père voulait la marier dans les règles, et il perdait la face s'il s'y refusait. Ne pas doter une fille selon son rang eut été avouer qu'on n'en avait pas les moyens. On pouvait aussi se demander si elle présentait un défaut caché, si planait sur elle une suspicion d'illégitimité, ou encore si son père ne l'aimait pas de manière excessive, pour lui-même (les Romains n'avaient aucune complaisance envers les désirs incestueux, dont la réalisation leur paraissait un sacrilège pouvant entraîner la colère des dieux contre tous les humains). Il est donc tout à fait imaginable que beaucoup de citoyens aient refusé de s'encombrer de filles s'ils prévoyaient qu'ils ne pourraient pas les marier de manière profitable pour leurs alliances politiques et économiques%
% [2]
\footnote{En Asie les mêmes causes produisent \emph{aujourd'hui} encore les mêmes effets. En Inde la nécessité de doter les filles, en Chine le fait qu'elles sortent de leur famille en se mariant et ne peuvent donc plus prendre en charge leurs parents âgés (la plupart ne jouissant pas d'une retraite), conduisent avec l'aide des échographies (mais l'infanticide à la naissance aboutit au même résultat) à un sex-ratio très déséquilibré, avec un fort excès de garçons, et à des trafics de femmes jeunes et pauvres.}%
. 

 Or, à Rome comme dans les cités grecques, seuls les citoyens de naissance \emph{légitime} pouvaient exercer des magistratures et seul les citoyens \emph{ingénus} (nés libres) pouvaient porter les armes. Les autres, les affranchis, ceux qui avaient commencé leur vie comme esclaves, et a fortiori ces derniers, étaient considérés comme sans courage, prêts à choisir de vivre en esclave plutôt qu'à mourir pour la cité. 

 L'empereur était d'abord un chef de guerre : du temps de la république romaine, le titre d'\latin{Imperator} désignait le {\emph{commandant en chef vainqueur devant l'ennemi}}. C'est par ses succès militaires que César avait conquis le pouvoir. Auguste, son successeur, était chef des armées. Sa stratégie de conquête du pouvoir exigeait des victoires militaires, et d'abord contre ses rivaux. Accepter une armée faible était consentir à sa propre défaite à plus ou moins proche échéance. Il ne pouvait donc se satisfaire d'une armée d'affranchis. Il ne pouvait pas non plus envisager de confier son armée à des magistrats (officiers) dont on ne connaissait pas le père. Il lui fallait un maximum de citoyens ingénus et beaucoup de citoyens de naissance légitime. 

 Un autre facteur a pu avoir une importance significative dans les décisions d'Auguste : chez les aristocrates qu'il avait dépouillés d'une grande part de leur pouvoir antérieur, la morale stoïcienne occupait à cette époque une position dominante. Chez les personnes de qualité, la maîtrise des comportements était devenue une fin en soi, ce qui avait entraîné des effets jusque dans la morale conjugale : en effet \emph{selon les stoïciens} le mariage est le lieu unique de la sexualité autorisée et vertueuse. Tous les comportements sexuels qui n'y trouvent pas place sont moralement condamnables. La chasteté pré conjugale des garçons est aussi souhaitable que celle des filles, tout comme le maintien de leur virginité jusqu'au mariage. Pour Sénèque, contemporain de Néron : {\emph{l'injustice la plus grave envers une épouse est d'avoir une maîtresse}%
% [3]
\footnote{\emph{Lettres à Lucilius}, XV, 94, 26, et 95, 37, \fsc{SéNèQUE}}%
}. Pour lui, {\emph{c'est être adultère envers sa propre femme que de l'aimer d'un amour trop ardent}}. Quant à Épictète, il accusait l'adultère d'être contraire à la nature, car les êtres humains sont nés pour être fidèles%
% [4]
\footnote{\fsc{Épictète}, 2, 4 ; 3,7 ; 16, 21.}%
. 

 Pour Musonius Rufus, moraliste et philosophe du premier siècle, {\emph{il faut que ceux qui ne sont ni voluptueux ni mauvais considèrent que les plaisirs amoureux sont justes dans le seul mariage et pour procréer, parce que licites, tandis que ceux qui assouvissent le seul plaisir sont injustes et criminels, même dans le mariage}%
% [5]
\footnote{Musonius Rufus, VII}%
}. Pour lui encore, {\emph{s'il semble \emph{[à quelqu'un qu'il n'est]} ni honteux ni déplacé qu'un maître ait des relations avec sa propre esclave, surtout s'il se trouve qu'elle n'est pas mariée, que celui-ci considère comment il apprécierait que son épouse ait des relations avec un esclave mâle. Cela ne paraîtrait-il pas totalement intolérable \emph{[...]} ? \emph{[...]} Quel besoin y a-t-il ici de dire que c'est un acte de dévergondage et rien d'autre pour un maître d'avoir des relations avec un esclave ? Tout le monde le sait}%
%[6]
\footnote{Musonius Rufus, VII}%
}. Comme on peut s'y attendre compte tenu de ces prémisses, il pensait que les relations sexuelles entre hommes sont {\emph{une chose monstrueuse et contraire à la nature}}. Jusqu'à quel point ces idées s'incarnaient-elles dans les comportements ? Même les stoïciens les plus dévots pouvaient à l'occasion s'écarter de leurs propres règles. Cela ne veut pas dire qu'elles n'avaient pas d'importance. 


 Selon Géraldine \fsc{PUCCINI-DELBEY} (\emph{La vie sexuelle à Rome}, 2007, p. 69-70), {\emph{l'homme romain doit se montrer officiellement un bon époux et respecter son épouse} [...]}. D'une morale civique, les Romains passent ainsi à une {\emph{morale du couple}}, changement qui se produit en un siècle ou deux. \tempuwave{Paul \fsc{VEYNE} résume} cette mutation de la manière suivante : la première morale disait : {\emph{se marier est un des devoirs du citoyen}} ; la seconde : {\emph{si l'on veut être un homme de bien, il ne faut faire l'amour que pour avoir des enfants ; l'état de mariage ne sert pas à des plaisirs vénériens}}. La seconde morale postule une amitié, une affection durable entre deux personnes de bien qui ne font l'amour que pour perpétuer l'espèce. C'est, aux yeux de l'historien, la naissance du mythe du couple, du {\emph{mythe de l'amour conjugal}}. La femme, jusque là nécessaire pour faire des enfants et arrondir le patrimoine, devient une amie, {\emph{la compagne de toute une vie}}. Son époux doit la respecter comme une amie qui lui est inférieure. \tempuwave{Paul \fsc{VEYNE} résume} (\emph{Histoire de la vie privée}, Tome I, p. 47) cette évolution en affirmant que les Romains sont passés d'une {\emph{bisexualité de viol}} à une {\emph{hétérosexualité de reproduction}}, une morale de la conjugalité que le christianisme reprendra par la suite à son compte.

 Si l'on en croit Emmanuel \fsc{TODD}, il s'agissait du déclin du patriarcat romain et de la résurgence de la famille nucléaire originelle, souterrainement agissante dans les représentations romaines malgré le poids du modèle patriarcal incarné dans le Droit romain, en particulier dans les couches de la populations qui détenaient peu de pouvoir et de biens à transmettre. 


\section{Promotion des naissances ingénues}

 Un maître Romain pouvait affranchir les enfants qu'une esclave lui avait donnés. Ils ne seraient jamais ses enfants légitimes, puisqu'ils avaient été esclaves, mais ils devenaient des citoyens. Ils étaient ses affranchis, et il était leur patron, conservant de ce fait une forte autorité sur eux. Il pouvait enfin les désigner comme héritiers. S'il reconnaissait officiellement dès leur naissance les enfants de ses esclaves concubines, ils étaient considérés comme des ingénus qui n'avaient jamais été esclaves, et comme ses enfants, même s'ils étaient illégitimes. 

 Auguste voulait augmenter le nombre des jeunes ingénus aux dépens de celui des jeunes affranchis, aussi a-t-il décidé%
% [7]
\footnote{\latin{Leges Juliae de adulteriis coercendis} et \latin{De maritandis ordinibus} de \mbox{18 av. J.-C.}, et \latin{lex Papia Poppaea} de \mbox{9 ap. J.-C.} L'empereur était l'une des sources principales du Droit. Il était donc de coutume d'écrire et de dire que {\emph{Auguste a décidé que...}}, même quand il se bornait à entériner une proposition que lui avaient faite les juristes qui travaillaient dans ses bureaux (parmi les meilleurs de son époque).}
que :
\begin{enumerate}
%A)
\item seuls pourraient hériter de leurs parents éloignés ou de personnes non apparentées les citoyens (mariés ou non) qui avaient donné le jour à au moins un enfant ingénu, les citoyennes (mariées ou non) qui avaient donné le jour à trois enfants ingénus, et les affranchies qui, après leur affranchissement, avaient donné le jour à quatre ingénus, enfants déclarés, reconnus, et non abandonnés (légitimes ou non, vivants ou morts, nés viables ou non, garçons ou filles). On peut penser que cette loi a donné de fortes chances d'être élevés aux trois premiers enfants de chaque citoyen, même aux filles ;
% B)
\item les citoyennes qui avaient donné le jour à trois enfants ingénus (quatre pour les affranchies), et qui n'avaient pas de \latin{pater familias} (père ou patron), seraient dispensées de tutelle. À partir de leurs 25 ans (l'âge de la majorité légale des hommes \latin{sui juris}), elles pouvaient gérer elles-mêmes leurs affaires et se marier ou se remarier à leur gré : elles aussi pouvaient donc devenir juridiquement majeures. 
\end{enumerate}


\section{Répression de l'adultère féminin}

 Sans la fidélité des épouses la paternité des maris est incertaine. Les adultères c'est-à-dire les relations \emph{passagères} entre une femme mariée et un autre homme que son mari dérangeaient le bon ordre une société construite sur la base des lignées et de leurs alliances (ce qui en soi n'implique pas le patriarcat). Une femme mariée qui avait des relations sexuelles avec son propre esclave était aussi adultère que si son amant avait été un homme libre. Dans un ordre d'idée assez proche, si une femme libre (célibataire ou mariée, citoyenne ou non) prenait pour amant un esclave qui ne lui appartenait pas, elle était elle aussi coupable d'\latin{adulterium} alors qu'aucun mari n'était en jeu. Si malgré les avertissements du maître de l'esclave la coupable persistait dans ses relations elle devenait elle-même l'esclave du maître de son amant%
% [8]
\footnote{\emph{Sénatus-consulte} de Claude, \mbox{52 ap. J.-C.} … comme si le fait qu'une femme aille vers un homme pour lui demander un rapport sexuel signifiait qu'elle se met sous son autorité.}%
.

 Par contre le fait qu'une épouse quitte son mari pour aller vivre \emph{durablement} avec \emph{un seul} autre homme libre ne tombait pas sous le coup de la loi. C'était un droit que possédait toute citoyenne romaine, mais jusqu'à Auguste la dot de la femme qui agissait ainsi restait dans les mains de celui qu'avait choisi son père, ce qui pouvait l'empêcher de trouver un nouveau parti avantageux, et donc de concevoir de nouveaux enfants légitimes. C'est pourquoi Auguste a dans le même mouvement facilité le divorce et sanctionné lourdement les adultères. Il a décidé que les citoyennes (les actes des autres femmes n'importaient pas puisqu'ils n'avaient pas d'effet sur l'ordre social) qui quitteraient leur mari pour une relation durable (un nouveau mariage ou un concubinage stable) garderaient le bénéfice de leur dot. 

 Une femme convaincue d'adultère devenait infâme. Elle perdait le droit au manteau des matrones, ce qui rendait public son statut. Une infâme n'était pas astreinte au devoir de fidélité et ses relations sexuelles ne regardaient plus qu'elle, mais elle ne pouvait pas s'appuyer sur la loi pour protéger son intégrité corporelle contre les agressions sexuelles. Celui qui la violait n'attentait plus l'honneur d'aucun homme. Son mariage était automatiquement rompu et elle n'avait pas le droit d'en contracter un autre. La moitié de sa dot et le tiers de ses biens passaient au fisc impérial, et elle était exilée, assignée à résidence dans une île%
% [9]
\footnote{… par exemple en Corse.}%
. Son complice masculin était également frappé d'infamie et puni de la même façon : confiscation de la moitié de ses biens et relégation dans une autre île. Une femme accusée d'adultère pouvait éteindre les poursuites engagées contre elle en prenant l'initiative de se faire inscrire sur la liste des prostituées : elle n'en était que plus sûrement frappée d'infamie. 

 Tout mari trompé devait sans tarder envoyer à son épouse une lettre de répudiation, sans quoi il était considéré comme proxénète et devenait infâme, et ses enfants aussi. Cela s'accompagnait de la confiscation de tout ou partie de leur fortune. L'obligation de dénonciation s'appliquait aussi au père et aux frères de l'épouse : tous ceux qui avaient connaissance d'un adultère se devaient de le dénoncer sous peine d'être eux aussi condamnés à l'infamie.

 Un mari trompé n'avait pas le droit de tuer son épouse, puisqu'il n'avait pas autorité complète sur elle (elle dépendait de son propre \latin{pater familias}). Par contre il pouvait tuer l'amant en toute impunité si celui-ci était un infâme, ou un esclave, ou un affranchi attaché à sa famille (son affranchi ou celui de son épouse, ou celui de leurs pères ou mères respectifs, ou de l'un ou l'autre de leurs enfants). Dans tous les autres cas de figure, ou bien s'il tuait son épouse, il commettait certes un homicide, mais la loi lui accordait de larges circonstances atténuantes. 

 Contrairement au mari et à la condition qu'il soit aussi son \latin{pater familias}, le père de l'épouse infidèle avait le droit de tuer sa fille, à la condition de tuer son amant aussi. Sinon il pouvait être accusé de meurtre.

 Jusqu'à quel point cette législation a-t-elle été appliquée ? Dans quelle mesure a-t-elle seulement eu un rôle dissuasif à l'encontre de la prostitution clandestine ? On pouvait compter sur le fisc pour tout mettre en œuvre pour faire rentrer de l'argent dans les caisses de l'état, et celui qui venait des adultères, quoique impur, était aussi bon à prendre que celui des taxes sur les vespasiennes ou sur les affranchissements d'esclaves, aussi les dénonciations étaient-elles encouragées financièrement. Elles avaient d'autant plus d'intérêt pour les finances de l'empereur et celles des dénonciateurs qu'elles concernaient des personnes plus riches. Mais le Droit romain prévoyait que les accusateurs qui ne parvenaient pas à prouver leurs accusations étaient très sévèrement punis, ce qui pouvait mettre un frein aux accusations gratuites lorsque les juges n'étaient pas corruptibles . 


\section{Promotion du mariage}

 Puisque le concubinage pouvait accroître le nombre des citoyens ingénus, Auguste n'a rien fait pour l'interdire aux hommes mariés. Par contre pour augmenter le nombre des enfants légitimes il a promu le mariage. Il a décidé que : 
\begin{enumerate}
% A)
\item tous les citoyens qui n'étaient pas mariés entre 25 et 60 ans et toutes les citoyennes qui n'étaient pas mariées entre 20 et 50 ans devaient payer un impôt spécial. Au-delà de ces âges le remariage n'était pas encouragé%
%[10]
\footnote{\emph{La vie sexuelle à Rome}, Géraldine \fsc{PUCCINI-DELBEY}, 2007, p. 48.}
: il n'était plus convenable de convoler alors que la conception des enfants était hors d'atteinte pour les femmes, et que les hommes étaient trop âgés pour assumer l'éducation de leurs fils. Cela devenait un motif de scandale ;
% B)
\item les citoyennes qui prendraient l'initiative de divorcer pourraient garder l'intégralité de leur dot, de telle façon qu'elles retrouvent un mari sans difficulté et continuent d'enfanter des citoyens légitimes ;
% C)
\item pour parer à l'éventualité qu'il n'y ait pas suffisamment de citoyennes nubiles pour tous les citoyens célibataires, Auguste a promu une solution de remplacement : il reconnaissait comme valide l'union des citoyens avec une de leurs propres esclaves, à la condition qu'ils l'aient affranchie pour l'épouser avant l'âge de douze ans, âge minimum légal du mariage pour les filles (même si l'âge moyen au mariage des femmes était beaucoup plus tardif). Affranchies pour être épousées, ces fillettes devenaient aussitôt des citoyennes, ce qui était un grand privilège. De cette façon, leurs enfants à venir naîtraient de deux citoyens libres et mariés. Ils seraient de plein droit citoyens, ingénus, et de naissance légitime. 
\end{enumerate}

 Comme toutes les autres femmes libres, ces affranchies avaient le droit de divorcer le plus simplement du monde, en quittant leur mari. Mais Auguste pénalisait l'exercice de ce droit : si elles quittaient leur époux \emph{contre son gré}, elles perdaient le \latin{connubium}, le droit d'épouser un (autre) citoyen romain%
% [11]
\footnote{... parce que ce faisant, elles se soustrayaient à l'autorité de leur tuteur. Le maître d'une esclave devenait en effet de droit son tuteur en même temps que son patron quand il l'affranchissait.}%
, ce qui implique que leurs enfants à venir ne seraient pas des citoyens. Elles étaient donc incitées à s'acquitter de leur tâche d'épouse conformément aux désirs de leur « maître et seigneur ».

 Le concubinage monogame stable entre deux citoyens produisait à peu près les mêmes effets qu'un mariage. Pour contracter mariage il n'y avait pas de cérémonie officielle non plus. La différence la plus significative entre le concubinage et le mariage était l'absence de dot, même si chez les Romains il existait des mariages en bonne et due forme sans dot (à la différence des Grecs). Pendant un bon millénaire encore, la seule preuve d'une union, en dehors de la parole des témoins, sera le contrat notarié qui enregistrait la dot%
% [12]
\footnote{On peut en rapprocher la \emph{Kétouba}, contrat de mariage écrit sans lequel un mariage juif n'est pas valide.}%
. Le concubinage était selon \fsc{VEYNE} le mariage des gens de peu, des pauvres, des gens sans importance, de ceux qui n'appartenaient pas à une famille aristocratique et qui n'avaient aucun espoir d'arriver un jour à des positions en vue. Les enfants (reconnus par leur père) qui en naissaient n'étaient pas considérés comme illégitimes tant que leurs parents n'étaient pas frappés d'un interdit de mariage (inceste, infamie,~etc.). 

 En ce qui concerne les interdits de mariage, cette période, où les mariages entre cousins étaient autorisés (contrairement aus usages romains antérieurs) innove sur un point : sous le règne de l'Empereur Claude (41-54 de notre ère), un sénatus-consulte (une réponse du Sénat, consulté sur un point de Droit) autorise le mariage d'une nièce et de son oncle paternel (pratique commune dans l'aire héllénistique). Cette exception aux interdits de mariage était un cadeau fait à l'Empereur, désireux d'épouser sa nièce. Elle semble n'avoir jamais complètement perdu chez les romains son aura de scandale. La tendance ensuite s'inversera, et on constate un élargissement du cercle des parents prohibés à l'époque des juristes classiques (de la fin du \siecle{2} au début du \siecle{3}). Le Droit positif étend les interdictions matrimoniales aux \latin{adfines} (parents par alliance), et au milieu du \siecle{4} la \latin{fratris filia}, puis la \latin{consobrina} (cousine)sont à nouveau interdites, ainsi que les \latin{adfines} de même génération.

 Une preuve du fait qu'Auguste est sans doute parvenu à peu près aux buts qu'il s'était fixés (... mais peut-être aussi à d'autres objectifs non recherchés initialement, mais appréciés tout de même ?) est que ses lois sur le mariage vont être appliquées pendant trois siècles et demi. 

 Cela dit, pour obtenir les enfants réglementaires, les Romains et les Romaines ne s'y prenaient pas toujours comme on s'y attend lorsqu'on a l'imaginaire façonné par les siècles d'indissolubilité du mariage ultérieurs. La fréquence des divorces et des remariages était grande, et selon \fsc{VEYNE} (2001) c'était souvent à plusieurs épouses que les hommes demandaient les enfants dont ils avaient un impérieux besoin (comme aujourd'hui). Des maris divorçaient pour prêter celles qui avaient prouvé leur fécondité à leurs amis, qui les épousaient à leur tour pour légitimer les enfants attendus. Les premiers pouvaient ré-épouser à nouveau les femmes ainsi prêtées une fois accompli leur service génésique. Lorsque la fécondité du futur époux était incertaine, ils pouvaient même les prêter déjà enceintes. 


\section{Limitation des affranchissements}

 Pour que le corps social ne soit pas trop vite envahi par les affranchis (comme par autant d'immigrés mal assimilés et provoquant chez les « vieux Romains » le sentiment désagréable de n'être plus chez eux à Rome), Auguste a restreint le droit des maîtres d'affranchir leurs esclaves%
% [13]
\footnote{... et rendu plus difficiles les fraudes à la citoyenneté romaine (statut qui était avantageux) ? Pour faire à coup sûr de son fils mineur un citoyen romain, un pérégrin pouvait le vendre à un citoyen romain non infâme (par exemple à un associé en affaires). Après quoi celui-ci n'avait plus qu'à l'affranchir. Avant cette réglementation, cela pouvait aller vite. Par contre s'il fallait vivre sous le statut d'esclave jusqu'à l'âge de trente ans c'était nettement moins intéressant.}%
. Il a décidé que ceux-ci ne pourraient obtenir la citoyenneté romaine que s'ils étaient affranchis après leurs 30 ans, à la condition que leur maître ait plus de 20 ans, et qu'il ne les affranchisse pas en masse. 

 Quand l'une ou l'autre de ces conditions n'était pas remplie, leur affranchissement leur demeurait acquis, mais ils n'obtenaient que le droit \emph{latin}, nettement moins favorable que la citoyenneté romaine. L'objectif de la mesure était de limiter l'acquisition de la nationalité romaine à ceux qui avaient eu suffisamment de temps pour s'assimiler. Il est vrai que pour devenir un citoyen romain, il suffisait aux hommes de statut latin de faire un enfant à une citoyenne.
 
 

% Le 09.03.2015 :
% Empire, Empereur
% Romain
% Le 03.03.2015 :
% Antiquité
% Moyen Âge
% ~etc.
% ~\%



\chapter[Évolutions du droit civil romain sous l'Empire]{Évolutions\\du droit civil romain sous l'Empire}


\section{Des droits pour les femmes}

 On a vu plus haut qu'Auguste avait décidé que les citoyennes qui avaient donné le jour à trois enfants ingénus (quatre pour les affranchies), qui avaient 25 ans ou plus, et qui n'avaient plus de \latin{pater familias}, étaient dispensées de tutelle. Elles pouvaient gérer elles-mêmes leurs affaires financières ou commerciales, et se marier ou se remarier à leur gré. Pour la première fois à Rome des citoyennes pouvaient ne plus être mineures à vie et accéder à la majorité légale. Cela concernait les citoyennes sans \latin{pater familias}, les orphelines de père, celles dont le père ne possédait pas ou plus les droits d'un \latin{pater familias} (esclave affranchi, citoyen déchu de ses droits par une condamnation, vendu comme esclave,~etc.), et les affranchies dont le patron était décédé ou condamné à l'infamie. 

 Mais elles n'accédaient à la majorité légale que par leur utérus, et cela n'en faisait pas pour autant les égales des hommes. Cela ne leur donnait en effet aucun droit sur la conduite d'autrui, même pas sur leurs propres enfants mineurs, qui recevaient un tuteur. Cela ne leur donnait pas non plus la plénitude des droits civiques : elles ne pouvaient ni voter dans les assemblées politiques, ni porter plainte pour autrui ni plaider devant les tribunaux, ni donner la plénitude du statut de citoyen à ceux de leurs esclaves qu'elles affranchissaient, ni adopter un enfant,~etc.
 
 
\section{Des droits pour les enfants}

 Au cours du premier siècle de notre ère une loi a autorisé un fils ou une fille en puissance de \latin{pater familias} à porter plainte contre ce dernier si celui-ci abusait de son pouvoir. Le magistrat pouvait décider d'émanciper cet « enfant » sous puissance (qui pouvait avoir quarante ans et plus). Si nécessaire il pouvait condamner le père à subvenir aux besoins de cet « enfant » qu'il venait d'émanciper mais qui ne possédait rien en propre (sauf sa solde lorsqu'il était militaire, ou son salaire s'il était fonctionnaire). 
 
 C'est à cette époque qu'a été mise au point la notion de \emph{prestation alimentaire}, distincte des problèmes de transmission de pouvoir, de succession et d'héritage. Cette prestation était fondée sur la seule \latin{pietas}, la piété (familiale), c'est-à-dire l'affection \emph{naturelle} qui doit régner entre tous les membres d'une même famille \emph{biologique}, liés par le sang, indépendamment des liens juridiques. La prestation était due entre ascendants et descendants, dans les deux sens, sans tenir compte du côté paternel ou maternel de la parenté ni de la légitimité des unions ni des filiations. Les mères y étaient astreintes au même titre que les pères. L'obligation alimentaire n'était supprimée ni par le refus de reconnaître un enfant, ni par son émancipation : « \emph{qui fait l'enfant doit le nourrir} ». 

 Peu à peu l'émancipation a (donc ?) cessé d'impliquer l'exclusion hors de la famille, et l'enfant émancipé a progressivement obtenu de la justice un droit à l'héritage de son père. De ce fait, alors qu'elle avait commencé par être une sanction, l'émancipation est devenue un cadeau qu'un père faisait à son fils.


\section{Des droits pour les esclaves}

 Face à une vieille noblesse républicaine, très réticente devant le nouveau régime, les premiers Empereurs romains ont organisé l'administration sur la base de leur propre domesticité (qu'ils se sont transmise de l'un à l'autre par héritage). Ils ont fait gérer l'Empire par leurs milliers d'esclaves et d'affranchis, qui par nécessité leur étaient tout dévoués. Les plus qualifiés de ces employés ont pu aider le droit à s'infléchir en faveur des esclaves. 

 Chacun des Empereurs successifs savait comment il avait lui-même conquis le pouvoir, souvent dans un climat de guerre civile. En effet en l'absence de mécanismes clairs et acceptés de dévolution du pouvoir, à côté des légions qui soutenaient la candidature de leur général (dans l'espoir de se partager les bénéfices de sa victoire), les postulants à l'Empire avaient recours à leurs armées d'esclaves pour gagner leur bataille plus ou moins sanglante contre leurs rivaux. Ils savaient de quel secours leur avaient été ces hommes qui n'avaient aucun lien juridique avec la cité et que la loi contraignait à ne rendre de comptes qu'à leur maître. Ils ne pouvaient donc pas accepter que leurs rivaux potentiels, c'est-à-dire beaucoup de monde, acquièrent trop de puissance. Ils avaient un intérêt direct et urgent à s'immiscer dans la relation entre les maîtres et les esclaves, et à faire en sorte que ces derniers deviennent des sujets de leur Empire comme les autres. 

 Les lois de l'Empire ont essayé de manière répétitive d'empêcher les ventes de citoyens ingénus mineurs à quelque âge que ce soit. Elles n'y sont jamais totalement parvenues, mais elles les déclaraient invalides, ce qui permettait à l'esclave qui prétendait être né libre et n'avoir pas donné son accord à son asservissement de contester en justice son statut d'esclave sans être torturé comme il l'était à chaque fois qu'il formulait toute autre plainte. Il lui fallait évidemment apporter ensuite la preuve de son ingénuité : chaque cité tenait ses registres d'état civil pour les enfants de naissance libre et reconnus.

 Claude, Empereur de 41 à 54 de notre ère, a interdit la mise à mort des esclaves vieux ou malades et décidé que les esclaves qui avaient été abandonnés par leurs maîtres (pour impotence, maladie, chômage,~etc.) seraient définitivement libres, et que leurs anciens maîtres ne pourraient plus mettre en avant leurs droits antérieurs s'ils survivaient à leur abandon. 

 Néron (fils adoptif de Claude) a donné au Préfet de la ville de Rome le pouvoir de recevoir et d'instruire les plaintes des esclaves contre les injustices de leurs maîtres. Jusque là ils n'avaient aucun droit de porter plainte pour eux-mêmes. À partir de ce moment ils ont eu un recours contre les mauvais traitements.

 La \latin{lex petronia} a subordonné à l'accord du Préfet la vente des esclaves aux proxénètes ou aux \emph{lanistes} (organisateurs des jeux du cirque). En effet lorsque les esclaves concernés n'étaient pas volontaires ces ventes étaient des sanctions très sévères. En ce qui concerne la vente au laniste il s'agissait en pratique d'une quasi-condamnation à mort.

 En 83, sous Domitien, un sénatus-consulte a interdit la castration des esclaves et ordonné de confisquer la moitié des biens des maîtres qui se livreraient à cet exercice. En 138 sous Hadrien les sanctions pour le même délit ont été aggravées : cela montre à la fois la persistance de la politique impériale, et celle des pratiques interdites par ces décisions. 

 À cette époque cela faisait très longtemps que l'assassin d'un esclave était poursuivi comme homicide, sauf quand il s'agissait de son maître lui-même. Au milieu du \siecle{2} de notre ère, l'Empereur Antonin le Pieux a qualifié d'homicide la mise à mort d'un esclave sur l'ordre de son maître, quand celui-ci n'avait pas d'abord obtenu l'accord d'un magistrat qualifié (Claude l'avait également interdit, ce qui tendrait à prouver là aussi que ce crime n'avait pas disparu). Désormais un maître pouvait être condamné pour ce motif : l'exercice de la justice domestique dans le cadre de la \latin{familia} était placé sous le contrôle de l'État.

 D'autre part les esclaves n'étaient plus tenus aussi rigoureusement que par le passé à l'écart du mariage, ou du moins de la vie en couple mixte stable. On ne regardait plus comme une absurdité de parler d'eux en termes de mari et de femme, même si pour un juriste pointilleux il ne s'agissait que d'une cohabitation%
% [1]
\footnote{\latin{Contubernium}, soit le fait de partager la même « chambrée » ou le même « poste d'équipage ».}
sans effets légaux : en effet, les droits du maître sur les deux partenaires et sur leurs enfants restaient entiers. Cette tolérance était-elle due à une diminution de l'approvisionnement du marché aux esclaves, qui donnait de la valeur aux enfants que ces derniers pouvaient concevoir et élever (et le moins cher et le plus sûr moyen pour que ces enfants soient aussi bien soignés qu'il était possible, était que leurs géniteurs les élèvent eux-mêmes) ? Ou bien était-ce un effet du stoïcisme, dont la morale conjugale triomphait sous le Haut Empire, et qui soulignait l'appartenance des esclaves à l'humanité commune ({Épictète} lui-même avait été esclave) ? Ou les deux ? 
 
 
\section{Tous Romains, tous égaux ?}

 En 212 de notre ère l'Empereur Caracalla accorde à tous les habitants de l'Empire la citoyenneté romaine. Ce faisant il leur impose aussi l'institution du \latin{pater familias} et les lois romaines sur le mariage (bien mitigées par rapport à ce qui se passait sous la République). 

 Cela aurait pu entraîner qu'il n'existe plus que deux statuts, celui de citoyen libre et celui d'esclave, si l'on n'avait pas constaté le renforcement de l'aristocratie et de ses privilèges, et s'il n'y avait pas eu de plus en plus de barbares étrangers, de peuples allogènes à l'intérieur même des frontières de l'Empire, immigrés qui obéissaient à leurs propres droits et qui assumaient les activités militaires que ne voulaient plus faire les citoyens romains : ce qu'on désigne sous le nom de « grandes invasions », et qui ont commencé dès le \siecle{3} de notre ère. 

 La distinction entre les nobles, {\latin{clarissimi}} (« les plus beaux, les plus illustres ») et les autres, tous qualifiés de {\latin{humili}} (« humbles ») ou de {\latin{pauperes}} (pauvres), même quand ils étaient financièrement à l'aise, mais sans pouvoir civique, a été affirmée de plus en plus fortement. C'étaient les sénateurs et les chevaliers, ainsi que leurs familles et leurs descendants pendant les trois premières générations, qui suivaient l'exercice de la magistrature que leur avait accordé leur titre : cette noblesse était héréditaire mais se perdait si on ne la retrempait pas périodiquement dans l'exercice des responsabilités (comme dans la Chine ancienne).

 Les nobles étaient les seuls citoyens romains qui n'avaient pas reçu d'Auguste le droit d'épouser des affranchies : la protection de leur pureté familiale était la seule qui comptait vraiment pour le sort de la Cité. À partir de Caracalla, ils sont aussi les seuls à conserver la plénitude des droits traditionnels des citoyens romains. Jusque là, aucun de ces derniers ne pouvait être soumis à la torture judiciaire, à la « \emph{question} ». Désormais cette protection n'était plus reconnue qu'aux seuls aristocrates, comme s'ils étaient devenus les seuls authentiques représentants des vieux Romains : les nobles du Moyen Âge hériteront de cette protection. Ce que la citoyenneté romaine avait gagné en extension, elle l'avait donc perdu en consistance, et sur un point très significatif le statut des citoyens ordinaires se rapprochait désormais de celui des esclaves.

 Caracalla n'en étendait pas moins à tous les \emph{pères} de l'Empire l'interdiction faite depuis fort longtemps aux citoyens romains de vendre leurs enfants nés libres. Dioclétien a interdit à nouveau aux pères \emph{et aux mères} de vendre leurs enfants ... sans plus de succès que Caracalla. Car il ne suffisait pas d'interdire ces ventes, il aurait fallu en supprimer les causes. En effet la demande d'esclaves n'avait pas disparu, pas plus que la misère. Pour les prolétaires, la vente de leurs enfants demeurait l'ultime moyen de survivre libre.
 
 

% Le 03.03.2015 :
% Antiquité
% Moyen Âge
% ~etc.
% ~\%



\chapter{Les chrétiens}


 Les chrétiens sont apparus avant le milieu du premier siècle de notre ère%
% [1]
\footnote{Sources :\\
Peter \fsc{BROWN}, \emph{Le renoncement à la chair, virginité, célibat et continence dans le christianisme primitif}, 2002.\\
Alexandre \fsc{FAIVRE}, \emph{Naissance d'une hiérarchie, les premières étapes du cursus clérical}, 1977.\\
Collectif, \emph{Aux origines du christianisme}, 2000.\\
A.~\fsc{HAMMAN}, \emph{La vie quotidienne des premiers chrétiens}, p. 95-197, 1971.\\
Aline \fsc{ROUSSELLE}, \emph{La contamination spirituelle, science, droit et religion dans l'Antiquité}, 1998.}%
. Au tout début ils ne se distinguaient des autres juifs que par leur jugement sur la personne de Jésus. Pour eux il était le Messie, qu'il n'y avait donc plus à attendre, et le Seigneur, c'est-à-dire Dieu.  \emph{Thomas lui répondit : "Mon Seigneur et mon Dieu !" Jésus lui dit : "parce que tu me vois, tu crois. Heureux ceux qui croiront sans avoir vu." (Jn, 20, 28-29)}. Ils tenaient la bible juive pour leur référence \emph{(l'Ancien Testament)}. Ils ont mis plus d'une génération à réaliser qu'ils n'étaient plus des juifs comme les autres. Ceux-ci auraient peut-être toléré leur prétention de voir en Jésus le messie mais ils ne pouvaient accepter de l'assimiler à Dieu. 



Dans le même temps ils ont élaboré un culte original pour lequel ils se sont créé un clergé permanent et ils se sont organisés hiérarchiquement. Ils ont ajouté à la bible juive les livres du \emph{Nouveau Testament} : les quatre \emph{Évangiles}, les \emph{Actes des apôtres}, \emph{L'Apocalypse} et diverses \emph{épîtres} (celles de Paul de Tarse d'abord et surtout). Ils ont élaboré les premiers éléments d'une théologie, poussés par la nécessité d'articuler les contradictions de leurs croyances qui heurtaient le sens commun : le Dieu d'Abraham, Jésus le Christ, l'Esprit Saint, une incarnation, une mort infamante, une résurrection... 

Jusqu'en 313 leurs croyances et leurs pratiques n'ont pas été reconnues par Rome et ils ont été l'objet de persécutions plus ou moins épisodiques et rigoureuses. À partir de la fin du \siecle{1}, se reconnaître chrétien et refuser de sacrifier aux dieux civiques (ce qui était permis aux juifs) était en effet un délit suffisant pour être mis à mort sans autre forme de procès%
% [2]
\footnote{Les chrétiens n'avaient pas obtenu la même reconnaissance que les juifs (que le versement du \latin{fiscus judaïcus} libérait de l'obligation de sacrifier à l'empereur et aux dieux des cité) et ils refusaient de se soumettre à ce tribut personnel. En le payant ils auraient évité d'être sanctionnés pour leur refus des idoles, mais ils tenaient à ne pas être confondus avec eux. Les pouvoirs publics ne les recherchaient pas systématiquement, mais ils ne pourchassaient pas non plus ceux qui les maltraitaient. Au contraire, les magistrats instruisaient les dénonciations qui leur étaient transmises.}
ce qui ne les a pas empêchés de se répandre jusqu'à devenir une des composantes incontournables de l'Empire, même si elle restait minoritaire. 

 Les premières communautés chrétiennes (les premières « églises%
% [3]
\footnote{\enquote{Église} vient du mot grec \emph{ecclésia}, qui signifie « assemblée » : ce mot désigne le rassemblement par convocation des citoyens libres, tandis que le mot \emph{sunagoguè}, de sens voisin et dont vient \enquote{synagogue}, désigne plutôt une réunion organisée par accord mutuel des participants.}
») fonctionnaient sur le modèle des communautés juives (des synagogues). Leur représentation du monde, fidèle en cela aussi au judaïsme, empruntait ses traits essentiels à la famille : un seul dieu père, une communauté qui se définit comme une famille de frères et de sœurs,~etc. Mais au fil du temps les différences sont devenues de plus en plus évidentes. Pour une part importante ces différences étaient liées à la façon dont les chrétiens abordaient la vie sexuelle et à la place qu'ils lui donnaient … ou ne lui donnaient pas. 


\section{Indissolubilité du mariage}

 Conformément au droit romain le mariage des chrétiens reposait sur la volonté des seuls époux%
% [5]
\footnote{Quand à Rome deux personnes voulaient se marier, il leur suffisait de dire tous les deux leur intention de convoler en présence de témoins dans le cadre d'une simple fête domestique. Aucune cérémonie plus officielle n'était nécessaire, même si les mots prononcés étaient plus ou moins codifiés (très proches en fait de ce que nous disons aujourd'hui, puisque c'est de là que nous l'avons reçu), et si l'échange d'anneaux (nos alliances) était de règle.}%
. L'expression publique de leur choix réciproque suffisait. Il n'existait rien qui ressemble à un mariage religieux, mais au sein de leur communauté leur accord public entrainait des conséquences qui excédaient les conséquences ordinaires d'un mariage entre non-chrétiens. Ils avaient la même horreur de l'inceste que les autres romains, qui refusaient le mariage entre cousins accepté par les orientaux%
% [6]
\footnote{... ce que prouvent les réactions au mariage de l'Empereur Claude (41-54 de notre ère) avec sa nièce. Même si cet exemple a été imité il heurtait le sentiment des romains, alors qu'en Orient cela ne faisait guère problème.}%
. Par contre ils tenaient pour valides des unions jugées impossibles ou même interdites par la loi romaine : entre sénateur et affranchie, entre citoyen et barbare, et même ils reconnaissaient la validité des mariages des esclaves entre eux ou avec des personnes libres,~etc.

Les quatre évangiles sont d'accord pour dire que le Christ enseignait que l'union conjugale est indissoluble. Cette doctrine est si éloignée des conceptions de l'époque, juives ou autres, et elle posait tant de problèmes, qu'elle a toutes les chances d'appartenir à son enseignement le plus authentique :

\begin{displayquote}
\emph{S'approchant, des pharisiens lui demandèrent : \enquote{Est-il permis à un mari de répudier sa femme ?} C'était pour le mettre à l'épreuve. Il leur répondit : \enquote{Qu'est-ce que Moïse vous a prescrit ? --- Moïse, dirent-ils, a permis de rédiger un acte de divorce et de répudier.} Alors Jésus leur répliqua : \enquote{c'est en raison de votre caractère intraitable qu'il a écrit pour vous cette prescription. Mais à l'origine de la création, Dieu les fit homme et femme. Ainsi donc l'homme quittera son père et sa mère, et les deux ne feront qu'une seule chair. Ainsi ils ne sont plus deux, mais une seule chair. Eh bien ! Ce que Dieu a uni, l'homme ne doit point le séparer.} Rentrés à la maison les disciples l'interrogèrent de nouveau sur ce point. Et il leur dit : \enquote{Quiconque répudie sa femme et en épouse une autre, commet un adultère à l'égard de la première ; et si une femme répudie son mari et en épouse un autre, elle commet un adultère.}} (Mc~10,~2-12)

{\emph{Si c'est elle \emph{[l'épouse]} qui se sépare de son mari et qui devient la femme d'un autre, elle commet un adultère.}} (Mt~5,~32b)

{\emph{Quiconque répudie sa femme et en épouse une autre commet un adultère, et celui qui épouse une femme répudiée par son mari commet un adultère.}} (Lc,~16,~18)
 \end{displayquote} 

  Il n'était pas possible pour ceux qui écoutaient cet enseignement de le tenir pour nul et non avenu quelles que soient les oppositions qu'il suscitait et les difficultés qu'il créait et créerait à l'avenir :

\begin{displayquote}[Mt~19,~10-12]
\emph{Les disciples lui dirent : « Si telle est la condition de l'homme envers la femme, il n'est pas expédient de se marier. » Et lui de leur répondre : « Tous ne comprennent pas ce langage, mais ceux-là seulement à qui c'est donné. Il y a en effet des eunuques qui sont nés ainsi du sein de leur mère, il y a des eunuques qui le sont devenus par l'action des hommes, et il y a des eunuques qui se sont rendus tels en vue du royaume des cieux. Comprenne qui pourra !}
\end{displayquote}
 
 Ceci dit pour le Christ le mariage n'est que pour cette terre :

\begin{displayquote}[Luc 20, 34-36]

[...] \emph{ceux qui auront été jugés dignes d'avoir part à l'autre monde et à la résurrection des morts ne prennent ni femme ni mari \emph{[...]} car ils sont pareils aux anges \emph{[...]}}
\end{displayquote}

 Pour Paul de Tarse le modèle du mariage était l'union du Christ avec son Église, union qui accomplissait l'alliance de Dieu et d'Israël (« ancienne alliance ») dont les prophètes avaient à maintes reprises dans le passé exprimé les fluctuations dans le langage de l'amour humain. Ce modèle identifiait l'homme au Christ et la femme à l'Église. Le mari se devait d'aimer son épouse :

\begin{displayquote}[Eph. 5,25-33]
\emph{Maris, aimez vos femmes comme le Christ a aimé l'Église : il s'est livré pour elle, afin de la sanctifier en la purifiant par le bain d'eau qu'une parole accompagne ; car il voulait se la présenter à lui-même toute resplendissante, sans tache ni ride ni rien de tel, mais consacrée et sans reproche. De la même façon les maris doivent aimer leurs femmes comme leurs propres corps. L'amoureux de sa femme s'aime lui-même. Or nul n'a jamais haï sa propre chair ; on la nourrit au contraire et on en prend bien soin. C'est justement ce que le Christ fait pour l'Église : ne sommes-nous pas les membres de son corps ? « Voici donc que l'homme quittera son père et sa mère pour s'attacher à sa femme, et les deux ne feront qu'une seule chair » : ce mystère est de grande portée ; je veux dire qu'il s'applique au Christ et à l'Église. Bref, en ce qui vous concerne, que chacun aime sa femme comme soi-même, et que la femme révère son mari.}%
%  [4] 
\footnote{... que cette épître ait été écrite par Paul lui-même ou par un de ses disciples est aujourd'hui en débat. Pour notre objet l'important est qu'elle ait été reçue comme venant de lui, exprimant sa doctrine, et qu'elle ait été intégrée dans le Canon de l'Église, la liste officielle de ses textes de référence.}%
.
\end{displayquote}

 Pour Paul, Dieu est fidèle en dépit et au-delà de toutes les infidélités d'Israël. De la même façon le Christ a été fidèle jusqu'à la mort. C'est ainsi qu'il argumentait son refus de tout remariage tant qu'un ex conjoint était vivant :

\begin{displayquote}[I Cor 7,10-12]

[...] \emph{que la femme ne se sépare pas de son mari -- en cas de séparation qu'elle ne se remarie pas ou qu'elle se réconcilie avec son mari -- et que le mari ne répudie pas sa femme.}
\end{displayquote}

L'union ne pouvait être dissoute que par la mort :

\begin{displayquote}[I Cor 7,39]
\emph{La femme demeure liée à son mari aussi longtemps qu'il vit ; mais si le mari meurt elle est libre d'épouser qui elle veut, dans le Seigneur seulement} (c'est-à-dire parmi les membres de la communauté chrétienne).
\end{displayquote}
 
 Les devoirs des époux étaient réciproques et égaux. L'union d'un homme et d'une femme était exclusive, ce qui interdisait la polygynie : \emph{Que chaque homme ait sa femme et chaque femme son mari.} (I Cor 7,2). En bon juif, Paul de Tarse n'avait rien contre les relations sexuelles, à la condition qu'elles s'inscrivent dans le cadre bien tempéré d'une vie conjugale régulière, excluant les pratiques de nature à empêcher la fécondation, et les avortements. Quant à la sexualité hors mariage, hétérosexuelle ou homosexuelle, il ne s'y intéressait que pour la condamner sans appel, comme les moralistes juifs et stoïciens de son époque. La fidélité était donc exigée des hommes mariés, et tous leurs écarts étaient qualifiés d'adultères, désormais aussi coupables moralement que ceux des épouses, alors que jusque là ils n'étaient adultères selon la loi que s'ils avaient une relation sexuelle avec la femme légitime d'un autre homme. 
 
 Même si leurs obligations réciproques étaient identiques, Paul ne mettait aucunement en question la soumission des femmes aux hommes, pas plus qu'il n'était prêt à leur donner un quelconque pouvoir de représentation dans les assemblées. Pour lui comme pour toute l'Antiquité, païenne ou juive, la famille était une institution hiérarchisée et non une démocratie, et c'était l'homme qui la dirigeait et non la femme : \emph{Le chef de tout homme, c'est le Christ ; le chef de la femme, c'est l'homme ; et le chef du Christ c'est Dieu.} (I Cor 1).


Dans sa première Apologie, Justin résume la position des églises vers l'an 155 :

\begin{displayquote}
\emph{Voici ce qu'il \emph{[Jésus]} dit de la chasteté : « Quiconque aura regardé une femme pour la convoiter a déjà commis l'adultère dans son cœur. » Et : « Que si votre œil droit vous scandalise ; arrachez-le et jetez-le loin de vous ; il vaut mieux n'avoir qu'un œil et entrer dans le royaume des cieux, qu'avoir deux yeux et être jeté dans le feu éternel. » Et : « Celui qui épouse la femme répudiée par un autre homme commet un adultère. » Et : « Il y a des eunuques sortis tels du sein de leur mère ; il y en a que les hommes ont fait eunuques, et il y en a qui se sont faits eunuques eux-mêmes en vue du royaume des cieux ; mais tous n'entendent pas cette parole. » Ainsi ceux qui, selon la loi des hommes, contractent un second mariage après leur divorce, comme ceux qui regardent une femme pour la convoiter, sont coupables aux yeux de notre maître; il condamne le fait et jusqu'à l'intention de l'adultère ; car Dieu voit non seulement les actions de l'homme, mais même ses plus secrètes pensées. Et pourtant combien d'hommes et de femmes sont parvenus à plus de soixante et soixante-dix années, qui, nourris depuis leur berceau dans la foi du Christ, sont restés purs et irréprochables durant leur longue carrière ! Ce fait se retrouve dans les peuples de toute contrée ; je m'engage à le prouver.}
\end{displayquote}
 
 On a vu qu'à Rome tout témoin d'un adultère féminin devait dénoncer les coupables. Le mari d'une adultère devait la répudier sans tarder, et elle était condamnée à l'infamie. Sinon il était lui aussi condamné à l'infamie comme proxénète, ainsi que ses enfants à naître. En tant qu'infâme il perdait son autorité sur ses enfants déjà nés et une fraction importante de ses biens était confisquée. Son mariage était dissous même s'il continuait de cohabiter avec son épouse. Les premiers chrétiens ne pouvaient pas faire l'impasse sur des lois civiles aux effets aussi redoutables : passer outre aurait été héroïque, surtout si des enfants étaient impliqués. Au surplus ils est probable qu'ils pensaient comme tous leurs contemporains et que l'adultère féminin leur paraissait très grave et honteux, beaucoup plus grave et plus honteux que celui des maris. Il est donc probable qu'ils répudiaient eux aussi les épouses infidèles. D'ailleurs l'évangile de Matthieu accepte la répudiation de l'épouse en cas de « fornication » ou de « prostitution » \latin{(porneia)} :

\begin{displayquote}
\emph{Il a été dit d'autre part : Celui qui répudie sa femme doit lui remettre un acte de divorce. Eh bien ! Moi je vous dis : quiconque répudie sa femme, hormis le cas de fornication, la voue à devenir adultère ; et si quelqu'un épouse une répudiée, il commet un adultère.} (Mt 5,31-32)

\emph{Or je vous le dis : quiconque répudie sa femme -- je ne parle pas de la prostitution -- et en épouse une autre, commet un adultère.} (Mt 19, 9)
\end{displayquote}

 Mais le vrai problème n'était ni la répudiation ni le divorce, c'était le remariage. Il ne s'agissait pas de décider si l'adultère mettait fin à un mariage, de toute façon déjà dissous par la loi civile, mais s'il permettait au conjoint innocent de contracter validement un nouveau mariage. Si la loi romaine faisait aux maris des femmes adultères l'obligation de répudier celles-ci, elle ne les obligeait pas à se remarier. S'ils ne le faisaient pas ils subissaient les conséquences des Lois d'Auguste et ils étaient taxés comme célibataires. Ce n'est que s'ils n'avaient pas encore le quota réglementaire d'enfants que les conséquences devenaient sérieuses : ils devaient alors renoncer à hériter de personnes qui ne faisaient pas partie de leur famille. Ceci dit s'ils n'avaient pas d'amis aisés susceptibles de les coucher sur leurs testaments la perte n'était pas grande. Il était surtout problématique d'interdire toute vie sexuelle et peut-être toute descendance à des hommes encore jeunes au seul motif que leurs épouses leur avaient été infidèles. 

 Les discussions entre théologiens et évêques ont donc porté sur ce que l'on devait entendre par « fornication » et par « prostitution ». Au fil du temps, ils se sont mis d'accord sur l'idée que par ces mots, le Christ avait voulu désigner non pas l'infidélité de l'un ou de l'autre des époux, mais la transgression des interdits de mariage, c'est-à-dire tout ce qui se rapproche de l'inceste%
% [7]
\footnote{cf. A.-M.~\fsc{GERARD}, 1989, p. 878.}%
. Au troisième siècle au plus tard c'est cette interprétation qui avait pris l'ascendant, mais la discussion est restée ouverte jusqu'au milieu du Moyen Âge. Même quand un remariage après divorce était admis, et cela semble n'avoir pas été rare jusqu'à la réforme grégorienne (milieu du Moyen Âge), ce n'était qu'une concession faite en vue d'éviter de plus grands maux%
% [8]
\footnote{La lecture littérale de Matthieu n'a jamais été oubliée notamment en Orient, et c'est cette lecture que choisira la réforme protestante.}%
.

 En stricte doctrine chrétienne le viol ne déshonorait ni ne souillait la victime : \emph{rien de ce qui est hors de l'homme et qui entre dans l'homme ne peut le souiller ; mais ce qui sort de l'homme voilà ce qui souille l'homme} (Marc 7, 15). D'autre part le suicide comme la répudiation étaient interdits. Cela étant le ressenti des personnes concernées ne pouvait s'affranchir des représentations communes : ce qu'on nommait l'honneur de la victime et celui de sa parentèle étaient en jeu. Comment redonner une bonne réputation aux victimes et à leur famille en dépit de ces représentations sans employer les moyens radicaux qui avaient cours jusque là, sans faire disparaître les victimes ? Comment soigner les traumatismes des femmes violées et restaurer leur image en elles-mêmes et chez les autres ?  

 De même que les juifs, les chrétiens cherchaient à marier leurs enfants à des membres de la communauté chrétienne, sans exclure formellement les mariages avec un non chrétien. Le seul cas où le remariage après divorce était autorisé par Paul concernait justement les unions où un conjoint non chrétien mettait des obstacles à la pratique religieuse de son conjoint chrétien (« Privilège Paulin »).


\section{Valorisation du célibat et de la continence}

 Les évangiles valorisaient le célibat de manière implicite en mettant en valeur deux célibataires : Jean le Baptiste et Jésus. Par ailleurs et surtout on y trouvait divers discours explicites en faveur du célibat, notamment le récit suivant qui est commun à trois évangiles sur quatre (Matthieu 19, 16-22 ; Marc 10, 17-22 ; Luc 18, 18-23) et qui fait donc partie du noyau de traditions et de paroles autour desquelles s'est articulée la prédication du premier demi-siècle de l'Église :

\begin{displayquote}[Mt~19,~16-22]
\emph{Or voici qu'un homme s'approcha et lui dit : « Maître, que dois-je faire de bon pour posséder la vie éternelle ? » Jésus lui dit : « Qu'as-tu à m'interroger sur ce qui est bon ? Un seul est le Bon. Que si tu veux entrer dans la vie, observe les commandements. -- Lesquels ? » lui dit-il. « Eh bien », reprit Jésus : « Tu ne tueras pas, tu ne commettras pas d'adultère, tu ne voleras pas, tu ne porteras pas de faux témoignage ; honore ton père et ta mère, et tu aimeras ton prochain comme toi-même. » Le jeune lui dit : « Tout cela, je l'ai gardé ; que me manque-t-il encore ? -- Si tu veux être parfait, lui dit Jésus, va, vends ce que tu possèdes, donne-le aux pauvres, et tu auras un trésor aux cieux ; puis viens, suis-moi. » Quand il entendit cette parole, le jeune homme s'en alla contristé, car il avait de grands biens.}
\end{displayquote}

 La vie religieuse ou le célibat consacré des siècles futurs ont là leur origine. Appliquer à la lettre la suggestion de Jésus \emph{(vends ce que tu possèdes)} impliquait en effet de n'avoir plus d'héritage à transmettre et donc plus jamais d'enfants, sauf à manquer à tous les devoirs d'un père, ce qui est une position que nul exégète \emph{sérieux}%
% [10]
\footnote{Jack \fsc{GOODY} note que dans son ouvrage \latin{Contra avaritiam}, Salvien, prêtre marseillais du \siecle{5}, à une période où le christianisme est devenu la religion d'état des populations romaines, conseille aux parents de laisser leurs biens à l'Église plutôt qu'à leurs enfants, car \emph{mieux vaut la souffrance des enfants en ce monde que la damnation des parents dans l'autre} (p. 107). Cette position antisociale n'est vraisemblablement qu'un médiocre artifice de rhétorique chez un auteur porté par ailleurs aux exagérations et à l'hyperbole : à la même époque Saint Augustin, dont l'autorité est sans commune mesure avec celle de Salvien, refusait formellement à l'Église le droit d'accepter tout legs fait au détriment d'un fils (Jack \fsc{GOODY}, p. 101). Il conseillait de (ne) léguer à l'Église (que) la part d'\emph{un} fils, ce qui diminuait d'autant plus le montant des legs qu'il y avait plus d'héritiers vivants. Le problème n'était pas seulement théorique : nombreuses ont été au fil des siècles les réactions des autorités civiles pour empêcher l'enthousiasme des plus fanatiques des dévots, ou la terreur de la damnation éternelle des mourants, ou leur désir de régler des comptes avec leurs enfants, de dépouiller leurs héritiers légitimes.}
n'a jamais prêtée à l'auteur de ce texte. 

 Lorsque les chrétiens valorisaient le célibat et la chasteté, ce n'était pas sans échos dans le monde gréco-romain des premiers siècles de notre ère : les philosophes stoïciens et les médecins d'alors étaient soucieux de ne pas donner au sexe plus de place qu'il n'en méritait et de maîtriser les passions, au premier rang desquelles la passion amoureuse. Les juifs aussi avaient leurs \emph{nazirs} et leurs \emph{esséniens}. 

 Pour Paul de Tarse les personnes continentes étaient moins exposées aux dangers moraux et aux angoisses que ceux et celles qui choisissaient le mariage :

\begin{displayquote}[I~Cor~7,~25-28]
\emph{Pour ce qui est des vierges, je n'ai pas d'ordre du Seigneur, mais je donne un avis en homme qui, par la miséricorde du Seigneur, est digne de confiance. J'estime donc qu'en raison de la détresse présente, c'est l'état qui convient ; oui, c'est pour chacun ce qui convient. Es-tu lié à une femme ? Ne cherche pas à rompre. N'es-tu pas lié à une femme ? Ne cherche pas de femme. Si cependant tu te maries, tu ne pèches pas ; et si la jeune fille se marie, elle ne pèche pas. Mais ceux-là connaîtront des épreuves en leur chair, et moi, je voudrais vous les épargner.}
\end{displayquote}

\begin{displayquote}[I~Cor~7,~32-35]
\emph{Je voudrais vous voir exempts de soucis. L'homme qui n'est pas marié a souci des affaires du Seigneur, des moyens de plaire au Seigneur. Celui qui s'est marié a souci des affaires du monde, des moyens de plaire à sa femme ; et le voilà partagé. De même la femme sans mari, comme la jeune fille, a souci des affaires du Seigneur ; elle cherche à être sainte de corps et d'esprit. Celle qui s'est mariée a souci des affaires du monde, des moyens de plaire à son mari. Je vous dis cela dans votre propre intérêt, non pour vous tendre un piège, mais pour vous porter à ce qui est digne et qui attache sans partage au Seigneur.}
\end{displayquote}

 Selon lui ceux qui supportaient la continence et qui en faisaient le choix étaient libérés de toute attache terrestre, et dégagés des soucis du monde : c'était un point de vue très stoïcien. Ils choisissaient « la meilleure part » d'où sa réticence devant les remariages, sauf pour les veufs et veuves jeunes et sans enfants. En effet en dépit de sa préférence pour la continence il ne pensait pas que celle-ci était faite pour tout le monde ni qu'elle était sans risques%
% [11]
\footnote{Cf. Blaise \fsc{Pascal} : \emph{L'homme n'est ni ange, ni bête, et le malheur veut que qui veut faire l'ange fait la bête.} \emph{(Pensées)}}
:

\begin{displayquote}[I Cor 7, 8-9]
\emph{J'en viens maintenant à ce que vous m'avez écrit. Il est beau pour l'homme de ne pas toucher à la femme. Toutefois en raison du péril d'impudicité, que chaque homme ait sa femme et chaque femme son mari. Que l'homme s'acquitte de son devoir envers sa femme, et pareillement la femme envers son mari. La femme ne dispose pas de son corps, mais le mari. Pareillement, le mari ne dispose pas de son corps, mais sa femme. Ne vous refusez pas l'un à l'autre, si ce n'est d'un commun accord, pour un temps, afin de vaquer à la prière ; puis reprenez la vie commune, de peur que Satan ne profite, pour vous tenter, de votre manque de maîtrise. Ce que je dis là est une concession, non un ordre. Je voudrais que tout le monde fût comme moi ; mais chacun reçoit de Dieu son don particulier, l'un celui-ci, l'autre celui-là. Je dis toutefois aux célibataires et aux veuves qu'il leur est bon de demeurer comme moi. Mais s'ils ne peuvent se maîtriser qu'ils se marient : mieux vaut se marier que de brûler.}
\end{displayquote}

 Il conseillait le célibat une fois satisfait le désir d'une descendance et obtenu le droit d'hériter qui en découlait pour les citoyens romains, une fois passées la jeunesse et ses orages, et on était vite vieux à une époque où les femmes commençaient souvent à avoir des enfants dès 13 ou 14 ans, où la moitié de ceux-ci mouraient avant leurs vingt ans, et où ceux qui atteignaient cet âge avaient de fortes chances d'être déjà orphelins de père. 


\section{Désacralisation de la fécondité, valorisation de chaque vivant}

 Comme les juifs les chrétiens refusaient que le mariage ait pour fin la continuité du culte des ancêtres. Mais contrairement à eux ils refusaient d'accorder une valeur religieuse à la fécondité individuelle : ni les Évangiles ni les Épîtres retenues par le Canon des écritures chrétiennes n'en parlent. Nul, même marié, n'était à leurs yeux tenu de concevoir des enfants. C'est le peuple chrétien tout entier, et non chaque famille, qui devait croître et se multiplier. Cette position leur permettait de refuser le divorce et la polygamie, alors qu'il leur aurait été difficile de maintenir ces refus si la fécondité avait été posée comme un devoir pour chaque individu. La seule obligation à laquelle chaque chrétien devait se soumettre était de n'opposer aucune barrière à sa fécondité au cours des actes sexuels dans lesquels il s'engageait. 

 Comme les juifs, l'Église désapprouvait moralement l'abandon, mais elle interdisait aussi l'avortement quel qu'en soit le motif. Dans sa première Apologie (vers 155) Justin (100-165) écrivait à l'empereur à propos des abandons et ventes d'enfants :

\begin{displayquote}
\emph{Quant à nous, loin de commettre aucune impiété, aucune vexation, nous regardons comme un crime odieux l'exposition des enfants nouveau-nés ; parce que d'abord nous voyons que c'est les vouer presque tous, non seulement les jeunes filles, mais même les jeunes garçons, à une prostitution infâme ; car de même qu'autrefois on élevait des troupeaux de bœufs et de chèvres, de brebis et de chevaux, de même on nourrit aujourd'hui des troupes d'enfants pour les plus honteuses débauches. Des femmes aussi et des êtres d'un sexe douteux, livrés à un commerce que l'on n'ose nommer, voilà ce qu'on trouve chez toutes les nations du Globe. Et au lieu de purger la terre d'un scandale pareil, vous en profitez, vous en recueillez des tributs et des impôts !}

 \emph{... Quant à l'exposition des enfants, il est un motif encore qui nous la fait abhorrer. Nous craindrions qu'ils ne fussent pas recueillis, et que notre conscience restât ainsi chargée d'un homicide. Au reste, si nous nous marions, c'est uniquement pour élever nos enfants ; si nous ne nous marions pas, c'est pour vivre dans une continence perpétuelle}.
\end{displayquote}

 Dans un plaidoyer en faveur des chrétiens adressé à l'empereur Marc-Aurèle, Athénagoras%
% [13]
\footnote{Cité par Albert \fsc{DUPOUX}, \emph{Sur les pas de Monsieur Vincent, 300 ans d'histoire parisienne de l'enfance abandonnée}, 1958, p.~5.}
exposait ainsi la position de l'Église : \emph{Nous tenons pour homicides les femmes qui se font avorter, et nous pensons que c'est tuer un enfant que de l'exposer.}  Deux générations plus tard Tertullien (160-245) écrivait que \emph{l'homme existe avant la naissance, de même que le fruit est tout entier dans la graine}. La seule méthode acceptable pour limiter le nombre des enfants était donc la continence. Celui qui ne pouvait élever plus d'enfants qu'il n'en avait déjà se devait de « s'abstenir de sa femme ».
 


 % Le 03.03.2015 :
% Antiquité
% Moyen Âge
% ~etc.
% ~\%






\section{Laïcs et laïques consacrés}

 Si la fécondité et le mariage cessaient d'être des obligations morales, alors les jeunes gens avaient le droit moral de s'y refuser, non plus en raison du manque d'attrait du parti proposé par leur père, ce qui était parfaitement admis à Rome, mais en raison d'une préférence pour le célibat « en vue de Dieu » et pour la continence, chose qui jusque là n'avait pas de sens. C'était donner aux jeunes gens le droit de choisir une vie indépendante du désir de leur \latin{pater familias}. C'était aller directement contre l'intérêt des familles tel que celles-ci le concevaient : en droit les jeunes gens qui avaient un \latin{pater familias} étaient mariés par celui-ci. Si le consentement des futurs époux était obligatoire à Rome, en réalité il était présumé acquis à partir du moment où ils ne protestaient pas trop bruyamment. Les parents de ce temps aussi n'hésitaient pas à déterminer eux-mêmes, en toute bonne conscience, le destin de leurs enfants. Au même moment et en dépit des interdictions légales des enfants de tous âges étaient vendus ou abandonnés aux prêteurs sur gage par des parents sans ressources, ce qui ne leur laissait guère de chances de se marier.

 Étant donné que quel que soit son âge une fille non mariée vivait sous l'autorité de son père ou de son tuteur, si elle voulait rester célibataire il fallait qu'elle le convainque de ne pas la marier. Or les représentations juives et romaines faisaient au contraire un devoir aux pères de marier leurs filles. Lorsqu'ils tardaient à s'acquitter de ce devoir, la rumeur publique était prompte à faire courir le bruit que la jeune personne manquait d'attraits physiques, ou qu'un problème de santé rendait dangereuse ou impossible pour elle la procréation, ou que des problèmes financiers empêchaient la constitution d'une dot, ou bien encore que l'amour du père pour sa fille était excessif, sinon coupable. 

 Comment comprendre alors que Paul de Tarse ait pu écrire que s'il le trouvait bon un père pouvait vouer sa fille au célibat au lieu de la marier ? En effet selon lui :

\begin{displayquote}[I~Cor~7,~36-38]
\emph{Si pourtant quelqu'un estime qu'il n'est pas honorable pour sa fille de dépasser l'âge du mariage, et qu'il doit en être ainsi, qu'il fasse ce qu'il veut ; il ne pèche pas : qu'ils se marient. De même celui qui est fermement décidé en son cœur, à l'abri de toute contrainte et libre de son choix, s'il décide de garder sa fille, il agira bien. Ainsi celui qui marie sa fille fait bien, et celui qui ne la marie pas fait mieux}.
\end{displayquote}

À qui Paul apportait-il son appui ? ... aux pères contre leurs filles ? ... ou bien aux pères et aux filles contre le reste du monde ? Ceci dit ce texte a plus d'une fois servi au fil des siècles à imposer à des filles un choix de vie qui n'était pas le leur. 
 
 Le choix du célibat pouvait être financièrement coûteux : du fait des lois d'Auguste ceux qui le choisissaient étaient exclus des héritages et soumis à l'impôt sur les célibataires. Mais surtout il était reconnu aux pères le droit d'émanciper ou de déshériter les jeunes récalcitrants. Ces jeunes étaient alors sans ressources, ce qui les contraignait à travailler pour autrui comme salarié (mercenaire), ce qui rapportait peu, ou à se vendre comme esclaves. Les filles qui choisissaient le célibat se condamnaient en outre à demeurer toute leur vie des mineures légales, alors que celles qui donnaient le jour à trois enfants étaient dispensées de tutelle à partir du décès de leur \latin{pater familias} et pouvaient gérer elles-mêmes leur fortune personnelle, même lorsqu'elles étaient mariées : une fois veuves elles n'avaient plus de comptes à rendre à personne sur l'emploi de leurs biens. Quant aux pauvres, qui ne pouvaient espérer aucun héritage, ils avaient encore moins que les autres intérêt à demeurer célibataires s'ils voulaient préparer leurs vieux jours. 

 Mais même si le mariage cessait d'être une obligation, qu'est-ce que cela changeait pour les femmes dans un monde où les filles et les femmes honorables ne pouvaient travailler qu'à leur domicile ou sous le regard direct d'un père, d'un frère ou d'un époux ? Comment les veuves et les épouses répudiées sans père et sans fortune personnelle (et la pauvreté des concubines abandonnées était encore pire) pouvaient-elles subsister honorablement, sinon en se mettant sous la protection d'un concubin ou d'un époux ? À quel autre moyen non infamant pouvaient-elles recourir pour ne pas en être réduites par la faim à vendre leurs enfants et à se vendre elles-mêmes ? À quoi bon dire du mal du remariage ? Se remarier ne valait-il pas mieux que se prostituer ?

 Les problèmes matériels posés par le choix du célibat étaient identiques à ceux des candidats au baptême qui vivaient jusque là dans une situation personnelle désapprouvée par l'Église : personnel des temples païens, fabricants d'idoles et \latin{d'ex-voto} païens, prostitués des deux sexes, acteurs, danseurs et danseuses, gladiateurs, organisateurs de spectacles,~etc. Aucun d'eux n'était exclu du salut puisque le baptême lavait tous les péchés antérieurs, mais ils ne pouvaient être accueillis dans la communauté qu'à la condition de cesser de vivre comme ils l'avaient fait jusque là. Une prostituée qui renvoyait ses clients, une maquerelle, un sculpteur d'idoles, un acteur, un gladiateur qui cessaient leur activité mouraient de faim. Il n'était pas imaginable de les renvoyer en s'en lavant les mains. 

 Les premiers chrétiens ne pouvaient pas se contenter de proclamer la valeur du célibat et de la continence. S'ils y croyaient vraiment il leur fallait le rendre possible. Leurs propres enseignements leur interdisaient de se désintéresser de la situation de ceux et surtout de celles qui faisaient ce choix sur leurs conseils. Ils devaient leur procurer les moyens de mettre en œuvre l'aspiration qu'ils avaient suscitée ou se taire. Les évêques ont donc été conduits à prendre fait et cause pour tous ceux de leurs fidèles qui faisaient un choix de vie radical. Si nécessaire ils se sont mis en avant pour soutenir les jeunes dans leurs combats contre leurs familles. 

 Les \emph{vierges consacrées} (parfois nommées \emph{diaconesses} dans l'aire grecque) sont apparues très tôt, dès le deuxième siècle. A la différence des diacres, des prêtres et des évêques, elles n'ont jamais fait pas partie des clercs. C'étaient des laïques, à l'instar des futurs moines, dont elles étaient l'équivalent féminin. Elles n'étaient pas \emph{ordonnées} pour mener à bien une mission ni pour tenir une place dans la hiérarchie. Elles choisissaient le célibat pour se consacrer à la prière et au service de la communauté, notamment à la catéchisation des femmes et au service des malades. Leurs vœux étaient reçus par leur évêque au cours d'une cérémonie qui a très tôt été interprétée comme un mariage mystique avec le Christ, auquel elles se vouaient. Le fait qu'à compter de ce jour elles portaient le voile (d'où le nom de la cérémonie : « prise de voile ») affichait cette interprétation de leur choix de vie à leurs contemporains. Il s'agissait en effet du voile des matrones qui proclamait qu'elles avaient un \latin{dominus}, un seigneur et maître. Cela les rapprochait des vestales, vouées au célibat et à la virginité, qui elles aussi portaient un voile qui avertissait les passants que leur corps était consacré, et qu'y toucher était sacrilège.


\section{Clergé et continence}

 En ce qui concerne les desservants du culte%
% [2]
\footnote{Pour cette partie je me suis en particulier servi de \emph{Naissance d'une hiérarchie, les premières étapes du cursus clérical,} d'Alexandre \fsc{FAIVRE}, 1977.}
l'Église n'a pas adopté les pratiques des synagogues. Les rabbins sont des sages, des savants, des exégètes et des juges, et non des prêtres, des sacrificateurs. Ils ne président au culte qu'avec l'assentiment des fidèles. Jusqu'au milieu du Moyen Âge au moins ils ne seront pas payés. Ils sont choisis par les fidèles eux-mêmes, et non reçus d'une autorité extérieure à la communauté. Ils ont moins un pouvoir normatif qu'un pouvoir d'influence, gagné par leur réputation de compétence, par leur savoir. Ils possèdent une autorité intellectuelle, morale et juridique acquise, reconnue par leurs pairs et non octroyée, même si la notion d'ordination par imposition des mains ne leur était pas étrangère. Cela signifie que le culte des synagogues était fondamentalement différent de celui des temples antiques. Il ne prétendait pas remplacer les sacrifices offerts au Temple de Jérusalem : bien au contraire si les prières faisaient référence aux sacrifices offerts au Temple, c'était pour rappeler que ces sacrifices n'étaient pas possibles, et qu'on ne pouvait qu'accepter la réalité de leur absence. 

 Au contraire le rite chrétien de la \emph{fraction du pain} se présentait comme le renouvellement littéral des gestes inaugurés par Jésus à la \emph{Cène}, selon le récit des Évangiles. Il n'était pas interprété comme un geste symbolique, destiné seulement à faire mémoire d'un événement historique une fois pour toutes réalisé. Il avait d'emblée été compris par les premiers chrétiens comme un geste efficace, performatif, comme un geste qui produisait ce qu'il énonçait, qui réalisait à la lettre ce qu'il disait : \emph{prenez et mangez : ceci est mon corps \emph{[...]} Buvez-en tous : car ceci est mon sang, le sang de l'alliance, qui va être répandu pour une multitude en rémission des péchés}. (Matthieu, 26, 28). L'Eucharistie a d'emblée été un rite surdéterminé. Pour Paul le « repas du Seigneur » est un repas pris en commun et c'est une anticipation du festin eschatologique, à la fin des temps (1~Co~11, 17-34). C'est le rappel des repas pris avec Jésus avant sa mort, mais aussi après sa résurrection. C'est un repas où le Seigneur se donne, en personne, à son peuple comme nourriture. C'est « la Pâque du Seigneur » où Jésus est l'agneau pascal dont le sang consacre ceux qui le reçoivent (1~Co~5, 7). C'est un renouvellement de l'alliance (1~Co~11, 25) : dans la \emph{Tora} le sang de l'alliance scellait la communion entre le peuple et Dieu, le sang du Christ instaure une communion encore plus profonde. C'est un sacrifice de louange. C'est donc à plusieurs titres que Paul de Tarse faisait de l'Eucharistie un sacrifice bien avant la destruction du Temple, et alors qu'à Jérusalem le \emph{sacrifice perpétuel} continuait d'être tous les jours offert, et que selon les \emph{Actes des Apôtres} les membres de la communauté judéo-chrétienne locale continuaient d'y assister assidûment.

 Mais en 70 de notre ère la destruction du Temple a contraint les juifs à cesser tout sacrifice sanglant : cela a permis aux chrétiens de penser et de dire que le \emph{sacrifice perpétuel} continuait désormais dans un autre lieu et sous une autre forme, non sanglante. Le clergé chrétien s'est pensé comme le successeur du clergé du Temple et a pris modèle sur lui. L'évêque s'est identifié au grand prêtre, les prêtres chrétiens aux prêtres juifs, les diacres et les autres ministres du culte aux lévites, d'autant plus soucieux de se conformer à leurs modèles qu'ils prétendaient les remplacer et ne pouvaient ignorer que par ailleurs ils récusaient ces mêmes modèles.

 À ses débuts l'Église avait disqualifié la notion d'impureté rituelle, notamment par la voix de Paul qui en bon pharisien savait de quoi il parlait, et qui ne voyait pas d'inconvénient à manger des viandes consacrées aux idoles, du moment que cela ne choquait pas les esprits faibles (Rom, 14, 2-3; Cor I, 8, 4-13). Mais la notion d'impureté a fait un retour en force avec la constitution d'un personnel religieux permanent, spécialisé, consacré par l'imposition des mains et mis à part : un \emph{clergé}. Le clergé chrétien va en effet s'imposer de respecter scrupuleusement les règles de pureté qui s'appliquaient aux prêtres du Temple. Il va surenchérir par rapport à ces règles et cela va renforcer la sacralisation du sexe et la valorisation de la continence (même si cette imitation n'est que l'un des éléments qui y ont contribué). En effet ses membres étaient en permanence susceptibles de toucher les « vases sacrés ». S'ils prenaient modèle sur le Temple cela exigeait d'eux qu'ils soient en permanence dans l'état de pureté rituelle exigé du Grand Prêtre durant ses semaines de service. Il fallait donc qu'ils ne soient jamais souillés par « l'impureté » sexuelle. Les clercs vont donc très tôt être voués à la continence perpétuelle, à partir de leur ordination diaconale. 

 Avant qu'ils aient atteint l'âge où ils devaient être ordonnés diacres les clercs pouvaient se marier et cohabiter avec une épouse pour avoir des enfants. L'imposition des mains (ordination diaconale) ne se faisait en effet qu'à un âge relativement avancé, 30 ans, si ce n'est 40 (prêtre vient de presbytre, c'est-à-dire « ancien, âgé »). Cela laissait un temps suffisant à un clerc marié dès ses 18 ou 20 ans pour avoir des enfants. Il était donc normal de rencontrer des fils de diacres, de prêtres ou d'évêques. Mais si un enfant leur naissait après leur ordination, les membres du clergé devaient en principe être démis de leurs fonctions. Ceci dit bien des clercs mineurs demeuraient leur vie entière dans des fonctions qui n'exigeaient pas la continence. 


\section{Il n'y a plus ni esclave ni homme libre ?}


 
 
 En Ga 3, 28, Paul de Tarse écrit : \emph{Il n'y a ni juif ni grec, il n'y a ni esclave ni homme libre, il n'y a ni homme ni femme ; car tous vous ne faites qu'un dans le Christ Jésus}. Il développe la même idée en 1 Co 12, 13 : du point de vue de Dieu, il n'y a aucune différence entre les esclaves et les hommes libres. Ils sont égaux en dignité, en valeur spirituelle et promis au même salut après leur mort. Sur cette terre il en était autrement : les chrétiens ne condamnaient pas l'esclavage, ce en quoi ils étaient d'accord avec la totalité des peuples de l'Antiquité. Ce n'étaient pas des révolutionnaires politiques et ils n'avaient qu'une confiance limitée dans les institutions et les pouvoirs humains. Ils croyaient que les changements de structure n'ont de chance d'atteindre leurs objectifs que si les cœurs ont d'abord été transformés. Avec Paul ils pensaient que le désir du mal est profondément inscrit en l'homme%
% [6]
\footnote{\emph{... en effet vouloir le bien est à ma portée, mais non pas l'accomplir : puisque je ne fais pas le bien que je veux et commets le mal que je ne veux pas.} (Rm, 7, 18-19).}%
. Pour lui comme pour les apôtres, pour ses maîtres pharisiens ou pour les stoïciens l'esclavage le plus grave était celui du péché%
%[7]
\footnote{\emph{En vérité, en vérité, je vous le dis, tout homme qui commet le péché est un esclave. Or l'esclave n'est pas pour toujours dans la maison, le fils y est pour toujours. Si donc le fils vous affranchit, vous serez réellement libres. \emph{(Jn 8,34-36).} Ne savez-vous pas qu'en vous offrant à quelqu'un comme esclaves pour obéir, vous devenez les esclaves du maître à qui vous obéissez, soit du péché pour la mort, soit de l'obéissance pour la justice ? Mais grâces soient rendues à Dieu ; jadis esclaves du péché, vous vous êtes soumis cordialement à la règle de doctrine à laquelle vous avez été confiés, et, affranchis du péché, vous avez été asservis à la justice (j'emploie une comparaison humaine en raison de votre faiblesse naturelle) car si vous avez jadis offert vos membres comme esclaves à l'impureté et au désordre de manière à vous désordonner, offrez-les de même aujourd'hui à la justice pour les sanctifier.} (Rm 6,16-19).}
(des désirs non contrôlés par la raison). La vie n'était qu'un bref passage%
%[8]
\footnote{\emph{J'estime en effet que les souffrances du temps présent ne sont pas à comparer à la gloire qui doit se révéler en nous.} (Rm, 8, 18).}% 
, et le salut religieux, la \emph{vie éternelle}, était plus important que le bonheur ou la réussite terrestre. Il n'était pas déterminant pour le salut d'être libre ou d'être esclave. L'essentiel était déjà gagné par la mort du Christ : \emph{Vous avez été achetés%
%[9] 
\footnote{La rédemption était le paiement d'un marché, d'une rançon, le rachat d'une créance par un rédempteur, c'est-à-dire un entrepreneur prêt à risquer ses capitaux dans une affaire, ou dans le rachat d'un esclave à son maître pour le libérer.} 
cher} (1 Co 6, 20). En conséquence Paul enseignait aux esclaves que Dieu lui-même avait accepté qu'ils soient placés là où ils étaient, et il leur recommandait la patience, l'honnêteté, le travail bien fait, indépendamment de toute récompense terrestre, non pour leur maître, mais pour Dieu ... et pour la bonne réputation des chrétiens. 

 Dans le même temps Paul enseignait%
% [10]
\footnote{Paul a maintes fois répété le même enseignement -- ce qui veut peut-être dire que la question lui a souvent été posée ?} 
aux maîtres que les esclaves ont la même valeur qu'eux%
%[11]
\footnote{\emph{Aussi bien est-ce en un seul esprit que nous tous avons été baptisés pour ne former qu'un seul corps, juifs ou grecs, esclaves ou hommes libres, et tous nous avons été abreuvés d'un seul esprit.} (I Cor 12, 13).}% 
, et il leur faisait un devoir de les traiter \emph{comme des collaborateurs} et \emph{comme des frères en Jésus-Christ} :

\begin{displayquote}[Col 3, 22-25 ; 4,1]
\emph{Esclaves, obéissez en tout à vos maîtres d'ici-bas, non d'une obéissance tout extérieure qui cherche à plaire aux hommes, mais en simplicité de cœur, par crainte du Maître. Quel que soit votre travail, faites-le avec âme, comme pour le Seigneur et non pour des hommes, sachant que le Seigneur vous récompensera en vous faisant ses héritiers. C'est le Seigneur Christ que vous servez; qui se montre injuste sera certes payé de son injustice, sans qu'il soit fait acception de personne. Maîtres, accordez à vos esclaves le juste et l'équitable, sachant que vous aussi, vous avez un maître au ciel.}
\end{displayquote}

 Les épîtres de Pierre défendaient les mêmes thèses :

\begin{displayquote}[Première Épître de Pierre 2, 18-23]
\emph{Vous, les domestiques \emph{[(esclaves)]} soyez soumis à vos maîtres, avec un profond respect, non seulement aux bons et aux bienveillants, mais aussi aux difficiles. Car c'est une grâce que de supporter par égard pour Dieu des peines que l'on souffre injustement. Quelle gloire en effet à supporter les coups si vous avez commis une faute, mais si faisant le bien vous supportez la souffrance c'est une grâce auprès de Dieu. Or c'est à cela que vous avez été appelés, car le Christ aussi a souffert pour vous, vous laissant un modèle afin que vous suiviez ses traces, lui qui n'a pas commis de faute. Et il ne s'est pas trouvé de fourberie dans sa bouche ; lui qui insulté ne rendait pas l'insulte, souffrant ne menaçait pas, mais s'en remettait à celui qui juge avec justice}.
\end{displayquote}

 Mais puisque la liberté est meilleure que la servitude, Paul rappelait aux maîtres qu'affranchir leurs esclaves était une bonne œuvre. Les communautés chrétiennes pouvaient racheter ceux de leurs membres qui étaient esclaves, notamment quand la nature de leur emploi, ou la dureté et l'injustice de leurs maîtres, mettaient en danger leur vie, leur santé, leur vertu ou leur foi. Il était cependant conseillé aux esclaves chrétiens de ne pas revendiquer trop bruyamment leur rachat par la communauté : ses ressources n'y auraient pas suffi, et on se serait vite demandé si leur piété n'était pas trop intéressée.

 Paul et ses successeurs auraient-ils pu s'engager plus loin ? Y ont-ils jamais pensé ?  On pourra approfondir ce sujet avec Peter \fsc{GARNSEY}, \emph{Conceptions de l'esclavage, d'Aristote à Saint Augustin}, Les belles lettres, Paris, 2004.



\section{Le service des pauvres}

 Comme n'importe quelle minorité de l'Empire les chrétiens se devaient de prendre en charge leurs pauvres, malades, et infirmes. Chaque église locale prenait en charge les siens. En cas de catastrophe elle pouvait demander des secours aux autres églises. Elle enregistrait ses pauvres, titulaires d'un droit à secours, sur la liste des personnes qu'elle entretenait \emph{(matricule)}, à côté des membres du clergé. Dans la Rome du \siecle{3}, alors que la religion chrétienne était encore illicite, c'était déjà par milliers que se comptaient les pauvres, les malades et les vieillards inscrits sur les listes de l'évêque. Chacun d'eux, pauvres et clercs entretenus, valides ou non, faisait en quelque sorte partie de la \latin{familia} de l'évêque. 

 Si toutes les églises de l'Empire faisaient de même, aucune ne concurrençait l'Annone sur son propre terrain. Les bénéficiaires n'étaient ni des hommes honorables ni des citoyens, c'étaient des indigents, des vieillards et des infirmes, beaucoup de femmes et de petits enfants, des esclaves et des concubines abandonnés, des étrangers sans ressources, exilés ou bannis... Leur sexe et leur statut ne comptaient pas. Ils étaient recrutés sur critères « sociaux », ce qui pouvait attirer les plus pauvres vers l'église de leur cité : c'est probablement pour cela que lorsque l'empereur Julien (361-363) a entrepris de restaurer les cultes traditionnels il a cherché à engager le clergé païen dans cette voie.

 Pourquoi les chrétiens donnaient-ils aux pauvres ? Selon la Bible Dieu préférait à tous les sacrifices l'assistance faite aux pauvres. Selon les évangiles il regardait ce qui était fait aux pauvres, aux malades, aux étrangers, aux prisonniers… comme si c'est à lui-même que c'était fait :

\begin{displayquote}[Matthieu, 25, 31-40]
\emph{Quand le Fils de l'homme viendra dans sa gloire, escorté de tous les anges, alors il prendra place sur son trône de gloire. Devant lui seront rassemblées toutes les nations, et il séparera les gens les uns des autres, tout comme le berger sépare les brebis des boucs. Il placera les brebis à sa droite, et les boucs à sa gauche. Alors le Roi dira à ceux de droite : » venez, les bénis de mon père, recevez en héritage le royaume qui vous a été préparé depuis la fondation du monde. Car j'ai eu faim et vous m'avez donné à manger, j'ai eu soif et vous m'avez donné à boire, j'étais un étranger et vous m'avez accueilli, nu et vous m'avez vêtu, malade et vous m'avez visité, prisonnier et vous êtes venus me voir. » Alors les justes lui répondront : « Seigneur, quand nous est-il arrivé de te voir affamé et de te nourrir, assoiffé et de te désaltérer, étranger et de t'accueillir, nu et de te vêtir, malade ou prisonnier et de venir te voir ? ». Et le roi leur fera cette réponse : « En vérité je vous le dis, dans la mesure où vous l'avez fait à l'un de ces plus petits de mes frères, c'est à moi que vous l'avez fait. »}
\end{displayquote}

%et :
%
\begin{displayquote}[Matthieu 25, 45-46]
\emph{Alors il leur répondra : « en vérité je vous le dis, dans la mesure où vous ne l'avez pas fait à l'un de ces plus petits, à moi non plus vous ne l'avez pas fait. » Et ils s'en iront, ceux-ci à une peine éternelle, et les justes à la vie éternelle.}
\end{displayquote}

 Le christianisme n'étant pas reconnu par l'état les églises n'étaient pas habilitées à recevoir des donations. Cela n'a pas empêché les dons d'avoir lieu ni un patrimoine de se constituer, sous couvert de prête-noms. Bien avant le quatrième siècle on constate l'existence d'une « zone grise », où les églises sont tolérées et où un patrimoine leur est tacitement reconnu par les autorités, même si une persécution pouvait tout remettre en question de temps en temps. 

 Les donateurs les plus généreux étaient les femmes de l'aristocratie romaine, veuves ou divorcées, libérées de toute tutelle, et n'ayant pas vocation à recevoir un \emph{honneur} c'est-à-dire moins susceptibles que les hommes d'être contraintes à assumer les frais d'une \emph{évergésie}). La loi les disait \latin{sui juris} (autonomes juridiquement) si elles avaient donné le jour à trois enfants au moins, et si leur \latin{pater familias} était décédé. Elles préféraient se mettre sous la protection morale de l'évêque de leur choix plutôt que de se soumettre à un nouveau mari auquel il leur serait très difficile de s'opposer, sans que cela garantisse pour autant qu'il soit plus intéressé par leur personne que par leur dot%
% [5]
\footnote{La dot, contribution du père de l'épouse aux dépenses du ménage, appartenait à cette dernière, mais elle était gérée par le mari. Par contre une femme \latin{sui juris} pouvait gérer elle-même tout le reste de sa fortune personnelle. La coutume voulait que la totalité des biens d'une femme parvienne à ses seuls enfants, mais il fallait pour cela qu'elle teste activement en leur faveur. Cela lui laissait en théorie la possibilité de tester pour d'autres qu'eux.}%
. Leur refus de se remarier était le moyen de leur liberté. Il leur permettait d'échapper aux grossesses et aux accouchements. Il évitait à leurs enfants d'être mal traités par un beau-père et de rentrer en conflit pour leur héritage (maternel) avec ceux d'un nouveau lit. Et enfin il leur laissait la liberté de jouer un rôle social valorisé par leur communauté. Les palais de ces grandes dames pouvaient offrir aux femmes sans ressources un refuge contre un monde « machiste » : c'est ainsi que se sont agrégés les noyaux des premières communautés religieuses féminines. 

 Quelques hommes imitaient ces nombreuses donatrices dans la mesure où leurs responsabilités de chefs de famille (ce que les femmes n'étaient pas) leur en laissaient la possibilité. C'était d'abord le cas des hommes mariés \emph{sans enfant}.

 Les chrétiens aisés étaient donc invités à infléchir en ce sens leurs activités d'évergètes. Ils étaient les plus capables d'accueillir la communauté dans leurs vastes demeures, et les plus susceptibles de disposer des ressources nécessaires pour subvenir aux besoins des plus pauvres. Comme chaque église locale élisait ses clercs et son évêque, les chrétiens avaient tendance à les choisir parmi les plus riches d'entre eux, selon la tradition antique de confier les responsabilités civiques aux notables. 

 En contrepartie des moyens de vivre qu'il offrait à ses protégés, à ses « clients », l'évêque avait comme tout \emph{patron} romain le droit de leur demander des services. Sans charges de famille, les célibataires étaient disponibles le jour et la nuit, et ils n'avaient ni épou(x)se ni enfants à prendre en charge. Leur entretien pouvait donc être nettement plus économique que celui de personnes mariées de niveau social égal. Ils (elles) pouvaient se vouer à la pauvreté sans être irresponsables face à une descendance. Dans ce cas le coût de leur travail pouvait être compétitif avec celui des esclaves. Au premier rang des services que les laïcs voués à la continence pouvaient rendre, il y avait les tâches d'assistance. La première initiative institutionnelle de la toute première communauté chrétienne (à Jérusalem) avait été de créer un corps de clercs spécialisés dans les tâches d'assistance, les \emph{diacres} (dont rien ne nous dit qu'ils étaient célibataires), dont la tâche était de visiter les malades à domicile, et de gérer l'assistance, et d'abord l'assistance aux veuves (\emph{Actes des Apôtres} 6, 1-6). 



Au nombre des pauvres figuraient d'abord les veuves. L'assistance aux veuves consistait en distribution d'argent, de vêtements, de nourriture,~etc. Dans un monde où les femmes ne pouvaient exercer une activité honnête que dans un cadre familial, une veuve pauvre qu'aucun parent ne recueillait était dans une détresse totale, surtout si elle avait des enfants à charge. 

 La situation était encore pire pour une ex-concubine. Le cas de la veuve qui prostitue sa fille parce que ni l'une ni l'autre n'ont aucun autre moyen de vivre est un poncif de la littérature, de l'Antiquité au \siecle{19}. La mère qui en était réduite là n'avait le plus souvent aucun choix : si elle vendait sa fille comme esclave, celle-ci n'échapperait probablement pas à la prostitution, sauf à être vendue comme esclave-concubine (mais il fallait trouver un protecteur), et elle perdrait sa liberté, tandis que sa mère n'aurait aucune ressource. En la prostituant la mère lui faisait perdre \emph{seulement} sa réputation. 

 Les veuves sans enfants à charge secourues par l'Église pouvaient s'employer au service des œuvres de l'évêque. Elles pouvaient s'occuper des enfants orphelins de père et de mère, des malades, des infirmes, vieillards et insensés,~etc. Elles étaient chargées des fonctions que seule une femme pouvait remplir sans faire jaser : soins aux femmes et aux petits enfants, visites aux femmes à domicile (instruction religieuse, soutien psychologique, assistance matérielle),~etc. Pour remplir ces missions on choisissait des matrones d'expérience, de bonne réputation (c'est-à-dire non infâmes) et d'un âge suffisant pour inspirer le respect (âge « canonique »). 

 Au fil des trois premiers siècles de notre ère le \emph{rôle social des veuves} soutenues par l'Église s'est effiloché. Sauf exception elles ont peu à peu cessé d'assumer des tâches de service auprès des femmes et des enfants. On a cessé d'attendre d'elles autre chose qu'une vie réglée, et si possible pieuse et édifiante. Leurs fonctions étaient désormais assumées par les vierges consacrées, et notamment tout ce qui a trait à l'assistance, qui dans l'aire catholique a presque exclusivement reposé sur des religieuses (et des religieux) jusqu'au \siecle{19} inclus. 



On a vu que les juifs considéraient que c'était une œuvre pieuse que de prendre en charge l'éducation des orphelins pauvres. D'ailleurs les non-chrétiens en faisaient autant. La loi romaine prévoyait que les mineurs orphelins de père (libres, ingénus ou affranchis) soient confiés à l'autorité d'un tuteur. La tutelle des orphelins était une charge et un devoir civiques, dont seules quelques professions étaient exemptées. Le tuteur était responsable de la gestion des biens du mineur et de son éducation, mais il n'était pas chargé de la financer lui-même. Les dépenses étaient supportées par l'héritage de l'enfant. Que se passait-il quand un orphelin ne possédait rien, quand ses parents ne lui avaient laissé aucun héritage, ce qui devait être fréquent ? 

 Quels étaient les orphelins assumés par les communautés dont les textes ecclésiaux les plus anciens parlent si souvent? Étaient-ce les enfants des martyrs ? Probablement pas : ce n'est qu'à quelques moments exceptionnels et brefs que le nombre de ceux-ci a été significativement élevé. Il s'agissait plus vraisemblablement des enfants des fidèles sans ressources décédés d'accident, de misère ou de maladie avant qu'eux-mêmes ne soient capables de gagner leur vie, ce qui était une situation banale à l'époque. Ceux dont les deux parents étaient morts n'étaient pas donnés en adoption. Les chrétiens s'y refusaient au même titre que les juifs : ces enfants appartenaient déjà à une famille connue même si celle-ci n'avait plus d'autres membres vivants qu'eux-mêmes. 

 Les orphelins secourus par l'Église pouvaient aussi être les enfants des veuves signalées plus haut : en effet les textes associent constamment \emph{les veuves et les orphelins}. Les uns et les autres étaient dans la misère et couraient un grave risque de tomber dans l'esclavage ou/et la prostitution (cela restera une constante : les plus pauvres sont toujours les femmes seules avec enfants). Selon la loi la tutelle d'un orphelin de père ne pouvait pas être exercée par sa mère : les femmes ne pouvaient exercer aucune autorité sur autrui, même sur leurs propres enfants, sinon par délégation (du père ou du tuteur).

 Sous quelle forme les communautés apportaient-elles aux enfants sans parents leur soutien matériel et moral ? On peut imaginer bien des formules, en fonction de l'âge des enfants et du contexte, depuis la nourrice, salariée jusqu'à ce que l'enfant puisse commencer à travailler (très tôt, comme tous les pauvres d'alors), ou la mise en apprentissage chez un professionnel aux frais de la communauté, jusqu'à la prise en charge collective de grands enfants dans la maison de l'évêque et sous son contrôle. 

 Certains de ces orphelins pouvaient être investis de manière spéciale : les plus vifs d'esprit, ou ceux dont les parents avaient laissé le meilleur souvenir. Certains garçons trouvaient une place dans le clergé : orientation naturelle s'ils n'avaient pas eu d'autre figure paternelle que des membres de ce même clergé.



Les malades et les infirmes de la communauté étaient visités à leur domicile. Ce service de proximité qui ne sépare pas le sujet de son milieu était l'un des premiers services assuré par les veuves auprès des femmes. Les diacres rendaient le même service aux hommes. Rien n'interdisait aux clercs ordonnés de pratiquer la médecine, du moment qu'ils ne versaient pas le sang. 

Vers 260 Denys d'Alexandrie vante le comportement des chrétiens face à une des épidémies de son temps : 

\begin{displayquote}
\emph{La peste fondit sur la ville, objet d'épouvante. La plupart de nos frères ne s'écoutaient pas eux-mêmes, visitant sans précaution les malades, leur donnant leurs soins dans le Christ. Chez les païens il en était tout autrement ; ceux qui commençaient à être malades, on les chassait, on fuyait ceux qui étaient les plus chers, on jetait sur la route des gens à demi-morts.}
\end{displayquote}

L'aide de la communauté se bornait-elle à ses propres membres ? 

 Une prise en charge totale était inévitable quand il s'agissait de personnes sans domicile, de voyageurs malades, de vieillards sans famille, d'esclaves abandonnés, d'orphelins sans fortune et sans parents, d'insensés trop agités ou trop régressés... Comme les moyens de l'hospitalité individuelle n'étaient pas illimités, il était dans l'ordre des choses que des formes collectives de prise en charge aient progressivement été mises en place, au moins dans les grands centres, sur le modèle alors bien connu des hôpitaux de garnison. 



Toute l'Antiquité, chrétiens y compris, croyait aux démons et à leur capacité de nuisance sur le corps comme sur l'esprit. Les chrétiens incluaient souvent parmi les démons tous les dieux païens, auxquels ils croyaient eux aussi à leur façon. Ceux qui selon les conceptions du temps étaient possédés par un démon étaient le plus souvent des malades mentaux agités, des épileptiques atteints de grand mal, des autistes, des psychotiques délirants, des déments, de grands attardés mentaux et des hystériques en crise. Dès le \siecle{2} des énergumènes (littéralement : « possédés par un démon ») étaient hébergés à plein temps dans des locaux dépendant de l'Église (qui s'en chargeait auparavant ? Les familles les gardaient-elles ou les abandonnaient-elles à la rue ou bien dans un temple aux bons soins d'Esculape ou de tel ou tel autre dieu, c'est-à-dire à la charge de la charité des fidèles et de la bonne volonté des desservants du temple ?) 

 Dont le but de chasser leurs démons ces malades bénéficiaient d'exorcismes quotidiens dont étaient chargés leurs gardiens. La fonction d'exorciste ne s'est individualisée que lentement. Longtemps chaque chrétien a été jugé assez compétent pour chasser les démons en imposant les mains sur le front des possédés, sans avoir besoin d'une ordination spéciale. Selon les Constitutions Apostoliques (une compilation rédigée durant les années 380 de textes réglementaires ou liturgiques plus anciens : \fsc{FAIVRE}, \emph{Naissance d'une hiérarchie}, 1977, p. 118) : \emph{l'exorciste n'est pas ordonné, car être exorciste dépend de la bienveillance, de la bonne volonté et de la grâce de Dieu donnée par le Christ par la venue du Saint-Esprit. Celui qui a reçu le don de guérison est manifesté par une révélation de Dieu et la grâce qui est en lui est visible pour tous.}

 Comment cette fonction a-t-elle évolué par la suite ? Les \latin{statuta ecclesiae antiqua} sont une œuvre privée et non la transcription de décisions faisant autorité (d'un concile, du Pape, ou d'un évêque). Ils n'en sont pas moins un témoin de l'état des pratiques et des opinions du temps de leur rédaction : vers 476-485. Ils font aux exorcistes un devoir d'être patients et bons avec leurs protégés, et d'imposer les mains à chacun d'eux tous les jours. Ce geste s'accompagnait d'une prière. Ils avaient aussi le devoir de leur apporter leur nourriture quotidienne à l'heure prescrite, sans les faire attendre : « Les énergumènes qui séjournent dans la maison de Dieu doivent recevoir en temps voulu leur pitance quotidienne qui leur est apportée par les exorcistes. » (canon 64). Cette dernière remarque suggère que beaucoup des patients présentaient des traits de retard mental ou de régression massifs qu'on faisait travailler à la mesure de leurs capacités (exemple : \latin{pavimenta domorum dei energumeni everrant} : « les énergumènes balaieront le pavement des églises ») et que les exorcistes étaient plutôt des garde-malades vigoureux que des experts en pathologie mentale ou en théologie. 

 Comme les infirmiers psychiatriques de naguère les exorcistes n'ont jamais été placés bien haut dans la hiérarchie du clergé (clercs « mineurs »). Au fil du temps ils ont été relégués avec leurs patients de plus en plus loin de l'autel, au plus bas niveau, près de la porte. À partir du \siecle{6} ou du \siecle{7} les plus savants des théologiens cesseront de croire que le démon se cache derrière les états pathologiques ordinaires. Les actes d'exorcisme encore pratiqués, devenus tout à fait rares, seront réservés à des prêtres nommés pour cela. 



En raison de la valorisation évangélique de la pauvreté et de la souffrance, et en raison de leurs pratiques d'assistance, les chrétiens ne pouvaient qu'attirer tous les accidentés de l'existence : ceux et celles qui avaient tout perdu, qui ne pouvaient plus travailler, les vieillards sans enfants qui n'avaient plus la force de gagner leur pain, les esclaves abandonnés (« libérés ») par leurs maîtres parce que trop vieux, infirmes ou malades, les concubines abandonnées sans enfants pour les recueillir, les prostituées âgées, les exilés inconsolables, les bannis de leur cité,~etc. 

 Les premières Églises ont adopté la même règle que les synagogues face aux coreligionnaires (face aux « frères ») en déplacement, en voyage d'affaire, en mission pour leur communauté, et face aux vagabonds : les voyageurs valides étaient reçus comme des hôtes pendant trois jours, après quoi ils étaient invités à poursuivre leur chemin ou à gagner leur vie en se mettant au travail, conformément au mot de Saint-Paul : \emph{celui qui ne travaille pas n'a pas droit de manger}. 

 Quant à ceux qui étaient hors d'état de continuer la route ils étaient soignés durant le temps qu'il fallait pour qu'ils se remettent : cela pouvait durer des mois ou des années. Cela pouvait durer tout ce qui restait d'une vie.



Les chrétiens de l'Antiquité rencontraient de multiples occasions de s'occuper de captifs :
\begin{enumerate}
% 1°)
\item ils subissaient sporadiquement des persécutions qui n'étaient pas toujours évitables. À chaque retour de flamme de ces persécutions les plus convaincus, les plus provocateurs ou les moins chanceux de ces fidèles se retrouvaient en prison ou étaient condamnés aux mines ou aux bêtes du cirque ;
% 2°)
\item d'autres parmi les fidèles étaient emprisonnés pour des crimes ou des délits de droit commun, pour dettes,~etc. Quel qu'ait pu être le motif de leur captivité, la communauté se devait de les visiter, si nécessaire sur le site lointain des mines où ils et elles purgeaient leur peine. Elle les soutenait moralement et les encourageait à tenir bon dans leur foi et la pratique religieuse. Elle essayait d'adoucir leur sort, par exemple en soudoyant les gardiens pour qu'ils leur procurent de meilleures conditions de vie ;
% 3°)
\item enfin les chrétiens étaient exposés au même titre que tous leurs contemporains au risque d'être enlevés, avec la perspective d'être vendus au loin comme esclaves. Ce risque était plus élevé pendant les périodes de trouble, et pendant tous les voyages. Mais les enlèvements étaient à craindre même en ville, notamment les enlèvements d'enfants. Pour revoir libres ceux qui avaient été enlevés il fallait négocier et rassembler une rançon. Les églises locales faisaient ce qu'elles pouvaient en fonction de leurs moyens. 
\end{enumerate}



En accord avec les différentes cultures antiques la Bible faisait un devoir à quiconque était présent de donner une sépulture décente à toute personne décédée, sans aucune exception. L'importance des rites funéraires
 était essentielle, et ne pas ensevelir un mort était un \emph{sacrilège} (cf. Antigone), en dépit de l'impureté qui touchait celui qui s'en chargeait. Le soin d'autrui n'était terminé que quand son corps avait été enseveli selon les règles. Même les plus pauvres cotisaient pour se payer une place dans un tombeau collectif et pour que leur soit rendu le culte mortuaire approprié. 
 Pour que tous les pauvres aient droit à une inhumation décente les églises finançaient un service collectif d'inhumation. Les synagogues en faisaient semble-t-il autant. Encore une fois les premières communautés chrétiennes se sont coulées dans le moule juif. Les fossoyeurs avaient le devoir d'enterrer tous les morts inconnus ou indigents trouvés dans les terrains vagues, au bord des routes ou sur les rivages. Le signe de l'importance symbolique de cette fonction, c'est qu'on s'est longtemps demandé s'ils faisaient ou non partie des clercs mineurs. 



Quand les Écritures prescrivaient de prendre soin des orphelins, est-ce que ce mot recouvrait les enfants abandonnés anonymement, les enfants trouvés ? L'étymologie n'interdit pas de le penser. Le grec \latin{orphanos} désignait en effet l'enfant privé de l'un ou l'autre de ses deux parents, notamment de son père. En ce sens les enfants exposés étaient des orphelins, même quand leurs parents étaient bien vivants. Pas plus que les juifs les chrétiens n'avaient le droit moral d'abandonner leurs enfants, même si ceux-ci étaient trop nombreux ou mal formés. Les écrivains chrétiens soutenaient comme un fait d'observation quotidienne que les païens laissaient mourir beaucoup de leurs enfants. Mais les fidèles pouvaient avoir les mêmes raisons que les autres d'exposer des nouveaux-nés, sauf à supposer qu'ils aient tous et toujours vécu dans une rigueur morale impeccable, ce qui est peu vraisemblable%
%[16] 
\footnote{Les Épîtres de Paul contiennent nombre de passage où il sermonne vertement ses ouailles pour des fautes morales caractérisées (par exemple la \emph{Première épître aux Corinthiens}, chapitres 5 et 6).} 
: enfant né d'un adultère, d'un inceste, d'un viol, d'une rencontre sexuelle sans lendemain, difficulté d'accepter un enfant mal formé,~etc. Rien ne permet non plus d'assurer que l'assistance de l'église mettait tous les chrétiens à l'abri de la misère, et qu'aucun n'était écrasé sous le poids matériel ou psychologique de ses enfants. On ne saura jamais quel était le pourcentage de leurs enfants que les chrétiens abandonnaient : peut-être était-il faible dans les communautés petites et ferventes, où tout le monde connaissait tout le monde, et surtout si l'assistance mutuelle y fonctionnait correctement ? 

 Les apologistes antiques du christianisme craignaient que la plupart des enfants abandonnés ne soient prostitués (ce qui implique d'ailleurs qu'ils ne croyaient pas qu'ils étaient promis à la mort%
% [17]
\footnote{John \fsc{BOSWELL}, \emph{Au bon cœur des inconnus}, 1993.}%
). Il était louable de les recueillir pour les protéger de ce risque. Mais quel était le statut des enfants ainsi pris en charge ? En effet nul ne pouvait prouver qu'ils étaient nés de conceptions régulières, et les règles de pureté de la \emph{Tora} les classaient parmi les \emph{mamzerim}, les impurs de naissance. La situation de ceux dont on avait bien connu les parents de leur vivant (« orphelins pauvres ») paraissait autrement digne d'intérêt, et le restera jusqu'au \siecle{19}. 

 On peut formuler plusieurs hypothèses :
\begin{enumerate}
% 1°) 
\item On a vu que les juifs étaient opposés à l'adoption telle que la pratiquaient les païens. L'adoption proprement dite, celle qui d'un étranger fait le fils et l'héritier d'une famille et de ses ancêtres (adoption « plénière ») disparaîtra dès que les chrétiens seront en mesure d'orienter les décisions impériales : cela suggère que les chrétiens campaient sur la même position que les juifs. Il paraît donc difficile de croire que les enfants abandonnés aient pu être régulièrement adoptés par eux.
% 2°)
\item Par contre, ils pouvaient donner à ces enfants le statut d'\latin{alumnii}, sans confondre l'entrée dans leur \latin{familia} avec l'entrée dans leur famille. Les païens et les juifs le faisaient bien ! Cela n'en faisait pas leurs enfants, sinon en un sens spirituel. Ils pouvaient tout de même les établir dans la vie. C'était une pratique socialement et religieusement valorisée.
% 3°)
\item Enfin rien ne leur interdisait non plus d'élever les enfants abandonnés dans le statut d'esclave, ce qui était une pratique traditionnelle. 
\end{enumerate} 

 En droit celui qui voulait prostituer un enfant trouvé était d'ailleurs obligé de lui donner le statut d'esclave, sans quoi son corps était protégé par la loi. On peut supposer que la plupart des fidèles ne s'autorisaient pas cette pratique, étant donné la véhémence avec laquelle les écrivains chrétiens de l'époque la dénoncent comme une abomination païenne. L'accueil des enfants trouvés pouvait être vu comme une bonne œuvre, dans le contexte de cette époque, du moment que l'accueillant s'interdisait d'exploiter le corps de l'enfant et qu'il s'efforçait de lui donner une bonne éducation. Celui qui prenait un enfant trouvé pour en faire son esclave était certes moins généreux que celui qui l'élevait en ingénu. Mais désormais ce nouveau-né sans droits n'était plus l'enfant de personne ni de nulle part, il faisait partie d'une \latin{familia} dont il recevait le nom et il trouvait une place non infamante, si petite fut-elle.

 Il n'y a pas de raison pour que ce type d'accueil ait laissé des traces dans les archives. Les enfants concernés n'entraient en effet dans les registres d'état civil que lorsqu'ils avaient reçu la liberté d'une personne libre. Sinon ils n'étaient pas enregistrés. La suite de l'histoire suggère pourtant que cette troisième solution a été fréquente, sinon la plus fréquente : dès Constantin les enfants trouvés ont en effet été mis à la charge des autorités civiles (cités, rois, seigneurs...), et ils ont aussi été comptés au nombre de leurs dépendants (« leurs hommes »), comme les futurs serfs et parmi eux. 
 
















% Le 03.03.2015 :
% Antiquité
% Moyen Âge
% ~etc.
% ~\%



\chapter{Une contre-société chrétienne}


\section{Laïcs et laïques consacrés}

 Si la fécondité et le mariage cessaient d'être des obligations morales, alors les jeunes gens avaient le droit moral de s'y refuser, non plus en raison du manque d'attrait du parti proposé par leur père, ce qui était parfaitement admis à Rome, mais en raison d'une préférence pour le célibat « en vue de Dieu » et pour la continence, chose qui jusque là n'avait pas de sens. C'était donner aux jeunes gens le droit de choisir une vie indépendante du désir de leur \latin{pater familias}. C'était aller directement contre l'intérêt des familles tel que celles-ci le concevaient : en droit les jeunes gens qui avaient un \latin{pater familias} étaient mariés par celui-ci. Si le consentement des futurs époux était obligatoire à Rome, en réalité il était présumé acquis à partir du moment où ils ne protestaient pas trop bruyamment. Les parents de ce temps aussi n'hésitaient pas à déterminer eux-mêmes, en toute bonne conscience, le destin de leurs enfants. Au même moment et en dépit des interdictions légales des enfants de tous âges étaient vendus ou abandonnés aux prêteurs sur gage par des parents sans ressources, ce qui ne leur laissait guère de chances de se marier.

 Étant donné que quel que soit son âge une fille non mariée vivait sous l'autorité de son père ou de son tuteur, si elle voulait rester célibataire il fallait qu'elle le convainque de ne pas la marier. Or les représentations juives et romaines faisaient au contraire un devoir aux pères de marier leurs filles. Lorsqu'ils tardaient à s'acquitter de ce devoir, la rumeur publique était prompte à faire courir le bruit que la jeune personne manquait d'attraits physiques, ou qu'un problème de santé rendait dangereuse ou impossible pour elle la procréation, ou que des problèmes financiers empêchaient la constitution d'une dot, ou bien encore que l'amour du père pour sa fille était excessif, sinon coupable. 

 Comment comprendre alors que Paul de Tarse ait pu écrire que s'il le trouvait bon un père pouvait vouer sa fille au célibat au lieu de la marier ? En effet selon lui :

\begin{displayquote}[I~Cor~7,~36-38]
\emph{Si pourtant quelqu'un estime qu'il n'est pas honorable pour sa fille de dépasser l'âge du mariage, et qu'il doit en être ainsi, qu'il fasse ce qu'il veut ; il ne pèche pas : qu'ils se marient. De même celui qui est fermement décidé en son cœur, à l'abri de toute contrainte et libre de son choix, s'il décide de garder sa fille, il agira bien. Ainsi celui qui marie sa fille fait bien, et celui qui ne la marie pas fait mieux}.
\end{displayquote}

À qui Paul apportait-il son appui ? ... aux pères contre leurs filles ? ... ou bien aux pères et aux filles contre le reste du monde ? Ceci dit ce texte a plus d'une fois servi au fil des siècles à imposer à des filles un choix de vie qui n'était pas le leur. 
 
 Le choix du célibat pouvait être financièrement coûteux : du fait des lois d'Auguste ceux qui le choisissaient étaient exclus des héritages et soumis à l'impôt sur les célibataires. Mais surtout il était reconnu aux pères le droit d'émanciper ou de déshériter les jeunes récalcitrants. Ces jeunes étaient alors sans ressources, ce qui les contraignait à travailler pour autrui comme salarié (mercenaire), ce qui rapportait peu, ou à se vendre comme esclaves. Les filles qui choisissaient le célibat se condamnaient en outre à demeurer toute leur vie des mineures légales, alors que celles qui donnaient le jour à trois enfants étaient dispensées de tutelle à partir du décès de leur \latin{pater familias} et pouvaient gérer elles-mêmes leur fortune personnelle, même lorsqu'elles étaient mariées : une fois veuves elles n'avaient plus de comptes à rendre à personne sur l'emploi de leurs biens. Quant aux pauvres, qui ne pouvaient espérer aucun héritage, ils avaient encore moins que les autres intérêt à demeurer célibataires s'ils voulaient préparer leurs vieux jours. 

 Mais même si le mariage cessait d'être une obligation, qu'est-ce que cela changeait pour les femmes dans un monde où les filles et les femmes honorables ne pouvaient travailler qu'à leur domicile ou sous le regard direct d'un père, d'un frère ou d'un époux ? Comment les veuves et les épouses répudiées sans père et sans fortune personnelle (et la pauvreté des concubines abandonnées était encore pire) pouvaient-elles subsister honorablement, sinon en se mettant sous la protection d'un concubin ou d'un époux ? À quel autre moyen non infamant pouvaient-elles recourir pour ne pas en être réduites par la faim à vendre leurs enfants et à se vendre elles-mêmes ? À quoi bon dire du mal du remariage ? Se remarier ne valait-il pas mieux que se prostituer ?

 Les problèmes matériels posés par le choix du célibat étaient identiques à ceux des candidats au baptême qui vivaient jusque là dans une situation personnelle désapprouvée par l'Église : personnel des temples païens, fabricants d'idoles et \latin{d'ex-voto} païens, prostitués des deux sexes, acteurs, danseurs et danseuses, gladiateurs, organisateurs de spectacles,~etc. Aucun d'eux n'était exclu du salut puisque le baptême lavait tous les péchés antérieurs, mais ils ne pouvaient être accueillis dans la communauté qu'à la condition de cesser de vivre comme ils l'avaient fait jusque là. Une prostituée qui renvoyait ses clients, une maquerelle, un sculpteur d'idoles, un acteur, un gladiateur qui cessaient leur activité mouraient de faim. Il n'était pas imaginable de les renvoyer en s'en lavant les mains. 

 Les premiers chrétiens ne pouvaient pas se contenter de proclamer la valeur du célibat et de la continence. S'ils y croyaient vraiment il leur fallait le rendre possible. Leurs propres enseignements leur interdisaient de se désintéresser de la situation de ceux et surtout de celles qui faisaient ce choix sur leurs conseils. Ils devaient leur procurer les moyens de mettre en œuvre l'aspiration qu'ils avaient suscitée ou se taire. Les évêques ont donc été conduits à prendre fait et cause pour tous ceux de leurs fidèles qui faisaient un choix de vie radical. Si nécessaire ils se sont mis en avant pour soutenir les jeunes dans leurs combats contre leurs familles. 

 Les \emph{vierges consacrées} (parfois nommées \emph{diaconesses} dans l'aire grecque) sont apparues très tôt, dès le deuxième siècle. A la différence des diacres, des prêtres et des évêques, elles n'ont jamais fait pas partie des clercs. C'étaient des laïques, à l'instar des futurs moines, dont elles étaient l'équivalent féminin. Elles n'étaient pas \emph{ordonnées} pour mener à bien une mission ni pour tenir une place dans la hiérarchie. Elles choisissaient le célibat pour se consacrer à la prière et au service de la communauté, notamment à la catéchisation des femmes et au service des malades. Leurs vœux étaient reçus par leur évêque au cours d'une cérémonie qui a très tôt été interprétée comme un mariage mystique avec le Christ, auquel elles se vouaient. Le fait qu'à compter de ce jour elles portaient le voile (d'où le nom de la cérémonie : « prise de voile ») affichait cette interprétation de leur choix de vie à leurs contemporains. Il s'agissait en effet du voile des matrones qui proclamait qu'elles avaient un \latin{dominus}, un seigneur et maître. Cela les rapprochait des vestales, vouées au célibat et à la virginité, qui elles aussi portaient un voile qui avertissait les passants que leur corps était consacré, et qu'y toucher était sacrilège.


\section{Clergé et continence}

 En ce qui concerne les desservants du culte%
% [2]
\footnote{Pour cette partie je me suis en particulier servi de \emph{Naissance d'une hiérarchie, les premières étapes du cursus clérical,} d'Alexandre \fsc{FAIVRE}, 1977.}
l'Église n'a pas adopté les pratiques des synagogues. Les rabbins sont des sages, des savants, des exégètes et des juges, et non des prêtres, des sacrificateurs. Ils ne président au culte qu'avec l'assentiment des fidèles. Jusqu'au milieu du Moyen Âge au moins ils ne seront pas payés. Ils sont choisis par les fidèles eux-mêmes, et non reçus d'une autorité extérieure à la communauté. Ils ont moins un pouvoir normatif qu'un pouvoir d'influence, gagné par leur réputation de compétence, par leur savoir. Ils possèdent une autorité intellectuelle, morale et juridique acquise, reconnue par leurs pairs et non octroyée, même si la notion d'ordination par imposition des mains ne leur était pas étrangère. Cela signifie que le culte des synagogues était fondamentalement différent de celui des temples antiques. Il ne prétendait pas remplacer les sacrifices offerts au Temple de Jérusalem : bien au contraire si les prières faisaient référence aux sacrifices offerts au Temple, c'était pour rappeler que ces sacrifices n'étaient pas possibles, et qu'on ne pouvait qu'accepter la réalité de leur absence. 

 Au contraire le rite chrétien de la \emph{fraction du pain} se présentait comme le renouvellement littéral des gestes inaugurés par Jésus à la \emph{Cène}, selon le récit des Évangiles. Il n'était pas interprété comme un geste symbolique, destiné seulement à faire mémoire d'un événement historique une fois pour toutes réalisé. Il avait d'emblée été compris par les premiers chrétiens comme un geste efficace, performatif, comme un geste qui produisait ce qu'il énonçait, qui réalisait à la lettre ce qu'il disait : \emph{prenez et mangez : ceci est mon corps \emph{[...]} Buvez-en tous : car ceci est mon sang, le sang de l'alliance, qui va être répandu pour une multitude en rémission des péchés}. (Matthieu, 26, 28). L'Eucharistie a d'emblée été un rite surdéterminé. Pour Paul le « repas du Seigneur » est un repas pris en commun et c'est une anticipation du festin eschatologique, à la fin des temps (1~Co~11, 17-34). C'est le rappel des repas pris avec Jésus avant sa mort, mais aussi après sa résurrection. C'est un repas où le Seigneur se donne, en personne, à son peuple comme nourriture. C'est « la Pâque du Seigneur » où Jésus est l'agneau pascal dont le sang consacre ceux qui le reçoivent (1~Co~5, 7). C'est un renouvellement de l'alliance (1~Co~11, 25) : dans la \emph{Tora} le sang de l'alliance scellait la communion entre le peuple et Dieu, le sang du Christ instaure une communion encore plus profonde. C'est un sacrifice de louange. C'est donc à plusieurs titres que Paul de Tarse faisait de l'Eucharistie un sacrifice bien avant la destruction du Temple, et alors qu'à Jérusalem le \emph{sacrifice perpétuel} continuait d'être tous les jours offert, et que selon les \emph{Actes des Apôtres} les membres de la communauté judéo-chrétienne locale continuaient d'y assister assidûment.

 Mais en 70 de notre ère la destruction du Temple a contraint les juifs à cesser tout sacrifice sanglant : cela a permis aux chrétiens de penser et de dire que le \emph{sacrifice perpétuel} continuait désormais dans un autre lieu et sous une autre forme, non sanglante. Le clergé chrétien s'est pensé comme le successeur du clergé du Temple et a pris modèle sur lui. L'évêque s'est identifié au grand prêtre, les prêtres chrétiens aux prêtres juifs, les diacres et les autres ministres du culte aux lévites, d'autant plus soucieux de se conformer à leurs modèles qu'ils prétendaient les remplacer et ne pouvaient ignorer que par ailleurs ils récusaient ces mêmes modèles.

 À ses débuts l'Église avait disqualifié la notion d'impureté rituelle, notamment par la voix de Paul qui en bon pharisien savait de quoi il parlait, et qui ne voyait pas d'inconvénient à manger des viandes consacrées aux idoles, du moment que cela ne choquait pas les esprits faibles (Rom, 14, 2-3; Cor I, 8, 4-13). Mais la notion d'impureté a fait un retour en force avec la constitution d'un personnel religieux permanent, spécialisé, consacré par l'imposition des mains et mis à part : un \emph{clergé}. Le clergé chrétien va en effet s'imposer de respecter scrupuleusement les règles de pureté qui s'appliquaient aux prêtres du Temple. Il va surenchérir par rapport à ces règles et cela va renforcer la sacralisation du sexe et la valorisation de la continence (même si cette imitation n'est que l'un des éléments qui y ont contribué). En effet ses membres étaient en permanence susceptibles de toucher les « vases sacrés ». S'ils prenaient modèle sur le Temple cela exigeait d'eux qu'ils soient en permanence dans l'état de pureté rituelle exigé du Grand Prêtre durant ses semaines de service. Il fallait donc qu'ils ne soient jamais souillés par « l'impureté » sexuelle. Les clercs vont donc très tôt être voués à la continence perpétuelle, à partir de leur ordination diaconale. 

 Avant qu'ils aient atteint l'âge où ils devaient être ordonnés diacres les clercs pouvaient se marier et cohabiter avec une épouse pour avoir des enfants. L'imposition des mains (ordination diaconale) ne se faisait en effet qu'à un âge relativement avancé, 30 ans, si ce n'est 40 (prêtre vient de presbytre, c'est-à-dire « ancien, âgé »). Cela laissait un temps suffisant à un clerc marié dès ses 18 ou 20 ans pour avoir des enfants. Il était donc normal de rencontrer des fils de diacres, de prêtres ou d'évêques. Mais si un enfant leur naissait après leur ordination, les membres du clergé devaient en principe être démis de leurs fonctions. Ceci dit bien des clercs mineurs demeuraient leur vie entière dans des fonctions qui n'exigeaient pas la continence. 


\section{Il n'y a plus ni esclave ni homme libre ?}


 
 
 En Ga 3, 28, Paul de Tarse écrit : \emph{Il n'y a ni juif ni grec, il n'y a ni esclave ni homme libre, il n'y a ni homme ni femme ; car tous vous ne faites qu'un dans le Christ Jésus}. Il développe la même idée en 1 Co 12, 13 : du point de vue de Dieu, il n'y a aucune différence entre les esclaves et les hommes libres. Ils sont égaux en dignité, en valeur spirituelle et promis au même salut après leur mort. Sur cette terre il en était autrement : les chrétiens ne condamnaient pas l'esclavage, ce en quoi ils étaient d'accord avec la totalité des peuples de l'Antiquité. Ce n'étaient pas des révolutionnaires politiques et ils n'avaient qu'une confiance limitée dans les institutions et les pouvoirs humains. Ils croyaient que les changements de structure n'ont de chance d'atteindre leurs objectifs que si les cœurs ont d'abord été transformés. Avec Paul ils pensaient que le désir du mal est profondément inscrit en l'homme%
% [6]
\footnote{\emph{... en effet vouloir le bien est à ma portée, mais non pas l'accomplir : puisque je ne fais pas le bien que je veux et commets le mal que je ne veux pas.} (Rm, 7, 18-19).}%
. Pour lui comme pour les apôtres, pour ses maîtres pharisiens ou pour les stoïciens l'esclavage le plus grave était celui du péché%
%[7]
\footnote{\emph{En vérité, en vérité, je vous le dis, tout homme qui commet le péché est un esclave. Or l'esclave n'est pas pour toujours dans la maison, le fils y est pour toujours. Si donc le fils vous affranchit, vous serez réellement libres. \emph{(Jn 8,34-36).} Ne savez-vous pas qu'en vous offrant à quelqu'un comme esclaves pour obéir, vous devenez les esclaves du maître à qui vous obéissez, soit du péché pour la mort, soit de l'obéissance pour la justice ? Mais grâces soient rendues à Dieu ; jadis esclaves du péché, vous vous êtes soumis cordialement à la règle de doctrine à laquelle vous avez été confiés, et, affranchis du péché, vous avez été asservis à la justice (j'emploie une comparaison humaine en raison de votre faiblesse naturelle) car si vous avez jadis offert vos membres comme esclaves à l'impureté et au désordre de manière à vous désordonner, offrez-les de même aujourd'hui à la justice pour les sanctifier.} (Rm 6,16-19).}
(des désirs non contrôlés par la raison). La vie n'était qu'un bref passage%
%[8]
\footnote{\emph{J'estime en effet que les souffrances du temps présent ne sont pas à comparer à la gloire qui doit se révéler en nous.} (Rm, 8, 18).}% 
, et le salut religieux, la \emph{vie éternelle}, était plus important que le bonheur ou la réussite terrestre. Il n'était pas déterminant pour le salut d'être libre ou d'être esclave. L'essentiel était déjà gagné par la mort du Christ : \emph{Vous avez été achetés%
%[9] 
\footnote{La rédemption était le paiement d'un marché, d'une rançon, le rachat d'une créance par un rédempteur, c'est-à-dire un entrepreneur prêt à risquer ses capitaux dans une affaire, ou dans le rachat d'un esclave à son maître pour le libérer.} 
cher} (1 Co 6, 20). En conséquence Paul enseignait aux esclaves que Dieu lui-même avait accepté qu'ils soient placés là où ils étaient, et il leur recommandait la patience, l'honnêteté, le travail bien fait, indépendamment de toute récompense terrestre, non pour leur maître, mais pour Dieu ... et pour la bonne réputation des chrétiens. 

 Dans le même temps Paul enseignait%
% [10]
\footnote{Paul a maintes fois répété le même enseignement -- ce qui veut peut-être dire que la question lui a souvent été posée ?} 
aux maîtres que les esclaves ont la même valeur qu'eux%
%[11]
\footnote{\emph{Aussi bien est-ce en un seul esprit que nous tous avons été baptisés pour ne former qu'un seul corps, juifs ou grecs, esclaves ou hommes libres, et tous nous avons été abreuvés d'un seul esprit.} (I Cor 12, 13).}% 
, et il leur faisait un devoir de les traiter \emph{comme des collaborateurs} et \emph{comme des frères en Jésus-Christ} :

\begin{displayquote}[Col 3, 22-25 ; 4,1]
\emph{Esclaves, obéissez en tout à vos maîtres d'ici-bas, non d'une obéissance tout extérieure qui cherche à plaire aux hommes, mais en simplicité de cœur, par crainte du Maître. Quel que soit votre travail, faites-le avec âme, comme pour le Seigneur et non pour des hommes, sachant que le Seigneur vous récompensera en vous faisant ses héritiers. C'est le Seigneur Christ que vous servez; qui se montre injuste sera certes payé de son injustice, sans qu'il soit fait acception de personne. Maîtres, accordez à vos esclaves le juste et l'équitable, sachant que vous aussi, vous avez un maître au ciel.}
\end{displayquote}

 Les épîtres de Pierre défendaient les mêmes thèses :

\begin{displayquote}[Première Épître de Pierre 2, 18-23]
\emph{Vous, les domestiques \emph{[(esclaves)]} soyez soumis à vos maîtres, avec un profond respect, non seulement aux bons et aux bienveillants, mais aussi aux difficiles. Car c'est une grâce que de supporter par égard pour Dieu des peines que l'on souffre injustement. Quelle gloire en effet à supporter les coups si vous avez commis une faute, mais si faisant le bien vous supportez la souffrance c'est une grâce auprès de Dieu. Or c'est à cela que vous avez été appelés, car le Christ aussi a souffert pour vous, vous laissant un modèle afin que vous suiviez ses traces, lui qui n'a pas commis de faute. Et il ne s'est pas trouvé de fourberie dans sa bouche ; lui qui insulté ne rendait pas l'insulte, souffrant ne menaçait pas, mais s'en remettait à celui qui juge avec justice}.
\end{displayquote}

 Mais puisque la liberté est meilleure que la servitude, Paul rappelait aux maîtres qu'affranchir leurs esclaves était une bonne œuvre. Les communautés chrétiennes pouvaient racheter ceux de leurs membres qui étaient esclaves, notamment quand la nature de leur emploi, ou la dureté et l'injustice de leurs maîtres, mettaient en danger leur vie, leur santé, leur vertu ou leur foi. Il était cependant conseillé aux esclaves chrétiens de ne pas revendiquer trop bruyamment leur rachat par la communauté : ses ressources n'y auraient pas suffi, et on se serait vite demandé si leur piété n'était pas trop intéressée.

 Paul et ses successeurs auraient-ils pu s'engager plus loin ? Y ont-ils jamais pensé ?  On pourra approfondir ce sujet avec Peter \fsc{GARNSEY}, \emph{Conceptions de l'esclavage, d'Aristote à Saint Augustin}, Les belles lettres, Paris, 2004.



\section{Le service des pauvres}

 Comme n'importe quelle minorité de l'Empire les chrétiens se devaient de prendre en charge leurs pauvres, malades, et infirmes. Chaque église locale prenait en charge les siens. En cas de catastrophe elle pouvait demander des secours aux autres églises. Elle enregistrait ses pauvres, titulaires d'un droit à secours, sur la liste des personnes qu'elle entretenait \emph{(matricule)}, à côté des membres du clergé. Dans la Rome du \siecle{3}, alors que la religion chrétienne était encore illicite, c'était déjà par milliers que se comptaient les pauvres, les malades et les vieillards inscrits sur les listes de l'évêque. Chacun d'eux, pauvres et clercs entretenus, valides ou non, faisait en quelque sorte partie de la \latin{familia} de l'évêque. 

 Si toutes les églises de l'Empire faisaient de même, aucune ne concurrençait l'Annone sur son propre terrain. Les bénéficiaires n'étaient ni des hommes honorables ni des citoyens, c'étaient les indigents, les vieillards et les infirmes, beaucoup de femmes et de petits enfants, des esclaves et des concubines abandonnés, des étrangers sans ressources, exilés ou bannis... Leur sexe et leur statut ne comptaient pas. Ils étaient recrutés sur critères « sociaux », ce qui pouvait attirer les plus pauvres vers l'église de leur cité : c'est peut-être pour cela que lorsque l'empereur Julien a entrepris pendant son court règne de refonder les cultes traditionnels il a cherché à engager le clergé païen dans cette voie.

 Pourquoi les chrétiens donnaient-ils aux pauvres ? Selon la Bible YHWH préférait à tous les sacrifices l'assistance faite aux pauvres. Selon les évangiles Dieu regardait ce qui était fait aux pauvres, aux malades, aux étrangers, aux prisonniers… comme si c'est à lui-même que c'était fait :

\begin{displayquote}[Matthieu, 25, 31-40]
\emph{Quand le Fils de l'homme viendra dans sa gloire, escorté de tous les anges, alors il prendra place sur son trône de gloire. Devant lui seront rassemblées toutes les nations, et il séparera les gens les uns des autres, tout comme le berger sépare les brebis des boucs. Il placera les brebis à sa droite, et les boucs à sa gauche. Alors le Roi dira à ceux de droite : » venez, les bénis de mon père, recevez en héritage le royaume qui vous a été préparé depuis la fondation du monde. Car j'ai eu faim et vous m'avez donné à manger, j'ai eu soif et vous m'avez donné à boire, j'étais un étranger et vous m'avez accueilli, nu et vous m'avez vêtu, malade et vous m'avez visité, prisonnier et vous êtes venus me voir. » Alors les justes lui répondront : « Seigneur, quand nous est-il arrivé de te voir affamé et de te nourrir, assoiffé et de te désaltérer, étranger et de t'accueillir, nu et de te vêtir, malade ou prisonnier et de venir te voir ? ». Et le roi leur fera cette réponse : « En vérité je vous le dis, dans la mesure où vous l'avez fait à l'un de ces plus petits de mes frères, c'est à moi que vous l'avez fait. »}
\end{displayquote}

%et :
%
\begin{displayquote}[Matthieu 25, 45-46]
\emph{Alors il leur répondra : « en vérité je vous le dis, dans la mesure où vous ne l'avez pas fait à l'un de ces plus petits, à moi non plus vous ne l'avez pas fait. » Et ils s'en iront, ceux-ci à une peine éternelle, et les justes à la vie éternelle.}
\end{displayquote}

 Le christianisme n'étant pas reconnu les églises n'étaient pas habilitées à recevoir des donations. Cela n'a pas empêché les dons d'avoir lieu ni un patrimoine de se constituer, sous couvert de prête-noms. Bien avant le quatrième siècle on constate l'existence d'une « zone grise », où les églises sont tolérées et où un patrimoine leur est tacitement reconnu par les autorités, même si une persécution pouvait tout remettre en question de temps en temps. 

 Les donateurs les plus généreux étaient les femmes de l'aristocratie romaine, veuves ou divorcées, libérées de toute tutelle, et non susceptibles d'être ordonnées à un \emph{honneur} (non susceptibles d'être contraintes à assumer les frais d'une \emph{évergésie}). La loi les disait \latin{sui juris} (autonomes juridiquement) si elles avaient donné le jour à trois enfants au moins, et si leur \latin{pater familias} était décédé. Elles préféraient se mettre sous la protection morale de l'évêque de leur choix plutôt que de se soumettre à un nouveau mari auquel il leur serait très difficile de s'opposer, sans que cela garantisse pour autant qu'il soit plus intéressé par leur personne que par leur dot%
% [5]
\footnote{La dot, contribution du père de l'épouse aux dépenses du ménage, appartenait à cette dernière, mais elle était gérée par le mari. Par contre une femme \latin{sui juris} pouvait gérer elle-même tout le reste de sa fortune personnelle. La coutume voulait que la totalité des biens d'une femme parvienne à ses seuls enfants, mais il fallait pour cela qu'elle teste activement en leur faveur. Cela lui laissait en théorie la possibilité de tester pour d'autres qu'eux.}%
. Leur refus de se remarier était le moyen de leur liberté. Il leur permettait d'échapper aux grossesses et aux accouchements. Il évitait à leurs enfants d'être mal traités par un beau-père et de rentrer en conflit pour leur héritage (maternel) avec ceux d'un nouveau lit. Et enfin il leur laissait la liberté de jouer un rôle social valorisé par leur communauté. Les palais de ces grandes dames pouvaient offrir aux femmes sans ressources un refuge contre un monde « machiste » : c'est ainsi que se sont agrégés les noyaux des premières communautés religieuses féminines. 

 Quelques hommes imitaient ces nombreuses donatrices dans la mesure où leurs responsabilités de chefs de famille (ce que les femmes n'étaient pas) leur en laissaient la possibilité. C'était d'abord le cas des hommes mariés \emph{sans enfant}.

 Les chrétiens aisés étaient donc invités à infléchir en ce sens leurs activités d'évergètes. Ils étaient les plus capables d'accueillir la communauté dans leurs vastes demeures, et les plus susceptibles de disposer des ressources nécessaires pour subvenir aux besoins des plus pauvres. Comme chaque église locale élisait ses clercs et son évêque, les chrétiens avaient tendance à les choisir parmi les plus riches d'entre eux, selon la tradition antique de confier les responsabilités civiques aux notables. 

 En contrepartie des moyens de vivre qu'il offrait à ses protégés, à ses « clients », l'évêque avait comme tout \emph{patron} romain le droit de leur demander des services. Sans charges de famille, les célibataires étaient disponibles le jour et la nuit, et ils n'avaient ni épou(x)se ni enfants à prendre en charge. Leur entretien pouvait donc être nettement plus économique que celui de personnes mariées de niveau social égal. Ils (elles) pouvaient se vouer à la pauvreté sans être irresponsables face à une descendance. Dans ce cas le coût de leur travail pouvait être compétitif avec celui des esclaves. Au premier rang des services que les laïcs voués à la continence pouvaient rendre, il y avait les tâches d'assistance. La première initiative institutionnelle de la toute première communauté chrétienne (à Jérusalem) avait été de créer un corps de clercs spécialisés dans les tâches d'assistance, les \emph{diacres} (dont rien ne nous dit qu'ils étaient célibataires), dont la tâche était de visiter les malades à domicile, et de gérer l'assistance, et d'abord l'assistance aux veuves (\emph{Actes des Apôtres} 6, 1-6). 

\subsection{Veuves}

Au nombre des pauvres figuraient d'abord les veuves. L'assistance aux veuves consistait en distribution d'argent, de vêtements, de nourriture,~etc. Dans un monde où les femmes ne pouvaient exercer une activité honnête que dans un cadre familial, une veuve pauvre qu'aucun parent ne recueillait était dans une détresse totale, surtout si elle avait des enfants à charge. 

 La situation était encore pire pour une ex-concubine. Le cas de la veuve qui prostitue sa fille parce que ni l'une ni l'autre n'ont aucun autre moyen de vivre est un poncif de la littérature, de l'Antiquité au \siecle{19}. La mère qui en était réduite là n'avait le plus souvent aucun choix : si elle vendait sa fille comme esclave, celle-ci n'échapperait probablement pas à la prostitution, sauf à être vendue comme esclave-concubine (mais il fallait trouver un protecteur), et elle perdrait sa liberté, tandis que sa mère n'aurait aucune ressource. En la prostituant la mère lui faisait perdre \emph{seulement} sa réputation. 

 Les veuves sans enfants à charge secourues par l'Église pouvaient s'employer au service des œuvres de l'évêque. Elles pouvaient s'occuper des enfants orphelins de père et de mère, des malades, des infirmes, vieillards et insensés,~etc. Elles étaient chargées des fonctions que seule une femme pouvait remplir sans faire jaser : soins aux femmes et aux petits enfants, visites aux femmes à domicile (instruction religieuse, soutien psychologique, assistance matérielle),~etc. Pour remplir ces missions on choisissait des matrones d'expérience, de bonne réputation (c'est-à-dire non infâmes) et d'un âge suffisant pour inspirer le respect (âge « canonique »). 

 Au fil des trois premiers siècles de notre ère le \emph{rôle social des veuves} soutenues par l'Église s'est effiloché. Sauf exception elles ont peu à peu cessé d'assumer des tâches de service auprès des femmes et des enfants. On a cessé d'attendre d'elles autre chose qu'une vie réglée, et si possible pieuse et édifiante. Leurs fonctions étaient désormais assumées par les vierges consacrées, et notamment tout ce qui a trait à l'assistance, qui dans l'aire catholique a presque exclusivement reposé sur des religieuses (et des religieux) jusqu'au \siecle{19} inclus. 

\subsection{Orphelins}

On a vu que les juifs considéraient que c'était une œuvre pieuse que de prendre en charge l'éducation des orphelins pauvres. D'ailleurs les non-chrétiens en faisaient autant. La loi romaine prévoyait que les mineurs orphelins de père (libres, ingénus ou affranchis) soient confiés à l'autorité d'un tuteur. La tutelle des orphelins était une charge et un devoir civiques, dont seules quelques professions étaient exemptées. Le tuteur était responsable de la gestion des biens du mineur et de son éducation, mais il n'était pas chargé de la financer lui-même. Les dépenses étaient supportées par l'héritage de l'enfant. Que se passait-il quand un orphelin ne possédait rien, quand ses parents ne lui avaient laissé aucun héritage, ce qui devait être fréquent ? 

 Quels étaient les orphelins assumés par les communautés dont les textes ecclésiaux les plus anciens parlent si souvent? Étaient-ce les enfants des martyrs ? Probablement pas : ce n'est qu'à quelques moments exceptionnels et brefs que le nombre de ceux-ci a été significativement élevé. Il s'agissait plus vraisemblablement des enfants des fidèles sans ressources décédés d'accident, de misère ou de maladie avant qu'eux-mêmes ne soient capables de gagner leur vie, ce qui était une situation banale à l'époque. Ceux dont les deux parents étaient morts n'étaient pas donnés en adoption. Les chrétiens s'y refusaient au même titre que les juifs : ces enfants appartenaient déjà à une famille connue même si celle-ci n'avait plus d'autres membres vivants qu'eux-mêmes. 

 Les orphelins secourus par l'Église pouvaient aussi être les enfants des veuves signalées plus haut : en effet les textes associent constamment \emph{les veuves et les orphelins}. Les uns et les autres étaient dans la misère et couraient un grave risque de tomber dans l'esclavage ou/et la prostitution (cela restera une constante : les plus pauvres sont toujours les femmes seules avec enfants). Selon la loi la tutelle d'un orphelin de père ne pouvait pas être exercée par sa mère : les femmes ne pouvaient exercer aucune autorité sur autrui, même sur leurs propres enfants, sinon par délégation (du père ou du tuteur).

 Sous quelle forme les communautés apportaient-elles aux enfants sans parents leur soutien matériel et moral ? On peut imaginer bien des formules, en fonction de l'âge des enfants et du contexte, depuis la nourrice, salariée jusqu'à ce que l'enfant puisse commencer à travailler (très tôt, comme tous les pauvres d'alors), ou la mise en apprentissage chez un professionnel aux frais de la communauté, jusqu'à la prise en charge collective de grands enfants dans la maison de l'évêque et sous son contrôle. 

 Certains de ces orphelins pouvaient être investis de manière spéciale : les plus vifs d'esprit, ou ceux dont les parents avaient laissé le meilleur souvenir. Certains garçons trouvaient une place dans le clergé : orientation naturelle s'ils n'avaient pas eu d'autre figure paternelle que des membres de ce même clergé.

\subsection{Malades}

Les malades et les infirmes de la communauté étaient visités à leur domicile. Ce service de proximité qui ne sépare pas le sujet de son milieu était l'un des premiers services assuré par les veuves auprès des femmes. Les diacres rendaient le même service aux hommes. Rien n'interdisait aux clercs ordonnés de pratiquer la médecine, du moment qu'ils ne versaient pas le sang. 

Vers 260 Denys d'Alexandrie vante le comportement des chrétiens face à une des épidémies de son temps : 

\begin{displayquote}
\emph{La peste fondit sur la ville, objet d'épouvante. La plupart de nos frères ne s'écoutaient pas eux-mêmes, visitant sans précaution les malades, leur donnant leurs soins dans le Christ. Chez les païens il en était tout autrement ; ceux qui commençaient à être malades, on les chassait, on fuyait ceux qui étaient les plus chers, on jetait sur la route des gens à demi-morts.}
\end{displayquote}

L'aide de la communauté se bornait-elle à ses propres membres ? 

 Une prise en charge totale était inévitable quand il s'agissait de personnes sans domicile, de voyageurs malades, de vieillards sans famille, d'esclaves abandonnés, d'orphelins sans fortune et sans parents, d'insensés trop agités ou trop régressés... Comme les moyens de l'hospitalité individuelle n'étaient pas illimités, il était dans l'ordre des choses que des formes collectives de prise en charge aient progressivement été mises en place, au moins dans les grands centres, sur le modèle alors bien connu des hôpitaux de garnison. 

\subsection{Énergumènes}

Toute l'Antiquité, chrétiens y compris, croyait aux démons et à leur capacité de nuisance sur le corps comme sur l'esprit. Les chrétiens incluaient souvent parmi les démons tous les dieux païens, auxquels ils croyaient eux aussi à leur façon. Ceux qui selon les conceptions du temps étaient possédés par un démon étaient le plus souvent des malades mentaux agités, des épileptiques atteints de grand mal, des autistes, des psychotiques délirants, des déments, de grands attardés mentaux et des hystériques en crise. Dès le \siecle{2} des énergumènes (littéralement : « possédés par un démon ») étaient hébergés à plein temps dans des locaux dépendant de l'Église (qui s'en chargeait auparavant ? Les familles les gardaient-elles ou les abandonnaient-elles à la rue ou bien dans un temple aux bons soins d'Esculape ou de tel ou tel autre dieu, c'est-à-dire à la charge de la charité des fidèles et de la bonne volonté des desservants du temple ?) 

 Dont le but de chasser leurs démons ces malades bénéficiaient d'exorcismes quotidiens dont étaient chargés leurs gardiens. La fonction d'exorciste ne s'est individualisée que lentement. Longtemps chaque chrétien a été jugé assez compétent pour chasser les démons en imposant les mains sur le front des possédés, sans avoir besoin d'une ordination spéciale. Selon les Constitutions Apostoliques (une compilation rédigée durant les années 380 de textes réglementaires ou liturgiques plus anciens : \fsc{FAIVRE}, \emph{Naissance d'une hiérarchie}, 1977, p. 118) : \emph{l'exorciste n'est pas ordonné, car être exorciste dépend de la bienveillance, de la bonne volonté et de la grâce de Dieu donnée par le Christ par la venue du Saint-Esprit. Celui qui a reçu le don de guérison est manifesté par une révélation de Dieu et la grâce qui est en lui est visible pour tous.}

 Comment cette fonction a-t-elle évolué par la suite ? Les \latin{statuta ecclesiae antiqua} sont une œuvre privée et non la transcription de décisions faisant autorité (d'un concile, du Pape, ou d'un évêque). Ils n'en sont pas moins un témoin de l'état des pratiques et des opinions du temps de leur rédaction : vers 476-485. Ils font aux exorcistes un devoir d'être patients et bons avec leurs protégés, et d'imposer les mains à chacun d'eux tous les jours. Ce geste s'accompagnait d'une prière. Ils avaient aussi le devoir de leur apporter leur nourriture quotidienne à l'heure prescrite, sans les faire attendre : « Les énergumènes qui séjournent dans la maison de Dieu doivent recevoir en temps voulu leur pitance quotidienne qui leur est apportée par les exorcistes. » (canon 64). Cette dernière remarque suggère que beaucoup des patients présentaient des traits de retard mental ou de régression massifs qu'on faisait travailler à la mesure de leurs capacités (exemple : \latin{pavimenta domorum dei energumeni everrant} : « les énergumènes balaieront le pavement des églises ») et que les exorcistes étaient plutôt des garde-malades vigoureux que des experts en pathologie mentale ou en théologie. 

 Comme les infirmiers psychiatriques de naguère les exorcistes n'ont jamais été placés bien haut dans la hiérarchie du clergé (clercs « mineurs »). Au fil du temps ils ont été relégués avec leurs patients de plus en plus loin de l'autel, au plus bas niveau, près de la porte. À partir du \siecle{6} ou du \siecle{7} les plus savants des théologiens cesseront de croire que le démon se cache derrière les états pathologiques ordinaires. Les actes d'exorcisme encore pratiqués, devenus tout à fait rares, seront réservés à des prêtres nommés pour cela. 

\subsection{Voyageurs, vagabonds, mendiants}

En raison de la valorisation évangélique de la pauvreté et de la souffrance, et en raison de leurs pratiques d'assistance, les chrétiens ne pouvaient qu'attirer tous les accidentés de l'existence : ceux et celles qui avaient tout perdu, qui ne pouvaient plus travailler, les vieillards sans enfants qui n'avaient plus la force de gagner leur pain, les esclaves abandonnés (« libérés ») par leurs maîtres parce que trop vieux, infirmes ou malades, les concubines abandonnées sans enfants pour les recueillir, les prostituées âgées, les exilés inconsolables, les bannis de leur cité,~etc. 

 Les premières Églises ont adopté la même règle que les synagogues face aux coreligionnaires (face aux « frères ») en déplacement, en voyage d'affaire, en mission pour leur communauté, et face aux vagabonds : les voyageurs valides étaient reçus comme des hôtes pendant trois jours, après quoi ils étaient invités à poursuivre leur chemin ou à gagner leur vie en se mettant au travail, conformément au mot de Saint-Paul : \emph{celui qui ne travaille pas n'a pas droit de manger}. 

 Quant à ceux qui étaient hors d'état de continuer la route ils étaient soignés durant le temps qu'il fallait pour qu'ils se remettent : cela pouvait durer des mois ou des années. Cela pouvait durer tout ce qui restait d'une vie.

\subsection{Captifs}

Les chrétiens de l'Antiquité rencontraient de multiples occasions de s'occuper de captifs :
\begin{enumerate}
% 1°)
\item ils subissaient sporadiquement des persécutions qui n'étaient pas toujours évitables. À chaque retour de flamme de ces persécutions les plus convaincus, les plus téméraires ou les moins chanceux de ces fidèles se retrouvaient en prison ou étaient condamnés aux mines ou aux bêtes du cirque ;
% 2°)
\item d'autres parmi les fidèles étaient emprisonnés pour des crimes ou des délits de droit commun, pour faillite,~etc. Quel qu'ait pu être le motif de leur captivité, la communauté se devait de les visiter, si nécessaire sur le site lointain des mines où ils et elles purgeaient leur peine. Elle les soutenait moralement et les encourageait à tenir bon dans leur foi et la pratique religieuse. Elle essayait d'adoucir leur sort, par exemple en soudoyant les gardiens pour qu'ils leur procurent de meilleures conditions de vie ;
% 3°)
\item enfin les chrétiens étaient exposés au même titre que tous leurs contemporains au risque d'être enlevés, avec la perspective d'être vendus au loin comme esclaves. Ce risque était plus élevé pendant les périodes de trouble, et pendant tous les voyages. Mais les enlèvements étaient à craindre même en ville, notamment les enlèvements d'enfants. Pour revoir libres ceux qui avaient été enlevés il fallait négocier et rassembler une rançon. Les églises locales faisaient ce qu'elles pouvaient en fonction de leurs moyens. 
\end{enumerate}

\subsection{Morts sans sépulture}

En accord avec les différentes cultures antiques la Bible faisait un devoir à quiconque était présent de donner une sépulture décente à toute personne décédée, sans aucune exception. L'importance des rites funéraires
 était essentielle, et ne pas ensevelir un mort était un \emph{sacrilège} (cf. Antigone), en dépit de l'impureté qui touchait celui qui s'en chargeait. Le soin d'autrui n'était terminé que quand son corps avait été enseveli selon les règles. Même les plus pauvres cotisaient pour se payer une place dans un tombeau collectif et pour que leur soit rendu le culte mortuaire approprié. 
 Pour que tous les pauvres aient droit à une inhumation décente les églises finançaient un service collectif d'inhumation. Les synagogues en faisaient semble-t-il autant. Encore une fois les premières communautés chrétiennes se sont coulées dans le moule juif. Les fossoyeurs avaient le devoir d'enterrer tous les morts inconnus ou indigents trouvés dans les terrains vagues, au bord des routes ou sur les rivages. Le signe de l'importance symbolique de cette fonction, c'est qu'on s'est longtemps demandé s'ils faisaient ou non partie des clercs mineurs. 

\subsection{Enfants trouvés}

Quand les Écritures prescrivaient de prendre soin des orphelins, est-ce que ce mot recouvrait les enfants abandonnés anonymement, les enfants trouvés ? L'étymologie n'interdit pas de le penser. Le grec \latin{orphanos} désignait en effet l'enfant privé de l'un ou l'autre de ses deux parents, notamment de son père. En ce sens les enfants exposés étaient des orphelins, même quand leurs parents étaient bien vivants. Pas plus que les juifs les chrétiens n'avaient le droit moral d'abandonner leurs enfants, même si ceux-ci étaient trop nombreux ou mal formés. Les écrivains chrétiens soutenaient comme un fait d'observation quotidienne que les païens laissaient mourir beaucoup de leurs enfants. Mais les fidèles pouvaient avoir les mêmes raisons que les autres d'exposer des nouveaux-nés, sauf à supposer qu'ils aient tous et toujours vécu dans une rigueur morale impeccable, ce qui est peu vraisemblable%
%[16] 
\footnote{Les Épîtres de Paul contiennent nombre de passage où il sermonne vertement ses ouailles pour des fautes morales caractérisées (par exemple la \emph{Première épître aux Corinthiens}, chapitres 5 et 6).} 
: enfant né d'un adultère, d'un inceste, d'un viol, d'une rencontre sexuelle sans lendemain, difficulté d'accepter un enfant mal formé,~etc. Rien ne permet non plus d'assurer que l'assistance de l'église mettait tous les chrétiens à l'abri de la misère, et qu'aucun n'était écrasé sous le poids matériel ou psychologique de ses enfants. On ne saura jamais quel était le pourcentage de leurs enfants que les chrétiens abandonnaient : peut-être était-il faible dans les communautés petites et ferventes, où tout le monde connaissait tout le monde, et surtout si l'assistance mutuelle y fonctionnait correctement ? 

 Les apologistes antiques du christianisme craignaient que la plupart des enfants abandonnés ne soient prostitués (ce qui implique d'ailleurs qu'ils ne croyaient pas qu'ils étaient promis à la mort%
% [17]
\footnote{John \fsc{BOSWELL}, \emph{Au bon cœur des inconnus}, 1993.}%
). Il était louable de les recueillir pour les protéger de ce risque. Mais quel était le statut des enfants ainsi pris en charge ? En effet nul ne pouvait prouver qu'ils étaient nés de conceptions régulières, et les règles de pureté de la \emph{Tora} les classaient parmi les \emph{mamzerim}, les impurs de naissance. La situation de ceux dont on avait bien connu les parents de leur vivant (« orphelins pauvres ») paraissait autrement digne d'intérêt, et le restera jusqu'au \siecle{19}. 

 On peut formuler plusieurs hypothèses :
\begin{enumerate}
% 1°) 
\item On a vu que les juifs étaient opposés à l'adoption telle que la pratiquaient les païens. L'adoption proprement dite, celle qui d'un étranger fait le fils et l'héritier d'une famille et de ses ancêtres (adoption « plénière ») disparaîtra dès que les chrétiens seront en mesure d'orienter les décisions impériales : cela suggère que les chrétiens campaient sur la même position que les juifs. Il paraît donc difficile de croire que les enfants abandonnés aient pu être régulièrement adoptés par eux.
% 2°)
\item Par contre, ils pouvaient donner à ces enfants le statut d'\latin{alumnii}, sans confondre l'entrée dans leur \latin{familia} avec l'entrée dans leur famille. Les païens et les juifs le faisaient bien ! Cela n'en faisait pas leurs enfants, sinon en un sens spirituel. Ils pouvaient tout de même les établir dans la vie. C'était une pratique socialement et religieusement valorisée.
% 3°)
\item Enfin rien ne leur interdisait non plus d'élever les enfants abandonnés dans le statut d'esclave, ce qui était une pratique traditionnelle. 
\end{enumerate} 

 En droit celui qui voulait prostituer un enfant trouvé était d'ailleurs obligé de lui donner le statut d'esclave, sans quoi son corps était protégé par la loi. On peut supposer que la plupart des fidèles ne s'autorisaient pas cette pratique, étant donné la véhémence avec laquelle les écrivains chrétiens de l'époque la dénoncent comme une abomination païenne. L'accueil des enfants trouvés pouvait être vu comme une bonne œuvre, dans le contexte de cette époque, du moment que l'accueillant s'interdisait d'exploiter le corps de l'enfant et qu'il s'efforçait de lui donner une bonne éducation. Celui qui prenait un enfant trouvé pour en faire son esclave était certes moins généreux que celui qui l'élevait en ingénu. Mais désormais ce nouveau-né sans droits n'était plus l'enfant de personne ni de nulle part, il faisait partie d'une \latin{familia} dont il recevait le nom et il trouvait une place non infamante, si petite fut-elle.

 Il n'y a pas de raison pour que ce type d'accueil ait laissé des traces dans les archives. Les enfants concernés n'entraient en effet dans les registres d'état civil que lorsqu'ils avaient reçu la liberté d'une personne libre. Sinon ils n'étaient pas enregistrés. La suite de l'histoire suggère pourtant que cette troisième solution a été fréquente, sinon la plus fréquente : dès Constantin les enfants trouvés ont en effet été mis à la charge des autorités civiles (cités, rois, seigneurs...), et ils ont aussi été comptés au nombre de leurs dépendants (« leurs hommes »), comme les futurs serfs et parmi eux. 
 
















%E1 Le tournant Constantinien 
%E2 Constantin et le droit des personnes 
%F1 entrée en scène des barbares 
%F2 Les sociétés du Bas-Empire et du Haut Moyen Âge 
%F3 l'esclavage chez les chrétiens de l'antiquité tardive et du haut moyen-âge 
%F4 clercs et religieux 
%F5 Le "mariage constantinien" 
%F6 Familles de chair 
%F7 Familles spirituelles 
%G1 Les familles de l'Ancien Régime entre autorités civiles et religieuses 
%G2 Les devoirs des pères de l'Ancien Régime 
%G3 Création d'une police des familles (XIVème-XVIIIème siècles)

\part[Les familles de l'ancien régime]{Les familles de l'ancien régime}

\chapter[Entre autorités civiles et religieuses]{Entre autorités civiles et religieuses}



\section{Le mariage constantinien}



Le « mariage constantinien » c'est le mariage des romains de l'antiquité tardive tel qu'il a été modifié par Constantin et ses successeurs pour l'accommoder aux conceptions chrétiennes. Il télescope plusieurs fonctions distinctes sur une seule personne : l'époux est à la fois le détenteur des droits juridiques de son épouse (son curateur), son amant (à qui on recommande la mesure), le géniteur de ses enfants, le détenteur des droits de ces enfants mineurs, et le responsable de leur éducation, c'est-à-dire leur père légal. Symétriquement, une épouse est la seule femme capable de donner à son époux des enfants légitimes, des héritiers, quel que soit le nombre de ses concubines. Pour être légitime chaque enfant doit être l'enfant biologique de ses parents légaux (leur enfant « naturel » au sens antique du terme). Et par définition seuls les enfants légitimes ont droit à une part d'héritage et à succéder à leurs parents.

Si le mariage constantinien, c'est le mariage tel que le conçoit l'église, à cela s'ajoute un élément essentiel, qui est que les autorités civiles traitent l'Eglise  comme l'une des sources du droit, ce qui donne à ses doctrines force de loi. 

Au terme d'une longue évolution commencée par Constantin,  à la fin du Moyen-Âge il n'y a plus de droit civil du mariage dans une Europe totalement christianisée où les seuls dissidents religieux reconnus sont les juifs. L'église a gagné suffisamment d'ascendant moral et de prestige intellectuel pour faire prévaloir une grande partie de sa doctrine. A partir de la réforme Grégorienne (XIème siècle) le mariage et la famille sont de son ressort exclusif. Les autorités civiles se soumettent, bon gré mal gré, à ses décisions : c'est elle qui reconnaît les mariages, qui ordonne les séparations, qui reconnaît les nullités de mariage, qui décide des interdits de mariage, et qui accorde des dispenses, etc...


Les religieux sont voués au célibat par choix : leurs voeux de pauvreté, d'obéissance à un supérieur et de chasteté sont incompatibles avec une vie de famille. Pour les pretres et les éveques, qui ne prononcent pas de voeux de chasteté, s'ils sont malgré tout interdits de mariage c'est pour les empecher d'avoir des fils légitimes, de leur transmettre leurs bénéfices (cure, éveché, etc...), et de créer des dynasties cléricales. 
 
En dehors de ces exceptions le célibat est licite, mais chez les jeunes gens sans enfants, en bonne santé et disposant de moyens matériels suffisants, il est suspect d'égoïsme, de libertinage ou de désirs « contraires à la nature » (homosexualité dont la mise en acte a toujours été condamnée moralement). Quels que soient les préférences individuelles la copulation n'est légitime que dans l'état de mariage monogame, qui est le seul moyen acceptable de répondre au \emph{"croissez et multipliez"} de la Genèse).  Comme la fin première du mariage est la procréation d'enfants légitimes les remariages sont déconseillés (quoique autorisés) quand cette fin n'est pas atteignable étant donné l'âge ou l'état de santé des conjoints. 
 
 

 \emph{C'est le mariage qui fonde la famille, et non la présence d'enfants}, même si l'Eglise met l'accueil des enfants au premier rang des « fins du mariage ». De son point de vue, le mariage crée  une parenté nouvelle, \emph{une seule chair}, entre les époux, \emph{qu'ils soient féconds ou non}. Cette parenté « par alliance » a des effets directs et immédiats sur les membres des parentèles des époux (frères, sœurs,~etc.) : elle étend le cercle des partenaires qui leur sont désormais définitivement interdits, même si l'un des époux décède.

 Selon la doctrine chrétienne, identique sur ce point au droit romain, ce sont les époux qui s'unissent l'un à l'autre : cela implique qu'ils soient capables de discernement (âge suffisant, santé mentale) et libres de leur personne : célibataires ou veufs, non esclaves, non engagés par contrat dans une entreprise qui empêcherait la vie commune, à l'abri de toute pression, libres de tout vœux religieux, sexuellement aptes au mariage.Contrairement au droit romain elle en est progressivement venue à ne reconnaître la réalité juridique d'un mariage que lorsqu'il a été consommé. 
 
 L'Église a toujours soutenu contre les parents que les jeunes gens ont le pouvoir de se marier validement sans leur accord, même si elle admettait qu'en leur désobéissant ces jeunes gens les déliaient de leur devoir de les établir dans la vie. 
 
 Contrairement au droit romain l'Eglise pose depuis sa fondation que le "oui" que se donnent les époux est irrévocable et elle enseigne que le mariage est indissoluble. Le divorce est interdit\footnote{...pour les chrétiens, mais pas pour les juifs soumis aux tribunaux rabbiniques.} quel que soit le motif. Seules sont possibles les actions en nullité de mariage, ou les actions en séparation (\emph{divortium}) avec interdiction de se remarier du vivant du conjoint. C'est sans doute l'un des points où les frictions entre clercs et laics ont été les plus vives (cf. Henri VIII, etc...).



 Les deux époux se doivent réciproquement fidélité. Même si les infidélités du mari ne sont pas sanctionnées par la loi, au contraire de celles de l'épouse, tout est fait pour qu'il n'ait aucun intérêt à entretenir des maîtresses. Cela ne lui interdit pas d'avoir des rapports avec des prostitué(e)s, rapports qui par nature ne s'inscrivent pas dans la durée et sont moins menaçants pour l'épouse. 
  
  Chacun des deux époux reconnaît à l'autre un droit sur son propre corps et a l'obligation de satisfaire ses désirs sexuels autant qu'il est en son pouvoir , ce qui veut dire que l'épouse doit accepter les étreintes de son mari, quoi qu'elle puisse penser des risques de grossesse et de santé à quoi cela l'expose, et quels que soient ses propres désirs. Ceci dit la modération est prêchée aux maris, qui se voient prescrire la continence de nombreux jours par an. Le \emph{devoir conjugal} n'est exempt de faute morale que si aucun obstacle n'est mis à la fécondation (éjaculation ailleurs que dans le vagin, pessaire, douche intime, ~etc.). 

Il n'est pas permis de se débarrasser des enfants non voulus par l'avortement ou par l'infanticide. Sauf indigence extrême il n'est pas non plus permis de s'en débarrasser par l'exposition ni la vente. Une femme n'a donc pas à craindre qu'on l'oblige contre son gré à avorter ou à abandonner son nouveau-né. Mais sa fécondité ne lui appartient pas plus qu'elle n'appartient à son mari, et pas plus que son mari elle n'a droit de vie ou de mort sur l'enfant qu'elle porte.


 Il n'est pas possible de répudier une épouse présumée stérile (en cas de stérilité dans un couple, c'est celle de la femme qui est toujours suspectée en premier) au motif de sa stérilité supposée. 
Les couples stériles, dont le nombre n'est pas du tout négligeable jusqu'à l'avènement de la médecine moderne, 20 % environ, et ceux dont aucun enfant n'ont atteint vivant l'âge adulte, sont invités à consacrer aux bonnes œuvres, aux pauvres et à l'Église les ressources qu'ils auraient transmises à leurs héritiers s'ils en avaient eus.

En cas de décès du père, c'est la mère qui, chez les Romains et à partir de 390, exerce la tutelle de ses enfants mineurs (et d'eux seuls) si elle a cinquante ans et plus, et du moins tant qu'elle ne se remarie pas, ce qui est le cas général des veuves dotées d'enfants\footnote{les femmes chargées d'enfants trouvent un compagnon beaucoup moins facilement que les autres. C'est encore vrai aujourd'hui.}. À partir de 390 une femme n'est donc plus considérée comme incapable par nature de représenter juridiquement une autre personne qu'elle-même. 

Les épouses savent qu'il est, sinon impossible, du moins difficile de les chasser de leur maison ou de leur imposer de cohabiter avec une concubine\footnote{... même si pour les hommes dont la puissance excède de beaucoup celle du commun des mortels, aristocrates, rois, la question peut se présenter différemment, et si les amours ancillaires sont de tous les temps.}% 
. Elles sont à peu près assurées que les infidélités de leur époux n'entraîneront de conséquences graves ni pour elles, ni pour leurs enfants, ni pour le futur héritage de ceux-ci. Tout au plus des « aliments » devront-ils être versés aux enfants nés des maîtresses de leurs maris, mais jusqu'au XXème siècle cela ne portera que sur d'assez petites sommes et seulement jusqu'à ce qu'ils soient mis au travail : 8-10 ans. Il n'est pas question de financer leur établissement dans la vie. 

 Tous les enfants nés hors mariage sont pénalisés. Même si la prise en charge d'\latin{alumnii} et leur installation dans l'existence est une bonne œuvre, il n'est guère possible pour un homme de se faire des héritiers sans se marier. La \emph{légitimation par mariage subséquent} est la seule exception de plein droit\footnote{... jusqu'au \siecle{20}. Les enfants irréguliers légitimés par les empereurs, les rois ou les papes, ne l'étaient pas de plein droit mais à la faveur d'une grâce, qui pouvait être refusée sans justification, et qui n'allait pas toujours sans contreparties coûteuses.} 
à la pénalisation des enfants nés hors mariage, et ses conditions sont strictes. Chacun, quelque puissant qu'il soit, doit savoir que s'il a l'imprudence de faire un enfant hors mariage ou dans un mariage contesté par son curé, par son évêque, par son seigneur, par le roi ou par sa propre parentèle, il ne pourra pas le faire reconnaître comme un de ses héritiers sans combat ou sans procès. Cet enfant ne pourra probablement pas lui succéder. L'exhérédation totale ou partielle des enfants illégitimes est restée jusqu'à la fin du \siecle{20} le premier frein apporté au désir des hommes de se procurer une descendance autrement qu'avec une femme épousée en bonne et due forme.

 Contrairement au droit romain de l'antiquité le concubinage n'est pas traité comme une union de second ordre, propre aux gens qui n'ont pas d'héritage à transmettre, mais comme un état de fornication durable. Il n'est possible à un homme de légitimer les enfants qu'il a obtenus d'une concubine qu'en l'épousant (\emph{mariage subséquent}), et à la condition de ne pas etre déjà marié. Et la perspective de se retrouver avec un enfant à charge, seule, sans aucun espoir d'une légitimation (ni même d'une aide significative venant du père de l'enfant lorsqu'il était déjà marié puisque aucune donation au-delà des frais d'éducation n'était autorisée depuis Constantin) a été un obstacle majeur à l'exercice d'une sexualité féminine en dehors du mariage ou avant le mariage. 


La première des méthode de limitation des naissances autorisées est la continence. Elle est fiable et donne des résultats immédiats. A plus long terme on peut y ajouter  le retard de l'âge au mariage, notamment celui des filles : si au lieu de se marier à 16 ans comme beaucoup de celles qui sont bien dotées elles attendent 24 ans en travaillant et en économisant sou à sou pour se constituer la petite dot qui leur permettra de se marier elles "gagnent" huit années sans grossesses\footnote{...soit 4 enfants environ en moyenne, sans contraception et avec allaitement.}. 

Pour diminuer le nombre de leurs descendants les parents peuvent aussi utiliser l'entrée en religion de certains de leurs enfants. Cela leur permet de concentrer leurs ressources sur l'établissement des autres. Meme s'il faut ordinairement une "dot" pour entrer dans les couvents et monastères son montant dépend de la notoriété de l'établissement, et il suffit de ne pas viser trop haut pour faire des économies sérieuses et mieux doter les filles qu'il convient de marier. 
Les voeux des religieux et religieuses sont reconnus par les autorités civiles. Le voeu de pauvreté les enlève définitivement tout droit à tout héritage. Le voeu d'obéissance à leur supérieur les délie de leur devoir d'obéissance à leurs parents. Il est possible d'etre relevé de ces voeux, mais cela demande une procédure lourde et aléatoire. Une fois un enfant placé au couvent il lui sera difficile de brouiller les stratégies familiales.  
   



 



Les laïcs n'ont jamais totalement épousé les points de vue des clercs et jusqu'à la Réforme Grégorienne (\siecle{11}), l'Église n'avait pas le monopole du droit familial. Les écarts entre le droit religieux (droit \emph{canon}) et les lois civiles n'ont jamais été nuls, pour ne pas parler de \emph{l'à peu près} avec lequel ces lois étaient respectées. Ce que j'appelle le mariage constantinien est donc un modèle qui n'a jamais été pleinement réalisé, et surtout pas sous Constantin. Pourtant il s'est peu à peu incarné dans les pratiques et les représentations, et il n'a peut-être jamais été aussi bien respecté que durant les derniers siècles de notre ancien régime.

Malgré sa rigueur, et même si elle n'a souvent été suivie que de manière plutôt lâche la morale familiale et sexuelle enseignée par l'Église est devenue la norme au fil des siècles . Si du \crmieme{11} au \siecle{16} des mouvements de contestation religieuse se succèdent, qui culmineront avec la réforme protestante, rares sont ceux qui à cette période mettent vraiment en question la morale familiale et sexuelle enseignée par l'Église. Au contraire les opposants s'appuient sur elle pour critiquer les écarts des clercs avec leurs propres principes. Cette morale a été formulée dès les premiers temps de l'Église, mais les règles de droit qui en découlent ont mis au moins un millénaire à s'imposer comme le droit commun.  Certaines régions s'y sont conformées avec beaucoup de rigueur tandis que d'autres ont été beaucoup plus tolérantes avec les irrégularités.  Le christianisme a certes contribué à façonner les sociétés d'ancien régime, mais il en a été lui-même fortement influencé. Il lui a été demandé de bénir, et même de sacraliser, leurs mécanismes et leurs logiques, et de les conforter dans leur fonctionnement, et c'est ce qu'il a souvent fait. 


 
 
 \section{traitement des naissances illégitimes}

 Les contraintes et limites imposées à la reproduction par le roi et par l'Église n'ont jamais été acceptées totalement ni par tous. C'était notamment le cas dans la noblesse. Depuis l'Antiquité tardive et durant tout le Moyen Âge elle était tenue, par elle-même et par les autres ordres de la société, pour une \emph{race} supérieure qui transmettait ses vertus par son sang. Cette antique croyance n'accordait pas d'importance au statut juridique ou religieux de l'union des parents. Elle coexistait sereinement dans les têtes avec le modèle canonique judéo-chrétien. 

 Les membres les plus puissants de la noblesse, et d'abord les rois eux-mêmes, n'ont jamais cessé de pratiquer une polygamie de fait qui leur donnait de nombreux enfants, de second rang certes, mais parfois bien utiles à défaut ou en complément d'enfants légitimes et valeureux. Jusqu'au \siecle{11} les différences faites dans les familles puissantes entre enfants légitimes et bâtards nés du chef de famille étaient faibles (capacité d'hériter, de succéder,~etc.). En effet le sang du père ennoblissait. Cette conception était un héritage des mœurs d'inspiration germanique du haut Moyen Âge. Le nombre des bâtards nobles semble même avoir crû au \siecle{15}. Par comparaison les bourgeois reconnaissaient beaucoup moins d'enfants illégitimes. De 1400 à 1649 les rois de France ont reconnu 24~\% d'enfants naturels tandis que les grands officiers, employés roturiers de la maison du roi, n'en avouaient que 10,3~\%. 

 Alors que le mariage des grands seigneurs ne répondait habituellement qu'à des critères politiques, les enfants qu'ils avaient conçus avec leurs maîtresses, \emph{enfants de l'amour}, étaient fréquemment \emph{réussis}. S'ils étaient légitimés, ces enfants pouvaient leur permettre des alliances profitables. Or les rois d'Europe, héritiers en cela aussi de l'empereur de Rome, pouvaient légitimer les « bâtards » par \emph{lettres royaux}. Les bâtards des familles aristocratiques ont donc souvent été légitimés par le roi ou par mariage subséquent. Même s'ils ne l'étaient pas, cela n'a pas fait problème pendant longtemps. Souvent ils n'ont été légitimés qu'après la mort de leur père, \latin{ad honores}, c'est-à-dire pour accéder aux honneurs qu'ils détenaient, c'est-à-dire pour leur succéder dans les emplois d'intérêt public, les charges qu'ils exerçaient. Par contre aucune reine, princesse du sang ou femme d'officier n'a pu légitimer d'enfant naturel autrement que par un mariage subséquent : c'est l'homme qui ennoblissait, c'est lui qui légitimait. 

 Les cas d'illégitimité susceptibles de bénéficier d'une légitimation par mariage subséquent avaient été étendus par les rois au-delà des critères de Constantin, de façon à inclure les enfants nés d'une relation passagère, et ceux conçus dans le cadre d'un enlèvement suivi d'un viol (enfants dont le géniteur n'était pas forcément le futur mari). Par ce biais la fiction retrouvait une place dans la filiation : l'adoption par l'époux de la mère redevenait possible. 

 Et pourtant les déclarations officielles de l'Église stigmatisaient tous les enfants illégitimes. Le Concile de Bourges (1031) confirmait les jugements des conciles antérieurs (« semence maudite »). Et l'Église continuait d'écarter de la prêtrise les fils de prêtres, sauf dispense (à vrai dire facilement accordée). Et elle était en cela d'accord avec le reste de la société. À partir du \siecle{11}, le mot « bâtard » devient un terme de mépris, une injure. Dès le \siecle{12}, avec la renaissance du droit romain, le bâtard n'appartient plus à aucune famille même dans les pays de droit coutumier\footnote{En gros les pays situés au nord de la Loire, opposés aux pays de droit (romain) écrit, situés au sud de la Loire.} 
: pas même celle de sa mère. 

 
 L'Hôpital du Saint Esprit de Paris était initialement un hospice destiné à toutes les personnes démunies. À la fin du Moyen Âge il était devenu l'orphelinat de Paris. Vers le milieu du \siecle{15} le roi lui a demandé de prendre en charge les enfants abandonnés de Paris. Tout roi qu'il fût ses demandes réitérées ont été récusées par les maîtres de l'un des hôpitaux de sa propre ville : \emph{"...en faisant prévaloir les statuts de fondation et la bonne réputation dont jouissent les enfants qu'il entretient et éduque déjà."} Si tous les orphelins d'origine inconnue lui étaient conduits, les gens de métier qui viennent chercher un apprenti, ou les jeunes compagnons qui y prennent femme ne seraient plus assurés de la légitimité, et partant de la moralité de l'adolescent (\fsc{SAUNIER}, \emph{Le « pauvre malade » dans le cadre hospitalier médiéval, France du Nord}, vers 1300-1500, 1993, p. 53).
 Réellement convaincu par ces arguments, ou bien de guerre lasse, le roi a confirmé les maîtres de l'Hôpital du Saint-Esprit dans l'idée que leurs statuts et eux-mêmes se faisaient de leur mission : en 1445 il a donc accepté qu'on n'y admette que les \emph{orphelins et orphelines nés en loyal mariage} et à qui on ne peut reprocher \emph{la tache de bâtardise}, car, selon un autre argument fourni par les maîtres de l'hôpital, \emph{"...ſy on y admettoit des baſtards, il ſeroit à craindre que la division ne ſe miſt bientôt dans cette maiſon par les reproches continuels que les enfants légitimes feroient aux baſtards".}
(\emph{Lettres patentes de Charles~VII du 7 août 1445, A.A.P. Saint-Esprit, L, II,} p. 32 ; cité par \fsc{SAUNIER}, id. p. 212).
 
 
 À cette époque les maîtres du Saint-Esprit faisaient remettre tous les enfants exposés qu'on leur présentait aux paroisses sur le territoire desquelles on les avait trouvés, alors qu'ils acceptaient de prendre en charge les adolescents légitimes (orphelins pauvres) qui sortaient convalescents de l'Hôtel-Dieu. Même les léproseries excluaient les bâtards \emph{"...parce que les gardes des maladreries diſaient que les bâtards n'avaient pas de lignage, ni n'étaient à hériter de nul droit par quoy ils ne ſe pouvaient aider de leur maiſon, pas plus qu'un étranger qui ſerait venu d'Eſpagne"} \fsc{SAUNIER}, id. p. 213.
Mais ces mêmes léproseries admettaient sans réserves les lépreux sans ressources s'ils étaient de naissance légitime : l'indigence leur faisait encore moins peur que l'illégitimité.



Être un « bâtard » était une tare, et semble avoir été de plus en plus pénalisant du Moyen Âge au \siecle{18} : est-ce pour des raisons religieuses ? ou plutôt parce que la société reposait sur l'alliance des familles, alliance que protégeait la mise hors jeu des enfants nés en dehors de ce cadre ? Dans une société où l'on n'était fils ou fille de quelqu'un que si l'on était né de deux parents légitimement mariés, un jeune de naissance illégitime ou né de parents inconnus (traité lui aussi de bâtard) portait une tare indélébile. Il était considéré comme fils de personne, hors parenté, même s'il vivait au foyer de l'un de ses deux parents (la mère en général). Un enfant non légitime ne pouvait ni succéder à un membre de sa parentèle dans un office, ni en hériter, sauf si aucun autre héritier légitime n'y trouvait ombrage. Personne ne se portait caution pour lui. Sans père il ne pouvait pas apporter sa contrepartie dans un système d'alliance. Il était condamné à une position marginale, du moins par rapport à celle de ses éventuels demi-frères et sœurs. En contrepartie de ces exclusions il n'était pas non plus contraint de se porter caution pour un parent. Il n'avait aucune autorisation parentale à demander pour convoler : ses géniteurs comme ses tuteurs ne pouvaient pas le lui interdire.

 

 Jusqu'à la fin de l'ancien régime les « bâtards » étaient exclus de nombreux métiers. En règle générale les corporations les refusaient, de la même façon et pour les mêmes raisons que la prêtrise leur était interdite. Maître Jacques \fsc{Ducros}, avocat au Parlement de Bordeaux, et premier Consul d'Agen en 1659, écrit dans ses \href{http://www.babordnum.fr/files/original/859d36685f2d7b2f871c648ea08bd103.pdf}{\emph{Réflexions singulières sur l'ancienne coutume d'Agen}}  :
%
\begin{displayquote}

{[...] \emph{les batards n'ont pas le bonheur de paſſer pour des domeſtiques%
% [11] 
\footnote{Ici « domestiques » signifie « appartenant à une maison », pas forcément comme salarié au service du maître. Cela inclut aussi tous les enfants et parents vivant dans la maison.} 
ny d'auoir rien en propre dans les maiſons. Ils ſont les productions du vice \& les enfans d'iniquité. Les peres les forment dans les tenebres \emph{[et]} les meres en cachent la conception. A meſme qu'ils sont nez , ces infames producteurs les deſauoüent. Les enfans legitimes cherchent le iour \& la lumiere, les illegitimes la nuit \& l'obſcurité. A proprement parler ce ſont des excremens, deſquels à meſme que la nature les chaſſe \& les pouſſe dehors, on couure d'ordure \& de ſaleté : ils n'ont ny nom ny race ny famille , c'eſt pourquoy ils ne peuuent eſtre admis au nombre des proches de ſang de conſanguinité ny d'allience}}%
%[12]
\footnote{\fsc{CAPUL}, Thèse, tome II, p. 111.%
\label{notecapul111}}%
.

\end{displayquote}


 \fsc{CAPUL} rapporte que lors des États généraux de 1614, le Tiers-état d'Agenais demande au roi : \enquote{\emph{que toutes lettres de légitimation ſeront deſnyees a tous enfens nez d'inceſte, d'adultère ou filz de prebſtres, et qu'on n'y aura aucun eſgard, ſoit pour ſucceſion, dignites, offices, bennefices (eccléſiaſtiques) et tous autres droitz}}%
% [13]
\footnote{%\fsc{CAPUL}, Thèse, tome II, p. 111.}%
Voir note \ref{notecapul111}.}%
. Les places désirables étaient trop peu nombreuses pour se montrer généreux. 
Les bourgeois prospères qui représentaient leurs concitoyens de l'Agenais, et qui exprimaient probablement l'opinion publique de leur époque, ne s'identifiaient en aucune façon aux enfants nés des unions sexuelles illicites, pourtant innocents des actes de leurs pères et mères. Ils ne toléraient pas que les « bâtards » soient confondus avec les enfants légitimes, et surtout pas avec les orphelins. Ces députés tenaient fermement à ce qu'aucun passe-droit ne puisse désavantager leurs propres fils dans la course aux honneurs, et leurs filles dans la chasse aux maris. C'était la défense la plus intransigeante de la morale conjugale qui servait leurs intérêts, puisqu'elle leur permettait d'écarter une partie des concurrents nobles ou bourgeois de leurs propres enfants.  

Ceci dit leur démarche auprès du roi  conforte l'idée que les interdits qui pesaient sur les « fruits du péché » pouvaient assez aisément être tournés avec de l'argent et/ou de l'entregent, mais cela ne concernait que les rares enfants illégitimes qui étaient investis par des parents puissants ou fortunés. Ainsi Erasme, (1469-1536), « prince des humanistes », âme de la « république des lettres » de son temps, était-il fils de prêtre. Fils d'un père cultivé il reçut une instruction soignée dans les écoles monastiques de son temps et entra en 1688 chez les chanoines de Saint-Augustin, où il fut ordonné prêtre en 1492. S'il ne fit pas une brillante carrière dans les allées du pouvoir temporel, comme évêque ou cardinal, c'est parce qu'il refusa les offres qu'on lui en fit au profit de la recherche intellectuelle et théologique, où il réussit il est vrai de manière exceptionnelle. Sa bâtardise et le statut ecclésiastique de son père ne semblent avoir fait problème à personne.

 
 
 

\section{Conflits de juridictions}

 Le droit romain n'a jamais totalement disparu dans les pays de droit écrit, au Sud de la Loire ou en Italie, mais à partir de sa redécouverte au \siecle{12} il a connu une nouvelle faveur en tant que modèle et outil de réflexion. La Renaissance a vu le triomphe du droit tel que les empereurs chrétiens l'ont mis en forme%
% [4]
\footnote{Justinien~I\ier{} (483 - 565) ou Justinien le Grand, empereur de Byzance de 527 à 565, essaya de restaurer l'unité de l'empire romain. Il a ordonné et dirigé une compilation du droit romain, le \latin{Corpus iuris civilis}, qui est l'une des bases du droit civil de divers pays européens.}% 
. Si bien qu'au bout de plus d'un millénaire, c'étaient encore les choix de Constantin et de ses successeurs immédiats qui modelaient en profondeur les mœurs familiales européennes : celles-ci n'ont jamais été aussi conformes à ses décrets qu'aux \crmieme{16} et \siecle{17}. À partir de la Renaissance et jusqu'au \siecle{20} les femmes \emph{mariées} ont été pratiquement réduites à un statut de mineures. Par rapport au Moyen Âge le retour en faveur du droit romain a appesanti l'autorité des pères sur les enfants, et contribué à enlever aux femmes, et surtout aux épouses, une part significative des libertés économiques et juridiques que le Moyen Âge leur avait reconnues.

 Du \crmieme{13} au \siecle{18} les autorités civiles reprennent peu à peu une grande partie du terrain qu'elles avaient concédé aux autorités religieuses au fil du haut Moyen Âge. Le \emph{Concordat de Bologne} (1516) accorde au roi de France le droit de nommer les titulaires des principaux bénéfices (évêques et abbés et abbesses), en dépit la règle qui depuis l'Antiquité voulait qu'ils soient désignés (élus) par leur communauté. 

 À partir du \siecle{14} un petit nombre de curés ont commencé de tenir des \emph{registres de catholicité} où ils enregistraient les baptêmes (autant dire les naissances dans un monde où tous sont baptisés) et parfois aussi les décès et les mariages. L'intérêt de ces registres était de faire foi dans les procès éventuels, même si manquaient les témoins capables de répondre à des questions concernant par exemple l'âge des personnes, ou leurs liens de parenté,~etc. En raison de cet intérêt quelques évêques ont ordonné à tous leurs curés d'en faire autant. \emph{L'Ordonnance de Villers-Cotterêts} (1539) a généralisé à tous les curés du royaume de France l'obligation d'enregistrer par écrit tous les baptêmes. \emph{L'Ordonnance de Blois} (1579) la complète en ordonnant que soient également notés sur ces registres tous les mariages et tous les décès, ce qui permettait de lutter contre les bigames. À cela s'ajoute l'obligation légale d'une publication des bans préalable au mariage, préconisée depuis longtemps par les conciles, mais appliquée de manière irrégulière, afin que ceux qui connaîtraient un empêchement au mariage projeté puissent le déclarer en temps utile. Par ces diverses initiatives et par d'autres les autorités civiles ont repris pied dans le domaine matrimonial. Le contrôle des unions importait en effet au moins autant aux rois et aux parents qu'à l'Église, et les mariages avaient un effet déterminant sur le bon fonctionnement de la société civile, sur la paix des familles et sur l'organisation économique. Les juges royaux ont cherché et trouvé, ou reçu du roi, des moyens de contester certaines des décisions des juges ecclésiastiques, si bien qu'on a pu observer un retour progressif du contentieux des mariages devant les tribunaux civils. 

 Le conflit le plus rude entre les autorités civiles et les autorités religieuses a porté sur la place à donner à l'autorité des parents sur les unions. Selon l'Église catholique les conjoints s'unissaient irrévocablement l'un à l'autre par leur « oui », devant le prêtre qui n'était qu'un témoin représentant l'Église, un témoin privilégié à partir du moment où il tenait les registres d'état civil (le curé de la paroisse de l'un ou l'autre des époux sauf dispense). La position traditionnelle de l'Église était que l'autorisation parentale n'était pas nécessaire pour la validité du mariage, même si elle déconseillait aux jeunes gens de s'en passer et si elle ne contestait pas aux parents le droit de déshériter les contrevenants. Les autorités civiles et les familles pensaient au contraire qu'un mariage, alliance entre deux familles et contrat civil, ne pouvait pas être valide, quel que soit l'âge des conjoints, sans l'accord formel de leurs parents. Du point de vue de ces derniers (et des clercs eux-mêmes lorsqu'ils ne parlaient pas en tant que représentants de l'Église, mais en tant que membres d'une famille particulière) l'absence de cet accord était une preuve du manque de bon sens, de l'immaturité des deux jeunes concernés, ou de la perversité de l'un d'eux (cf. la réaction du chanoine Fulbert, oncle d'Héloïse, face au mariage secret, et néanmoins canoniquement valide, de sa nièce avec Abélard). 

 Malgré la pression du roi de France, et le besoin qu'ils avaient de son appui et de ses armées, les évêques rassemblés en concile à Trente ont refusé de modifier la doctrine traditionnelle. Le roi a donc promulgué \emph{l'édit de 1556} qui ne contestait pas la validité religieuse des mariages célébrés sans l'accord des parents \emph{(mariages clandestins)} mais qui les déclarait civilement illégaux. Il confirmait le droit traditionnellement accordé aux parents de déshériter les enfants qui se rendaient coupables de tels mariages. Il décidait surtout que l'instigateur ou l'instigatrice d'un tel mariage (c'est-à-dire celui qui avait à y gagner, en principe le plus pauvre) pouvait être condamné à la peine de mort pour \emph{rapt}, ce qui réglait radicalement la question de l'indissolubilité du mariage. Cet édit a été en vigueur jusqu'à la Révolution, et semble avoir réglé le problème à la satisfaction des pères et des mères de familles. 
 
 
 La position des protestants était très proche de celle du roi de France. Pour eux le mariage n'était pas un sacrement mais seulement un contrat entre deux personnes, par nature révocable, et du ressort des seules autorités civiles. Selon Luther (Traubüchlein, 1529) : « \emph{il faut laisser à chaque ville et à chaque pays ses us et coutumes tels qu'ils sont pratiqués \emph{[... tous ces usages]} c'est aux princes et aux magistrats qu'il appartient de les établir et de les régler} ». Le roi et les pères et mères étaient d'accord sur l'idée que le choix d'un conjoint était trop important pour être laissé à la discrétion des futurs époux. Mais le fait de dénier au mariage le poids d'un sacrement et de n'y voir qu'un contrat ne lui enlevait pas une certaine forme d'indissolubilité. Même Henri~VIII n'avait pas rompu le lien entre l'Église d'Angleterre et Rome pour divorcer, mais pour faire reconnaître par les évêques de son royaume la nullité de son premier mariage. 
 
 À partir du moment où le mariage était invalide lorsque les parents des futurs conjoints ne lui apportaient pas leur approbation, comme pour tout autre contrat, il était du devoir de ceux-ci de se soumettre à la volonté de ceux-là dans ce domaine comme dans tous les autres. Une fois leurs parents d'accords, au terme de négociations plus ou moins âpres, et après que le père de la mariée ait remis celle-ci en mains propres à son futur gendre, les époux se devaient de respecter la volonté de leurs auteurs et de rester ensemble en dépit des difficultés éventuelles. C'est pour cela que tout en reconnaissant aux époux le droit de divorcer, les protestants leur imposaient tant de conditions (en Angleterre il y fallait entre autres un acte du parlement) que leurs divorces étaient en réalité difficiles à obtenir et coûteux (800 livres au \siecle{19} en Angleterre) et donc rares : en Angleterre 184 divorces entre 1715 et 1852, pour 9 millions d'habitants environ ; au Massachusetts, état américain bien plus libéral, 143 divorces entre 1692 et 1786 pour \nombre{300000} habitants environ (un et demi par an !). 
 

\chapter{Les devoirs des parents}


 \section{Des pères et des rois}


À partir du \siecle{4} dans l'empire romain, ce n'est plus d'abord et avant tout par la relation de pouvoir qu'il exerce sur les membres de sa maison que le père est juridiquement défini. En effet, il est soumis au devoir de \emph{piété} 
\footnote{La piété était l'affection réciproque et le respect mutuel entre les divers membres de la famille nucléaire, y compris le devoir d'assistance.} 
à l'égard de ses enfants au même titre qu'ils le sont à son égard, et autant qu'eux. L'accent se déplace sur sa responsabilité envers eux. 

La paternité est exaltée en liaison avec la paternité divine, mais la valorisation des parentalités spirituelles à travers le culte de Saint Joseph (père adoptif de Jésus selon les évangiles) affirme la prééminence de la relation éducative sur la reproduction biologique. Au même moment la maternité est très fortement idéalisée à travers le culte de la Vierge Marie. Au total c'est la famille nucléaire, le couple et ses enfants légitimes, qui est sacralisée. Cela s'exprime entre autres dans le culte de la « sainte famille » qui prend son essor au \siecle{17} : c'est en 1665 qu'est fondée la \emph{confrérie de la Sainte Famille} (la fête religieuse de la sainte famille n'a été instaurée qu'en 1893). 


 A la fin du Moyen Âge le fonctionnement des familles semble avoir eu tendance à se rigidifier dans un patriarcat de plus en plus rigoureux, en même temps que les états montaient en puissance et que les doctrines absolutistes gagnaient de l'audience. Vécu comme un père par ses sujets, le roi s'identifiait à son tour à tous les pères de famille. Eux et lui étaient autant de représentants de Dieu « le Père » et ils se confortaient les uns les autres. 

Qu'ils soient protestants ou catholiques les philosophes, les théoriciens du droit et les chantres de l'absolutisme (Jean Bodin, Omer Talon, Bossuet, Thomas Hobbes,~etc.) soutenaient la nécessité d'un pouvoir fort, incarné par un souverain absolu, c'est-à-dire sans contre-pouvoirs significatifs. On peut supposer que cela découlait pour une part de l'expérience des guerres, civiles ou entre états, dans lesquelles les européens se sont laissés entraîner par les divergences entre options religieuses coexistantes. Cette expérience a révélé la violence mortelle que peuvent provoquer les identités, notamment religieuses. Elle a aussi révélé les limites de la capacité du pape et des évêques à réguler pacifiquement les conflits d'interprétation, ce qui les a délogés de leur millénaire position d'autorité et de partenaires autonomes et incontournables des pouvoirs civils. Bon gré mal gré les européens s'en sonr donc remis à "\emph{César}", quitte à s'accomoder de son despotisme ("\emph{cujus regio ejus religio}") et des injustices de la raison d'état. Tout valait mieux à leurs yeux que les désordres d'un monde où chacun serait un loup pour l'autre. 

Les ecclésiastiques eux-mêmes se sont ralliés à cette position : les protestants bien entendu, qui déniaient toute autorité particulière à l'éveque de Rome et confortaient les princes dans leurs désirs de contrôler les cultes, mais aussi les catholiques. Ainsi Pierre de Bérulle (fondateur de l'Ordre de l'Oratoire) écrivait en 1623, dans un discours\footnote{\emph{Discours de l'État et des grandeurs de Jésus}.} au Roi (Louis~XIII)  :
    \enquote{\emph{un monarque est un Dieu selon le langage de l'écriture : un Dieu non par essence mais par puissance ; un Dieu non par nature mais par grâce ; un Dieu non pour toujours mais pour un temps. Un Dieu non pour le Ciel mais pour la Terre. Un Dieu non subsistant, mais dépendant de celui qui est le subsistant par soi-même ; qui étant le Dieu des Dieux, fait les rois Dieux en ressemblance, en puissance et en qualité, Dieux visibles, images du Dieu invisible}}. Jusqu'au milieu du \siecle{18} l'image de l'autorité était globalement positive, et l'exercice que les rois et les pères (et avec eux les « pères spirituels » de tous ordres) faisaient de leurs pouvoirs était regardé comme légitime et bénéfique. Dans ce cadre de pensée s'opposer au souverain comme aux pères c'était faire preuve de présomption et peut-être s'opposer à Dieu lui-même. 
    
    
    Dans quelle mesure cette vision du pouvoir et de la paternité a-t-elle rejailli sur l'image que les gens d'alors se faisaient de Dieu ? Ils prêtaient en effet à celui-ci une dureté ou même une cruauté impitoyable : exigences morales inflexibles, poids de la culpabilisation, arbitraire de la grâce, prédestination, terrorisme de la damnation, croyance en la rareté des élus,~etc. Mais on peut aussi bien se demander si ce ne sont pas les thèses des théologiens de la fin du Moyen Age qui ont renforcé l'absolutisme des pères et des rois ? Ils valorisaient en effet sans limites la toute-puissance divine. Selon Jean-Claude Monod\footnote{ in \emph{La querelle de la sécularisation : théologie politique et philosophie de l'histoire de hegel à Blumenberg}, Jean-Claude Monod, Paris, Vrin, 2002.} : "\emph{L'importance du nominalisme à la fin du Moyen Âge tient à ce que ce courant de pensée a mis en crise le système scolastique en voulant pousser l'homme à une capitulation sans condition dans l'acte de foi, et a retiré à la théologie toute tâche de médiation entre la connaissance et la foi. Ainsi en est-il de la souveraineté absolue de Dieu : volonté insaisissable et opaque "potentia absoluta", le Dieu du nominalisme et ses "décrets" se situent au-delà de toute tentative de compréhension par l'esprit humain. Tout ce qui est fait peut être défait, toute loi peut être suspendue, nulle garantie ne doit être attendue de Dieu, dont l'entendement est incommensurable au nôtre et dont dépend pourtant entièrement notre salut."} Comment penser la liberté des individus si Dieu connaît à l'avance tout leur avenir ? Comment peuvent-ils être responsables de leurs actes ? Comment imaginer un Dieu bon s'il n'est lié par aucune exigence de justice ? etc.  
    
    
    L'époque était au renforcement de toutes les autorités laïques puisque les autorités religieuses avaient failli. Les pères se voyaient donc rappeler leur devoir de maintenir leur maison en bon ordre, dans le respect des lois civiles et religieuses. On attendait d'eux qu'ils le fassent sans faiblesse, et pour y parvenir il leur était reconnu une grande part de la puissance paternelle des romains. Dans les pays de Droit écrit, revenus avant la fin du Moyen Âge à une application stricte du Droit romain, leur puissance ne cessait qu'avec leur mort. Leur mission éducative impliquait le \emph{droit de correction}. On considérait que c'était pour eux un devoir moral et social que de corriger les enfants \emph{et les épouses} indisciplinés. Jusqu'au \siecle{18} (au moins) il était admis qu'une tendresse excessive était plus dommageable, et donc plus coupable, qu'une sévérité excessive : « {qui aime bien châtie bien} ». Montaigne nous dit qu'il fut placé de sa naissance à l'âge de quatre ans chez des bûcherons, puis mis en pension en collège à partir de six ans. Il dit s'être trouvé mieux de cette enfance loin de sa famille... parce qu'il lui semblait que son père était « trop tendre%
% [1] 
\footnote{En justifiant la décision de son père par son « excès de tendresse » Montaigne nous fournit un bel exemple de ce qu'on désigne aujourd'hui sous le nom de « fidélité » ou de « loyauté » des enfants, et des trésors de compréhension dont ils sont prêts à faire preuve face à toutes les décisions, quelles qu'elles soient, que leurs parents ont pu prendre.} 
» !

 Le roi soutenait l'autorité des époux sur leur épouse et leurs enfants, et leur prêtait main-forte s'ils le demandaient, entre autres moyens par les \emph{lettres de cachet} ordonnant sans jugement\footnote{...ancêtre des placements administratifs actuels, dont il faut reconnaître qu'ils sont mieux contrôlés qu'alors par les autorités judiciaires. Il est infiniment plus aisé aujourd'hui de mettre en question leur pertinence parce que nous n'idéalisons plus la parole des pères ni des autres autorités. Au contraire nous les tenons en suspicion.} l'incarcération de l'enfant récalcitrant, mineur ou majeur, ou de l'épouse indigne, volage ou frivole ou de mauvais caractère,~etc. S'il le jugeait nécessaire, il pouvait se substituer de sa propre initiative%
% [2] 
\footnote{De même que lorsqu'il s'agit de ses enfants un père n'attend pas d'être saisi : par définition il parle en leur nom et à leur place (et le Droit romain lui donne ce droit même quand ils sont adultes).} 
aux pères défaillants dans leur fonction de faire régner l'ordre dans leurs familles. 
 Mais il se devait aussi de contrôler qu'ils n'abusaient pas de leurs pouvoirs : leur droit de correction n'était pas un droit de vie ou de mort. Jamais les parents n'ont été autorisés à estropier leurs enfants, et l'appui donné par la force publique à leurs décisions n'était pas automatique.

 

\section{Les transmissions}

    
   Le devoir des pères est d'établir ses enfants dans la vie. Il doit donc transmettre : transmettre ses biens, ses charges et fonctions, transmettre ses connaissances. Il peut se faire aider par plus compétent que lui dès que c'est nécessaire.
   
   Il n'est pas question ici de faire une histoire de l'enseignement, mais seulement d'en esquisser les traits qui ont directement rapport à notre sujet\footnote{\\\fsc{FURET} et \fsc{OZOUF}, \emph{Lire et écrire, l'alphabétisation des français de Calvin à Jules Ferry}, 1977.
\\Maurice \fsc{CAPUL}, \emph{Internat et internement sous l'ancien régime, contribution à l'histoire de l'éducation spéciale}, Thèse d'état, CTNERHI-PUF, Paris, 1983-1984.
\\Martine \fsc{SONNET}, {« Une fille à éduquer », in \emph{Histoire des femmes en Occident}, III, \siecles{16}{18}}, Collectif, sous la direction de Georges \fsc{DUBY} et Michelle \fsc{PERROT}, 2002, Chapitre 4, p. 131 à 168.
\\Sous la direction de Marie-Madeleine \fsc{COMPERE} et Philippe\fsc{SAVOIE}, \emph{L’établissement scolaire. Des collèges d'humanités à l'enseignement secondaire, XVIe-XXe siècles}, numéro spécial 90 de la revue \emph{Histoire de l’éducation}, mai 2001
\\ Marie-Madeleine \fsc{COMPERE}, \emph{Du collège au lycée. Généalogie de l'enseignement secondaire français (1500-1850)}
Collection Archives (n° 96), Gallimard, 1985.
\\sous la dir. de Marie-Madeleine \fsc{COMPERE} et d'André \fsc{CHERVEL}, \emph{Les Humanités classiques}, Paris : Institut national de la recherche pédagogique, 1997.
\\Marie-Madeleine \fsc{COMPERE},	\emph{L'histoire de l'éducation en Europe : essai comparatif sur la façon dont elle s'écrit} Paris : Institut national de recherche pédagogique ; Bern : P. Lang, 1995. }.

 Pour l'énorme majorité (99\%) des jeunes la durée de l'enseignement se réduisait à quelques années et à l'apprentissage des rudiments de la lecture et de l'écriture, mais il ne faut pas oublier l'apprentissage d'un métier dont bénéficiaient beaucoup de jeunes (la plupart ?) de façon informelle et gratuite auprès de leur père ou d'un oncle (dont tous les paysans, les pécheurs, les fils d'artisans ou de commerçants, etc...), ou à titre onéreux auprès d'un maitre d'apprentissage. 
 
L'ensemble du domaine scolaire était sous le contrôle et à la charge des évêques. Des décisions royales répétées au fil des siècles (à commencer par Charlemagne) confirmaient ceux-ci dans leurs droits et aussi dans leurs obligations : en accord avec les autorités civiles ils remplissaient d'autant plus naturellement cette mission de service public que jusqu'à la fin du moyen-âge la plupart de ceux qui avaient suivi un enseignement poussé (les "humanités") faisaient partie du clergé. En principe il était gratuit. Le financement venait de subventions, de dons, ou de taxes affectées, ou du produit de fondations qui faisaient partie des biens ecclésiastiques. 

 
En dehors des monastères qui ont toujours formé leurs propres novices, le réseau des écoles s'est développé depuis le début du Moyen Âge (cf. troisième \emph{Concile de Vaison}, 529) à partir des \emph{écoles cathédrales}, d'une part vers l'enseignement élémentaire avec les \emph{petites écoles} (écoles primaires), d'autre part vers l'enseignement supérieur (incluant à cette époque ce qu'à partir du \siecle{19} on appellera l'enseignement \emph{secondaire}) avec les \emph{universités} et leurs \emph{collèges}.  Elles étaient placées sous le contrôle du Chapitre de la Cathédrale. L'un des chanoines exerçait cette responsabilité : l'\emph{écolâtre},  le \emph{chantre} ou le \emph{chancelier}... C'est lui qui jusqu'à la fin de l'ancien régime agréera tous les candidats à l'enseignement, agrément sans lequel nul n'avait le droit d'enseigner sur le territoire sous sa juridiction. 





À partir des derniers siècles du Moyen Âge des \emph{petites écoles} paroissiales existaient dans toutes les villes importantes, fondées par les curés, ou par les municipalités, et ordinairement par les deux à la fois. A côté des connaissances profanes (d'abord la lecture, le calcul, souvent l'écriture, mais pas toujours) on enseignait aussi la religion, les disciplines du corps et de l'esprit, les bonnes manières de se conduire. Leur mission était en effet d'éduquer autant que d'enseigner. L'instruction, une fois entendu qu'elle se devait d'inclure la religion, était considérée comme la meilleure défense contre l'envie de mal faire. Curés ou pasteurs protestants, parents et autorités locales étaient d'accord sur ce point. D'autre part les citadins voyaient aussi en elle la meilleure arme pour trouver et pour garder un travail, ce qui avait à la fois un intérêt économique et un intérêt social. À partir de la Réforme et du Concile de Trente cette foi en l'enseignement s'est exprimée en un véritable apostolat (c'est pourquoi divers ordres enseignants ont été créés au fil des siècles).


 Les petites écoles s'adressaient aux « enfants des pauvres », c'est-à-dire, dans le langage d'alors, à tous ceux dont les ressources étaient précaires, ceux qui n'avaient pas de rentes, de quelque nature qu'elles soient, et qui devaient gagner leur vie en travaillant. Il s'agissait donc de l'essentiel de la population des villes. Mais les petites écoles ne pouvaient pas toujours être complètement gratuites (pas plus que les universités). Elles étaient donc à la portée des bourgeois aisés, des commerçants et artisans, mais pas toujours à celle des autres. Lorsque les paroisses ne pouvaient pas exempter ces derniers des frais de scolarité, ce qui était le cas lorsque l'ensemble de leurs paroissiens étaient réellement pauvres, seuls de généreux donateurs et surtout des ordres religieux pouvaient les prendre en charge (cf.  les « écoles de charité »). Les religieux bénéficiaient en effet d'une sécurité financière, d'une surface sociale et d'un entregent que ne pouvaient avoir des particuliers ou des communes pauvres. Certains ordres avaient d'ailleurs explicitement été créés pour assurer gratuitement l'enseignement des indigents.

Beaucoup d'enfants n'étaient pourtant pas scolarisés, même dans les villes où l'enseignement était gratuit : leurs parents avaient trop besoin du produit de leur travail, ou bien ils ne voyaient aucune utilité à un apprentissage scolaire. Même aux yeux de ceux qui envoyaient leurs enfants à l'école il n'était pas toujours évident qu'il faille que ceux-ci soient scolarisés avec assiduité pendant plusieurs années. Beaucoup, et peut-être même la plupart, se contentaient des quelques mois ou années nécessaires pour apprendre à lire et/ou à écrire. Jusqu'à la fin de l'ancien régime l'instruction des paysans (plus de 90~\% de la population) n'était pas jugée nécessaire. Étant donné le niveau des techniques alors en usage l'illettrisme n'avait pas d'incidence sur leur productivité. D'autre part leurs maîtres et seigneurs craignaient qu'une instruction même minime ne les rende « raisonneurs » et « arrogants ». En l'absence d'école les plus brillants pouvaient être distingués par leur curé (qui avait depuis l'antiquité l'obligation de faire à tous le catéchisme et de repérer les plus alertes d'esprit) et recevoir de lui les bases nécessaires pour aller au collège. Quant aux enfants de famille aisée, bourgeois et aristocrates, leur première scolarité se faisait traditionnellement auprès d'un précepteur recruté par leurs parents.  



Les écoles cathédrales et les écoles monastiques ont été créées pour fournir l'Église en clercs, mais elles ont toujours reçu un petit contingent d'élèves promis à la vie civile. Charlemagne leur en a fait une obligation. À partir du \siecle{10} la croissance des villes a provoqué la demande d'une instruction de niveau universitaire (à l'époque le secondaire et le supérieur n'étaient pas encore distingués). À partir du \siecle{12} les universités se sont créées comme des corporations autogérées de professeurs indépendants, avec l'appui des autorités civiles. Elles étaient sous le contrôle de l'évêque du lieu et leur personnel comme leurs étudiants bénéficiaient des avantages et exemptions attachés aux clercs (à cette époque la majorité d'entre eux étaient religieux ou prêtres ou destinés à le devenir). En cas de conflits entre l'évêque et l'Université le pape arbitrait. Dans le cadre des universités ont été créés des collèges caractérisés d'abord par la vie en internat à l'intention des étudiants pauvres, sur le modèle des écoles monastique (p. ex. le collège qui deviendra la Sorbonne). Leur mission était de permettre aux jeunes gens sans fortune d'entrer dans le clergé (puisqu'il y fallait un titre universitaire) mais peu à peu ces collèges d'universités sont devenus des lieux d'enseignement en complément des cours publics, et ils ont de ce fait été recherchés par d'autres candidats. L'enseignement que nous nommons « secondaire » est donc sorti de l'enseignement « universitaire ».

Même si la gestion d'un internat était lourde et source de tracas beaucoup de pédagogues en avaient une image positive, mais entre l'externat des collèges, gratuit ou presque, et la pension des internats l'écart des coûts était énorme
\footnote{Selon Martine \fsc{SONNET}, la pension d'un seul enfant, garçon ou fille, représentait presque la totalité du salaire d'un ouvrier (« une fille à éduquer », Chapitre 4 de \emph{L'Histoire des femmes en Occident}, III, \siecles{16}{18}, p. 146). C'est pourquoi en 1760 les internats parisiens n'accueillaient que 13~\% de la population scolarisée de la ville, et il semble qu'il en était de même ailleurs.}% 
. Aux familles qui ne pouvaient payer les frais d'une pension, c'est-à-dire la plupart, seul l'externat était accessible, en vivant en ville chez ses parents\footnote{... d'où la pression des municipalités pour créer un collège ou un « petit collège », et au minimum une classe de latin, une « régence latine », pour gagner quelques années de scolarité sans recourir à la pension.} ou chez un parent, ou chez un logeur peu exigeant. 

C'est en se transformant profondément que la formule du collège s'est généralisée à partir du \siecle{16} avec à cette époque une majorité d'élèves externes promis à la vie laïque, avec une réduction forte de la dispersion des âges, et un classement des élèves par niveaux, etc.. De nombreux collèges ont été créés à la demande des municipalités et/ou des évêques. Les initiatives étaient très décentralisées et les créations partaient le plus souvent des besoins et des demandes locales, et d'abord des demandes des parents d'élèves potentiels. Les enseignants étaient recrutés au sein du clergé diocésain local en fonction des compétences et des titres universitaires. Au fil du temps la gestion de beaucoup de ces nouveaux collèges a été confiée par leurs fondateurs à des ordres religieux spécialisés : surtout les jésuites et les oratoriens, à côté desquels d'autres ordres comme les bénédictins ou les dominicains ont aussi joué un certain rôle. Jusqu'à leur expulsion (1765) les jésuites ont eu un rôle prépondérant avec leurs immenses collèges de \nombre{2000} élèves, tous externes, qui recevaient une scolarité gratuite valorisant le latin comme une langue vivante et utilisant des méthodes d'enseignement actives. 

Mais tous les collèges n'étaient pas de plein exercice, avec des classes de tous les niveaux jusqu'à la classe de philosophie comprise. Beaucoup d'entre eux (les "petits collèges") se contentaient des quelques niveaux de base, quand la commune ne se bornait pas à entretenir un seul professeur de latin (une \emph{régence latine}) pour les quelques élèves concernés. Les élèves désireux d'aller jusqu'au bout du cursus secondaire allaient le terminer dans un plus gros établissement,  Beaucoup de parents se contentaient d'une scolarité réduite à quelques années de collège. 

Au fil des générations la scolarité secondaire a eu tendance à se décentraliser sans que pour autant le nombre d'élèves concernés n'ait augmenté. Ce dernier est resté étonnement stable pendant très longtemps : moins de un pour cent de la population des jeunes garçons d'âge scolaire\footnote{Selon un rapport établi en 1843 par A. F. Villemain  et cité par Antoine LEON (\emph{Histoire de l'enseignement en France}, Que Sais-je ?, PUF, Paris, 1967) il existait à la veille de la Révolution 562 collèges avec \nombre{73000} élèves, dont \nombre{40000} boursiers : 178 collèges congréganistes et 384 collèges dépendant des universités ou gérés par des communes ou des particuliers. En 1812 il y avait 36 lycées et 337 collèges publics avec \nombre{44000} élèves, et \nombre{1000} autres institutions et pensionnats privés pour \nombre{27000} élèves, soit \nombre{71000} élèves au total. En 1880 il y avait environ \nombre{150000} élèves dans les lycées et collèges, et \nombre{500000} en 1940.}. 

Pour les parents des collégiens l'éducation était un investissement familial, même quand ils se destinaient à devenir des clercs\footnote{Les membres du clergé, qui n'étaient pas astreints au voeu de pauvreté, pouvaient faire des carrières très lucratives et ainsi aider matériellement leurs frères et soeurs. Même les religieux et religieuses pouvaient aider à caser l'un ou l'autre de leurs neveux et nièces, ce qui pouvait conduire au "népotisme", c'est-à-dire à l'art d'avantager abusivement ses neveux.}, ce qui était le cas d'une proportion significative jusqu'au \siecle{18}. Cela justifiait qu'ils soient improductifs pendant leurs années de scolarité. Ceux qui étaient sans ressources mais dont les dons intellectuels étaient évidents pouvaient bénéficier de bourses, surtout ceux qui se destinaient à entrer dans les ordres.



Les collèges proposaient aux jeunes garçons de s'investir dans la découverte du savoir, et celui-ci était ressenti par leurs enseignants et leurs parents comme quelque chose qui en valait la peine. Ils entraient dans une aristocratie de l'esprit. À l'époque dans toute l'Europe l'enseignement secondaire et supérieur se faisait en latin. Sans lui on savait peut-être lire, mais on n'en demeurait pas moins un \emph{illettré} qui ne connaissait pas les \emph{belles lettres}
\footnote{C'est en latin qu'Héloïse et Abélard se sont écrit toute leur vie. C'est en latin que la République des Lettres de la Renaissance correspondait d'un bout de l'Europe à l'autre. Dans toute l'Europe les thèses de doctorat seront encore soutenues en latin durant la plus grande partie du \siecle{19}.}. C'était la langue vivante, la langue de communication des communautés intellectuelles du temps. 

Mais depuis l'\emph{ordonnance de Villers-Cotterêts} (1539) qui imposait le français comme langue administrative du Royaume, il n'était plus possible de tenir un \emph{office} public si on ne le maîtrisait pas suffisamment. La langue française n'était encore que le patois de l'Île-de-France, domaine du roi. Partout ailleurs c'était une langue étrangère qui allait mettre très longtemps à déloger les langues locales des places et des marchés. Même si dans les collèges l'accent était mis sur le latin l'enseignement du français était donc incontournable pour exercer un office au service du roi. 

Quant aux jeunes filles de famille aisée, la clôture des couvents leur interdisait toute rencontre avec les jeunes gens de leur âge et protégeait leur « vertu » et leur réputation en attendant que leurs parents les marient. Sauf quand elles se destinaient à être religieuses la durée de leur séjour au couvent était très inférieure à celle de leurs frères dans leurs collèges : un an pour préparer leur communion solennelle, deux ou trois au plus. Les jeunes filles bien dotées étaient mariées bien plus tôt que les autres. Le savoir qui leur était dispensé était nettement moins poussé que celui que recevaient leurs frères, même si les novices bénéficiaient d'un enseignement qui en faisait des lettrées, des sœurs « de chœur », capables au minimum de chanter les offices en comprenant le latin qu'elles chantaient, et d'enseigner aux jeunes pensionnaires.

\section{La correction paternelle}

 Les jeunes étaient sous l'autorité de leurs parents jusqu'à leur majorité. Qu'ils soient orientés vers des études plus ou moins longues, ou mis au travail dès qu'ils pouvaient gagner leur pain, c'est leur père (à défaut leur mère) qui en décidait. Mais  tous n'entraient pas docilement dans les projets parentaux : jeunes en fugue de leur classe d'école ou de leur atelier, fréquentations suspectes, beuveries, insultes et de voies de faits, exclusion pour indiscipline de leur collège ou de leur apprentissage, vols domestiques, inconduite sexuelle,  jeunes « libertins », c'est-à-dire rétifs à toute mesure éducative,~etc. 

 À la demande de leurs parents, ces jeunes pouvaient être traités comme des délinquants. Pour les enfants difficiles des familles aisées il y avait des solutions payantes dans les sections des collèges et internats contemporains affectés à la « correction ». Ceux qui n'en avaient pas les moyens étaient internés avec les délinquants condamnés. Leurs parents payaient une pension qui tenait compte de leurs ressources. 

 À partir de la fin du \siecle{17} et de plus en plus souvent au fil du \crmieme{18}, les \emph{enfants de famille}, garçons et filles mineurs \emph{et majeurs}, qui avaient commis de vrais actes de délinquance, mais aussi ceux qui donnaient simplement du mécontentement à leurs parents par leurs fréquentations, leur mauvaise conduite, leur indocilité, leur violence aveugle ou leur absence de sens commun (« insensés »), leurs dépenses inconsidérées, ou leurs dettes de jeu, pouvaient, sur la demande de ces derniers,  faire l'objet d'une \emph{lettre de cachet}, c'est-à-dire d'une \emph{décision administrative d'internement} dans un hôpital, une prison, une forteresse, un couvent, un collège, ou même leur déportation aux colonies. Les lettres de cachet, qui ont une origine très ancienne, bien antérieure au \siecle{17}, pouvaient aussi être accordées à l'encontre de conjoints aux comportements répréhensibles : cette mesure a beaucoup plus souvent frappé les épouses que les époux. 

 L'autorité publique n'était pas obligée d'accorder satisfaction aux demandes qui lui était faites, et restait seule juge de l'opportunité de la mesure. Elle était surtout sollicitée à Paris, notamment par les couches populaires, contrairement aux provinces où l'internement administratif était moins facile à obtenir et où les couches populaires n'y avaient guère recours. Même si au fil du temps les lettres de cachet ont fait l'objet de critiques de plus en plus virulentes et si les autorités publiques y répugnaient de plus en plus, les demandes se sont faites de plus en plus nombreuses au fil du \siecle{18}. 

 En effet les familles sollicitaient ces lettres comme une grâce : cela leur évitait la honte causée par la publicité du recours à la justice, le coût d'un procès, et la publicité de la mesure d'enfermement. La réputation du jeune (ou de l'adulte) ainsi placé pouvait s'en relever plus facilement. Cela évitait le contrôle par la justice de la nature exacte des faits incriminés et de la proportionnalité des sanctions aux dommages et délits constatés. Cela permettait à l'occasion à d'authentiques délinquants bien nés d'échapper à peu de frais aux conséquences normales de leurs actes. 

Mais cela permettait aussi aux parents abusifs d'exercer des pressions sur leurs enfants rétifs à leurs projets (ce qui expliquait les critiques de plus en plus virulentes des lettres de cachet au fil du \siecle{18}), à une époque où le consentement des parents était exigé à tout âge et pour tout mariage sous peine d'exhérédation, et où bien des entrées en religion étaient imposées par eux sans tenir compte des désirs du ou de la jeune concerné. Et cela confortait aussi l'autorité des épouses sur leurs épouses

\section{Des enfants « adoptifs » ?}

 On a vu que dans le but de défendre le mariage monogame et indissoluble, l'Église a tout fait depuis l'Antiquité pour que les enfants illégitimes ne puissent pas devenir des héritiers de plein exercice. C'est pour cette raison que l'adoption était interdite, et pourtant... De l'Antiquité à la fin de l'ancien régime, on peut observer en nombre non négligeable des situations plus ou moins proches d'une adoption, où une personne, souvent un ecclésiastique (cf. \hbox{Villon}, adopté par un chanoine), souvent aussi un couple sans enfants, exerçaient la puissance paternelle sur un enfant qui n'était pas né d'eux et qu'ils élevaient jusqu'à sa majorité. C'était par exemple le cas à Lyon, où les recteurs de l'Hôtel-Dieu « adoptaient » ainsi des orphelins. 

 Ces situations d'\latin{alumnii} (adoptions simples) étaient parfois sanctionnées par des actes juridiques où les nourriciers faisaient un legs à l'enfant devant un procureur fiscal, et où ils s'engageaient à l'élever, instruire et établir matériellement à leurs frais comme leur propre enfant. Pour autant cela ne faisait pas de lui un membre de leur famille ni un héritier. 

 En principe seul un enfant légitime sans parents pouvait bénéficier de ce dispositif. Souvent, probablement le plus souvent, il était orphelin, mais des enfants légitimes pouvaient aussi être abandonnés solennellement par leurs parents, qui reconnaissaient par écrit qu'ils renonçaient à leur puissance paternelle, et à l'héritage de leur enfant s'il décédait. Pour autant ce dernier ne changeait ni de parenté ni de nom. Quand il possédait des biens, l'adoptant, tel un tuteur, les gérait jusqu'à sa majorité et il était responsable sur ses propres biens de sa gestion. 

 Les enfants abandonnés, nés de parents inconnus, ont très longtemps été exclus de ce genre de prise en charge, parce qu'ils étaient suspects d'être illégitimes. Ainsi en était-il à Lyon jusqu'en 1765. Ensuite ils y ont été traités comme les autres. Ce n'est que dans la deuxième moitié du \siecle{18} que les idées ont changé sur ce point : à partir des années 1760-1770.}. Pourtant il était courant que des personnes accueillent pour l'élever un enfant abandonné à eux confié par un hôpital ou par une paroisse, qu'elles refusent d'être rémunérées pour l'élever, qu'elles le gardent jusqu'à sa majorité et qu'elles l'établissent dans la vie, ce qui en fait ressemblait beaucoup à la situation des enfants nés légitimes et juridiquement « adoptés ». Si aucun de leurs héritiers légitimes ne s'y opposait, elles faisaient de lui l'un de leurs héritiers. Mais il n'était pas question pour cet enfant d'hériter d'une fonction impliquant l'exercice public du pouvoir. 
Derrière les mots employés il n'est pas toujours facile de reconnaître les situations réelles : adoption simple ? tutelle ? parrainage ?
\footnote{Cf. Jean-Pierre \fsc{GUTTON}, \emph{Histoire de l'adoption en France}, 1993.} 

\section{Les enfants illégitimes}

 Alors que les grossesses n'avaient pas à être déclarées, à partir de 1556 obligation est faite par Henri~II de déclarer toutes les grossesses illégitimes (et elles seules), sous peine pour les filles non mariées et les femmes veuves depuis plus d'un an qui seraient enceintes d'être accusées d'infanticide si leur enfant décédait avant son baptême\footnote{... qui avait valeur officielle de déclaration de naissance puisque les curés avaient reçu peu de temps auparavant l'obligation de tenir les \emph{registres de catholicité}, ou registres de baptême, ancêtres directs des registres d'état civil.} 
(crime en principe puni de mort). Dans la déclaration devait figurer le nom du père allégué par la mère, sauf refus de celle-ci. Cette déclaration renforçait la position de la mère face à l'homme qui l'avait engrossée, et celle de son enfant, et permettait les actions en justice. Cette décision royale a été rappelée par Henri~III en 1585, et renforcée par Louis~XIV. C'est ainsi qu'en 1708 ce dernier ordonnait encore aux curés de la rappeler en chaire tous les trois mois. 

 Si elle l'a été si souvent, c'est qu'elle n'a jamais été observée de manière rigoureuse. Il semble même que la majorité des grossesses illégitimes n'aient jamais été déclarées. En dépit de la sévérité des peines annoncées les mères préféraient oublier de se signaler à l'attention des autorités lorsqu'elles pensaient pouvoir mieux défendre leurs intérêts et leur réputation (et ceux de leur enfant) par un arrangement discret avec le géniteur (ex. : mariage, pension alimentaire, octroi d'une dot,~etc.) ou par un abandon discret. Combien parmi les veuves et filles dont l'enfant est décédé sans baptême ont-elles effectivement subi les peines prévues ? Il ne semble pas que les autorités aient poursuivi ce genre d'infraction avec beaucoup d'énergie : le plus souvent les tribunaux accordaient de larges circonstances atténuantes aux « coupables » déférées devant elles
\footnote{Frédéric \fsc{Chauvaud}, Jacques-Guy \fsc{Petit}, Jean-Jacques \fsc{Yvorel}, \emph{Histoire de la justice de la Révolution à nos jours}, Presses universitaires de Rennes, 2007.}% 
. 

 Tout enfant, même illégitime, avait le droit d'exiger de ses auteurs des « aliments » c'est-à-dire des moyens de vivre. Un vieil adage juridique, toujours cité, disait en effet que \emph{qui fait l'enfant doit le nourrir}. Le représentant naturel de l'enfant né hors mariage est sa mère, et \emph{protéger celle-ci était aussi protéger l'enfant}. Les actions de la mère%
% [14] 
\footnote{Nommée « fille-mère », et n'ayant droit qu'au titre de « mademoiselle » jusqu'au milieu du \siecle{20}. Ce n'est pas un enfant qui pouvait faire d'elle une femme, une « dame », mais un époux en règle.} 
contre le géniteur qui l'avait délaissée étaient encouragées et soutenues, notamment par les hôpitaux, qui en cas d'abandon de l'enfant devaient en assumer seuls la charge. Elle pouvait entreprendre une \latin{actio provisionis} : demande de provisions pour frais de grossesse ou d'accouchement. Si plusieurs hommes avaient partagé à la même période son intimité ils pouvaient être solidairement responsables de l'enfant. Elle pouvait aussi tenter une \latin{actio susceptionnis partus} ou \latin{actio captionis} : action qui demandait de condamner le géniteur à assumer les frais de l'éducation de l'enfant, sur lequel il ne recevait pour autant aucune autorité. 

 \latin{L'actio dotis} prévoyait que le coupable d'un viol épouse la célibataire qu'il avait déflorée, surtout s'il l'avait engrossée. S'il refusait de l'épouser, ce qui était son droit, il devait payer une dot à la mère et financer l'entretien de l'enfant. Il en était de même si le géniteur était déjà engagé ailleurs (mariage, vœux religieux, ordination sacerdotale). Quel que soit son statut (célibataire, marié, clerc, moine, noble, roturier ou serf) il était et demeurait responsable de la vie de l'enfant et devait donc le nourrir. Même si le géniteur n'était pas père légal il restait \latin{nutritor}.

 Par contre les enfants adultérins étaient toujours traités comme des enfants abandonnés, comme s'ils n'avaient ni père, ni mère, ni \latin{nutritor}. Ils n'avaient aucun droit vis-à-vis de leurs deux géniteurs, dans la famille desquels ils n'entraient pas et auxquels ils ne pouvaient pas réclamer des aliments%
% [15]
\footnote{Ceci dit la loi n'interdisait pas à leurs auteurs de prendre librement l'initiative de pourvoir à leur éducation.}% 
. Ils étaient exclus de toute possibilité de légitimation, même par mariage, puisque leurs géniteurs ne pourraient pas se marier, même après la mort de l'époux qui faisait obstacle à leur mariage. 

 Étaient encore plus rigoureusement exclus de toute légitimation les enfants nés d'une relation incestueuse.


\chapter{Organisation d'une police des pauvres}

 À la fin du Moyen Âge il était courant que les mendiants représentent 10~\% de la population 
\footnote{José \fsc{CUBERO}, \emph{Histoire du vagabondage}, 1998, p. 8.}% 
. Les solutions en vigueur depuis la fin de l'antiquité pour traiter l'indigence et les malheurs individuels, pensées pour de petites communautés rurales où tous se connaissent
\footnote{José \fsc{CUBERO}, 1998, p. 42 et suivantes.}% 
, n'étaient plus à l'échelle des problèmes en un temps où les villes débordaient de leurs murailles anciennes, et où les États modernes se constituaient, imposant plus d'ordre, de rigueur et de contrôles, et rognant peu à peu les larges marges jusque là consenties entre les principes et les pratiques réelles. 

 Face à la pauvreté dès 1350 apparaissent les signes avant-coureurs d'un changement des mentalités et des pratiques. Les vagabonds ne sont plus assimilés aux pèlerins du moyen-âge mais sont de plus en plus souvent considérés comme des fauteurs de trouble. On commence à parler de « {bons} » et de « {mauvais pauvres} ». Les « {bons pauvres} » ou « {pauvres honteux} » ont le droit moral de mendier parce qu'il leur est impossible de travailler, et parce qu'ils restent rattachés à leur cadre villageois, à leur paroisse d'origine : enfants, infirmes, malades, vieillards... Ils ne se soustraient pas au contrôle de leur communauté. Les \emph{mauvais pauvres} sont ceux qui ont force et santé mais qui fuient le travail par paresse ou par goût de l'errance
\footnote{... ou par refus de conditions de travail par trop inacceptables (mais cela c'est notre point de vue du \siecle{21}, ce n'était pas celui des décideurs d'alors).} 
loin de tous les cadres sociaux, sans aveu. On soupçonne les vagabonds de vivre dans la débauche et de commettre nombre de délits (notamment des vols). On a peur de leur nombre qui favorise la mendicité agressive et qui intimide les personnes sans défense (enfants, jeunes filles, femmes, vieillards). On les accuse de contrefaire maladies ou infirmités, d'enlever des enfants pour exciter la pitié des passants
\footnote{José \fsc{CUBERO}, 1998, p. 70.}% 
, et même de mutiler ces derniers pour obtenir plus d'aumônes%
%[5]
\footnote{Bronislaw \fsc{GEREMEK} fait état de procès tenus dans la région parisienne en 1449 où ont été condamnés des criminels qui avaient successivement enlevé plusieurs enfants à leurs parents, enfants auxquels ils avaient crevé les yeux et coupé bras ou jambes, pour en tirer profit en mendiant. (in \emph{Les marginaux parisiens aux \crmieme{14} et \crmieme{15} siècles}, Paris, 1976).}% 
. 

 Dès le \siecle{14} les hôpitaux refusent de plus en plus souvent les vagabonds%
% [6]
\footnote{José \fsc{CUBERO}, 1998, p. 68.}% 
, tandis que de nombreuses mesures de police tentent de les contrôler et surtout de les chasser. À l'intention des petits délinquants, des vagabonds et autres chômeurs sans ressources avouables on fait des expériences multiples de travaux « forcés », travaux d'utilité publique, ou même galères du roi%
%[7]
\footnote{En 1456 les États du Languedoc prévoient cette peine pour les vagabonds invétérés. \emph{En 1486, Charles~VIII étend cette mesure à l'ensemble du royaume.} La condamnation aux galères, résurgence de la condamnation antique aux mines, \latin{ad metallas}, se substitue alors dans la plupart des cas à la peine de mort, jusque là appliquée largement en l'absence de peines plus adaptées : \emph{avec la peine des galères... le Moyen Âge renoue avec la notion antique de l'esclavage... Seul le travail rédempteur peut éviter les galères[7] à ces mendiants valides et vagabonds qui menacent la paix}, José \fsc{CUBERO}, p. 78 idem.}% 
. Le fait que ces décisions d'expulsion aient été périodiquement reformulées montre et leur relative inefficacité, et la persistance des représentations qui les sous-tendent.

 Les cités, en expansion, sont dirigées par leurs bourgeois, commerçants, artisans, juristes et autres détenteurs d'offices. Leur expérience personnelle les porte à tenir pour synonymes les vertus familiales et « bourgeoises » : fidélité, économie, sens de l'effort, contrôle de soi et prévoyance. Pour eux un sou est un sou : contrairement aux aristocrates ils ne valorisent ni le panache, ni le faste, ni la prodigalité. Les anathèmes des religieux contre la richesse, traditionnels, ne les impressionnent plus, sauf lorsqu'ils sont à l'article de la mort, et ils sont fiers de leur fortune. Ils ont la tranquille assurance de ceux qui ont réussi. À leurs yeux les autres n'ont qu'à en faire autant, et ils se font forts de le leur enseigner.
 
 
 Cela s'accompagne  de l'exercice d'un contrôle rigoureux sur les comportements, les siens et ceux des autres. Dans la plus grande partie des sociétés européennes c'est à la suite des réformes protestante et catholique que les écarts seront les plus faibles entre la morale sexuelle et conjugale officielle et les pratiques réelles. Ce sera le moment où tous les laïcs ou presque se marieront et feront des enfants. Ce sera le moment où les taux de naissances illégitimes et de conceptions pré conjugales seront au plus bas de toute l'histoire européenne : de 1650 à 1750, Normandie : 2 à 3~\% d'enfants illégitimes ; bassin parisien : 1~\% ; Languedoc et Bretagne : 1 à 2~\%. Angleterre sous Cromwell : moins de 1~\% ; en 1600 : 3,2~\%. Ces taux impliquent un haut degré de contrôle social, exercé conjointement par les familles, par les autorités civiles et par les autorités religieuses.

 Au moment où les peuples d'Europe sont en train de se cliver entre catholiques et protestants apparaissent simultanément dans tous les grands États européens des mesures très semblables pour contrôler pauvres et vagabonds. Pas un seul instant la marche vers la rationalisation du contrôle des pauvres et l'organisation de leur mise au travail, forcé si nécessaire, n'a été entravée ou modifiée par les guerres de religion, et tous les États concernés connaissent des évolutions à peu de choses près superposables : mêmes représentations, mêmes solutions, mêmes réussites et mêmes impuissances. 

\subsection{Création du placement d'office des mineurs de famille}
 Le 22 Avril 1532 le Parlement de Paris ordonne une fois de plus que tous ceux qui dans cette ville peuvent travailler et n'ont ni emploi ni revenus avouables seront contraints à entrer dans les ateliers publics qu'il organise pour eux. Ils travailleront enchaînés deux à deux, gardés rigoureusement et employés aux travaux d'utilité publique les plus rudes. On reconnaît là les pratiques des bagnes%
% [8]
\footnote{... décrites par exemple par Philippe \fsc{HENWOOD} dans \emph{Bagnards à Brest} : « l'accouplement » des bagnards enchaînés, deux par deux, p. 40, 41 et 42,~etc.}% 
. \emph{Mais l'Ordonnance royale du 22 avril 1532 est fondamentale en ceci qu'elle ordonne le placement d'office des enfants des vagabonds arrêtés.} L'autorité parentale peut désormais être disqualifiée en l'absence de tout autre délit que le vagabondage. Ce n'était pas la première fois que des essais de ce genre étaient tentés (exemple : Reims, 1454) mais cette fois il s'agit de le faire à Paris, où se trouve la plus grande concentration de vagabonds du royaume (environ un tiers) et l'Ordonnance est signée par le roi. Elle donne aux \emph{bureaux des pauvres}, où siègent des représentants des autorités ecclésiastiques et judiciaires, une part de l'autorité de l'État. Ils exercent une fonction d'autorité sur tous les pauvres, dont ils peuvent et doivent contrôler non seulement l'incapacité de travailler, mais aussi la correction des pratiques conjugales, éducatives et religieuses. Contrôle et assistance sont désormais liés, et les assujettis ont peu de recours judiciaires possibles : ils subissent une justice d'exception. 

 L'Ordonnance royale de 1566 étend l'interdiction de la mendicité à tout le royaume de France et met les pauvres à la charge de leur paroisse d'origine (\emph{domicile de secours} : seul lieu où l'indigent a droit aux secours) ce qui leur interdit de vagabonder. Elle prévoit que les \emph{bureaux des pauvres} et autres \emph{aumônes générales} doivent si nécessaire organiser et financer des ateliers pour donner du travail aux indigents valides. Entre 1550 et 1600, des forces de police spéciales placées sous l'autorité directe des {bureaux des pauvres} (souvent appelées \emph{archers de l'Hôpital}) sont chargées de traquer la mendicité, de poursuivre hors de l'hôpital et d'arrêter les vagabonds, de récupérer les enfants placés par les bureaux des pauvres lorsqu'ils ont fugué de leur lieu de placement, et de faire régner l'ordre dans les hospices et hôpitaux. 

 Au \siecle{17} les expériences réalisées et les réflexions entretenues par les divers acteurs de l'assistance et du contrôle social confluent dans l'idée qu'il convient de regrouper en une seule administration centralisée les hôpitaux et les hospices, et d'y renfermer tous les indigents qui ne peuvent se prendre en charge seuls, en raison de leur immaturité, de leurs infirmités ou maladies, ou bien en raison de leurs comportements%
% [9]
\footnote{Sources principales :
\\Collectif sous la direction de Jean \fsc{IMBERT}, \emph{L'histoire des hôpitaux en France}, 1982.
\\Maurice \fsc{CAPUL}, \emph{Internat et internement sous l'ancien régime, contribution à l'histoire de l'éducation spéciale}, Thèse d'État, 4 tomes, Tomes 1 et 2, \emph{Les enfants placés}, Tome 3 et 4, \emph{La pédagogie des maisons d'assistance}, 1983-1984.
\\Michel \fsc{FOUCAULT}, \emph{Folie et déraison : histoire de la folie à l'âge classique}, 1961.
\\Michel \fsc{FOUCAULT}, \emph{Surveiller et punir, naissance de la prison}, 1975.
\\Bronislaw \fsc{GEREMEK}, \emph{La potence ou la pitié, l'Europe et les pauvres du Moyen Âge à nos jours}, 1987.
\\Jean \fsc{IMBERT}, \emph{Le droit hospitalier de l'ancien régime}, 1993.
\\Jacques \fsc{TENON}, \emph{Mémoires sur les hôpitaux de Paris}, 1788.}. 

\subsection{Création de l'Hôpital général}

 Louis~XIV ordonne en 1656 la création d'un \emph{Hôpital Général} dans toutes les grandes villes du royaume, et le 14 juin 1662 l'établissement d'un hôpital général dans \emph{toutes les villes et gros bourgs}. Les directeurs, nommés à vie, reçoivent des pouvoirs administratifs et de police pour accomplir leurs missions : \emph{tout pouvoir d'autorité, de direction, d'administration, commerce, police, juridiction, corrections et châtiments sur tous les pauvres de Paris, tant en dehors qu'au-dedans de l'hôpital général \emph{[...]} sans que l'appel puisse être reçu des ordonnances qui seront par eux rendues} [...] Les administrateurs de l'hôpital jugent sans appel, à charge pour eux \emph{si lesdits pauvres méritent peine afflictive plus grande que le fouet, de le mettre es mains du juge ordinaire pour à la requête du procureur d'office leur procez estre fait et parfait}. 





 Que le mouvement de création des Hôpitaux généraux se soit poursuivi à la demande des autorités locales, et pas seulement en France, jusqu'à la fin du \siecle{18} montre que cette formule de l'institution fermée et à l'écart du monde correspondait%
% [10] 
\footnote{Maurice \fsc{CAPUL}, idem, T III, p 301.} 
bien aux conceptions de l'époque : partout en Europe on observait à cette période le même mouvement. Les \anglais{Poor Laws} anglaises ordonnaient en 1661 ou 1662 l'enfermement des pauvres dans des \anglais{Workhouses} qui sont l'exact pendant (en plus dur ?) des hôpitaux généraux. Il en était de même à Berlin,~etc.

 Les contemporains essayaient de ne pas avoir personnellement affaire à ces institutions dont le régime n'était pas fait pour être désirable. Par contre ils approuvaient leur utilisation pour mettre à l'écart les indésirables et pour éviter les catastrophes en cas de disette ou de crise de l'emploi. 

 Et pourtant il y avait des listes d'attente pour entrer à l'hôpital et il fallait souvent patienter avant d'y être admis. Une recommandation était ordinairement nécessaire (très souvent celle de son curé). Un certain nombre de personnes, pauvres mais non indigentes, acceptaient même de payer pension pour y entrer, ce qui laisse à penser que même si les conditions de vie y étaient rudes (mais ces personnes-là n'étaient pas astreintes au travail forcé) il y avait encore pire ailleurs. Pour elles l'Hôpital Général fonctionnait comme une maison de retraite (cf. les « petites maisons » dans le cadre de celui de Paris), et assumait une forme de prise en charge qui existait déjà avant sa propre création.

 Quant à ceux des mendiants et vagabonds qui troublaient l'ordre public par leurs débordements, ils ne venaient pas à l'hôpital de leur plein gré et leurs comportements le traduisaient, aussi les employés des hôpitaux généraux ne faisaient-ils aucun effort pour les garder. Au bout d'un siècle d'expériences cela conduira les Intendants du roi à créer à partir de 1768 à l'intention de cette population les \emph{dépôts de mendicité}, dépôts qui seront à l'origine des futures \emph{prisons départementales}%
% [11]
\footnote{Leur histoire est complexe et s'étend sur une bonne part du \siecle{19}. Voir entre autres : \emph{Lieux d'hospitalité : hospices, hôpital, hostellerie}, ouvrage collectif sous la direction d'Alain \fsc{MONTANDON}, P.U. Blaise Pascal, 2001.}% 
.






 En ce qui concerne les enfants les plus jeunes la croyance en la vertu éducatrice et rééducatrice de l'internat est à cette époque à son apogée. Les décideurs n'ont pas encore compris l'importance des relations interpersonnelle (corps à corps et cœur à cœur) dans la construction d'une personnalité d'enfant. Ils n'ont pas plus compris combien est déterminante, pour l'investissement de quelque enseignement que ce soit, la différence entre le placement en internat scolaire choisi par les parents, et l'internement d'office ordonné contre leur gré par une instance administrative ou judiciaire. Ils n'ont pas compris non plus la différence qui existe entre la prise en charge des enfants sans famille (orphelins ou abandonnés) qui ni les uns ni les autres n'ont plus de parents, et celle des enfants qui connaissent leurs parents mais à qui on prétend interdire de s'identifier à eux. 


\subsection{Enfants trouvés et abandonnés}

Les enfants abandonnés pris en charge par les institutions d'assistance pouvaient avoir été déposés dans un lieu public ou dans le « tour » d'un hôpital, ou confiés par leur père ou leur mère, ou volontairement « perdus » par eux dans un lieu inconnu%
%[18]
\footnote{L'histoire du \emph{Petit Poucet}, racontée par \fsc{Perrault} dans les \emph{Contes de ma mère l'oye} (1697) a parfois correspondu à une réalité, pour des enfants très jeunes incapables de dire de quelle commune ils venaient ni comment s'appelaient leurs parents.}% 
. Beaucoup de nouveaux-nés étaient abandonnés par leurs mères dans les services d'accouchement des hôpitaux, que seules fréquentaient les indigentes qui ne pouvaient accoucher à leur propre domicile ni chez une sage-femme. D'autres tout-petits n'étaient pas abandonnés à proprement parler. Il s'agissait par exemple d'enfants dont les pères ou/et mères étaient incarcérés dans les « \emph{lieux de force} » (dont la prison pour femmes de \emph{La Force} qui faisait partie de l'hôpital de la Salpêtrière) pour vagabondage, prostitution ou autres actes de délinquance, et qui ne pouvaient donc pour un temps s'occuper d'eux. Dès que l'incarcération durait un temps significatif (un an ?) la restauration des droits parentaux devenait impossible. 

 À part ce cas les enfants abandonnés pouvaient être repris par leurs parents. Il fallait évidemment que leur abandon n'ait pas été anonyme pour que ce retour soit possible. En fait ces \emph{retours en famille}étaient rares, les causes de l'abandon, et d'abord la misère, persistant dans la plupart des cas.
 De nombreux enfants entraient à l'Hôpital bien après leur petite enfance : « \emph{... dans la généralité de Lyon, le plus grand nombre d'enfants présentés aux hôpitaux par leurs parents ont une dizaine d'années…} » À cet âge la plupart des enfants « de famille » travaillaient déjà. Ceux qui étaient confiés à l'hôpital étaient donc souvent ceux qui étaient jugés inaptes au travail. Certains d'entre eux se présentaient d'eux-mêmes à l'hôpital. 

 Au-dessous de 4 à 5 ans les enfants de l'Hôpital sont placés en nourrice. Une fois finie la petite enfance, le placement en institution est préféré. Les administrateurs croient que leurs Hôpitaux offrent des possibilités d'éducation nettement supérieures à une famille nourricière, pour des raisons variées, dont la modestie du niveau culturel des nourrices et de leur maris, qui sont le plus souvent paysans ou ouvriers agricoles, et parce que l'hôpital fournit une scolarité qu'on ne trouve pas à la campagne. Ils estiment aussi que les possibilités de trouver un emploi sont plus grandes en ville. Peut-être ne se sentent-ils pas non plus le droit de déraciner pour toujours des jeunes citadins en les laissant vivre à la campagne, surtout s'ils ont de la parenté dans la ville ? Mais il faut aussi tenir compte du fait que le prix de journée de l'hôpital est à l'époque nettement inférieur au salaire d'une nourrice.

 Tous les enfants de 6 ans et plus, non placés chez un maître artisan ou un nourricier, vivent dans les murs de l'hôpital. Même quand ils ont une famille, les enfants placés en sont plus ou moins radicalement coupés, \emph{même quand leurs parents sont placés dans le même établissement}. Les clôtures internes de l'hôpital sont aussi hautes que son mur d'enceinte%
% [19]
\footnote{Il n'est pour en être persuadé que de visiter la chapelle de l'Hôpital de La Salpêtrière.} 
. Pour nombre d'enfants cette coupure est définitive. 

 En dépit d'un souci éducatif certain%
% [20] 
\footnote{... manifesté à Paris par 5 heures 30 d'enseignement par jour, durant six jours par semaine, ce qui n'a rien à envier aux écoles primaires d'aujourd'hui... mais aussi un nombre d'élèves très élevé pour un seul maitre.} 
l'encadrement humain des jeunes placés est extrêmement réduit (d'où la modestie du prix de journée), ce qui contraint les relations entre les jeunes et les adultes à être formelles, distantes et souvent impersonnelles%
%[21]
\footnote{Selon l'expression de Maurice \fsc{CAPUL} : \emph{pour les pauvres, les moyens de la pédagogie étaient pauvres}.}% 
. Contrairement aux jeunes « de famille » inscrits par leurs parents dans les collèges contemporains, il ne s'agit pas d'intégrer ces jeunes à la « grande » culture ni de leur donner les moyens de penser plus ou moins librement : il s'agit seulement, comme dans les petites écoles, de leur donner les rudiments de la lecture et de l'écriture, et d'en faire de bons pauvres.

\subsection{« Correctionnaires »}

Les mineurs « correctionnaires » sont les jeunes qu'il faut « corriger », ceux dont les comportements font problème, c'est-à-dire les délinquants, rebelles et opposants : mineurs condamnés par décision de justice, faux saulniers de moins de 14 ans, vagabonds, mendiants, prostitué(e)s, « enfants de bohême ». Les enfants au dessus de 6 ans sont soumis aux mêmes règles de droit que les adultes. Dès l'âge de 8 ou 10 ans la peine de mort peut leur être appliquée si une « malignité » exceptionnelle justifie de les exclure du bénéfice de l'excuse de minorité. Les jeunes délinquants sont ordinairement condamnés à un temps d'incarcération déterminé : de quelques mois à 20 ans et plus. Mais ils peuvent aussi être enfermés pour une durée indéterminée : aussi longtemps que l'administration estimera qu'ils ne seront pas suffisamment amendés, jusqu'à leurs 25 ans et plus. Les jeunes garçons condamnés aux galères pour des délits commis sans l'excuse de minorité ne peuvent y être envoyés avant leurs 15 ou 16 ans. Ils attendent donc à l'hôpital d'avoir atteint l'âge d'aller au bagne, soumis au régime des autres correctionnaires, mais le temps qu'ils passent à l'hôpital ne compte pas comme temps d'exécution de la peine

\subsection{« Religionnaires »}

À partir de la \emph{Révocation de l'Édit de Nantes} (1685) ce terme désigne les enfants des protestants rebelles à la conversion au catholicisme qu'on exige d'eux%
%[22]
\footnote{L'Angleterre avait précédé la France dans la persécution des dissidents religieux et leur exclusion de toutes les charges et fonctions officielles. La légitimité des mariages des protestants n'est plus reconnue, ce qui fait de leurs enfants des bâtards incapables d'hériter. Ils se voient retirer leurs droits parentaux. Pour cette raison dès l'âge de sept ans leurs enfants leur sont enlevés. C'était l'application stricte du principe \latin{cujus regio, cujus religio}, « {un roi, une foi, une loi} ». Il faudra attendre la fin du \siecle{18} pour que la tolérance apparaisse comme une vertu et non comme une faiblesse.}

 À partir de cette date il est demandé aux hôpitaux généraux d'enfermer et rééduquer les membres de la « \emph{religion prétendue réformée} » (RPR) si aucune autre solution n'est possible. Les enfants de ceux qui ne peuvent payer sont placés en hôpital général, avec les correctionnaires. Les autres sont placés aux frais de leurs parents dans une section de correction d'un collège (catholique comme tous les collèges du royaume à partir de la Révocation), avec les enfants indisciplinés ou récalcitrants des mêmes milieux sociaux qu'eux. Ils y sont soumis à une pression morale ouverte ou insidieuse, brutale ou habile, pour les pousser à abjurer la religion de leurs parents et à se convertir au catholicisme. Leur sortie de l'hôpital ou du collège dépend en grande partie de leur « conversion ». 

 Selon Maurice \fsc{CAPUL}, cette politique a été poursuivie activement de 1685 au milieu du \siecle{18}, en dépit du fait qu'elle ne donnait que des résultats insatisfaisants : selon les observateurs du temps elle produisait des adultes peu consistants, qui ne savaient plus à quoi ils croyaient, ou des sceptiques qui ne croyaient plus à rien. D'autre part elle jetait la discorde au sein des familles et la brouille entre les parents et les enfants. Elle va se déliter peu à peu après le milieu du \siecle{18}, mais ce n'est qu'en 1787 que \emph{l'Édit de Versailles} y met un terme en créant un état-civil laïque, qui rend aux enfants protestants leur légitimité. Le roi prend officiellement acte de la tolérance dont le culte protestant avait fini par bénéficier à cette date%
% [23] 
\footnote{Depuis l'affaire Calas (condamné par le parlement de Toulouse à être roué, exécuté en 1762) et l'intervention de Voltaire (qui avait entrainé sa réhabilitation en 1765) la répression du protestantisme s'était adoucie : dans l'opinion publique la légitimité avait changé de camp.} 
dans la réalité quotidienne. Les dispositions de cet édit concernent aussi les français de confession juive.





%H1 Contestation des familles par les Lumières
%H2 La révolution française et les familles
%I1 La famille du Code Napoléon
%I2 La police des familles au XIXème siècle
%J1 3ème et 4ème républiques, principales décisions dans le domaine des familles
%J2 Séparation des Eglises et de l'Etat
%J3 Contestation de la famille du Code Napoléon
%J4 L'Etat, providence des familles ? 

\part{Des Lumières au baby-boom}

% Le 28.02.2015 :
% Antiquité
% Moyen Âge
% _, --> ,
% Le 24.02.2015 :
% ~etc.
% Moyen-Âge
% ~\%


\chapter[Contestation des autorités établies par le mouvement des Lumières]{Contestation des autorités établies par le mouvement des Lumières}


 Les penseurs qui se réclamaient des « lumières » de la raison, et qui entendaient tout leur soumettre se sont attaqués à l'argument d'autorité et à tous les dogmes, à « l'obscurantisme ». Leur audience est allée croissant au fur et à mesure qu'avançait leur siècle, et surtout à partir de 1760-1765, moment charnière d'une grande importance. À partir de cette date, et du moins dans la population « éclairée », un certain nombre de faits se mettent à poser insupportablement problème, et des solutions jusque là inenvisageables deviennent évidentes. En lien avec ces phénomènes on assiste à partir du milieu du \siecle{18} au décollage économique d'une France qui se développe à grands pas%
% [1]
\footnote{Sources principales :
\\José \fsc{CUBERO}, \emph{Histoire du vagabondage du Moyen Âge à nos jours},1998.
\\Jacques \fsc{DONZELOT}, \emph{La police des familles}, 1977.
\\Bronislaw \fsc{GEREMEK}, \emph{La potence ou la pitié, l'Europe et les pauvres du Moyen Âge à nos jours}, 1987.
\\Jack \fsc{GOODY}, \emph{L'évolution de la famille et du mariage en Europe}, 1985.
\\Jean-Philippe \fsc{LÉVY} et André \fsc{CASTALDO}, \emph{Histoire du droit civil}, 2002.}% 
.

 
\section{Contestation de l'autonomie de l'Église}

 Outre le service du culte dans les quarante mille paroisses du pays, l'Église subvenait aux besoins de l'assistance (hôpitaux, hospices, enfants des hôpitaux placés en nourrice, et une part notable de l'assistance au domicile) et de l'enseignement (petites écoles, collèges et universités), dont une assez grande part était gratuite. Depuis la fin de l'Antiquité, les prêtres, religieux et religieuses fournissaient la majeure partie du personnel des hôpitaux, des collèges et des universités. Les hôpitaux étaient fondés comme les couvents et les collèges, dans la plupart de cas sur des initiatives individuelles. 

 Sauf exception leurs ressources étaient similaires : ils vivaient des revenus de biens en capital reçus de leurs bienfaiteurs (le plus souvent des personnes privées) et ressortissant du régime juridique et fiscal des biens ecclésiastiques (biens de \emph{mainmorte}). L'Église possédait%
% [2] 
\footnote{Cf. François \fsc{BLUCHE}, \emph{L'ancien régime, institutions et sociétés}, 1993, p. 71-72.} 
les églises, les cures, et les bâtiments nécessaires à l'activité des hôpitaux, collèges et universités. Elle possédait aussi des biens de toute nature (7~\% des terres du royaume, maisons de rapport,~etc.) dont les revenus subvenaient aux besoins de fonctionnement de ces diverses institutions. 

 Au fil du \siecle{18} ce modèle a été de plus en plus sévèrement critiqué. L'opinion publique considérait que la gestion des biens ecclésiastiques était négligente et entachée d'amateurisme. Il existait de grandes disparités de revenus entre communautés religieuses : certaines étaient plus riches que nécessaire pendant que d'autres vivaient dans une gêne extrême. Elle dénonçait le tribut prélevé sur les revenus de ces biens par les rentes de situation, les emplois fictifs, et d'abord le plus criant, c'est-à-dire le système de la \emph{commende}.

 Dans ce système mis en place à la Renaissance, le roi de France avait obtenu du Pape que les revenus d'une abbaye ou d'un couvent (rarement ceux d'un hôpital), soient attribués à un \emph{bénéficier} de son choix au même titre que les autres \emph{bénéfices} ecclésiastique (évêchés et cures%
% [3]
\footnote{C'était une façon pour le roi de récupérer les revenus des biens qui avaient été donnés par les autorités civiles aux ordres religieux, une façon déguisée de soumettre les religieux à une imposition, alors qu'ils étaient théoriquement non imposables. Cf. \fsc{MINOIS}, 1989.}% 
). Il n'était pas nécessaire que le \emph{bénéficier} appartienne à la maison concernée, ni qu'il y réside, et il n'y exerçait aucune autorité spirituelle. Il fallait et il suffisait qu'il soit homme et tonsuré, ce qui ne l'engageait à rien en termes de vie religieuse, à part le port de la tenue ecclésiastique et l'interdiction de se marier (ce qui ne voulait pas dire faire voeu de chasteté). Par définition le bénéficiaire de cette nomination n'avait pas non plus fait vœu de pauvreté. Il suffisait qu'il laisse aux moines de quoi vivre, après quoi il pouvait consommer tout le reste. L'abbé commendataire avait financièrement intérêt à ce qu'il y ait le moins de moines ou de religieuses possible, et à minimiser les dépenses d'entretien et toutes les aumônes aux pauvres et autres dépenses improductives, tandis que de leur côté les religieux avaient intérêt à tirer de leurs biens le maximum de revenus pour qu'il leur en reste assez après le prélèvement du commendataire, ce qui les poussait à être exigeants face à leurs fermiers et locataires. 

 Par ailleurs on reprochait aux fondations religieuses d'être trop nombreuses et d'accaparer sans cesse plus de biens puisqu'elles n'avaient pas d'héritiers, ce qui donnait à leurs gestionnaires un pouvoir d'influence excessif sur la société, au détriment parfois des objectifs des autorités civiles et de l'intérêt commun, en stérilisant une part excessive de la richesse nationale. Les économistes de cette époque pensaient que le total des richesses existantes était fixe, inextensible. La croissance d'une famille nouvelle (charnelle ou spirituelle) impliquait donc à leurs yeux l'appauvrissement de toutes les autres : \emph{"un des principaux objets de notre attention, ce sont les inconvénients de la multiplication des établissements des gens de mainmorte et la facilité qu'ils trouvent à acquérir des fonds naturellement destinés à la subsistance et à la conservation des familles, \emph{[...]} qui ont souvent le déplaisir de s'en voir privées \emph{[...]} en sorte qu'une très grande part des fonds de notre moyenne se trouve actuellement possédés par eux ..."} (Édit du 25 Août 1749). 

 Mais ces problèmes patrimoniaux n'avaient rien de nouveau. S'ils ont été mis en avant avec détermination durant la seconde moitié du \siecle{18}, c'est que le monopole de l'Église sur les fonctions d'assistance et d'enseignement n'allait plus de soi : son autorité morale était contestée avec vigueur. 


\section{Contrôle des autorités civiles sur les congrégations religieuses}

 Il semble, sans qu'on puisse en faire une règle générale, que beaucoup de monastères aient été dans un état pitoyable à partir du milieu du \siecle{18} : effectifs squelettiques, ferveur discutable ou absente, indiscipline, non respect des règles de l'ordre... Les causes sont sans doute nombreuses et l'état général des esprits à l'époque des lumières n'inclinait peut-être pas à la vie contemplative. D'autre part l'utilisation traditionnelle des monastères au service de la régulation des familles était en train de tomber en désuétude. Si les critiques contre les vocations forcées, aussi anciennes que le phénomène lui-même, pouvaient enfin être entendues c'est peut-être que les pères de famille n'avaient plus autant besoin qu'auparavant de l'Église pour caser leurs enfants surnuméraires, soit que les pratiques de limitation du nombre d'héritiers qui se répandaient rapidement à cette époque%
% [5] 
\footnote{Banalisation des abandons. Recours aux méthodes de prévention des naissances disponibles alors : \emph{coïtus interruptus} certainement, douches vaginales,~etc.} 
aient diminué le nombre des enfants à établir, soit que le développement économique de la fin de l'Ancien Régime ait offert aux cadets de famille des perspectives plus alléchantes que l'entrée en religion ?

 Est-ce pour cela que commençait de paraître scandaleuse%
% [6] 
\footnote{Le roman \emph{La religieuse} de \fsc{DIDEROT} paraît en 1782 : il exprime cet état d'esprit. Cette \emph{effrayante satire des couvents}, selon l'auteur lui-même, raconte l'histoire d'une fille contrainte par sa famille à prendre le voile et à vivre dans un couvent aux modes de vie et de penser terrifiants. Une de ses propres sœurs était religieuse.} 
l'idée qu'on puisse à vingt ans aliéner sa liberté de manière définitive en prononçant des vœux perpétuels, sans tenir compte des évolutions psychologiques et intellectuelles qu'une vie peut entraîner ? Les \emph{philosophes} n'épargnaient d'ailleurs pas davantage l'indissolubilité du mariage, qui leur paraissait une oppression du même ordre. Mais dans le cas des religieux cela leur paraissait d'autant plus monstrueux, que depuis la fin de l'Antiquité le droit privait ceux-ci de tous leurs droits familiaux et en faisait des morts civils. 

 D'autre part il est symptomatique que le vœu d'obéissance à une autorité étrangère (le Pape) ait été l'un des griefs principaux formulés contre les jésuites. À une période où il s'est peut-être créé plus d'internats éducatifs qu'à aucune autre, la société civile (par l'intermédiaire du Parlement de Paris) entendait exercer un contrôle sur les contenus de l'enseignement, et ne plus laisser les mains libres à des corps de spécialistes comme les jésuites, qui se situeraient au-dessus de la nation ou de l'État (ou du roi), même au nom d'une légitimité religieuse supranationale. On reprochait aussi au mode de vie des religieux d'être inadapté aux institutions d'éducation ou d'assistance. À partir de leur expulsion en 1761, les nombreux collèges des jésuites ont été repris en main par les représentants du pouvoir civil, qui les ont confié aux ecclésiastiques de leur choix. Le mandat qu'ils entendaient donner aux nouveaux professeurs des collèges était de préparer les jeunes « de famille » à vivre dans le siècle, non à l'ombre des cloîtres. Il ne s'agissait plus d'en faire des clercs. Ils entendaient que le monde contemporain, avec ses réalités matérielles et ses techniques profanes, soit introduit dans les internats éducatifs%
% [7]
\footnote{C'est à cette époque que sont créées les premières grandes écoles, puisque l'université refusait de développer en son sein les enseignements techniques de niveau supérieur dont la société d'alors commençait à avoir un besoin impérieux (exemples : Chirurgie, Ponts et Chaussées, Mines,~etc.).}%
.

 L'aspiration des auteurs de la fin du \siecle{18} à la maîtrise de soi, comme leur amour de l'ordre, étaient aussi grands que ceux de leurs prédécesseurs : ce qui changeait, c'est qu'ils fondaient leurs projets sur une représentation idéalisée des républiques antiques et non plus sur saint Augustin. Ce qui changeait, c'était la valorisation du modèle militaire%
% [8] 
\footnote{Les premières écoles militaires datent aussi de cette période. Les casernes deviennent des lieux de dressage rationnel (cf. le « \emph{drill} » prussien) qui transforme une « piétaille » indisciplinée et timorée en une machine de guerre efficace.} 
aux dépens du modèle monastique. Il ne s'agissait plus de former des âmes pour le service de Dieu mais de former des corps et des caractères pour la Cité, l'État, la Nation. Ils parlaient de vertu « spartiate », « romaine », « républicaine » ou « citoyenne » ...

 Le premier juin 1739, sur requête du Parlement de Metz, Louis~XV publiait un édit spécial à destination de la Lorraine {[...] \emph{pour empêcher que par des voies indirectes on ne fasse de nouveaux établissements sans autorisation, soit pour empêcher les communautés autorisées de faire sans permission de nouveaux acquêts}}. Il interdisait qu'aucune donation, qu'aucun legs et qu'aucune rente ne soit plus acceptés à l'avenir par une communauté religieuse sans permission, et qu'aucune acquisition ne soit faite par une communauté, qu'aucune communauté nouvelle ne soit créée sans une enquête préalable \emph{de commodo et incommodo}. Il défendait à quiconque de se faire prête-nom pour des religieux. Il donnait droit aux gens lésés par des dons ou des ventes non autorisés de réclamer leurs biens aux communautés concernées,~etc. En 1749, Louis~XV étendait par édit ces règles à l'ensemble du royaume : \emph{il ne sera fait aucun nouvel établissement, chapitre, séminaire, communauté religieuse quelconque même sous prétexte d'hospice, de quelque qualité que ce soit, sans permission expresse par lettres patentes enregistrées}. Il s'agissait de distinguer entre les fondations d'intérêt public (dont les hôpitaux et les hospices, mais seulement au cas par cas, et s'ils étaient approuvés par les autorités civiles) et les autres (couvents et monastères divers). 

 Les autorités civiles contestaient à l'Église le droit de s'opposer à l'autorité de l'État, garant de \emph{l'intérêt général} et de la tranquillité publique, ainsi que le montre l'arrêt du conseil du 24 mai 1766 : [considérant que] \emph{s'il appartient à l'autorité spirituelle d'examiner et d'approuver les instituts religieux dans l'ordre de la religion ; si elle seule peut consacrer les vœux, en dispenser ou en relever dans le for intérieur, la puissance temporelle a le droit de déclarer abusifs et non véritablement émis les vœux qui n'auraient pas été formés suivant les règles canoniques et civiles, comme aussi d'admettre ou de ne pas admettre les ordres religieux suivant qu'ils peuvent être utiles ou dangereux dans l'État, même d'exclure ceux qui s'y seraient établis contre lesdites règles ou qui deviendraient nuisibles à la tranquillité publique,~etc.}

 Dans le même esprit, de 1766 à 1784 la \emph{Commission royale des Réguliers} supprimait 458 monastères et couvents%
% [9] 
\footnote{François \fsc{BLUCHE}, idem, p. 68.} 
sans en référer à Rome et en dépit de l'opposition d'une grande partie du clergé français. Sur son instigation, un édit royal de 1773 supprimait \emph{l'exemption} qui depuis le haut Moyen Âge interdisait aux évêques (nommés par le Roi, contrôlés par lui, et qui donc le représentaient) d'exercer leur autorité sur tous les couvents et monastères de leurs diocèses. Il leur confiait la mission de les contrôler. 

 En 1768 la Commission royale des Réguliers repoussait à 21 ans (18 ans pour les filles) l'âge à partir duquel les postulants avaient le droit de prononcer des vœux solennels. L'objectif était de protéger la liberté des jeunes gens contre toutes les formes d'oppression : celle des pères était sans doute visée au moins autant que celle des couvents qui n'étaient généralement pas à un ou deux ans près, et qui avaient peu à gagner à s'encombrer de membres malheureux, aigris ou révoltés.

 Par ailleurs à la fin de l'ancien régime la part d'héritage donnée aux postulants religieux a été limitée par les autorités civiles pour empêcher des surenchères, des luttes de prestige, et pour ne pas immobiliser trop d'argent dans des institutions en principe vouées à la pauvreté. Les familles pouvaient se borner à payer une pension viagère pour leur fils ou fille, sans qu'il soit plus question d'accroître définitivement le capital du couvent.

 Cela étant dit les relations entre les autorités civiles et le monde religieux ne se résumaient pas à ces antagonismes. C'était bien plus complexe, et les religieux ne venaient pas d'un autre monde, au contraire ils étaient en majorité issus des couches aisées et bien insérées de la population. En 1761, c'est à des ecclésiastiques séculiers que les parlementaires confient les collèges dont ils venaient d'expulser les jésuites. Ils partageaient la même sympathie pour les thèses jansénistes et gallicanes. Ces clercs, soumis aux évêques, soumis eux-mêmes au roi, avaient toute leur confiance. De la même façon {Tenon} exprimait en 1788 toute son estime pour les religieuses hospitalières que sa carrière l'amenait à côtoyer jour après jour dans les hôpitaux de Paris. Alors qu'il participait de sa place au mouvement de réflexion, de rationalisation et de modernisation des Lumières, il n'imaginait pas un instant se passer de leurs services. 


\section{Contestation de la puissance des pères}

 En conséquence du retour au droit romain à partir du \siecle{12}, la puissance paternelle avait été restaurée dans toute sa force à partir de la fin du Moyen Âge, dans les pays de droit écrit surtout, mais aussi dans le reste de la France%
% [10]
\footnote{Cf. \emph{Histoire des pères et de la paternité}, Collectif, 1990, édition 2000.}% 
. Aussi longtemps que le père vivait il conservait sa puissance de décision dans les domaines essentiels de la vie de ses enfants, même devenus adultes (mariage, achats et ventes de pièces du patrimoine...). 

 Mais le roi soleil, qui avait donné un éclat incomparable à la monarchie de droit divin, avait également révoqué l'édit de Nantes, et c'est lui qui avait ordonné les persécutions qui s'en étaient suivies pendant des générations contre les membres de la « {religion prétendue réformée} » (RPR). C'est donc lui qui avait attaqué la fonction paternelle dans la personne de ceux qu'il avait disqualifiés aux yeux de leurs propres enfants en ne reconnaissant pas leurs unions conjugales comme légitimes. Cela faisait de ces enfants des bâtards et les empêchait d'hériter des biens de leurs parents, ce qui gênait beaucoup leur établissement dans la vie. C'est lui qui avait enlevé aux parents réformés le droit d'élever leurs enfants en émancipant ces derniers dès l'âge de raison (7 ans). En faisant tout cela il avait placé les représentations idéologiques et (surtout ?) le pouvoir de l'État au-dessus des pères qu'il prétendait pourtant défendre : il s'était conduit comme un père abusif%
% [11]
\footnote{Maurice \fsc{CAPUL}, \emph{Infirmités et hérésies, les enfants placés sous l'ancien régime} (tome II), 1989, 1990.}% 
.

 Au contraire les auteurs des Lumières voulaient que les pères soient au service de l'épanouissement de leurs enfants, et que leur autorité ne s'exerce que durant le temps où ces derniers étaient incapables de se conduire seuls. Ils voulaient qu'ils appuient leur autorité sur l'affection plutôt que sur la crainte. En 1762 paraissait \emph{L'Émile} et son succès était immédiat, ce qui prouve combien ce roman était en accord avec l'air de son temps. Jean-Jacques \fsc{ROUSSEAU}%
% [12] 
\footnote{Compte tenu de leurs expériences personnelles, ni Rousseau ni Voltaire ni D'Alembert ne pouvaient supporter que puisse exister quelque chose comme un droit supérieur, divin, des pères. De même plusieurs des personnages emblématiques de la Révolution Française ont eu maille à partir avec leur père et avec le droit de correction paternelle tel qu'il pouvait s'exercer sous l'Ancien Régime : Mirabeau, Sade...} 
y proposait une nouvelle image de l'enfance et des rapports parents--enfants. 

 Le regard que les « {philosophes} » portaient sur l'enfance avait-il pour autant radicalement changé ? Même s'ils ne parlaient plus en termes de péché, les pédagogues et philosophes du \siecle{18} ne montraient pas beaucoup plus de vraie confiance en la bonté \emph{naturelle} des enfants que ceux des siècles précédents. Pour \fsc{ROUSSEAU} l'enfant nouveau-né ne portait plus la marque d'un quelconque péché originel, mais il le jugeait sans défenses face aux tentations et trop aisément corruptible par la société, c'est-à-dire d'abord par son entourage immédiat. Cela se traduisait dans \emph{l'Émile} par une pédagogie aussi peu spontanée et naturelle que l'internat le plus contrôlé. Et l'incroyable phobie de la masturbation masculine et féminine qui a régné à partir du \siecle{18} et jusqu'aux années trente du \crmieme{20}, phobie qui a fait déraisonner tant de médecins et de spécialistes de l'éducation, ne se fondait pas sur des raisons religieuses, que ces autorités qui se voulaient scientifiques récusaient et n'auraient jamais reconnues. Par contre elle fournissait aux parents et aux éducateurs tous les arguments légitimes, fondés en raison, pour exercer sur les enfants et adolescents et sur leurs premiers pas dans la découverte de leurs corps et leur sexualité une surveillance intrusive. Il faudra attendre \fsc{FREUD} pour que changent les représentations.


\section{Banalisation des abandons}

 Si l'on décomptait \nombre{312} abandons à Paris en 1670, du temps de Monsieur Vincent de Paul, on en dénombrait \nombre{5842} en 1790, pour environ \nombre{600000} habitants. Ils représentaient 40~\% des naissances parisiennes en 1772, et 33 à 34~\% à la veille de la révolution%
% [13]
\footnote{Pour l'ensemble de la France de 2010 de tels taux donneraient un nombre d'abandon de l'ordre de \nombre{600000} enfants (six cent mille), soit trois fois plus que le nombre actuel d'IVG, et au bas mot 600 fois plus que le nombre d'abandons actuels.}%
. À la fin du \siecle{18} les mœurs ont donc beaucoup changé.

 Pour être juste il faut dire aussi que le nombre des abandons dans les villes était artificiellement gonflé et cela d'autant plus qu'elles étaient grandes. À Paris c'était clairement le cas. On y envoyait des enfants de plusieurs \emph{centaines} de kilomètres à la ronde. Néanmoins la croissance du nombre et du pourcentage des abandons depuis l'époque de Monsieur Vincent était indiscutable et massive. La grande période de l'abandon d'enfant, la période où il a été utilisé de la façon la plus massive et la moins contestée, se situe entre 1760 et 1860%
% [14]
\footnote{... en 1810 : \nombre{55700} abandons sur toute la France ; en 1833 : \nombre{164000} abandons.}%
.

 À cette époque, le poids des interdits religieux avait diminué, au moins dans les villes et dans certaines campagnes, dont celles du bassin parisien, ce qui facilitait à la fois les relations sexuelles hors mariage et le refus des géniteurs masculins de « réparer » en cas de grossesse, comme c'était l'usage jusque là quand un mariage, des vœux religieux ou l'inégalité des conditions ne s'y opposaient pas. La croissance des villes et des fabriques, ateliers, mines et autres industries nouvelles concourait au relâchement de la pression sociale sur les comportements individuels. Le nombre des naissances hors mariage non légitimées par mariage subséquent avait donc augmenté. 

 Mais même dans les couples stables, concubins ou mariés, le recours à l'abandon s'était généralisé. Les scrupules religieux avaient cessé de le freiner. On s'était mis à pratiquer l'abandon des nouveaux-nés dans tous les milieux et de plus en plus souvent à visage découvert. L'abandon était devenu un droit pour tous, exercé sans honte, sans questions, sans enquête, sans formalités, sans poursuites, même en dehors des cas de nécessité vitale, et c'est ce qui était nouveau. C'était devenu un moyen comme un autre de régulation des familles. 

 Le grand public culpabilisait d'autant moins l'abandon qu'il croyait en \emph{la bonté de l'éducation donnée par les hôpitaux}. \fsc{Rousseau} explique dans ses \emph{Confessions} que s'il a abandonné ses cinq enfants, c'est parce que \emph{tout pesé, je choisis pour mes enfants le mieux ou ce que je crus l'être. J'aurais voulu, je voudrais encore avoir été élevé et nourri comme ils l'ont été}. Il croyait que les nouveaux-nés abandonnés avaient de réelles chances de survie. Cette croyance semble à l'époque avoir été très largement partagée. Les administrateurs des hôpitaux étaient presque les seuls à savoir combien la réalité était loin de cet idéal. 

 Les enfants pouvaient être abandonnés à tout âge, et le lien était évident entre le nombre des abandons et les crises économiques. Le manque de ressources des parents, leur maladie, le chômage, ou encore le veuvage, expliquent que des enfants n'étaient pas abandonnés à la naissance, mais après un certain nombre de mois ou d'années. Les mères seules étaient dans une situation économique particulièrement fragile : à travail égal les femmes étaient \emph{beaucoup} moins bien payées que les hommes. 

 Parmi les nouveaux-nés des villes placés en nourrice%
% [15] 
\footnote{La grande majorité des enfants des villes, même non abandonnés, vivaient alors leurs premières années en placement nourricier rural : \emph{1780 : Le lieutenant de police Lenoir constate, non sans amertume, que sur les \nombre{21000} enfants qui naissent annuellement à Paris, \nombre{1000} à peine sont nourris par leur mère. \nombre{1000} autres, des privilégiés, sont allaités par des nourrices à demeure. Tous les autres quittent le sein maternel pour le domicile plus ou moins lointain d'une nourrice mercenaire. Nombreux sont les enfants qui mourront sans avoir jamais connu le regard de leur mère. Ceux qui reviendront quelques années plus tard sous le toit familial découvriront une étrangère : celle qui leur a donné le jour.} (cité par Élisabeth \fsc{BADINTER}). Les propos du lieutenant de police montrent aussi qu'en 1780, les bébés et l'allaitement maternel font désormais partie des sujets de préoccupation légitimes d'un haut fonctionnaire.} 
à la campagne par leurs parents, un certain nombre entraient dans la catégorie des enfants abandonnés si leurs parents ne payaient plus les gages de la nourrice. Lorsque celle-ci n'obtenait pas de réponse à ses réclamations, il allait de soi qu'elle remettait l'enfant à l'hôpital le plus proche : elle n'avait pas reçu de mandat pour faire autre chose, et elle avait besoin de son salaire. 


\section{Valorisation de l'éducation familiale et maternelle}

 Traditionnellement les enfants placés en nourrice par les hôpitaux y revenaient quand leur petite enfance était achevée. Mais dès 1696 le \emph{bureau de l'Hôpital} observait que : [...] \emph{les enfants qu'on ramène à 4 ans à Paris s'accoutument mal à l'air de la capitale et qu'il en meurt beaucoup. On pense qu'il serait bon de les laisser un an de plus à la campagne...} C'est pourquoi au fil du \siecle{18} leur séjour à la campagne s'est prolongé.

 De nouveaux règlements sont édictés en 1761 par l'Hôpital des Enfants Trouvés de Paris%
% [16]
\footnote{Comme celui-ci exerce un rôle de modèle national puisqu'il reçoit le tiers des indigents du royaume, et que le roi suit de très près ce qui se passe dans la ville dont il est le seigneur, ces règlements vont avoir une postérité importante.}% 
. En ce qui concerne ces enfants-là, le placement dans des familles nourricières est désormais mis sur le même pied que l'internat de l'hôpital. On lui reconnaît une valeur au moins égale, au nom de la vie qu'il permet de sauver. 

 À cette époque le sort ordinaire des enfants ordinaires était de commencer très tôt à travailler chez leurs parents ou chez le maître où ceux-ci les avaient placés : souvent dès l'âge de 6 ans. Dans les villes l'entrée au travail attendait dans les meilleurs cas l'issue de la scolarité dans une petite école, scolarité qui durait fort peu de temps. Si le jeune était placé chez un maître, celui-ci avait une large délégation de l'autorité parentale. Il en était ainsi depuis l'Antiquité pour la majorité de la population, pour tous les humbles. Le sort des enfants placés en nourrice paraissait donc naturel et normal, à défaut d'être désirable.

 Il s'agissait d'insérer l'enfant dans un milieu naturel rural ou artisanal, et si possible de lui donner une famille, même si l'adoption demeurait impensable et impossible. Comme l'observera en 1790 \fsc{LA ROCHEFOUCAULT-LIANCOURT}, du Comité de Salut Public, une génération après la prise de décision de ne pas ramener ces jeunes à l'hôpital : \emph{presque tous les enfants conservés par les nourrices sont gardés dans leurs maisons jusqu'à ce qu'ils se marient, y sont traités comme leurs propres enfants, le plus grand nombre tourne bien et ils deviennent de bons habitants des campagnes}. Les enfants placés en nourrice restaient les enfants de l'hôpital, employeur des nourrices, qui avait pleine autorité sur eux, et exerçait l'autorité paternelle jusqu'à leur majorité (25 ans) ou leur mariage (pour les filles). 

 D'emblée ce système a été jugé satisfaisant, mis à part le fait que beaucoup de garçons avaient tendance à s'en aller avant d'avoir eu 25 ans, pour gagner de l'argent. D'autre part un certain nombre de garçons étaient \emph{... renvoyés par le nourricier}. Comme toujours les filles posaient nettement moins de problèmes de discipline que les garçons. 

 Désormais l'objectif était de faire grandir un futur sujet pour le service du roi et de l'État. Il était encore moins question qu'auparavant d'enlever systématiquement leurs enfants aux indigents pour les placer dans un Hôpital coûteux et à la valeur éducative douteuse. À un moment de crise économique (1770) le ministre Turgot a ordonné qu'on mette en place dans chaque paroisse un \emph{bureau d'aumône}, ou \emph{bureau de charité}, à l'intention des pauvres domiciliés, et d'eux seuls, avec pour mission de redistribuer des taxes levées sur les propriétaires aisés de la paroisse%
% [17]
\footnote{Ces institutions n'existaient alors que dans certaines paroisses, même si selon des décisions vieilles de plusieurs siècles, et jamais abrogées, elles auraient dû exister partout. Inutiles durant les périodes de bonne santé économique, elles étaient de celles qu'il fallait refonder constamment.}% 
. Pour secourir les pauvres dociles, les bons pauvres, les femmes seules chargées de famille, les journaliers au chômage, pour protéger les jeunes filles pauvres et en danger de « se perdre » dans la prostitution, sans pour autant les héberger ni les prendre en charge totalement, il a fait ouvrir, ou plutôt rouvrir, des \emph{ateliers de charité}. Les pauvres y travaillaient comme ils l'auraient fait à l'hôpital, et ils continuaient à vivre à leur domicile, dans leur communauté. Ils n'étaient pas déracinés, ni désocialisés, et cela coûtait moins cher. 

 Pour éviter les abandons, les hôpitaux (ou du moins certains hôpitaux) aidaient financièrement les mères indigentes à nourrir chez elles leur propre enfant. Cela ne leur coûtait pas plus cher que de mettre un enfant abandonné en nourrice, mais en termes de survie c'était bien plus efficace : ainsi la mortalité des bébés de Rouen vivant avec leur mère, alors que celles-ci étaient secourues à domicile par l'Hôpital Général (c'étaient donc des indigentes) ne dépassait pas 18,7~\% entre 1777 et 1789%
% [18]
\footnote{Selon Élisabeth \fsc{BADINTER}.}% 
. À la même période ceux des enfants qui étaient mis en nourrice par leurs parents, avec l'aide matérielle du même Hôpital (c'étaient donc des indigents eux aussi), subissaient une mortalité de 38,1~\%. Quant à ceux qui étaient abandonnés à ce même hôpital, il en mourait plus de 90~\%. À Lyon il en était de même à la même période : les bébés nourris par les mères qui ont été secourues à domicile par le bureau de bienfaisance maternelle n'ont subi de 1785 à 1788 qu'un taux de mortalité de 16~\% avant l'âge d'un an. Ces taux étaient très bons pour l'époque, même comparés à ceux des familles non indigentes dont les mères nourrissaient elles-mêmes : il est vrai qu'il s'agissait d'enfants uniques (sans quoi leurs mères n'étaient plus jugées dignes, « méritantes », de bénéficier d'une telle mesure), qu'elles avaient voulu les garder et les élever, et qu'elles avaient le temps de s'en occuper.

 Des recherches véritablement scientifiques ont été menées afin de diminuer la mortalité infantile en collectivité. À partir de 1784 une expérience a été conduite dans l'une des salles de la Couche de Paris sous l'impulsion et le contrôle du corps médical. Elle avait pour principe d'augmenter le taux d'encadrement et l'intensité des relations des bébés avec les soignants. Cette expérience s'est poursuivie pendant 4 ans (1784-1788) sous le contrôle de l'Académie de Médecine. Elle a conduit à une baisse significative du taux de la mortalité. Aussi avec l'approbation de la même faculté de médecine (1788) ces pratiques ont-elles connu un début de généralisation timide : un tel dispositif était en effet fort coûteux, et la Révolution a suspendu sa mise en œuvre.

 Voici ce qu'écrivait en l'an XI \fsc{CAMUS}, membre du Conseil qui avait dans ses attributions les maisons d'enfants trouvés, dans son \emph{Rapport au Conseil général des hospices sur les hôpitaux et hospices, les secours à domicile, la direction des nourrices} : \emph{Peut-être est-il beaucoup plus difficile de suppléer aux soins de la mère et de la nourrice qu'à leur lait. On est assez avancé dans les connaissances chimiques pour composer une boisson qui ait la qualité du lait de femme, même avec les variations que le lait éprouve pendant la durée de l'allaitement%
% [19] 
\footnote{En réalité à cette date aucun essai n'avait réussi. Tout au plus savait-on à peu près compléter un allaitement insuffisant par des bouillies, et cela ressortait de l'art des mères et des nourrices plus que de l'expertise des médecins.} 
; mais ces tendres soins d'une femme pour l'enfant auquel elle donne une partie de sa substance, cette gestation entre les bras, ces embrassements continus, ces baisers fréquents : en un mot, cette espèce d'incubation qui doit suivre la sortie du sein de la mère, voilà ce qu'on n'obtient ni avec des combinaisons chimiques, ni avec des règlements, ni avec des gages.}%
%[20] 
\footnote{Cité par \fsc{DUPOUX}, idem, p. 136 et 181, 1958.} 




% 28.02.2015 :
% haut Moyen Âge
% _, --> ,
% ~etc.
% Antiquité


\chapter{La Révolution française et les familles}


 La première démarche des représentants de la nation réunis en 1789 a été de rédiger une \emph{Déclaration des droits de l'homme et du citoyen}. Ils ont commencé par refuser les privilèges \emph{et les désavantages} fondés sur les circonstances de la conception, de la naissance, sur le statut des parents, sur la religion ou l'absence de religion. Selon l'article~1 de la Déclaration de 1789, \emph{les hommes naissent et demeurent libres et égaux en droit. Les distinctions sociales ne peuvent être fondées que sur l'utilité commune}. L'article~6 de la même Déclaration précise que \emph{tous les citoyens étant égaux aux yeux de la loi sont également admissibles à toutes les dignités, places et emplois publics, selon leur capacité et sans autres distinctions que celles de leurs vertus et de leurs talents}. En conséquence, personne ne naît plus esclave ni serf, et aucun nouveau-né ne doit être traité différemment des autres, quoi qu'aient pu commettre ses parents et quelles qu'aient pu être les circonstances de sa naissance : conception hors mariage, adultère, inceste,~etc. 


\section{Limitation de la puissance paternelle}

 La législation révolutionnaire sur la famille a une histoire complexe mais ses acteurs étaient d'accord sur l'essentiel. Ils avaient d'abord en ligne de mire la puissance paternelle\footnote{Cf. \emph{l'Histoire des pères et de la paternité}, voir en particulier le chapitre XI (p. 289 à 328) : « La volonté d'un homme » écrit par Jacques \fsc{MULLIEZ}.}. 
Certains d'entre eux allaient jusqu'à affirmer que les enfants appartenaient à l'état avant d'appartenir à leurs parents. Dans le même ordre d'idée les plus radicaux auraient voulu que tous les jeunes soient pris en charge en internat dès l'âge de 5 ans, pour les préserver de l'influence néfaste de leurs parents, suspects d'être « contre-révolutionnaires » ou « obscurantistes », et pour en faire des citoyens conformes à leurs désirs : répéter en somme pour tous les français ce que Louis~XIV avait cherché en vain à faire avec les protestants et autres « déviants ». En fait ces extrémistes étaient peu nombreux. La majorité tenait à ce que la nation contrôle l'éducation de sa jeunesse, mais elle était ouverte à une large liberté de l'enseignement, et en dépit des péripéties plus ou moins chaotiques vécues par certains l'essentiel du corps enseignant en place à la fin de l'ancien régime (en grande partie constitué d'ecclésiastiques) a formé l'armature des écoles privées ou publiques et des collèges de la Révolution.  

 En 1790 l'Assemblée constituante avait aboli les \emph{lettres de cachet}, dont la plus grande part était octroyée par les autorités civiles dans l'intérêt des chefs de famille. Ceci dit, le père, ou la mère si elle était seule, ou le tuteur (et eux seuls) pouvaient demander à un juge d'emprisonner pour un temps un enfant qui leur créait des "sujets de mécontentement". Mais cet internement n'était renouvelable qu'une seule fois pour un jeune de moins de 16 ans, et un jeune récalcitrant ne pouvait être interné plus d'une année entre 16 et 21 ans pour ce seul motif. D'autre part les parents devaient d'abord obtenir l'accord des \emph{tribunaux de la famille}, qui délibéraient \emph{sous l'autorité d'un juge professionnel}, même si leurs membres étaient recrutés au sein de la famille élargie (et à défaut dans le voisinage immédiat). Ces tribunaux étaient par ailleurs chargés de rétablir la concorde dans les foyers en conflit. 


\section{Privatisation des vœux perpétuels et ouverture du droit au divorce}

 La Constitution de 1791 refusait de reconnaître une valeur juridique aux vœux prononcés par les religieux et fermait tous les couvents. Dans la même logique, les révolutionnaires refusaient de reconnaître tout aspect religieux au mariage et d'y voir autre chose qu'un contrat civil, révocable comme tout autre contrat. En conséquence, la loi du 20 septembre 1792 supprimait la \emph{séparation de corps}, qui sentait trop le catholicisme ...

 ... tandis qu'elle autorisait le divorce par \emph{consentement mutuel} et le divorce \emph{sur demande d'un seul époux}, demande qu'elle autorisait de manière très large et d'abord pour \emph{convenance personnelle}. 

 Le divorce est à ce moment-là devenu aussi facile et plus rapide qu'aujourd'hui (2018). Jusqu'à l'an VII on observe \emph{en ville} un divorce pour 5 mariages ; ensuite un divorce pour 3 mariages. L'inflation du nombre des divorces, non anticipée par la plupart de ceux qui les avaient facilités, a choqué bien des sensibilités. Si les citadins ont recouru très largement du nouveau droit, les habitants des campagnes ne l'ont guère utilisé, d'où l'on conclura sans risque de se tromper qu'on ne divorce pas d'une terre obtenue par mariage aussi facilement que du conjoint qui l'a procurée. C'est une constante de l'histoire : comme le dit le Talmud : \emph{" malheur à celui qui est mal marié et ne peut rembourser la dot de son épouse "}. Ceci dit les campagnes n'avaient pas été gagnées au même degré que les villes par les critiques des philosophes contre l'indissolubilité du mariage, que ce soit pour des raisons religieuses ou parce que celle-ci allait en réalité dans le sens de l'intérêt des familles. 

 


\section{Autonomisation des enfants majeurs}

 La Révolution affirmait l'égalité entre les héritiers et elle la défendait contre tout droit d'aînesse. Pour ce motif et pour empêcher les parents d'exercer une pression indirecte sur les actes de leurs enfants majeurs, la liberté des testateurs était très limitée.
 
 En 1792 l'âge de la majorité a été abaissé de 25 à 21 ans, et surtout les enfants majeurs ont été totalement déliés de la puissance paternelle. Leur capacité juridique a été reconnue comme pleine et entière, qu'il s'agisse d'aliéner leurs biens ou de s'engager dans n'importe quel contrat. Ils pouvaient notamment se marier ou divorcer librement, si nécessaire en passant outre à l'opposition de leurs parents, sans risquer d'être déshérités pour autant. 

 


\section{Nul ne peut être parent contre son gré}

 Les révolutionnaires assimilaient les enfants illégitimes aux enfants légitimes, qu'ils soient adultérins, incestueux ou nés hors mariage de personnes libres de tout engagement ou empêchement, \emph{à la condition expresse qu'ils aient été reconnus par au moins l'un de leurs deux géniteurs}. 
 Mais ils affirmaient aussi que \emph{nul ne peut être parent contre son gré}. Nul, ni femme ni homme, ne devait être contraint à reconnaître un enfant pour sien. L'enfant ne devait être reconnu que volontairement et librement. En cas de naissance hors mariage, une décision libre de chacun des géniteurs était nécessaire pour qu'il devienne parent. La seule exception était le viol avec enlèvement, auquel cas le coupable perdait son droit de ne pas reconnaître l'enfant et de ne pas assumer de responsabilité financière vis à vis de lui. 
De là découlait que dans le temps même où les révolutionnaires accordaient aux enfants illégitimes le droit à entrer dans la famille du parent qui les reconnaissait, et d'hériter de lui à égalité avec un enfant légitime, ils écartaient toute possibilité de \emph{recherche de paternité naturelle}, même pour l'allocation de simples \emph{aliments}. 
 
 On peut s'étonner de cette rigueur à l'encontre des enfants nés hors mariage, comparée à la propension des tribunaux d'ancien régime à traiter favorablement toutes les accusations des mères célibataires portées contre leurs amants réels ou supposés : ils ne reconnaissaient pas la liberté de \emph{ne pas reconnaître} l'enfant dont on ne veut pas. Si les tribunaux "d'avant" écoutaient d'une oreille complaisante les accusations des mères célibataires, et si celles-ci avaient objectivement intérêt à accuser des hommes riches, ceux-ci s'en tiraient ordinairement sans trop de dommages. En effet ils ne risquaient pas de voir entrer les enfants concernés dans leur famille, ni de devoir les compter au nombre de leurs héritiers. Même s'ils l'avaient voulu les lois de l'ancien régime le leur interdisaient. Jusqu'à la Révolution une recherche en paternité se soldait dans le pire des cas, que l’homme condamné soit vraiment le géniteur de l’enfant concerné ou qu'il ait échoué à prouver le contraire, par l’obligation de verser des frais de \emph{gésine} puis \emph{d'aliments} proportionnés au statut social de la mère, jusqu'à ce que l'enfant puisse gagner son pain, à douze ans au plus tard. Par contre à partir du moment où les lois nouvelles ne faisaient plus de différence entre les enfants naturels et les enfants légitimes une recherche en paternité naturelle entraînait de tout autres conséquences sur les familles. 

 Dans le même esprit, l'adultère féminin ne posait pas problème aux révolutionnaires tant que la légitimité de l'enfant conçu n'était pas dénoncée par le mari, suivant le vieux principe du droit romain qui voulait que l'époux de la mère était le père de tous les enfants de celle-ci nés pendant leur union. Le mari d'une femme adultère avait le pouvoir de dénoncer sa paternité et il était le seul dans ce cas : ni l'épouse, ni son amant ne pouvaient le faire, même s'ils le voulaient. 

 Dans l'esprit des hommes de la Révolution, la contrepartie du droit de ne pas être parent si et de refuser de reconnaître un enfant né de ses œuvres, était l'ouverture aux enfants non reconnus d'un large droit à être adoptés dès leur plus jeune âge. En donnant aux personnes sans enfants le droit de se faire ainsi des successeurs et des héritiers ils espéraient résoudre le problème posé par le grand nombre d'enfants abandonnés de cette période. 

 


\section{Démembrement de l'Hôpital Général}

 Le 19 avril 1801 (an IX), le \emph{Conseil général des hospices} réorganisait administrativement les établissements hospitaliers. Les différentes fonctions assurées indistinctement jusque là étaient administrativement démembrées et réparties entre des institutions indépendantes et spécialisées%
% [4]
\footnote{En fait toutes les populations contrôlées par l'ancien Hôpital Général (hôpitaux, hospices, prisons, nourrices et même services d'assistance au domicile) et toutes les institutions nées de son éclatement, sont restées jusqu'au début du \siecle{20} sous la tutelle du Ministère de l'Intérieur, chargé par ailleurs de la police et des cultes. Les personnes incarcérées, quel que soit leur âge, dépendront du ministère de l'intérieur jusqu'en 1911, date où elles passeront sous l'autorité du ministère de la Justice.}%
, dont les ressources allaient être de plus en plus exclusivement assurées par l'impôt. Le classement des établissements établi en 1801 est à l'origine de celui qui a cours aujourd'hui, même si à l'intérieur de chacune de ses catégories de profondes évolutions ont depuis lors transformé le traitement des problèmes des personnes prises en charge :
%\begin{enumerate}[leftmargin=*,itemsep=0pt]
\begin{itemize}
%1) 
\item pour les prévenus, pour les hommes condamnés à de courtes peines, et pour toutes les femmes condamnées : les prisons ; 
\item pour les hommes condamnés à de longues peines, les bagnes ;
\item pour les malades mentaux (c'est-à-dire ceux désignés comme tels par leurs familles ou les autorités civiles, après confirmation du diagnostic par les médecins aliénistes) : les hôpitaux psychiatriques (dont l'architecture, l'organisation interne et le personnel présentaient une grande proximité avec ceux des prisons) ;
\item pour les malades pauvres : les hôpitaux\footnote{Le même jour, le 16 avril 1801, le Conseil général des hospices supprimait les lits de plus d'une personne. C'était la survivance d'un archaïsme que Tenon tenait dès 1788 pour une aberration nuisible à la cure des malades. Mais à Paris cette situation perdurait partout, et surtout dans les sections des indigents des hôpitaux généraux.}, réservés aux malades pauvres ou sans famille, aux personnes en voyage loin de chez elles, et aux militaires éloignés de toute infirmerie de garnison. Les riches préféraient se faire soigner à domicile ou dans des "cliniques" privées payantes, comme toujours ;
\item pour les personnes incapables de gagner leur vie : vieillards sans ressources, infirmes et enfants non abandonnés dont on connaît les parents (qu'ils soient ou non vivants) : les hospices ;
\item pour tous les nourrissons, d'une part, et pour les enfants abandonnés (pupilles) jusqu'à leur majorité d'autre part : placements en nourrice sous l'autorité des hospices mentionnés ci-dessus ;
\item les mineurs vagabonds ou délinquants étaient emprisonnés en tant que délinquants comme les adultes, et avec les adultes. 
 \end{itemize}

 




% Le 18.03.2015 :
% Moyen Âge
% Antiquité
% droit-Droit
% Le 24.02.2015 :
% ~etc.
% Moyen-Âge
%~\%



\chapter{La famille du Code Napoléon}

\section{Suppression du divorce}
 En 1802 Napoléon signait avec le Pape un Concordat qui reconnaissait la religion catholique comme la \emph{religion de la majorité des Français}. À ce titre, il reconnaissait à cette religion une vocation à être l'une des sources du droit et à l'État le droit de nommer les évêques. Il prenait acte de l'expropriation par les révolutionnaires des propriétés de l'Église. En contrepartie l'État s'engageait à salarier et loger les ministres du culte (avant la révolution c'était l'une des revendications du bas clergé). Au nom de l'égalité, l'État reconnaissait aussi les églises protestantes et le judaïsme, et salariait également les pasteurs protestants et les rabbins.
 
 Napoléon a fait réorganiser le droit civil par les professeurs de Droit les plus réputés en faisant une synthèse de la législation révolutionnaire et du droit coutumier de l'ancien régime.  Le \emph{Code Civil} (ou \emph{Code Napoléon}) paraît en 1804. Que ce soit en souci de conformité avec le droit Canon, dans le cadre du Concordat, ou en réaction aux innovations révolutionnaires, qui n'avaient eu guère de succès en dehors de la population des villes, très minoritaire, il restaure presque intégralement la \emph{famille constantinienne}, fondée sur l'union monogame et (quasi) indissoluble d'un homme et d'une femme, et qui exclut de l'héritage tous les enfants adultérins. Il la défend contre tous les courants centrifuges qui pourraient menacer son unité et donc la fragiliser. 
 
 La société, ou du moins une grande partie de celle-ci, se sent attaquée lorsqu'un mariage est menacé, comme du temps d'Auguste, et elle réprouve le divorce, même quand elle le permet. En 1804 le Code Civil supprime le divorce pour \emph{incompatibilité d'humeur} et pose tant de conditions au divorce par \emph{consentement mutuel} qu'il devient très rare, environ cinquante par an, alors qu'en l'an~VII de la Révolution le nombre des divorces dans les villes était le tiers de celui des mariages. Il comprend le divorce comme la sanction d'une faute : adultère du partenaire, condamnation à une peine infamante, excès, sévices ou injures graves... Il restaure la \emph{séparation de corps}, qui interdit le remariage. D'autre part les époux divorcés n'ont plus le droit de se remarier l'un avec l'autre (comme dans le droit juif). Enfin l'époux condamné pour adultère se voit interdire à vie d'épouser son ou sa complice. On reconnaît là une règle de droit instituée par Constantin et ses successeurs immédiats, et jamais abrogée ensuite jusqu'à la Révolution. 



 La Restauration poursuit dans le même sens et supprime le droit au remariage après divorce dès 1816
\footnote{Sous l'ancien régime chacun était soumis dans le domaine familial au droit de sa propre religion, ce qui permettait aux protestants français (à partir du moment où leur religion était tolérée), aux protestants étrangers résidents permanents (en tout temps), et aux juifs, de rompre une union selon leurs propres règles et d'en contracter légalement une nouvelle : les "sans-religion" n'avaient pas de place dans ce modèle. En raison du principe de l'universalité de la loi institué par la Révolution, et donc de l'impossibilité de reconnaître des droits particuliers à certains citoyens, le divorce a été interdit à tous les français par le Code Napoléon quelle que soit leur religion ou leur absence de religion.}. Aux époux mal mariés il ne restait plus que la séparation, comme avant la Révolution. Pour l'obtenir, le demandeur\footnote{qui était le plus souvent une demanderesse : ce n'est pas d'aujourd'hui que les femmes demandent le divorce plus souvent que les hommes.} devait invoquer la faute de son conjoint. L'accord des deux partenaires ne suffisait pas. Les femmes accusaient ordinairement leurs maris de les maltraiter, physiquement ou moralement. Les hommes invoquaient le plus souvent l'adultère de leur épouse. Aux yeux de la loi les infidélités masculines n'étaient des injures graves que s'ils introduisaient leurs maîtresses sous le toit conjugal. 

 Selon une tradition française ancienne, la garde des enfants était ordinairement remise quel que soit leur âge à celui des parents qui était jugé non coupable : pour ce motif elle était le plus souvent confiée aux mères (en Angleterre au contraire les enfants ont été assez systématiquement remis à leur père jusqu'au milieu du \siecle{19}, comme sous l'empire romain). Dans tous les cas de figure, c'est le père qui devait subvenir aux besoins des enfants. Comme toujours depuis l'Antiquité la condamnation d'un conjoint à une peine infamante permettait au conjoint innocent d'obtenir la séparation et la garde des enfants.

\section{Restauration de l'autorité des pères}

 Les familles du Code Napoléon étaient presque aussi patriarcales que celles de l'ancien régime. Certes les points de friction les plus irritants de l'Ancien Régime avaient disparu : les jeunes gens étaient libres de leurs choix professionnels à partir de 21 ans, et il n'était guère possible de les déshériter... 
 
 ...mais beaucoup d'entre eux travaillaient sous l'autorité de leur père dans son entreprise, son atelier, sa boutique ou son exploitation agricole. Ils devaient attendre son décès ou son retrait volontaire, ce qui les maintenait dans sa dépendance jusque dans le choix de leur conjoint, choix d'autant plus contrôlé qu'il restait souvent l'une des clés de leur établissement professionnel. Ils avaient besoin de l'accord de leurs parents pour se marier, quel que soit leur âge, et ne pouvaient passer outre à leur refus qu'à certaines conditions. 

 Dans le principe, les droits parentaux (ce qu'on appelait la \emph{puissance paternelle}) étaient reconnus à chacun des deux parents mais au nom de l'unité du commandement jugée nécessaire à toute institution la cellule familiale était confiée à la direction du mari, et les épouses étaient sous la tutelle de leurs maris. Seuls les hommes participaient à la vie publique et pouvaient exercer le pouvoir politique. Tant qu'ils étaient vivants et non déchus de leurs droits pour condamnation infamante ou pour maltraitance grave de leurs enfants, ou pour démence, ou pour absence, c'étaient eux qui exerçaient la puissance paternelle. Ce n'est qu'en cas d'absence, de séparation à leurs torts, de condamnation à une peine infamante, ou de décès, que les mères pouvaient les remplacer, et encore devaient-elles dans certaines circonstances être assistées dans l'exercice de ce droit par un ou plusieurs membres mâles de la famille de leur époux. 
 
 
 Le Code Civil de 1804 donnait au père le droit de faire appel au juge s'il estimait que son autorité n'était pas respectée par son enfant mineur\footnote{Cf. Pascale \fsc{QUINCY-LEFEBVRE}, « Une autorité sous tutelle. La justice et le droit de correction des pères sous la Troisième République », in \emph{Lien social et politiques, Politiques du père,} RIAC, 37, Printemps 1997, p. 99-109.}%
. Il pouvait faire enfermer un de ses enfants de moins de 16 ans pendant un mois (renouvelable s'il le jugeait nécessaire). Le mineur « de famille » interné pour ce motif était traité comme les délinquants du même âge. S'il avait 16 ans et plus (majorité pénale), ou s'il possédait des biens, ou si son père était remarié, il bénéficiait de plus de garanties : le magistrat pouvait accepter, réduire ou refuser la demande d'incarcération. Mais à partir de seize ans celle-ci pouvait durer six mois renouvelables. Même si la loi mettait des limites au droit de correction, le juge n'avait qu'une assez faible liberté d'appréciation : \emph{il se devait} d'apporter son aide au père qui la sollicitait. En cas de décès du père et si la mère ne s'était pas remariée, c'est elle qui exerçait le droit de correction paternelle \emph{avec l'accord des deux plus proches parents du défunt}. 

 Les lettres de cachet avaient certes disparu, mais la Justice restait \emph{tenue} de fournir son aide aux parents qui la lui demandaient pour contenir et corriger les mineurs dont la conduite préoccupait ces derniers. Durant la majeure partie du \siecle{19} elle l'a fait sans trop se poser de questions. Rapportées au nombre de jeunes français, le nombre des mesures administratives de \emph{correction paternelle} était d'ailleurs limité : quelques milliers par an tout au plus. Et il y avait de grandes disparités dans le nombre des recours au juge suivant les régions et suivant les milieux sociaux. Ils étaient beaucoup plus fréquents dans les familles populaires de Paris que partout ailleurs : plus de la moitié des mesures%
%[4]
\footnote{Cf. Pascale \fsc{QUINCY-LEFEBVRE}, article cité, p. 99.}%
. Ailleurs on se débrouillait autrement avec les jeunes « récalcitrants », fugueurs, « paresseux », « libertins », ou « vicieux » (c'était le langage de l'époque). Il était peut-être plus facile d'élever un adolescent à la campagne ou dans des villes beaucoup plus petites, plus paisibles et moins bouillonnantes de sollicitations que Paris ? Et surtout nulle part ailleurs qu'à Paris n'existait la même tradition de proximité, et même de familiarité, avec la personne du souverain, ce qui facilitait les recours. Quant aux bourgeois, de Paris ou d'ailleurs, ils disposaient toujours de toute une gamme d'internats pour mettre un peu de distance entre eux-mêmes et leurs adolescents trop difficiles à élever, et pour offrir à ceux-ci une rencontre avec des éducateurs professionnels en principe plus sereins et moins impliqués.

 Jusqu'à 1882 l'école n'était pas obligatoire et les enfants des classes populaires qui n'y allaient pas commençaient à travailler très tôt. Comme toujours, s'ils ne les employaient pas eux-mêmes leurs pères les plaçaient chez un patron et touchaient leurs gains jusqu'à leur majorité. Les enfants qui avaient été scolarisés étaient mis au travail dès qu'ils avaient fini leur temps d'école (dix/douze ans). C'est dans ce cadre que doivent être interprétés les reproches formulés par les pères. Les mineurs fugueurs ou vagabonds fuyaient parfois moins l'autorité paternelle que l'atelier, la boutique, l'usine ou la maison bourgeoise où ils (elles) avaient été placés.

 De même qu'en 1801 on avait séparé les aliénés des délinquants, sous la Restauration on a séparé autant que faire se pouvait les mineurs, délinquants et vagabonds, des majeurs, pour éviter qu'ils ne soient maltraités ou « pervertis » par eux. On a donc créé des établissements de correction (ou de redressement) spécialisés dans la prise en charge et la rééducation des délinquants et vagabonds mineurs : prisons spéciales vers 1820 (quartiers spécialisés au sein des prisons, la Petite Roquette,~etc.) puis en 1830 pénitenciers pour mineurs, puis à partir de 1840 les \emph{Colonies agricoles et pénitentiaires} privées. Comme aux siècles précédents, depuis le début du \crmieme{19} les fugueurs, les vagabonds, les prostitués et les mendiants de moins de seize ans (mineurs pénaux) étaient arrêtés par la force publique (du moins s'ils causaient du trouble à l'ordre public). À Paris ils étaient conduits à la Préfecture de police. Ceux qui étaient condamnés allaient en prison. Ceux qui étaient acquittés mais que leurs parents ne réclamaient pas étaient déférés à l'autorité judiciaire. Ils allaient en Colonie Pénitentiaire.

 Les jeunes de la correction paternelle étaient placés à la Petite Roquette pour les garçons, au couvent des dames de Saint-Michel pour les filles. En province ils étaient placés en maison d'arrêt avec les détenus de tous les âges (d'où le moindre recours des parents à cette mesure ?). 

 Face à leurs jeunes « indisciplinés », les familles plus aisées recouraient à des internats scolaires comme aux siècles précédents, sans faire appel à la Justice. Ainsi à partir de 1850 à côté de la Colonie agricole et pénitentiaire de Mettray existait une \emph{Maison Paternelle} réputée, créée par le même fondateur que la Colonie Pénitentiaire, et qui fonctionnait toujours vers 1910, jusqu'à ce que le suicide d'un pensionnaire la fasse fermer. Elle était vouée à la correction des fils des familles suffisamment aisées pour en payer la pension.

\section{Interdiction des recherches en paternité}

 Le Code Napoléon (1804) ramenait les « bâtards » non reconnus par mariage subséquent à leur situation antérieure à la Révolution. Il interdisait la reconnaissance des enfants adultérins et incestueux par leurs géniteurs. Même lorsqu'ils avaient été reconnus par ceux-ci il les excluait de leur succession, donc de leur famille, et ne leur reconnaissait que leur droit traditionnel à des legs « alimentaires ». 

 L'adoption était autorisée par le Code Napoléon, mais il s'agissait uniquement de \emph{l'adoption d'adultes majeurs} par des personnes de 50 ans et plus, et non d'enfants mineurs ni de nouveaux-nés. Contrairement aux vœux des révolutionnaires, il ne s'agissait pas en principe de donner une famille à un enfant sans parents, mais de répondre au besoin d'enfant d'une famille en mal d'héritier. Ces adoptions seront rares durant tout le \siecle{19} et jusqu'en 1923 : 114,4 par an en moyenne de 1840 à 1886 pour toute la France, dont 49,4 enfants naturels, reconnus ou non, et 17,76 neveux, nièces et autres alliés
\footnote{« Statistiques des adoptions au \siecle{19} d'après les comptes généraux de l'administration de la justice civile », tableau cité dans \emph{l'avis présenté au nom de la Commission des affaires sociales sur la proposition de loi, adoptée par l'Assemblée Nationale, relative à l'adoption}, \no~298, session ordinaire du Sénat de 1995-1996, annexe au procès-verbal de la séance du 28 mars 1996.}
. L'objectif de ces adoptions était d'abord de transmettre un patrimoine
\footnote{Plus de la moitié des adoptants étaient des rentiers, terme qui désignait notamment les personnes qui s'étaient retirées des affaires après avoir vendu leur entreprise ou leur commerce et qui vivaient des rentes produites par leur capital : c'étaient en somme des retraités.}
. 

 Le nouveau Code durcissait encore l'interdiction révolutionnaire des recherches en paternité naturelle. Pourtant les recherches en maternité naturelle restaient autorisées. Il n'acceptait les recherches en paternité qu'en cas d'enlèvement, mais il les excluait en cas de viol sans enlèvement,~etc. Comme preuves de la paternité il n'acceptait que les aveux formels écrits par le père, ou bien la cohabitation prolongée du père avec la mère, ou encore la \emph{possession d'état}%
%[7]
\footnote{Situation où le mineur est élevé comme son enfant par le père supposé, même s'il ne l'a pas formellement reconnu.}%
,~etc. Dans ces conditions aucun homme ou presque ne pouvait être contraint contre son gré à reconnaître un enfant naturel, ni condamné à verser une pension alimentaire, même quand tout le monde savait parfaitement à quoi s'en tenir sur ses responsabilités. Il ne restait aux enfants naturels qu'à espérer que leur géniteur veuille bien prendre librement l'initiative de les reconnaître. 

 Cela ne pouvait que pousser les mères célibataires à ne pas garder leur enfant et à l'abandonner anonymement, et cette conséquence était acceptée sans état d'âme. Le placement à la campagne des enfants abandonnés était jugé satisfaisant par tout le monde, et le "bâtard" était un obstacle presque insurmontable à la « rédemption » de sa mère par le mariage, sauf s'il était légitimé par le mariage de celle-ci avec son géniteur (à la rigueur avec un autre homme), ce qui restait la solution préférée. 


\section{le mariage des intérêts}
 
 Depuis les temps les plus reculés le premier objectif des jeunes gens raisonnables n'était pas tant de vivre mieux que leurs parents et de s'enrichir, que de se maintenir au même niveau de fortune qu'eux, et de ne pas tomber dans l'indigence, de ne pas être un \emph{déclassé}. Au \siecle{19} (et probablement en était-il de même auparavant) un homme dépensait plus s'il était célibataire que s'il était marié, sauf à employer une bonne "à tout faire" (souvent nommée "gouvernante"). Il était plus rentable pour un homme employé à plein temps d'entretenir une « ménagère » à domicile que de manger tous les jours au restaurant, de faire blanchir son linge ~etc. En dehors de sa dot, très mince ou inexistante dans les milieux populaires, une épouse pouvait fournir beaucoup de services qu'il était coûteux de se procurer sur le marché. Il était donc avantageux pour les jeunes gens dotés d'un emploi et pour les jeunes filles sans fortune de se mettre en ménage, même sans parler de l'échange de prestations sexuelles ou de désir d'enfant. Mais s'il pouvait être préférable de se marier que de ne pas le faire, il fallait aussi éviter de compromettre, par enthousiasme naïf, par imprudence ou par sottise, les bases économiques d'un futur couple et le statut social des enfants à venir. 
 
 Si un trop grand nombre de ces derniers pouvait dégrader irrémédiablement la situation économique d'une famille, le recul de l'âge au mariage permettait aux filles sans fortune de le limiter tout en se constituant une dot par leur travail. C'est pourquoi leur âge moyen au mariage était bien plus élevé que celui des riches héritières. D'autre part à partir d'un certain âge (peu à peu reculé par la loi) et jusqu'à leur majorité les enfants contribuaient à leur tour aux revenus du ménage, et en l'absence de retraite (sauf pour les fonctionnaires) ils étaient une garantie pour les vieux jours de leurs parents (leurs "bâtons de vieillesse"). 

 Le choix du mariage d'inclination, fondé sur l'amour passion et non sur la raison (c'est-à-dire l'intérêt) était la marque des imprévoyant(e)s. Entre mariage d'inclination et concubinage les liens paraissaient évidents. C'est ainsi que s'unissaient ceux qui ne possédaient que leurs bras, les ouvriers, les manœuvres, les valets, les ouvrières et les servantes, etc. Ceux qui se mettaient en ménage avant d'avoir « assis » leur « situation » se condamnaient à « tirer le diable par la queue ». Selon les moralistes, avec lesquels faisaient chorus tous les parents angoissés, la soumission des jeunes imprévoyants à leurs appétits charnels et à leurs affects leur faisait courir le risque de gâcher leur vie, de connaître la misère et de perdre un jour la main sur leurs propres enfants, ainsi qu'il en avait toujours été depuis le début du monde. Ils risquaient en effet de ne pas pouvoir les élever et de devoir les abandonner aux institutions d'assistance. Ils ne pourraient pas les « établir » en leur donnant un capital matériel, ou en finançant leur apprentissage professionnel auprès d'un maître qualifié, ou en les mettant à l'école, même gratuite, puisqu'ils seraient contraints de les placer chez un maître dès que leur âge le permettrait. En cas de chômage et de disette, ils seraient contraints de les envoyer mendier. Ils ne pourraient pas compter sur ces enfants, condamnés à être pauvres à leur tour, pour soutenir leur propre vieillesse. Ils risquaient de finir leurs jours dans la solitude et la misère, affective et matérielle, des hospices.

 Au contraire les parents prévoyants établissaient leurs enfants dans un mariage profitable grâce à leurs économies, à leurs relations et à des stratégies complexes : échanges simultanés et réciproques d'enfants, de terres, de droits d'exploitation, d'entreprises, de commerces, de gérances, d'offices (ministériels), etc\footnote{... sans compter jusqu'à la Révolution l'entrée en religion de ceux qu'ils ne pouvaient ou ne voulaient pas marier de manière conforme à leur milieu social.}... Voilà pourquoi l'accord des parents était demandé depuis la Renaissance pour tout mariage : selon le Code civil de 1804 l'âge à partir duquel le mariage était autorisé était de 15 ans pour les filles et 18 ans pour les garçons, mais l'âge où l'accord des parents cessait d'être exigé était bien plus tardif : 21 ans pour les filles et 25 ans pour les garçons.
 
 Des stratégies familiales si complexes ne pouvaient pas toujours tenir compte des préférences sexuelles ou amoureuses de chacun, et on n'en faisait pas grief aux parents : les femmes s'en consoleraient avec leurs enfants ou la religion, les hommes avec le travail, l'argent, le pouvoir, les prostituées ou les maîtresses (le recours à celles-là étant toujours préférable, du point de vue des épouses, au choix de celles-ci). Les patrimoines étaient verrouillés contre les effets des infidélités des uns et des autres. Jusqu'au début du XXème siècle une épouse ne pouvait introduire d'enfant adultérin dans sa famille que si son mari le voulait bien, mais en ce cas sa paternité sur cet enfant devenait absolument inattaquable : le géniteur n'avait aucun recours. Quant aux enfants illégitimes du mari, ils ne pouvaient pas être légitimés et menacer l'héritage des enfants de l'épouse. 

 Sauf emploi salarié stable et suffisamment rémunérateur (au service de l'état si possible) la pérennité des couples raisonnables était favorisée par la synergie des ressources que leurs familles respectives avaient sagement et laborieusement conjointes (même les militaires de carrière étaient invités à épouser des filles bien dotées). Leurs parents étaient les premiers à tenir fermement à ce qu'ils, et elles plus encore, ne mettent pas ces arrangements en danger par des comportements imprudents ou des passions irréfléchies, d'où leur accord profond sur ce point avec les autorités morales et religieuses de l'époque. Même en cas de dissensions l'intérêt matériel des époux était le plus souvent de rester unis, quitte à accepter des renoncements ou des compromis sur les vrais désirs de chacun, et à cultiver comme un des fondements du vivre-ensemble une dose convenable d'hypocrisie : d'ailleurs, depuis l'antiquité païenne, il était inconvenant d'afficher publiquement une affection trop vive entre les conjoints. 

 Certes, l'impossibilité de placer les préférences individuelles avant tout autre critère pouvait faire souffrir, et l'amour passion comme la liberté de choix du conjoint faisaient rêver. Les œuvres littéraires du passé reflètent la prégnance de ces représentations. Ainsi, pour ne prendre qu'un seul exemple, la plupart des intrigues de Molière reposent sur le refus d'un mariage arrangé. Les romans de Jane Austen sont des archétypes parmi les milliers d'autres fondés sur les "problèmes de coeur" de jeunes gens et de jeunes filles, apparemment libres de leurs choix et en réalité exctrêmement contraints. Les contraintes économiques étaient indépassables, en dépit des souffrances et des renoncements qu'elles entraînaient. Cela n'empêchait pas la société de continuer siècle après siècle à fonctionner sur le même mode. 
 


\chapter{La police des familles au \siecle{19}}


\section{Les enfants trouvés et abandonnés}

 Le Décret Impérial du 19 janvier 1811 concernant les enfants trouvés ou abandonnés et les orphelins pauvres a créé le Service des enfants trouvés et abandonnés, qui sera nommé Service des enfants assistés à partir de 1866. Il ordonnait qu'il y ait \emph{dans chaque arrondissement un hospice où les enfants trouvés pourront être reçus}. Ont donc été désignés des hospices dépositaires où les nouveaux-nés étaient abandonnés et des dépôts départementaux accueillant au sein de ces mêmes hospices les enfants plus grands. 

 L'abandon anonyme, par l'intermédiaire d'un tour, restait la norme, d'autant plus que \emph{nul ne peut être parent contre son gré} et que les recherches en paternité étaient désormais totalement interdites. L'administration ne faisait pas de distinction entre enfants trouvés (ceux dont on ne connaît pas les parents) et enfants abandonnés (ceux dont on les connaît), ni en fonction de leur légitimité réelle ou supposée.

 Le service prenait en charge les enfants sans famille (trouvés ou abandonnés) et les orphelins pauvres. Il assumait également les enfants dont les deux parents étaient prévenus ou condamnés à une peine de prison. Lorsque ceux-ci étaient incarcérés pour plus d'un an leurs enfants étaient définitivement classés parmi les enfants abandonnés, et traités comme tels%
% [1]
\footnote{Leurs parents ne pouvaient plus les reprendre à leur sortie de prison. Ils étaient considérés comme « infâmes » (la prison est une peine infamante), comme de mauvais exemple, et à ce titre incapables de prendre en main l'éducation d'enfants. On a vu que c'est un accessoire (!) de peine hérité de l'antiquité via l'ancien régime : à Rome le condamné à une peine de travail forcé telle que celle des mines devenait en effet \emph{ipso facto} un esclave \emph{(esclave de la peine)} et il perdait de ce fait tous ses droits civiques et parentaux. Il cessait d'être le père légal de ses enfants et le mari de son épouse. Au \siecle{19} si une personne était condamnée à une peine infamante, le divorce était de droit pour son conjoint innocent : il n'avait pas besoin d'autre motif.}% 
. Il ne s'occupait des enfants âgés de plus de douze ans, abandonnés, orphelins, enfants d'indigents ou de vagabonds, que s'il les avait déjà pris en charge avant leurs douze ans. Si un mineur pauvre de plus de douze ans était incapable de travailler, quel qu'en soit le motif, il pouvait être admis à l'hospice mais en ce cas c'était à cause de son incapacité à travailler, et de son manque de ressources, et non parce qu'il était mineur. Le décret de 1811 prévoyait que les enfants incapables d'être placés chez un maître, parce que malades chroniques, estropiés ou infirmes, malades mentaux, ou retardés intellectuels, etc. resteraient à l'hospice [...] \emph{où on les fera travailler autant que faire se pourra}.

 Le Service confiait les pupilles délinquants à la Justice, exerçant ainsi son droit de correction paternelle comme n'importe \emph{bon père de famille} d'alors était tenu de le faire. Le jeune ainsi remis à l'administration compétente allait en prison (en prison pour mineurs s'il en existait une dans le département) ou en \emph{maison de redressement}, ou en \emph{colonie agricole}, etc. 

 Ses règles de fonctionnement reprenaient, pour l'essentiel, le \emph{règlement concernant les enfans-trouvés} promulgué en 1761, un demi-siècle plus tôt, par le \emph{bureau de l'hôpital} de Paris. Alors que le \emph{Code Napoléon} (1804) interdisait les adoptions de mineurs, l'objectif suivant a été inscrit dans tous les règlements du service dès l'an XIII (1805) : \emph{créer une famille nouvelle à la place de celle qui l'abandonne}%
% [2]
\footnote{In l'A.P. en 1900, p. 349.}% 
. Pour les administrateurs et autres responsables du service, un des objectifs du placement était la greffe du pupille dans un nouveau milieu : \emph{on pourrait citer de nombreux exemples d'enfants déjà grands qui, réclamés par leurs parents, refusent absolument de se séparer de leur famille d'adoption. Il n'est pas rare que des nourriciers dotent un enfant assisté, lui réservent une part ou la totalité de leur héritage. Ainsi, l'enfant abandonné, favorisé au point de vue des soins et des précautions matériels, trouve ordinairement chez ses nourriciers une affection, un attachement qui lui rendent véritablement une famille.}%
\footnote{Idem, p. 340.} 

 Il est troublant de voir qu'au \siecle{19} et une partie de \crmieme{20}, les mots \emph{adoption, père, mère} et \emph{parents} sont constamment employés par les professionnels des placements pour parler des nourriciers, là où ces mots ne sont évidemment pas appropriés. Cela a pourtant été une pratique courante et presque une règle pendant au moins un siècle et demi%
% [4]
\footnote{Les textes écrits par \fsc{Soulé} et \fsc{Noël} entre 1955 et 1965 montrent que cet usage extensif du vocabulaire de la parenté a persisté sans aucun changement jusqu'à eux, jusqu'à ce que depuis une génération se généralise l'idée que tout placement est fait pour préparer le retour vers les parents et que le non-retour est un échec.}% 
. Il est vrai que la plupart des parents des enfants placés étaient fermement tenus à l'écart. Cela contribuait au sentiment des pupilles de constituer une famille avec leurs nourriciers, sentiment d'autant plus facile à verbaliser que la loi interdisait leur adoption. Ils étaient placés dans un entre-deux insatisfaisant, mais qui dans une certaine mesure pouvait être protecteur. Pour les nourriciers cet entre-deux était pratique et confortable. Ils n'avaient à se confronter ni à une adoption en bonne et due forme, ni à des parents réels et vivants. 

 C'est en raison de cet état d'esprit que les enfants qui avaient une famille connue, même constituée de parents décédés (orphelins pauvres), n'étaient pas placés en nourrice au-delà de leur petite enfance. Ils étaient voués à vivre en collectivité dans les hospices dépositaires jusqu'à leur placement professionnel. Au fil du siècle de nombreuses voix, laïques comme religieuses, se sont élevées pour réclamer qu'ils soient distingués des enfants trouvés et abandonnés, et pour qu'il leur soit fourni des conditions d'éducation plus soignées. Dès la première moitié du siècle ces protestations vont se traduire par la création de nombreux \emph{orphelinats} privés qui fonctionneront jusqu'au \siecle{20} à la manière des internats scolaires. 

 Durant tout le \siecle{19} les placements en nourrice ont fonctionné à la satisfaction générale. Les enfants placés à la campagne grandissaient et s'adaptaient \emph{comme on l'attendait d'eux} aussi bien sur le plan professionnel que sur le plan social ou scolaire%
% [7]
\footnote{In l'A.P. en 1900, p. 346. Selon DUPOUX, p. 193, en 1898 68,5~\% des enfants de l'Assistance Publique présentés au certificat d'études l'avaient obtenu, à une époque où plus de la moitié d'une classe d'âge sortait du primaire sans ce diplôme, et où il était donc probablement aussi discriminant que le baccalauréat d'aujourd'hui : aujourd'hui 60~\% des jeunes obtiennent un baccalauréat... mais seulement quelques pour cent de ceux que place l'ASE. Pourtant la moitié de ces derniers obtient aux épreuves psychométriques standardisées des résultats qui les situent dans la zone normale et parfois au-dessus.}% 
. Beaucoup s'intégraient définitivement dans les milieux où l'administration les avait transplantés. L'assistance aux enfants de cette période donnait le spectacle d'une espèce d'équilibre : chacun croyait qu'il savait ce qu'il faisait. Les enfants n'étaient là que parce que leurs parents les avaient confiés à l'institution ou parce qu'ils avaient été dans les formes légales qualifiés d'incompétents ou de dangereux. L'institution apportait un secours indispensable à des enfants qui sans cela seraient ou morts ou dans une grande misère. Elle était sûre d'elle et avait une excellente image dans le public, d'autant plus que le taux de survie des enfants abandonnés a augmenté au fil du siècle de manière extrêmement spectaculaire.

 Pour les mères qui abandonnaient leur enfant la question de son adoption par d'autres ne se posait pas. Leur acte était donc loin d'avoir le sens de \emph{consentement à l'adoption} qui lui serait donné aujourd'hui. Même au moment où elles mettaient leur enfant au tour beaucoup de mères ne croyaient pas que la séparation était définitive, et glissaient par exemple des signes de reconnaissance dans ses langes%
% [8]
\footnote{\fsc{DUPOUX}, idem, p. 200. Le procès-verbal d'abandon rédigé lorsqu'il était découvert décrivait l'enfant et toute sa vêture, en notant soigneusement tous les signes distinctifs.}% 
. Même si le lieu du placement leur était caché, même si les contacts directs et les correspondances ont été interdits jusqu'à la fin du \siecle{19}, au fil des années de plus en plus de parents ont cherché à reprendre l'enfant qu'ils avaient abandonné, du moins tant qu'il avait moins de quatre ou cinq ans. Ceci étant dit il ne faut pas oublier que cette démarche demeurait très minoritaire.


\section{La prévention des abandons}

 Le nombre des abandons a cru rapidement au début du \siecle{19}, \nombre{55700} enfants trouvés en 1810, \nombre{164000} en 1833. À l'époque cette augmentation a été imputée à l'anonymat de l'abandon, permis par les tours, plutôt qu'à l'accroissement du nombre des femmes réduites à survivre misérablement dans les conditions du travail salarié d'alors (elles étaient payées \emph{beaucoup} moins que les hommes pour le même travail), ou à l'impossibilité où elles se trouvaient de recourir aux recherches en paternité pour obtenir l'aide des géniteurs de leurs enfants.

 Face à l'augmentation du nombre des abandons, deux réactions ont été opposées : la diminution puis la fermeture des tours, d'une part, et d'autre part l'aide aux mères. Certains départements ont créé des \emph{Secours préventifs contre l'abandon}, organisés sur le modèle de ce qui s'était fait dans quelques villes avant la Révolution, à l'intention des mères seules et sans ressources, célibataires pour la plupart (« filles mères », comme les nommait alors l'administration). Mis en place en 1837 à Paris, ils ont été généralisés à tous les départements à partir de 1850 (arrêté du 23 décembre). Un certain nombre des jeunes enfants ainsi \emph{secourus}, une minorité, étaient placés chez une nourrice choisie par leur mère elle-même, conformément aux pratiques des populations citadines de l'époque. Le but était en ce dernier cas que ces mères puissent exercer une activité professionnelle sans abandonner leur enfant pour autant.

 Les derniers tours encore en fonction ont été fermés en 1861. Désormais les abandons devaient se faire \emph{à bureau ouvert}. L'anonymat de l'abandon restait possible si la personne qui déposait l'enfant refusait de donner l'état-civil de celui-ci et éventuellement le sien, mais dans le cas contraire on notait l'identité des parents, le plus souvent celle de la mère seule. Cette identité est restée secrète, même pour l'enfant devenu adulte, jusqu'aux années 1980. L'abandon à bureau ouvert n'a donc guère changé la situation des pupilles (ce n'était pas son objectif) par contre il a eu un effet visible sur le nombre des abandons, qui a rapidement et fortement décru%
% [9]
\footnote{Les avortements (clandestins) ont-ils augmenté dans les mêmes proportions ? Selon Stanislas \fsc{DU MORIEZ} (\emph{L'avortement}, 1912) et Edmond \fsc{PIERSON} (\emph{La dépopulation de la France}, 1913) de 1826 à 1880 les tribunaux français ont traité \nombre{9300} affaires d'avortements, dont \nombre{1020} ont donné lieu à des sanctions ; de 1881 à 1909 ils ont traité \nombre{14731} affaires d'avortement, dont \nombre{715} ont donné lieu à des peines diverses. Le faible nombre d'affaires par rapport à ce qu'on suppose être le nombre des avortements, et surtout la faiblesse du pourcentage des condamnations effectives (12 et 5~\%) est à noter : les journaux de la Belle Époque sont remplis de petites annonces de sages-femmes proposant leurs services de manière à peine voilée pour supprimer les grossesses indésirables : en dépit de la stigmatisation morale qui frappait les avorteurs et avorteuses, il existait en fait une relative tolérance qui disparaîtra après la première guerre mondiale. Quant au nombre effectif d'avortements provoqués, il était à cette époque estimé par les auteurs ci-dessus au minimum à \nombre{200000} par an, et au maximum à \nombre{1000000} (un million) et plus. Autrement dit, ils n'en savaient presque rien, ce qui n'est pas étonnant pour une pratique clandestine et qui demande peu de moyens techniques.}% 


\section{Des clivages idéologiques durables autour des familles}

 Les secours préventifs contre l'abandon ont provoqué de très virulents débats parlementaires. Ceux qui les critiquaient pensaient que rien ne vaut un couple conjugal légitime (rural si possible). Ils pensaient que l'aide destinée à l'enfant était le plus souvent détournée de son objet pour \emph{ alimenter la débauche} des mères et de leurs amants, qu'elle n'assurait pas l'avenir d'enfants \emph{sans pères} et donc \emph{sans repères}, et qu'au contraire elle entretenait \emph{une masse d'enfants vagabonds indisciplinés, qui encombrent les cités, constituent un péril social, et dont il faut à grands frais punir les méfaits ou réprimer l'audace toujours croissante}%
% [10]
\footnote{de~\fsc{GERANDO}, cité par \fsc{BIANCO} et \fsc{LAMY}, 1980.}% 
. Avant d'encourager les \emph{filles mères} à garder leurs enfants illégitimes il fallait donc moraliser leur vie et leur donner un « tuteur » en les mariant à un homme travailleur, sobre et économe. Il convenait de faire passer les couples de concubins devant monsieur le maire, et si possible devant monsieur le curé. Lorsque cela n'était pas possible il était préférable pour l'ordre public et pour l'État de confier les bébés sans père aux placements nourriciers ruraux, qui aux yeux des participants de cette sensibilité étaient parfaitement au point : [...] \emph{le service des enfants assistés fournit au contraire \emph{[...]} une race honnête, vigoureuse, fixée à la campagne, fournissant un contingent peu élevé de criminalité}%
%[11]
\footnote{Idem.}%
.

 Ceux qui défendaient les secours préventifs contre l'abandon étaient plus sensibles à la détresse des mères et aux risques pour l'enfant qu'entraîne la coupure d'avec sa famille, même réduite à une seule personne. Ils ne pensaient pas que l'absence d'un époux rendait les mères incompétentes. Ils ne pensaient pas que le soutien d'un père soit irremplaçable. Ils ne pensaient pas qu'un enfant sans père était condamné à l'inadaptation et à la délinquance. Ils croyaient que la société pouvait fournir une aide suffisante aux mères et aux enfants pour que cela n'arrive pas. Ce n'était pas un débat nouveau, puisqu'on l'observait dès le \siecle{18}. Ceux qui n'attachent pas une importance déterminante au mariage et à la naissance légitime et qui regardent la paternité avec une certaine distance ont tendance à sympathiser avec les idées d'égalité et d'autonomie individuelle promues par les Lumières et la Révolution. Même sans aller jusqu'à désigner l'État comme la seule instance qui ait une autorité légitime sur les enfants, ils sont plus ouverts que les autres à l'idée qu'il puisse légitimement exercer un contrôle sur tous les parents. Ceux qui au contraire tiennent pour essentiel que l'enfant grandisse dans une famille fortement structurée autour d'un couple mixte, avec des rôles différenciés (père, mère, enfants), ont plus tendance à n'être que peu ou pas du tout séduits par les discours révolutionnaires et considèrent plus facilement comme abusif que l'État cherche à s'immiscer dans la relation entre les parents et les enfants. Ils sont aussi plus enclins que les autres à supporter que la loi fasse des différences entre les enfants légitimes et les autres au nom de la défense de l'institution familiale. Il y a là une ligne de partage que l'on retrouve aujourd'hui encore.
 



% Le 18 mars 2015 :
% Antiquité
% Moyen Âge
% ~etc.


\chapter{III\ieme{} et IV\ieme{} républiques}


 Depuis la fin du \siecle{19} il s'est produit beaucoup d'évènements qui ont eu un impact décisif, directement ou indirectement, sur les familles et pour commencer voici les décisions essentielles :
\footnote{Sources :
\\Ouvrage collectif de l'Administration générale de l'Assistance Publique, \emph{L'Assistance Publique en 1900}, écrit à l'occasion de l'Exposition Universelle de Paris de 1900, composé et imprimé par les pupilles de la Seine de l'école d'Alembert à Montévrain, Paris, 1900. Consultable au Musée Social, Paris, VIIème.
\\Collectif, sous la direction de Michel \fsc{CHAUVIERE}, Pierre \fsc{LENOËL}, Eric \fsc{PIERRE}, \emph{Protéger l'enfant, Raison juridique et pratiques socio-judiciaires (\crmieme{19} et \siecle{20}{}s)}, Presses Universitaires de Rennes, 1996.
\\Collectif, sous la direction de Jean \fsc{DELUMEAU} et Daniel \fsc{ROCHE}, \emph{Histoire des pères et de la paternité}, Larousse, 1990, édition 2000.
\\Collectif, sous la direction de Jean \fsc{IMBERT}, \emph{Histoire des hôpitaux en France}, Privat, 1982, 559 p.
\\Collectif, sous la direction d'Alain \fsc{BURGUIERE}, Christine \fsc{KLAPISH-ZUBER}, Martine \fsc{SEGALEN}, Françoise \fsc{ZONABEND}, \emph{Histoire de la famille, 3, Le choc des modernités}, Armand Colin Éditeur, Paris, 1986.
\\\fsc{BROUSOLLE} Paul, \emph{Délinquance et déviance, brève histoire de leurs approches psychiatriques}, Privat, Toulouse, 1978.
\\\fsc{CUBERO} José, \emph{Histoire du vagabondage du Moyen Âge à nos jours}, Imago, Paris, 1998.
\\\fsc{DONZELOT} Jacques, \emph{L'invention du social, essai sur le déclin des passions politiques}, Seuil, Paris, 1994.
\\\fsc{DUPOUX} Albert, \emph{Sur les pas de Monsieur Vincent, 300 ans d'histoire parisienne de l'enfance abandonnée}, Édité par la Revue de l'Assistance Publique, Paris, 1958.
\\Patrice \fsc{PINELL}, Markos \fsc{ZAFIROPOULOS}, \emph{Un siècle d'échecs scolaires (1882-1982}), Les éditions ouvrières, Paris, 1983.}% 
 :

\begin{description}
\item[1880] Création d'un enseignement secondaire public pour les filles, calqué sur le modèle de celui des garçons. 

\item[1881] Obligation pour chaque commune de mettre à la disposition de ses administrés une école gratuite et laïque.

%1881 : 
Création du \emph{Service des enfants moralement abandonnés}. Il a pour objet de recevoir les jeunes de 12 à 16 ans sans support familial, pénalement mineurs, non secourus puisque le service ne recevait pas de nouveaux entrants après l'âge de douze ans, et n'ayant que la mendicité et le vol pour subsister : mineurs arrêtés pour {\emph{vagabondage et autres menus délits, et aussi ceux que leurs parents se montraient incapables de diriger}}.

\item[1882] Obligation scolaire pour les garçons et filles de 6 à 12 ans. Avant leurs 12 ans (11 ans s'ils ont le certificat d'études) il est interdit aux pères de placer leurs enfants chez un employeur, ou de les employer eux-mêmes à plein-temps.

\item[1884] Réouverture du droit au divorce (uniquement pour faute, comme en 1804).

\item[1886] Laïcisation du personnel enseignant des écoles publiques. 

\item[1889] La loi du 24 juillet {\emph{sur la protection judiciaire des enfants maltraités et moralement abandonnés}} précise les conditions de la déchéance de la puissance paternelle. Cette déchéance totale est prononcée par un Juge :
\begin{enumerate}[leftmargin=*,itemsep=0pt]
% 1°)
\item facultativement pour inconduite des parents,
% 2°)
\item facultativement en cas de mauvais traitements ou de délaissement de l'enfant,
% 3°)
\item et de plein droit dans le cas de certaines condamnations infamantes (ce qui était le cas depuis l'Antiquité).
\end{enumerate}

% 1889 : 
Cette loi du 24 juillet 1889 %{\emph{sur la protection judiciaire des enfants maltraités et moralement abandonnés}} 
confie à l'administration (c'est-à-dire à l'Assistance Publique) la tutelle des enfants maltraités, victimes de crimes, ou de délits, ou délaissés. Le service les prend en charge même s'ils sont âgés de plus de 12 ans à leur entrée. Ces enfants sont retirés autoritairement à leurs parents et deviennent des pupilles comme les autres. Ils sont traités à l'instar des autres enfants du service. Quel que soit leur âge, autant que faire se peut ils seront placés en nourrice, pour de longues durées, et dans tous les cas ils seront coupés de leurs parents déchus.

\item[1893] Les femmes séparées de corps ont la pleine capacité civile : elles récupèrent les droits qu'elles avaient quand elles étaient célibataires.

%\item[$\!\!\!$] À partir de \textbf{1896} 
\item[1896] {\emph{Les familles indigentes mises devant la nécessité d'abandonner \emph{[sont]} autorisées après enquête à correspondre directement avec enfants et nourriciers}}. D'autre part les {\emph{enfants de parents internés \emph{[sont désormais]} considérés comme n'ayant pas été abandonnés volontairement}}, et les correspondances directes entre parents et enfants sont autorisées.

\item[1897] Les femmes mariées peuvent être témoins dans les actes civils et notariés.

\item[1901] Loi sur les associations à but non lucratif. Leur fondation est libre, basée sur la notion de contrat entre personnes. Elles ne peuvent recevoir ni dons ni legs.

\item[1904] Dénonciation unilatérale du Concordat de 1802.

% 1904 : 
Autorisation donnée aux amants condamnés pour adultère de s'épouser après leur(s) divorce(s) ou le décès du conjoint trompé.

% 1904 : 
 Loi du 12 avril : majorité pénale à 18 ans au lieu de 16, élargissement de l'excuse de minorité, affirmation de la nécessité de faire passer l'éducatif avant le répressif pour les mineurs pénaux.

\item[1907] La loi du 13 juillet permet aux femmes mariées de toucher et de gérer elles-mêmes leur propre salaire, au lieu qu'il soit remis à leur mari comme c'était la règle durant tout le \siecle{19}. 

\item[1912] Autorisation des recherches en paternité naturelle. Les enfants naturels peuvent demander des aliments à chacun de leurs géniteurs : ce texte vise essentiellement les pères, et les mères peuvent agir au nom de leurs enfants. 

% 1912 : 
La loi du 22 juillet créée des tribunaux spéciaux pour enfants et adolescents. Elle pose les premiers jalons de la liberté surveillée

\item[1913] Mesures d'assistance en faveur des femmes en couche nécessiteuses, et des familles nombreuses nécessiteuses.

\item[1917] Une femme peut être nommée tutrice et siéger au conseil de famille

\item[1920] Une femme mariée peut adhérer à un syndicat sans l'autorisation de son mari.

% 1920 : 
Toute forme de propagande anticonceptionnelle ou de publicité pour des instruments de lutte anticonceptionnelle est interdite (préservatifs, pessaires, diaphragmes,~etc. qui restent néanmoins disponibles en pharmacie).

\item[1921] La loi ouvre la possibilité de prononcer une déchéance partielle de l'autorité paternelle. 

\item[1923] L'adoption des enfants abandonnés (sans limite d'âge inférieure) est ouverte aux couples mariés. Nommée légitimation adoptive, elle n'annule pas le passé de l'enfant.

\item[1924] Identité complète des programmes d'études dans le secondaire féminin et masculin.

\item[1925] L'A.P. commence à placer en nourrice les jeunes enfants (âgés de moins de quatre ans, dans un premier temps) placés \emph{en dépôt} par leurs parents et elle les y laisse grandir. 

\item[1932] \emph{Allocations familiales} (pour tous les enfants).

\item[1931] Les femmes peuvent être nommées (élues ?) juges.

\item[1935] Le décret-loi du 30 octobre sur {\emph{la correction paternelle et l'assistance éducative}} institue l'assistance éducative à domicile. 

% 1935 : 
Le \emph{vagabondage} des mineurs cesse d'être un délit, (contrairement à la mendicité et au racolage qui demeurent des délits). 

\item[1938] La femme mariée acquiert certains des droits des femmes célibataires : droit à une carte d'identité, à un passeport, à ouvrir un compte en banque sans l'autorisation de son époux.

\item[1941] Allocation de salaire unique \emph{(parmi les textes promulgués sous l'occupation, ne comptent que ceux qui ont été confirmés à la Libération : plusieurs d'entre ces derniers avaient été préparés bien avant la guerre)}. 

% 1941 : 
Ouverture des hôpitaux à tous, quels que soient leurs revenus : depuis longtemps déjà les hôpitaux recevaient des malades qui payaient leur séjour et les soins qui leur étaient dispensés. Ils payaient eux-mêmes ou c'est un tiers qui le faisait : militaires (dès l'ancien régime), accidentés du travail (1897), assurés sociaux (1928),~etc. En 1900 cela contribuait pour 20~\% aux recettes des hôpitaux. En 1940 40~\% des hospitalisés donnaient lieu à un remboursement. Le 21 décembre 1941 il est décidé d'étendre cette possibilité à tout le monde, sans maintenir d'exclusive. Comme bien des décisions de cette époque, ce n'était que la mise en œuvre de décisions préparées dés 1938, c'est pourquoi cette orientation n'a pas été remise en question à la libération.

\item[1943] La loi du 15 avril 1943 donne un droit aux secours aux enfants \emph{qui ont un père, même quand celui-ci est valide}. Le droit des parents au « dépôt » volontaire de leurs enfants à l'Assistance Publique est élargi.

\item[1944] Octroi du \emph{droit de vote} aux femmes. 

\item[1945] L'ordonnance du 2 février crée le corps des juges pour enfants, pour les jeunes de moins de 18 ans. Elle crée l'éducation surveillée à l'intention des mineurs délinquants.

\item[1946] Création des \emph{allocations prénatales}.

% 1946 : 
La constitution déclare \emph{égaux} les droits des hommes et des femmes.
\end{description}

\section{Séparation de l'église catholique et de l'État}


 La France n'était jusque là jamais sortie du cadre intellectuel et moral du catholicisme dans lequel elle s'était constituée, sauf durant quelques années pendant la Révolution française. Les mouvements de laïcisation de la fin du Moyen Âge et de la Renaissance, comme ceux du \siecle{18}, avaient travaillé sur les limites entre ce qui revenait aux pouvoirs civils et ce qui revenait au personnel ecclésiastique, mais ils n'avaient pas fondamentalement mis en question la place de la religion catholique comme source du Droit. Le Concordat de \hbox{Napoléon} avait remanié cette situation sans la modifier radicalement. Les autres confessions et les « sans religion » ne représentaient en 1804 qu'un très faible pourcentage de la population. Même si le degré d'identification des français à l'Église fluctuait beaucoup suivant les régions et les milieux sociaux, la population française était très majoritairement catholique. A la fin du \siecle{19} les religieux étaient bien plus nombreux qu'à la fin de l'ancien régime : {\emph{\nombre{81000} religieux en 1789, \nombre{13000} en 1808, \nombre{160000} en 1878}%
% [1]
\footnote{Christian \fsc{SORREL}, \emph{La République contre les congrégations – Histoire d'une passion française 1899-1904}, éd. du Cerf 2003, p. 12. Dans \emph{L'ancien régime, institutions et sociétés} (Le livre de poche, 1993, p.68) François \fsc{BLUCHE} donne des chiffres différents, mais du même ordre de grandeur pour les religieux : \emph{le monde ecclésiastique comprenait, à l'extrême fin de l'ancien régime, un peu moins de \nombre{140000} membres. Le clergé régulier (religieux et religieuses, moines et moniales) regroupait quelques \nombre{59000} âmes (dont \nombre{28000} femmes)... Le clergé séculier représentait quelque \nombre{80000} hommes d'Église (\nombre{139} prélats, environ \nombre{10000} chanoines et les \nombre{70000} prêtres assurant le culte des \nombre{40000} paroisses).}}%
}... La Révolution avait supprimé les monastères et les couvents, et confisqué tous leurs biens, et le Concordat n'avait prévu aucun cadre juridique pour les congrégations religieuses. Et pourtant d'innombrables congrégations nouvelles avaient été créées durant tout le siècle, tandis que beaucoup parmi les anciennes s'étaient relevées de leur état de langueur du \siecle{18}. Les « congréganistes » s'investissaient d'abord et avant tout dans l'enseignement, alors en plein essor, notamment dans le primaire, et aussi et comme toujours dans les services hospitaliers, eux aussi en expansion : en 1847 il y avait en France plus de sept mille religieuses hospitalières, à la fin du Second Empire plus de dix mille, en 1905 plus de douze mille.

 C'est justement là que le bât blessait : le programme des républicains qui avaient conquis le pouvoir en 1879 faisait de la solidarité et de l'enseignement des outils essentiels de gouvernement%
% [2]
\footnote{Cf. \emph{L'invention du social, essai sur le déclin des passions politiques}, Jacques \fsc{DONZELOT}, 1994.}% 
, et il n'était pas question pour eux de les laisser aux mains des employés permanents de l'Église. Ils voulaient retirer à celle-ci les points d'appui institutionnels sur lesquels elle avait assis son influence depuis Constantin. À partir du moment où la gauche radicale l'a emporté dans les urnes, l'histoire des familles comme celle du traitement de la pauvreté a changé de direction. La nouvelle majorité s'est donnée pour mission des tâches traditionnellement dévolues à la Providence%
%[3]
\footnote{Le terme « Providence » est l'un des noms de Dieu.}%
. Au lieu de déplorer les malheurs et les injustices de la \emph{vallée de larmes} où vivraient les hommes, tout en comptant sur un \emph{au-delà} paradisiaque ou infernal pour régler à chacun son compte, ses membres ont estimé du devoir de l'État de s'attaquer lui-même aux sources des malheurs individuels, et d'abord aux injustices sociales, sans se reposer sur les initiatives privées, expressions de la Providence, et de procurer aux citoyens sinon le bonheur du moins un droit effectif à une aide efficace, afin de prévenir le malheur quand c'est possible et de soulager les souffrances quand cela ne l'est pas. 
 Ils s'attaquaient résolument et en pleine connaissance de cause à son autorité sur les esprits.

 À côté de mesures de portée limitée ou relativement symbolique%
% [4]
\footnote{Exemples : suppression de l'obligation du repos dominical (rétabli dès 1906 sous la pression des associations ouvrières) ; sécularisation des cimetières ; suppression des prières publiques constitutionnelles et de tous les signes religieux présents dans les lieux publics ; imposition du service militaire aux religieux et aux séminaristes ; exclusion des membres du clergé des commissions d'enseignement des hôpitaux en tant que membres de droit : curés chargés d'une paroisse, évêques,~etc.}% 
, les républicains ont d'emblée exclu les facultés de théologie des universités publiques, et le personnel religieux du corps enseignant universitaire. L'institution de l'obligation scolaire jusqu'à 12 ans était devenue inéluctable à cette époque%
%[5]
\footnote{Selon \fsc{FURET} et \fsc{OZOUF}, dès le milieu du \siecle{19} près des trois quarts des enfants français sont scolarisés. Chaque commune était depuis Guizot astreinte à l'obligation de fournir une école primaire publique à ses habitants, mais pas à en garantir la laïcité, d'ailleurs souvent refusée par la majorité de la population, comme la suite de l'histoire l'a montré. Le nombre d'enfants scolarisés en 1850 dans les écoles primaires (héritières des petites écoles des siècles précédents) représentait 73~\% du nombre des enfants de la tranche d'âge des 6-13 ans. Il en représentait même 105~\% en 1876-1877 : plus de 100~\%, ce qui s'explique par les enfants scolarisés avant 6 ans et après 13 ans (\fsc{FURET} et \fsc{OZOUF}, 1977, p. 173). Par conséquent en 1880 les enfants d'âge scolaire non scolarisés ne représentaient plus qu'une petite minorité. Mais ces chiffres moyens couvraient des disparités extrêmement grandes :
%\begin{itemize}
\begin{enumerate}[label=\alph*.,itemsep=0pt]
%A)
\item entre régions (le nord et l'est étaient très scolarisés depuis des siècles, au contraire du sud et de l'ouest, très peu scolarisés),
% B)
\item entre villes et campagnes,
% C)
\item parmi les régions rurales elles-mêmes, entre celles de civilisation exclusivement orale comme la Bretagne (valorisant la parole « vivante », et se défiant de la parole « morte », c'est-à-dire écrite), le Pays Basque, la Catalogne,~etc. et celles (de langue française) largement pénétrées par l'écrit,
% D)
\item et au moins autant entre classes sociales.
\end{enumerate}
%\end{itemize}

 Rien ne permettait de penser que ces petites minorités réfractaires à l'école d'alors étaient prêtes à rejoindre spontanément et rapidement le mouvement général, ce qui justifiait d'obliger par la loi les parents à scolariser leurs enfants.}% 
, mais il n'en était pas de même de la laïcité de l'enseignement. Celle-ci était évidemment une arme contre l'Église et son influence dans le domaine scolaire. Alors qu'à cette époque les femmes étaient les plus fidèles soutiens de l'Église, les républicains ont créé pour elles un enseignement secondaire public et laïque similaire en (presque) tout point à celui des garçons. Il s'agissait à la fois de lutter contre l'influence des congrégations en leur interdisant tout enseignement avant de les expulser, et contre la vision traditionnelle d'une femme soumise au contrôle masculin pour l'accès au savoir et à la culture. 

 En légalisant le divorce en 1884, les républicains affranchissaient le mariage civil des règles du Droit Canon. En 1904 l'adultère cesse d'être une faute contre la société dans son ensemble et n'est plus qu'une affaire privée : une fois libérés de leurs unions antérieures, les amants adultères ont le droit de s'unir légalement, ce qui leur était interdit à vie depuis l'empereur Justinien --- interdit qui les empêchait de légitimer leurs enfants déjà nés (adultérins) et leurs enfants encore à naître. 

 Le titre III de la loi 1901 sur les associations a refusé aux congrégations religieuses la liberté d'association. Par conséquent à partir de sa promulgation toutes les congrégations existantes ont été dans l'obligation d'obtenir une autorisation législative, ce dont elles s'étaient le plus souvent passées depuis le début du \siecle{19}. Cette autorisation a été refusée à la grande majorité d'entre elles, \emph{ipso facto} dissoutes. Les congrégations enseignantes, dont les effectifs étaient de beaucoup supérieurs à celui des religieuses hospitalières, ont toutes été interdites%
% [6]
\footnote{Sur un nombre de plus de \nombre{1300} congrégations, \nombre{140} congrégations masculines et \nombre{888} congrégations féminines ont été dissoutes. Cela a concerné plus de cent cinquante mille personnes dont 80~\% de femmes...}% 
. Seules ont été épargnées les congrégations hospitalières ... et toutes les congrégations implantées aux colonies. 

 La laïcisation des hôpitaux a commencé dès 1879, mais elle ne pouvait se faire qu'au rythme de la formation du personnel laïc d'encadrement et infirmier, ce qui demandait d'abord de créer les écoles d'infirmières et de surveillantes nécessaires, puisque les noviciats des congrégations hospitalières en avaient jusque là tenu le rôle. On observe quelques créations vers 1880, puis en 1899 est prise la décision de créer une école d'infirmière dans toutes les villes de faculté. Ceci étant dit la laïcisation de chaque institution dépendait d'abord et surtout de la couleur politique du conseil municipal dont elle dépendait.

 En 1900 les religieuses formaient la plus grande part du personnel soignant, par contre en 1975 l'ensemble du personnel soignant (sans compter les autres employés des hôpitaux) se comptait à près de \nombre{300000} personnes. Les religieuses ne représentaient plus à cette date qu'une toute petite minorité vieillissante. C'est qu'il ne s'agissait plus des anciens hôpitaux et hospices voués essentiellement aux indigents. Désormais il s'agissait d'établissements industriels, la réalisation concrète des « machines à guérir » dont les penseurs de la fin du \siecle{18} avaient rêvé. Les clients avaient complètement changé, et l'échelle aussi : en 1975 on trouvait dans les hôpitaux publics plus de \nombre{10000} (dix mille) médecins des hôpitaux à plein temps et \nombre{26900} internes, beaucoup plus qu'il n'y avait de personnels soignants, tous statuts confondus, dans tous les hôpitaux du \siecle{19}%
% [8]
\footnote{Cf. Jean \fsc{IMBERT}, \emph{Histoire des hôpitaux en France}, 1982.}% 
.
\section{Critiques de gauche et anarchistes de la famille}

 

Selon Jacques \fsc{Donzelot} depuis la Belle Époque des « militants », qu'il classe parmi les anarchistes ou à côté d'eux, ont mis {\emph{"...en place les petites machines de guerre contre la famille \emph{[... que sont]} la célébration de l'union libre, \emph{[...]} la distribution des produits anticonceptionnels et \emph{[...]} la propagande pour la grève des ventres}\footnote{Idem p. 163.}.}\footnote{Jacques \fsc{DONZELOT}, \emph{La police des familles}, 1977, 220 pages. Chapitre 5, « La régulation des images », p. 154 à 211.}" 
Parmi eux on trouvait, à côté des militants de base, des médecins comme Adolphe \fsc{Pinard}, des écrivains comme Octave \fsc{Mirbeau}, des hommes politiques de gauche comme Léon \fsc{Blum}, des savants comme Paul \fsc{Langevin}, soucieux {\emph{"...d'incorporer l'hygiène et donc le contrôle des naissances dans le fonctionnement des institutions."}} On trouvait également en première ligne la \emph{Ligue des droits de l'homme} et la \emph{Société de prophylaxie sanitaire et morale}, dirigées toutes deux par le docteur \fsc{Sicard~de Plauzolles}. Ils s'exprimaient dans divers ouvrages tels que \emph{La fonction sexuelle} (1908) du même docteur, ou \emph{Du mariage} de Léon \fsc{BLUM} (1908). 

Leur discours \emph{"...est à peu près celui-ci : puisque la famille est détruite par les nécessités économiques de l'ordre social actuel, il faut que la collectivité remplace le père pour assurer la subsistance de la mère et des enfants. Au père se substituera ainsi la mère comme chef de la famille ; puisqu'elle en est le centre fixe, la matrice et le cœur, elle en sera la tête. Les enfants seront sous sa tutelle, centralisée par l'autorité publique. Tous porteront le nom de leur mère ; ainsi les enfants nés d'une même femme mais de pères différents auront le même nom ; aucune différence n'existera plus entre légitimes et bâtards. L'influence de l'homme sur la femme et sur les enfants sera en rapport avec l'amour et l'estime qu'il inspirera ; il n'aura d'autorité que par sa valeur morale : il n'aura de place au foyer que celle qu'il méritera..\footnote{Idem, p. 164.}.} 

 Pour les plus radicaux de ces théoriciens, tels l'avocat Ernest \fsc{TARBOURIECH} (in \emph{La cité future, essai d'une utopie scientifique}, 1902), socialiste marxiste et collectiviste : \emph{"La puissance paternelle aura disparu... Le père et la mère n'auront sur leur progéniture aucun droit d'aucune sorte, mais seulement des devoirs qui peuvent ainsi se formuler : aider l'état dans la tâche qui lui incombe vis à vis des jeunes générations. L'éducation et l'instruction, affaires d'état, seront réglées souverainement par l'état au mieux. Les médecins représentant la communauté confieront chaque enfant à la personne qui donnera les soins les plus tendres et les plus éclairés. La loi présumera que cette personne est la mère mais cette présomption si naturelle... ne sera pas... de Jure... mais... susceptible de preuve contraire.
 ...l'autorité médico-judiciaire pourra intervenir}..." à tout instant jusqu'à la majorité du mineur (p. 309). 
 
 En résumé  c'est l'État, détenteur des moyens de production et pourvoyant aux besoins de la totalité de la population, active comme inactive, qui gère les effectifs de ses employés. Il dirige donc la reproduction et l'éducation. Il ordonne l'euthanasie des nouveaux-nés jugés par une commission scientifique (le "juge médical") mal conformés, "vicieux", "tarés", ou voués au crime ou à l'impuissance économique \emph{"...pendant cette période où ils ne sont pas encore une personnalité."} (p. 397). Le même "juge médical" accorde ou refuse aux individus le droit à une vie sexuelle. L'état prescrit de déclarer toutes les grossesses et il les surveille. C'est lui qui décide si la génitrice est apte à collaborer avec l'état dans la mission d'élever le futur citoyen. Il peut à tout moment la remplacer au profit d'un éleveur ou d'un éducateur offrant plus de garanties qu'elle (éventuellemet le géniteur de l'enfant). Il s'agit pour \fsc{tarbouriech} d'étendre à toute la société le régime de la tutelle, et à toutes les mères l'attribution des secours éducatifs et du contrôle sanitaire, afin qu'elles soient payées comme nourrices de leurs propres enfants et qu'elles les élèvent non pour elles mais pour l'État et sous son controle. Il étend à tous les enfants le régime des "enfants assistés" (pupilles)  de l'Assistance Publique : \emph{"Bref,} selon \fsc{Donzelot}  \emph{...une gestion médicale de la sexualité libérera la femme et les enfants de la tutelle patriarcale, cassera le jeu familial des alliances et des filiations au profit d'une emprise plus grande de la collectivité sur la reproduction et d'une prééminence de la mère. Soit un féminisme d'état\footnote{Une telle utopie serait-elle vraiment un "féminisme (d'Etat)" ? Il faut n'avoir aucune idée de la dépendance des "nourrices" de l'Assistance face à l'administration pour le croire. La responsabilité de l'éducation reviendrait entièrement à l'Etat, et la puissance enlevée aux pères ne serait pas donnée aux mères, qui ne détiendraient qu'une délégation d'autorité parentale, révocable à tout instant. Face à leurs enfants elles auraient moins de garanties juridiques qu'avec le Code Napoléon, tout patriarcal que soit celui-ci.}."\footnote{Idem, p. 164.}}
 



 De la Belle Epoque à la fin du baby-boom les néo-malthusiens se sont opposés aux « populationnistes ». Ceux-ci se recrutaient dans la bourgeoisie traditionnelle, attachée pour de multiples raisons à la transmission de son patrimoine, mais aussi parmi {"[...] \emph{les ligues de pères de famille, la Ligue des mères de familles nombreuses, l'Association des parents d'élèves des lycées et collèges, l'École des parents, l'Union des assistantes sociales, les organisation scoutes, les ligues d'hygiène morale, d'assainissement des kiosques de journaux, des abords des lycées,~etc}\footnote{Idem, p. 162.}".} Les membres de ces groupes de pression défendaient la répartition traditionnelle des rôles sexués et des pouvoirs au sein de la famille. Ils pensaient en effet que plus la structure familiale était forte, plus elle avait de chances d'être prolifique, et de bien réaliser sa mission éducative. Ils luttaient {\emph{"...contre tout ce qui peut fragiliser la famille : le divorce, les pratiques anticonceptionnelles, l'avortement."}\footnote{Idem.}}




 Bien des mesures décidées à cette époque par la Gauche au pouvoir allaient dans le sens des néo-malthusiens et contre les populationnistes. Le divorce%
% [8] 
\footnote{... pour faute seulement, parce que l'opinion d'alors n'acceptait pas d'autre motif, mais divorce tout de même : d'où jusqu'aux années 1970 tout un folklore de manœuvres vaudevillesque pour fabriquer en commun une « faute » légalisable (lettres d'injures...) même quand les conjoints étaient d'accord sur l'objectif.} 
permettait aux épouses maltraitées, délaissées ou bafouées, de sortir de la prison où le mariage les retenait jusque là. 

 Le premier objectif de l'obligation scolaire était certes de répandre le savoir, la culture commune, et de ne laisser personne à l'écart de cette richesse, mais un effet pleinement assumé, et même désiré, de cette obligation était aussi de contraindre toutes les familles à accepter l'entrée en leur sein de points de vue extérieurs. Elles ne pouvaient plus élever leur enfant à l'écart du monde. 

 La création d'un enseignement secondaire public pour filles calqué sur celui des garçons promouvait l'égalité complète des filles et des garçons, même si en 1880 on en était loin. C'était un choix historique, une rupture dans la répartition sexuée traditionnelle des tâches et compétences. 

 L'obligation scolaire interdisait aux parents de placer leurs enfants chez un employeur avant leurs 12 ans (11 ans s'ils avaient obtenu le certificat d'études) ou de les employer eux-mêmes à plein-temps%
% [9]
\footnote{Après le rapport de \fsc{VILLERMÉ} sur le travail des enfants, la loi du 22 mars 1841 avait fixé pour la première fois une limite d'âge au-dessous de laquelle il était interdit aux employeurs, et donc (indirectement) aux parents, de mettre les enfants au travail. La première borne avait été posée à l'âge de 8 ans. Elle avait été plus ou moins respectée mais ce n'en était pas moins le début d'une lente progression. La loi sur la scolarité obligatoire s'inscrivait comme une nouvelle étape dans cette progression, et l'exploitation du travail de l'enfant par ses parents commençait d'apparaître comme une forme de maltraitance.}% 
. 

 Quant au droit des femmes mariées à gérer leur propre salaire, c'était une part de souveraineté symbolique en moins pour les maris. En fait dans bien des ménages populaires c'étaient les femmes qui tenaient les cordons de la bourse, d'un commun accord entre conjoints (en a-t-il toujours été ainsi ? Les épouses semblent être presque toujours chargées de gérer les réserves, les resserres et les greniers, ce que symbolise le fait qu'on leur confiait les clés).

 À partir de 1912 les enfants sans père reçoivent le droit de demander des aliments à leur géniteur \emph{(recherche en paternité naturelle)}. Une mère célibataire n'est plus sans recours devant celui qui l'a laissée seule avec son enfant, qu'elle représente devant la justice. Cela entraîne pour corollaire qu'une femme mariée n'est plus aussi à l'abri qu'avant des conséquences matérielles et sociales des frasques pré ou extra conjugales de son conjoint. Pour autant un enfant illégitime ne peut toujours pas hériter de son père. 


\section{Mise en question du droit de correction}

 À aucun moment de l'histoire les parents n'ont été autorisés à faire subir à leurs enfants \emph{tout} ce qu'ils pouvaient imaginer. La tolérance à leurs abus de pouvoir a pu varier au fil des siècles, mais ils n'ont jamais eu le droit de les estropier, pas plus qu'ils n'avaient le droit d'estropier les enfants des autres, ni d'en faire leurs partenaires sexuels. Mais la Justice a toujours beaucoup de difficultés à les poursuivre lorsque les traces des sévices ne se voient pas, ou dans les cas de négligence simple, d'abandon moral. En faisant du délaissement et de la maltraitance des délits, la loi {\emph{sur la protection judiciaire des enfants maltraités et moralement abandonnés}} a permis de prononcer la déchéance des droits parentaux pour ces seuls motifs. 

 Le fait de ne pas juger les mineurs et les majeurs selon les mêmes critères est sans doute aussi vieux que la justice elle-même. Refuser de traiter les fautes des mineurs autrement que celles des adultes serait manquer de bon sens. Ce qui fait la différence, c'est l'âge de la coupure entre l'irresponsabilité complète, l'atténuation de la responsabilité \emph{(l'excuse de minorité)} et la responsabilité pleine et entière. Ce sont aussi les peines encourues : nature des peines, durée... La minorité pénale était fixée à 16 ans depuis l'ancien régime. La loi du 12 avril 1904 la repousse de 16 à 18 ans, et elle affirme la prééminence de l'éducatif sur le répressif.

 La loi du 28 juin 1904 s'inscrit dans le courant d'idées qui a confié les enfants \emph{moralement abandonnés} à l'Assistance Publique. Elle ordonne que les pupilles \emph{difficiles} soient confiés non plus à des prisons, mais à des écoles professionnelles publiques ou privées. Ce texte confirme à l'administration du service des enfants assistés, détentrice de la puissance paternelle sur les pupilles, le droit de désigner ceux qu'elle garderait et ceux qu'elle refuserait d'assumer et pour lesquels elle solliciterait l'aide de la Justice. Au même moment c'étaient encore en principe les pères qui définissaient ce qui sous leur toit était indiscipline et insoumission à leur autorité.

 La loi du 24 juillet 1889 {\emph{sur la protection judiciaire des enfants maltraités et moralement abandonnés}} donne aux juges la possibilité de prononcer la déchéance totale de la puissance paternelle pour inconduite des parents, en cas de mauvais traitements ou de délaissement de l'enfant (et de plein droit dans le cas de certaines condamnations infamantes). La même loi confie à l'administration (c'est-à-dire à l'Assistance Publique) la tutelle des enfants maltraités, victimes de crimes ou de délits ou délaissés. Le service les prend en charge même s'ils sont âgés de plus de 12 ans à leur entrée. Ces enfants deviennent des pupilles comme les autres. Ils sont traités à l'instar des autres enfants du service. Quel que soit leur âge, autant que faire se peut ils seront placés en nourrice, pour de longues durées, et dans tous les cas ils seront totalement coupés de leurs parents déchus.

 Une nouveauté majeure est introduite en 1912, avec la création des tribunaux pour enfants (sans magistrats spécialisés) et la création de la liberté surveillée%
% [10]
\footnote{Cf. l'ouvrage collectif \emph{Protéger l'enfant} (1996), qui aborde les problèmes de la jeunesse sous l'angle de la \emph{protection judiciaire}. Il présente un résumé de l'histoire de celle-ci, et des débats d'idée et des conflits de pouvoir qui ont présidé à sa naissance et qui la traversent encore...}%
. Cette nouveauté avait été précédée depuis les années 80 par tout un mouvement d'idées, notamment chez les magistrats chargés de l'application du droit de correction paternelle. Il y a en effet un lien direct entre la dénonciation de l'indignité des pères (cf. la loi de 1889 sur la déchéance paternelle), et la mise en cause du droit de correction%
%[11]
\footnote{Pascale \fsc{QUINCY-LEFEBVRE}, « Une autorité sous tutelle. La justice et le droit de correction des pères sous la troisième république », in \emph{Lien social et politiques-RIAC}, 37, Printemps 1997, p. 99 à 109.}% 
. Ceux qui s'intéressaient à ce problème ne contestaient en aucune façon l'existence d'enfants \emph{insoumis}, difficiles à élever et qui provoquaient le \emph{légitime} mécontentement de leurs parents. Ils estimaient par contre que c'était un problème qui débordait le cadre familial, parce qu'on pensait qu'en règle générale ceux qui étaient insoumis à leurs parents ne faisaient pas de bons citoyens, et risquaient de devenir délinquants, c'est pourquoi l'état ne pouvait s'en désintéresser. Ils estimaient surtout qu'il n'était pas possible de s'en tenir à la parole du parent, et qu'il fallait s'assurer par une enquête approfondie de la réalité et de la nature des problèmes. 

 D'autre part ils estimaient que la prison n'était pas un outil de correction efficace, et qu'il fallait fournir aux jeunes insoumis une prestation éducative de durée suffisante pour obtenir d'eux un amendement réel. Ils pensaient que cette prestation devait être fournie par un internat sous le contrôle de la Justice et non sous celui des pères. Ils accusaient en effet ceux-ci (ceux du moins qui réclamaient à la justice son aide, c'est-à-dire ceux des pères, tous de milieu populaire, qui ne pouvaient supporter les frais d'une pension dans l'un des internats privés dont c'était la spécialité) d'être trop prompts à retirer leurs enfants (comme ils en avaient le droit) dès que ceux-ci semblaient suffisamment \emph{intimidés} par l'incarcération. Ils les suspectaient de n'avoir qu'un seul but, celui de mettre le plus vite possible leurs enfants au travail pour toucher leur salaire. Aux yeux des réformateurs, les droits des pères (éducatifs ou financiers) importaient moins que l'intérêt des enfants, qui était de recevoir une bonne éducation durant le temps nécessaire et avec la sévérité qui convenait, et que l'intérêt de la société, qui était de voir conduire à son terme la \emph{correction des insoumis}. 

 C'est donc du fait des juges et non à la demande de la société que la correction paternelle est peu à peu tombée en désuétude. Ils ont pris l'habitude dès les années 1890 de demander systématiquement une enquête pour vérifier si le parent demandeur avait vraiment des \emph{sujets de mécontentement très graves}, et s'il n'était pas plutôt un parent \emph{indigne}. Ils ont ainsi retiré aux parents leur droit de qualifier eux-mêmes de fautifs les comportements de leurs enfants. Puis la loi de 1889 leur a donné la possibilité non seulement de refuser aux parents indignes une demande de correction paternelle, mais encore de leur retirer la garde de l'enfant. Enfin la loi de 1904 les a explicitement autorisés à mettre les pupilles indisciplinés en maison de correction pendant plusieurs années (c'était déjà le sort des pupilles indisciplinés ou récalcitrants du \siecle{19}). Ces pupilles pouvaient être les enfants de parents déclarés \emph{indignes} une fois que leur déchéance était prononcée. Les parents étaient souvent qualifiés d'indignes parce qu'ils laissaient la bride sur le cou de leur enfant en ne le contrôlant pas d'assez près, ou parce qu'ils entravaient les efforts des éducateurs qui tentaient de les amender. Il semble qu'à cette époque les juges et les premiers travailleurs sociaux déploraient plus leur laxisme que leur autoritarisme. 

 En 1921 une loi ouvre la possibilité de prononcer une \emph{déchéance partielle} de l'autorité paternelle. Une déchéance totale des droits parentaux était une mesure aux effets quasi irréversibles. À partir de 1921 les magistrats n'ont plus été réduits au tout ou rien d'une telle mesure face aux parents qu'ils jugeaient incompétents, délinquants ou négligents. Au contraire ils pouvaient prononcer une déchéance partielle et provisoire, non seulement là où la déchéance totale aurait été injustifiée, mais même là où ils y auraient recouru par nécessité en l'absence d'une mesure plus souple. Le nombre de ces décisions a donc crû rapidement. Cela ne s'est pas traduit par un accroissement important du nombre de jeunes placés, mais par un changement du statut de beaucoup d'enfants placés : le nombre des pupilles a décru au fur et à mesure qu'augmentait celui des enfants en garde, sans que l'effectif total ne se modifie sensiblement. Pendant ce temps le nombre des abandons ne cessait de diminuer.

 On a vu que dès la fin du \siecle{19} des juges avaient commencé d'ordonner des enquêtes pour évaluer la pertinence des demandes de correction paternelle. En 1923 le succès de cette pratique a conduit à la création à Paris, où étaient traitées les deux tiers des demandes de correction paternelle faites en France, d'un service social réalisant pour le tribunal des enquêtes débordant largement la matérialité des faits reprochés par les parents à leur enfant. Désormais la demande d'intervention des parents était entendue comme l'expression d'un dysfonctionnement dans la famille, qui dépassait largement le mineur concerné. Cela entraînait une enquête sociale, c'est-à-dire l'introduction au sein de la famille, d'un observateur extérieur mandaté par les juges. À partir de cette base ces derniers se sont donné le droit de conseiller les parents face aux problèmes que leur posaient leurs enfants : dans la plupart des cas cela les conduisait à mettre en œuvre une action non judiciaire, confiée sous leur contrôle à des institutions privées. Il s'agissait très souvent d'une \emph{action éducative en milieu ouvert}, mais ils pouvaient aussi prendre l'initiative de placer en établissement de correction les mineurs qui leur semblaient en avoir besoin, entre autres au titre des lois de 1889 et de 1921 sur la déchéance paternelle, et de 1904 sur les pupilles difficiles ou vicieux. Tous ces placements écartaient le contrôle paternel.

 Les magistrats n'accédaient plus à la demande de correction paternelle que dans un nombre de cas de plus en plus petit : environ un cas sur quatre ou cinq en 1917, un sur dix dans les années trente. Ils ont ainsi vidé de sa substance le droit de correction paternelle. Le nombre des mineurs placés à ce titre n'a donc cessé de baisser jusqu'à devenir marginal, comme le montre la table \vref{ord-corr-pat}.
 
\begin{table}[h]
\centering
\caption{Ordonnances de correction paternelle}
\label{ord-corr-pat}
\begin{tabular}{ccc}
 & France & part \\
Année & entière & Seine \\
\hline
1881 & 1192 & 63,7~\% \\
1891 & 737 & 59,6~\% \\
1901 & 731 & 50,0~\% \\
1911 & 644 & 66,9~\% \\
1921 & 270 & 83,3~\% \\
1931 & 60 & 66,7~\%
\end{tabular}
\end{table}
 
 Le décret-loi du 30 octobre 1935 sur {\emph{la correction paternelle et l'assistance éducative}} institue l'assistance éducative à domicile. Il entérine les changements qui travaillaient depuis deux générations l'exercice du droit de correction paternelle. Il a également dépénalisé le vagabondage des mineurs (les fugues simples, sans délits caractérisés) ce qui suggère que ces deux ordres de faits se recouvraient. Désormais les jeunes vagabonds ne ressortissaient plus de la Justice, mais mais d'une assistance placée sous le contrôle des juges. 

\section{Vers une assistance non punitive ?}

Les mesures d'assistance en faveur des femmes en couche et des familles nécessiteuses, mais aussi les allocations ouvertes à toutes les familles (allocations familiales, salaire unique, allocations prénatales, etc.), les consultations de nourrissons, les crèches, et les améliorations progressives des conditions de travail, tout cela a facilité la vie des familles. Depuis le début du siècle le nombre des abandons a décru régulièrement et massivement.


Depuis le milieu du \siecle{19}, l'administration pouvait verser aux mères seules une aide afin qu'elles placent elles-mêmes leur enfant chez une nourrice de leur choix. Dans le même esprit, le placement chez une nourrice directement salariée par l'assistance publique a de plus en plus souvent été perçu comme une forme de secours à la famille, et non plus comme le remplacement d'une famille par une autre. On aidait la mère en lui fournissant une nourrice, là où les familles citadines non indigentes de l'époque se la procuraient elles-mêmes. À partir de 1924, l'assistance publique de Paris a commencé de placer en nourrice des enfants non abandonnés, dont les parents n'étaient pas déchus de leurs droits, des enfants qui n'étaient pas des pupilles. 

 Pour cela, il avait fallu franchir une barrière psychologique et oser placer en nourrice pour une durée indéterminée, l'enfant qu'une femme pourrait reprendre un jour. Cela allait contre les pratiques antérieures de l'assistance publique, mais (et ce n'est sans doute pas un hasard) c'est également à partir de l'année 1924 que la loi a permis d'adopter les enfants mineurs. C'est à partir de cette date que le service a la possibilité de procéder à des adoptions d'enfants abandonnés. Face à la réalité d'adoptions authentiques (quel que soit leur nombre réel, quelques centaines par ans semble-t-il), l'illusion que le placement dans une famille nourricière salariée était une espèce d'adoption ne pouvait plus tenir. Il ne pouvait plus être question pour un « nourricier » de prendre la place d'un parent dans le cœur de l'enfant, mais seulement de fournir à ce dernier une assistance pendant un temps plus ou moins long. 

 Au motif que les liens avec leurs parents n'étaient pas coupés, l'administration pouvait laisser les petits enfants concernés en collectivité (on a vu que c'était sa position traditionnelle face aux enfants qui avaient des parents), mais :
\begin{enumerate}
%a)
\item elle priverait alors autant de nourrices de leur emploi alors que les régions pauvres où elles vivaient avaient besoin de ces emplois et que le nombre des enfants abandonnés avait déjà beaucoup baissé depuis le début du siècle,
% b)
\item les nourrices coûtaient moins cher que les internats,
% c)
\item d'autre part, et surtout, on savait qu'en collectivité l'état de santé des petits enfants (0~à 4 ans) se dégrade inexorablement et rapidement au fil du temps, comme l'expérience l'avait régulièrement démontré depuis plusieurs siècles. Dès que le placement courait le risque d'être durable (plus de quelques semaines), il fallait donc autant que possible éviter aux plus jeunes les « dépôts » des enfants de l'Assistance Publique et les « orphelinats ».
\end{enumerate}
 Les jeunes concernés ont donc assez systématiquement été placés en famille d'accueil. On a rapidement observé, comme il était prévisible au vu de l'histoire antérieure de l'assistance, que le placement en famille d'accueil des tout-petits donnait de très bons résultats en ce qui concerne la santé physique et psychologique (indissociables à cet âge). Cette observation a assuré le succès de cette formule. Peu à peu les âges d'admission ont été assouplis. Les enfants qui ont des parents et qui gardent un lien avec eux ont pu entrer en famille d'accueil à un âge de plus en plus avancé, et tous ont fini par bénéficier de cette formule. 

 Le droit des parents au « dépôt » de leurs enfants à l'A.P. a été élargi par la loi du 15 avril 1943. Cette loi ouvrait un droit aux secours (dont le placement est l'une des formes possibles) aux enfants \emph{qui ont un père}, même quand celui-ci est valide et donc capable de travailler. Elle impliquait qu'un homme qui ne peut subvenir financièrement aux besoins de sa famille n'était pas pour autant disqualifié comme époux et comme père. Il n'avait pas pour autant à être sanctionné comme un débiteur insolvable à écarter de sa partenaire et de ses enfants. Il convenait plutôt de l'assister.

 On peut supposer que les séparations familiales et les privations de cette période de rationnement avaient facilité cette décision. L'un de ses objectifs a pu être de fournir une aide aux femmes et aux enfants des prisonniers de guerre retenus en Allemagne (valides certes, mais enfermés au loin et travaillant sans rémunération dans la tradition plurimillénaire de l'esclavage des vaincus). Mais il s'agissait aussi de reconnaître l'évolution des pratiques réelles des services. Preuve en est que cette réglementation n'a pas été abrogée après la guerre. 

 C'était aussi un corollaire du fait qu'on se libérait un peu de la représentation patriarcale du monde qui dominait les siècles précédents : si les hommes n'étaient plus les patriarches tout-puissants qu'on avait pensé qu'ils étaient (ou voulu qu'ils soient), leur impuissance financière ne justifiait plus leur éviction.
 
 \section{Construction de l'État-providence}
 
 À côté des drames qui en ont fait aussi une période tragique (deux guerres mondiales, diverses guerres locales, les guerres de décolonisation et au moins une grande crise économique) le demi-siècle qui va de 1910 à 1960 a vu la fin silencieuse d'une civilisation rurale millénaire (ce qui a représenté un drame d'une autre nature pour bien des gens) et révolutionné la vie quotidienne : création des média de masse, construction de banlieues concentrationnaires, progrès fulgurants de la lutte contre les maladies infectieuses, début de la croissance explosive qui a caractérisé les « trente glorieuses »,~etc. Il a vu l'essor du salariat, celui de multiples caisses d'assurances sociales et de retraite (déjà initié à petite échelle dès la fin du \siecle{19}) puis leur extension à l'ensemble de la population. Il a vu la création des allocations familiales. Il a vu le début de la démocratisation des enseignements secondaire et supérieur, qui étaient à l'époque des outils indiscutables d'ascension sociale. 

 Ces années étaient marquées par la conviction qu'il était possible d'aller vers un monde meilleur, lorsque les forces du mal seraient vaincues (guerres mondiales, guerres coloniales, capitalisme, fascisme, communisme, etc.) et ce monde semblait alors à portée de main. Apparue après la seconde guerre mondiale, l'expression \emph{État-providence} (en Angleterre le « \anglais{Welfare State} », état de bien-être par opposition à l'état de guerre) exprimait un aspect de ce projet : à défaut de faire descendre {\emph{ici-bas}} le paradis, procurer à tous au moins un solide filet de sécurité contre l'indigence et l'abandon social. 

 Cet effort de longue haleine, commencé par endroits dès le \siecle{19}, a obtenu des résultats très significatifs, grâce auxquels à partir de 1945 l'ensemble de la population européenne (entre autres), et en particulier les travailleurs les plus pauvres, a bénéficié d'assurances sociales et de retraites par répartition qui mutualisaient les risques. Ces systèmes imposés par les États rendaient en principe inutile le recours à l'assistance et à la bienfaisance comme la prise en charge des indigents par leur propre parentèle. 

 Le montant des aides financières accordées à tous les parents pour la prise en charge de leurs enfants a connu son apogée entre 1945 et 1965. Elles protégeaient de l'indigence les enfants des pauvres mieux qu'on ne l'avait jamais fait jusqu'alors.  
 
 Jusqu'aux années soixante du vingtième siècle, la législation, les mœurs, les discours dominants et le niveau élevé des aides matérielles à la famille et à la procréation, étaient conformes aux vœux des « populationnistes ». Les allocations familiales n'ont jamais été aussi fortes qu'alors par rapport aux salaires de base, ce qui a contribué à permettre à beaucoup de femmes de rester chez elles élever plus d'enfants qu'elles n'en auraient eu sans cela. Le marché de l'emploi aurait probablement permis à beaucoup d'entre elles de travailler au dehors de leur famille. Si les allocations étaient si substantielles, c'est bien parce que l'atmosphère nataliste d'alors était favorable à cette représentation des familles. Il s'agissait pour l'État de promouvoir les naissances, ce qui justifiait d'aider les familles prolifiques et de ne pas heurter de front les idées de ceux qui les représentaient. C'est dans cette atmosphère idéologique très favorable aux couples conjugaux, aux familles et aux associations qui les représentaient que les enfants du « \anglais{baby-boom} » ont été conçus et élevés. 
 





\chapter{L'État, providence des familles ?}


 À côté des drames qui en ont fait aussi une période noire (deux guerres mondiales, diverses guerres locales, les guerres de décolonisation et au moins une grande crise économique) le demi-siècle qui va de 1910 à 1960 a vu la fin silencieuse d'une civilisation rurale millénaire (ce qui a représenté un drame d'une autre nature pour bien des gens) et révolutionné la vie quotidienne : création des média de masse, construction de banlieues concentrationnaires, progrès fulgurants de la lutte contre les maladies infectieuses, début de la croissance explosive qui a caractérisé les « trente glorieuses »,~etc. Il a vu l'essor du salariat, celui de multiples caisses d'assurances sociales et de retraite (déjà initié à petite échelle dès la fin du \siecle{19}) puis leur extension à l'ensemble de la population. Il a vu la création des allocations familiales. Il a vu le début de la démocratisation des enseignements secondaire et supérieur, qui étaient à l'époque des outils indiscutables d'ascension sociale. 

 Ces années étaient marquées par la conviction qu'il était possible d'aller vers un monde meilleur, lorsque les forces du mal seraient vaincues (guerres mondiales, guerres coloniales, capitalisme, communisme, etc.) et ce monde semblait alors à portée de main. Apparue après la seconde guerre mondiale, l'expression \emph{État Providence} (en Angleterre le « \emph{welfare state} », état de bien-être par opposition à l'état de guerre) exprimait un aspect de ce projet : à défaut de faire descendre {\emph{ici-bas}} le paradis, procurer à tous au moins un solide filet de sécurité contre l'indigence et l'abandon social. 

 Cet effort de longue haleine, commencé dès le \siecle{19}, a obtenu des résultats très significatifs, grâce auxquels à partir de 1945 l'ensemble de la population, et en particulier les travailleurs les plus pauvres, a bénéficié d'assurances sociales et de retraites par répartition qui mutualisaient les risques. Ces systèmes imposés par les états rendaient en principe inutile le recours à l'assistance et à la bienfaisance comme la prise en charge des indigents par leur propre parentèle : il s'agissait de l'immense majorité de la population, et d'abord des salariés. 

 Le montant des aides financières accordées à tous les parents pour la prise en charge de leurs enfants a connu son apogée entre 1945 et 1965. Elles protégeaient de l'indigence les enfants des pauvres mieux qu'on ne l'avait jamais fait jusqu'alors. 

 Les mesures d'assistance en faveur des femmes en couche et des familles nécessiteuses, mais aussi les allocations ouvertes à toutes les familles (allocations familiales, salaire unique, allocations prénatales, etc.), les consultations de nourrissons, les crèches, et les améliorations progressives des conditions de travail, tout cela a facilité la vie des familles et permis que depuis le début du siècle le nombre des abandons décroisse régulièrement et massivement.

 Depuis le milieu du \siecle{19}, l'administration pouvait verser aux mères seules une aide afin qu'elles placent elles-mêmes leur enfant chez une nourrice de leur choix. Dans le même esprit, le placement chez une nourrice directement salariée par l'assistance publique a de plus en plus souvent été perçu comme une forme de secours, et non plus comme le remplacement d'une famille par une autre. On aidait la mère en lui fournissant une nourrice, là où les familles citadines non indigentes de l'époque se la procuraient elles-mêmes. À partir de 1924, l'assistance publique de Paris a commencé de placer en nourrice des enfants non abandonnés, dont les parents n'étaient pas déchus de leurs droits, des enfants qui n'étaient pas des pupilles. 

 Pour cela, il avait fallu franchir une barrière psychologique et oser placer en nourrice pour une durée indéterminée, l'enfant qu'une femme pourrait reprendre un jour. Cela allait contre les pratiques antérieures de l'assistance publique, mais (et ce n'est sans doute pas un hasard) c'est également à partir de l'année 1924 que la loi a permis d'adopter les enfants mineurs. C'est à partir de cette date que le service a la possibilité de procéder à des adoptions d'enfants abandonnés. Face à la réalité d'adoptions authentiques (quel que soit leur nombre réel, quelques centaines par ans semble-t-il), l'illusion que le placement dans une famille nourricière salariée était une espèce d'adoption ne pouvait plus tenir. Il ne pouvait plus être question pour un « nourricier » de prendre la place d'un parent dans le cœur de l'enfant, mais seulement de fournir à ce dernier une assistance pendant un temps plus ou moins long. 

 Au motif que les liens avec leurs parents n'étaient pas coupés, l'administration pouvait laisser les petits enfants concernés en collectivité (on a vu que c'était sa position traditionnelle face aux enfants qui avaient des parents), mais :
\begin{enumerate}
%a)
\item elle priverait alors autant de nourrices de leur emploi alors que les régions pauvres où elles vivaient avaient besoin de ces emplois et que le nombre des enfants abandonnés avait déjà beaucoup baissé depuis le début du siècle,
% b)
\item les nourrices coûtaient moins cher que les internats,
% c)
\item d'autre part, et surtout, on savait qu'en collectivité l'état de santé des petits enfants (0~à 4 ans) se dégrade inexorablement et rapidement au fil du temps, comme l'expérience l'avait régulièrement démontré depuis plusieurs siècles. Dès que le placement courait le risque d'être durable (plus de quelques semaines), il fallait autant que possible éviter aux plus jeunes les « dépôts » des enfants de l'Assistance Publique et les « orphelinats ».
\end{enumerate}
 Les jeunes concernés ont donc assez systématiquement été placés en famille d'accueil. On a rapidement observé, comme il était prévisible au vu de l'histoire antérieure de l'assistance, que le placement en famille d'accueil des tout-petits donnait de très bons résultats en ce qui concerne la santé physique et psychologique (indissociables à cet âge). Cette observation a assuré le succès de cette formule. Peu à peu les âges d'admission ont été assouplis. Les enfants qui ont des parents et qui gardent un lien avec eux ont pu entrer en famille d'accueil à un âge de plus en plus avancé, et tous ont fini par bénéficier de cette formule. 

 Le droit des parents au « dépôt » de leurs enfants à l'A.P. a été élargi par la loi du 15 avril 1943. Cette loi ouvrait un droit aux secours (dont le placement est l'une des formes possibles) aux enfants \emph{qui ont un père}, même quand celui-ci est valide et donc capable de travailler. Elle impliquait qu'un homme qui ne peut subvenir financièrement aux besoins de sa famille n'était pas pour autant disqualifié comme époux et comme père. Il n'avait pas pour autant à être sanctionné comme un débiteur insolvable à écarter de sa partenaire et de ses enfants. Il convenait plutôt de l'assister.

 On peut supposer que les séparations familiales et les privations de cette période de rationnement avaient facilité cette décision. L'un de ses objectifs a pu être de fournir une aide aux femmes et aux enfants des prisonniers de guerre retenus en Allemagne (valides certes, mais enfermés au loin et travaillant sans rémunération dans la tradition plurimillénaire de l'esclavage des vaincus). Mais il s'agissait aussi de reconnaître l'évolution des pratiques réelles des services. Preuve en est que cette réglementation n'a pas été abrogée après la guerre. 

 C'était un corollaire (paradoxal) du fait qu'on sortait peu à peu de la représentation patriarcale du monde qui dominait les siècles précédents : si les hommes n'étaient plus les patriarches tout-puissants qu'on avait pensé qu'ils étaient (ou voulu qu'ils soient), leur impuissance financière ne justifiait plus leur éviction.
 
 



%K1 démantèlement de la famille traditionnelle
%K2 Victoire du mariage d'inclination
%K3 "Le corps des femmes est à elles"
%K4 Incestes et paradoxes
%K5 Perplexités éducatives
%K6 Désarrois masculins
%K7 Inertie des pratiques
%L1 Un enfant pour quoi ? pour qui ?
%L2 Qui pour accueillir l'enfant ?
%L3 Droit à l'enfant ?
%L4 Progrès ou régressions ? 

\part{Depuis 1960, le temps des expériences}

% Le 19 mars 2015 :
% ~etc.
% Moyen Âge
% Antiquité
% " --> « ou » ou \enquote{}
% même


\chapter[Démantèlement de la famille traditionnelle]{Démantèlement\\de la famille traditionnelle}


 

\section{Révolution dans le droit}

\begin{description}

\item[1961] Une mesure administrative qui sur le moment n'a pas frappé beaucoup d'esprits, mais dont l'importance symbolique n'en est pas moins significative (les grandes fractures commencent souvent par une fissure imperceptible à l'œil nu) : le ministère de l'éducation nationale supprime le caractère obligatoire de l'enseignement du droit romain dans le programme de la licence de droit, obligation qui datait de la création des études de droit dans les universités aux \crmieme{11} et \siecle{12}s. Ce corpus n'est plus qu'une option facultative parmi d'autres. 

\item[1965] La loi du 13 juillet lève les derniers obstacles à l'exercice d'une activité commerciale par les femmes mariées sans la tutelle de leurs maris. Ceux-ci ne gèrent plus de droit les biens propres de leurs épouses (dot,~etc.). Elles n'ont plus à obtenir leur autorisation pour exercer une profession séparée, quelle qu'elle soit.

\item[1966] La loi du 11 juillet sur l'adoption assimile les enfants adoptés aux enfants légitimes non adoptés (adoption plénière). 

%\item[1966] la (même)
Cette
loi du 11 juillet ouvre le droit à l'adoption plénière à une personne seule, qu'elle soit célibataire ou non et quelles que soient ses préférences sexuelles, d'au moins 28 ans.

\item[1967] La loi \fsc{NEUWIRTH} dépénalise la prévention des naissances : elle autorise la publicité concernant les méthodes anticonceptionnelles (interdite depuis les années 20), et elle autorise leur mise à disposition du public :
%la première visée et la principale était la « pilule » anticonceptionnelle 
la « pilule » anticonceptionnelle était principalement visée,
qui venait d'être mise au point. L'accord du mari n'est pas nécessaire, son refus n'a pas d'effet.

\item[1972] La loi du 3 janvier fait entrer les enfants naturels dans la famille du ou des parents qui les ont reconnus. À quelques restrictions près -- enfants adultérins, elle leur ouvre un droit à l'héritage égal à celui des enfants légitimes.

%\item[1972] 
La puissance paternelle est abolie au profit de l'autorité parentale. En cas de séparation, cette autorité est conservée à égalité par chacun des deux parents. 

%\item[1972] 
Une loi ordonne l'égalité des salaires féminins avec les salaires masculins.

\item[1974] L'âge de la majorité légale est abaissé de 21 à 18 ans. 

\item[1975] Loi du 30 juin relative aux institutions sociales et médicosociale : les usagers et les familles doivent être associés au fonctionnement de l'établissement qui les prend en charge (il doit les « prendre en compte » ).  Le pa(ma)ternalisme des institutions d'aide sociale est un peu affaibli.

%\item[1975] 
À côté du divorce pour faute, la loi du 11 juillet ouvre la possibilité de divorcer par consentement mutuel ou pour rupture de la vie commune. Par ailleurs, cette loi met les deux époux à égalité en matière de choix résidentiel et en matière de contribution aux charges du mariage.

%\item[1975] 
La loi \fsc{WEILL} dépénalise l'avortement (\emph{interruption volontaire de grossesse} ou IVG). La loi ne demande pas l'avis des maris éventuels.

%\item[1975] 
Les épouses ne sont plus tenues de faire usage du nom de leur mari dans la vie quotidienne et les relations avec l'administration.

%\item[1975] 
L'adultère féminin est dépénalisé. Ce n'est plus un délit qui concerne la société, ce n'est qu'un affront privé qui ne concerne que le mari.

\item[1976] Loi du 22 décembre relative aux conditions d'adoption : la présence d'enfants légitimes n'est plus un obstacle à l'adoption, même si leur avis est entendu. 

\item[1978] La loi du 6 janvier donne à tout individu majeur le droit de connaître le contenu de tout dossier administratif le concernant. Cela concerne notamment tous les enfants abandonnés.

\item[1983] Un arrêt de la cour de cassation du 21 mars 1983 admet la légalité de la garde conjointe de l'enfant après divorce.

%\item[1983] 
Loi sur l'égalité professionnelle entre femmes et hommes.

\item[1984] La loi du 6 juin relative aux \emph{droits des familles dans leurs rapports avec les services chargés de la protection de la famille et de l'enfance}, et au statut des pupilles de l'État, donne aux parents des droits plus étendus face à l'administration. L'autorité des parents sur leurs enfants placés à l'ASE est confortée dans tous les domaines (sauf limites définies expressément par un juge).

\item[1985] La loi du 23 décembre 1985 met les deux parents à égalité dans la gestion des biens de l'enfant : ils exercent cette tâche conjointement quand ils exercent en commun l'autorité parentale. Sinon l'un des deux l'exerce sous le contrôle du juge.

\item[1987] L'autorité parentale est redéfinie par la loi du 22 juillet (loi \fsc{MALHURET}) en termes de \emph{responsabilité parentale ordonnée à l'intérêt de l'enfant}. Elle est à égalité assumée par chacun des deux parents, qu'ils cohabitent ou pas.

\item[1989] \emph{Convention Internationale des Droits de l'Enfant} promulguée dans le cadre de l'ONU le 20 novembre : reconnaît le droit de tout mineur à une famille, et ses droits face à sa propre famille.

%\item[1989] 
Loi du 10 juillet \emph{relative à la prévention des mauvais traitements à l'égard des mineurs et à la protection de l'enfance}. Elle prévoit que le délai de prescription ne court qu'à partir de la majorité pour les mineurs victimes de violences.

\item[1993] L'autorité parentale conjointe devient la règle pour les couples de concubins comme pour les couples mariés.

\item[1996] Convention européenne du 25 janvier sur l'exercice des droits de l'enfant. Elle donne le droit aux enfants mineurs de donner leur avis sur les mesures qui les concernent lors du divorce de leurs parents.

\item[1999] Création du PACS : pacte civil de solidarité, ouvert aux couples hétérosexuels et aux couples homosexuels.

\item[2000] La pilule {\emph{du lendemain}} est en vente libre dans les pharmacies, et distribuée gratuitement aux mineures par les infirmières scolaires sur simple demande de la mineure, sans demander l'avis de ses parents, et sans qu'ils en soient informés.

\item[2002] Sur décision de la Cour Européenne de Justice les dernières discriminations juridiques que subissaient en matière d'héritage les enfants adultérins et incestueux sont effacées. Seuls sont distingués les enfants nés des incestes parent--enfant, qui ne peuvent être reconnus que par un seul de leurs deux parents. Ils doivent néanmoins être traités absolument en tout le reste, et d'abord en ce qui concerne l'héritage, comme leurs éventuels demi-frères ou sœurs. 

%\item[2002] 
Loi du 4 mars : {[...] \emph{les parents associent l'enfant aux décisions qui le concernent, selon son âge et son degré de maturité}}. L'administration de la famille par les deux parents doit être démocratique.

\item[2005] La loi autorise les femmes à donner à leurs enfants leur propre nom à égalité avec leur mari à compter de janvier 2005. 

%\item[2005] 
Interdiction du mariage des filles avant dix-huit ans (traditionnel âge au mariage des garçons).

\item[2013] Loi \fsc{Taubira} : ouverture du mariage aux couples de même sexe.

\item[2013] Remboursement à 100~\% de l'IVG.

\item[2014] Suppression de l'exigence d'une « détresse » pour reconnaître à une femme enceinte son droit à un avortement. 
\end{description}
 
 \section{Le corps des femmes est à elles}


 La pilule anticonceptionnelle ("la pilule") a été autorisée en France quelques années seulement après sa mise au point : en 1967. Et elle l'a été par une assemblée de députés dans laquelle il n'y avait pratiquement que des pères de famille, qui ont apparemment plus pensé aux intérêts de leurs épouses et de leurs filles qu'à la défense du patriarcat. Dès ce moment la pilule a été largement utilisée par toutes les femmes majeures, célibataires ou mariées, et par bien des mineures. Grâce à elle, les femmes pouvaient prendre l'initiative d'une rencontre sexuelle sans obérer leur avenir. Cela a permis de constater que si c'était sans risque de grossesse, bien des parents ne refusaient pas que leurs filles aient une vie sexuelle hors mariage. Il n'était plus nécessaire de donner une valeur à la virginité ou à la chasteté des femmes non mariées et les filles n'étaient plus contraintes par le risque de grossesse de fuir les garçons ni de nier, de réprimer ou de refouler leurs propres désirs sexuels
\footnote{En Janvier ou février 1968, les garçons de la cité universitaire de Nanterre réclament bruyamment le droit d'entrer dans les chambres des filles de la cité, jusque là sanctuaires (en principe) inviolés. Ce fait divers, grand-guignolesque à nos yeux d'aujourd'hui, n'en a pas moins servi de détonateur à la chaîne d'évènements mémorables qui ont culminé au mois de Mai de cette année-là. Il se trouve que la disponibilité (alors toute nouvelle, mais déjà répandue comme une trainée de poudre) de la pilule anticonceptionnelle, venait d'enlever à cette revendication une part du caractère scandaleux (à tout le moins angoissant pour les pères et mères des jeunes filles) qu'elle aurait eu peu de temps auparavant. Le sexe librement recherché pour lui-même devenait un jeu sans enjeu dramatique. Ce n'est que bien après ce printemps-là que le SIDA est venu lui rendre une gravité nouvelle.}% 
. Elles ont reçu une liberté égale à celle de leurs frères, et l'âge moyen de leurs premiers rapports sexuels (autant qu'on puisse le connaître) est rapidement passé de 21 à 17 ans (comme eux).

 La « pilule » a-t-elle été inventée dès que son emploi est apparu comme acceptable ? Ou bien est-ce plutôt le contraire ? On peut en effet se demander pourquoi les préservatifs, disponibles en vente libre en pharmacie depuis le début du \siecle{20} (officiellement en tant qu'outil de prévention contre les maladies vénériennes) n'ont pas été employés en France comme un outil de prévention des naissances, alors qu'ils l'étaient dans d'autres pays ? Et pourquoi la pilule ne s'est pas heurtée à la même réticence ? 

 C'est que ce sont les femmes qui ont la maîtrise de cet outil-là : les députés leur ont en effet accordé le droit de prendre la pilule anticonceptionnelle même en cas de désaccord avec leurs maris. Elles \emph{peuvent} la prendre sans le dire à leurs partenaires. Elles \emph{peuvent} aussi cesser de la prendre sans les prévenir. Et\emph{ ils n'y peuvent rien}. Avec l'appui de la législation et des pouvoirs publics, le Planning Familial, héritier des néo-malthusiens, s'emploie à rendre effective cette liberté pour toutes les femmes, mineures comme majeures. 

 C'est dans la foulée de cette première mesure que l'avortement a été autorisé par la loi. Il ne s'agissait plus de l'avortement à la romaine : celui de l'épouse ou de l'esclave sur l'ordre du \emph{pater familias}, ou avec son accord exprès. Désormais une femme peut prendre seule l'initiative d'un avortement, en dépit du refus de son compagnon, comme elle peut garder leur enfant même s'il lui demande d'avorter\footnote{Certes en accouchant « sous X",  mode d'accouchement aux origines très anciennes et ouvert aux femmes mariées autant qu'aux autres, puisqu'elles n'ont pas à donner leur identité, les femmes ont toujours pu priver de paternité, en toute légalité, le géniteur de l'enfant qu'elles portaient, mais il est peu vraisemblable que cela ait été leur objectif losqu'elles recouraient à cette procédure et l'immense majorité des accouchements sous X a concerné et concerne encore des femmes seules et sans soutien.}.
 
  \section{Personne n'est illégitime}
 
 Aujourd'hui tout se passe comme s'il n'existait plus que des enfants légitimes : tout enfant a vocation à faire partie de la famille de chacun de ses deux géniteurs quelles que soient les circonstances de sa conception. Tout enfant a vocation à hériter de ses deux parents à égalité avec ses éventuels demi-frères et demi-sœurs. Tout enfant est un « enfant de famille ». On peut aussi bien dire que tous les enfants sont devenus « naturels » et que la notion même de légitimité s'est évaporée, réduite à un mot sans épaisseur puisqu'il n'a plus de prise sur rien, puisque les effets concrets de la légitimité sont les mêmes que ceux de l'illégitimité, et inversement.
 \begin{displayquote}
\emph{"Cette distinction légitimité-illégitimité était totalement structurante de la société. Aujourd'hui il n'est pas un pays qui n'ait soit complètement aboli cette distinction, soit s'apprête à l'abolir. C'est un changement majeur des rapports entre famille et société qui montre que nous sommes face à des changements de la structure sociale elle-même"}.
\footnote{Irène \fsc{THERY}, \enquote{Peut-on parler d'une crise de la famille ? Un point de vue sociologique}, \emph{Neuropsychiatrie de l'enfance et de l'adolescence}, 2001, 49, 492-501, p. 403.}% } 
\end{displayquote}

La cheville qui depuis \nombre{1600} ans tenait ensemble tout le système de la famille constantinienne a été retirée, et cela se passe apparemment à la satisfaction de tous. Puisque ni les parents ni les enfants ne risquent plus aucun désagrément du fait d'une naissance illégitime, à quoi bon le mariage, surtout quand on est convaincu que le seul couple légitime c'est celui qui repose sur l'accord quotidien de deux volontés. Depuis plus d'une génération, le nombre d'enfants nés hors mariage a donc progressé en même temps que croissait leur assimilation aux enfants nés dans le mariage. Aujourd'hui ils représentent la moitié des naissances. 

 Dans la France d'aujourd'hui, l'illégitimité a cessé d'être honteuse et il n'est plus socialement utile que le nom que l'on porte atteste qu'on a été reconnu par un homme. Une loi de 2001 autorise (donc ?) les couples mariés à donner à leurs enfants le patronyme de la mère à la place de celui du père, ou bien à côté de lui (texte complété par la loi \fsc{Taubira} de 2013). Ce changement est significatif, puisque la pratique traditionnelle n'était pas celle-là (contrairement à l'Espagne, par exemple) tout comme est significatif le moment où il a été institué. 


 
 \section{Victoire du mariage d'amour}


 
 Aujourd'hui, à la condition de posséder une vraie qualification professionnelle (capital intellectuel), il n'est plus besoin de capitaux pour s'établir. Rien ne vaut un « bon » métier : un métier qui implique beaucoup de savoirs et de savoir-faire, dans un secteur d'activité porteur. Il n'est plus honteux de « servir ». Au fil du \siecle{20}, le salariat s'est souvent révélé plus sûr que la possession de capitaux ou d'outils de production, surtout au service de l'État. D'autre part la scolarité est désormais gratuite ou presque jusqu'aux niveaux les plus élevés (même si ce n'est souvent pas vrai aux niveaux les plus élevés). Les parents ont intégré cette logique : depuis la Libération, le taux de scolarisation n'a cessé de s'accroître bien au-delà de la fin de l'obligation scolaire, qui elle-même est passée de 14 (1936) à 16 ans (1959). Le nombre des diplômés de l'enseignement supérieur a explosé. Le nombre de bacheliers se situe actuellement entre 60~\% et 80~\% d'une classe d'âge, contre 5~\% à la Libération et 8~\% en 1960. Même si ce diplôme s'est largement dévalué et ne peut plus depuis longtemps procurer un emploi à lui seul, le niveau de culture moyen a indiscutablement progressé. 

 Aujourd'hui la rentabilité du travail domestique a été fortement réduite par les innovations techniques, commerciales et sociales du \siecle{20} : infrastructures collectives (électricité, tout à l'égout, eau courante) ; machines qui économisent le temps de travail (chauffage central, machine à laver le linge, la vaisselle, cuisines équipées électriques ou au gaz, aspirateur,~etc.) ; grande distribution qui rend non-compétitive l'auto production en couture, en jardinage vivrier, en préparation des aliments,~etc. Sans oublier les écoles maternelles et les garderies d'enfants. Si l'on peut dire que les ménagères ont été libérées d'une grande partie du poids des tâches domestiques, cela signifie aussi qu'elles été réduites au chômage technique, ce que traduit le fait que c'est au même moment qu'on observe la fin des « bonnes ». Pour contribuer significativement aux ressources de leur ménage les épouses doivent désormais travailler au dehors de leur foyer. Cela leur a ouvert la possibilité de se trouver un autre emploi que celui d'être la « maîtresse de maison » d'un homme mais on peut aussi bien dire que cela les  y a contraintes. 
 Elles y ont d'autant plus été contraintes que depuis la libéralisation du divorce elles ne peuvent plus être assurées, comme l'étaient leurs grand-mères, de leur position d'épouse titulaire d'un homme nommément désigné. Les filles de la bourgeoisie ont compris que leur avenir serait mieux assuré par un « bon » métier que par une « belle » dot, un « beau » parti, et par le « grand » mariage qui était jusqu'alors le point de focalisation de tous les désirs familiaux, le signe et le sommet de la réussite féminine (celle des filles et celle des mères). Toutes ont compris que grâce aux savoirs et aux diplômes elles seraient libres : indépendantes des désirs d'un homme et de sa bonne volonté, à l'abri des effets matériels des répudiations, en mesure de prendre l'initiative et de sortir des situations affectives ou familiales dans lesquelles elles ne trouveraient pas leur compte. Dans la course au diplôme les filles se sont (donc ?) montrées significativement plus déterminées que les garçons (quant aux décrochages scolaires de ces derniers, leurs causes n'ont probablement rien à voir avec les motifs de la détermination des filles). 

 S'il n'est plus nécessaire pour « s'établir » d'avoir l'appui financier ni des relations de ses parents, alors le mariage ne scelle plus l'alliance (économique surtout) de deux familles : alors rien n'exige plus que les jeunes gens subissent un mariage arrangé, un mariage d'argent et d'entregent. Ils peuvent sans risque \emph{matériel} s'offrir le luxe de n'être pas raisonnables et de baser leur couple sur la seule passion amoureuse : aujourd'hui les autres stratégies ne sont pas plus raisonnables que celle-là.

 C'est en tout cas tellement devenu notre logique que cela révolutionne notre compréhension du mariage lui-même. Si celui-ci se définit d'abord comme l'union de deux personnes qui s'aiment, alors la question de la durée perd de son sens. L'authenticité des désirs inscrits dans les actes posés ici et maintenant a plus d'importance que la fidélité à une promesse ancienne. L'infidélité conjugale n'est plus une offense à un ordre public qui ne se donne plus pour but de sanctuariser les familles. Ce n'est plus qu'une offense privée, le signe d'un désaccord entre deux associés. La séduction devient une obligation permanente. L'accord du conjoint à une relation charnelle ne peut plus être tenu pour acquis d'avance, par contrat. La notion de \emph{devoir conjugal} s'est vidée de son sens, et la loi ne le reconnaît plus. La notion de viol entre époux prend du sens, et comme tout viol c'est un délit punissable par la loi. 

 Si c'est l'amour mutuel qui fonde le couple, alors sa fécondité potentielle perd de son importance. Que le couple soit constitué d'un homme et d'une femme ne va plus sans le dire. La reconnaissance publique d'un couple de deux hommes ou de deux femmes n'est plus impossible à penser. 

 Mais si c'est l'enfant qui fait la famille, et s'il héritera de ses deux parents quoi qu'il arrive, alors à quoi bon se marier ?

 
 \section{Nouveaux jugements sur les violences sexuelles}

L'abondance actuelle, depuis les années 1985-1990, des discours sur les \emph{abus} sexuels intra familiaux 
\footnote{... comme s'il y avait un usage correct du sexe entre les générations différentes au sein des familles ?} 
signifie-t-elle qu'il s'en commet plus qu'autrefois ? Si l'on en croit le témoignage de Jeannine \fsc{NOEL} (1965) il est permis d'en douter : selon elle entre le quart et le tiers des adolescentes placées à l'Hôpital Hospice Saint Vincent de Paul
\footnote{À cette époque c'était encore le Foyer de l'Enfance de Paris (anciennement « dépôt de l'Assistance Publique ») recevant (souvent avant une orientation ailleurs) tous les jeunes dont les parents ne pouvaient pas s'occuper ou de l'autorité desquels ils avaient été soustraits par décision de justice. On plaçait et place toujours dans les foyers de l'enfance les jeunes qui n'ont pas d'autre lieu où aller, quelle que soit la raison qui les a mis dans cette situation.} 
 au cours des années cinquante du \siecle{20} avaient été confrontées à des problèmes de ce genre : la situation ne semble pas être pire aujourd'hui. 
 
Par contre depuis un demi-siècle toutes les formes de violences sexuelles, qu'elles soient extra ou intra familiales, ont été regardées avec un oeil nouveau. Meme si ce n'est que très progressivement que l'on a pris conscience de la gravité de leurs effets sur leurs victimes il nous est devenu moins difficile de nous identifier aux souffrances de celles-ci. Cela s'est traduit par la requalification de certaines actes délictueux, et surtout par une nouvelle façon d'écouter les plaignants et plaignantes et d'accorder crédit à leur parole.  
 

 Alors pourquoi n'est-ce qu'aujourd'hui que le caractère absolu des secrets professionnels imposé aux professionnels susceptibles de découvrir des violences sexuelles sur mineurs a été mis en question ? Pourquoi n'est-ce qu'aujourd'hui que l'évocation des sévices intra familiaux obtient un tel effet ? Avait-on peur d'ébranler l'autorité et la représentation d'une institution familiale sacralisée, et préférait-on lui sacrifier ses victimes ? Pensait-on que ces délits et ces crimes, aussi condamnables qu'ils étaient, ne pouvaient être traités pénalement, et qu'il était préférable de les recouvrir du \emph{manteau de Noé} ? 
 
 
 \section{Désarroi masculin}


 
S'il veut une femme et/ou des enfants un homme ne peut plus s'y prendre aujourd'hui comme naguère. Il ne lui sert plus à rien de demander à un futur beau-père la main de sa fille, de lui demander un transfert d'autorité, puisque ce dernier ne la détient plus et ne peut donc plus la donner. D'ailleurs lui-même n'a plus besoin d'un gendre pour légitimer les petits enfants que sa fille lui donnera et pour en faire des héritiers, puisqu'il n'y a plus de fonctions interdites aux enfants illégitimes et donc plus d'enfants illégitimes. Il n'y a donc plus d'intérêt commun entre beau-père et gendre, et le soupirant doit négocier seul et sans intermédiaire avec la femme dont il recherche les faveurs. Il n'aura d'elle des enfants que si elle le veut bien. Et elle pourra d'autant plus facilement le quitter en emmenant leurs enfants communs (ou le pousser hors du domicile familial) que l'absence d'un homme à côté d'une mère ne fait plus problème, tandis que la présence de celle-ci semble encore indispensable\footnote{Cela changera peut-être si on constate des aptitudes au "maternage" chez les pères célibataires ou les couples homosexuels masculins ?}. 
 
 Les ressources dont disposent les hommes (puissance économique, compétences culturelles et professionnelles, pouvoir politique,~etc.) ont la vertu de les rendre désirables. Ils se doivent comme toujours d'être « ceux qui peuvent », ceux qui ne sont pas marqués par le manque ou la défaillance. Plus ils sont intellectuellement et professionnellement qualifiés, plus ils ont de probabilités d'être mariés. C'est le contraire pour les femmes, ce qui suggère que dès qu'elles ont les moyens de leur indépendance elles n'ont plus intérêt à être mariées. Cela confirme la solidité de la répartition traditionnelle des rôles masculins et féminins.
 
 On a vu que « l'obligation de résultat », l'obligation de fécondité, qui pesait sur les seules femmes mariées a été supprimée par Constantin, qui a exclu la stérilité des motifs de divorce. La loi impériale romaine a ensuite confirmé l'interdit fait aux chrétiens de se remarier après divorce. À l'obligation de fécondité des épouses s'est substituée une obligation de moyens pour chacun des deux époux : ne pas mettre d'obstacle aux fécondations autre que l'abstinence 
\footnote{En France les relevés démographiques montrent l'érosion progressive du respect de cette obligation, et l'extension depuis trois siècles des pratiques anticonceptionnelles : ce que les anciens moralistes nommaient les « \emph{funestes secrets} ».}. Si les femmes mariées ont ainsi été protégées contre la répudiation et contre la privation de leurs enfants, par contre la loi ne les autorisait pas plus qu'avant à se dérober au « devoir conjugal » lorsque leur mari l'exigeait, ni aux grossesses qui en découleraient, et à leurs risques, sauf à demander une séparation. 

 Depuis 1967, même si leurs maris le désirent, les femmes mariées ne sont plus tenues par la loi de laisser libre cours à leur fécondité. Aujourd'hui le corps des femmes est à elles, \emph{y compris l'embryon ou le fœtus, qui juridiquement en fait partie depuis 1975}, comme c'était le cas dans le droit romain antique. La loi ne se soucie plus de soutenir le désir masculin en ce domaine. Même si elles sont leurs épouses, même s'ils sont les géniteurs de l'enfant qu'elles portent, même si elles avaient été d'accord pour le concevoir avec eux, les hommes n'ont plus le droit d'exiger des femmes qu'elles donnent naissance à cet enfant. Elles peuvent choisir d'avorter ou de l'abandonner à la naissance en dépit du désir du géniteur de l'enfant. On est au plus loin du droit du \emph{pater familias} romain de faire surveiller la grossesse et l'accouchement de son épouse (ou ex épouse), pour qu'elle ne puisse pas lui dérober un enfant né de ses œuvres.

 Dans le même temps ont été supprimées toutes les limites légales qui pouvaient interdire le rattachement d'un enfant naturel à un homme. Comme sous l'ancien régime une mère qui le demande recevra toujours l'appui de la justice pour rechercher le géniteur de son enfant (sauf insémination avec donneur), quelle que soit la situation personnelle de cet homme, mais désormais cela se fera avec une efficacité  imparable. Aucun père n'est plus « \emph{incertus} ». Vivant ou mort son ADN le désignera, sauf lorsque la mère veut cacher son identité à son enfant ou aux tiers. Si la mère le veut, le géniteur sera contraint d'assumer financièrement un enfant qui héritera de lui à part entière, contrairement à ce qui se passait jusqu'au \siecle{19}. Mais cela ne lui donnera pas forcément le moindre droit sur l'éducation de l'enfant : en ce sens cela n'en fera pas un père. Si une femme qui accouche « sous X » refuse de laisser à son enfant des renseignements sur sa propre identité, elle en a le droit. 

 Pour l'essentiel, on peut donc dire que la maîtrise de la génération est passée du côté des femmes. La famille monoparentale d'aujourd'hui, c'est le plus souvent la famille \emph{moins} le père. Dans la grande majorité des séparations (85~\%) ce sont les mères qui gardent les enfants. Est-ce pour ces raisons que l'initiative des divorces vient des femmes beaucoup plus souvent (trois fois sur quatre) que des hommes  ? 




  \section{Police des familles ?}
 
 
Selon Jacques \fsc{Donzelot} (\emph{La police des familles}, 1977), nous sommes passés du gouvernement « des » familles au gouvernement « par les » familles.  Aujourd'hui le pouvoir royal des pères sur leurs enfants est mort, et celui des mères en même temps et ils ne leur reste plus que celui que l'état leur concède, à la condition qu'ils se conforment aux modèles promus par ce dernier.

\begin{displayquote}
{\emph{Ce qui caractérise la loi de 1970 (qui substitue l'autorité parentale à la puissance paternelle) ce sont trois concepts au centre de la réforme, celui « d'égalité » des époux et parents, celui « d'intérêt de l'enfant » et enfin celui de « contrôle judiciaire » devenu nécessaire pour arbitrer d'éventuels conflits entre les parents, entre parents et enfants. On assiste à un recentrage des positions de chacun des membres de la famille. Au centre l'enfant, en face de lui, responsables de lui, ses parents. Entre les deux des médiateurs, les spécialistes judiciaires}%
\footnote{Françoise \fsc{HURSTEL}, \emph{La déchirure paternelle}, p. 117.}%. 
}.
\end{displayquote}

Nous avons assisté à la délégitimation de la justice domestique, du droit des deux parents à régler eux-mêmes sans tiers extérieur tous les conflits intra familiaux. Lorsqu'ils ne réussissent pas à se mettre d'accord entre eux ou avec leurs enfants, ils sont désormais contraints (par leur égalité elle-même) à recourir à un tiers extérieur pour arbitrer leur différend.  

De nouveaux personnages se sont imposés au sein des familles. Sous l'autorité des juges les travailleurs sociaux et les experts (psychologues, psychiatres, médiateurs,~etc.) sont entrés dans le champ, jusque là bien clos, des familles ordinaires, des familles non stigmatisées au préalable comme défaillantes (en ce qui concerne les familles reconnues officiellement comme incompétentes ou délinquantes, c'est depuis toujours que les représentants de la société y avaient leurs entrées). Ils font régner la bonne parole et les bonnes pratiques et vérifient que les familles adoptent les bonnes pratiques dans la prise en charge de leurs enfants, pratiques définies par les mêmes personnages. Quand ils l'estiment nécessaire ils ont l'appui des autorités pour faire passer le message.


 

 Assiste-t-on à la disparition de la sphère privée, cette sphère de la vie de chacun qui se définit par le fait que tant qu'il n'enfreint aucune loi, il n'a aucun compte à rendre sur ce qui s'y passe, et surtout pas à l'État et à ses représentants ? 

 Tout ce qui concerne les enfants est-il entré dans le domaine public, alignant le traitement de l'ensemble des familles sur celui qui était autrefois réservé aux seuls « cas sociaux », et mettant implicitement en cause l'aptitude des parents à défendre suffisamment bien (en \enquote{\emph{bons pères de famille}}) l'intérêt de leurs propres enfants ?



\section{Inertie des pratiques}


 Dans la réalité les changements ne sont pas (encore ?) aussi importants que dans l'idée que l'on s'en fait. Depuis une génération le nombre de mariage diminue innexorablement : en 1990, 90~\% des couples existants étaient mariés, en 1999, année où le Pacs est entré dans les pratiques ils n'étaient plus que 83~\% (\emph{Histoires de familles, histoires familiales}, INSEE, 1999). 
 Le Pacs, à l'origine pensé pour les couples homosexuels, est en réalité le plus souvent choisi par des couples mixtes (dix-neuf pacs sur vingt sont contractés par eux), dont près de la moitié finit par se marier, et la somme des Pacs et des mariages est plus élevée que le nombre des seuls mariages avant la création du Pacs. Le lien entre naissances et mariage semble solide : à la naissance du deuxième enfant 86~\% des couples sont mariés, et 93~\% au troisième. On est donc, avec le Pacs, tout près d'un mariage à l'essai.

 Le nombre des divorces se situe aujourd'hui entre le tiers et la moitié de celui des mariages. Ce nombre est élevé ou bas suivant le point de vue. Sur 29~millions d'adultes vivant en couple, mariés ou non, 26~millions (90~\%) en sont \emph{toujours} à leur première expérience de couple, et pour l'instant les recompositions de familles concernent \emph{seulement} 3~millions de personnes sur 29. C'est que le nombre de couples mariés de tous âges (le « stock ») est si important que les divorces n'en représentent pas plus de 1~\% par an : 99~\% des gens qui étaient mariés au premier janvier le sont encore au 31 décembre qui suit (mais qu'en sera-t-il de ces chiffres dans une génération ?). 

 En 2006, 1,2~millions de mineurs vivaient en famille recomposée, soit 9~\% de l'ensemble des mineurs. Parmi ces mineurs, \nombre{400000} sont nés des deux membres du nouveau couple. Ceux-là vivaient donc avec leurs deux parents, bien que dans une famille "recomposée". À la même date, 2,2~millions de mineurs vivaient au sein d'une famille monoparentale (six fois sur sept avec leur mère), tandis que 10,25~millions de mineurs
vivaient avec leur père et leur mère (dont les \nombre{400000} enfants vivant au sein de familles recomposées et nés du couple nouveau). 
 


\newlength{\lcol}
\setlength{\lcol}{0.16666667\textwidth}
\addtolength{\lcol}{-2\tabcolsep}


\begin{table}[!ht]% [!htb]
%\centering
\begin{minipage}{\textwidth} 
\caption[Cadre de vie des jeunes en 1999]%
{Cadre de vie des jeunes en 1999%
\footnote{Sources :
« Histoires de familles, histoires familiales », \emph{Les cahiers de l'INED}, \no 156 ;
\emph{Recensement de la population}, INSEE, 1999, p. 281.} }
\label{tableau-cadre-vie-1999}
\begin{tabular}{*{6}{>{\hspace{0pt}\centering\arraybackslash}b{\lcol}}}
Âge des jeunes (années) & Vivant avec les deux parents de naissance & Avec un parent seul%
\footnote{Familles monoparentales.}
 & Avec un parent et un beau-parent%
\footnote{Familles recomposées.}
 & Autres situations%
\footnote{En internat, en appartement, en chambre, chez un logeur, en placement ASE, en prison, en hôpital,~etc.}
 & Total\\
\hline
 0-4     & 85,0 & 11,1 & 1,8 & 2,2  & 100~\% \\
 5-9     & 77,7 & 15,6 & 5,2 & 1,5  & 100~\% \\
 10-14 & 72,7 & 17,5 & 8,4 & 1,5  & 100~\% \\
 15-19 & 68,5 & 18,7 & 8,6 & 4,1  & 100~\% \\
 20-24 & 43,5 & 11,5 & 4,3 & 40,6 & 100~\% \\
\hline
 0-17  & 76,5 & 15,7 & 6,0 & 1,8  & 100~\%
\end{tabular}
\end{minipage}
\end{table}

%CADRE DE VIE DES JEUNES EN 1999[6]
% 
%\emph{Age des jeunes}
%\emph{ (années)}
%\emph{Vivant avec ses deux parents de naissance}
%\emph{Avec un parent seul[7]}
%\emph{Avec un parent et un beau-parent[8]}
%\emph{Autres situations [9]}
%\emph{Total}
%\emph{0-4}

  

\makeatletter
\if@twoside
\begin{table}[t]% [!htb]
\else
\begin{table}[!t]% [!htb]
\fi
\makeatother
%\centering




\begin{minipage}{\textwidth} 
\caption[Cadre de vie des jeunes en 2004-2007]%
{Cadre de vie des jeunes en 2004-2007%
\footnote{Source : \emph{Moyenne annuelle des enquêtes emploi de 2004 à 2007}, INSEE.} }



\label{tableau-cadre-vie-2004-2007}

\begin{tabular}{*{6}{>{\hspace{0pt}\centering\arraybackslash}b{\lcol}}}
Âge des jeunes (années) & Vivant avec les deux parents de naissance & Avec un parent seul & Avec un parent et un beau-parent & Autres situations & Total\\
\hline
 0-6     & 82,2 & 10,1 & 7,2 & 0,5  & 100~\% \\
 7-13   & 72,8 & 16,6 & 9,9 & 0,7  & 100~\% \\
 14-17 & 66,9 & 19,0 & 9,8 & 4,4  & 100~\%
\end{tabular}

\end{minipage}

\end{table}

%CADRE DE VIE DES JEUNES EN 2004/2007[11]
% 
%\emph{Age des jeunes (années)}
%\emph{Vivant avec ses deux parents de naissance}
%\emph{Avec un parent seul[12]}
%\emph{Avec un parent et un beau-parent}
%\emph{Autres situations[13]}
%\emph{Total}
%\emph{0-6}
 
 
 
 
 L'évolution des comportements n'a rien de fulgurant. Vivre séparé de l'un de ses deux géniteurs reste une situation minoritaire : pour l'instant les trois quarts des mineurs vivent sous le même toit que leurs \emph{deux} parents \emph{de naissance} (dont les deux tiers des mineurs de 15 ans à 18 ans).
 
 Mais les évolutions actuelles sont aussi des évolutions symboliques : il n'y a peut-être (à vérifier) jamais eu autant d'enfants qu'aujourd'hui à vivre jusqu'à leur majorité avec leur père et leur mère de naissance et pourtant les familles ne sont plus pensées comme l'alliance irréversible de deux lignées, ni comme des institutions aux limites intangibles, mais comme des associations d'individus à géométrie variable. Les enfants d'aujourd'hui apprennent très tôt que les couples mixtes sont fragiles, qu'on rencontre aussi des couples mariés de même sexe, qu'amour ne rime pas avec toujours, que les princes et les princesses n'ont pas forcément beaucoup d'enfants, et qu'ils se séparent souvent avant la fin de leur histoire. Ils apprennent à dissocier parentalité et conjugalité, ou plutôt ils n'apprennent plus à les associer de manière indéfectible. À côté des scénarii traditionnels de leurs jeux d'imagination (le gendarme et le voleur, le client et la marchande, le malade et le docteur, l'indien et le cow-boy,~etc.) ils disposent maintenant du jeu du mariage et du divorce.

 Sous l'Ancien Régime c'était le contraire : en droit civil comme en droit canon, les mariages étaient indissolubles. Par contre, la mortalité d'alors, très élevée par rapport à celle d'aujourd'hui, faisait que plus de la moitié des époux étaient séparés par la mort avant même que leurs enfants n'aient atteint leurs vingt ans, et à cet âge il était normal d'être orphelin d'au moins un de ses deux parents. La durée moyenne effective des couples conjugaux était faible, environ quinze ans, comparée à celle d ceux des couples d'aujourd'hui qui n'ont pas divorcé, autour de cinquante ans. 

 La Révolution avait autorisé et facilité le divorce \emph{par consentement mutuel}, et à la suite de cette décision le taux de divorces observé dans les villes (mais \emph{seulement dans les villes}) avait rapidement atteint le niveau actuel. Mais contrairement à ce qui s'était passé dès l'an~III, aujourd'hui personne ne semble s'en inquiéter. Personne ne se donne plus pour objectif d'enrayer ce phénomène comme ce fut le cas avec le Code Napoléon, pendant la plus grande partie du \siecle{19} et sous le régime de Vichy (1940-45). Il ne s'agit plus de punir un coupable, ou deux, ni de chercher à prouver aux conjoints qu'ils peuvent respecter leurs engagements conjugaux au prix de quelques accommodements. Au contraire, les lois accompagnent ce mouvement de « \emph{démariage}
\footnote{Cf. Irène \fsc{THERY}, et son livre du même nom.}, et le divorce par consentement mutuel est devenu le modèle du bon divorce. 

 C'est en majeure partie du fait des divorces que les personnes seules avec enfants ont crû en nombre et en visibilité depuis 1970. En effet, le pourcentage de veufs et de veuves en leur sein a beaucoup baissé, au contraire de celui des divorcés : 9 fois sur 10 il s'agit de femmes seules avec enfants. 
 


 


 
\section{Problèmes de transmission}


 Les décideurs du passé n'avaient guère de difficultés à se mettre d'accord sur l'éducation des enfants et adolescents : jusqu'au milieu du \siecle{20} a régné dans le domaine éducatif un assez grand consensus autour de règles communes et de limites peu ou pas discutées. Cela se traduisait par exemple par le fait qu'au même moment les établissements éducatifs du \crmieme{19} ou du \siecle{20} présentaient partout à des nuances près les mêmes modes de fonctionnement, les mêmes limites, la même séparation des sexes, les mêmes styles de communication, qu'ils se réfèrent à un corps de doctrine religieux ou à une laïcité stricte. Les décideurs ne doutaient pas de leur droit à imposer leurs analyses aux parents qu'ils jugeaient négligents, délinquants, ou mal pensants. Ils valorisaient l'éducation au sein de la famille, mais au nom même de celle-ci ils plaçaient sans hésiter les enfants loin de leurs parents lorsque les conditions de leur éducation leur paraissaient compromises. C'est pourquoi l'époque actuelle est atypique en ce que depuis un bon demi-siècle elle hésite dans l'idée qu'elle se fait de l'intérêt de l'enfant, dans le choix de ce qu'elle veut lui transmettre, et dans les modalités de la transmission. 

 Il n'existe pas de savoir scientifique sur ce qu'est l'intérêt de l'enfant : il existe certes des connaissances scientifiques de plus en plus affinées sur les liens entre telle mesure éducative et tel résultat, telle performance, tel taux de morbidité,~etc. mais l'intérêt de l'enfant c'est bien autre chose. Il est lié aux fins que les hommes se donnent, qui ne sont pas scientifiques, mais politiques, philosophiques, religieuses, éthiques... 

 En l'absence de croyance partagée sur ce qu'est l'intérêt de l'enfant, l'accord minimal se fait sur l'idée qu'il faut avant tout ne pas lui nuire. Le reproche majeur qu'encourt un éducateur ce n'est plus de le « gâter » par sa complaisance et par son manque de fermeté. Le principal des risques actuels de son métier, c'est d'être accusé de le maltraiter par des exigences excessives. Parfois les enfants semblent assimilés à une population à libérer (par le droit) de l'arbitraire oppressif que les parents, les enseignants et les institutions d'assistance et de rééducation exerceraient sur eux%
% [1]
\footnote{Références : 
\\Gérard \fsc{MENDEL}, \emph{Pour décoloniser l'enfant, socio psychanalyse de l'autorité}, 1971.
\\Alain \fsc{RENAUT}, \emph{La libération des enfants, contribution philosophique à une histoire de l'enfance}, 2002.}% 
.

 Les « \emph{événements de mai 1968} » ont rendu visible et accéléré une remise en question des institutions, des hiérarchies et de l'argument d'autorité, mise en question qui avait commencé bien avant : avec les « maîtres du soupçon » ? Dès 1889 et la possibilité de déchoir les pères indignes ? Dès la Révolution française et l'exécution de Louis~XVI ? Avec les Lumières et Rousseau ? Avec la Renaissance et Rabelais,~etc. ? C'est aussi que bien des personnes et des institutions revêtues d'autorité avaient montré leurs propres limites, au nom de l'ordre et de l'obéissance, au cours des guerres et sous les régimes totalitaires dont le \siecle{20} a connu quelques beaux exemples, tandis que la psychanalyse soulignait la place centrale du désir du sujet dans son propre développement. D'autre part différentes recherches et expériences scientifiques avaient montré que les méthodes autoritaires d'éducation pouvaient être nocives, et que les méthodes non autoritaires de direction des groupes (Moren, Rogers) comme les pédagogies basées sur la découverte et l'initiative (Freinet, Montessori,~etc.) pouvaient être plus efficaces que les autres.

 Lorsqu'un anonyme inspiré a écrit sur un mur : « \emph{il est interdit d'interdire} », cette proposition jaillie d'on ne sait où a donc été reprise comme une évidence. Pour quelle raison ce paradoxe a-t-il à ce point fait vibrer la génération née au sortir de l'occupation ? ... sans doute parce qu'il était le corollaire d'un autre slogan aussi fameux de la même période : « \emph{jouissons sans entraves} ». 

 Les enfants se sont vus reconnaître le droit à la parole sur tout ce qui les concerne, notamment leurs orientations, mais aussi le droit à une vie sexuelle active dès l'âge de quinze ans (y compris le droit au secret médical, y compris pour les filles le droit si elles le désirent de mener à bien une grossesse ou de choisir une IVG,~etc.) tandis que leurs parents ont été sommés par la loi de les conduire démocratiquement vers une indépendance aussi précoce que possible (sauf dans le diomaine financier). 

 Depuis 1882, la durée de l'obligation scolaire s'est allongée et l'âge minimum de la mise au travail est passé de douze à seize ans afin de donner aux jeunes le maximum de chances d'insertion, de les protéger de toute exploitation au travail, et d'écrêter les différences entre milieux scolaires et sociaux différents. Depuis la création du Collège unique (\emph{Réforme \fsc{Haby}}, 1975), tous les enfants bénéficient de ce qui était un privilège jusqu'aux années soixante du \siecle{20}. Les études longues sont plus que jamais la voie royale vers la réussite personnelle. Grâce à leur quasi gratuité, tous ceux qui en ont les moyens intellectuels et le désir (et aussi des parents suffisamment aisés pour subvenir à leurs besoins matériels jusqu'à la fin de leurs études) ont des chances sérieuses de pouvoir en faire. Quant à savoir s'ils sont contents de l'extension de l'âge de 12 ans à 14 ans, puis à 16 ans, de leurs 5 ou 7 heures journalières de fréquentation des enseignants, il est assez évident que nombre d'entre eux, et notamment de garçons, ne la vivent pas bien et le font bruyamment savoir.

 Dans un contexte de concurrence scolaire généralisée, les richesses financières et culturelles des parents ne peuvent plus suppléer aussi massivement qu'autrefois à l'incompétence d'un jeune ou à son absence d'implication personnelle (même si elles jouent beaucoup). C'est pourquoi même s'ils sont toujours soucieux de l'avenir de leurs enfants, la pression qu'ils exercent a changé de lieu d'application : du contrôle rigoureux de leur sexualité pré conjugale, autrefois impératif pour leur futur établissement, et désormais sans importance, à l'exigence de performances scolaires aussi brillantes que possible, désormais sans alternative. C'est que rien n'a changé, bien au contraire, dans les règles du jeu qui permettent d'accéder aux meilleures sections des grands lycées et aux plus réputées des grandes écoles françaises et par là aux emplois les mieux payés, les plus attrayants ou les plus influents. Les jeunes n'ont donc plus guère à réprimer leurs désirs sexuels ni à supporter la culpabilité qui s'y attachait, devant un dieu ou devant leurs parents (sauf sans doute les jeunes aux tendances homosexuelles). Par contre il leur faut satisfaire à des normes exigeantes d'autonomie, de productivité intellectuelle et de compétitivité. Ceux qui n'y parviennent pas vivent une « honte » qui peut être au moins aussi insupportable que les anciennes culpabilités. Il n'y a sans doute pas moins de pression parentale aujourd'hui qu'autrefois, et il n'est peut-être pas plus agréable d'y être soumis, ni plus facile d'y satisfaire... 

 Sans parler de la responsabilité qui repose sur les épaules des enfants sur qui l'on compte, à défaut d'autre liens, pour donner sens à la vie de leurs parents :

\begin{displayquote}
\emph{Encore plus importante, naturellement, cette question : qu'est-ce qu'un enfant ? Le paradoxe est ici encore plus important car on n'a jamais autant prêté attention à l'enfant, on ne s'est jamais autant soucié de lui et on n'a jamais autant désenfantisé l'enfant.}
 
\emph{Désenfantiser l'enfant, comme s'il n'était possible de le concevoir comme notre égal qu'en le concevant comme notre semblable}[...]

\emph{L'enfant soutien de famille : ceci évoque un renversement tout à fait fondamental. \emph{[...]} la parentification des enfants dans les familles recomposées, c'est-à-dire un mouvement nouveau où, de façon tout à fait inattendue, la prise en compte de l'enfance aboutit à un déni d'enfance et où l'infantilisation du monde des adultes aboutit à une parentification du monde des enfants.}

 [... cette question encore paradoxale :] \emph{est-ce à l'enfant de dire qui appartient ou qui n'appartient pas à sa propre famille ?}%
% [2]
\footnote{Irène \fsc{THERY}, « Peut-on parler d'une crise de la famille ? un point de vue sociologique », \emph{Neuropsychiatrie de l'enfance et de l'adolescence}, 2001, 49, 492-501.} 
\end{displayquote}



 
 
 

 
 


\chapter{Victoire du mariage d'amour}


 Jusqu'au \siecle{19}, le premier objectif des jeunes gens raisonnables n'était pas tant de vivre mieux que leurs parents et de s'enrichir, que de réussir au moins à reproduire le même mode de vie qu'eux, et de ne pas tomber dans l'indigence. Pour cela, ils n'avaient le choix qu'entre un mariage arrangé par leurs parents, ou pas de mariage du tout, et si le plus souvent (pas toujours) il valait mieux à tous points de vue se marier que ne pas le faire, il leur fallait aussi éviter de compromettre, par enthousiasme naïf, par imprudence ou par sottise, les bases économiques de leur futur couple et le statut social de leurs enfants à venir. 

 Selon les moralistes d'alors le choix du mariage d'inclination, fondé sur l'amour passion et non sur la raison (c'est-à-dire l'intérêt) était la marque des imprévoyant(e)s. Entre mariage d'inclination et concubinage les liens paraissaient évidents. C'est ainsi que s'unissaient ceux qui ne possédaient que leurs bras, les ouvriers, les manœuvres, les valets, les ouvrières et les servantes, etc. Ceux qui se mettaient en ménage avant d'avoir « assis » leur « situation » se condamnaient à « tirer le diable par la queue ». Selon les mêmes moralistes, avec lesquels faisaient chorus tous les parents angoissés (et dans beaucoup de moralistes, même célibataires, il y a un parent angoissé), la soumission des jeunes imprévoyants à leurs appétits charnels leur faisait courir le risque de gâcher leur vie, de connaître la misère et de perdre un jour la main sur leurs propres enfants, ainsi qu'il en avait toujours été depuis le début du monde. 

 Ils risquaient en effet de ne pas pouvoir les élever et de devoir les abandonner aux institutions d'assistance. Ils ne pourraient les « établir », ni en leur donnant un capital matériel, ni en finançant leur apprentissage professionnel auprès d'un maître qualifié, ni en les mettant à l'école, même gratuite, puisqu'ils seraient contraints de les placer chez un maître dès que leur âge le permettrait. En cas de chômage et de disette, ils les enverraient mendier. Ils ne pourraient pas compter sur ces enfants, condamnés à être pauvres à leur tour, pour soutenir leur propre vieillesse. Ils risquaient de finir leurs jours dans la solitude et la misère, affective et matérielle, des hospices.

 Au contraire les parents prévoyants établissaient leurs enfants dans un mariage profitable grâce à leurs économies, à leurs relations et à des stratégies complexes : échanges simultanés et réciproques d'enfants, de terres, de droits d'exploitation, d'entreprises, de gérances, d'offices (ministériels), etc. sans compter jusqu'au \siecle{18} l'entrée en religion plus ou moins volontaire de ceux qu'ils ne pouvaient ou ne voulaient pas marier de manière conforme à leur milieu social. 

 Ces stratégies complexes ne pouvaient pas toujours tenir compte des préférences sexuelles ou amoureuses de chacun, et on n'en faisait pas grief aux parents. Les femmes s'en consoleraient avec leurs enfants ou la religion, les hommes avec le travail, le pouvoir, les prostituées ou les maîtresses (le recours à celles-là et aux "maisons closes" étant toujours préférable, du point de vue des épouses, au choix de celles-ci). Les patrimoines étaient verrouillés contre les effets des infidélités des uns et des autres. Une épouse ne pouvait introduire d'enfant adultérin dans sa famille que si son mari le voulait bien, mais en ce cas la paternité de celui-ci devenait absolument inattaquable : le géniteur n'avait aucun recours. Quant aux enfants illégitimes du mari, ils ne pouvaient en aucun cas être légitimés ni menacer l'héritage des enfants légitimes. Les épouses pouvaient dormir tranquilles (sur ce point en tout cas) même quand leurs maris découchaient. 

 La pérennité des couples raisonnables était favorisée par la synergie des ressources que leurs familles respectives avaient sagement et laborieusement conjointes. Leurs parents étaient les premiers à tenir fermement à ce qu'ils, et elles plus encore, ne mettent pas ces arrangements en danger par des comportements imprudents ou des passions irréfléchies, d'où leur accord profond avec les autorités morales et religieuses de l'époque. L'intérêt matériel des époux était le plus souvent de rester ensemble, quitte à accepter des renoncements ou des compromis sur les vrais désirs de chacun, et à promouvoir comme un des fondements du savoir-vivre une dose convenable d'hypocrisie : d'ailleurs, dès l'antiquité païenne, il était très inconvenant d'afficher publiquement une affection trop vive entre conjoints. 

 Certes, l'impossibilité de placer les préférences individuelles avant tout autre critère pouvait faire souffrir, et l'amour passion comme la liberté de choix du conjoint faisaient rêver. Les œuvres littéraires du passé reflètent la prégnance de ces représentations. Ainsi, pour ne prendre qu'un seul exemple, la plupart des intrigues de Molière reposent sur le refus d'un mariage arrangé. les romans de Jane Austen surnagent comme des modèles parmi des milliers d'autres fondés sur les "problèmes de coeur" de jeunes gens et surtout de jeunes filles, apparemment libres de leurs choix, plus libres en Angleterre qu'en France au moins au premier regard, et en réalité excessivement contraints. Les gens raisonnables savaient que le choix du conjoint n'était habituellement pas libre. Ce n'était que du rêve. Les contraintes économiques étaient indépassables, en dépit des souffrances et des renoncements qu'elles entraînaient. Cela n'empêchait pas la société de continuer siècle après siècle à fonctionner sur le même mode. 

 Aujourd'hui, à la condition de posséder une vraie qualification professionnelle (capital intellectuel), il n'est plus besoin de capitaux pour s'établir. Rien ne vaut un « bon » métier : un métier qui implique beaucoup de savoirs et de savoir-faire, dans un secteur d'activité porteur. Il n'est plus honteux de « servir ». Au fil du \siecle{20}, le salariat s'est souvent révélé plus sûr que la possession de capitaux ou d'outils de production, surtout au service de l'État. D'autre part la scolarité est désormais gratuite ou presque jusqu'aux niveaux les plus élevés (même si ce n'est souvent pas vrai aux niveaux les plus élevés). Les parents ont intégré cette logique : depuis la Libération, le taux de scolarisation n'a cessé de s'accroître bien au-delà de la fin de l'obligation scolaire, qui elle-même est passée de 14 (1936) à 16 ans (1959). Le nombre des diplômés de l'enseignement supérieur a explosé. Le nombre de bacheliers se situe actuellement entre 60~\% et 80~\% d'une classe d'âge, contre 5~\% à la Libération et 8~\% en 1960. Même si ce diplôme s'est largement dévalué et ne peut plus depuis longtemps procurer un emploi à lui seul, le niveau de culture moyen a indiscutablement progressé. 

 Au \siecle{19}, un homme dépensait plus s'il était célibataire que s'il était marié, sauf à employer une « bonne à tout faire ». Il était plus rentable d'entretenir une « ménagère » à domicile que de manger tous les jours au restaurant, de faire blanchir son linge,~etc. En dehors de sa dot (très mince ou inexistante dans les milieux populaires), une épouse fournissait gratuitement une somme de prestations qu'il eût été coûteux de se procurer sur le marché. Aujourd'hui la rentabilité du travail domestique a été fortement réduite par les innovations techniques, commerciales et sociales du \siecle{20} : infrastructures collectives (électricité, tout à l'égout, eau courante) ; machines qui économisent le temps de travail (chauffage central, machine à laver le linge, la vaisselle, cuisines équipées électriques ou au gaz, aspirateur,~etc.) ; grande distribution qui rend non-compétitive l'auto production en couture, en jardinage vivrier, en préparation des aliments,~etc. Sans oublier les écoles maternelles et les garderies d'enfants. 

 Si les ménagères ont été libérées d'une grande partie du poids des tâches domestiques, elles ont en même temps été réduites au chômage technique : c'est au même moment qu'on observe la fin des « bonnes ». Pour contribuer significativement aux ressources de leur ménage elles doivent désormais travailler au dehors de leur foyer. Cela leur a ouvert la possibilité de -- mais on peut tout aussi bien dire que cela les a contraintes à -- se trouver d'autres emplois que d'être la « maîtresse de maison » titulaire d'un homme. 

 Elles y ont d'autant plus été poussées, que depuis la libéralisation du divorce elles ne peuvent plus être assurées, comme l'étaient leurs grand-mères, de leur position d'épouse en titre d'un homme nommément désigné. Les filles de la bourgeoisie ont compris que leur avenir serait mieux assuré par un « bon » métier que par une « belle » dot, un « beau » parti, et par le « grand » mariage qui était jusqu'alors le point de focalisation de tous les désirs familiaux, le signe et le sommet de la réussite féminine. Toutes ont compris que grâce aux savoirs et aux diplômes elles seraient libres : indépendantes des désirs d'un homme et de sa bonne volonté, à l'abri des effets matériels des répudiations, en mesure de prendre l'initiative et de sortir des situations affectives ou familiales dans lesquelles elles ne trouveraient pas leur compte. Dans la course au diplôme les filles se sont (donc ?) montrées significativement plus déterminées que les garçons (quant aux décrochages scolaires de ces derniers, ils ont bien entendu d'autres causes). 

 S'il n'est plus nécessaire pour « s'établir » d'avoir l'appui financier de ses parents ni de leurs relations, alors le mariage ne scelle plus l'alliance (économique et éventuellement politique) de deux familles : alors rien n'exige plus que les jeunes gens subissent un mariage arrangé, un mariage d'argent et d'entregent. Ils peuvent sans risque \emph{matériel} excessif s'offrir le luxe de n'être pas raisonnables et de baser leur couple sur la seule passion amoureuse.

 C'est en tout cas tellement devenu notre logique que cela révolutionne notre compréhension du mariage lui-même. Si celui-ci se définit d'abord comme l'union de deux personnes qui s'aiment, alors la question de la durée perd de son sens. L'authenticité des désirs inscrits dans les actes posés ici et maintenant a plus d'importance que la fidélité à une promesse ancienne. L'infidélité conjugale n'est plus une offense à un ordre public qui ne se donne plus pour but de sanctuariser les familles. Ce n'est plus qu'une offense privée, le signe d'un désaccord entre deux associés. La séduction devient une obligation permanente. L'accord du conjoint à une relation charnelle ne peut plus être tenu pour acquis d'avance, par contrat. La notion de \emph{devoir conjugal} s'est vidée de son sens, et la loi ne le reconnaît plus. La notion de viol entre époux prend du sens, et comme tout viol c'est un délit punissable par la loi. 

 Si c'est l'amour mutuel qui fonde le couple, alors sa fécondité potentielle perd de son importance. Que le couple soit constitué d'un homme et d'une femme ne va plus sans le dire. La reconnaissance publique d'un couple de deux hommes ou de deux femmes n'est plus impossible à penser. 

 Mais si c'est l'enfant qui fait la famille, à quoi bon se marier ?


% 28.02.2015 :
% ~etc.
% Moyen Âge
% _, --> ,
% Antiquité




 









\chapter{Perplexités éducatives}


 Les décideurs du passé n'avaient guère de difficultés à se mettre d'accord sur l'éducation des enfants et adolescents : jusqu'au milieu du \siecle{20} a régné dans le domaine éducatif un assez grand consensus autour de règles communes et de limites peu ou pas discutées. Cela se traduisait par exemple par le fait qu'au même moment les établissements éducatifs du \crmieme{19} ou du \siecle{20} présentaient partout à des nuances près les mêmes modes de fonctionnement, les mêmes limites, la même séparation des sexes, les mêmes styles de communication, qu'ils se réfèrent à un corps de doctrine religieux ou à une laïcité stricte. Les décideurs ne doutaient pas de leur droit à imposer leurs analyses aux parents qu'ils jugeaient négligents, délinquants, ou mal pensants. Ils valorisaient l'éducation au sein de la famille, mais au nom même de celle-ci ils plaçaient sans hésiter les enfants loin de leurs parents lorsque les conditions de leur éducation leur paraissaient compromises. C'est pourquoi l'époque actuelle est atypique en ce que depuis un bon demi-siècle elle hésite dans l'idée qu'elle se fait de l'intérêt de l'enfant, dans le choix de ce qu'elle veut lui transmettre, et dans les modalités de la transmission. 

 Il n'existe pas de savoir scientifique sur ce qu'est l'intérêt de l'enfant : il existe certes des connaissances scientifiques de plus en plus affinées sur les liens entre telle mesure éducative et tel résultat, telle performance, tel taux de morbidité,~etc. mais l'intérêt de l'enfant dépend de bien autre chose. Il est lié aux fins que les hommes se donnent, qui ne sont pas scientifiques, mais politiques, philosophiques, religieuses, éthiques... 

 En l'absence de croyance partagée sur ce qu'est l'intérêt de l'enfant, l'accord minimal se fait sur l'idée qu'il faut avant tout ne pas lui nuire. Le reproche majeur qu'encourt un éducateur ce n'est plus de le « gâter » par sa complaisance et par son manque de fermeté. Le principal des risques actuels de son métier, c'est d'être accusé de le maltraiter par des exigences excessives. Parfois les enfants semblent assimilés à une population à libérer (par le droit) de l'arbitraire oppressif que les parents, les enseignants et les institutions d'assistance et de rééducation exerceraient sur eux%
% [1]
\footnote{Références : 
\\Gérard \fsc{MENDEL}, \emph{Pour décoloniser l'enfant, socio psychanalyse de l'autorité}, 1971.
\\Alain \fsc{RENAUT}, \emph{La libération des enfants, contribution philosophique à une histoire de l'enfance}, 2002.}% 
.

 Les « \emph{événements de mai 1968} » ont rendu visible et accéléré une remise en question des institutions, des hiérarchies et de l'argument d'autorité, mise en question qui avait commencé bien avant : avec les « maîtres du soupçon » ? Dès 1889 et la possibilité de déchoir les pères indignes ? Dès la Révolution française et l'exécution de Louis~XVI ? Avec les Lumières et Rousseau ? Avec la Renaissance et Rabelais,~etc. ? C'est aussi que bien des personnes et des institutions revêtues d'autorité avaient montré leurs propres limites, au nom de l'ordre et de l'obéissance, au cours des guerres et sous les régimes totalitaires dont le \siecle{20} a connu quelques beaux exemples, tandis que la psychanalyse soulignait la place centrale du désir du sujet dans son propre développement. D'autre part différentes recherches et expériences scientifiques avaient montré que les méthodes autoritaires d'éducation pouvaient être nocives, et que les méthodes non autoritaires de direction des groupes (Moren, Rogers) comme les pédagogies basées sur la découverte et l'initiative (Freinet, Montessori,~etc.) pouvaient être plus efficaces que les autres.

 Lorsqu'un anonyme inspiré a écrit sur un mur : « \emph{il est interdit d'interdire} », cette proposition jaillie d'on ne sait où a donc été reprise comme une évidence. Pour quelle raison ce paradoxe a-t-il à ce point fait vibrer la génération née au sortir de l'occupation ? ... sans doute parce qu'il était le corollaire d'un autre slogan aussi fameux de la même période : « \emph{jouissons sans entraves} ». 

 Les enfants se sont vus reconnaître le droit à la parole sur tout ce qui les concerne, notamment leurs orientations, mais aussi le droit à une vie sexuelle active dès l'âge de quinze ans (y compris récemment le droit au secret médical, et y compris pour les filles le droit de mener à bien une grossesse ou de choisir une IVG,~etc.) tandis que leurs parents ont été sommés par la loi de les conduire démocratiquement vers une indépendance aussi précoce que possible. 

 Depuis 1882, la durée de l'obligation scolaire s'est allongée et l'âge minimum de la mise au travail est passé de douze à seize ans afin de donner aux jeunes le maximum de chances d'insertion, de les protéger de toute exploitation au travail, et d'écrêter les différences entre milieux scolaires et sociaux différents. Depuis la création du Collège unique (\emph{Réforme \fsc{Haby}}, 1975), tous les enfants bénéficient de ce qui était un privilège jusqu'aux années soixante du \siecle{20}. Les études longues sont plus que jamais la voie royale vers la réussite personnelle. Grâce à leur quasi gratuité, tous ceux qui en ont les moyens intellectuels et le désir (et aussi des parents suffisamment aisés pour subvenir à leurs besoins matériels jusqu'à la fin de leurs études) ont des chances sérieuses de pouvoir en faire. Quant à savoir s'ils sont contents de l'extension de l'âge de 12 ans à 14 ans, puis à 16 ans, de leurs 5 ou 7 heures journalières de fréquentation des enseignants, il est assez évident que nombre d'entre eux, et notamment de garçons, ne la vivent pas bien et le font bruyamment savoir.

 Dans un contexte de concurrence scolaire généralisée, les richesses financières et culturelles des parents ne peuvent plus suppléer aussi massivement qu'autrefois à l'incompétence d'un jeune ou à son absence d'implication personnelle. C'est pourquoi même s'ils sont toujours soucieux de l'avenir de leurs enfants, la pression qu'ils exercent a changé de lieu d'application : du contrôle rigoureux de leur sexualité pré conjugale, autrefois impératif pour leur futur établissement, et désormais sans importance, à l'exigence de performances scolaires aussi brillantes que possible, désormais sans alternative. C'est que rien n'a changé, bien au contraire, dans les règles du jeu qui permettent d'accéder aux meilleures sections des grands lycées et aux plus réputées des grandes écoles françaises et par là aux emplois les mieux payés, les plus attrayants ou les plus influents. Les jeunes n'ont donc plus guère à réprimer leurs désirs sexuels ni à supporter la culpabilité qui s'y attachait, devant un dieu ou devant leurs parents. Par contre il leur faut satisfaire à des normes exigeantes d'autonomie, de productivité et de compétitivité. Ceux qui n'y parviennent pas vivent une « honte » qui peut être au moins aussi insupportable que les anciennes culpabilités. Il n'y a sans doute pas moins de pression parentale aujourd'hui qu'autrefois, et il n'est peut-être pas plus agréable d'y être soumis, ni plus facile d'y satisfaire... 

 Sans parler de la responsabilité qui repose sur les épaules des enfants sur qui l'on compte, à défaut d'autre liens, pour donner sens à la vie de leurs parents :

\begin{displayquote}
\emph{Encore plus importante, naturellement, cette question : qu'est-ce qu'un enfant ? Le paradoxe est ici encore plus important car on n'a jamais autant prêté attention à l'enfant, on ne s'est jamais autant soucié de lui et on n'a jamais autant désenfantisé l'enfant.}
 
\emph{Désenfantiser l'enfant, comme s'il n'était possible de le concevoir comme notre égal qu'en le concevant comme notre semblable}[...]

\emph{L'enfant soutien de famille : ceci évoque un renversement tout à fait fondamental. \emph{[...]} la parentification des enfants dans les familles recomposées, c'est-à-dire un mouvement nouveau où, de façon tout à fait inattendue, la prise en compte de l'enfance aboutit à un déni d'enfance et où l'infantilisation du monde des adultes aboutit à une parentification du monde des enfants.}

 [... cette question encore paradoxale :] \emph{est-ce à l'enfant de dire qui appartient ou qui n'appartient pas à sa propre famille ?}%
% [2]
\footnote{Irène \fsc{THERY}, « Peut-on parler d'une crise de la famille ? un point de vue sociologique », \emph{Neuropsychiatrie de l'enfance et de l'adolescence}, 2001, 49, 492-501.} 
\end{displayquote}




\chapter{Désarrois masculins}


 Notre retour sur l'histoire montre à quel point la situation actuelle est révolutionnaire. Le mariage avait pour but, essentiel sinon unique, de fabriquer des pères et de leur donner des enfants. L'ancien Droit romain semblait réduire les femmes à n'être que des ventres au service des hommes. Ceux-ci répudiaient celles qui ne leur avaient pas donné les héritiers qu'ils voulaient, ou prenaient des concubines. En cas de séparation, le Droit leur attribuait systématiquement tous leurs enfants%
% [1]
\footnote{... ce qu'il a fait jusqu'au milieu du \siecle{19} dans les pays anglo-saxons.}% 
. Ils pouvaient assumer eux-mêmes leur éducation, ou les confier à leur propres parents. Ils pouvaient même les confier à leurs ex épouses, mais toujours sous leur propre autorité et à leurs frais.

 Si jusqu'à ces dernières décennies le mariage alliait deux lignées en associant un homme et une femme dans le cadre d'une division sexuelle du travail indiscutée, la première de ses fonctions, ressentie comme incontournable et justifiée par la survie des individus aussi bien que celle de l'espèce, était de donner des enfants aux hommes%
% [6]
\footnote{{\emph{La fonction principale du mariage était d'ailleurs de fabriquer du père} [...]}, Irène \fsc{THERY}, idem.}% 
. Ils ne pouvaient en effet donner le jour à des semblables, mais ils n'en avaient pas moins un impérieux besoin pour s'occuper d'eux jusqu'à leur mort même quand ils ne pourraient plus subvenir à leurs propres besoins et pour leur succéder. Cela impliquait de donner de la valeur au fait que les enfants aient un père et non un géniteur anonyme. 
 
 Tout était (donc ?) fait pour décourager les femmes de concevoir des enfants sans en passer par un homme publiquement désigné (jusqu'à l'infériorité des salaires féminins à travail égal ?). « À cause des enfants », (grâce aux enfants ?) dont l'avenir, le statut et l'installation dans l'existence dépendaient plus d'eux que d'elles, les hommes tenaient les femmes en leur « main ». Le mariage permettait à presque tous ceux qui le désiraient (c'est-à-dire la plupart des hommes) d'avoir des enfants bien à eux et qui ne leur seraient contestés par personne et d'abord par leur mère. Il leur permettait aussi de s'attacher une femme et les services de tous ordres que seule une femme pouvait alors fournir. 
 
 Mais la réciproque était vraie aussi : le mariage permettait aux femmes d'avoir des enfants sans être obligées de les élever seules, dans la pauvreté et l'illégitimité. Quant à celles qui y attachaient du prix, il leur permettait de s'attacher solidement un homme%
% [7]
\footnote{... ce que symbolisaient depuis l'antiquité les anneaux que s'échangeaient les conjoints, et ce qu'exprimait sur le mode burlesque des expressions comme {\emph{se laisser mettre le grappin dessus}}, ou {\emph{se passer la corde au cou}}. Il est symptomatique que c'étaient les hommes qui employaient ces expressions : dans les représentations d'alors ce sont les femmes qui cherchaient le plus activement et anxieusement à se marier, et le jour de leur mariage était en quelque sorte celui de leur triomphe.}
 
 Un homme qui désire des enfants ne peut plus s'y prendre aujourd'hui comme naguère. Il ne lui sert plus à rien de demander à un futur beau-père la main de sa fille, de lui demander un transfert d'autorité, puisque ce dernier ne la détient plus et ne peut donc plus la donner. D'ailleurs lui-même n'a plus besoin d'un gendre pour légitimer les petits enfants que sa fille lui donnera et pour en faire des héritiers, puisqu'il n'y a plus de fonctions interdites aux enfants illégitimes et donc plus d'enfants illégitimes. Il n'y a donc plus d'intérêt commun entre beau-père et gendre, et le soupirant doit négocier seul et sans intermédiaire avec la femme dont il recherche les faveurs. Il n'aura d'elle des enfants que si elle le veut bien. Et elle pourra d'autant plus facilement le quitter en emmenant leurs enfants communs (ou le pousser hors du domicile familial) que l'absence d'un homme à côté d'une mère ne fait plus problème, tandis que la présence de celle-ci semble encore presque indispensable (mais cela changera peut-être si on constate des compétences maternantes au sein des couples masculins ?). 
 
 En ce qui concerne les hommes, les ressources dont ils disposent (puissance économique, puissance militaire, compétences professionnelles,~etc.) ont la vertu de les rendre désirables autant que leurs qualités physiques et psychologiques, et on l'a toujours su. Plus ils sont intellectuellement et professionnellement qualifiés, plus ils ont de chances d'être mariés. C'est le contraire pour les femmes. C'est peut-être une preuve que celles-ci n'ont pas intérêt au mariage dès qu'elles ont les moyens de leur indépendance ? au contraire des hommes ? Le fait que dans certains pays d'Europe un nombre significatif de femmes aux ressources au dessus de la moyenne semble choisir aujourd'hui de ne pas avoir d'enfants montre que pour elles en tout cas la famille et le mariage n'ont pas d'attraits. 
 
 Pour le moment, ces évolutions n'infirment pas la répartition traditionnelle des rôles érotiques masculins et féminins : les hommes se doivent encore d'être « ceux qui peuvent », ceux qui ne sont pas marqués par le manque ou la défaillance (pouvoir politique, financier, intellectuel, militaire, puissance sexuelle...) à défaut de quoi ils n'exercent guère d'attrait sur la plupart des femmes, tandis que lorsque celles-ci ne s'éprouvent pas, au moins un peu, comme « celles qui n'ont pas » (pas tout), comme celles qui « ne peuvent pas » (pas toutes seules), elles n'ont pas besoin des hommes (mais peut-être n'exercent-elles pas non plus d'attrait sur eux ?).
 
 On a vu que « l'obligation de résultat », l'obligation de fécondité, qui pesait sur les seules femmes mariées a été supprimée par Constantin, qui a exclu la stérilité des motifs de divorce. La loi impériale romaine a ensuite confirmé l'interdit fait aux chrétiens de se remarier après divorce. À l'obligation de fécondité des épouses s'est substituée une obligation (religieuse) de moyens pour chacun des deux époux de ne pas mettre d'obstacle autre que l'abstinence%
% [2] 
\footnote{En France les relevés démographiques montrent l'érosion progressive du respect de cette obligation, et l'extension depuis trois siècles des pratiques anticonceptionnelles : ce que les anciens moralistes nommaient les « \emph{funestes secrets} ».} 
à une conception et à une naissance. Si les femmes mariées ont ainsi été protégées contre la répudiation et contre la privation de leurs enfants, par contre la loi ne les autorisait pas plus qu'avant à se dérober au « devoir conjugal » lorsque leur mari l'exigeait, ni aux grossesses qui en découleraient, et à leurs risques, sauf à demander une séparation. 

 Depuis 1967, même si leurs maris le désirent, les femmes mariées ne sont plus tenues par la loi de laisser libre cours à leur fécondité. Aujourd'hui le corps des femmes est à elles, y compris l'embryon ou le fœtus, qui juridiquement en fait partie depuis 1975, comme c'était le cas dans le droit romain antique. 

 La loi ne se soucie plus de soutenir le désir masculin en ce domaine. Même si elles sont leurs épouses, même s'ils sont les géniteurs de l'enfant qu'elles portent, même si elles avaient été d'accord pour le concevoir avec eux, les hommes n'ont plus le droit d'exiger des femmes qu'elles leur donnent cet enfant. Elles peuvent choisir d'avorter ou de l'abandonner à la naissance contre le gré du père de l'enfant. On est au plus loin du droit du \emph{pater familias} romain de faire surveiller la grossesse et l'accouchement de son épouse (ou ex épouse), pour qu'elle ne puisse pas lui dérober un enfant né de ses œuvres.

 Dans le même temps ont été supprimées toutes les limites légales qui pouvaient interdire le rattachement d'un enfant naturel à son géniteur, à l'exception des inséminations artificielles avec donneur, ou IAD. Une mère qui le demande recevra toujours l'appui de la justice pour rechercher le géniteur de son enfant, quelle que soit la situation personnelle de cet homme, comme sous l'ancien régime, mais désormais cela se fera avec une efficacité  imparable. Aucun père n'est plus « \emph{incertus} ». Vivant ou mort son ADN le désignera, sauf lorsque la mère veut cacher son identité à l'enfant ou aux tiers (mais si une mère qui accouche « sous X » refuse de laisser à son enfant des renseignements sur sa propre identité, elle en a le droit). Si la mère le veut, le géniteur sera contraint d'assumer financièrement un enfant qui héritera de lui à part entière, contrairement à ce qui se passait jusqu'au \siecle{19}. Mais cela ne lui donnera pas forcément le moindre droit sur l'éducation de l'enfant : en ce sens cela n'en fera pas un père.

 Pour l'essentiel, on peut donc dire que la maîtrise de la génération est passée du côté des femmes. La famille monoparentale d'aujourd'hui, c'est assez ordinairement la famille \emph{moins} le père. Dans la majorité des séparations (85~\%) ce sont les mères qui gardent les enfants. Est-ce pour ces raisons que l'initiative des divorces vient beaucoup plus souvent des épouses que des maris ? Beaucoup d'hommes ont plus à perdre que leurs femmes au divorce, et surtout les plus pauvres. 

 Les mères ont toujours eu une place de choix dans les représentations : elles sont traditionnellement du côté de l'accueil de la vie et de son entretien, de l'intime, de la tendresse, du cœur. Mais aujourd'hui cette idéalisation n'est plus contrebalancée par l'idéalisation symétrique des pères des siècles classiques. Aujourd'hui la déploration des déficiences des pères, de leurs fragilités et de leur irresponsabilité, est un passage obligé de tout discours sur la famille, tandis que l'idée qu'ils puissent exercer une force ou une puissance dans leur relation à des enfants renvoie automatiquement à des représentations de violence et de maltraitance. Quand on parle sans les spécifier des violences conjugales ou intra familiales, il va de soi qu'il s'agit des violences masculines, alors que l'observation montre que les femmes sont très capables de concurrencer les hommes dans ce domaine aussi. 

 D'ailleurs maintenant que le capital le plus utile c'est le capital intellectuel, maintenant que l'avenir des enfants se prépare à coup d'études longues, financées en grande partie par la collectivité, sous la houlette de professionnels de l'enseignement et sous le contrôle de l'État, qu'est-ce qu'un père pourrait bien transmettre à ses enfants (à part ses biens) sans menacer leur autonomie ?

 Dans l'effritement de l'autorité des pères, Françoise \fsc{HURSTEL} pointe trois moments clé : la loi de 1889 contre les « \emph{parents indignes} », la loi de 1935 abolissant le droit de « \emph{correction paternelle} » et la loi de 1938 abolissant la « \emph{puissance maritale} ». Ont été abolies toutes les dispositions juridiques sur lesquelles était fondé dans le passé l'exercice masculin d'un rôle patriarcal. Le résultat est que « [...] \emph{nous ne savons plus ce qu'est la place d'un père et ce que sont ses fonctions} », et que « \emph{ce ne sont pas des petits bouts de la paternité qui ont changé, mais l'ensemble du système a muté avec la mort du \emph{pater familias}.} »%
% [3]
\footnote{Françoise \fsc{HURSTEL}, « Penser la paternité contemporaine dans le monde occidental : quelles places et quelles fonctions du père pour le devenir humain, sujet et citoyen des enfants ? », in \emph{Neuropsychiatrie de l'enfance et de l'adolescence}, 53 (2005) 224-230.} 

 Autrefois (jusqu'aux années 60 du siècle dernier ?) c'est l'excès de présence et de poids des pères qui faisait problème. Aujourd'hui on déplore qu'ils ne soient jamais assez présents, ou jamais là où il faut. Françoise \fsc{HURSTEL} soutient que cela est l'effet de ces changements, et non leur cause. Si les lois suivaient l'évolution des mœurs, alors la promulgation d'une loi serait le signe que les esprits sont prêts à l'accueillir. Dans cette hypothèse, pendant les années précédant la promulgation de chacune des lois ci-dessus, on aurait dû observer un mouvement de l'opinion publique stigmatisant les parents indignes, le recours abusif au droit de correction paternelle, ou le scandale que constitue l'existence d'une puissance maritale. Selon elle ce n'est pas ainsi que cela s'est passé, au contraire. Ce n'est qu'à partir de la promulgation de la loi de 1889 que la presse aurait commencé de dénoncer les carences des pères « indignes%
% [4]
\footnote{« \emph{alcoolique, pauvre, inculte et violent} », Françoise \fsc{HURSTEL}, \emph{la déchirure paternelle}, p. 113.} 
 ».

 Et de même ce n'est que vers 1942 que les spécialistes de l'éducation auraient commencé de dénoncer les pères sans autorité, tandis que la notion de carence n'aurait envahi les écrits qu'à partir de 1950 :
 
\begin{displayquote}
\emph{C'est donc quelques années après la promulgation de ces lois faisant disparaître des textes juridiques les termes de puissance (maritale) et ceux de correction paternelle tout en maintenant ceux de chef et d'autorité (paternelle), qu'est décrite cette figure d'un père manquant d'autorité et de sévérité ; et que les spécialistes admonestent les pères d'une position qui est bien celle de chef de famille.}
\end{displayquote}

 Selon elle, l'opinion publique n'aurait donc appelé aucune de ces lois de ses vœux. Ces réformes n'auraient été imaginées, réclamées, et parfois discrètement expérimentées que par les seuls experts, médecins, administrateurs, juges et travailleurs sociaux directement intéressés à leur mise en œuvre. Pour Françoise \fsc{HURSTEL}, tous les discours sur les déficiences des pères actuels ne sont que des productions imaginaires qui coexistent avec des réalités qui n'ont pas grand-chose à voir avec eux. En effet, les enquêtes sur le terrain ne montrent rien qui permette de croire que les pères d'aujourd'hui seraient dans l'ensemble moins attentifs et moins présents que ne l'étaient ceux du passé%
% [5]
\footnote{... mais cela exige d'éviter les biais méthodologiques. Il faut notamment que ces enquêtes ne se placent pas consciemment ou inconsciemment du seul point de vue des mères. Cf. Germain \fsc{DULAC}, « La configuration du champ de la paternité : politiques, acteurs et enjeux », in \emph{Politiques du père, numéro spécial de Lien social et politiques}, (n° 37) 1997, p. 133-142.}%
. Certes il y a des pères qui sont incompétents, irresponsables ou délinquants, mais cela n'a rien de nouveau, et rien ne permet d'affirmer qu'il y en ait plus qu'autrefois. Les discours ne portent pas tant sur ce que font réellement les pères que sur ce qu'ils devraient faire dans l'idéal pour être de bons pères. 

 Pour elle, il s'agit, à l'aide de ces discours, d'asseoir l'autorité de ceux qui prétendent savoir ce qu'est un bon père et qui sont les bons pères :
 
\begin{displayquote} 
\emph{Du point de vue de la paternité les hommes de la période contemporaine n'auront pas été gâtés. Je propose une image pour illustrer ce que peut être la notion de carence : lorsqu'un homme devient père, il endosse un pardessus plein de trous et de soupçons..., plus précisément une image de plus en plus dévalorisée, et cela quelle que soit la valeur personnelle de l'homme qui assume une telle fonction. Et ce qui les caractérise est un discours dévalorisant des spécialistes ; tellement dévalorisant qu'il apparaît, en fait, comme un discours de l'exclusion des pères... au profit du super père spécialiste. Si les pères peuvent être dits carents \emph{[en Droit, le père « carent » est celui qui ne laisse rien à ses enfants, qui ne leur laisse aucun héritage]}, c'est parce qu'ils sont relégués à cette place par ceux-là mêmes qui normalisent les pratiques autour de l'enfant. Nous dirons que ces pères carents sont en fait d'abord des pères exclus par les théoriciens de l'éducation.}%
% [6]
\footnote{Françoise \fsc{HURSTEL}, \emph{la déchirure paternelle}, p. 112-113.} 

[...] \emph{Ainsi les signifiants inscrits dans la loi produisent des effets imaginaires qui se repèrent dans les représentations collectives, les modèles normatifs du père et les pratiques sociales.} 

\emph{Je ferai ici un pas de plus et avancerai ceci : non seulement les signifiants des lois produisent des effets imaginaires, mais encore les lois elles-mêmes ne sont connues que par le biais de ces productions...}

\emph{Les figures du père carent semblent bien avoir une fonction sociale et idéologique importante, celle d'être l'une de ces fonctions sociales qui rendent compte et qu'il y a du père dans notre société (au sens du père symbolique et de la fonction paternelle) et qu'il y a du changement dans les montages qui instituent le père... bref, elles seraient un mode d'historicisation d'une structure.}

\emph{Mais en retour cet imaginaire du père marquera chaque homme ayant à assumer la fonction paternelle, chaque mère appelée à reconnaître qu'il y a du père pour son enfant.}%
% [7]
\footnote{Idem, p. 113-115.}
\end{displayquote}

 Il est ordinaire et au fond assez normal que les adolescents, garçons et filles, soient en état d'incertitude identitaire, avec tous les malaises que cela implique, mais ils supportent encore moins bien les incertitudes identitaires de leurs adultes de référence que les leurs propres. Ce n'est pas un hasard si ce sont les garçons qui expriment aujourd'hui le plus durement leur désarroi : violences contre les personnes et les biens, prises de risques inconsidérées, désinvestissement scolaire, etc. Ils ont besoin que les adultes (hommes et femmes) reconnaissent que c'est une puissance valeureuse qui croît en eux et non une violence erratique, brutale et destructrice, juste bonne à être périodiquement sacrifiée en holocauste aux dieux de la guerre.

 Puisque le patriarcat est mort et que les femmes ne retourneront plus dans des gynécées, sinon contraintes et forcées, et puisque dans le domaine familial aussi le droit à l'égalité s'impose comme le principe de base indiscutable, il faudra inventer (ou découvrir, ou redécouvrir) pour les hommes une place qui soit aussi désirable que celle des femmes : des points de vue et des désirs spécifiquement masculins sur les enfants sont-ils acceptables ? Mais dans un environnement allergique à tout ce qui ressemble à du paternalisme, qu'est-ce qu'un homme est autorisé à désirer concernant des enfants ? Les hommes sont-ils fondés à dire quelque chose sur les enfants ? Sont-ils fondés à dire quelque chose aux enfants ? Il faudra sans doute commencer par admettre qu'il existe des valeurs masculines, ou une manière masculine de faire vivre les valeurs universelles.

 Derrière le problème de la paternité se profile la question « à qui appartient l'enfant ? ». Il ne s'appartient pas à lui-même, sauf à supprimer le statut de mineur. On ne peut pas plus dire qu'il n'appartient à personne. Du point de vue des enfants, n'appartenir à personne (ou appartenir à une institution) c'est être abandonné. Depuis très longtemps les enfants n'appartiennent plus aux seuls pères. Est-ce qu'ils appartiennent désormais aux seules mères ? ... ou bien aux deux parents, comme le dit la loi ? ... ou bien à l'ensemble de ceux qui les élèvent en leur donnant leur argent et leur temps, dont les beaux-pères et belles-mères ? ... ou bien encore à l'État ?




% 28.02.2015 :
% haut Moyen Âge
% _, --> ,
% ~etc.
% Antiquité
% ~\%


\chapter{Inertie des pratiques}


 Dans la réalité les changements ne sont pas (encore ?) aussi importants que dans l'idée que l'on s'en fait. Si depuis une génération le nombre de mariage diminue indiscutablement%
% [4]
\footnote{En 1990, 90~\% des couples existants étaient mariés, en 1999, année où le Pacs est entré dans les pratiques ils n'étaient plus que 83~\%. \emph{Histoires de familles, histoires familiales}, INSEE, 1999.}% 
, le Pacs, à l'origine pensé pour les couples homosexuels, est le plus souvent choisi par des couples mixtes (dix-neuf pacs sur vingt sont contractés par des couples mixtes), dont la moitié environ finit par se marier, et au total la somme des Pacs et des mariages est plus élevée que le nombre des seuls mariages avant la création du Pacs. Le lien entre naissances et mariage semble solide : à la naissance du deuxième enfant 86~\% des couples sont mariés, et 93~\% au troisième. On n'est pas loin, avec le Pacs, d'un mariage à l'essai.

 Si par ailleurs le nombre des divorces se situe aujourd'hui entre le tiers et la moitié de celui des mariages, il faut considérer que ce nombre est à la fois élevé et bas. Sur 29~millions d'adultes vivant en couple, mariés ou non, 26~millions (90~\%) en sont \emph{encore} à leur première expérience de couple, et pour l'instant les recompositions de familles concernent \emph{seulement} 3~millions de personnes sur 29. C'est que le nombre de couples mariés de tous âges (le « stock ») est si important que les divorces n'en représentent pas plus de 1~\% par an : 99~\% des gens qui étaient mariés au premier janvier le sont encore au 31 décembre qui suit (mais qu'en sera-t-il de ces chiffres dans une génération ?).

 En 2006, 1,2~millions de mineurs vivent en famille recomposée, soit 9~\% de l'ensemble des mineurs. Parmi ces mineurs, \nombre{400000} sont nés du couple qui s'est « recomposé ». Ceux-là vivent donc avec leurs deux parents, bien que dans une famille "recomposée". À la même date, 2,2~millions de mineurs vivent au sein d'une famille monoparentale (six fois sur sept avec leur mère), tandis que 10,25~millions de mineurs%
% [5] 
\footnote{... dont les \nombre{400000} enfants vivant au sein de familles recomposées et nés du couple nouveau.} 
vivent avec leur père et leur mère (mariés ou non). 
 


\newlength{\lcol}
\setlength{\lcol}{0.16666667\textwidth}
\addtolength{\lcol}{-2\tabcolsep}


\begin{table}[!ht]% [!htb]
%\centering
\begin{minipage}{\textwidth} 
\caption[Cadre de vie des jeunes en 1999]%
{Cadre de vie des jeunes en 1999%
\footnote{Sources :
\\« Histoires de familles, histoires familiales », \emph{Les cahiers de l'INED}, \no 156 ;
\\\emph{Recensement de la population}, INSEE, 1999, p. 281.} 
}
\label{tableau-cadre-vie-1999}
\begin{tabular}{*{6}{>{\hspace{0pt}\centering\arraybackslash}b{\lcol}}}
Âge des jeunes (années) & Vivant avec les deux parents de naissance & Avec un parent seul%
\footnote{Familles monoparentales.}
 & Avec un parent et un beau-parent%
\footnote{Familles recomposées.}
 & Autres situations%
\footnote{En internat, en appartement, en chambre, chez un logeur, en placement ASE, en prison, en hôpital,~etc.}
 & Total\\
\hline
 0-4     & 85,0 & 11,1 & 1,8 & 2,2  & 100~\% \\
 5-9     & 77,7 & 15,6 & 5,2 & 1,5  & 100~\% \\
 10-14 & 72,7 & 17,5 & 8,4 & 1,5  & 100~\% \\
 15-19 & 68,5 & 18,7 & 8,6 & 4,1  & 100~\% \\
 20-24 & 43,5 & 11,5 & 4,3 & 40,6 & 100~\% \\
\hline
 0-17  & 76,5 & 15,7 & 6,0 & 1,8  & 100~\%
\end{tabular}
\end{minipage}
\end{table}

%CADRE DE VIE DES JEUNES EN 1999[6]
% 
%\emph{Age des jeunes}
%\emph{ (années)}
%\emph{Vivant avec ses deux parents de naissance}
%\emph{Avec un parent seul[7]}
%\emph{Avec un parent et un beau-parent[8]}
%\emph{Autres situations [9]}
%\emph{Total}
%\emph{0-4}

 La comparaison %de ce tableau avec le suivant 
des tables \vrefbetterrange{tableau-cadre-vie-1999}{tableau-cadre-vie-2004-2007} 
montre que l'évolution des familles et de leurs comportements n'a rien de fulgurant. Vivre séparé de l'un de ses deux géniteurs reste une situation minoritaire : pour l'instant les trois quarts des mineurs vivent sous le même toit que leurs \emph{deux} parents \emph{de naissance} (dont les deux tiers des mineurs de 15 ans à 18 ans).

\makeatletter
\if@twoside
\begin{table}[t]% [!htb]
\else
\begin{table}[!t]% [!htb]
\fi
\makeatother
%\centering

\begin{minipage}{\textwidth} 
\caption[Cadre de vie des jeunes en 2004-2007]%
{Cadre de vie des jeunes en 2004-2007%
\footnote{Source : \emph{Moyenne annuelle des enquêtes emploi de 2004 à 2007}, INSEE.} 
}
\label{tableau-cadre-vie-2004-2007}

\begin{tabular}{*{6}{>{\hspace{0pt}\centering\arraybackslash}b{\lcol}}}
Âge des jeunes (années) & Vivant avec les deux parents de naissance & Avec un parent seul & Avec un parent et un beau-parent & Autres situations & Total\\
\hline
 0-6     & 82,2 & 10,1 & 7,2 & 0,5  & 100~\% \\
 7-13   & 72,8 & 16,6 & 9,9 & 0,7  & 100~\% \\
 14-17 & 66,9 & 19,0 & 9,8 & 4,4  & 100~\%
\end{tabular}

\end{minipage}

\end{table}

%CADRE DE VIE DES JEUNES EN 2004/2007[11]
% 
%\emph{Age des jeunes (années)}
%\emph{Vivant avec ses deux parents de naissance}
%\emph{Avec un parent seul[12]}
%\emph{Avec un parent et un beau-parent}
%\emph{Autres situations[13]}
%\emph{Total}
%\emph{0-6}
 
 Mais les évolutions actuelles sont aussi (sont d'abord ?) symboliques : peut-être n'y a-t-il jamais eu autant d'enfants qu'aujourd'hui à vivre jusqu'à leur majorité avec leur père et leur mère de naissance \tempuwave{(?)}, et pourtant les familles ne sont plus pensées comme l'alliance irréversible de deux lignées, ni comme des institutions aux limites intangibles, mais comme des associations d'individus à géométrie variable. Les enfants d'aujourd'hui apprennent très tôt que les couples mixtes sont fragiles, qu'on rencontre aussi des couples mariés de même sexe, qu'amour ne rime pas avec toujours, que les princes et les princesses n'ont pas forcément beaucoup d'enfants, et qu'ils se séparent souvent avant la fin de leur histoire. Ils apprennent à dissocier parentalité et conjugalité, ou plutôt ils n'apprennent plus à les associer de manière indéfectible. À côté des scénarii traditionnels de leurs jeux d'imagination (le gendarme et le voleur, le client et la marchande, le malade et le docteur,~etc.) ils disposent maintenant du jeu du mariage et du divorce.

 Sous l'Ancien Régime c'était le contraire : en droit civil comme en droit canon, les mariages étaient indissolubles. Par contre, la mortalité d'alors, très élevée par rapport à celle d'aujourd'hui, faisait que plus de la moitié des époux étaient séparés par la mort avant même que leurs enfants n'aient atteint leurs vingt ans, et à cet âge il était normal d'être orphelin d'au moins un de ses deux parents. La durée moyenne effective des couples conjugaux était faible, environ quinze ans, comparée à celle des couples d'aujourd'hui qui n'ont pas divorcé, autour de cinquante ans. 

 La Révolution avait autorisé et facilité le divorce \emph{par consentement mutuel}, et à la suite de cette décision le taux de divorces observé dans les villes (mais \emph{seulement dans les villes}) avait rapidement atteint le niveau actuel. Mais contrairement à ce qui s'était passé dès l'an~III, aujourd'hui personne ne semble s'en inquiéter. Personne ne se donne plus pour objectif d'enrayer ce phénomène comme ce fut le cas avec le Code Napoléon, pendant la plus grande partie du \siecle{19} et sous le régime de Vichy (1940-45). Il ne s'agit plus de punir un coupable, ou deux, ni de chercher à prouver aux conjoints qu'ils peuvent respecter leurs engagements conjugaux au prix de quelques accommodements. Au contraire, les lois accompagnent ce mouvement de « \emph{démariage}%
% [14]
\footnote{Cf. Irène \fsc{THERY}, et son livre du même nom.}
» et le divorce par consentement mutuel est devenu le modèle. 

 C'est en majeure partie du fait des divorces que les personnes seules avec enfants ont crû en nombre et en visibilité depuis 1970. En effet, le pourcentage de veufs et de veuves en leur sein a beaucoup baissé, au contraire de celui des divorcés : 9 fois sur 10 il s'agit de femmes seules avec enfants. Les appeler « familles monoparentales » comme on le fait depuis une génération, eut semblé absurde du temps tout proche où c'était le mariage et non l'enfant qui fondait les familles. 
 












\backmatter

% Le 02.03.2015 :
% Antiquité
% Moyen Âge
% ~etc.
% ~\%

\part*{Conclusion}

\chapter[Conclusion]{}


 Celui qui embrasse d'un seul regard le passé des familles et de la reproduction humaine ne peut pas ne pas observer que le mouvement de l'histoire semble par moments « repasser les plats » et reproduire des configurations déjà observées durant des périodes antérieures. Pourtant à bien y regarder c'est à chaque fois sous une forme originale et imprévue.  
 
 Ainsi, nous avons pu remarquer : 
\begin{itemize}

\item qu'une grande partie du Moyen Âge européen a accordé plus d'autonomie aux femmes et aux enfants que ne l'avait jamais fait une Antiquité grecque et romaine au patriarcat exemplaire ;

\item qu'au rebours de cette tendance, les \crmieme{17} et \siecle{18}s ont atteint un sommet dans le renforcement du pouvoir des pères sur toute leur famille, épouse comprise, sur le modèle du \latin{pater familias} de l'Antiquité tardive, renforcement initié par les professeurs de droit civil et religieux du onzième et du douzième siècle. Sous la pression de la compétition des deux Réformes ennemies et jumelles pour diriger les consciences et les comportements, et avec l'appui des États modernes (à moins que ce ne soit le contraire ?), les sociétés européennes ont atteint alors dans le domaine de la reproduction un niveau de conformité avec le droit canon inégalé jusque là ;

\item que la suite de cet apogée des maîtrises patriarcales au sein des familles a été la réaction anti pères, anti familles et anticléricale des Lumières et de la Révolution française : \frquote{\emph{divorce facile, autorité paternelle partagée avec la mère et contrôlée par la justice, égalité de l'enfant naturel avec l'enfant légitime, plénitude de l'adoption, administration commune des époux, diminution des incapacités dues à l'âge, les exemples sont nombreux, dans le droit de la famille, à témoigner du surprenant modernisme du législateur révolutionnaire%
% [2]
\footnote{Marcel \fsc{GARAUD}, \emph{La révolution française et la famille}, 1978, p. 191.}%
}} ;

\item qu'à cette réaction, le code civil de Napoléon a réagi en restaurant l'essentiel du droit familial d'inspiration romaine de l'ancien régime ;
 
\item qu'en fait c'est la troisième République qui a inscrit durablement dans le droit français de la famille les changements que voulaient faire les révolutionnaires ;
 
\item que c'est pourtant dans le cadre de l'État providence initié par cette même troisième république, et mis en œuvre par la quatrième, que se sont le mieux épanouies les familles traditionnelles : \enquote{\emph{Les années 1945-1965, qu'on pourrait appeler les « vingt glorieuses » de la famille (ce moment où s'impose un modèle familial, quasiment unique, où presque tout le monde se marie, où la famille est relativement féconde, où elle est stable avec un taux de divorces de moins d'un mariage sur dix, et où elle est organisée selon un principe assez strict de partage des rôles sexués, masculin et féminin) : Cette période de 1945-1965 qui nous sert souvent de référence pour penser la famille « traditionnelle » et lui opposer la famille actuelle, a été à bien des égards un moment historique exceptionnel%
%[1]
\footnote{Irène \fsc{THERY}, « Peut-on parler d'une crise de la famille ? Un point de vue sociologique », \emph{Neuropsychiatrie de l'enfance et de l'adolescence}, 2001, 49, p. 492-501.}%
}} \mbox{-- }moment où la réalité vécue par les familles a été plus proche du modèle traditionnel qu'à toute autre période ;

\item que ce sont justement ceux qui sont nés à ce moment-là, les enfants du baby-boom, qui ont joyeusement et férocement poussé la critique des familles si exemplairement traditionnelles dont ils sortaient ;
 
 Ils ont plébiscité sans réserves les conceptions des révolutionnaires français dans le domaine familial.
 
\end{itemize}
 
 Nous avons vu en quoi la famille constantinienne était une synthèse originale des pratiques matrimoniales des grecs, des juifs, des romains et des chrétiens. Cet amalgame avait pris en une seule génération, au milieu du \siecle{4}. Il avait « pris » comme une mayonnaise ou un mortier peuvent se coaguler en une forme stable et résistante aux déformations.  Si l'on fait abstraction des inflexions apportées par le Code Napoléon, de la création en 1884 du divorce pour faute et des adoucissements apportés à la belle époque au sort des enfants illégitimes, il a fonctionné jusqu'aux années cinquante du \siecle{20}, soit près de \nombre{1600} ans. 
 
 Cette antiquité vénérable donnait en quelque sorte à la famille constantinienne l'air d'être « naturelle », mais cela n'a pas empêché toutes les règles de droit sur lesquelles elle était fondée d'être abrogées entre 1965 et 1985, c'est-à-dire en un temps encore plus bref que celui qui avait été nécessaire pour les mettre en place. 
 
 La famille constantinienne n'était que l'une des modalités des familles patriarcales, et pas la plus typique. Mais elle est en train d'entraîner toutes les autres dans sa chute, même celles des sociétés qui ne l'ont pas pratiquée. Elles sont toutes travaillées par une même lame de fond partie de l'occident et qui déferle sur la planète. Le domaine de la reproduction humaine est aujourd'hui en pleine révolution.
  
 Lorsque ce remue-ménages sera fini tout redeviendra-t-il comme avant, à la façon dont le Code civil n'a gardé du droit révolutionnaire que ce que Napoléon et ses juristes ont choisi d'en préserver, tandis que pour l'essentiel ils restauraient le droit (romain) de l'ancien régime, parfois en le durcissant ? 
 
 Ou bien n'en sommes-nous qu'au début d'un processus dont l'issue est à proprement parler inimaginable pour ceux qui ont grandi dans le monde révolu des familles traditionnelles ?

 Sommes-nous parvenus au terminus indépassable de l'histoire des mœurs et de la reproduction humaine ? ... ou seulement à l'étape actuelle d'une course indéfinie, d'une histoire encore à écrire ? 
 
 La seule chose qui soit certaine c'est qu'un retour pur et simple aux pratiques du passé n'est pas possible. C'est le moment de se souvenir que selon Michel \fsc{FOUCAULT}, l'histoire avance par crises entre lesquelles règnent des périodes de stabilité, définies par un cadre de pensée (\emph{épistémè}, ou paradigmes) à chaque fois différent (cf. \emph{Les mots et les choses}). Il faut certes que les règles ainsi adoptées soient tenables, qu'elles ne provoquent pas d'effets trop pervers ni de dysfonctionnements trop insupportables ... mais les humains sont rusés et inventifs et capables d'interpréter de façon créative n'importe quel système.

 En même temps que pan à pan s'effondre dans la loi, dans les têtes et dans les comportements ce qui reste de la synthèse constantinienne, les traits de ce qui va la remplacer commencent sans aucun doute à se dessiner sous nos yeux, même lorsque nous ne savons pas les reconnaître.

On peut penser que bien des historiens, des sociologues et des moralistes du passé auraient donné cher pour être à notre place devant l'expérience d'écologie humaine en grandeur réelle dans laquelle le mouvement de l'histoire nous entraîne inexorablement ? Il sera en tout cas passionnant d'observer comment les enfants et les petits enfants des « baby-boomers » reprendront à leur propre compte toutes les questions actuellement en crise.



\tableofcontents
\listoftables

\end{document}
