% Le 18 mars 2015 :
% Antiquité
% ~etc.
% \latin



\chapter{La police des familles au \siecle{19}}


\section{Les enfants trouvés et abandonnés}

 Le Décret Impérial du 19 janvier 1811 concernant les enfants trouvés ou abandonnés et les orphelins pauvres a créé le Service des enfants trouvés et abandonnés, qui sera nommé Service des enfants assistés à partir de 1866. Il ordonnait qu'il y ait \emph{dans chaque arrondissement un hospice où les enfants trouvés pourront être reçus}. Ont donc été désignés des hospices dépositaires où les nouveaux-nés étaient abandonnés et des dépôts départementaux accueillant au sein de ces mêmes hospices les enfants plus grands. 

 L'abandon anonyme, par l'intermédiaire d'un tour, restait la norme, d'autant plus que \emph{nul ne peut être parent contre son gré} et que les recherches en paternité étaient désormais totalement interdites. L'administration ne faisait pas de distinction entre enfants trouvés (ceux dont on ne connaît pas les parents) et enfants abandonnés (ceux dont on les connaît), ni en fonction de leur légitimité réelle ou supposée.

 Le service prenait en charge les enfants sans famille -- trouvés ou abandonnés -- et les orphelins pauvres. Il assumait également les enfants dont les deux parents étaient prévenus ou condamnés à une peine de prison. Lorsque ceux-ci étaient incarcérés pour plus d'un an leurs enfants étaient définitivement classés parmi les enfants abandonnés, et traités comme tels%
% [1]
\footnote{Leurs parents ne pouvaient plus les reprendre à leur sortie de prison. Ils étaient considérés comme « infâmes » (la prison est une peine infamante), comme de mauvais exemple, et à ce titre incapables de prendre en main l'éducation d'enfants. On a vu que c'est un accessoire (!) de peine hérité de l'Antiquité via l'ancien régime : à Rome le condamné à une peine de travail forcé telle que celle des mines devenait en effet \latin{ipso facto} un esclave \emph{(esclave de la peine)} et il perdait de ce fait tous ses droits civiques et parentaux. Il cessait d'être le père légal de ses enfants et le mari de son épouse. Au \siecle{19} si une personne était condamnée à une peine infamante, le divorce était de droit pour son conjoint innocent : il n'avait pas besoin d'autre motif.}% 
. Il ne s'occupait des enfants âgés de plus de douze ans, abandonnés, orphelins, enfants d'indigents ou de vagabonds, que s'il les avait déjà pris en charge avant leurs douze ans. Si un mineur pauvre de plus de douze ans était incapable de travailler, quel qu'en soit le motif, il pouvait être admis à l'hospice mais en ce cas c'était à cause de son incapacité à travailler, et de son manque de ressources, et non parce qu'il était mineur. Le décret de 1811 prévoyait que les enfants incapables d'être placés chez un maître, parce que malades chroniques, estropiés ou infirmes, malades mentaux, ou retardés intellectuels,~etc. resteraient à l'hospice [...] \emph{où on les fera travailler autant que faire se pourra}.

 Le Service confiait les pupilles délinquants à la Justice, exerçant ainsi son droit de correction paternelle comme n'importe \emph{bon père de famille} d'alors était tenu de le faire. Le jeune ainsi remis à l'administration compétente allait en prison (en prison pour mineurs s'il en existait une dans le département) ou en \emph{maison de redressement}, ou en \emph{colonie agricole},~etc. 

 Ses règles de fonctionnement reprenaient, pour l'essentiel, le \emph{règlement concernant les enfans-trouvés} promulgué en 1761, un demi-siècle plus tôt, par le \emph{bureau de l'hôpital} de Paris. Alors que le \emph{Code Napoléon} (1804) interdisait les adoptions de mineurs, l'objectif suivant a été inscrit dans tous les règlements du service dès l'an XIII (1805) : \emph{créer une famille nouvelle à la place de celle qui l'abandonne}%
% [2]
\footnote{In l'A.P. en 1900, p. 349.}% 
. Pour les administrateurs et autres responsables du service, un des objectifs du placement était la greffe du pupille dans un nouveau milieu : \emph{on pourrait citer de nombreux exemples d'enfants déjà grands qui, réclamés par leurs parents, refusent absolument de se séparer de leur famille d'adoption. Il n'est pas rare que des nourriciers dotent un enfant assisté, lui réservent une part ou la totalité de leur héritage. Ainsi, l'enfant abandonné, favorisé au point de vue des soins et des précautions matériels, trouve ordinairement chez ses nourriciers une affection, un attachement qui lui rendent véritablement une famille.}%
\footnote{Idem, p. 340.} 

 Il est troublant de voir qu'au \siecle{19} et une partie de \crmieme{20}, les mots \emph{adoption, père, mère} et \emph{parents} sont constamment employés par les professionnels des placements pour parler des nourriciers, là où ces mots ne sont évidemment pas appropriés. Cela a pourtant été une pratique courante et presque une règle pendant au moins un siècle et demi%
% [4]
\footnote{Les textes écrits par \fsc{Soulé} et \fsc{Noël} entre 1955 et 1965 montrent que cet usage extensif du vocabulaire de la parenté a persisté sans aucun changement jusqu'à eux, jusqu'à ce que depuis une génération se généralise l'idée que tout placement est fait pour préparer le retour vers les parents et que le non-retour est un échec.}% 
. Il est vrai que la plupart des parents des enfants placés étaient fermement tenus à l'écart. Cela contribuait au sentiment des pupilles de constituer une famille avec leurs nourriciers, sentiment d'autant plus facile à verbaliser que la loi interdisait leur adoption. Ils étaient placés dans un entre-deux insatisfaisant, mais qui dans une certaine mesure pouvait être protecteur. Pour les nourriciers cet entre-deux était pratique et confortable. Ils n'avaient à se confronter ni à une adoption en bonne et due forme, ni à des parents réels et vivants. 

 C'est en raison de cet état d'esprit que les enfants qui avaient une famille connue, même constituée de parents décédés (orphelins pauvres), n'étaient pas placés en nourrice au-delà de leur petite enfance. Ils étaient voués à vivre en collectivité dans les hospices dépositaires jusqu'à leur placement professionnel. Au fil du siècle de nombreuses voix, laïques comme religieuses, se sont élevées pour réclamer qu'ils soient distingués des enfants trouvés et abandonnés, et pour qu'il leur soit fourni des conditions d'éducation plus soignées. Dès la première moitié du siècle ces protestations vont se traduire par la création de nombreux \emph{orphelinats} privés qui fonctionneront jusqu'au \siecle{20} à la manière des internats scolaires. 

 Durant tout le \siecle{19} les placements en nourrice ont fonctionné à la satisfaction générale. Les enfants placés à la campagne grandissaient et s'adaptaient \emph{comme on l'attendait d'eux} aussi bien sur le plan professionnel que sur le plan social ou scolaire%
% [7]
\footnote{In l'A.P. en 1900, p. 346. Selon DUPOUX, p. 193, en 1898 68,5~\% des enfants de l'Assistance Publique présentés au certificat d'études l'avaient obtenu, à une époque où plus de la moitié d'une classe d'âge sortait du primaire sans ce diplôme, et où il était donc probablement aussi discriminant que le baccalauréat d'aujourd'hui : aujourd'hui 60~\% des jeunes obtiennent un baccalauréat... mais seulement quelques pour cent de ceux que place l'ASE. Pourtant la moitié de ces derniers obtient aux épreuves psychométriques standardisées des résultats qui les situent dans la zone normale et parfois au-dessus.}% 
. Beaucoup s'intégraient définitivement dans les milieux où l'administration les avait transplantés. L'assistance aux enfants de cette période donnait le spectacle d'une espèce d'équilibre : chacun croyait qu'il savait ce qu'il faisait. Les enfants n'étaient là que parce que leurs parents les avaient confiés à l'institution ou parce qu'ils avaient été dans les formes légales qualifiés d'incompétents ou de dangereux. L'institution apportait un secours indispensable à des enfants qui sans cela seraient ou morts ou dans une grande misère. Elle était sûre d'elle et avait une excellente image dans le public, d'autant plus que le taux de survie des enfants abandonnés a augmenté au fil du siècle de manière extrêmement spectaculaire.

 Pour les mères qui abandonnaient leur enfant la question de son adoption par d'autres ne se posait pas. Leur acte était donc loin d'avoir le sens de \emph{consentement à l'adoption} qui lui serait donné aujourd'hui. Même au moment où elles mettaient leur enfant au tour beaucoup de mères ne croyaient pas que la séparation était définitive, et glissaient par exemple des signes de reconnaissance dans ses langes%
% [8]
\footnote{\fsc{DUPOUX}, idem, p. 200. Le procès-verbal d'abandon rédigé lorsqu'il était découvert décrivait l'enfant et toute sa vêture, en notant soigneusement tous les signes distinctifs.}% 
. Même si le lieu du placement leur était caché, même si les contacts directs et les correspondances ont été interdits jusqu'à la fin du \siecle{19}, au fil des années de plus en plus de parents ont cherché à reprendre l'enfant qu'ils avaient abandonné, du moins tant qu'il avait moins de quatre ou cinq ans. Ceci étant dit il ne faut pas oublier que cette démarche demeurait très minoritaire.


\section{La prévention des abandons}

 Le nombre des abandons a cru rapidement au début du \siecle{19}, \nombre{55700} enfants trouvés en 1810, \nombre{164000} en 1833. À l'époque cette augmentation a été imputée à l'anonymat de l'abandon, permis par les tours, plutôt qu'à l'accroissement du nombre des femmes réduites à survivre misérablement dans les conditions du travail salarié d'alors (elles étaient payées \emph{beaucoup} moins que les hommes pour le même travail), ou à l'impossibilité où elles se trouvaient de recourir aux recherches en paternité pour obtenir l'aide des géniteurs de leurs enfants.

 Face à l'augmentation du nombre des abandons, deux réactions ont été opposées : la diminution puis la fermeture des tours, d'une part, et d'autre part l'aide aux mères. Certains départements ont créé des \emph{Secours préventifs contre l'abandon}, organisés sur le modèle de ce qui s'était fait dans quelques villes avant la Révolution, à l'intention des mères seules et sans ressources, célibataires pour la plupart (« filles mères », comme les nommait alors l'administration). Mis en place en 1837 à Paris, ils ont été généralisés à tous les départements à partir de 1850 (arrêté du 23 décembre). Un certain nombre des jeunes enfants ainsi \emph{secourus}, une minorité, étaient placés chez une nourrice choisie par leur mère elle-même, conformément aux pratiques des populations citadines de l'époque. Le but était en ce dernier cas que ces mères puissent exercer une activité professionnelle sans abandonner leur enfant pour autant.

 Les derniers tours encore en fonction ont été fermés en 1861. Désormais les abandons devaient se faire \emph{à bureau ouvert}. L'anonymat de l'abandon restait possible si la personne qui déposait l'enfant refusait de donner l'état-civil de celui-ci et éventuellement le sien, mais dans le cas contraire on notait l'identité des parents, le plus souvent celle de la mère seule. Cette identité est restée secrète, même pour l'enfant devenu adulte, jusqu'aux années 1980. L'abandon à bureau ouvert n'a donc guère changé la situation des pupilles (ce n'était pas son objectif) par contre il a eu un effet visible sur le nombre des abandons, qui a rapidement et fortement décru%
% [9]
\footnote{Est-ce que les avortements, clandestins à l'époque, ont augmenté dans les mêmes proportions ? Selon Stanislas \fsc{DU MORIEZ} (\emph{L'avortement}, 1912) et Edmond \fsc{PIERSON} (\emph{La dépopulation de la France}, 1913) cités par René \fsc{LE MEE} (dans "une affaire de "faiseuse d'ange" à la fin du XIXè siècle" in \emph{communications}, dénatalité : l'antériorité française, 1800-1914, 44,1986) de 1826 à 1880 les tribunaux français ont traité \nombre{9300} affaires d'avortements, dont \nombre{1020} ont donné lieu à des sanctions ; de 1881 à 1909 ils ont traité \nombre{14731} affaires d'avortement, dont \nombre{715} ont donné lieu à des peines diverses. Le faible nombre d'affaires par rapport à ce qu'on suppose être le nombre des avortements, et surtout la faiblesse du pourcentage des condamnations effectives (12 et 5~\%) est à noter. Les journaux de la Belle Époque sont remplis de petites annonces de sages-femmes proposant leurs services de manière à peine voilée pour supprimer les grossesses indésirables : en dépit de la stigmatisation morale qui frappait les avorteurs et avorteuses, il existait en fait une relative tolérance qui disparaîtra après la première guerre mondiale. Quant au nombre effectif d'avortements provoqués, il était à cette époque estimé par les auteurs ci-dessus au minimum à \nombre{200000} par an, et au maximum à \nombre{1000000} (un million) et plus. Autrement dit, ils ne savaient pas, ce qui n'est pas étonnant pour une pratique clandestine et qui demande peu de moyens techniques.}% 


\section{Des clivages idéologiques durables autour des familles}

 Les secours préventifs contre l'abandon ont provoqué de très virulents débats parlementaires. Ceux qui les critiquaient pensaient que rien ne vaut un couple conjugal légitime (rural si possible). Ils pensaient que l'aide destinée à l'enfant était le plus souvent détournée de son objet pour \emph{alimenter la débauche} des mères et de leurs amants, qu'elle n'assurait pas l'avenir d'enfants \emph{sans pères} et donc \emph{sans repères}, et qu'au contraire elle entretenait \emph{une masse d'enfants vagabonds indisciplinés, qui encombrent les cités, constituent un péril social, et dont il faut à grands frais punir les méfaits ou réprimer l'audace toujours croissante}%
% [10]
\footnote{de~\fsc{GERANDO}, cité par \fsc{BIANCO} et \fsc{LAMY}, 1980.}% 
. Avant d'encourager les \emph{filles mères} à garder leurs enfants illégitimes il fallait donc moraliser leur vie et leur donner un « tuteur » en les mariant à un homme travailleur, sobre et économe. Il convenait de faire passer les couples de concubins devant monsieur le maire, et si possible devant monsieur le curé. Lorsque cela n'était pas possible il était préférable pour l'ordre public et pour l'État de confier les bébés sans père aux placements nourriciers ruraux, qui aux yeux des participants de cette sensibilité étaient parfaitement au point : [...] \emph{le service des enfants assistés fournit au contraire \emph{[...]} une race honnête, vigoureuse, fixée à la campagne, fournissant un contingent peu élevé de criminalité}%
%[11]
\footnote{Idem.}%
.

 Ceux qui défendaient les secours préventifs contre l'abandon étaient plus sensibles à la détresse des mères et aux risques pour l'enfant qu'entraîne la coupure d'avec sa famille, même réduite à une seule personne. Ils ne pensaient pas que l'absence d'un époux rendait les mères incompétentes. Ils ne pensaient pas que le soutien d'un père soit irremplaçable. Ils ne pensaient pas qu'un enfant sans père était condamné à l'inadaptation et à la délinquance. Ils croyaient que la société pouvait fournir une aide suffisante aux mères et aux enfants pour que cela n'arrive pas. Ce n'était pas un débat nouveau, puisqu'on l'observait dès le \siecle{18}. Ceux qui n'attachent pas une importance déterminante au mariage et à la naissance légitime et qui regardent la paternité avec une certaine distance ont tendance à sympathiser avec les idées d'égalité et d'autonomie individuelle promues par les Lumières et la Révolution. Même sans aller jusqu'à désigner l'État comme la seule instance qui ait une autorité légitime sur les enfants, ils sont plus ouverts que les autres à l'idée qu'il puisse légitimement exercer un contrôle sur tous les parents. Ceux qui au contraire tiennent pour essentiel que l'enfant grandisse dans une famille fortement structurée autour d'un couple mixte, avec des rôles différenciés (père, mère, enfants), ont plus tendance à n'être que peu ou pas du tout séduits par les discours révolutionnaires et considèrent plus facilement comme abusif que l'État cherche à s'immiscer dans la relation entre les parents et les enfants. Ils sont aussi plus enclins que les autres à supporter que la loi fasse des différences entre les enfants légitimes et les autres au nom de la défense de l'institution familiale. Il y a là une ligne de partage que l'on retrouve aujourd'hui encore.
 

