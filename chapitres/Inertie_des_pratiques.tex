% 28.02.2015 :
% haut Moyen Âge
% _, --> ,
% ~etc.
% Antiquité
% ~\%


\chapter{Inertie des pratiques}


 Dans la réalité les changements ne sont pas (encore ?) aussi importants que dans l'idée que l'on s'en fait. Si depuis une génération le nombre de mariage diminue indiscutablement%
% [4]
\footnote{En 1990, 90~\% des couples existants étaient mariés, en 1999, année où le Pacs est entré dans les pratiques ils n'étaient plus que 83~\%. \emph{Histoires de familles, histoires familiales}, INSEE, 1999.}% 
, le Pacs, à l'origine pensé pour les couples homosexuels, est le plus souvent choisi par des couples mixtes (dix-neuf pacs sur vingt sont contractés par des couples mixtes), dont la moitié environ finit par se marier, et au total la somme des Pacs et des mariages est plus élevée que le nombre des seuls mariages avant la création du Pacs. Le lien entre naissances et mariage semble solide : à la naissance du deuxième enfant 86~\% des couples sont mariés, et 93~\% au troisième. On n'est pas loin, avec le Pacs, d'un mariage à l'essai.

 Si par ailleurs le nombre des divorces se situe aujourd'hui entre le tiers et la moitié de celui des mariages, il faut considérer que ce nombre est à la fois élevé et bas. Sur 29~millions d'adultes vivant en couple, mariés ou non, 26~millions (90~\%) en sont \emph{encore} à leur première expérience de couple, et pour l'instant les recompositions de familles concernent \emph{seulement} 3~millions de personnes sur 29. C'est que le nombre de couples mariés de tous âges (le « stock ») est si important que les divorces n'en représentent pas plus de 1~\% par an : 99~\% des gens qui étaient mariés au premier janvier le sont encore au 31 décembre qui suit (mais qu'en sera-t-il de ces chiffres dans une génération ?).

 En 2006, 1,2~millions de mineurs vivent en famille recomposée, soit 9~\% de l'ensemble des mineurs. Parmi ces mineurs, \nombre{400000} sont nés du couple qui s'est « recomposé ». Ceux-là vivent donc avec leurs deux parents, bien que dans une famille "recomposée". À la même date, 2,2~millions de mineurs vivent au sein d'une famille monoparentale (six fois sur sept avec leur mère), tandis que 10,25~millions de mineurs%
% [5] 
\footnote{... dont les \nombre{400000} enfants vivant au sein de familles recomposées et nés du couple nouveau.} 
vivent avec leur père et leur mère (mariés ou non). 
 


\newlength{\lcol}
\setlength{\lcol}{0.16666667\textwidth}
\addtolength{\lcol}{-2\tabcolsep}


\begin{table}[!ht]% [!htb]
%\centering
\begin{minipage}{\textwidth} 
\caption[Cadre de vie des jeunes en 1999]%
{Cadre de vie des jeunes en 1999%
\footnote{Sources :
\\« Histoires de familles, histoires familiales », \emph{Les cahiers de l'INED}, \no 156 ;
\\\emph{Recensement de la population}, INSEE, 1999, p. 281.} 
}
\label{tableau-cadre-vie-1999}
\begin{tabular}{*{6}{>{\hspace{0pt}\centering\arraybackslash}b{\lcol}}}
Âge des jeunes (années) & Vivant avec les deux parents de naissance & Avec un parent seul%
\footnote{Familles monoparentales.}
 & Avec un parent et un beau-parent%
\footnote{Familles recomposées.}
 & Autres situations%
\footnote{En internat, en appartement, en chambre, chez un logeur, en placement ASE, en prison, en hôpital,~etc.}
 & Total\\
\hline
 0-4     & 85,0 & 11,1 & 1,8 & 2,2  & 100~\% \\
 5-9     & 77,7 & 15,6 & 5,2 & 1,5  & 100~\% \\
 10-14 & 72,7 & 17,5 & 8,4 & 1,5  & 100~\% \\
 15-19 & 68,5 & 18,7 & 8,6 & 4,1  & 100~\% \\
 20-24 & 43,5 & 11,5 & 4,3 & 40,6 & 100~\% \\
\hline
 0-17  & 76,5 & 15,7 & 6,0 & 1,8  & 100~\%
\end{tabular}
\end{minipage}
\end{table}

%CADRE DE VIE DES JEUNES EN 1999[6]
% 
%\emph{Age des jeunes}
%\emph{ (années)}
%\emph{Vivant avec ses deux parents de naissance}
%\emph{Avec un parent seul[7]}
%\emph{Avec un parent et un beau-parent[8]}
%\emph{Autres situations [9]}
%\emph{Total}
%\emph{0-4}

 La comparaison %de ce tableau avec le suivant 
des tables \vrefbetterrange{tableau-cadre-vie-1999}{tableau-cadre-vie-2004-2007} 
montre que l'évolution des familles et de leurs comportements n'a rien de fulgurant. Vivre séparé de l'un de ses deux géniteurs reste une situation minoritaire : pour l'instant les trois quarts des mineurs vivent sous le même toit que leurs \emph{deux} parents \emph{de naissance} (dont les deux tiers des mineurs de 15 ans à 18 ans).

\makeatletter
\if@twoside
\begin{table}[t]% [!htb]
\else
\begin{table}[!t]% [!htb]
\fi
\makeatother
%\centering

\begin{minipage}{\textwidth} 
\caption[Cadre de vie des jeunes en 2004-2007]%
{Cadre de vie des jeunes en 2004-2007%
\footnote{Source : \emph{Moyenne annuelle des enquêtes emploi de 2004 à 2007}, INSEE.} 
}
\label{tableau-cadre-vie-2004-2007}

\begin{tabular}{*{6}{>{\hspace{0pt}\centering\arraybackslash}b{\lcol}}}
Âge des jeunes (années) & Vivant avec les deux parents de naissance & Avec un parent seul & Avec un parent et un beau-parent & Autres situations & Total\\
\hline
 0-6     & 82,2 & 10,1 & 7,2 & 0,5  & 100~\% \\
 7-13   & 72,8 & 16,6 & 9,9 & 0,7  & 100~\% \\
 14-17 & 66,9 & 19,0 & 9,8 & 4,4  & 100~\%
\end{tabular}

\end{minipage}

\end{table}

%CADRE DE VIE DES JEUNES EN 2004/2007[11]
% 
%\emph{Age des jeunes (années)}
%\emph{Vivant avec ses deux parents de naissance}
%\emph{Avec un parent seul[12]}
%\emph{Avec un parent et un beau-parent}
%\emph{Autres situations[13]}
%\emph{Total}
%\emph{0-6}
 
 Mais les évolutions actuelles sont aussi (sont d'abord ?) symboliques : peut-être n'y a-t-il jamais eu autant d'enfants qu'aujourd'hui à vivre jusqu'à leur majorité avec leur père et leur mère de naissance \tempuwave{(?)}, et pourtant les familles ne sont plus pensées comme l'alliance irréversible de deux lignées, ni comme des institutions aux limites intangibles, mais comme des associations d'individus à géométrie variable. Les enfants d'aujourd'hui apprennent très tôt que les couples mixtes sont fragiles, qu'on rencontre aussi des couples mariés de même sexe, qu'amour ne rime pas avec toujours, que les princes et les princesses n'ont pas forcément beaucoup d'enfants, et qu'ils se séparent souvent avant la fin de leur histoire. Ils apprennent à dissocier parentalité et conjugalité, ou plutôt ils n'apprennent plus à les associer de manière indéfectible. À côté des scénarii traditionnels de leurs jeux d'imagination (le gendarme et le voleur, le client et la marchande, le malade et le docteur,~etc.) ils disposent maintenant du jeu du mariage et du divorce.

 Sous l'Ancien Régime c'était le contraire : en droit civil comme en droit canon, les mariages étaient indissolubles. Par contre, la mortalité d'alors, très élevée par rapport à celle d'aujourd'hui, faisait que plus de la moitié des époux étaient séparés par la mort avant même que leurs enfants n'aient atteint leurs vingt ans, et à cet âge il était normal d'être orphelin d'au moins un de ses deux parents. La durée moyenne effective des couples conjugaux était faible, environ quinze ans, comparée à celle des couples d'aujourd'hui qui n'ont pas divorcé, autour de cinquante ans. 

 La Révolution avait autorisé et facilité le divorce \emph{par consentement mutuel}, et à la suite de cette décision le taux de divorces observé dans les villes (mais \emph{seulement dans les villes}) avait rapidement atteint le niveau actuel. Mais contrairement à ce qui s'était passé dès l'an~III, aujourd'hui personne ne semble s'en inquiéter. Personne ne se donne plus pour objectif d'enrayer ce phénomène comme ce fut le cas avec le Code Napoléon, pendant la plus grande partie du \siecle{19} et sous le régime de Vichy (1940-45). Il ne s'agit plus de punir un coupable, ou deux, ni de chercher à prouver aux conjoints qu'ils peuvent respecter leurs engagements conjugaux au prix de quelques accommodements. Au contraire, les lois accompagnent ce mouvement de « \emph{démariage}%
% [14]
\footnote{Cf. Irène \fsc{THERY}, et son livre du même nom.}
» et le divorce par consentement mutuel est devenu le modèle. 

 C'est en majeure partie du fait des divorces que les personnes seules avec enfants ont crû en nombre et en visibilité depuis 1970. En effet, le pourcentage de veufs et de veuves en leur sein a beaucoup baissé, au contraire de celui des divorcés : 9 fois sur 10 il s'agit de femmes seules avec enfants. Les appeler « familles monoparentales » comme on le fait depuis une génération, eut semblé absurde du temps tout proche où c'était le mariage et non l'enfant qui fondait les familles. 
 

