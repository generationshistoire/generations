% 28.02.2015 :
% haut Moyen Âge
% _, --> ,
% ~etc.
% Antiquité
% ~\%


\chapter[Les sociétés du Bas-Empire et du haut Moyen Âge]{Les sociétés du Bas-Empire\\et du haut Moyen Âge}


 Nous ne pouvons lire que les textes et inscriptions qui nous sont parvenus, or l'écriture était à la fin de l'Antiquité et au début du Moyen Âge un privilège et une distinction. Compte tenu de la diminution progressive du nombre des laïcs cultivés, les clercs, et surtout les moines, en devenaient peu à peu les spécialistes. C'est par leur truchement, c'est à travers leurs yeux que nous sommes aujourd'hui condamnés à regarder leur monde. Beaucoup d'entre eux étaient eux-mêmes issus des familles de guerriers, de l'aristocratie de la naissance. Leur société était l'héritière de l'empire de Caracalla dans lequel, si tous les hommes libres étaient devenus citoyens romains, seuls les nobles \latin{(clarissimi)} jouissaient de la totalité des droits autrefois garantis par la citoyenneté. Ainsi les serfs et les esclaves ne pouvaient pas être ordonnés, sauf à être affranchis au préalable pour les délier de leur dépendance à leur maître. Que tous les baptisés (hommes et femmes, esclaves ou libres ...) aient la même valeur aux yeux de Dieu, ce qu'ils enseignaient, n'était pas une raison suffisante pour que l'égalité soit recherchée sur cette terre. Au contraire, ici-bas les hiérarchies leur paraissaient naturelles, nécessaires et crées par Dieu en vue du bien commun. Leur point de vue était conforté par les écrits de Paul de Tarse ou ceux d'Augustin.

 Dans la lignée de l'Antiquité grecque et romaine, et donc du mépris des hommes libres pour les tâches serviles, ils pensaient que l'activité intellectuelle avait plus de valeur que le travail manuel, juste bon pour ceux qui ne possédaient ni revenus fonciers ni savoirs, ce qui allait de pair en l'absence d'écoles gratuites. À leurs yeux étaient associés, sauf exception dûment soulignée, le \emph{sang vil}, la lâcheté et l'incapacité à tenir parole, le paganisme (religion des \latin{pagani}, des paysans) et la sorcellerie, la servilité morale et les \emph{tâches serviles}... Ils trouvaient naturel que coïncident le \emph{sang noble} et les \emph{tâches nobles}, telles que l'étude et le \emph{service divin} (prêtres, évêques, moines de chœur chantant les offices en latin), le sang des aristocrates et l'aptitude à prêter serment, à dire le vrai, à tenir sa parole, à s'engager par contrat : si l'on en croit les \emph{Vies de saints} écrites au haut Moyen Âge, rares étaient ceux d'entre ces derniers qui \emph{n'étaient pas} issus de haute noblesse. Les moines qui les rédigeaient étaient \emph{presque} incapables d'imaginer qu'un personnage digne d'être mis sur les autels puisse ne pas être né d'un puissant seigneur et d'une noble et pieuse dame. 

 Chez les Germains comme chez les Celtes, c'est la naissance qui déterminait la valeur. À leurs yeux la société reposait sur le \emph{sang}, c'est-à-dire les ascendants, la lignée, l'hérédité. Il existait quelques lignées nobles, descendantes en partie de l'aristocratie romaine, en partie des aristocraties barbares, et de plus en plus des deux à la fois, distinguées de toutes les autres, celles des multitudes de personnes au sang vil, sans parents dignes de mémoire. La société s'organisait en un système qui serait un jour théorisé (par des clercs) comme l'union de ceux qui prient (et qui prêchent et enseignent), de ceux qui combattent (et qui dirigent), et de ceux qui nourrissent tout le monde (ceux qui transpirent et œuvrent de leurs mains et qui paient taxes et dîmes). Ces derniers étaient d'abord les « vilains », ceux qui habitaient les \latin{villas}, c'est-à-dire les paysans : 95~\% de la population d'alors.

 Les plus humbles n'ont laissé de traces directes que pour les archéologues. On ne peut donc savoir quelles étaient leurs propres représentations. Jusqu'où avaient-ils la possibilité ne pas s'identifier à l'image que les savants de leur époque, tous clercs, avaient d'eux-mêmes ?

 Il faudra attendre le \siecle{12} pour que la croissance des villes, celle des populations et celle des économies permettent une renaissance des civilités, sous des formes inconnues de l'Antiquité, mais aussi brillantes dans certains domaines. 
 
 Les chrétiens apportaient une philosophie de l'histoire, une explication totale du monde et une morale pour tous les instants. Comme l'avait voulu Constantin, ils fournissaient une idéologie unificatrice à l'empire. Mais à la fin de l'Antiquité celle-ci était encore loin d'avoir imprégné la culture et les mœurs. Pour \fsc{DUBY} il faudra attendre le \siecle{12} pour qu'elle soit véritablement intériorisée par l'ensemble des populations. Pourtant dès le \siecle{6}, « chrétien » désignait une identité (une « ethnie » en langage médiatique actuel) au même titre que « Romain », et les deux identités tendaient à se confondre. 

 De son côté l'Empire romain influençait profondément les chrétiens, qui avaient calqué leur organisation territoriale sur lui, avec une hiérarchie religieuse parallèle à la hiérarchie civile. Le christianisme avec ses représentations entrait en résonance avec les conceptions des empereurs, de la même manière que le système impérial lui-même exprimait sans doute le \emph{style de communication} des gens de cette époque%
%[1]
\footnote{Peter \fsc{BROWN}, 1999}%
. Au \siecle{4}, dans les vastes basiliques offertes par Constantin, le nombre des participants, la structure hiérarchique de l'assemblée et le style des homélies, la minutie des rituels, la pompe, le décorum, les luminaires, l'encens, la musique et les chants, tout rappelait les splendeurs des cérémonies des temples romains, grecs ou égyptiens. Au même moment et pour encore un tout petit peu de temps ces derniers continuaient de déployer leurs fastes immémoriaux. 

 Les évêques étaient assez régulièrement issus des familles de sénateurs ou de chevaliers, qui fournissaient ses magistrats à l'Empire, élus par leur clergé et par les membres importants de leur église locale. Certains avaient été eux-mêmes de hauts fonctionnaires avant leur ordination (cf. Ambroise de Milan, ancien préfet). Les évêques parlaient comme les préfets, avec la même conscience de la grandeur de leur mission et de leur légitimité, et la même rhétorique particulière du Bas Empire. Le ton de leurs écrits était en consonance avec celui des mandements et rescrits impériaux. Ils enseignaient et admonestaient leurs ouailles, ils écrivaient et géraient leurs églises avec la même logique intellectuelle, le même esprit juridique, la même conscience professionnelle et le même sens de la grandeur de leur tâche et de leur fonction que les magistrats et fonctionnaires d'alors. Ils présidaient le culte chrétien comme leurs pères avaient présidé les sacrifices des religions civiques dans le cadre de leur \latin{cursus honorum}. 

 Pendant les siècles du haut Moyen Âge, en raison de l'effondrement du système d'enseignement public antique, provoqué en grande partie par la déchéance des villes désertées par l'essentiel de leurs citoyens et passées aux mains des rois barbares, ce sont les clercs qui ont tenté avec plus ou moins de succès de maintenir les traditions littéraires et administratives romaines. De plus en plus souvent ils sont devenus les seuls experts de l'écriture, de la littérature et de l'éloquence, capables d'occuper les emplois de lettrés, et à ce titre ils peuplaient les chancelleries des grands.
 
 % Le 10 mars 2015 :
% Moyen Âge
% ~etc.
% Antiquité
% Romain

% Le 02.03.2015 :
% ~\%
% ~etc.
% Antiquité
% Moyen Âge

% 28.02.2015 :
% haut Moyen Âge
% _, --> ,
% Antiquité
% ~etc.
% ~\%




\section{Le clergé chrétien}


 Les règles de recrutement et de discipline cléricale de l'Église se sont précisées au cours des premiers siècles%
% [1]
\footnote{Sources : Georges \fsc{MINOIS}, \emph{Les religieux en Bretagne sous l'Ancien Régime}, 1989. Léo \fsc{MOULIN}, \emph{La vie quotidienne des religieux au Moyen Âge, \siecles{10}{15}}, 1978. Michel \fsc{PARISSE}, \emph{Les nonnes au Moyen Âge}, 1983.}% 
. Dès le \siecle{4} ces règles reflètent l'état définitif de la doctrine permanente, qu'on retrouvera telle quelle et quasi inchangée dans sa formulation du début du \siecle{20} (droit canon de 1917). Selon les Décrétales du Pape Innocent I (401-417) il est interdit d'admettre au diaconat et à la prêtrise : 1°) ceux qui ont épousé une femme non vierge ; 2°) ceux qui ont épousé une veuve ; 3°) ceux qui ont été mariés deux fois, quelles que soient les circonstances ; 4°) ceux qui se sont fait soldats après leur baptême, qui ont accepté de toucher des armes dont certaines ont versé le sang, et surtout ceux qui ont accepté de verser le sang. Même si l'interdiction faite aux chrétiens d'être militaires a été levée par l'Église à partir du ralliement de Constantin (entre 313 et 315), le sang restait sacré, donc impur, et impur aussi celui qui le versait, même pour la bonne cause (chirurgiens) ; 5°) ceux qui, magistrats, ont jugé ou plaidé dans des procès où ils ont requis ou prononcé la peine de mort (même motif que le cas précédent : le juge est condamné à faire couler le sang : infliger la \emph{question}, c'est-a-dire torturer, était alors considéré comme nécessaire pour découvrir la vérité, et donc inévitable; infliger des peines mineures comme le fouet; condamner à mort...). Ce n'était pas le risque de l'erreur judiciaire qui était en jeu, c'était encore une fois le contact avec le sang et le contact avec la mort ;6°) les pécheurs qui ont été condamnés à une pénitence (« pécheurs publics », nouveaux infâmes) ; 7°) ceux qui ont donné des jeux publics \latin{(munera)}, où du sang (humain ou animal) a coulé ; 8°) ceux qui ont exercé des sacerdoces païens, et qui ont donc eux-mêmes sacrifié aux dieux ; 9°) ceux qui se sont mutilés eux-mêmes, ce qui vise surtout l'auto castration. Ces derniers ont à la fois versé leur propre sang et mutilé leur corps à l'instar des \latin{galles} (prêtres de Cybèle).

 On peut comparer trait pour trait ces règles avec celles du Lévitique qui régissaient les lévites et les prêtres du Temple de Jérusalem. C'est la même logique. Dans les discussions sur ces sujets les textes de la Tora ont servi d'arguments décisifs. En effet l'imitation du clergé du Temple s'est faite au fil du temps de plus en plus consciente et volontaire. Et pourtant plus il se voulait identique au clergé du Temple, moins le clergé chrétien lui ressemblait ! L'exigence du célibat lui imprimait en effet une physionomie tout à fait inédite. 

 On a vu que la continence perpétuelle était exigée des diacres et prêtres dès les premiers siècles afin qu'ils soient toujours prêts à toucher les « choses sacrées » (vases et linges sacrés, offrandes, pain consacré,~etc.), non souillés par l'impureté rituelle produite par le coït. Ce qui est remarquable c'est que cet argumentaire a emporté l'adhésion. Pourtant l'organisation du service du Temple de Jérusalem montrait une voie de compromis évidente, le service par roulement. D'autre part la notion même de pureté et de souillure religieuse, qui ne se confond pas avec celle de faute morale \emph{(péché)}, avait été mise en question par le Christ lui-même. On peut en déduire que le refus du service par roulement était motivé par des raisons autrement impérieuses que la difficulté de mettre en place un tour de service. 

 Dès l'élection du remplaçant de l'apôtre Judas et l'institution des diacres, les apôtres avaient estimé que personne ne se donne à soi-même une mission%
% [2]
\footnote{Cf. selon le livre des \emph{Actes des Apôtres} les difficultés de Paul de Tarse pour faire admettre par les apôtres sa mission auto proclamée auprès des gentils et ses prétentions au titre d'apôtre.} 
ni ne la tient de sa naissance%
%[3]
\footnote{... de même que nul ne peut (en stricte doctrine) se dire chrétien par sa naissance : il faut que chaque enfant en passe par le baptême, comme le premier converti venu.}% 
, que c'est l'Église qui appelle, et Dieu à travers elle. C'est pourquoi la succession dans le même poste ecclésiastique du père au fils, de l'oncle au neveu, sans être interdite n'a jamais été reconnue comme un droit, au contraire du droit à hériter d'un « honneur », d'une terre ou d'une entreprise, et encore moins comme un modèle. Passés les premiers siècles elle a au contraire été vue comme une irrégularité grosse de dangers. 

 Si la haute administration de l'Empire romain tardif et des royaumes barbares qui lui ont succédé est devenue vers le \siecle{10} la noblesse héréditaire du Moyen Âge, c'est parce que ceux qui étaient nommés par les autorités civiles à un emploi public ont fini par obtenir le droit de désigner eux-mêmes leur successeur, ce qui signifie que « l'honneur » (responsabilités et biens servant à les rémunérer) qui leur avait été conféré par les souverains est entré dans leur patrimoine, à la faveur de l'affaiblissement de ces mêmes souverains, système d'où est sortie la \emph{féodalité}. Mais si les membres d'une hiérarchie peuvent donner leur poste (leur "honneur") à un de leurs héritier c'est qu'ils en sont devenus propriétaires et c'est toujours au détriment du sommet de la hiérarchie, désormais obligée de composer avec une autre source de légitimité qu'elle-même. Inversement, c'est toujours pour défendre ou renforcer son autorité qu'un souverain refuse que soit limité son pouvoir de nommer et de démettre.
 
  Si les membres du clergé avaient été autorisés à cohabiter avec une épouse ils auraient eu autant de bonnes ou de mauvaises raisons que les autres membres des couches dirigeantes de chercher à transmettre leur poste à leurs fils,  et le risque eut été grand de voir se constituer une caste sacerdotale à côté de la caste aristocratique, à la mode indienne ou hébraïque. On connaît d'ailleurs un certain nombre de grandes familles de l'Antiquité et du haut Moyen Âge dont des membres se sont succédé sur le même siège épiscopal pendant plusieurs générations : Sylvère, pape de 536 à 537, était le fils légitime d'Horsmidas, pape de 514 à 523 (il était né avant l'ordination de ce dernier). 

 Au contraire, du point de vue d'une institution comme l'Eglise, le célibat est idéal : 1°) Un célibataire est plus disponible puisqu'il n'a pas à plaire à sa femme, ni à s'occuper de ses enfants (cf. Paul de Tarse). 2°) Un célibataire sans enfants a moins de besoins matériels qu'un homme marié et donc il \emph{peut} coûter moins cher. Il n'y a pas à constituer de dot pour ses filles ni à établir ses garçons ... 3°) ... qui pourraient prétendre avoir des droits sur le poste de leur père. 4°) N'ayant pas à craindre pour ses proches, ni à les établir dans la vie, un clerc sans attaches familiales est plus résistant aux manoeuvres d'intimidations ou de corruption de ses contemporains. 5°) Par ailleurs il serait inconvenant que des histoires de famille puissent interférer dans les affaires de l'Église. 6°) Enfin une paroisse, un diocèse, un monastère ne sont ni des bâtiments ni des biens fonciers. Ces institutions sont des ensembles de fidèles, c'est-à-dire d'âmes immortelles, qui ne peuvent par nature appartenir à une personne, ou à une famille, à la façon dont la force de travail des serfs (mais non leurs âmes) appartenait à leur seigneur. Les souverains et autres personnages puissants du \siecle{6} et des siècles suivants ont régulièrement utilisé leur influence pour conférer l'épiscopat à des serviteurs laïcs afin de les récompenser pour leurs loyaux services (ou bien pour les neutraliser par cet « honneur » qui leur interdisait en principe tout retour aux armes). L'interdiction de transmettre par héritage les biens de l'Eglise (biens de mainmorte) était un moyen de contrer ce type d'empiètement des autorités civiles, et au minimum d'en limiter les effets.

Voilà pourquoi la doctrine ecclésiale a toujours voulu, malgré des pesanteurs individuelles et collectives qui ont entrainé de nombreux écarts à la règle, que les clercs ne soient pas issus du monde des laïcs. Mais voilà aussi autant de raisons concrètes et matérielles d'attribuer une valeur spirituelle au célibat et à la continence perpétuelle. Les deux niveaux de logique étaient indissolublement liés\footnote{ Quelques points de repère :
 306 : Concile d’Elvire, Espagne, décret § 43 : un prêtre qui a eu une relation charnelle avec sa femme la nuit précédant une messe sera démis de sa fonction.
325 : Concile de Nicée : après son ordination, un prêtre n'a plus le droit de se marier.
385 : Le pape Sirice Il  décrète que les prêtres mariés ne doivent pas avoir de relations charnelles avec leur épouse. 
567 : 2ème Concile de Tours : si un clerc est trouvé couché dans son lit avec sa femme il encourt une excommunication pour une année et la réduction à l’état laïc.
580 :  le pape Pélage II (520-590) choisit de tolérer le mariage des prêtres tant qu’ils ne transfèrent pas les biens de l’Eglise à leurs épouses et à leurs enfants.}.
 
 
 Cela a eu des conséquences très importantes sur la société toute entière. En effet il s'est constitué en son sein une espèce de "caste" non héréditaire recrutée dans les autres castes, à laquelle le recours aux armes était interdit, cultivant au contraire le savoir et la culture, et au sein de laquelle les carrières n'étaient pas déterminées par la naissance, même si ces trois idéaux, toujours recherchés, n'ont jamais été totalement atteints. Les savoirs cultivés dans les institutions d'Église souffraient de limitations certaines et l'héritage antique n'a pas été transmis sans pertes. Tous les clercs n'étaient pas savants, et les plus compétents n'obtenaient pas toujours les promotions auxquelles leurs talents les auraient qualifiés. De même tous les princes de l'Église n'étaient pas à la hauteur de leur charge, et ils ne se tenaient pas toujours à l'écart des lutte pour le pouvoir temporel. Mais c'est tout de meme au sein du corps des moines et des prêtres que se trouvaient les personnages les plus savants de leur époque, et pour ceux qui n'avaient pas les privilèges de la naissance c'est au sein de l'Église qu'ils avaient le plus de chances de promotions. Joseph \fsc{MORSEL} voit dans cet élitisme ecclésiastique et le modèle qu'elle a fourni, profondément intériorisé, l'une des causes principales du développement ultérieur de l'Europe et de son avance sur les autres civilisations (in \emph{L'Histoire du Moyen Âge est un sport de combat}, texte publié au format pdf sur Internet à l'adresse \url{http://lamop.univ-paris1.fr/IMG/pdf/SportdecombatMac.pdf}).
 
 \section{Les religieux}
 Le mouvement monastique s'est développé à partir des expériences des premiers ermites qui ont fui le monde dès le \siecle{3} dans les déserts d'Égypte, et des premières veuves et vierges consacrées qui en ont fait autant à l'ombre des cathédrales et sous la protection des évêques. Il continuera de se développer à un rythme soutenu jusqu'au foisonnement de la fin du Moyen Âge. Il prouvait par sa floraison que la continence \emph{perpétuelle} était possible%
% [2]
\footnote{... même si elle doit parfois s'appuyer sur \emph{l'impuissance de famine}, cf. Aline \fsc{Rousselle}, 1998, p. 203 - 224}% 
, et cela non seulement pour les femmes, de qui depuis toujours on l'exigeait au gré des besoins de leur famille, mais aussi pour les hommes. Saint Augustin, évêque de la fin du \siecle{4} et du début du \siecle{5}, vivait en moine avec ses collaborateurs immédiats, dans une communauté d'où sortiront un jour les chapitres de chanoines officiant dans toutes les cathédrales. Au même moment les hôpitaux s'organisaient dans l'esprit des monastères. Ils étaient construits comme des églises dans lesquelles seraient logés des malades et si le concile de Nicée voulait que leur personnel soit recruté parmi les religieux.

 S'appuyant sur les lettres de Paul de Tarse et les paroles du Christ, l'Église défendait le droit des jeunes de consacrer volontairement et librement leur vie à Dieu, alors qu'ils étaient encore \emph{dans la main} de leur père. Dans ce cas elle défendait leur droit de recevoir leur part d'héritage sans pour autant suivre la voie prévue par leurs parents, part d'héritage sans laquelle leur liberté de choix serait restée formelle. Cela leur permettait de s'engager dans le monastère ou l'hôpital de leur choix en faisant don à leur communauté de leur part d'héritage%
% [3]
\footnote{C'est ainsi qu'était mis en pratique la proposition de donner tous leurs biens aux pauvres faite par le Christ à ceux qui voulaient choisir la perfection (parabole du « jeune homme riche ») : en effet leur nouvelle famille spirituelle n'était constituée que de membres qui avaient fait vœu de pauvreté.}% 
. On peut supposer que ce n'est pas par hasard qu'en 320 Constantin avait abrogé les lois d'Auguste qui exigeaient d'avoir engendré trois enfants et d'être marié pour recevoir les héritages venant de personnes éloignées, et qu'il avait posé des limites au droit des pères de déshériter un enfant. Contrainte par sa propre logique, et fidèle sur ce point au droit romain, l'Église plaidait pour le consentement mutuel des fiancés et contre l'idée que celui de leurs parents était nécessaire pour que leur mariage soit valide%
%[4]
\footnote{Là aussi elle allait contre l'autorité des pères. Cet enseignement-là restait en travers de la gorge de bien des pères, mais aussi des ecclésiastiques eux-mêmes pour autant qu'ils s'identifiaient aux intérêts temporels de leur famille d'origine, cf. les avanies subies par Abélard, alors qu'il était encore laïc et donc épousable, du fait de l'ecclésiastique qui était oncle et tuteur d'Héloïse.}% 
.

 Les revenus des monastères, des évêchés et des hôpitaux étaient fondés sur des propriétés, terres, domaines,~etc., provenant des dons et des legs. Grâce aux rentes sur la terre%
% [5]
\footnote{Ressentie de l'Antiquité à la fin du Moyen Âge (au moins) comme le seul bien qui ne fait jamais défaut, et dont les fruits permettent de survivre quelle que soit la catastrophe économique qui puisse arriver (Paul \fsc{Veyne}, \emph{La société romaine}, chapitre).} 
et les immeubles (en nature ou en argent) il était possible sans recourir à l'impôt de « fonder » (en principe une fois pour toutes) des emplois \emph{(bénéfices)} de clercs, des écoles, des hôpitaux, des monastères,~etc. Ce mode de financement était hérité de l'Antiquité pré chrétienne. C'était déjà celui des temples païens. S'ajoutaient à ces revenus des contributions régulières notamment les différentes \emph{dîmes} versées par les fidèles, d'abord volontaires, puis obligatoires. Ainsi les institutions ecclésiastiques étaient autonomes et auto-suffisantes, sans courir les risques du marché, ni dépendre étroitement de généreux donateurs ou des pouvoirs locaux. Ce système ne faisait peser aucune charge récurrente sur le budget de la puissance publique et donnait aux institutions un maximum de liberté face aux pressions des pouvoirs publics. 

 Jusqu'à la fin du Moyen Âge une part de presque tous les héritages était donnée aux pauvres (c'est-à-dire à leur protectrice officielle : l'Église) pour \emph{le salut de l'âme} des donateurs. Il existait déjà chez les anciens des fondations identiques auprès des temples païens. Quant aux barbares ils admettaient comme les Égyptiens, les Celtes et les Germains que chaque mort emporte dans son tombeau des biens pour l'au-delà, ce qui du point de vue des chrétiens ou des juifs était un signe de superstition. Cette part des biens du mourant qu'il comptait emporter avec lui (jusqu'à un tiers de sa fortune ?) l'Église lui proposait d'en faire meilleur usage, en l'investissant dans les \emph{œuvres pies} (pieuses). 

 Selon Raymond Goody il y avait un lien entre la défense par l'Église de la liberté de choix de vie des jeunes, celle du droit des jeunes à une part d'héritage même en cas de désaccord paternel, celle des chrétiens à faire des donations (notamment dans leur testament) et le financement des institutions religieuses qui fournissaient les lieux où chercher la perfection. Selon lui la nécessité de trouver des ressources pour faire vivre les paroisses, monastères et hôpitaux a exercé une pression déterminante sur la définition même des règles du droit de la famille. Elle aurait contribué à ce que le droit de l'Église mette des limites au droit des pères à imposer leur volonté à leurs enfants. Elle aurait aussi et surtout contribué à étendre les degrés de parenté interdisant les mariages. Même si cette thèse paraît un peu extrême, comme toute thèse qui attribue à une cause unique un mouvement observable sur plus de dix siècles, elle n'en contient pas moins une part de vérité significative. 

 En dehors du travail de leurs membres, qui exigeait lui-même un minimum d'outils de production et d'abord de terres, le financement des monastères reposait sur les \emph{dots} des postulants, notamment dans les monastères féminins qui ne pouvaient bénéficier comme les monastères d'hommes des honoraires de messes offertes pour le repos de l'âme des défunts. Au décès du religieux sa dot demeurait acquise au monastère (du moins tant qu'elle a consisté en un capital et non en une rente). Celui-ci avait donc des chances de voir grossir peu à peu son capital. Cela permettait (dans les meilleurs des cas) d'accepter les postulants sans le sou et de consacrer le superflu au service des pauvres et des malades.
 

\section{Le « mariage constantinien »}


 J'appelle « mariage constantinien » le mariage romain tel qu'il a été modifié par Constantin et ses successeurs pour l'accommoder aux conceptions chrétienne, même si les laïcs n'ont jamais totalement épousé les points de vue des clercs. Jusqu'à la Réforme Grégorienne (\siecle{11}), l'Église n'avait d'ailleurs pas le monopole du droit familial, et les autorités civiles ne se sentaient pas obligées d'appuyer toutes ses prétentions dans un domaine aussi critique pour la transmission du pouvoir. Les écarts entre le droit religieux (droit \emph{canon}) et les lois civiles n'ont jamais été nuls, pour ne pas parler de \emph{l'à peu près} avec lequel ces lois étaient respectées. Ce que j'appelle le mariage constantinien est donc un modèle qui n'a jamais été pleinement réalisé, et surtout pas sous Constantin. Pourtant il tendra peu à peu à s'incarner dans les pratiques et les représentations, et il ne sera peut-être jamais aussi bien respecté que durant les derniers siècles de notre ancien régime.

 Dans le mariage constantinien, plusieurs fonctions distinctes sont télescopées sur une seule personne : un époux est à la fois le détenteur des droits juridiques de son épouse (son curateur), son amant (point trop empressé si possible), le géniteur de ses enfants, le détenteur des droits de ces enfants mineurs, et le responsable de leur éducation, c'est-à-dire leur père légal. Symétriquement, une épouse est la seule femme capable de donner à son époux des enfants légitimes, des héritiers, quel que soit le nombre de ses concubines. Pour être légitime chaque enfant doit être l'enfant biologique de ses parents légaux (leur enfant « naturel » au sens antique du terme). Et par définition seuls les enfants légitimes ont droit à une part d'héritage et à succéder à leurs parents.

 Le Bas-Empire et le haut Moyen Âge continuent de reconnaître sans discussion la validité des concubinages stables monogames non incestueux, et la légitimité civile et religieuse des enfants qui en naissent : Justinien les autorise à hériter, mais il ne fait ainsi que rappeler une règle de droit ancienne. Dans la pratique du Bas-Empire, le concubinage monogame est une forme de mariage souvent (presque toujours ?) employée par les personnes qui ne possèdent pas de patrimoine significatif et ne voient donc pas la nécessité de s'unir en public et solennellement, ni de passer devant un notaire. Dans le même sens, Augustin d'Hippone enseigne qu'une concubine qui se veut fidèle à son concubin lui est mariée devant Dieu de manière aussi légitime qu'une épouse en titre.

 Pour l'Église, \emph{c'est le mariage qui fonde la famille, et non la naissance des enfants}, même si elle met l'accueil des enfants au premier rang des « fins du mariage ». De son point de vue, le mariage crée en effet \emph{dès sa célébration} une parenté nouvelle, \emph{une seule chair}, entre les époux, \emph{qu'ils soient féconds ou non}. Cette parenté « par alliance » a des effets directs et immédiats sur les membres des parentèles des époux (frères, sœurs,~etc.) : elle étend le cercle des partenaires qui leur sont désormais définitivement interdits, même si l'un des époux décède.

 Selon la doctrine chrétienne, identique sur ce point au droit romain, ce sont les époux qui s'unissent l'un à l'autre : cela implique qu'ils soient capables de discernement (âge suffisant, santé mentale) et libres de leur personne : célibataires ou veufs, non esclaves, non engagés par contrat dans une entreprise qui empêcherait la vie commune, à l'abri de toute pression, libres de tout vœux religieux, sexuellement aptes au mariage. L'Église a toujours soutenu contre les parents que les jeunes gens ont le pouvoir de se marier validement sans leur accord, même si elle admettait qu'en leur désobéissant ces jeunes gens les déliaient de leur devoir de les établir dans la vie. 

 Contrairement au droit romain l'Église en est progressivement venue à ne reconnaître la réalité juridique d'un mariage que lorsqu'il a été consommé, assez probablement parce que chez les barbares (cf. chapitre suivant), même christianisés, les unions se construisaient en plusieurs étapes séparées par de très longs intervalles, les premières étapes (dont les promesses de fiançailles) ayant parfois lieu alors que les futurs époux étaient encore de très jeunes enfants. Pour les besoins des procès en nullité de mariage il a fallu trouver un critère décisif dans cette progression, et c'est la consommation du mariage qui a été retenue. 

 Selon les évêques et théologiens chrétiens, le célibat non consacré est licite, mais chez les jeunes gens sans enfants, en bonne santé et disposant de moyens matériels suffisants, il est suspect d'égoïsme, de libertinage ou de désirs « contraires à la nature » (homosexualité notamment dont la mise en acte a toujours été condamnée moralement, même si elle n'a été semble-t-il que rarement sanctionnée). Quels que soient les préférences individuelles la copulation n'est légitime que dans l'état de mariage monogame, qui est le seul moyen acceptable de répondre à l'ordre divin (\emph{croissez et multipliez} de la Genèse). C'est donc l'état normal de tous ceux qui ne sont pas ordonnés à un ministère ou engagés dans la vie religieuse. Mais comme la fin première du mariage est la procréation d'enfants légitimes les remariages sont déconseillés (quoique autorisés) quand cette fin est à priori inatteignable étant donné l'âge ou l'état de santé des conjoints. 

 À partir du \siecle{4} dans l'empire romain, ce n'est plus d'abord et avant tout par la relation de pouvoir qu'il exerce sur les membres de sa maison que le père est juridiquement défini. En effet, il est soumis au devoir de \emph{piété}%
% [2] 
\footnote{La piété était l'affection réciproque et le respect mutuel entre les divers membres de la famille nucléaire, y compris le devoir d'assistance.} 
à l'égard de ses enfants au même titre qu'ils le sont à son égard, et autant qu'eux. L'accent se déplace sur sa responsabilité envers eux. 

D'autre part chez les Romains (mais pas chez tous les barbares alliés à Rome) en cas de décès du père, c'est la mère qui, à partir de 390, exerce la tutelle de ses enfants mineurs (et d'eux seuls) si elle a cinquante ans et plus, et du moins tant qu'elle ne se remarie pas, ce qui est le cas général, indépendamment même des réserves ecclésiastiques face au remariage des veuves dotées d'enfants. À partir de 390 une femme n'est plus considérée comme incapable par nature de représenter juridiquement une autre personne qu'elle-même. Le fait qu'à partir de ce moment elle puisse exercer, de droit, la tutelle de ses enfants (même si c'est sous le contrôle éventuel et plus ou moins étroit de la famille de son mari), manifeste que les droits et les devoirs dits « paternels » sont en réalité dès ce moment ceux du couple parental, même si tant qu'il vit c'est le mari qui représente le couple face au monde extérieur%
%[3]
\footnote{... et ce sera le cas jusqu'aux années 60 du \siecle{20}.}% 
. Au fil des siècles, la mise en pratique de ce principe a varié de pays en pays en fonction de nombreux facteurs. Il est probable que plus l'héritage était mince et la famille de petite importance, plus le droit de la veuve non remariée à exercer en toute liberté la tutelle de ses enfants mineurs lui était reconnu, et inversement. Ainsi dans les familles riches et puissantes, il pouvait y avoir tellement d'intérêts matériels ou politiques en jeu que la veuve n'avait pas forcément beaucoup d'impact sur l'éducation de celui de ses fils qui devait prendre la succession de son mari dans ses fonctions publiques. 

 Jusqu'à Constantin la fécondité de chaque femme mariée appartenait sans limites à son mari. Désormais elle ne lui appartient plus. Il n'est plus permis de se débarrasser des enfants non voulus par l'avortement ou par l'infanticide. Sauf indigence extrême il n'est pas non plus permis de s'en débarrasser par l'exposition ni la vente. Une femme n'a donc plus autant à craindre qu'auparavant qu'on ne l'oblige contre son gré à avorter ou à abandonner son nouveau-né. Mais sa fécondité ne lui appartient pas non plus. Pas plus que son mari elle n'a droit de vie ou de mort sur l'enfant qu'elle porte. Chacun des époux reconnaît aussi à l'autre un droit sur son propre corps. Les deux époux se doivent réciproquement fidélité : c'est un devoir \emph{moral} pour l'homme autant que pour sa femme, et même si ses propres infidélités ne sont pas sanctionnées par la loi tout est fait pour qu'il n'ait aucun intérêt à entretenir des maîtresses%
% [1]
\footnote{Cela ne l'empêche évidemment pas d'avoir des rapports avec des prostitué(e)s, rapports qui par nature ne s'inscrivent pas dans la durée.}% 
. Chacun des deux époux a l'obligation de satisfaire autant qu'il est en son pouvoir les désirs sexuels de l'autre, ce qui veut dire que l'épouse doit accepter les étreintes de son mari, quoi qu'elle puisse penser des risques de grossesse et de santé à quoi cela l'expose, et quels que soient ses propres désirs. Ceci dit la modération est prêchée aux maris, qui se voient prescrire la continence de nombreux jours par an. Le \emph{devoir conjugal} n'est par ailleurs exempt de faute morale que si aucun obstacle n'est mis à la fécondation (coit interrompu, pessaire, douche intime, sodomie,~etc.). Les seuls moyens de contrôle des naissances autorisés par l'Église sont le retard de l'âge au mariage, le célibat et la continence, périodique ou totale. 

 Il n'est plus possible en principe (mais ce principe a mis de nombreux siècles à s'imposer en dépit de la lutte constante de l'Église) de répudier une épouse présumée stérile (en cas de stérilité dans un couple, c'est celle de la femme qui est toujours suspectée en premier). Ou bien les hommes ont la chance de vivre un mariage fécond et de voir au moins l'un de leurs fils légitimes atteindre l'âge adulte pour leur succéder, ou bien ils doivent renoncer à tout héritier direct tant que vit leur épouse%
% [4]
\footnote{Pour les maris les moins patients il ne restait plus que le « divorce à la carolingienne », c'est-à-dire l'assassinat de l'épouse. Cela ne pouvait se faire que si les institutions policières étaient faibles et les parents de l'épouse moins puissants que ceux du mari. Plus l'État était déliquescent, plus il était possible, comme toujours, de prendre des libertés avec toutes les règles de droit, à la condition de disposer de la force.}% 
. Les couples stériles (dont le nombre n'était pas du tout négligeable jusqu'à l'avènement de la médecine moderne, 20~\% environ) et ceux dont aucun enfant n'a atteint vivant l'âge adulte, sont invités à consacrer aux bonnes œuvres, aux pauvres et à l'Église les ressources qu'ils auraient transmises à leurs héritiers s'ils en avaient eus.

 Tous les enfants nés hors mariage sont pénalisés. En principe il n'est plus possible pour un homme de se faire des héritiers sans se marier, même si la prise en charge d'\latin{alumnii} et leur installation dans l'existence reste une bonne œuvre. La \emph{légitimation par mariage subséquent} est désormais la seule exception de plein droit%
% [5] 
\footnote{... jusqu'au \siecle{20}. Les enfants irréguliers légitimés par les empereurs, les rois ou les papes, ne l'étaient pas de plein droit mais à la faveur d'une grâce, qui pouvait toujours être refusée sans justification, et n'allait pas sans contreparties coûteuses.} 
à la pénalisation des enfants nés hors mariage, et ses conditions sont strictes. Chacun, quelque puissant qu'il soit, doit savoir que s'il a l'imprudence de faire un enfant hors mariage ou dans un mariage contesté par son curé, ou par son évêque, ou par son seigneur, par le roi ou par sa propre parentèle, il ne pourra pas le faire reconnaître comme un de ses héritiers sans combat ou sans procès. Cet enfant ne pourra sans doute pas lui succéder. L'exhérédation totale ou partielle des enfants illégitimes est restée jusqu'à la fin du \siecle{20} le premier frein apporté au désir des hommes de se procurer une descendance ailleurs qu'avec leur épouse légitime, même si d'innombrables exemples montrent que cette règle a mis des siècles à s'imposer.

 Et la perspective de se retrouver avec un enfant à charge, seule, sans aucun espoir d'une légitimation (ni même d'une aide significative venant du père de l'enfant lorsqu'il était déjà marié puisque aucune donation au-delà des frais d'éducation n'était plus autorisée depuis Constantin) a été un obstacle majeur à l'exercice d'une sexualité féminine en dehors du mariage ou avant le mariage. 

 Mais les épouses savent aussi qu'il est devenu, sinon impossible, du moins difficile de les chasser de leur maison ou de leur imposer de cohabiter avec une concubine%
% [6]
\footnote{... même si pour ceux dont la puissance excède de beaucoup celle du commun des mortels, aristocrates, rois, la question peut se présenter différemment, et si les amours ancillaires sont de tous les temps.}% 
. Elles sont à peu près assurées que les infidélités de leur époux n'entraîneront de conséquences graves ni sur elles, ni sur leurs enfants, ni sur le futur héritage de ceux-ci. Tout au plus des « aliments » devront-ils être versés aux enfants nés de leurs maîtresses, mais cela ne portera que sur d'assez petites sommes et seulement jusqu'à ce qu'ils soient mis au travail : 8-10 ans. Il n'est plus question de financer leur établissement dans la vie. 





\section{Familles de chair}


 À partir de Constantin les lois de l'Empire, puis celles des royaumes qui en Occident ont repris sa succession, se sont lentement alignées sur les conceptions chrétiennes du mariage et de la génération
\footnote{Cf. Georges \fsc{DUBY}, \emph{Le chevalier, la femme et le prêtre}, 1981.}% 
. Mais dans le même temps la vie familiale à la romaine était également mise à mal par les « barbares ». Ceux-ci ont introduit des pratiques différentes, principalement germaines
\footnote{Jean-Pierre \fsc{POLY}, \emph{Le chemin des amours barbares, Genèse médiévale de la sexualité européenne}, 2003.}% 
, sur les territoires de l'ancien Empire romain d'Occident. Le haut Moyen Âge est un temps de conflits, de coexistence et de compromis entre les droits et coutumes des royaumes « barbares » et le droit romain%
%[3]
\footnote{Cf. Pierre \fsc{PETOT}, \emph{La famille}, 1992.}% 
. On constate l'effacement progressif des traditions juridiques romaines, compensé dans une large mesure par l'élaboration (ou la résurgence) de pratiques non romaines, dites \emph{coutumières}, propres à chaque lieu et caractérisées d'abord par une très grande variété%
%[4]
\footnote{Mais lorsqu'il s'agira à partir du \siecle{12} de reconstruire un droit unifié et cohérent, à la fois dans le domaine civil \emph{(droit civil)} et dans le domaine religieux \emph{(droit Canon)}, c'est au \emph{Code de Justinien}, publié en 529 et 534, que les juristes savants vont se référer.}% 
. 

 Toutes ces réserves étant faites, et malgré une infinité de particularités tenant aux lieux et aux temps, les lignes de force du système articulant en \emph{chrétienté} les familles, les autorités civiles et les institutions d'Église sont demeurées les mêmes au-delà du Moyen Âge%
% [5]
\footnote{... et même au-delà, jusqu'à la Révolution Française, même si des évolutions très significatives ont eu lieu à partir de la Renaissance et des Réformes protestantes et catholiques. Le paradoxe c'est même que c'est aux \crmieme{17} et \crmieme{18} siècles que les familles se sont le plus étroitement conformées aux principes chrétiens, sous la garde conjointe, vigilante et de plus en plus efficace, des autorités religieuses et civiles.}% 
. Sur la délimitation de cette période de l'histoire et en ce qui concerne mon sujet je ne peux que constater que celle qui convient le mieux est celle du \emph{long Moyen Âge} de Jacques \fsc{LE GOFF} (\emph{Faut-il vraiment découper l'histoire en tranches ?} 2013). À bien des points de vue le Moyen Âge ne s'est achevé qu'avec la Révolution Française, même si la deuxième moitié du \siecle{18} (à partir d'un tournant que l'on peut situer vers 1760/1765) participait déjà du siècle suivant, notamment sur le plan des idées, avec le mouvement européen des \emph{Lumières}, mais aussi sur le plan des institutions, notamment dans le domaine administratif (cf. Pierre Legendre). Pour schématiser on pourrait dire qu'il y a une unité dans la période qui va de Constantin jusqu'à l'Encyclopédie (non comprise). 

\subsection{Contrôle de la sexualité et indissolubilté des unions : Saint-Augustin}

Les chrétiens sont-ils coupables, comme on le croit souvent, d'avoir diabolisé les plaisirs de la chair ? La réponse à cette question n'est pas simple. D'abord il faut dire si l'on en croit Paul Veyne qu'ils \emph{n'ont rien réprimé du tout, c'était déjà fait}. Le monde patriarcal des cités antiques n'était en rien un monde de liberté sexuelle, sauf pour les hommes libres, et encore. Puis les philosophes stoïciens étaient passés par là pour exiger des hommes libres eux-mêmes qu'ils orientent leurs désirs vers la seule procréation. Quant aux médecins ils allaient eux aussi dans le sens d'une grande modération dans l'activité sexuelle. 

 Ceci étant dit il est vrai que les théologiens chrétiens des premiers siècles ont longtemps été tentés par les thèses \emph{encratites}%
% [6]
\footnote{Selon \emph{Encyclopedia universalis} : Encratite est un "...\emph{terme signifiant « les continents » (du grec \emph{enkratès}) et désignant plusieurs sectes hérétiques de l'Église ancienne qui prônaient un rigorisme moral radical (interdiction du mariage, abstention de viande et de vin) fondé sur une condamnation de la matière et du corps considérés comme les œuvres d'un démiurge distinct du Dieu suprême. Tatien, d'abord disciple de Justin, à Rome, est traditionnellement tenu pour le fondateur, vers 170, de cette secte ascétique des encratites, probablement dans la région d'Édesse. L'encratisme fut alors proscrit sous ses diverses formes par de nombreux décrets de Théodose I\ier, à la fin du \siecle{4}, et de Théodose II, en 428.
La sévérité des mesures impériales suffirait à témoigner de l'importance de la secte à cette époque. L'encratisme s'est alors confondu avec le manichéisme et a trouvé des prolongements chez les Messaliens et les Bogomiles (et les Cathares). Le rigorisme que pratiquaient ses adeptes se voulait une négation de l'œuvre du \emph{démiurge} (dieu mauvais opposé au dieu bon). Les fondements doctrinaux de la secte consistaient dans le rejet de certaines parties des Écritures, en particulier de l'Ancien Testament, et dans un recours à des textes de la littérature apocryphe présentant des tendances ascétiques très marquées. Certaines positions doctrinales et liturgiques découlaient généralement de la conception encratiste de la création et de la matière : négation du salut d'Adam (Tatien), négation de la résurrection de la chair, docétisme en christologie, utilisation d'eau à la place du vin pour célébrer l'eucharistie. La ligne de démarcation entre l'encratisme et le gnosticisme est difficile à tracer : ce dernier est dans une large mesure marqué par un courant rigoriste, et l'encratisme semble avoir accueilli des spéculations d'origine gnostique}."
Richard \fsc{GOULET}}%
, proches de celles des manichéens qui soutenaient que la matière est mauvaise par nature, que l'âme préexiste au corps, et qu'avec la conception elle chute dans un monde matériel et charnel, lieu de l'esprit du mal
\footnote{... ce qui paradoxalement pouvait conduire les adeptes de ces doctrines à une licence effrénée puisque rien sous le ciel n'avait plus d'importance : « méprises et fais ce que tu veux ».}% 
. Le plus emblématique des théologiens encratites, Tatien (deuxième siècle après J.-C.), considéré comme hérétique par divers \emph{Pères de l'Église}, rejetait le mariage et condamnait l'usage de la viande et du vin comme de tout autre plaisir de la chair. Préconisant l'eau pour célébrer l'eucharistie à la place du vin, il recommandait de se garder de tout acte sexuel, et de ne pas faire d'enfants pour ne pas prolonger l'existence d'un monde qu'il jugeait mauvais.  Dans le même ordre d'idées, selon Robert Markus%
% [8] 
\footnote{Robert \fsc{MARKUS}, \emph{Au risque du christianisme, l'émergence du modèle chrétien (\siecles{4}{6})}, Cambridge University Press, 1990, réédition en Français, Presses Universitaires de Lyon, 2012.} 
divers auteurs du \siecle{4} (comme Jérôme, Ambroise ou Grégoire de Nysse..) pensaient qu'Adam et Ève \emph{avaient été créés sans sexe, avec une "innocence" que certains comparaient à l'innocence des enfants, et que s'ils n'avaient pas péché, les rapports sexuels n'auraient pas été requis pour reproduire et multiplier la race humaine}.

 Avant sa conversion à 32 ans, en 386, Saint-Augustin avait longtemps adhéré aux thèses des manichéens. Au fil des longues années durant lesquelles il a élaboré son oeuvre théologique (il a été évêque de 395 à 430) il a évolué jusqu'à s'opposer à elles avec autant de fermeté que de subtilité. Son contemporain Pélage (vers 350-420) lui aussi s'opposait au pessimisme manichéen, mais il enseignait que chacun peut atteindre à la sainteté par ses propres forces. Sa doctrine valorisait donc la volonté individuelle et minimisait l'utilité de la grâce divine. C'est dans le cadre de la polémique provoquée par cette thèse (condamnée en 418 par le seizième concile de Carthage) qu'Augustin a mis en forme sa doctrine du \emph{« péché originel »} en se fondant sur les versets de la Genèse qui racontent le \emph{péché d'Adam} et ses conséquences. Il lui fallait rendre compte du fait que la propension au péché, c'est-à-dire à la transgression des lois divines, est présente chez tous les hommes dès leur naissance : tous sont tentés de se donner à eux-même leurs propres règles et de se rebeller contre Dieu. Pour expliquer comment se propage de génération en génération cette inclination pour le mal il en est venu à faire de la reproduction humaine le \emph{lieu} de sa transmission, ce qui implicitement ou explicitement jetait le soupçon sur celle-ci. Pourtant dans ses dernières oeuvres il soutenait que l'union sexuelle et la reproduction ont d'emblée fait partie du plan de Dieu qui a créé les humains hommes et femmes, et non unisexes ou asexués, ce qui le conduisait à écrire que : "\emph{Je ne vois pas pourquoi il ne devrait pas y avoir de mariage honorable au Paradis}." Et il concluait : « \emph{Ce n'est pas la chair corruptible qui a rendu l'âme pécheresse, c'est l'âme pécheresse qui a rendu la chair corruptible. »}\footnote{Selon Robert Markus :  «\emph{ Ce qui devait être expliqué n'était pas l'existence de la sexualité, mais plutôt son mode de fonctionnement et l'impact du péché d'Adam sur la sexualité de ses descendants. » « Les problèmes que posait la sexualité n'étaient ni plus ni moins les mêmes que ceux que posait l'homme. »  « La tension que décrivait l'enseignement manichéen, entre deux natures différentes dans un conflit permanent, était maintenant transposée en terme de conflit interne avec soi-même... Ce qui est répréhensible et honteux dans la sexualité n'est pas son existence même, mais sa tendance à être hors de tout contrôle et à échapper à la raison... un tel constat impliquait une réhabilitation de la chair... Et au bout du compte, cela impliquait aussi une réhabilitation du mariage. »}} 
 
  A ses yeux la concupiscence charnelle était un mal moral, parce qu'elle tendait au péché, mais le mariage chrétien en faisait un bon usage. Dans ses ouvrages \emph{De  bono viduitatis} et \emph{De bono conjugali} Il mettait sans ambigüité la continence et la virginité au-dessus du mariage : « \emph{Nous appuyant donc sur la foi et sur la saine doctrine des Ecritures, nous disons que le mariage n'est point un péché, et cependant qu'il est un état moins parfait, non seulement que celui de la virginité, mais même que celui de la viduité} (l'état des veufs continents). \emph{Nous disons que la nécessité présente pour les époux, sans leur ôter le droit à la vie éternelle, les prive, par le fait même, de cette gloire par excellence réservée à la chasteté perpétuelle. Nous disons que dans cette vie le mariage n'est utile que pour ceux à qui la continence est impossible.} ». Mais il ne donnait pas de valeur à la virginité ou à la continence en elles-mêmes : « \emph{Ce que nous louons dans les vierges, ce n'est pas leur virginité même, c'est leur consécration à Dieu dans les exercices d'une pieuse continence.} » « \emph{...Cette proposition ne fait que confirmer celle qui établit pour les vierges une sainteté plus grande que pour les épouses, sainteté qui obtiendra une récompense proportionnée à son degré de mérite. La raison en est que la virginité permet de tourner vers Dieu toutes ses pensées. En effet, la femme fidèle, tout en observant les lois de la pudeur conjugale, ne peut pas ne penser qu'au Seigneur ; sa perfection est donc moindre, puisqu'elle a aussi les pensées du monde en cherchant à plaire à son mari. C'est d'elle que l'Apôtre a parlé en disant que le mariage lui impose la nécessité de penser aux choses du monde et de chercher à plaire à son époux.} »Et il affirmait la valeur intrinsèque du mariage pour ceux qui ne pouvaient être continents sans \emph{brûler} : « \emph{Toutefois j'affirme que le mariage est bon, non pas précisément parce qu'il produit une postérité, mais parce qu'il la produit dans l'honnêteté, dans le droit, dans la pudeur et pour le bien de la société ; parce qu'il sert à donner aux enfants une éducation commune, salutaire et constante ; parce qu'enfin les époux s'y gardent la fidélité et ne profanent point le sacrement.} »
  
Pour lui l'importance de la sexualité dans le plan de Dieu et les risques qu'elle faisait encourir impliquaient une ascèse à laquelle étaient conviés les époux au même titre que les religieux, ascèse aussi méritoire pour les uns que pour les autres, et au sérieux de laquelle contribuait l'indissolubilité du lien contracté entre les deux époux (ce à quoi il attribue un caractère sacrementel), et le refus de tout remariage quelles que soient les circonstances, hors décès de l'ex-conjoint. 

La doctrine augustinienne a fixé la doctrine de l'Eglise de Rome pour un millénaire au moins, mais elle n'a pas empêché les moines du haut moyen-âge, c'est-à-dire l'essentiel des intellectuels de leur temps, de dénigrer la féminité et l'exercice de la sexualité (de toutes les sexualités), \footnote{Il faudra arriver au \siecle{12} pour que ces points de vue soient en partie remplacés par une exaltation du mariage comme état de vie chrétien. }. Quant aux laïcs ils choisissaient parmi tous ces principes et toutes ces règles celles qui leur paraissaient les plus raisonnables ou les moins intenables. Au sein des sociétés chrétiennes il existait donc une tension permanente
\footnote{Jacques \fsc{ROSSIAUD}, \emph{Sexualités au Moyen Âge}, Éditions Jean-Paul Gisserot, Paris, 2012.}. Le remariage après divorce, du vivant du premier conjoint, ne semble avoir été en droit totalement éradiqué d'Occident qu'après les réformes Grégoriennes du milieu du Moyen Âge
\footnote{Cf. Georges \fsc{DUBY}, \emph{Le chevalier, la femme et le prêtre}, 1981.}% 
. Il a longtemps été reconnu comme valide dans ses effets par les autorités civiles, notamment en ce qui concernait la légitimité des enfants à naître, alors même qu'il était sanctionné comme une faute par l'Église (excommunication, pénitences publiques...) ou par les autorités civiles elles-mêmes (amendes, exil, confiscation de biens...).  

 Si Constantin avait ordonné que l'adultère d'une femme avec un esclave soit puni de la mort des deux complices, il avait autorisé le mari d'une femme adultère, entremetteuse ou empoisonneuse à la répudier tout en conservant sa dot, et à se remarier. À défaut de condamnation plus grave l'épouse adultère était reléguée dans une île. Dans les autres cas une femme répudiée conservait sa dot, et si son ex-époux se remariait elle pouvait « envahir » sa maison et prendre possession de la dot de la nouvelle élue. Une épouse pouvait elle aussi répudier un époux coupable d'homicide, d'empoisonnement, de violation de sépulture, et s'en aller avec sa dot. Elle la perdait dans les autres cas où elle prenait l'initiative de la séparation. L'empereur d'occident Honorius a fixé pour chaque époux trois paliers
\footnote{J.-P.~\fsc{POLY}, \emph{Le chemin des amours barbares}, p. 42.}% 
: si le mari répudie sa femme pour « crime grave » (c'est-à-dire les motifs précisés par Constantin) il garde sa dot et il peut se remarier; s'il la répudie pour « faute contre les mœurs » il reprend la donation qu'il lui a faite en l'épousant, mais doit rendre sa dot, et attendre deux ans pour se remarier; s'il la répudie pour d'autres motifs il perd dot et donation, et il ne peut plus se remarier. L'épouse peut de même quitter son mari pour « cause grave » (toujours les motifs de Constantin) et se remarier après un délai de cinq ans\footnote{Il faudra que les tribunaux de l'Église obtiennent vers le \crm{10}\ieme{} ou \siecle{11} le monopole sur les affaires concernant le mariage pour que l'adultère cesse d'autoriser le divorce et le remariage (même si ce n'était pas si simple ni sans notion de faute), et pour que le vieux mot latin \latin{divortium} prenne le sens de séparation sans droit au remariage du vivant de l'autre conjoint, sens qu'il a gardé jusqu'à la Révolution.}. 
 
  À l'occasion du sac de Rome de 410 par Alaric et ses wisigoths,  et des nombreux viols qui l'ont accompagné, Augustin reprend à son compte l'idée, banale à son époque comme à la nôtre, du moins sur le plan rationnel, que le viol ne « souille » pas la victime, mais seulement son auteur. Il en tire la conclusion qu'il n'est pas question que la victime soit punie pour un acte auquel elle n'a pas consenti : "\emph{La sainteté du corps ne consiste pas à préserver nos membres de toute altération et de tout contact... Ainsi donc, tant que l'âme garde ce ferme propos qui fait la sainteté du corps, la brutalité d'une convoitise étrangère ne saurait ôter au corps le caractère sacré que lui imprime une continence persévérante... Nous soutenons que lorsqu'une femme, décidée à rester chaste, est victime d'un viol sans aucun consentement de sa volonté, il n'y a de coupable que l'oppresseur... }" 
S'opposant aux usages valorisés par l'Antiquité en pareille situation il défend les Romaines victimes de viol qui ont choisi de ne pas se suicider : \emph{"Elles ont voulu vivre, pour ne point venger sur elles le crime d'autrui, pour ne point commettre un crime de plus, pour ne point ajouter l'homicide \emph{[d'elles-même]} à l'adultère; c'est en elles-mêmes qu'elles possèdent l'honneur de la chasteté, dans le témoignage de leur conscience ; devant Dieu, il leur suffit d'être assurées qu'elles ne pouvaient rien faire de plus sans mal faire, résolues avant tout à ne pas s'écarter de la loi de Dieu, au risque même de n'éviter qu'à grand-peine les soupçons blessants de l'humaine malignité"} (\emph{La Cité de Dieu}, livre 1, chapitres 18 et 19). C'est de manière paradoxale qu'Augustin s'y prend pour défendre les victimes de viol : si la honte de l'avoir subi sans pouvoir l'empêcher est de tous les temps, honte qui n'est pas en soi liée à un sentiment de culpabilité, il qualifie néanmoins cette honte de faiblesse, de sentiment compréhensible mais non fondé en raison, et contre lequel il convient de lutter. Ce faisant il reprend les argumentations traditionnelles, mais il leur ajoute l'interdiction du suicide, que celui-ci soit de honte, de protestation ou de désespoir. 
Le suicide a toujours été condamné par l'Église, comme il l'était par la Tora\footnote{ \emph{"...comme un péché grave, sauf chez les « fous » ou les victimes d'un « grand chagrin »"} selon le \emph{premier concile de Braga} qui s'est tenu vers 561. Il s'agissait alors pour l'Église de marquer une différence avec la mentalité héritée de la civilisation romaine qui voyait dans le suicide une mort comme une autre pour le désespéré et une voie honorable, un moyen de rachat pour le criminel. (Wikipédia). }. En déniant que le suicide soit une réaction acceptable à la détresse et à la douleur morale éprouvées par les victimes directes de viol, Augustin posait sur leurs épaules un fardeau qui a pu être insupportable à certaines. Mais en leur faisant un devoir moral de survivre, il déniait aussi aux victimes collatérales du viol (époux, enfants, parentèles, voisins) le droit moral de les tuer ou de les pousser au suicide pour apaiser leur propre honte de n'avoir pas été capables, eux non plus, d'empêcher le forfait.
 


\subsection{Phobie de l'inceste}

 Même si les mariages entre cousins germains, et entre nièce et oncle paternel, avaient fini par être autorisés pendant un temps sous l'empire, l'interdit de l'inceste était ressenti avec force à Rome. L'Église partageait cette horreur de l'inceste : elle avait même choisi de comprendre le mot \latin{Porneia} comme désignant exclusivement les unions incestueuses, considérant que la seule cause acceptable de nullité des mariages était la proximité excessive des époux. L'un de ses objectifs était d'écarter du sein des parentèles toute expression des désirs sexuels (hors couples mariés), avec les rivalités, jalousies et rancœurs qui les accompagnent, pour donner toute la place à la seule fraternité et à la \latin{caritas}%
% [10]
\footnote{\latin{Caritas} = amour désexualisé : souci du bien de l'autre, amour de l'autre pour lui-même (même s'il n'a rien d'aimable). Il est souvent rendu par « charité » (dérivé direct de \latin{caritas}), mot où nous ne percevons plus aujourd'hui beaucoup d'amour.}% 
. L'autre objectif était de renforcer le tissu social. Augustin d'Hippone formule ainsi sa pensée : \emph{L'union du mâle et de la femelle, pour autant qu'elle relève du genre humain, est une sorte de pépinière de charité. \emph{[...]} Une très juste raison de charité%
%[12] 
\footnote{\latin{caritas}}
invita les hommes \emph{[...]} à multiplier leurs liens de parenté ; un seul homme ne devait pas en concentrer trop en lui-même, il fallait les répartir entre des sujets différents ; ainsi leur grand nombre contribuerait à préserver plus efficacement les liens de la vie sociale. Père et beau-père sont, en effet, les noms de deux liens de parenté. Que chacun ait un homme pour père et un autre pour beau-père, la charité s'étend sur un plus grand nombre \emph{[...au lieu qu']} un seul homme eût été, pour ses enfants frères et sœurs mariés entre eux, père, beau-père et oncle \emph{[...]}, autres pour le même homme seront alors la sœur, l'épouse, la cousine ; autres le père, l'oncle, le beau-père ; autres la mère, la tante, la belle-mère. Ainsi, loin de se restreindre à un cercle étroit, le lien social s'étendra plus largement et sur plus de têtes par des alliances multiples%
%[13]
\footnote{Livre XV de \emph{la Cité de Dieu}, d'après la traduction de G.~\fsc{COMBES}.}% 
.}

 Plus l'on étend le périmètre de l'inceste plus il faut aller loin de sa famille de naissance pour trouver un conjoint. Cela diminue le risque que les descendants d'une personne (un homme dans le système patriarcal) ne deviennent si puissants, au moyen d'une endogamie stricte de sa descendance, qu'ils puissent menacer le reste de la société, n'ayant pas à composer entre des allégeances multiples. On peut à contrario évoquer les observations de Germaine \fsc{TILLON} dans \emph{le harem et les cousins}, 1966, et l'opposition qu'elle fait entre la « république des beaux-frères » et la « république des cousins ». L'Église s'opposait ainsi aux pratiques orientales, qui privilégiaient le mariage entre cousins (et même entre frères et sœurs en Égypte), comme aux coutumes germaniques qui favorisaient les unions préférentielles entre familles aux alliances redoublées de génération en génération%
% [11]
\footnote{Jean-Pierre \fsc{POLY}, \emph{Le chemin des amours barbares, Genèse médiévale de la sexualité européenne}, Perrin, 2003.}% 
. On ne peut pas dire qu'elle choisissait pour autant un système de parenté contre les autres, même si elle posait le couple entouré de ses enfants, la {\emph{sainte famille}}, au centre de ses préoccupations. Elle mettait seulement une limite contraignante aux systèmes familiaux qui cherchaient à se fermer sur eux-mêmes.

 Au tournant entre le \crmieme{4} et le \siecle{5} les empereurs Théodose, Arcadius et Honorius, ont tenté d'interdire le mariage entre cousins germains, mais ces interdits ont été levés quelques années plus tard par les empereurs (d'Orient) suivants, même si en accord avec Ambroise de Milan et l'évêque de Rome, Augustin plaidait pour cet interdit : \emph{Qui peut douter qu'il ne soit aujourd'hui plus honnête d'interdire le mariage même entre cousins germains ?} Il arguait de sa proximité excessive avec l'inceste fraternel, et du fait que même si les lois de l'Empire l'avaient effectivement autorisé, la coutume romaine n'y était pas favorable. Mais il n'en était pas de même en Orient, où le mariage entre cousins germains était traditionnellement tenu pour idéal, trop inscrit dans la culture pour que l'argumentation d'Augustin puisse y être entendue. En Occident la position d'Augustin, reprise siècle après siècle par l'église, finira néanmoins par triompher.

 À partir du \siecle{7} et surtout du \crmieme{11} au \crmieme{13} en Occident, l'Église entend la notion d'inceste de manière de plus en plus extensive, jusqu'au septième degré, comme le faisait le droit romain ancien. Par dessus le marché à partir du \siecle{7}, elle s'est ralliée progressivement, et non sans résistances même en son sein, à un mode de calcul de ces degrés qui excluait tous les descendants \emph{des arrière-grands-parents des arrière-grands-parents} du sujet concerné%
% [17]
\footnote{Cf. \emph{Histoire du droit civil}, Jean-Philippe \fsc{LEVY} et André \fsc{CASTALDO}, p. 93-95.}% 
, ce qui multipliait de façon exponentielle le nombre des personnes interdites, du moins dans les familles qui prétendaient connaître leurs ancêtres aussi loin dans le passé, celles dont la légitimité reposait sur leur ascendance. Les humbles n'avaient pas une telle prétention, et on n'attachait pas autant d'importance aux irrégularités formelles de leurs unions. En ce qui les concernait il suffisait que l'interdit porte sur toutes les personnes ressenties par eux comme faisant partie de leur parenté. 

 Non seulement les évêques et théologiens d'Occident y ont ajouté toutes les parentés par alliance, y compris beaux-frères et belles-sœurs, mais ils y ont aussi adjoint la \emph{parenté spirituelle} qui liait les parrains et marraines d'un même enfant, et la parentèle de ceux-ci, sans compter les \emph{parentés} illicites nées des rencontres extra conjugales. La phobie de l'inceste a donc conduit à des extrêmes absurdes qui créaient mécaniquement des situations impossibles dans les communautés étroites où les personnes se déplaçaient fort peu en dehors des familles aristocratiques, et où en l'absence de registres d'état-civil il était difficile ou impossible de faire des généalogies fiables. 

 La déliquescence des États et donc celle des cours de justice a fini par assurer à l'Église l'exclusivité du traitement des litiges touchant aux mariages entre le \crmieme{9} et le \siecle{12}. Au fur et à mesure que son influence sur le droit du mariage grandissait sa définition de l'inceste était bon gré mal gré intégrée par les familles dans leurs stratégies. A-t-elle été pour elle un outil de conquête du pouvoir ? L'extension des limites de l'inceste servait objectivement ses intérêts matériels et politiques en multipliant les risques de nullité et les demandes de \emph{dispense}%
% [18]
\footnote{En ce qui concernait ces interdits il s'agissait d'une question de discipline et non d'une règle de foi. L'Église se reconnaissait donc le droit d'en dispenser les fidèles qui en faisaient la demande, mais cela ne se faisait pas toujours sans frais.}% 
. C'est la thèse de Goody, et elle n'est pas invraisemblable (selon Poly elle est plausible, mais à partir du \siecle{11} seulement),

 ... mais il faut observer qu'au même moment les rois et les autres puissants s'appuyaient sur les mêmes principes pour s'immiscer dans les conflits au sein des familles de leurs dépendants, et pour gérer à leur convenance les transmissions des patrimoines, des héritages, et des fiefs. 

 C'est de la même façon que les autorités civiles se sont opposées à ce que les enfants illégitimes reçoivent le même traitement que les autres, notamment dans les héritages. Ils s'opposaient surtout à ce qu'ils puissent hériter des fonctions fournissant un surcroît d'\emph{honneur}, c'est-à-dire les fonctions de pouvoir. Rois et clercs ont mis des siècles à parvenir à cette fin, mais ils y sont parvenus. C'est qu'ils avaient des intérêts convergents dans l'affaire. Les puissants avaient intérêt à ce que les familles de leurs concurrents et de ceux qui dépendaient d'eux aient des difficultés à trouver un héritier, sachant qu'environ une femme sur cinq ayant l'âge de procréer n'était pas féconde, que suivant les « honneurs » à transmettre, les filles ne convenaient pas aussi bien qu'un garçon ou étaient exclues, suivant les législations ou les coutumes en vigueur (ex. la loi Salique chez les Francs), et que les enfants illégitimes ne pouvaient en hériter. Il était avantageux pour les puissants de plaider l'illégitimité des enfants de leurs ennemis promis à un riche héritage pour les en déposséder, et donner à un autre de leur choix l'honneur qui devait leur échoir. 

 Et l'extension à l'infini de l'inceste rendait paradoxalement plus facile, pour tous ceux qui pouvaient assumer un procès canonique, d'obtenir l'annulation d'un mariage pour inceste si une alliance plus profitable ou une femme plus désirable ou supposée plus féconde se présentait : si tout le monde était parent de tout le monde toutes les unions étaient incestueuses, et donc à la merci d'un procès gagné d'avance (en somme : « si vous ne voulez pas être piégé dans un mariage indissoluble épousez votre petite cousine »). 
 
 \subsection{Disparition de l'adoption}

 La première des donations à visée religieuse des païens était leur héritage. De droit c'est leur héritier qui était l'officiant de leur culte mortuaire. À partir du moment où les cultes païens ont été disqualifiés, puis interdits, il n'y avait plus de motif religieux de se procurer à toute force un héritier, puisqu'il n'y avait plus de culte des morts à assumer.  C'est l'Église, et non plus les familles, qui gérait le culte des défunts en même temps qu'elle veillait sur les corps rassemblés dans les cimetières et le sol des églises, ce pour quoi elle recevait des donations. Elle n'avait pas de raison de se soucier de la pérennité des lignées, au contraire, l'absence d'héritiers n'était de son point de vue qu'un malheur individuel, et seulement pour cette vie. Cette absence n'avait pas d'incidence sur le salut de l'âme des défunts après leur mort. On en revenait donc aux règles juives : pas de filiation « fictive ». L'adoption plénière, celle qui fabriquait des héritiers légitimes avec des étrangers, a presque totalement disparu de la scène, mais cela ne s'est pas fait du jour au lendemain et il y a fallu plusieurs siècles, d'autant plus que les familles détentrices d'un « honneur », d'une charge publique, à commencer par les rois et les \latin{domini}, les seigneurs, avaient impérativement besoin d'héritiers pour ne pas perdre leur position sociale, et n'étaient pas d'accord sur ce point avec les clercs. 


\subsection{Enfants en trop, enfants « irréguliers »}

 Ce n'est pas parce qu'elle était interdite (mais sans que des sanctions soient prévues) que l'exposition des enfants avait disparu. Les pauvres ont toujours eu recours à l'exposition et n'ont jamais été sanctionnés pour ce motif. Quant aux ventes d'enfants, interdites en principe, elles ne tombaient sous le coup de la loi que lorsqu'elles obéissaient à d'autres motifs que le dénuement%
% [19]
\footnote{... mais qui vendait son enfant pour d'autres raisons (sauf un enfant que le père de famille supposait né d'un adultère de son épouse) ?}% 
. Bien au contraire les acheteurs ont en réalité été encouragés par le fait que les parents qui abandonnaient étaient déchus de leurs droits. Il faudra que le servage disparaisse aux \crmieme{12} et \siecle{13} pour que les ventes d'enfants disparaissent aussi... Et c'est à partir de cette époque que le nombre des expositions de nouveaux-nés dans les villes va se mettre à poser de sérieux problèmes d'ordre public.

 En accord avec la Bible, l'Église a toujours interdit à ses fidèles l'avortement et l'infanticide, et Constantin a introduit cette interdiction dans le droit romain. Qu'en était-il en réalité ? Les avortements et les infanticides ont-ils d'un seul coup disparu ? Il est difficile de le croire. Les infanticides n'ont certainement pas disparu. Ainsi Grégoire de Tours (539-594) rapporte le cas d'une femme qui avait mis au monde un enfant monstrueux : \emph{Comme c'était pour beaucoup un sujet de moquerie de l'apercevoir, et qu'on demandait à la mère comment un tel enfant pouvait être né d'elle, elle confessait en pleurant qu'il avait été procréé pendant une nuit de dimanche. Et n'osant le tuer comme les mères ont coutume de le faire, elle l'élevait de même que s'il eût été conforme}%
% [20]
\footnote{... cité par D.~\fsc{ALEXANDRE-BIDON} et D.~\fsc{LETTE}, p. 27.}% 
. On croit en effet à cette époque que les naissances d'enfants mal conformés sont le résultat de relations sexuelles durant les périodes d'abstinence obligatoire, pendant le carême ou l'avent, pendant les règles%
%[21]
\footnote{Il en est de même pour la lèpre. Il est si difficile de ne pas savoir pourquoi le malheur vous frappe qu'on préfère encore s'en proclamer responsable.}% 
,~etc.

 Mais le plus suggestif c'est le « \emph{n'osant le tuer comme les mères ont coutume de le faire} ». Il faut remarquer la simplicité avec laquelle Grégoire de Tours rapporte ce qui est pour lui une évidence contre laquelle il ne s'indigne pas. Entre les règles morales, même celles qui étaient inscrites dans la loi, et les pratiques effectives, il y avait une marge, comme toujours, et l'infanticide est si aisé et si difficile à prouver. Les nouveaux-nés sont si fragiles, et il arrivait si souvent qu'ils soient étouffés par mégarde sans intention maligne lorsqu'ils partageaient le lit de leur mère, pour avoir plus chaud ou lui éviter de se relever la nuit,~etc. 

 Les avortements ont pu se raréfier en l'absence de médecins et de sages-femmes compétents et prêts à louer leurs services (à supposer que ces compétences se soient effectivement perdues chez les femmes d'expérience, ce qui est à prouver), mais les avortements n'ont jamais été ressentis comme des infanticides, et tout au plus comme des fautes lourdes. Les avortements précoces étaient d'autant moins culpabilisés que pour la plupart des théologiens du Moyen Âge comme pour ceux de l'Antiquité, l'animation du fœtus n'avait pas lieu au moment de la fécondation, mais bien plus tard, chacun défendant sa propre théorie (Cf. Maaike \fsc{VAN DER LUGT}, \frquote{L'animation de l'embryon humain et le statut de l'enfant à naître dans la pensée médiévale}, in \emph{L'embryon, formation et animation}, collectif, déc 2004, Paris, Vrin, p. 234-254). 

 Les enfants issus de simples mésalliances (sénateur--affranchie, femme libre--esclave,~etc.) ne posaient pas de problème religieux aux chrétiens, pas plus qu'aux juifs, même s'ils posaient des problèmes familiaux et sociaux, et même si le droit romain pourchassait ces mésalliances. Ceux dont l'Église réprouvait vraiment la naissance étaient ceux qui avaient été conçus dans le cadre d'une transgression de ses propres lois morales, les \emph{fruits du péché}. 

 Dans ce domaine, les règles de l'Église viennent presque intégralement des juifs. L'échelle de gravité des fautes est calquée sur l'échelle des \emph{mamzerim}. Y ont été ajoutés les enfants nés des personnes qui ont fait vœu de célibat, par analogie avec le sort des enfants illégitimes des prêtres du Temple de Jérusalem. 

 Les \emph{irrégularités de conception} étaient classées comme suit de la moins grave à la plus grave :
\begin{enumerate}
% a)
\item ceux qui ont été conçus dans le cadre d'un concubinage stable, monogame et sans interdit de mariage, et qui n'ont pas (encore) été régularisés par un mariage subséquent ;
% b)
\item ceux qui sont nés d'un rapport de hasard (fornication) ou d'un concubinage qui n'a pas duré ;
% c)
\item ceux qui ont été conçus alors que leur mère se prostituait (fornication) ;
% d)
\item ceux qui sont nés d'un adultère avéré (enfants adultérins) ;
% e)
\item ceux qui sont nés des relations coupables, consenties, d'un clerc ou d'une religieuse ayant fait vœu de célibat (sacrilège) ;
% f)
\item ceux qui sont nés du viol d'une femme mariée (sacrilège) ;
% g)
\item ceux qui sont nés du viol d'une vierge consacrée (sacrilège) ;
% h)
\item ceux qui sont nés d'un inceste. 
\end{enumerate}

 Tous ces enfants étaient illégitimes. Quand ils étaient le fruit des œuvres de leur père avec une servante ou une concubine, ils ont souvent été élevés dans la famille de leur père, au moins pendant le haut Moyen Âge : chez les Germains cela allait de soi. Par contre même s'ils étaient invités à pardonner à leurs épouses infidèles, les maris n'avaient pas l'obligation d'assumer les enfants adultérins de celles-ci. En ce cas l'abandon anonyme était un droit reconnu officiellement, même aux maris fortunés. Mais ils pouvaient aussi les assumer, comme en droit romain. Quant aux enfants nés d'un « sacrilège » ou d'un inceste il est vraisemblable qu'ils étaient le plus souvent traités comme des enfants abandonnés.


\subsection{Les éducations}

 Pour la plupart des enfants des villes (qui représentent peu de chose à l'époque) la petite enfance se passe à la campagne chez une nourrice. La mise en nourrice a concerné plus d'enfants que tous les internats éducatifs, collèges ou hôpitaux réunis, et de très loin. C'était en effet une nécessité absolue pour les femmes des villes qui exerçaient un métier : elles ne pouvaient consacrer à l'allaitement le temps nécessaire jusqu'au sevrage de l'enfant (à deux ans), et il n'y a eu jusqu'au \siecle{20} aucun substitut valable au lait féminin. La généralisation du nourrissage mercenaire reposait aussi sur la possibilité de gagner (à compétences égales) beaucoup plus d'argent en ville qu'à la campagne. Cela permettait aux citadines, même de ressources modestes, d'acheter le lait et le temps des paysannes. La croissance des villes a donc entraîné une augmentation massive du recours à la mise en nourrice.

 Cela n'a pu se faire aussi largement que parce qu'était peu ou pas perçue l'influence des premières relations de l'enfant avec sa mère ou un substitut sur la construction de sa personnalité : le bébé était censé n'avoir besoin que de lait. Le nourrissage mercenaire est donc une institution dont il a été fort peu parlé pendant des millénaires. Cette pratique n'était ni pensée, ni pensable. Elle était du côté des corps et de la nature, des réalités féminines, au même titre que la grossesse et l'accouchement, qui se faisaient aussi bien quand les hommes n'en parlaient pas, sinon mieux. Seul présentait de l'intérêt pour ces derniers ce qui commençait avec l'âge de raison (7 ans).

 Dès qu'ils ont l'âge de raison les enfants de ces temps ne sont plus regardés comme fondamentalement différents des adultes. Ils ne reçoivent aucune protection spéciale (protection du corps contre les gestes traumatiques, protection des yeux et des oreilles contre les spectacles traumatiques). L'éducation est rude et les sanctions sévères. Celui qui économiserait les verges et le fouet croirait mal faire. Les orphelins continuent d'être l'objet de toutes les attentions des autorités. Quant aux jeunes délinquants, condamnables dès 7 ans, ils perdent à 12 ans \emph{l'excuse de minorité}, qui de toute façon n'est pas automatique même avant cet âge. Ils sont en tout traités comme des adultes.

 Dans l'immense majorité des cas chacun apprend de son père le métier de son père. Dès 6 ans la plupart des enfants travaillent autant qu'ils le peuvent. A partir de cet âge un enfant de pauvre ne coûte plus guère. Sauf chez les riches et des puissants, dès 12 ans chacun gagne réellement le pain qu'il mange chez ses parents ou chez un maître. Le placement réciproque des adolescents chez des alliés des parents (oncles, surtout maternels, suzerain,~etc..) est un outil éducatif souvent employé (les jeunes nobles servent comme pages, les fils de paysans comme pâtres, les marins comme mousses,~etc.). Le placement en apprentissage chez un artisan (maître ès arts) n'est possible que si les parents paient l'apprentissage : c'est un luxe auquel les pauvres ne peuvent pas prétendre. Pour l'école il en est de même. 

 L'Antiquité grecque ou romaine avait élaboré à l'intention de ceux qui pouvaient se le payer un système complet d'enseignement (primaire, secondaire et supérieur). D'autre part un certain nombre de postes de professeur du secondaire étaient payés par les cités, et l'état romain finançait des chaires d'enseignement supérieur (Augustin d'Hippone en est un représentant illustre : il se décrit successivement élève, étudiant, enseignant et titulaire de chaire). Au \siecle{4} ce système continue de fonctionner, au \siecle{6} il est pratiquement en ruines en Occident, alors qu'il perdurera encore dix siècles à Byzance sans changements de structure. Les universités européennes ne relèveront le flambeau qu'à partir des derniers siècles du Moyen Âge. 

 En attendant seules résistent les écoles cathédrales et monastiques. Les premières écoles \emph{épiscopales} ou \emph{cathédrales} fleurissent au \siecle{4}. Elles ont pour principal objet de former les futurs clercs, mais les élèves peuvent à la fin du cursus refuser d'entrer dans le clergé. Le \emph{deuxième concile de Vaison} (529) prescrit à chaque prêtre chargé de paroisse de mettre en place une \emph{école paroissiale} à l'intention des jeunes les plus vifs d'esprit. Ce sont les premières attestations de \emph{petites écoles}. Elles sont d'abord destinées à alimenter \emph{l'école cathédrale} en sujets d'élite destinés à former le personnel ecclésiastique, et n'ont pas pour but d'apprendre à lire à tous comme c'est le cas chez les juifs : la vie religieuse du chrétien n'exige pas qu'il sache lire, il suffit qu'il sache entendre. Son activité professionnelle ne l'implique pas non plus : l'enseignement lettré (maîtrise du latin, langue de la culture et des savants) est inutile à qui ne sera pas clerc. Si l'on cherche le pouvoir il est alors plus efficace de connaître les armes que la rhétorique ou le droit. 
 
 Les \emph{écoles monastiques} apparaissent aussi au \siecle{4}, mais elles ne prennent en principe que des enfants destinés à devenir moines (« donnés » très jeunes à Dieu par leurs parents) et seuls apprennent le latin, les moines « de chœur », ceux qui chantent dans le chœur, ceux qui pourraient être ordonnés prêtres. Pourtant bien des écoles monastiques acceptent aussi quelques jeunes qui ne sont pas destinés à devenir moines, et Charlemagne leur en fera l'obligation.
 
 
 % Le 12 mars 2015 :
% \latin
% ~etc.
% Moyen Âge
% Antiquité


\section{Familles spirituelles}


 Si les jeunes gens pouvaient se marier validement sans l'accord de leurs parents, en bonne logique ils avaient aussi le droit de ne pas se marier. Devant le désir d'un jeune de devenir religieux l'autorité du père s'arrêtait si le jeune en appelait à son évêque : pour les garçons à partir de 14 ans, pour les filles à partir de 12 ans. Le choix de la vie religieuse émancipait ceux qui le faisaient avant l'âge de leur majorité, et les mettait sous sa protection : cela reposait évidemment sur une reconnaissance par les autorités civiles de la validité des vœux religieux. Cette reconnaissance leur a été accordée par les derniers empereurs romains (chrétiens) et a été reconduite jusqu'à la Révolution française.

 Ceux qui se sentaient attirés par une vie de célibataire consacré pouvaient proposer à une communauté religieuse de les coopter. Ce choix de vie entraînait des incidences légales importantes et définitives, puisque la loi civile l'entérinait. En effet en prononçant ses vœux (pauvreté, chasteté, et surtout obéissance) le moine ou la religieuse se mettaient sous la puissance du responsable de la communauté, comme s'ils s'étaient fait adroger. Ils étaient juridiquement exclus de leur famille de naissance, et de tout héritage à venir. Comme des mineurs ils ne pouvaient plus rien faire de leur propre initiative. S'ils ne pouvaient signer aucun contrat en leur nom propre, ils pouvaient toujours, de la même façon qu'un esclave ou qu'un fils en puissance de \latin{pater familias}, exercer au nom de leur supérieur(e) tout mandat qu'il lui convenait de leur confier. Une fois entrés dans la communauté, c'était en principe pour toujours. Ils ne pouvaient plus sortir de leur état. Ils pouvaient dans une certaine mesure changer de monastère et même d'ordre religieux, mais les \emph{gyrovagues} qui erraient de couvent en couvent étaient mal vus. 

 Aucun religieux ne possédait rien qui lui soit personnel, et pourtant beaucoup d'entre eux avaient reçu de leurs parents une part d'héritage sous forme d'argent, de terre,~etc. À leur entrée dans la communauté ils en avaient fait don (eux ou leurs parents) à la communauté, qui en contrepartie s'était engagée à les prendre en charge jusqu'à leur mort. Chaque communauté vivait de son travail et des revenus des biens qu'elle avait reçus en don : dots des religieux vivants \emph{et décédés}, loyers, récoltes, rentes et autres dons reçus de bienfaiteurs. Tous les biens appartenaient au monastère et celui-ci possédait le droit de posséder et d'exercer des actes juridiques en son nom propre. Si les religieux se succédaient de génération en génération, le monastère en tant qu'entité n'en persistait pas moins dans son être, unissant les morts et les vivants dans le même ensemble intemporel. Le modèle familial ainsi mis en œuvre était accepté en toute connaissance de cause ainsi que le montre l'emploi très précoce du vocabulaire de la famille : « père » (\emph{abba} = père en araméen = abbé), « mère », « frère », « sœur »,~etc. Mais cette famille n'avait pas d'héritiers à pourvoir et ses biens étaient inaliénables et insaisissables, protégés par le statut de la \emph{mainmorte}. 
 
 Ce mot a deux sens :
\begin{enumerate}
% a)
\item c'est le droit du seigneur de prendre les biens de son serf à sa mort. Les biens font \emph{échute}, c'est-à-dire réversion au seigneur qui en hérite. En ce sens les serfs sont \emph{gens de mainmorte}. Ce n'est pas le sens du mot \emph{mainmorte} qui nous concerne ici%
%[1]
\footnote{... même si un bon nombre des derniers serfs (fin \siecle{18}) appartenaient à des communautés religieuses de l'Est de la France, et si à cette époque les religieux ont eux aussi été nommés \emph{gens de mainmorte}, parce qu'incapables de transmettre des biens à des héritiers, non comme serfs d'un seigneur, mais comme religieux.}
 ;
% b)
\item on appelle aussi \emph{biens de mainmorte} ceux qui appartiennent à une personne juridique : ce sont les biens des collectivités qui ont le privilège de pérennité et n'ont pas à transmettre leurs biens à des héritiers. Cela conduisait à l'enrichissement progressif des communautés bien gérées... jusqu'au jour où leurs richesses devenaient trop tentantes pour les puissants du moment et leur étaient (re) prises par l'un d'entre eux : de ce point de vue l'histoire de la plupart des monastères est celle d'une suite de périodes d'accumulation et de moments de spoliation.
\end{enumerate} 

 À la fin de l'Antiquité il était admis que dès leur plus jeune âge (6 ou 7 ans) les parents puissent faire don à un monastère d'un ou plusieurs de leurs enfants, légitimes ou non%
% [2]
\footnote{Sources : Marc \fsc{BLOCH}, \emph{La société féodale}, Paris, 1939, 1994. Georges \fsc{DUBY}, \emph{Féodalité}, Paris, 1996, 1999. Collectif, \emph{L'homme médiéval}, Paris, 1989.}% 
. Ils accompagnaient le « don » de l'enfant d'un cadeau, souvent un bien foncier, qui devait permettre de subvenir à son entretien. Si l'enfant \emph{à Dieu donné} découvrait un jour que ce mode de vie ne lui convenait pas, il lui était extrêmement difficile d'en sortir. Le droit civil et les enseignements de l'Église se liguaient pour lui prêcher la résignation et lui barrer tout retour. Le jeune \emph{donné} à un monastère n'était d'ailleurs pas forcément plus contrarié dans ses choix que les jeunes esclaves, ou que les jeunes gens qui au même moment étaient mariés par leurs familles sans tenir compte de leur avis, ou qui devaient reprendre le métier de leur père. D'autre part, le \emph{don à Dieu} côtoyait des situations contemporaines par rapport auxquelles il représentait un progrès relatif (cf. {Boswell}) : abandon anonyme, infanticide, vente par les parents comme esclave,~etc. 


Les jeunes « donnés » ont pu à certaines périodes représenter une proportion importante de l'effectif des monastères, mais ceux-ci fournissaient aussi à ceux et celles qui n'étaient pas ou plus attirés par le mariage un moyen de l'éviter, alors que le célibat non consacré était mal accepté par la société civile. Cela permettait aux veuves d'échapper à la nécessité de se mettre sous la protection d'un mari. Cela donnait une chance aux femmes les plus douées de jouer un rôle public auquel elles n'auraient jamais pu rêver autrement. C'était le seul moyen pour les filles d'esquiver un mari grossier, mesquin ou brutal, et/ou d'éviter de risquer leur vie dans les grossesses et les accouchements%
% [3]
\footnote{... qui à l'époque faisaient mourir (en hôpital) une femme sur dix environ si l'on en croit les mémoires de \fsc{TENON}, ce qui ne représente pas une naissance sur dix, bien évidemment, mais d'une naissance sur 30 à une naissance sur 120 suivant les temps et les lieux (p. 242 et suivantes). Ce chiffre avait de quoi angoisser les jeunes filles, surtout celles de santé fragile, ou celles qui présentaient une malformation. Jacques Tenon était chirurgien à l'Hôtel-Dieu de Paris avant la Révolution. À la demande des autorités il a rédigé ses \emph{mémoires sur les hôpitaux de Paris} édités en 1788. Nous le citerons souvent.}%
. C'était un refuge pour les jeunes gens mal conformés ou de santé trop fragile. 

 D'un autre point de vue, l'entrée en religion d'un enfant légitime diminuait mécaniquement le nombre des petits-enfants à naître (qu'il faudrait « établir » un jour sur le capital familial). C'était donc une forme indirecte de contrôle des naissances. C'est l'une des raisons, sinon la première, pour lesquelles les seigneurs grands et petits ont créé tant de monastères : ils avaient un intérêt direct à disposer d'institutions où placer l'excédent de leur progéniture dans un cadre conforme à la dignité de leur famille, et sans contrevenir aux lois de l'Église, qu'ils avaient à peu près intériorisées. D'ailleurs si la politique familiale l'exigeait (par exemple si les enfants privilégiés dans un premier temps décédaient) il n'était pas impossible de relever de ses vœux et de faire sortir du cloître une fille ou un fils, à la condition qu'il n'ait pas été ordonné prêtre (mais le plus souvent les moines ne l'étaient pas : pour le droit canon, c'étaient des laïcs, à l'exception de ceux qui étaient ordonnés diacres ou prêtres).

 Par ailleurs même quand pour les personnages publics puissants il était difficile de faire d'un fils illégitime un successeur. D'autre part il était interdit d'ordonner prêtres les garçons illégitimes : pour eux le clergé séculier n'était un débouché envisageable qu'au prix de dispenses coûteuses. Au contraire les monastères ne manifestaient pas de réticences à les accueillir, tout comme ils accueillaient les enfants abandonnés. La vie des religieux est conçue comme une vie de purification. De plus c'est une vie cachée à l'écart du monde. Dans l'esprit du temps cela convenait parfaitement aux pécheurs et pécheresses repentis, aux natures perverties par le péché, et donc aux « impurs de naissance » ou aux clercs séculiers punis pour fautes graves. En outre, on considérait qu'ainsi les enfants illégitimes pouvaient racheter la faute de leurs parents. Il paraissait donc très louable de les vouer à la vie religieuse. Par ailleurs cela les excluait des jeux de pouvoir dont le monde profane est le théâtre. Il n'y avait plus à craindre de les voir parasiter les politiques familiales. C'est ainsi que les parents pouvaient estimer s'en sortir \emph{par le haut} du casse-tête créé par leurs enfants illégitimes ou surnuméraires.
 
 C'est aussi pour cela que tant de filles de rois, légitimes ou non, qui ne pouvaient sans déchoir être données en mariage à des aristocrates trop inférieurs en dignité à leur beau-père, et qui risquaient si on les mariait à de trop puissants seigneurs de donner naissance à des garçons d'ascendance royale susceptibles de menacer les héritiers légitimes du trône, se sont retrouvées abbesses d'abbayes royales, jusqu'au \siecle{18}. 

 La part d'héritage (un bien foncier, une somme acquise définitivement par le couvent dès la profession, la \emph{dot},~etc.) donnée par leurs familles aux futurs religieux était fonction de la fortune familiale et du prestige de la maison religieuse où ils entraient. Chaque ordre et chaque monastère possédaient une « cote » sur le marché des valeurs de prestige, ce qui justifiait un coût (et inversement). En règle générale la part d'héritage de celui ou celle qui entrait dans les ordres était bien plus petite que celle des autres enfants de sa famille. Il convenait en effet qu'une fois tout réglé il reste aux pères un bénéfice financier à faire entrer des enfants en religion. 

 Selon la plupart des règles une communauté vivait non seulement de ses rentes, mais aussi du travail de ses membres. Mais souvent seuls les \emph{convers}, enfants de pauvres reçus gratuitement sans part d'héritage (ou adultes qui se donnaient eux-mêmes ainsi), travaillaient de leurs mains : sauf dons intellectuels ou spirituels éclatants ils n'étaient instruits que superficiellement et ils effectuaient l'essentiel des besognes matérielles, tandis que les religieux mieux dotés par leurs parents étudiaient, apprenaient à lire et à écrire, apprenaient le latin, chantaient au chœur, copiaient les livres, enseignaient,~etc. Il y avait là une évidente sélection par l'argent et par la naissance. Pendant très longtemps il semble que nul n'y ait vu matière à scandale. C'est que depuis l'empire romain (dès Caracalla, sinon avant) les sociétés civiles contemporaines étaient très inégalitaires avec des castes et des hiérarchies civiles justifiées par l'idéologie de l'hérédité (du « sang »), à laquelle les barbares adhéraient autant sinon plus que les Romains. Il est vrai aussi que les exhortations de Paul de Tarse (entre autres) à demeurer à la place où Dieu vous a mis ne favorisaient pas la mise en question de l'ordre établi%
% [4]
\footnote{Il faudra attendre les ordres mendiants à partir du \siecle{12} pour que ces discriminations par l'argent au sein des ordres religieux soient dénoncées, mais non supprimées. François d'Assise, fondateur des franciscains, était fils de bourgeois, non d'aristocrate, ce qui lui avait sans doute donné un autre regard sur le caractère « naturel » des discriminations de caste. Elles ne semblent pas avoir été vécues comme insupportables avant le \crmieme{18} ou \siecle{19}, du moins pour la plupart des religieux qui avaient droit à l'écriture et qui ont laissé des témoignages : mais ce n'étaient pas eux qui étaient ainsi humiliés.}% 
. 

 Si les familles des bienfaiteurs et des fondateurs entretenaient des liens étroits avec « leur » monastère pour conserver la possibilité d'y placer des enfants, elles le faisaient aussi et au moins autant parce qu'elles comptaient sur les prières des religieux, sur leurs messes et sur leurs autres dévotions, offices divers qu'elles « fondaient » contre donation pour garantir le salut éternel des âmes de leurs membres. Chacun de ceux qui le pouvaient affectait une part de ses biens à ces donations comme leurs ancêtres pré chrétiens avaient affecté une part de leurs biens (un tiers selon Goody ?) aux sacrifices à faire après leur mort et aux objets qu'ils emportaient avec eux dans la tombe. C'est pourquoi jusqu'à la fin du Moyen Âge presque tous les testaments contenaient des donations pour le repos de l'âme du défunt, faites à une institution religieuse et/ou d'assistance, ce qui à l'époque était indiscernable : toutes les œuvres d'assistance étaient aussi des « \emph{œuvres pieuses »}. Cela a contribué à produire en quelques siècles un quasi-monopole de l'Église dans le domaine des testaments et des conflits qui y sont liés.
 
 
\section[L'esclavage chez les chrétiens de l'Antiquité tardive et du haut Moyen Âge]{L'esclavage chez les chrétiens\\de l'Antiquité tardive\\et du haut Moyen Âge}


 Aucun auteur antique n'est allé jusqu'à une condamnation de l'esclavage en tant que système. À cette époque, un monde sans esclaves n'était pas pensable. Seules certaines sociétés arriérées et misérables d'alors se passaient réellement d'esclaves, et elles n'avaient rien de désirable pour les autres. 

 Selon Jean \fsc{ANDREAU} et Raymond \fsc{DESCAT} (\emph{Esclave en Grèce et à Rome}, 2006, p. 220) : \emph{celui qui est allé le plus loin dans la condamnation de l'esclavage reste Grégoire de Nysse, au \siecle{4}. Non seulement il estimait que, devant Dieu, les esclaves sont les égaux des hommes libres, mais il regardait la possession d'esclaves comme un péché et un très grave péché. En effet, quoique toutes les créatures soient au service de Dieu et appartiennent à Dieu, le propriétaire d'esclaves s'est approprié certaines de ces créatures, ce qui revient à défier l'ordre divin et à revendiquer un droit qui ne peut être que celui de Dieu} [...] (Homélie IV sur l'Ecclésiaste, 2, 7) [...] \emph{Grégoire a-t-il libéré tous ses esclaves ? Nous n'en savons rien. Mais \emph{[...]} même lui n'a pas milité pour l'abolition de l'esclavage}. Selon les mêmes auteurs, deux groupes dissidents juifs, les Esséniens et les Thérapeutes, étaient opposés à l'esclavage, mais n'ont pas non plus milité en ce sens : \emph{aucun penseur antique ne l'a fait, et il n'y a jamais eu, dans l'Antiquité, de mouvement abolitionniste}... Saint Augustin interprète l'existence de l'esclavage comme une conséquence du péché originel, et pour lui comme pour les stoïciens le fait d'être esclave du péché était bien plus grave que celui d'être esclave d'un maître. 
 
 Même si l'Église a toujours soutenu le caractère méritoire des affranchissements elle ne s'est donc pas attaquée à l'institution de l'esclavage. Elle a reçu sans états d'âme des esclaves en donation, elle en a acheté, elle en a employé. Elle a même fait obligation aux évêques et prêtres qui libéraient un esclave de le racheter sur leur fortune personnelle ou de le remplacer par un autre esclave de valeur équivalente pour que le patrimoine de l’Église dont la gestion leur avait été confiée ne soit pas diminué (ex. conciles espagnols \crm{4}--\siecles{5}{6}). 

 Cela étant dit il y a des choses que l'Église ne supportait pas :
\begin{enumerate}
% 1°)
\item Qu'un esclave ne puisse pas devenir chrétien alors qu'il le désirait, et qu'il soit empêché de satisfaire aux prescriptions d'une vie chrétienne régulière (culte dominical, jeûnes,~etc.). Elle était contrainte par les lois civiles de demander l'autorisation expresse des maîtres avant tout baptême d'esclave, mais elle vivait comme une persécution le fait qu'ils la lui refusent.
% 2°)
\item Qu'un ou une esclave soit contraint à des pratiques contraires au Décalogue, entre autre dans le domaine sexuel. Elle n'acceptait pas qu'un esclave soit condamné à vie à un célibat non choisi, ou à une vie de promiscuité sexuelle, ou à des avortements ou à l'exposition de ses enfants,~etc.
% 3°)
\item Elle défendait le droit au mariage des esclaves. Pour « unir » deux esclaves, il suffisait que le maître permette que soit organisée une cérémonie interne à la \latin{familia} où toutes les personnes présentes, libres et esclaves, étaient les témoins des conjoints et faisaient la fête avec eux, mais cet acte était infra juridique et purement domestique. Cette union (\latin{contubernium}, ou compagnonnage de chambrée) n'avait pas la valeur d'un authentique mariage au-delà des murs du domaine, et le maître n'était pas tenu de la respecter. Elle ne donnait pas aux intéressés de droits parentaux. A contrario l'Église reconnaissait le mariage des esclaves du moment qu'il était monogame, fidèle et inscrit dans la durée, sans distinguer leur union de celle des personnes libres.
% 4°)
\item L'Église ne supportait pas qu'on sépare les couples d'esclaves, qu'on les vende chacun de son côté, ou qu'on leur rende la vie commune impossible. Elle ne supportait pas que les enfants des esclaves soient séparés de leurs parents, confiés à d'autres personnes contre leur gré, et encore moins vendus de leur côté. Le corollaire du droit au mariage, sans lequel ce droit n'a aucun sens, est en effet qu'il soit garanti à ceux qui se marient un minimum de maîtrise sur le temps à venir et de droits sur leur conjoint et sur leurs enfants. 
\end{enumerate}

 L'Église n'acceptait donc l'esclavage que pour autant que le statut des esclaves soit aménagé, de même que les juifs n'acceptaient l'esclavage d'un coreligionnaire que s'il était traité en mercenaire (ou en gagé pour dettes) et non asservi à perpétuité. Si l'on acceptait ces exigences, le statut des esclaves se rapprochait de celui des hilotes grecs et de divers autres statuts de dépendants, dans lesquels la force de travail de ceux-ci appartenait à leurs maîtres de manière héréditaire, mais pas leurs corps ni leurs droits parentaux. Les serfs du Moyen Âge \emph{n'étaient pas des esclaves}, mais le latin ne possède qu'un seul mot pour désigner ces deux statuts \latin{(servi)}, ce qui plaide en faveur d'une évolution progressive de l'un vers l'autre sur plusieurs siècles.

 Les esclaves constituaient une part relativement importante de la population du haut Moyen Âge. Ils étaient toujours l'objet d'achat et de vente, et leurs enfants appartenaient toujours au maître de leur mère. La réduction des chrétiens en esclavage par la force avait été interdite par les derniers empereurs, de même que depuis des siècles il n'était pas permis d'asservir des citoyens grecs ou romains ... mais rien n'empêchait personne de se vendre soi-même. À qui n'avait ni alliés ni capitaux ni culture ni savoir-faire rare, il n'était pas plus facile qu'auparavant de trouver de quoi vivre. Était toujours à bon droit traité comme esclave celui qui se reconnaissait comme tel. L'interdiction d'asservir les chrétiens ne concernait pas les tribunaux, libres de condamner des coupables à l'esclavage. S'il était interdit d'asservir des chrétiens nés libres, rien n'obligeait à libérer les chrétiens qui étaient nés esclaves, même si leur affranchissement était un acte si louable que les évêques s'y impliquaient personnellement. Et les interdits consécutifs au fait d'avoir été un esclave (interdits qui constituent la \emph{marque servile}) frappaient toujours les affranchis, montrant la persistance des anciennes représentations. D'ailleurs les institutions ecclésiastiques possédaient leurs propres esclaves sans y voir aucun mal. Le principal souci des évêques était d'empêcher que les esclaves chrétiens ne soient soumis à des maîtres païens ou juifs susceptibles de les détourner de la foi chrétienne, et qu'ils ne soient déportés dans des contrées non chrétiennes. 

 Si la réduction en esclavage de chrétiens libres (« ingénus ») était un crime pour l'Église et pour les pouvoirs civils, il n'en était pas de même en ce qui concernait les autres (juifs, païens, hérétiques et schismatiques divers...), qui pouvaient être asservis sans problème. A fortiori n'étaient pas affranchis non plus les païens capturés à la guerre ou à la chasse aux esclaves (parmi les attraits de la guerre, celui d'y faire des esclaves demeurait aussi important que par le passé) même si leurs maîtres chrétiens avaient le devoir moral de les faire baptiser, en vertu de leur autorité sur la totalité des membres de leur maison. D'ailleurs ne valait-il pas mieux, comme toujours, asservir les vaincus plutôt que de les passer au fil de l'épée, eux, leurs femmes et leurs enfants, lorsqu'ils n'avaient pas les moyens de payer une rançon ?

 Le commerce des esclaves, notamment non-chrétiens, a donc perduré bien au-delà du Moyen Âge, alimenté par des circuits divers (\anglais{slave} vient de « esclave »), même si à partir du milieu du Moyen Âge, l'esclavage en tant que tel n'a plus joué en Europe un rôle important, sauf exception locale (Espagne, pourtour de la Méditerranée). 

 Le statut des esclaves s'est pourtant insensiblement modifié au fil des siècles : la plupart des esclaves ruraux ont été \emph{chasés}, c'est-à-dire installés dans une \latin{casa}, une maison, avec la pièce de terre plus ou moins étendue que le maître y adjoignait, et une concubine attitrée prise dans sa \latin{familia}, comme les \emph{colons esclaves} de l'Antiquité romaine. Mais dans le même temps le statut de beaucoup des tenanciers libres, \emph{colons libres} d'un propriétaire, ou propriétaires indépendants \emph{(alleutiers)}, sans oublier les affranchis, s'est dégradé au fil du temps pour se rapprocher de celui des esclaves. Pour une part de la population, plus ou moins grande selon les lieux, le résultat de ces deux mouvements a été la généralisation du statut de \emph{serf}, qui attachait chacun de manière héréditaire à une terre ou à un office, et l'assujettissait au seigneur \latin{(dominus)} de cette terre ou de cet office. 

 Ils étaient possédés par leur emploi, ils n'étaient pas totalement libres d'employer leur temps et leurs forces à leur gré. Ils ne pouvaient ni s'en aller ni se soustraire aux ordres reçus. Ils devaient se marier sur le domaine. Une partie de leurs droits personnels appartenaient au seigneur. Par contre et contrairement aux esclaves, ils jouissaient du reste de leurs droits personnels, notamment conjugaux et parentaux ... mais certaines terres, certains offices au service des puissants valaient parfois qu'on s'asservisse pour eux. Certains « postes » de serfs étaient jugés très enviables, au même titre que certains esclaves de personnages puissants pouvaient provoquer des jalousies.

 Le servage était une promotion pour les esclaves, mais une régression pour les personnes libres. En acceptant leur dépendance, celles-ci voyaient s'aliéner une part très significative de leur liberté. En contrepartie, elles faisaient partie d'une communauté villageoise. Les gens des villages, serfs ou libres, n'étaient pas toujours incapables de faire bloc et d'exercer une pression sur leur seigneur, qui avait besoin de leur prospérité matérielle autant qu'ils avaient besoin de sa protection. Ils pouvaient dans une certaine mesure intervenir en tiers entre un serf et lui. En droit comme en fait, il était assez difficile au \latin{dominus} de chasser un serf de sa terre. 
 
 En France l'esclavage a disparu au profit du servage vers le \siecle{8}, et le 3 juillet 1315 Louis Le Hutin a décidé que sont libres tous les esclaves chrétiens qui posent le pied sur le territoire français. 

 Cela n'empêchera pas les Français de recourir à l'esclavage dans leurs colonies avec l'approbation des autorités civiles (cf. \emph{le code noir}) : il suffira d'empêcher les esclaves (et aussi \emph{tous} les noirs, esclaves ou non) de toucher le sol de France. La survie de l'esclavage sur les terres européennes, sans interruption depuis l'Antiquité, dans l'Europe du sud, et d'abord en Espagne, a préparé les esprits de la Renaissance à recourir  à l'esclavage des indiens, puis celui des noirs d'Afrique, pour exploiter des deux Amériques et les autres colonies européennes (Surinam,~etc.). 

 Ce n'est qu'à partir de la fin du \siecle{18} qu'une part significative des intellectuels sont tombés d'accord pour condamner l'esclavage sans lui donner aucune circonstance atténuante (cf. \emph{L'Encyclopédie}).
 

% 28.02.2015 :
% haut Moyen Âge
% _, --> ,
% Antiquité
% ~etc.
% ~\%


