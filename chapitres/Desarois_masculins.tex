
\chapter{Désarrois masculins}


 Notre retour sur l'histoire montre à quel point la situation actuelle est révolutionnaire. 
Si jusqu'à ces dernières décennies le mariage alliait deux lignées en associant un homme et une femme dans le cadre d'une division sexuelle du travail indiscutée, la première de ses fonctions, ressentie comme incontournable et justifiée par la survie des individus aussi bien que celle de l'espèce, était de donner des enfants aux hommes :
\emph{La fonction principale du mariage était d'ailleurs de fabriquer du père}, (Irène \fsc{THERY}, idem). En effet ils ne pouvaient donner le jour à des semblables mais ils n'en avaient pas moins un impérieux besoin de successeurs pour s'occuper d'eux jusqu'à leur mort même quand ils ne pourraient plus subvenir à leurs propres besoins, pour poursuivre les oeuvres qu'ils avaient entreprises et pour s'occuper de leur statut post-mortem, du sort de leurs mânes.  Presque partout ils pouvaient répudier les épouses qui ne leur avaient pas donné les héritiers mâles qu'ils voulaient et ne s'en privaient pas, ou bien ils leur adjoignaient des concubines\footnote{Il a fallu bien des siècles pour que cela ne soit plus le cas en chrétienté.}. C'est pourquoi l'ancien Droit romain semblait réduire les femmes à n'être que des ventres au service des hommes\footnote{…et il en était de même en Chine et dans toute sa sphère d'influence.}. Tout était fait pour décourager les femmes de concevoir des enfants sans en passer par un homme publiquement désigné (jusqu'à l'infériorité des salaires féminins à travail égal ?). Cela impliquait aussi de donner de la valeur au fait que les enfants aient un père désigné et non un géniteur anonyme. En cas de séparation, le Droit attribuait systématiquement aux pères tous leurs enfants\footnote{... ce qu'il a fait jusqu'au milieu du \siecle{19} dans les pays anglo-saxons.}. Ils pouvaient assumer eux-mêmes leur éducation, ou les confier à leur propres parents. Ils pouvaient même les confier à leurs ex épouses, mais toujours sous leur propre autorité et à leurs frais. Les hommes tenaient les femmes en leur dépendance « À cause des enfants » dont le statut et l'installation dans l'existence dépendaient plus d'eux que d'elles, et le mariage permettait à presque tous ceux qui le désiraient (c'est-à-dire la plupart des hommes) d'avoir des enfants bien à eux et qui ne leur seraient contestés par personne et d'abord par leur mère. Le mariage leur permettait aussi de s'attacher une femme et les services de tous ordres que seule une femme pouvait alors fournir (la pertinence de la répartition des tâches selon le sexe n'a guère été contestée jusqu'à la seconde moitié du XXème siècle). 
 
 Mais la réciproque était vraie aussi : le mariage permettait aux femmes d'avoir des enfants sans être obligées de les élever seules, dans la pauvreté et l'illégitimité. Quant à celles qui y attachaient du prix, il leur permettait de s'attacher solidement tel ou tel homme, son statut social et ses ressources\footnote{... ce que symbolisaient (depuis l'antiquité romaine) les anneaux que s'échangeaient les conjoints, et ce qu'exprimaient sur le mode burlesque des expressions comme {\emph{se laisser mettre le grappin dessus}}, ou {\emph{se passer la corde au cou}}. Il est symptomatique que c'étaient les hommes qui employaient ces expressions : dans les représentations d'alors ce sont les femmes qui cherchaient le plus activement et le plus anxieusement à se marier.}.
 
Un homme ne peut plus s'y prendre aujourd'hui comme naguère. Il ne lui sert plus à rien de demander à un futur beau-père la main de sa fille, de lui demander un transfert d'autorité, puisque ce dernier ne la détient plus et ne peut donc plus la donner. D'ailleurs lui-même n'a plus besoin d'un gendre pour légitimer les petits enfants que sa fille lui donnera et pour en faire des héritiers, puisqu'il n'y a plus de fonctions interdites aux enfants illégitimes et donc plus d'enfants illégitimes. Il n'y a donc plus d'intérêt commun entre beau-père et gendre, et le soupirant doit négocier seul et sans intermédiaire avec la femme dont il recherche les faveurs. Il n'aura d'elle des enfants que si elle le veut bien. Et elle pourra d'autant plus facilement le quitter en emmenant leurs enfants communs (ou le pousser hors du domicile familial) que l'absence d'un homme à côté d'une mère ne fait plus problème, tandis que la présence de celle-ci semble encore indispensable\footnote{Cela changera peut-être si on constate des aptitudes au "maternage" chez les pères célibataires ou les couples homosexuels masculins ?}. 
 
 En ce qui concerne les hommes, les ressources dont ils disposent (puissance économique, compétences culturelles et professionnelles, pouvoir politique,~etc.) ont la vertu de les rendre désirables. Plus ils sont intellectuellement et professionnellement qualifiés, plus ils ont de probabilités d'être mariés. Les hommes se doivent encore d'être « ceux qui peuvent », ceux qui ne sont pas marqués par le manque ou la défaillance.  C'est le contraire pour les femmes, ce qui suggère que dès qu'elles ont les moyens de leur indépendance elles n'ont pas intérêt à être mariées. Cela confirme à la fois la mise en question de la répartition traditionnelle des rôles masculins et féminins et sa solidité.
 
 On a vu que « l'obligation de résultat », l'obligation de fécondité, qui pesait sur les seules femmes mariées a été supprimée par Constantin, qui a exclu la stérilité des motifs de divorce. La loi impériale romaine a ensuite confirmé l'interdit fait aux chrétiens de se remarier après divorce. À l'obligation de fécondité des épouses s'est substituée une obligation (religieuse) de moyens pour chacun des deux époux de ne pas mettre d'obstacle autre que l'abstinence 
\footnote{En France les relevés démographiques montrent l'érosion progressive du respect de cette obligation, et l'extension depuis trois siècles des pratiques anticonceptionnelles : ce que les anciens moralistes nommaient les « \emph{funestes secrets} ».} 
à une conception et à une naissance. Si les femmes mariées ont ainsi été protégées contre la répudiation et contre la privation de leurs enfants, par contre la loi ne les autorisait pas plus qu'avant à se dérober au « devoir conjugal » lorsque leur mari l'exigeait, ni aux grossesses qui en découleraient, et à leurs risques, sauf à demander une séparation. 

 Depuis 1967, même si leurs maris le désirent, les femmes mariées ne sont plus tenues par la loi de laisser libre cours à leur fécondité. Aujourd'hui le corps des femmes est à elles, y compris l'embryon ou le fœtus, qui juridiquement en fait partie depuis 1975, comme c'était le cas dans le droit romain antique. La loi ne se soucie plus de soutenir le désir masculin en ce domaine. Même si elles sont leurs épouses, même s'ils sont les géniteurs de l'enfant qu'elles portent, même si elles avaient été d'accord pour le concevoir avec eux, les hommes n'ont plus le droit d'exiger des femmes qu'elles leur donnent cet enfant. Elles peuvent choisir d'avorter ou de l'abandonner à la naissance contre le gré du père de l'enfant. On est au plus loin du droit du \emph{pater familias} romain de faire surveiller la grossesse et l'accouchement de son épouse (ou ex épouse), pour qu'elle ne puisse pas lui dérober un enfant né de ses œuvres.

 Dans le même temps ont été supprimées toutes les limites légales qui pouvaient interdire le rattachement d'un enfant naturel à son géniteur, à l'exception des inséminations artificielles avec donneur, ou IAD. Une mère qui le demande recevra toujours l'appui de la justice pour rechercher le géniteur de son enfant, quelle que soit la situation personnelle de cet homme, comme sous l'ancien régime, mais désormais cela se fera avec une efficacité  imparable. Aucun père n'est plus « \emph{incertus} ». Vivant ou mort son ADN le désignera, sauf lorsque la mère veut cacher son identité à l'enfant ou aux tiers (mais si une mère qui accouche « sous X » refuse de laisser à son enfant des renseignements sur sa propre identité, elle en a le droit). Si la mère le veut, le géniteur sera contraint d'assumer financièrement un enfant qui héritera de lui à part entière, contrairement à ce qui se passait jusqu'au \siecle{19}. Mais cela ne lui donnera pas forcément le moindre droit sur l'éducation de l'enfant : en ce sens cela n'en fera pas un père.

 Pour l'essentiel, on peut donc dire que la maîtrise de la génération est passée du côté des femmes. La famille monoparentale d'aujourd'hui, c'est le plus souvent la famille \emph{moins} le père: dans la grande majorité des séparations (85~\%) ce sont les mères qui gardent les enfants. Est-ce pour ces raisons que l'initiative des divorces vient des épouses beaucoup plus souvent (trois fois sur quatre) que des maris  ? 


 
 
 



 


