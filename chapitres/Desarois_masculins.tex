
\chapter{Désarrois masculins}


 Notre retour sur l'histoire montre à quel point la situation actuelle est révolutionnaire. 
Si jusqu'à ces dernières décennies le mariage alliait deux lignées en associant un homme et une femme dans le cadre d'une division sexuelle du travail indiscutée, la première de ses fonctions, ressentie comme incontournable et justifiée par la survie des individus aussi bien que celle de l'espèce, était de donner des enfants aux hommes :
\emph{La fonction principale du mariage était d'ailleurs de fabriquer du père}, (Irène \fsc{THERY}, idem). En effet ils ne pouvaient donner le jour à des semblables mais ils n'en avaient pas moins un impérieux besoin de successeurs pour s'occuper d'eux jusqu'à leur mort même quand ils ne pourraient plus subvenir à leurs propres besoins, pour poursuivre les oeuvres qu'ils avaient entreprises et pour s'occuper de leur statut post-mortem, du sort de leurs mânes.  Presque partout ils pouvaient répudier les épouses qui ne leur avaient pas donné les héritiers mâles qu'ils voulaient et ne s'en privaient pas, ou bien ils leur adjoignaient des concubines\footnote{Il a fallu bien des siècles pour que cela ne soit plus le cas en chrétienté.}. C'est pourquoi l'ancien Droit romain semblait réduire les femmes à n'être que des ventres au service des hommes\footnote{…et il en était de même en Chine et dans toute sa sphère d'influence.}. Tout était fait pour décourager les femmes de concevoir des enfants sans en passer par un homme publiquement désigné (jusqu'à l'infériorité des salaires féminins à travail égal ?). Cela impliquait aussi de donner de la valeur au fait que les enfants aient un père désigné et non un géniteur anonyme. En cas de séparation, le Droit attribuait systématiquement aux pères tous leurs enfants\footnote{... ce qu'il a fait jusqu'au milieu du \siecle{19} dans les pays anglo-saxons.}. Ils pouvaient assumer eux-mêmes leur éducation, ou les confier à leur propres parents. Ils pouvaient même les confier à leurs ex épouses, mais toujours sous leur propre autorité et à leurs frais. Les hommes tenaient les femmes en leur dépendance « À cause des enfants » dont le statut et l'installation dans l'existence dépendaient plus d'eux que d'elles, et le mariage permettait à presque tous ceux qui le désiraient (c'est-à-dire la plupart des hommes) d'avoir des enfants bien à eux et qui ne leur seraient contestés par personne et d'abord par leur mère. Le mariage leur permettait aussi de s'attacher une femme et les services de tous ordres que seule une femme pouvait alors fournir (la pertinence de la répartition des tâches selon le sexe n'a guère été contestée jusqu'à la seconde moitié du XXème siècle). 
 
 Mais la réciproque était vraie aussi : le mariage permettait aux femmes d'avoir des enfants sans être obligées de les élever seules, dans la pauvreté et l'illégitimité. Quant à celles qui y attachaient du prix, il leur permettait de s'attacher solidement tel ou tel homme
\footnote{... ce que symbolisaient depuis l'antiquité les anneaux que s'échangeaient les conjoints, et ce qu'exprimait sur le mode burlesque des expressions comme {\emph{se laisser mettre le grappin dessus}}, ou {\emph{se passer la corde au cou}}. Il est symptomatique que c'étaient les hommes qui employaient ces expressions : dans les représentations d'alors ce sont les femmes qui cherchaient le plus activement et anxieusement à se marier, et le jour de leur mariage était en quelque sorte celui de leur triomphe.}.
 
Un homme ne peut plus s'y prendre aujourd'hui comme naguère. Il ne lui sert plus à rien de demander à un futur beau-père la main de sa fille, de lui demander un transfert d'autorité, puisque ce dernier ne la détient plus et ne peut donc plus la donner. D'ailleurs lui-même n'a plus besoin d'un gendre pour légitimer les petits enfants que sa fille lui donnera et pour en faire des héritiers, puisqu'il n'y a plus de fonctions interdites aux enfants illégitimes et donc plus d'enfants illégitimes. Il n'y a donc plus d'intérêt commun entre beau-père et gendre, et le soupirant doit négocier seul et sans intermédiaire avec la femme dont il recherche les faveurs. Il n'aura d'elle des enfants que si elle le veut bien. Et elle pourra d'autant plus facilement le quitter en emmenant leurs enfants communs (ou le pousser hors du domicile familial) que l'absence d'un homme à côté d'une mère ne fait plus problème, tandis que la présence de celle-ci semble encore indispensable\footnote{Cela changera peut-être si on constate des aptitudes au "maternage" chez les pères célibataires ou les couples homosexuels masculins ?}. 
 
 En ce qui concerne les hommes, les ressources dont ils disposent (puissance économique, compétences culturelles et professionnelles, pouvoir politique,~etc.) ont la vertu de les rendre désirables. Plus ils sont intellectuellement et professionnellement qualifiés, plus ils ont de probabilités d'être mariés. Les hommes se doivent encore d'être « ceux qui peuvent », ceux qui ne sont pas marqués par le manque ou la défaillance.  C'est le contraire pour les femmes, ce qui suggère que dès qu'elles ont les moyens de leur indépendance elles n'ont pas intérêt à être mariées. Cela confirme à la fois la mise en question de la répartition traditionnelle des rôles masculins et féminins et sa solidité.
 
 On a vu que « l'obligation de résultat », l'obligation de fécondité, qui pesait sur les seules femmes mariées a été supprimée par Constantin, qui a exclu la stérilité des motifs de divorce. La loi impériale romaine a ensuite confirmé l'interdit fait aux chrétiens de se remarier après divorce. À l'obligation de fécondité des épouses s'est substituée une obligation (religieuse) de moyens pour chacun des deux époux de ne pas mettre d'obstacle autre que l'abstinence 
\footnote{En France les relevés démographiques montrent l'érosion progressive du respect de cette obligation, et l'extension depuis trois siècles des pratiques anticonceptionnelles : ce que les anciens moralistes nommaient les « \emph{funestes secrets} ».} 
à une conception et à une naissance. Si les femmes mariées ont ainsi été protégées contre la répudiation et contre la privation de leurs enfants, par contre la loi ne les autorisait pas plus qu'avant à se dérober au « devoir conjugal » lorsque leur mari l'exigeait, ni aux grossesses qui en découleraient, et à leurs risques, sauf à demander une séparation. 

 Depuis 1967, même si leurs maris le désirent, les femmes mariées ne sont plus tenues par la loi de laisser libre cours à leur fécondité. Aujourd'hui le corps des femmes est à elles, y compris l'embryon ou le fœtus, qui juridiquement en fait partie depuis 1975, comme c'était le cas dans le droit romain antique. La loi ne se soucie plus de soutenir le désir masculin en ce domaine. Même si elles sont leurs épouses, même s'ils sont les géniteurs de l'enfant qu'elles portent, même si elles avaient été d'accord pour le concevoir avec eux, les hommes n'ont plus le droit d'exiger des femmes qu'elles leur donnent cet enfant. Elles peuvent choisir d'avorter ou de l'abandonner à la naissance contre le gré du père de l'enfant. On est au plus loin du droit du \emph{pater familias} romain de faire surveiller la grossesse et l'accouchement de son épouse (ou ex épouse), pour qu'elle ne puisse pas lui dérober un enfant né de ses œuvres.

 Dans le même temps ont été supprimées toutes les limites légales qui pouvaient interdire le rattachement d'un enfant naturel à son géniteur, à l'exception des inséminations artificielles avec donneur, ou IAD. Une mère qui le demande recevra toujours l'appui de la justice pour rechercher le géniteur de son enfant, quelle que soit la situation personnelle de cet homme, comme sous l'ancien régime, mais désormais cela se fera avec une efficacité  imparable. Aucun père n'est plus « \emph{incertus} ». Vivant ou mort son ADN le désignera, sauf lorsque la mère veut cacher son identité à l'enfant ou aux tiers (mais si une mère qui accouche « sous X » refuse de laisser à son enfant des renseignements sur sa propre identité, elle en a le droit). Si la mère le veut, le géniteur sera contraint d'assumer financièrement un enfant qui héritera de lui à part entière, contrairement à ce qui se passait jusqu'au \siecle{19}. Mais cela ne lui donnera pas forcément le moindre droit sur l'éducation de l'enfant : en ce sens cela n'en fera pas un père.

 Pour l'essentiel, on peut donc dire que la maîtrise de la génération est passée du côté des femmes. La famille monoparentale d'aujourd'hui, c'est assez ordinairement la famille \emph{moins} le père. Dans la majorité des séparations (85~\%) ce sont les mères qui gardent les enfants. Est-ce pour ces raisons que l'initiative des divorces vient des épouses beaucoup plus souvent (trois fois sur quatre) que des maris  ? Toujours est-il que beaucoup de ceux-ci ont plus à perdre que celles-là au divorce, et surtout les plus pauvres : ils y perdent d'autant plus que leurs ressources financières sont limitées et qu'il existe une grande dissymétrie dans l'investissement des hommes et des femmes dans tout ce qui concerne le "care" (autrement dit  ils n'ont pas plus appris que les autres hommes à s'occuper d'eux-mêmes et ils n'ont pas les moyens de se procurer les soins dont ils ont besoin autrement que dans le cadre du mariage traditionnel). 

 Les mères et la fonction maternelle ont toujours été valorisées dans les représentations communes : elles sont traditionnellement du côté de l'accueil de la vie et de son entretien, de l'intime, de la tendresse, du cœur (du care). Mais aujourd'hui cette idéalisation n'est plus contrebalancée par celle qui entourait les pères et la fonction paternelle des siècles classiques. Aujourd'hui la déploration des déficiences des pères, de leurs fragilités et de leur immaturité est un passage obligé de tout discours sur la famille, tandis que l'idée qu'ils puissent mettre en oeuvre leur force ou leur puissance dans une relation à des enfants suscite quasi-mécaniquement des représentations de violence et de maltraitance. Quand on parle sans les spécifier des violences conjugales, il va de soi qu'il s'agit des violences masculines, alors que l'observation sociologique montre que les femmes sont très capables de concurrencer les hommes de manière significative dans ce domaine aussi et qu'elles n'ont jamais été sans défenses. D'ailleurs maintenant que le capital le plus utile c'est le capital intellectuel, maintenant que l'avenir des enfants se prépare à coup d'études longues, financées en grande partie par la collectivité, sous la houlette de professionnels de l'enseignement et sous le contrôle de l'État, qu'est-ce qu'un père pourrait bien transmettre à ses enfants, à part ses biens, sans menacer leur autonomie ?

 Dans l'effritement de l'autorité des pères, Françoise \fsc{HURSTEL} pointe trois moments clé : la loi de 1889 contre les « \emph{parents indignes} », la loi de 1935 abolissant le droit de « \emph{correction paternelle} » et la loi de 1938 abolissant la « \emph{puissance maritale} ». Ont été abolies toutes les dispositions juridiques sur lesquelles était fondé dans le passé l'exercice masculin d'un rôle patriarcal. Le résultat est que « [...] \emph{nous ne savons plus ce qu'est la place d'un père et ce que sont ses fonctions} », et que « \emph{ce ne sont pas des petits bouts de la paternité qui ont changé, mais l'ensemble du système a muté avec la mort du \emph{pater familias}.} »%
% [3]
\footnote{Françoise \fsc{HURSTEL}, « Penser la paternité contemporaine dans le monde occidental : quelles places et quelles fonctions du père pour le devenir humain, sujet et citoyen des enfants ? », in \emph{Neuropsychiatrie de l'enfance et de l'adolescence}, 53 (2005) 224-230.} 

 Autrefois (jusqu'aux années 60 du siècle dernier ?) c'est l'excès de présence et de poids des pères qui faisait problème. Aujourd'hui on déplore qu'ils ne soient jamais assez présents, ou jamais là où il faut. Françoise \fsc{HURSTEL} soutient que ces discours sont l'effet de ces changements, et non leur cause. Si les lois suivaient l'évolution des mœurs, alors la promulgation d'une loi serait le signe que les esprits sont prêts à l'accueillir. Dans cette hypothèse, pendant les années précédant la promulgation de chacune des lois ci-dessus, on aurait dû observer un mouvement de l'opinion publique stigmatisant les parents indignes, le recours abusif au droit de correction paternelle, ou le scandale que constitue l'existence d'une puissance maritale. Selon elle ce n'est pas ainsi que cela s'est passé, au contraire. Ce n'est qu'à partir de la promulgation de la loi de 1889 que la presse aurait commencé de dénoncer les carences des pères « indignes%
% [4]
\footnote{« \emph{alcoolique, pauvre, inculte et violent} », Françoise \fsc{HURSTEL}, \emph{la déchirure paternelle}, p. 113.} 
 ».

 Et de même ce n'est que vers 1942 que les spécialistes de l'éducation auraient commencé de dénoncer les pères sans autorité, tandis que la notion de carence n'aurait envahi les écrits qu'à partir de 1950 :
 
\begin{displayquote}
\emph{C'est donc quelques années après la promulgation de ces lois faisant disparaître des textes juridiques les termes de puissance (maritale) et ceux de correction paternelle tout en maintenant ceux de chef et d'autorité (paternelle), qu'est décrite cette figure d'un père manquant d'autorité et de sévérité ; et que les spécialistes admonestent les pères d'une position qui est bien celle de chef de famille.}
\end{displayquote}

 Selon elle, l'opinion publique n'aurait donc appelé aucune de ces lois de ses vœux. Ces réformes n'auraient été imaginées, réclamées, et parfois discrètement expérimentées que par les seuls experts, médecins, administrateurs, juges et travailleurs sociaux directement intéressés à leur mise en œuvre. Pour Françoise \fsc{HURSTEL}, tous les discours sur les déficiences des pères actuels ne sont que des productions imaginaires qui coexistent avec des réalités qui n'ont pas grand-chose à voir avec eux. En effet, les enquêtes sur le terrain ne montrent rien qui permette de croire que les pères d'aujourd'hui seraient dans l'ensemble moins attentifs et moins présents que ne l'étaient ceux du passé%
% [5]
\footnote{... mais cela exige d'éviter les biais méthodologiques. Il faut notamment que ces enquêtes ne se placent pas consciemment ou inconsciemment du seul point de vue des mères. Cf. Germain \fsc{DULAC}, « La configuration du champ de la paternité : politiques, acteurs et enjeux », in \emph{Politiques du père, numéro spécial de Lien social et politiques}, (n° 37) 1997, p. 133-142.}%
. Certes il y a des pères qui sont incompétents, irresponsables ou délinquants, mais cela n'a rien de nouveau, et rien ne permet d'affirmer qu'il y en ait plus qu'autrefois. Les discours ne portent pas tant sur ce que font réellement les pères que sur ce qu'ils devraient faire dans l'idéal pour être de bons pères. 

Pour elle, il s'agit, à l'aide de ces discours, d'asseoir l'autorité de ceux qui prétendent savoir ce qu'est un bon père et qui sont les bons pères :
 
\begin{displayquote} 
\emph{Du point de vue de la paternité les hommes de la période contemporaine n'auront pas été gâtés. Je propose une image pour illustrer ce que peut être la notion de carence : lorsqu'un homme devient père, il endosse un pardessus plein de trous et de soupçons..., plus précisément une image de plus en plus dévalorisée, et cela quelle que soit la valeur personnelle de l'homme qui assume une telle fonction. Et ce qui les caractérise est un discours dévalorisant des spécialistes ; tellement dévalorisant qu'il apparaît, en fait, comme un discours de l'exclusion des pères... au profit du super père spécialiste. Si les pères peuvent être dits carents \emph{[en Droit, le père « carent » est celui qui ne laisse rien à ses enfants, qui ne leur laisse aucun héritage]}, c'est parce qu'ils sont relégués à cette place par ceux-là mêmes qui normalisent les pratiques autour de l'enfant. Nous dirons que ces pères carents sont en fait d'abord des pères exclus par les théoriciens de l'éducation.}%
% [6]
\footnote{Françoise \fsc{HURSTEL}, \emph{la déchirure paternelle}, p. 112-113.} 

[...] \emph{Ainsi les signifiants inscrits dans la loi produisent des effets imaginaires qui se repèrent dans les représentations collectives, les modèles normatifs du père et les pratiques sociales.} 

\emph{Je ferai ici un pas de plus et avancerai ceci : non seulement les signifiants des lois produisent des effets imaginaires, mais encore les lois elles-mêmes ne sont connues que par le biais de ces productions...}

\emph{Les figures du père carent semblent bien avoir une fonction sociale et idéologique importante, celle d'être l'une de ces fonctions sociales qui rendent compte et qu'il y a du père dans notre société (au sens du père symbolique et de la fonction paternelle) et qu'il y a du changement dans les montages qui instituent le père... bref, elles seraient un mode d'historicisation d'une structure.}

\emph{Mais en retour cet imaginaire du père marquera chaque homme ayant à assumer la fonction paternelle, chaque mère appelée à reconnaître qu'il y a du père pour son enfant.}%
% [7]
\footnote{Idem, p. 113-115.}
\end{displayquote}

 Il est ordinaire et au fond assez normal que les garçons et les filles traversent à l'adolescence des états d'incertitude identitaire, avec tous les malaises que cela implique, étant donnés toutes les métamorphoses par où ils doivent passer. Mais encore plus déstabilisantes pour eux sont les incertitudes identitaires de leurs adultes de référence. Si même les adultes ne savent plus ce qu'est un homme ou une femme, à quel modèle peuvent-ils se mesurer ? C'est pourquoi ce n'est pas par hasard si aujourd'hui ce sont les garçons et les jeunes hommes qui expriment le plus bruyament leur désarroi : violences contre eux-mêmes, contre les personnes et contre les biens, prises de risques inconsidérées, désinvestissement scolaire, etc. En effet les discours qu'ils peuvent entendre sur ce que c'est qu'un homme sont particulièrement insupportables pour eux: 
 
 \emph{Viols et violences, mépris et humiliation des femmes et des hommes dévalorisés qui leur sont assimilés, cynisme, manque de pensée et appauvrissement affectif : la représentation des hommes qui exsude d'une lecture attentive des recherches qui leur sont consacrées est suffocante. Quels que soient les champs disciplinaires et les orientations théoriques, la virilité désigne l'expression collective et individuelle de la domination masculine et ne saurait donc constituer une définition positive du masculin\footnote{Molinier Pascale \emph{Virilité défensive, masculinité créatrice}, in  \emph{Travail, genre et société}, n° 3, mars 2000.}.}



 


