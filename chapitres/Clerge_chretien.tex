% Le 02.03.2015 :
% ~\%
% ~etc.
% Antiquité
% Moyen Âge


\section{Le clergé chrétien}


 Les règles de recrutement et de discipline cléricale de l'Église se sont précisées au cours des premiers siècles%
% [1]
\footnote{Sources : Georges \fsc{MINOIS}, \emph{Les religieux en Bretagne sous l'Ancien Régime}, 1989. Léo \fsc{MOULIN}, \emph{La vie quotidienne des religieux au Moyen Âge, \siecles{10}{15}}, 1978. Michel \fsc{PARISSE}, \emph{Les nonnes au Moyen Âge}, 1983.}% 
. Dès le \siecle{4} ces règles reflètent l'état définitif de la doctrine permanente, qu'on retrouvera telle quelle et quasi inchangée dans sa formulation du début du \siecle{20} (droit canon de 1917). Selon les Décrétales du Pape Innocent I (401-417) il est interdit d'admettre au diaconat et à la prêtrise :
\begin{enumerate}
%  1°)
\item ceux qui ont épousé une femme non vierge ;
% 2°)
\item ceux qui ont épousé une veuve ;
% 3°)
\item ceux qui ont été mariés deux fois, quelles que soient les circonstances ;
% 4°)
\item ceux qui se sont fait soldats après leur baptême, qui ont accepté de toucher des armes dont certaines ont versé le sang, et surtout ceux qui ont accepté de verser le sang. Même si l'interdiction faite aux chrétiens d'être militaires a été levée par l'Église à partir du ralliement de Constantin (entre 313 et 315), le sang restait sacré, donc impur, et impur aussi celui qui le versait, même pour la bonne cause (chirurgiens) ;
% 5°)
\item ceux qui, magistrats, ont jugé ou plaidé dans des procès où ils ont requis ou prononcé la peine de mort (même motif que le cas précédent : le juge est condamné à faire couler le sang : infliger la \emph{question}, c'est-a-dire torturer, était alors considéré comme nécessaire pour découvrir la vérité, et donc inévitable; infliger des peines mineures comme le fouet; condamner à mort...). Ce n'était pas le risque de l'erreur judiciaire qui était en jeu, c'était encore une fois le contact avec le sang et le contact avec la mort ;
% 6°)
\item les pécheurs qui ont été condamnés à une pénitence (« pécheurs publics », nouveaux infâmes) ;
% 7°)
\item ceux qui ont donné des jeux publics \latin{(munera)}, où du sang (humain ou animal) a coulé ;
% 8°)
\item ceux qui ont exercé des sacerdoces païens, et qui ont donc eux-mêmes sacrifié aux dieux ;
% 9°)
\item ceux qui se sont mutilés eux-mêmes, ce qui vise surtout l'auto castration. Ces derniers ont à la fois versé leur propre sang et mutilé leur corps à l'instar des \latin{galles} (prêtres de Cybèle).
\end{enumerate}

 On peut comparer trait pour trait ces règles avec celles du Lévitique qui régissaient les lévites et les prêtres du Temple de Jérusalem. C'est la même logique. Dans les discussions sur ces sujets les textes de la Tora ont servi d'arguments décisifs. En effet l'imitation du clergé du Temple s'est faite au fil du temps de plus en plus consciente et volontaire. Et pourtant plus il se voulait identique au clergé du Temple, moins le clergé chrétien lui ressemblait ! L'exigence du célibat lui imprimait en effet une physionomie tout à fait inédite. 

 On a vu que la continence perpétuelle était exigée des diacres et prêtres dès les premiers siècles afin qu'ils soient toujours prêts à toucher les « choses sacrées » (vases et linges sacrés, offrandes, pain consacré,~etc.), non souillés par l'impureté rituelle produite par le coït. Ce qui est remarquable c'est que cet argumentaire a emporté l'adhésion. Pourtant l'organisation du service du Temple de Jérusalem montrait une voie de compromis évidente, le service par roulement. D'autre part la notion même de pureté et de souillure religieuse, qui ne se confond pas avec celle de faute morale \emph{(péché)}, avait été mise en question par le Christ lui-même. On peut en déduire que le refus du service par roulement était motivé par des raisons autrement impérieuses que la difficulté de mettre en place un tour de service. 

 Dès l'élection du remplaçant de l'apôtre Judas et l'institution des diacres, les apôtres avaient estimé que personne ne se donne à soi-même une mission%
% [2]
\footnote{Cf. selon le livre des \emph{Actes des Apôtres} les difficultés de Paul de Tarse pour faire admettre par les apôtres sa mission auto proclamée auprès des gentils et ses prétentions au titre d'apôtre.} 
ni ne la tient de sa naissance%
%[3]
\footnote{... de même que nul ne peut (en stricte doctrine) se dire chrétien par sa naissance : il faut que chaque enfant en passe par le baptême, comme le premier converti venu.}% 
, que c'est l'Église qui appelle, et Dieu à travers elle. C'est pourquoi la succession dans le même poste ecclésiastique du père au fils, de l'oncle au neveu, sans être interdite n'a jamais été reconnue comme un droit, au contraire du droit à hériter d'un « honneur », d'une terre ou d'une entreprise, et encore moins comme un modèle. Passés les premiers siècles elle a au contraire été vue comme une irrégularité grosse de dangers. 

 Si la haute administration de l'Empire romain tardif et des royaumes barbares qui lui ont succédé est devenue vers le \siecle{10} la noblesse héréditaire du Moyen Âge, c'est parce que ceux qui étaient nommés par les autorités civiles à un emploi public ont fini par obtenir le droit de désigner eux-mêmes leur successeur, ce qui signifie que « l'honneur » (responsabilités et biens servant à les rémunérer) qui leur avait été conféré par les souverains est entré dans leur patrimoine, à la faveur de l'affaiblissement de ces mêmes souverains, système d'où est sortie la \emph{féodalité}. Au même moment un clergé marié aurait eu les mêmes chances de devenir héréditaire, et le risque eut été grand de voir se constituer une caste sacerdotale à côté de la caste aristocratique, à la mode indienne ou hébraïque. On connaît d'ailleurs un certain nombre de grandes familles de l'Antiquité et du haut Moyen Âge dont des membres se sont succédé sur le même siège épiscopal pendant plusieurs générations : Sylvère, pape de 536 à 537, était le fils légitime d'Horsmidas, pape de 514 à 523 (né avant son ordination). D'autre part les souverains et autres puissants du \siecle{6} et des siècles suivants ont utilisé leur influence pour conférer l'épiscopat à des serviteurs laïcs afin de les récompenser pour leurs loyaux services, ou bien pour les neutraliser par cet « honneur » particulier, qui leur interdisait (en principe) tout retour aux armes. 

 Si les membres d'une échelle hiérarchique peuvent donner leur poste à un héritier c'est qu'ils en sont devenus propriétaires et c'est toujours au détriment du sommet de la hiérarchie, désormais obligé de composer avec une autre source de légitimité qu'elle-même. Inversement, c'est toujours pour défendre ou renforcer son autorité qu'un souverain refuse que soit limité son pouvoir de nommer et de démettre.

 Au contraire, du point de vue d'une institution, le célibat est idéal :
\begin{enumerate}
% 1°)
\item Un clerc célibataire est plus disponible puisqu'il n'a pas à plaire à sa femme, ni à s'occuper de ses enfants (cf. Paul de Tarse).
% 2°)
\item Un clerc célibataire et sans enfants a moins de besoins matériels qu'un clerc marié et donc il \emph{peut} coûter moins cher. La continence des clercs est d'abord économique en ce qu'il n'y a pas à constituer de dot pour les filles ni à établir les garçons ...
% 3°)
\item ... qui pourraient prétendre avoir des droits sur le poste de leur père.
% 4°)
\item N'ayant pas à craindre pour ses proches, ni à les établir dans la vie, un clerc sans attaches familiales est moins sensible aux pressions et séductions venant de la société civile.
% 5°)
\item Par ailleurs il serait inconvenant que des histoires de famille puissent interférer dans les affaires de l'Église.
% 6°)
\item Enfin une paroisse, un diocèse, un monastère ne sont pas des bâtiments ni des biens fonciers. Ces institutions sont des ensembles de fidèles, et pour l'Église aucun groupe de fidèles, c'est-à-dire d'âmes immortelles, ne peut appartenir à une personne, ou à une famille, à la façon dont la force de travail des serfs (mais non leurs âmes) appartenait à leur seigneur.
\end{enumerate} 

 Voilà pourquoi la doctrine ecclésiale a toujours voulu, malgré toutes les pesanteurs individuelles et collectives qui ont entrainé de nombreux écarts, que les clercs ne soient pas issus de familles de clercs, mais issus du monde des laïcs. Et voilà pour eux autant de raisons très concrètes d'attribuer une valeur spirituelle au célibat et à la continence perpétuelle, tandis que les religieux montraient par leur exemple que cet idéal n'était pas inatteignable. 

 Cela a eu des conséquences très importantes sur la société toute entière. En effet il s'est constitué en son sein une caste non héréditaire recrutée dans les autres castes, cultivant le savoir et la culture, et au sein de laquelle les carrières n'étaient pas déterminées par la naissance, même si ces deux idéaux, toujours poursuivis, n'ont jamais été totalement atteints. La culture cultivée dans les institutions d'Église souffrait de limitations certaines et l'héritage antique n'avait pas été transmis sans pertes. Tous les clercs n'étaient pas savants, et les plus compétents n'obtenaient pas toujours les promotions auxquelles leurs talents les auraient qualifiés. De même tous les princes de l'Église n'étaient pas à la hauteur de leur charge. Mais c'est au sein du corps des moines et des prêtres que se trouvaient les plus savants de leur époque, et pour ceux qui n'avaient pas les privilèges de la naissance c'est au sein de l'Église qu'ils avaient le plus de chances de promotions. 
 
 Joseph \fsc{MORSEL} voit dans cet élitisme ecclésiastique et le modèle qu'elle a fourni, profondément intériorisé, l'une des causes principales du développement ultérieur de l'Europe et de son avance sur les autres civilisations (in \emph{L'Histoire du Moyen Âge est un sport de combat}, texte publié au format pdf sur Internet à l'adresse \url{http://lamop.univ-paris1.fr/IMG/pdf/SportdecombatMac.pdf}).
 
 \section{Les religieux}
 Le mouvement monastique s'est développé depuis les premiers ermites qui ont fui le monde dès le \siecle{3} dans les déserts d'Égypte, et les premières veuves et vierges consacrées qui en ont fait autant à l'ombre des cathédrales, sous la protection des évêques. Il continuera de se développer à un rythme soutenu jusqu'au foisonnement de la fin du Moyen Âge. Il prouvait par sa floraison que la continence \emph{perpétuelle} était possible%
% [2]
\footnote{... même si elle doit parfois s'appuyer sur \emph{l'impuissance de famine}, cf. Aline \fsc{Rousselle}, 1998, p. 203 - 224}% 
, et cela non seulement pour les femmes, de qui depuis toujours on l'exigeait au gré des besoins de leur famille, mais aussi pour les hommes. Saint Augustin, évêque de la fin du \siecle{4} et du début du \siecle{5}, vivait en communauté avec ses collaborateurs immédiats, communauté d'où sortiront un jour les chapitres de chanoines présents dans toutes les cathédrales. Au même moment les hôpitaux s'organisaient dans l'esprit des monastères. Ils étaient construits comme des églises dans lesquelles seraient logés des malades et si l'on en croit le concile de Nicée leur personnel était recruté parmi les religieux.

 S'appuyant sur les lettres de Paul de Tarse et les paroles du Christ, l'Église défendait le droit des jeunes de consacrer volontairement et librement leur vie à Dieu, alors qu'ils étaient encore \emph{dans la main} de leur père. Dans ce cas elle défendait leur droit de recevoir leur part d'héritage sans pour autant suivre la voie prévue par leurs parents, part d'héritage sans laquelle leur liberté de choix serait restée formelle. Cela leur permettait de s'engager dans le monastère ou l'hôpital de leur choix en faisant don à leur communauté de leur part d'héritage%
% [3]
\footnote{C'est ainsi qu'était mis en pratique la proposition de donner tous leurs biens aux pauvres faite par le Christ à ceux qui voulaient choisir la perfection (parabole du « jeune homme riche ») : en effet leur nouvelle famille spirituelle n'était constituée que de membres qui avaient fait vœu de pauvreté.}% 
. On peut supposer que ce n'est pas par hasard qu'en 320 Constantin avait abrogé les lois d'Auguste qui exigeaient d'avoir engendré trois enfants et d'être marié pour recevoir les héritages venant de personnes éloignées, et qu'il avait posé des limites au droit des pères de déshériter un enfant. Contrainte par sa propre logique, et fidèle sur ce point au droit romain, l'Église plaidait pour le consentement mutuel des fiancés et contre l'idée que celui de leurs parents était nécessaire pour que leur mariage soit valide%
%[4]
\footnote{Là aussi elle allait contre l'autorité des pères. Cet enseignement-là restait en travers de la gorge de bien des pères, mais aussi des ecclésiastiques eux-mêmes pour autant qu'ils s'identifiaient aux intérêts temporels de leur famille d'origine, cf. les avanies subies par Abélard, alors qu'il était encore laïc et donc épousable, du fait de l'ecclésiastique qui était oncle et tuteur d'Héloïse.}% 
.

 Les revenus des monastères, des évêchés et des hôpitaux étaient fondés sur des propriétés, terres, domaines,~etc., provenant des dons et des legs. Grâce aux rentes sur la terre%
% [5]
\footnote{Ressentie de l'Antiquité à la fin du Moyen Âge (au moins) comme le seul bien qui ne fait jamais défaut, et dont les fruits permettent de survivre quelle que soit la catastrophe économique qui puisse arriver (Paul \fsc{Veyne}, \emph{La société romaine}, chapitre).} 
et les immeubles (en nature ou en argent) il était possible sans recourir à l'impôt de « fonder » (en principe une fois pour toutes) des emplois \emph{(bénéfices)} de clercs, des écoles, des hôpitaux, des monastères,~etc. Ce mode de financement était hérité de l'Antiquité pré chrétienne. C'était déjà celui des temples païens. S'ajoutaient à ces revenus des contributions régulières notamment les différentes \emph{dîmes} versées par les fidèles, d'abord volontaires, puis obligatoires. Ainsi les institutions ecclésiastiques étaient autonomes et auto-suffisantes, sans courir les risques du marché, ni dépendre étroitement de généreux donateurs ou des pouvoirs locaux. Ce système ne faisait peser aucune charge récurrente sur le budget de la puissance publique et donnait aux institutions un maximum de liberté face aux pressions des pouvoirs publics. 

 Jusqu'à la fin du Moyen Âge une part de presque tous les héritages était donnée aux pauvres (c'est-à-dire à leur protectrice officielle : l'Église) pour \emph{le salut de l'âme} des donateurs. Il existait déjà chez les anciens des fondations identiques auprès des temples païens. Quant aux barbares ils admettaient comme les Égyptiens, les Celtes et les Germains que chaque mort emporte dans son tombeau des biens pour l'au-delà, ce qui du point de vue des chrétiens ou des juifs était un signe de superstition. Cette part des biens du mourant qu'il comptait emporter avec lui (jusqu'à un tiers de sa fortune ?) l'Église lui proposait d'en faire meilleur usage, en l'investissant dans les \emph{œuvres pies} (pieuses). 

 Selon Raymond Goody il y avait un lien entre la défense par l'Église de la liberté de choix de vie des jeunes, celle du droit des jeunes à une part d'héritage même en cas de désaccord paternel, celle des chrétiens à faire des donations (notamment dans leur testament) et le financement des institutions religieuses qui fournissaient les lieux où chercher la perfection. Selon lui la nécessité de trouver des ressources pour faire vivre les paroisses, monastères et hôpitaux a exercé une pression déterminante sur la définition même des règles du droit de la famille. Elle aurait contribué à ce que le droit de l'Église mette des limites au droit des pères à imposer leur volonté à leurs enfants. Elle aurait aussi et surtout contribué à étendre les degrés de parenté interdisant les mariages. Même si cette thèse paraît un peu extrême, comme toute thèse qui attribue à une cause unique un mouvement observable sur plus de dix siècles, elle n'en contient pas moins une part de vérité significative. 

 En dehors du travail de leurs membres, qui exigeait lui-même un minimum d'outils de production et d'abord de terres, le financement des monastères reposait sur les \emph{dots} des postulants, notamment dans les monastères féminins qui ne pouvaient bénéficier comme les monastères d'hommes des honoraires de messes offertes pour le repos de l'âme des défunts. Au décès du religieux sa dot demeurait acquise au monastère (du moins tant qu'elle a consisté en un capital et non en une rente). Celui-ci avait donc des chances de voir grossir peu à peu son capital. Cela permettait (dans les meilleurs des cas) d'accepter les postulants sans le sou et de consacrer le superflu au service des pauvres et des malades.
 