% Le 02.03.2015 :
% ~\%
% ~etc.
% Antiquité
% Moyen Âge


\chapter{Clercs et religieux}


 Les règles de recrutement et de discipline cléricale de l'Église se sont précisées au cours des premiers siècles%
% [1]
\footnote{Sources : Georges \fsc{MINOIS}, \emph{Les religieux en Bretagne sous l'Ancien Régime}, 1989. Léo \fsc{MOULIN}, \emph{La vie quotidienne des religieux au Moyen Âge, \siecles{10}{15}}, 1978. Michel \fsc{PARISSE}, \emph{Les nonnes au Moyen Âge}, 1983.}% 
. Dès le \siecle{4} ces règles reflètent l'état définitif de la doctrine permanente, qu'on retrouvera telle quelle et quasi inchangée dans sa formulation du début du \siecle{20} (droit canon de 1917). Selon les Décrétales du Pape Innocent I (401-417) il est interdit d'admettre au diaconat et à la prêtrise :
\begin{enumerate}
%  1°)
\item ceux qui ont épousé une femme non vierge ;
% 2°)
\item ceux qui ont épousé une veuve ;
% 3°)
\item ceux qui ont été mariés deux fois, quelles que soient les circonstances ;
% 4°)
\item ceux qui se sont fait soldats après leur baptême, qui ont accepté de toucher des armes dont certaines ont versé le sang, et surtout ceux qui ont accepté de verser le sang. Même si l'interdiction faite aux chrétiens d'être militaires a été levée par l'Église à partir du ralliement de Constantin (entre 313 et 315), le sang restait sacré, donc impur, et impur aussi celui qui le versait, même pour la bonne cause (chirurgiens) ;
% 5°)
\item ceux qui, magistrats, ont jugé ou plaidé dans des procès où ils ont requis ou prononcé la peine de mort (même motif que le cas précédent : le juge est condamné à faire couler le sang : infliger la \emph{question}, c'est-a-dire torturer, était alors considéré comme nécessaire pour découvrir la vérité, et donc inévitable; infliger frd peines mineures comme le fouet; condamner à mort...). Ce n'était pas le risque de l'erreur judiciaire qui était en jeu, c'était encore une fois le contact avec le sang et le contact avec la mort ;
% 6°)
\item les pécheurs qui ont été condamnés à une pénitence (« pécheurs publics », nouveaux infâmes) ;
% 7°)
\item ceux qui ont donné des jeux publics \emph{(munera)}, où du sang (humain ou animal) a coulé ;
% 8°)
\item ceux qui ont exercé des sacerdoces païens, et qui ont donc eux-mêmes sacrifié aux dieux ;
% 9°)
\item ceux qui se sont mutilés eux-mêmes, ce qui vise surtout l'auto castration. Ces derniers ont à la fois versé leur propre sang et mutilé leur corps à l'instar des \emph{galles} (prêtres de Cybèle).
\end{enumerate}

 On peut comparer trait pour trait ces règles avec celles du Lévitique qui régissaient les lévites et les prêtres du Temple de Jérusalem. C'est la même logique. Dans les discussions sur ces sujets les textes de la Tora ont servi d'arguments décisifs. En effet l'imitation du clergé du Temple s'est faite au fil du temps de plus en plus consciente et volontaire. Et pourtant plus il se voulait identique au clergé du Temple, moins le clergé chrétien lui ressemblait ! L'exigence du célibat lui imprimait en effet une physionomie tout à fait inédite. 

 On a vu que la continence perpétuelle était exigée des diacres et prêtres dès les premiers siècles afin qu'ils soient toujours prêts à toucher les « choses sacrées » (vases et linges sacrés, offrandes, pain consacré,~etc.), non souillés par l'impureté rituelle produite par le coït. Ce qui est remarquable c'est que cet argumentaire a emporté l'adhésion. Pourtant l'organisation du service du Temple de Jérusalem montrait une voie de compromis évidente, le service par roulement. D'autre part la notion même de pureté et de souillure religieuse, qui ne se confond pas avec celle de faute morale \emph{(péché)}, avait été mise en question par le Christ lui-même. On peut en déduire que le refus du service par roulement était motivé par des raisons autrement impérieuses que la difficulté de mettre en place un tour de service. 

 Dès l'élection du remplaçant de l'apôtre Judas et l'institution des diacres, les apôtres avaient estimé que personne ne se donne à soi-même une mission%
% [2]
\footnote{Cf. selon le livre des \emph{Actes des Apôtres} les difficultés de Paul de Tarse pour faire admettre par les apôtres sa mission auto proclamée auprès des gentils et ses prétentions au titre d'apôtre.} 
ni ne la tient de sa naissance%
%[3]
\footnote{... de même que nul ne peut (en stricte doctrine) se dire chrétien par sa naissance : il faut que chaque enfant en passe par le baptême, comme le premier converti venu.}% 
, que c'est l'Église qui appelle, et Dieu à travers elle. C'est pourquoi la succession dans le même poste ecclésiastique du père au fils, de l'oncle au neveu, sans être interdite n'a jamais été reconnue comme un droit, au contraire du droit à hériter d'un « honneur », d'une terre ou d'une entreprise, et encore moins comme un modèle. Passés les premiers siècles elle a au contraire été vue comme une irrégularité grosse de dangers. 

 Si la haute administration de l'Empire romain tardif et des royaumes barbares qui lui ont succédé est devenue vers le \siecle{10} la noblesse héréditaire du Moyen Âge, c'est parce que ceux qui étaient nommés par les autorités civiles à un emploi public ont fini par obtenir le droit de désigner eux-mêmes leur successeur, ce qui signifie que « l'honneur » (responsabilités et biens servant à les rémunérer) qui leur avait été conféré par les souverains est entré dans leur patrimoine, à la faveur de l'affaiblissement de ces mêmes souverains, système d'où est sortie la \emph{féodalité}. Au même moment un clergé marié aurait eu les mêmes chances de devenir héréditaire, et le risque eut été grand de voir se constituer une caste sacerdotale à côté de la caste aristocratique, à la mode indienne ou hébraïque. On connaît d'ailleurs un certain nombre de grandes familles de l'Antiquité et du haut Moyen Âge dont des membres se sont succédé sur le même siège épiscopal pendant plusieurs générations : Sylvère, pape de 536 à 537, était le fils légitime d'Horsmidas, pape de 514 à 523 (né avant son ordination). D'autre part les souverains et autres puissants du \siecle{6} et des siècles suivants ont utilisé leur influence pour conférer l'épiscopat à des serviteurs laïcs afin de les récompenser pour leurs loyaux services, ou bien pour les neutraliser par cet « honneur » particulier, qui leur interdisait (en principe) tout retour aux armes. 

 Si les membres d'une échelle hiérarchique peuvent donner leur poste à un héritier c'est qu'ils en sont devenus propriétaires et c'est toujours au détriment du sommet de la hiérarchie, désormais obligé de composer avec une autre source de légitimité qu'elle-même. Inversement, c'est toujours pour défendre ou renforcer son autorité qu'un souverain refuse que soit limité son pouvoir de nommer et de démettre.

 Au contraire, du point de vue d'une institution, le célibat est idéal :
\begin{enumerate}
% 1°)
\item Un clerc célibataire est plus disponible puisqu'il n'a pas à plaire à sa femme, ni à s'occuper de ses enfants (cf. Paul de Tarse).
% 2°)
\item Un clerc célibataire et sans enfants a moins de besoins matériels qu'un clerc marié et donc il \emph{peut} coûter moins cher. La continence des clercs est d'abord économique en ce qu'il n'y a pas à constituer de dot pour les filles ni à établir les garçons ...
% 3°)
\item ... qui pourraient prétendre avoir des droits sur le poste de leur père.
% 4°)
\item N'ayant pas à craindre pour ses proches, ni à les établir dans la vie, un clerc sans attaches familiales est moins sensible aux pressions et séductions venant de la société civile.
% 5°)
\item Par ailleurs il serait inconvenant que des histoires de famille puissent interférer dans les affaires de l'Église.
% 6°)
\item Enfin une paroisse, un diocèse, un monastère ne sont pas des bâtiments ni des biens fonciers. Ces institutions sont des ensembles de fidèles, et pour l'Église aucun groupe de fidèles, c'est-à-dire d'âmes immortelles, ne peut appartenir à une personne, ou à une famille, à la façon dont la force de travail des serfs (mais non leurs âmes) appartenait à leur seigneur.
\end{enumerate} 

 Voilà pourquoi la doctrine ecclésiale a toujours voulu, malgré toutes les pesanteurs individuelles et collectives qui ont entrainé de nombreux écarts, que les clercs ne soient pas issus de familles de clercs, mais issus du monde des laïcs. Et voilà pour eux autant de raisons très concrètes d'attribuer une valeur spirituelle au célibat et à la continence perpétuelle, tandis que les religieux montraient par leur exemple que cet idéal n'était pas inatteignable. 

 Cela a eu des conséquences très importantes sur la société toute entière. En effet il s'est constitué en son sein une caste non héréditaire recrutée dans les autres castes, cultivant le savoir et la culture, et au sein de laquelle les carrières n'étaient pas déterminées par la naissance, même si ces deux idéaux, toujours poursuivis, n'ont jamais été totalement atteints. La culture cultivée dans les institutions d'Église souffrait de limitations certaines et l'héritage antique n'avait pas été transmis sans pertes. Tous les clercs n'étaient pas savants, et les plus compétents n'obtenaient pas toujours les promotions auxquelles leurs talents les auraient qualifiés. De même tous les princes de l'Église n'étaient pas à la hauteur de leur charge. Mais c'est au sein du corps des moines et des prêtres que se trouvaient les plus savants de leur époque, et pour ceux qui n'avaient pas les privilèges de la naissance c'est au sein de l'Église qu'ils avaient le plus de chances de promotions. 
 
 Joseph \fsc{MORSEL} voit dans cet élitisme ecclésiastique et le modèle qu'elle a fourni, profondément intériorisé, l'une des causes principales du développement ultérieur de l'Europe et de son avance sur les autres civilisations (in \emph{L'Histoire du Moyen Âge est un sport de combat}, texte publié au format pdf sur Internet à l'adresse \url{http://lamop.univ-paris1.fr/IMG/pdf/SportdecombatMac.pdf}).
 
 