% Le 03.03.2015 :
% Antiquité
% Moyen Âge
% ~etc.
% ~\%


\chapter[Contestation de la famille du Code Napoléon]{Contestation de la famille\\du Code Napoléon}


\section{Critiques théoriques}

 Dans \emph{la police des familles}%
% [1] 
\footnote{Jacques \fsc{DONZELOT}, \emph{La police des familles}, 1977, 220 pages. Chapitre 5, « La régulation des images », p. 154 à 211.}%
, Jacques \fsc{DONZELOT} expose les débats qui ont eu lieu des premières années du \siecle{20} jusqu'à la fin du baby-boom, entre « populationnistes » et « néo-malthusiens ». Ces étiquettes schématisent en fait des oppositions qui étaient bien plus complexes... Selon lui depuis la Belle Époque les « militants », qu'il classe à côté des anarchistes, ont mis {\emph{en place les petites machines de guerre contre la famille \emph{[... que sont]} la célébration de l'union libre, \emph{[...]} la distribution des produits anticonceptionnels et \emph{[...]} la propagande pour la grève des ventres}%
% [2]
\footnote{Idem p. 163.}% 
}. Parmi les néo-malthusiens de la Belle Époque, on trouvait à côté des militants de base des médecins comme Pinard, des écrivains comme Octave Mirbeau, des hommes politiques de gauche comme Léon Blum, des savants comme Paul Langevin, soucieux {\emph{d'incorporer l'hygiène et donc le contrôle des naissances dans le fonctionnement des institutions}}. On trouvait également en première ligne la Ligue des droits de l'homme et la Société de prophylaxie sanitaire et morale, dirigées toutes deux par le docteur \fsc{Sicard~de Plauzolles}. Ils s'exprimaient dans divers ouvrages tels que \emph{La fonction sexuelle} (1908) du même docteur, ou \emph{Du mariage} de Léon \fsc{BLUM} (1908). 

 Selon \fsc{DONZELOT} leur discours {[...] \emph{est à peu près celui-là : puisque la famille est détruite par les nécessités économiques de l'ordre social actuel, il faut que la collectivité remplace le père pour assurer la subsistance de la mère et des enfants. Au père se substituera ainsi la mère comme chef de la famille ; puisqu'elle en est le centre fixe, la matrice et le cœur, elle en sera la tête. Les enfants seront sous sa tutelle, centralisée par l'autorité publique. Tous porteront le nom de leur mère ; ainsi les enfants nés d'une même femme mais de pères différents auront le même nom ; aucune différence n'existera plus entre légitimes et bâtards. L'influence de l'homme sur la femme et sur les enfants sera en rapport avec l'amour et l'estime qu'il inspirera ; il n'aura d'autorité que par sa valeur morale : il n'aura de place au foyer que celle qu'il méritera... Bref, une gestion médicale de la sexualité libérera la femme et les enfants de la tutelle patriarcale, cassera le jeu familial des alliances et des filiations au profit d'une emprise plus grande de la collectivité sur la reproduction et d'une prééminence de la mère. Soit un féminisme d'état.}%
% [3]
\footnote{Idem, p. 164.}% 
}

 Pour les plus radicaux de ces théoriciens, tels que le docteur \fsc{TARBOURIECH} (\emph{La cité future}, 1902) : {\emph{le père et la mère n'auront sur leur progéniture aucun droit d'aucune sorte, mais seulement des devoirs \emph{[...]}. C'est l'État qui déclare l'homme ou la femme apte à collaborer dans la mission d'élever tel futur citoyen et qui peut à tout moment les remplacer s'ils ne remplissent pas une mission de façon convenable, au profit d'un éleveur ou d'un éducateur offrant plus de garanties. Il s'agit donc d'étendre à toute la société le régime de la tutelle%
% [4]
\footnote{Je souligne : il s'agit d'étendre à toutes les mères et à tous les pères le régime de la tutelle.}%
, à toutes les mères l'attribution des secours éducatifs et du contrôle sanitaire, afin qu'elles soient payées comme nourrices de leurs propres enfants et les élèvent non pour elles mais pour l'État.}%
%}}}%


 Aux néo-malthusiens s'opposaient les « populationnistes ». À côté de la bourgeoisie traditionnelle attachée pour de multiples raisons à la transmission de son patrimoine, on y trouvait {[...] \emph{les ligues de pères de famille, la Ligue des mères de familles nombreuses, l'Association des parents d'élèves des lycées et collèges, l'École des parents, l'Union des assistantes sociales, les organisation scoutes, les ligues d'hygiène morale, d'assainissement des kiosques de journaux, des abords des lycées,~etc.}%
% [6]
\footnote{Idem, p. 162.}% 
} Les membres de ce groupe de pression défendaient la répartition traditionnelle des rôles sexués et des pouvoirs au sein de la famille. Ils pensaient en effet que plus la structure familiale était forte, plus elle avait de chances d'être prolifique, et de bien réaliser sa mission éducative. Ils luttaient {\emph{contre tout ce qui peut fragiliser la famille : le divorce, les pratiques anticonceptionnelles, l'avortement.}%
%[7]
\footnote{Idem.} 
}


\section{Évolutions du Droit}

 Bien des mesures décidées à cette époque par la Gauche au pouvoir allaient dans le sens des néo-malthusiens et contre les populationnistes. Le divorce%
% [8] 
\footnote{... pour faute seulement, parce que l'opinion d'alors n'acceptait pas d'autre motif, mais divorce tout de même : d'où jusqu'aux années 1970 tout un folklore de manœuvres vaudevillesque pour fabriquer en commun une « faute » légalisable (lettres d'injures...) même quand les conjoints étaient d'accord sur l'objectif.} 
permettait aux épouses maltraitées, délaissées ou bafouées, de sortir de la prison où le mariage les retenait jusque là. 

 Le premier objectif de l'obligation scolaire était certes de répandre le savoir, la culture commune, et de ne laisser personne à l'écart de cette richesse, mais un effet pleinement assumé, et même désiré, de cette obligation était aussi de contraindre toutes les familles à accepter l'entrée en leur sein de points de vue extérieurs. Elles ne pouvaient plus élever leur enfant à l'écart du monde. 

 La création d'un enseignement secondaire public pour filles calqué sur celui des garçons promouvait l'égalité complète des filles et des garçons, même si en 1880 on en était loin. C'était un choix historique, une rupture dans la répartition sexuée traditionnelle des tâches et compétences. 

 L'obligation scolaire interdisait aux parents de placer leurs enfants chez un employeur avant leurs 12 ans (11 ans s'ils avaient obtenu le certificat d'études) ou de les employer eux-mêmes à plein-temps%
% [9]
\footnote{Après le rapport de \fsc{VILLERMÉ} sur le travail des enfants, la loi du 22 mars 1841 avait fixé pour la première fois une limite d'âge au-dessous de laquelle il était interdit aux employeurs, et donc (indirectement) aux parents, de mettre les enfants au travail. La première borne avait été posée à l'âge de 8 ans. Elle avait été plus ou moins respectée mais ce n'en était pas moins le début d'une lente progression. La loi sur la scolarité obligatoire s'inscrivait comme une nouvelle étape dans cette progression, et l'exploitation du travail de l'enfant par ses parents commençait d'apparaître comme une forme de maltraitance.}% 
. 

 Quant au droit des femmes mariées à gérer leur propre salaire, c'était une part de souveraineté symbolique en moins pour les maris. En fait dans bien des ménages populaires c'étaient les femmes qui tenaient les cordons de la bourse, d'un commun accord entre conjoints (en a-t-il toujours été ainsi ? Les épouses semblent être presque toujours chargées de gérer les réserves, les resserres et les greniers, ce que symbolise le fait qu'on leur confiait les clés).

 À partir de 1912 les enfants sans père reçoivent le droit de demander des aliments à leur géniteur \emph{(recherche en paternité naturelle)}. Une mère célibataire n'est plus sans recours devant celui qui l'a laissée seule avec son enfant, qu'elle représente devant la justice. Cela entraîne pour corollaire qu'une femme mariée n'est plus aussi à l'abri qu'avant des conséquences matérielles et sociales des frasques pré ou extra conjugales de son conjoint. Pour autant un enfant illégitime ne peut toujours pas hériter de son père. 


\section{Érosion du droit de correction}

 À aucun moment de l'histoire les parents n'ont été autorisés à faire subir à leurs enfants \emph{tout} ce qu'ils pouvaient imaginer. La tolérance à leurs abus de pouvoir a pu varier au fil des siècles, mais ils n'ont jamais eu le droit de les estropier, pas plus qu'ils n'avaient le droit d'estropier les enfants des autres, ni d'en faire leurs partenaires sexuels. Mais la Justice a toujours beaucoup de difficultés à les poursuivre lorsque les traces des sévices ne se voient pas, ou dans les cas de négligence simple, d'abandon moral. En faisant du délaissement et de la maltraitance des délits, la loi {\emph{sur la protection judiciaire des enfants maltraités et moralement abandonnés}} a permis de prononcer la déchéance des droits parentaux pour ces seuls motifs. 

 Le fait de ne pas juger les mineurs et les majeurs selon les mêmes critères est sans doute aussi vieux que la justice elle-même. Refuser de traiter les fautes des mineurs autrement que celles des adultes serait manquer de bon sens. Ce qui fait la différence, c'est l'âge de la coupure entre l'irresponsabilité complète, l'atténuation de la responsabilité \emph{(l'excuse de minorité)} et la responsabilité pleine et entière. Ce sont aussi les peines encourues : nature des peines, durée... La minorité pénale était fixée à 16 ans depuis l'ancien régime. La loi du 12 avril 1904 la repousse de 16 à 18 ans, et elle affirme la prééminence de l'éducatif sur le répressif.

 La loi du 28 juin 1904 s'inscrit dans le courant d'idées qui a confié les enfants \emph{moralement abandonnés} à l'Assistance Publique. Elle ordonne que les pupilles \emph{difficiles} soient confiés non plus à des prisons, mais à des écoles professionnelles publiques ou privées. Ce texte confirme à l'administration du service des enfants assistés, détentrice de la puissance paternelle sur les pupilles, le droit de désigner ceux qu'elle garderait et ceux qu'elle refuserait d'assumer et pour lesquels elle solliciterait l'aide de la Justice. Au même moment c'étaient encore en principe les pères qui définissaient ce qui sous leur toit était indiscipline et insoumission à leur autorité.

 La loi du 24 juillet 1889 {\emph{sur la protection judiciaire des enfants maltraités et moralement abandonnés}} donne aux juges la possibilité de prononcer la déchéance totale de la puissance paternelle pour inconduite des parents, en cas de mauvais traitements ou de délaissement de l'enfant (et de plein droit dans le cas de certaines condamnations infamantes). La même loi confie à l'administration (c'est-à-dire à l'Assistance Publique) la tutelle des enfants maltraités, victimes de crimes ou de délits ou délaissés. Le service les prend en charge même s'ils sont âgés de plus de 12 ans à leur entrée. Ces enfants deviennent des pupilles comme les autres. Ils sont traités à l'instar des autres enfants du service. Quel que soit leur âge, autant que faire se peut ils seront placés en nourrice, pour de longues durées, et dans tous les cas ils seront totalement coupés de leurs parents déchus.

 Une nouveauté majeure est introduite en 1912, avec la création des tribunaux pour enfants (sans magistrats spécialisés) et la création de la liberté surveillée%
% [10]
\footnote{Cf. l'ouvrage collectif \emph{Protéger l'enfant} (1996), qui aborde les problèmes de la jeunesse sous l'angle de la \emph{protection judiciaire}. Il présente un résumé de l'histoire de celle-ci, et des débats d'idée et des conflits de pouvoir qui ont présidé à sa naissance et qui la traversent encore...}%
. Cette nouveauté avait été précédée depuis les années 80 par tout un mouvement d'idées, notamment chez les magistrats chargés de l'application du droit de correction paternelle. Il y a en effet un lien direct entre la dénonciation de l'indignité des pères (cf. la loi de 1889 sur la déchéance paternelle), et la mise en cause du droit de correction%
%[11]
\footnote{Pascale \fsc{QUINCY-LEFEBVRE}, « Une autorité sous tutelle. La justice et le droit de correction des pères sous la troisième république », in \emph{Lien social et politiques-RIAC}, 37, Printemps 1997, p. 99 à 109.}% 
. Ceux qui s'intéressaient à ce problème ne contestaient en aucune façon l'existence d'enfants \emph{insoumis}, difficiles à élever et qui provoquaient le \emph{légitime} mécontentement de leurs parents. Ils estimaient par contre que c'était un problème qui débordait le cadre familial, parce qu'on pensait qu'en règle générale ceux qui étaient insoumis à leurs parents ne faisaient pas de bons citoyens, et risquaient de devenir délinquants, c'est pourquoi l'état ne pouvait s'en désintéresser. Ils estimaient surtout qu'il n'était pas possible de s'en tenir à la parole du parent, et qu'il fallait s'assurer par une enquête approfondie de la réalité et de la nature des problèmes. 

 D'autre part ils estimaient que la prison n'était pas un outil de correction efficace, et qu'il fallait fournir aux jeunes insoumis une prestation éducative de durée suffisante pour obtenir d'eux un amendement réel. Ils pensaient que cette prestation devait être fournie par un internat sous le contrôle de la Justice et non sous celui des pères. Ils accusaient en effet ceux-ci (ceux du moins qui réclamaient à la justice son aide, c'est-à-dire ceux des pères, tous de milieu populaire, qui ne pouvaient supporter les frais d'une pension dans l'un des internats privés dont c'était la spécialité) d'être trop prompts à retirer leurs enfants (comme ils en avaient le droit) dès que ceux-ci semblaient suffisamment \emph{intimidés} par l'incarcération. Ils les suspectaient de n'avoir qu'un seul but, celui de mettre le plus vite possible leurs enfants au travail pour toucher leur salaire. Aux yeux des réformateurs, les droits des pères (éducatifs ou financiers) importaient moins que l'intérêt des enfants, qui était de recevoir une bonne éducation durant le temps nécessaire et avec la sévérité qui convenait, et que l'intérêt de la société, qui était de voir conduire à son terme la \emph{correction des insoumis}. 

 C'est donc du fait des juges et non à la demande de la société que la correction paternelle est peu à peu tombée en désuétude. Ils ont pris l'habitude dès les années 1890 de demander systématiquement une enquête pour vérifier si le parent demandeur avait vraiment des \emph{sujets de mécontentement très graves}, et s'il n'était pas plutôt un parent \emph{indigne}. Ils ont ainsi retiré aux parents leur droit de qualifier eux-mêmes de fautifs les comportements de leurs enfants. Puis la loi de 1889 leur a donné la possibilité non seulement de refuser aux parents indignes une demande de correction paternelle, mais encore de leur retirer la garde de l'enfant. Enfin la loi de 1904 les a explicitement autorisés à mettre les pupilles indisciplinés en maison de correction pendant plusieurs années (c'était déjà le sort des pupilles indisciplinés ou récalcitrants du \siecle{19}). Ces pupilles pouvaient être les enfants de parents déclarés \emph{indignes} une fois que leur déchéance était prononcée. Les parents étaient souvent qualifiés d'indignes parce qu'ils laissaient la bride sur le cou de leur enfant en ne le contrôlant pas d'assez près, ou parce qu'ils entravaient les efforts des éducateurs qui tentaient de les amender. Il semble qu'à cette époque les juges et les premiers travailleurs sociaux déploraient plus leur laxisme que leur autoritarisme. 

 En 1921 une loi ouvre la possibilité de prononcer une \emph{déchéance partielle} de l'autorité paternelle. Une déchéance totale des droits parentaux était une mesure aux effets quasi irréversibles. À partir de 1921 les magistrats n'ont plus été réduits au tout ou rien d'une telle mesure face aux parents qu'ils jugeaient incompétents, délinquants ou négligents. Au contraire ils pouvaient prononcer une déchéance partielle et provisoire, non seulement là où la déchéance totale aurait été injustifiée, mais même là où ils y auraient recouru par nécessité en l'absence d'une mesure plus souple. Le nombre de ces décisions a donc crû rapidement. Cela ne s'est pas traduit par un accroissement important du nombre de jeunes placés, mais par un changement du statut de beaucoup d'enfants placés : le nombre des pupilles a décru au fur et à mesure qu'augmentait celui des enfants en garde, sans que l'effectif total ne se modifie sensiblement. Pendant ce temps le nombre des abandons ne cessait de diminuer.

 On a vu que dès la fin du \siecle{19} des juges avaient commencé d'ordonner des enquêtes pour évaluer la pertinence des demandes de correction paternelle. En 1923 le succès de cette pratique a conduit à la création à Paris, où étaient traitées les deux tiers des demandes de correction paternelle faites en France, d'un service social réalisant pour le tribunal des enquêtes débordant largement la matérialité des faits reprochés par les parents à leur enfant. Désormais la demande d'intervention des parents était entendue comme l'expression d'un dysfonctionnement dans la famille, qui dépassait largement le mineur concerné. Cela entraînait une enquête sociale, c'est-à-dire l'introduction au sein de la famille, d'un observateur extérieur mandaté par les juges. À partir de cette base ces derniers se sont donné le droit de conseiller les parents face aux problèmes que leur posaient leurs enfants : dans la plupart des cas cela les conduisait à mettre en œuvre une action non judiciaire, confiée sous leur contrôle à des institutions privées. Il s'agissait très souvent d'une \emph{action éducative en milieu ouvert}, mais ils pouvaient aussi prendre l'initiative de placer en établissement de correction les mineurs qui leur semblaient en avoir besoin, entre autres au titre des lois de 1889 et de 1921 sur la déchéance paternelle, et de 1904 sur les pupilles difficiles ou vicieux. Tous ces placements écartaient le contrôle paternel.

 Les magistrats n'accédaient plus à la demande de correction paternelle que dans un nombre de cas de plus en plus petit : environ un cas sur quatre ou cinq en 1917, un sur dix dans les années trente. Ils ont ainsi vidé de sa substance le droit de correction paternelle. Le nombre des mineurs placés à ce titre n'a donc cessé de baisser jusqu'à devenir marginal, comme le montre la table \vref{ord-corr-pat}.
 
\begin{table}[h]
\centering
\caption{Ordonnances de correction paternelle}
\label{ord-corr-pat}
\begin{tabular}{ccc}
 & France & part \\
Année & entière & Seine \\
\hline
1881 & 1192 & 63,7~\% \\
1891 & 737 & 59,6~\% \\
1901 & 731 & 50,0~\% \\
1911 & 644 & 66,9~\% \\
1921 & 270 & 83,3~\% \\
1931 & 60 & 66,7~\%
\end{tabular}
\end{table}
 
 Le décret-loi du 30 octobre 1935 sur {\emph{la correction paternelle et l'assistance éducative}} institue l'assistance éducative à domicile. Il entérine les changements qui travaillaient depuis deux générations l'exercice du droit de correction paternelle. Il a également dépénalisé le vagabondage des mineurs (les fugues simples, sans délits caractérisés) ce qui suggère que ces deux ordres de faits se recouvraient. Désormais les jeunes vagabonds ne ressortissaient plus de la Justice, mais mais d'une assistance placée sous le contrôle des juges. 



