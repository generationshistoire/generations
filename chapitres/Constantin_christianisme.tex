% Le 03.03.2015 :
% Antiquité
% Moyen Âge
% Au 24.02.2015 :
% ~etc.
% Moyen-Âge
% ~\%


\chapter{Le tournant constantinien}
\emph{Pour ce qui concerne mon sujet}, à savoir \emph{l'histoire de la génération}, l'ère qu'ouvre Constantin ne s'achèvera \emph{au plus tot} qu'avec les Lumières. Il s'agit là d'un domaine où la notion de \emph{long moyen âge} (cf. Le Goff) est pertinente. Du côté des doctrines et des représentations, fournies essentiellement par les clercs, théologiens, évêques et religieux, ce millénaire montre une grande continuité d'inspiration, si bien qu'il est malcommode de le diviser en tranches historiques successives même si les réformes grégorienne et protestante marquent deux césures très significatives. Par contre en ce qui concerne l'inscription des doctrines dans les pratiques quotidiennes, on observe de grandes particularités suivant les époques et suivant les lieux, mais il ne me sera pas possible d'entrer dans des "détails" qui ont suffi à remplir les vies et les oeuvres de nombreux chercheurs. Dans cette partie de mon récit je me bornerai donc à tenter de décrire les logiques en présence et leurs articulations, avec une centration progressive sur l'Europe et sur la France. 

\section{Le choix d'un homme ?}
En 313  Constantin (272-337) rompt avec la politique romaine traditionnelle, qui est aussi celle de ses rivaux dans la course au pouvoir suprême. Il ordonne l'arrêt des persécutions en cours contre les chrétiens dans la partie d'empire qui lui a été confiée \footnote{Sources :
\\Danièle \fsc{ALEXANDRE-BIDON} et Didier \fsc{LETT}, \emph{Les enfants au Moyen Âge, \siecles{5}{15}}, 1997.
\\Didier \fsc{LETT}, \emph{Famille et parenté dans l'Occident médiéval, \siecles{5}{15}}, 2000.
\\Peter \fsc{BROWN}, \emph{Genèse de l'Antiquité tardive}, 1978.
\\Peter \fsc{BROWN}, \emph{Pouvoir et persuasion dans l'Antiquité tardive, vers un empire chrétien}, 1992.
\\Peter \fsc{BROWN}, \emph{Le renoncement à la chair, virginité, célibat et continence dans le christianisme primitif}, 2002.
\\Jean-Michel \fsc{CARRIE}, Aline \fsc{ROUSSELLE}, \emph{L'Empire romain en mutation, des Sévères à Constantin}, 192-337, 1999.
\\Christian \fsc{DELACAMPAGNE}, \emph{Une histoire de l'esclavage, de l'Antiquité à nos jours}, 2002.
\\Jean \fsc{DURLIAT}, \emph{De l'Antiquité au Moyen Âge, l'Occident de 313 à 800}, 2002.
\\Alexandre \fsc{FAIVRE}, \emph{Naissance d'une hiérarchie, les premières étapes du cursus clérical}, 1977.
\\Bertrand \fsc{LANCON}, \emph{Le monde romain tardif, \siecles{3}{7} ap. J.C.}, 1992.
\\Henri-Irénée \fsc{MARROU}, \emph{Décadence romaine ou Antiquité tardive ? \siecles{3}{6}}, 1977.
\\Henri-Irénée \fsc{MARROU}, \emph{L'Église de l'Antiquité tardive}, 303-604, 1963.
\\Paul \fsc{PETIT}, \emph{Histoire générale de l'Empire romain, 3, le Bas-Empire (284-395)}, 1974.
\\Aline \fsc{ROUSSELLE}, \emph{La contamination spirituelle, science, droit et religion dans l'Antiquité}, 1998.},
et il inscrit leur religion dans la liste des cultes reconnus. 

Pourquoi l'empereur de Rome s'allie-t-il avec les chrétiens ? 
C'est sûrement pour étayer son pouvoir \footnote{cf. son rêve de la veille de la bataille décisive du Pont de Milvius, où il aurait entendu que : \latin{in hoc signo, vinces} (c'est à dire "\emph{sous cet étendard tu vaincras}"), à la suite duquel il aurait décidé de faire combattre ses soldats sous les enseignes du Christ.} mais c'est peut-être aussi par conviction personnelle. Les historiens en débattent toujours\footnote{Pour \fsc{VEYNE} c'est d'abord par conviction personnelle : \emph{Quand l'empire romain est devenu chrétien}, 2007.}. Il est de fait qu'il n'a été baptisé que peu de temps avant sa mort, ce qui ne plaide pas \emph{a priori} pour une foi ardente, mais cette façon de faire était fréquente chez les chrétiens de son temps. Elle lui laissait les mains plus libres pour se livrer aux basses oeuvres impliquées parfois par la conquête et l'exercice du pouvoir, ce dont il ne s'est pas privé. D'autre part elle lui permettait d'assumer la fonction de \emph{Grand Pontife} au même titre que les empereurs précédents\footnote{...qui depuis Jules César s'étaient attribué cette fonction. Ce dignitaire était le prêtre le plus important de Rome. Il avait le pas sur tous les desservants de tous les cultes autorisés, et il supervisait le bon fonctionnement de ceux-ci.}. Il a donc été le Grand Pontife jusqu'à sa mort. Ses successeurs ne se dessaisiront totalement de ce titre que vers 380. Est-ce en tant que Grand Pontife ou en tant qu'Empereur que Constantin a placé le culte des chrétiens au rang de religion officielle ? 

A-t-il été convaincu par une supériorité (dont le domaine serait à définir) du christianisme et/ou par le nombre ou l'influence des chrétiens ? C'est l'interprétation \emph{providentielle} qui a été proposée par les auteurs chrétiens depuis \emph{Eusèbe de Césarée} (mort en 339) et sa "\emph{Vie de Constantin}" jusqu'à nos jours. Selon Yvon Thébert\footnote{Yvon THEBERT, "A propos du "triomphe du christianisme", in \emph{Dialogues d'histoire ancienne}, vol 14, 1988} il faut se déprendre de cette interprétation qui à ses yeux n'est que propagande et se demander sans inhibition comment le christianisme a pu être privilégié par un postulant à l'empire qui en matière de \emph{religions à mystère} ou de \emph{religions du salut} n'avait que l'embarras du choix ? Pourquoi s'est-il appuyé sur une secte obscure, lentement et difficilement dégagée de sa matrice juive en dépit de sa prétention à l'universalité, dont le fondateur divin et humain à la fois (ce qui à cette époque était banal) a été condamné à une mort infamante par un magistrat \emph{romain} (ce qui l'était beaucoup moins), professant un monothéisme exclusif mais dont le dieu est composé de trois entités dont les natures et les relations sont un tel tourment pour la raison que depuis l'apparition de la secte elles provoquent chez les adeptes une insécurité dogmatique et des clivages incessants ?  Et comment son choix a-t-il pu être confirmé par tous ses successeurs à l'exception de son neveu Julien dont il avait fait assassiner le père ?  

Il faut se demander si ce n'est pas le choix de Constantin qui a été l'élément déterminant dans le triomphe du christianisme. Il faut même aller jusqu'à oser voir que c'est lui qui a organisé cette religion en fonction de ses propres intérêts et cela jusque dans les dogmes (cf. le concile de Nicée, centré sur la nature du christ, convoqué et présidé par Constantin\footnote{Il attendait du concile qu'il fasse la paix entre les Ariens et les autres chrétiens, mais le concile a condamné les thèses d'Arius. Constantin ne pouvait donc pas toujours obtenir des évêques tout ce qu'il voulait, alors qu'il avait une préférence pour ces thèses et qu'il semble avoir choisi de se faire baptiser par un évêque arien.}). En conséquence "\emph{ce n'est pas le christianisme qui submerge le pouvoir, c'est le pouvoir qui utilise le christianisme et qui, pour ce faire, va le modeler de façon décisive en fonction de ses besoins}". Et ce qui intéresse l'empereur, c'est que "\emph{Fondamentalement, cette église se distingue des autres par l'organisation rigide du pouvoir : tendances constantes au monarchisme épiscopal, embryons déjà vigoureux de la théorie de la suprématie romaine.}" Ce qui l'intéresse c'est la capacité de l'Eglise à encadrer la masse des fidèles dans le respect du pouvoir civil et à exténuer les désirs de transformation sociale, et c'est celle de l'eveque de Rome à ramener les déviants idéologiques dans la voie moyenne. Au terme de sa réflexion Thébert conclut (en marxiste orthodoxe) que : "\emph{Le catholicisme n'a pas conquis la société du Bas-Empire, il a été secrété par elle : il en est le produit, tout comme la morale ou l'art de cette époque}"\footnote{En réduisant l'Eglise du Bas-Empire à n'être que le produit d'une politique qui la dépasse cette interprétation est de nature à relativiser les reproches d'intolérance persécutrice que l'on lui fait souvent (Thébert le premier) : en exerçant une police de la pensée, elle n'aurait fait que ce que lui demandait l'empereur, sans le \emph{bras séculier} duquel ses anathèmes seraient restés sans force.}. Reste à comprendre par quel miracle les agrégats du bricolage religieux d'un empereur romain du IVème siècle ont pu prendre en masse plus durablement que le béton de ses grands chantiers de Constantinople et de Rome \footnote{Cet article iconoclaste et stimulant ne représente pas la position moyenne des spécialistes du Bas-Empire ni du christianisme ancien, comme le constate Patrick BOUCHERON  dans "Le Génie de l'athéisme" (in \emph{Afrique et Histoire}, 2005/1, vol. 3) et il a été peu cité et commenté.}.


\section{l'alliance du trône et de l'autel}


Quoi qu'il en soit de ses motivations Constantin a orienté l'histoire à venir de manière irréversible et conclu une alliance du trône et de l'autel qui en Europe comme à Byzance a perduré sans changements significatifs pendant plus d'un millénaire. À partir de sa victoire sur ses rivaux il a traité les évêques chrétiens comme les clergés des autres religions reconnues, c'est-à-dire comme un corps de magistrats religieux associés au pouvoir civil. Ils ont accepté avec gratitude leur nouveau statut comme un développement providentiel de \emph{l'histoire du salut}, comme la reconnaissance de leurs droits légitimes. Pour eux comme pour tous leurs contemporains, il était inimaginable que l'Etat se désintéresse des dieux et les relègue dans la sphère privée, au risque que l'un d'entre eux ne se venge cruellement de cette négligence discourtoise. 
 
 L'Église bénéficiait désormais de tous les droits des cultes reconnus par l'État : droit de recevoir dons et legs, inaliénabilité des biens fonciers, exemptions d'impôts,~etc, et Constantin l'a rapidement favorisée en la dotant de bâtiments et de propriétés terriennes\footnote{Selon les thèses (controversées mais séduisantes) de \fsc{DURLIAT} (2002), il pouvait s'agir de la \emph{propriété éminente} de \latin{villae}, qu'il faut distinguer de la propriété ordinaire, de la propriété \emph{utile} (c'est-à-dire le droit de vendre, et d'acheter la terre, le droit de la mettre en valeur et de jouir des fruits du sol). Selon lui la \latin{villa} était à cette époque \emph{une circonscription fiscale} peuplée notamment de cultivateurs (nommés \emph{colons} en langage administratif) qui pouvaient être propriétaires de leur exploitation agricole, ou simples tenanciers. À cette époque, le terme villa pouvait aussi désigner un château et ses terres, ce qui crée des confusions. Celui qui possédait la propriété \emph{éminente} d'une villa fiscale, nommé le \latin{dominus}, était chargé d'y faire la collecte des impôts : le \emph{cens}, les droits de mutation,~etc. Ceux-ci correspondaient selon les estimations de \fsc{DURLIAT} à environ 20~\% de l'ensemble des revenus des contribuables, fournis en monnaie, en nature, ou en corvées, et dont une partie revenait au \latin{dominus} pour prix de ses services. Il était en quelque sorte le percepteur de cette \latin{villa}. C'était une charge lucrative et honorable tout à la fois. Les grandes fortunes de l'empire romain reposaient sur ces propriétés très particulières, d'ailleurs non exclusives de la propriété utile des mêmes domaines. On pouvait recevoir ces charges de l'empereur, ou les transmettre par héritage, les vendre et les acheter (comme la \emph{ferme des impôts} sous l'Ancien Régime). Le \latin{dominus} assurait le lien entre les colons concernés et les administrations de l'État. Il était en quelque sorte le Seigneur de cet espace. Par certains aspects cela préfigurait les \emph{seigneuries} apparues à partir du milieu du Moyen Âge.}% 
. Pour ne pas les obliger à sacrifier aux dieux civiques, les clercs étaient exemptés des charges curiales, c'est-à-dire de l'obligation pour les plus fortunés de participer à la curie, au conseil municipal de leur cité
\footnote{... ce qui était à la fois un honneur et un impôt, puisqu'ils devaient financer de leurs propres deniers certaines des dépenses de celle-ci. Cf. P. \fsc{VEYNE}, 1976.}. L'exemption des charges curiales (dont bénéficiaient aussi un certain nombre de prêtres des cultes civiques) leur permettait de consacrer leur temps et leur fortune à d'autres formes d'\emph{évergésies} que celles que devaient traditionnellement pratiquer les curiales. L'empereur comptait sur eux pour investir dans l'assistance aux pauvres, la construction d'hôpitaux,~etc. 

Les évêques n'ont jamais eu tout pouvoir sur Constantin ni sur ses successeurs. C'était ordinairement le contraire. L'Église se définissait elle-même comme un partenaire qui n'avait pas vocation à l'exercice du pouvoir temporel. Quand un prêtre, un évêque ou un pape prétendait néanmoins gouverner les affaires courantes, ce qui n'a pas manqué de se produire, les autorités civiles le renvoyaient à l'Évangile : \emph{rendez à César ce qui est à César, et à Dieu ce qui est à Dieu} (Mt 22, 17-21). Mais en contrepartie de ce renoncement les clercs recevaient l'exclusivité sur le culte, sur les dogmes qui servaient de cadre de pensée à tous, sur les mœurs, notamment familiales, et progressivement sur l'enseignement. 

 Il n'existait pas à Rome de pouvoir judiciaire indépendant des autres administrations impériales. l'empereur était la source du droit et tous les jugements étaient faits en son nom. Comme tous les autres magistrats de l'Empire, dont les prêtres des autres religions reconnues, les évêques ont donc reçu délégation pour régler les litiges qu'on leur soumettait et qui ressortaient de leurs compétences. C'était reconnaître officiellement le rôle d'arbitrage qu'ils exerçaient déjà officieusement, notamment en matière familiale et doctrinale. Comme les prêtres des autres cultes reconnus jusque là ils ont reçu le pouvoir d'enregistrer les affranchissements d'esclaves : leurs actes écrits avaient désormais force de preuves. Leurs arbitrages devaient d'autant plus être respectés qu'ils avaient la faveur de l'empereur, comme en atteste le fait qu'ils ont reçu le pouvoir de juger seuls des fautes des clercs chrétiens (hors affaires criminelles).
 
 

D'une certaine façon avec Constantin peu de choses changeaient dans la nature des liens entre l'empereur-grand-pontife et sa religion préférée. Ainsi c'est lui qui en 325 a convoqué dans son palais de Nicée le premier concile "œcuménique"
\footnote{...qui devait rassembler tous les évêques vivants} 
de l'Église, parce qu'il voulait imposer un accord dogmatique entre les ariens et les non-ariens et c'est lui qui l'a présidé. A l'issue de ce concile c'est lui qui a donné force de lois à ses décisions en les contresignant. Lorsque l'un des successeurs de Constantin aura pris fait et cause pour l'hérésie arienne ce seront les catholiques qui subiront la défaveur du prince. Il s'agissait d'un échange de légitimités entre empereur et évêques, d'un étayage réciproque, conforme aux traditions antiques de confusion de la religion et de la cité. Dans la compréhension de l'œuvre législative réalisée sous le règne de Constantin il faut donc tenir compte non seulement de l'influence de l'Église, qui est souvent évidente, mais aussi de sa marque personnelle et de l'évolution générale des attitudes romaines face aux mœurs, au couple et à la famille. Tant que l'alliance des gouvernants et des églises perdurera la question de l'influence réciproque sera posée. Il est parfois très difficile de dire aujourd'hui lequel des deux partenaires était à l'origine d'un fait, d'une décision, d'une règle, d'une institution, en un mot d'un des symptômes de cette alliance. 

Selon son propre mot Constantin se considérait comme « l'évêque du dehors », l'évêque (\emph{episcopos} : le surveillant, l'inspecteur) des non chrétiens. Il estimait de son devoir de conduire l'ensemble de ses sujets, chrétiens ou non, à la vérité (telle qu'il la concevait), et à défaut de les convertir tous, ce qui était du ressort des évêques, il entendait au moins mettre les lois en accord avec les principes chrétiens (avec ceux du moins qu'il approuvait) et créer ainsi un milieu de vie qui favoriserait les conversions. Il ne s'est pas borné pas à favoriser le culte chrétien, il a aussi initié la séparation de l'État romain et des religions païennes. Celle-ci va s'effectuer par étapes entre 325 et 391. En 382 un de ses successeurs supprime les privilèges des vestales et des prêtres païens et interdit aux cités de financer les temples païens (nomination des prêtres, entretien des bâtiments, fourniture des offrandes pour les sacrifices, achat d'encens,~etc.). Progressivement les biens et les terres de ceux-ci sont confisqués au profit du trésor public. Un des petit-fils de Constantin décrète en 391 que seule la religion chrétienne est désormais autorisée. Tous les sujets de l'empire sont à partir de cette date fermement invités à se faire baptiser et à professer publiquement la foi que celui-ci leur désigne : la doctrine définie par l'évêque de Rome \footnote{... ce qui en soi est nouveau : il n'y avait pas de profession de foi consciente et articulée dans les religions antiques, du moins pas avant le triomphe du christianisme.}. Tous les autres cultes sont interdits. Seule la dissidence juive continuera d'être tolérée, comme une espèce de « butte témoin » de l'ancienne alliance, mais il lui sera interdit de faire des prosélytes, surtout parmi les chrétiens\footnote{La conversion des chrétiens ou des païens au judaïsme était interdite, de même que les mariages mixtes. Si un juif faisait circoncire son esclave chrétien cela entraînait \latin{ipso facto} l'affranchissement de ce dernier, s'il le réclamait. Dès 313 Constantin condamnait à mort les juifs qui lapidaient ceux de leurs religionnaires qui se convertissaient au christianisme.}. 

 Si Constantin n'avait pas récusé le titre de \latin{Pontifex Maximus}, ses successeurs n'ont plus porté ce titre, sauf Julien qui a tenté de remettre la religion traditionnelle à l'honneur. Malgré de fortes tentations et une sacralisation de leur fonction ambiguë, aucun empereur, aucun roi "très chrétien" n'osera
\footnote{... jusqu'à Henri~VIII, roi d'Angleterre.} 
se proclamer dignitaire de l'Église. C'est l'évêque de Rome qui héritera du titre de pontife. Cela n'empêchait pas les empereurs et les rois de se considérer comme les partenaires permanents et obligés des évêques, comme les soutiens de leur pouvoir et le bras armé qui défendait leurs enseignements. Soucieux d'ordre et d'unité ils seront étroitement associés à l'Église dans le choix des évêques, les arbitrages théologiques et disciplinaires, la convocation des conciles, la définition de la discipline ecclésiastique,~etc. Jusqu'au milieu du Moyen Âge leurs édits, décrets et codes concerneront le fonctionnement interne des églises et la discipline ecclésiastique au même titre que celui des autres corps de la société, et les décisions des conciles n'auront force exécutoire que pour autant qu'ils les auront approuvées et appuyées de leur autorité (cela est resté vrai en France au moins jusqu'à la révolution). Le sacre (et surtout l'onction) des empereurs et des rois sera l'objet d'une valorisation qui en fera presque un sacrement analogique à l'ordination des clercs. À Constantinople l'Église et l'Empereur vont être inséparables pendant un millénaire, étayés l'un par l'autre et se chargeant à eux deux de régir l'empire. 

Certes, en première analyse, l'évangélisation des populations non catholiques (les "barbares", les juifs et les chrétiens "hérétiques") a été l'œuvre de l'Église. Mais elle a été presque toujours été appuyée dans cette tâche par les pouvoirs publics. Il s'agissait d'abord et avant tout de convertir les rois et les puissants et les rapports de force militaires facilitaient les choses. Le reste suivait presque mécaniquement. En dépit de la doctrine ecclésiale qui voulait que les candidats au baptême agissent librement et de leur propre initiative, ils l'ont parfois fait sous la contrainte, et tous les évêques n'ont pas protesté contre les pressions subies par les « païens », les « hérétiques » (ariens en particulier) et les juifs (notamment en Espagne)
\footnote{Sur ce sujet on peut se référer notamment au livre de Bruno \fsc{DUMEZIL}, \emph{Les racines chrétiennes de l'Europe, Conversion et liberté dans les royaumes barbares \siecles{5}{8}}, Fayard, 2005, Paris.}. 

 Les décrets et rescrits promulgués par Constantin et ses successeurs n'ont pas fait disparaître les pratiques antérieures d'un trait de plume, ni transformé toutes les familles de l'empire romain, plus ou moins bien baptisées, ou pas baptisées du tout, en autant de \emph{saintes familles}\footnote{Sources : Jean-Pierre \fsc{LEGUAY}, \emph{L'Europe des états barbares, \siecles{5}{8}}, Belin, Paris, 2002. Jean-Pierre \fsc{POLY}, \emph{Le chemin des amours barbares, Genèse médiévale de la sexualité européenne}, 2003. Stéphane \fsc{LEBECQ}, \emph{Les origines franques, \siecles{5}{9}, Nouvelle histoire de la France Médiévale}, 1990.}. Les non chrétiens, encore majoritaires en 313, ont certes fait de la résistance face aux décrets de Constantin, mais les baptisés aussi. L'ordre public possède ses propres logiques : la société a toujours toléré ou soutenu bien des choses que les évêques réprouvaient. Inversement l'Église a défendu avec une persévérance millénaire des positions que la société n'a jamais cessé de considérer comme idéalistes, impraticables ou irresponsables. Les sociétés n'ont jamais été une pâte malléable dans les mains du clergé, qui était lui-même divisé sur bien des sujets de morale personnelle et familiale, et sur bien des points d'accord avec ses ouailles. En dépit du poids de l'Église, il n'y a jamais eu de coïncidence rigoureuse entre les lois civiles et les prescriptions religieuses. Face à celles-ci les autorités civiles ont suivant les cas adopté toutes les positions possibles : de la collaboration intéressée et active, et même pressante, quand cela allait dans le sens de leurs intérêts, à l'opposition franche, en passant par l'inertie sceptique.


\section{Constantin et le droit des personnes}


\begin{description}
\item[313] Constantin :
 \begin{enumerate}[leftmargin=*,itemsep=0pt]
%  a)
\item contrairement aux décisions de ses prédécesseurs, reconnaît aux citoyens indigents le droit de vendre leurs enfants, mais seulement à leur naissance ; 
% b)
\item décide que cette vente suspend la puissance paternelle. Le père garde le droit de récupérer son enfant, même contre le gré du possesseur, mais à la condition de fournir en échange un esclave de valeur équivalente, ou la somme correspondante ; 
% c)
\item reconnaît à celui qui recueille un nouveau-né exposé ou qui l'achète à son parent le droit d'en faire son esclave. Cette décision aggravait indiscutablement le sort juridique des enfants puisque le statut d'esclave ne pouvait plus être contesté, ni par les enfants concernés ni par leurs parents : s'ils parvenaient à les racheter, ces enfants étaient des affranchis, non des ingénus, avec toutes les limitations juridiques que cela entraînait. Il paraît plausible que cette décision ait été prise pour favoriser l'accueil de tous les enfants abandonnés sans exception. En effet jusque là les pères qui avaient abandonné, et non vendu, leurs enfants, pouvaient les récupérer à tout moment sans avoir à verser aucune contrepartie financière, puisqu'il était interdit d'asservir un citoyen né libre. Ils pouvaient donc être tentés de réclamer l'enfant qu'ils avaient exposé dès qu'il était capable de leur rapporter de l'argent, quitte à le revendre l'instant d'après : ce n'était pas impossible dans un monde esclavagiste. Cela pouvait suffire à dissuader les bonnes volontés et les spéculateurs de recueillir les enfants abandonnés ? 
\end{enumerate}
 
\item[315] Il décide (ou plutôt il rappelle ce qui était le cas jusque là) que la vente d'un ingénu est illégale dès qu'il ne s'agit plus d'un nouveau-né, et qu'elle ne peut effacer le statut initial d'un ingénu : l'intéressé peut donc en tout temps revendiquer sa liberté devant les tribunaux (à charge de fournir des preuves suffisantes).

% 315 
Il décide que les biens d'une mère défunte sont de droit la propriété de ses enfants. C'était d'ores et déjà l'usage établi, mais il fallait jusque là un testament maternel en bonne et due forme.

% 315 
Le mari devient le \emph{curateur} de sa femme, c'est lui et non plus le père de celle-ci qui gère ses affaires financières tant qu'elle n'a pas eu trois enfants et ne peut donc selon le Droit le faire elle-même. Était-ce un progrès ? Sûrement pour le mari, peut-être moins pour l'épouse qui ne pouvait plus se retourner vers son père ou son tuteur en cas de conflit avec son mari.

\item[316] Constantin revoit l'arsenal des peines prévues contre ceux qui enlèvent les enfants pour les vendre. Il leur promet la condamnation au travail forcé dans les mines%
%[1]
\footnote{Constantin a interdit la crucifixion, la condamnation aux jeux du cirque et celle des femmes aux lupanars, remplacés par la condamnation aux mines. En fait le travail des mines était très dur et particulièrement malsain (ex. emploi du feu en front d'exploitation pour désagréger la roche...) et on y mourait vite, ce n'était donc qu'un adoucissement relatif. L'interdit de Constantin avait l'avantage connexe de tarir l'une des sources des distractions offertes au public dans les jeux du cirque, condamnés depuis toujours par les chrétiens. Constantin interdisait aussi les mutilations du visage, ce qui ne l'empêchait pas de prévoir des peines terribles pour divers délits.}% 
. Cette décision n'était pas nouvelle, et ces rapts continueront aussi longtemps qu'il sera permis de vendre et d'acheter des esclaves.

\item[318] Il décide d'étendre la notion juridique de parricide à tous ceux, parents ou enfants, qui tuent un de leurs parents. Les parents infanticides seront désormais passibles de la peine de mort. 

 Dans le même mouvement \emph{il assimile l'avortement à un infanticide.}

 Par ailleurs à partir de Constantin, la puissance paternelle est retirée non seulement aux pères qui exposent leurs enfants, comme on l'a vu plus haut, mais aussi à ceux qui les prostituent et à ceux qui ont avec eux des relations sexuelles. Les pères déchus de leurs droits continuent de devoir assumer financièrement leurs enfants, mais ces derniers sont confiés à un tuteur.

\item[320] Il abroge les lois d'Auguste contre le célibat : dès l'âge de 25 ans, un homme ou une femme \emph{sans enfants}, célibataire ou non, \latin{sui juris}, peut recevoir tous les héritages venant de personnes extérieures à sa famille. Il n'est plus non plus question de sanctionner financièrement le célibat par un impôt spécial. Le refus du mariage ou du remariage n'est plus pénalisé.

\item[321] Constantin accorde à l'Église le droit de recevoir des legs et des successions, même par une simple déclaration orale. 

\item[324] Il dispense de tutelle les jeunes gens \latin{sui juris} (orphelins de père) qui ne sont pas infâmes%
%[2]
\footnote{... c'est-à-dire qu'il émancipe presque tous les jeunes citoyens : la plupart d'entre eux ne sont en effet pas infâmes, à l'exception des prostitué(e)s.}% 
, les filles dès 18 ans et les garçons dès 20 ans. Jusque là les garçons \latin{sui juris} avaient un tuteur jusqu'à leurs 25 ans, et seules les mères \latin{sui juris} de 3 enfants pouvaient être exemptées de tuteur à partir de leurs 25 ans. Les filles continuent de devoir obtenir l'accord de leurs parents (mère, oncles, grand-pères, frères (?)) pour se marier, \emph{mais elles reçoivent le droit de s'en passer pour entrer en religion}. Ce choix de vie est donc protégé contre les pressions des parents, alors que le choix du conjoint (et de la famille avec laquelle faire alliance) ne l'est pas : \emph{le mariage reste une alliance de deux familles}.

\item[325] Le concile de Nicée, approuvé par l'empereur, ordonne la création auprès de chaque évêque de « maisons de charité », pour les malades, les pauvres, les vieillards, les voyageurs, les pèlerins et les infirmes. Il confie la gestion de ces maisons à un « religieux du désert » c'est-à-dire à un moine, explicitement invité à considérer son travail quotidien au service des indigents et des malades comme une prière. En confirmant les décisions du concile Constantin donnait aux évêques la mission de réaliser en grand, à l'échelle de l'empire, ce qu'ils avaient expérimenté depuis le premier siècle et qu'ils affirmaient être au cœur de leur mission religieuse. Il faisait ainsi de l'assistance un service \emph{public} exercé par l'Église, ce qui donnait aux évêques le droit de réclamer aux autorités civiles des moyens à la mesure de leur mission de protecteurs des pauvres : dotations en argent, en domaines%
%[3] 
\footnote{... sur le statut juridique et fiscal desquels les discussions ne semblent pas terminées (cf. Jean \fsc{DURLIAT}, \emph{De l'Antiquité au Moyen Âge, l'Occident de 313 à 800}, 2002).} 
et en bâtiments%
%[4]
\footnote{En dépit du respect manifesté aux responsables ecclésiastiques, il s'agissait de moyens fournis par les autorités pour une mission et non de dons sans contrepartie, et il n'en sera jamais autrement. Les autorités civiles se sentiront toujours un droit de regard sur les moyens alloués aux évêques, de la même façon que les autorités des cités antiques n'ont jamais hésité à dépouiller les temples civiques de leurs trésors en cas de nécessité. Si Constantin a transféré une grande partie des biens des temples païens aux églises, c'est parce qu'il y voyait l'intérêt de son État. En dépit des protestations des clercs, les autorités civiles ne renonceront jamais longtemps à reprendre les moyens à eux confiés pour les affecter à d'autres fins lorsque cela leur paraitra aller dans le sens de l'intérêt général dont elles sont comptables.}% 
. Cela leur donnait un outil de conquête des esprits%
%[5]
\footnote{Lorsque Julien (empereur de 361 à 363) a tenté de restaurer les religions traditionnelles de l'empire, il a cherché à mettre en place un clergé païen hiérarchisé sous sa direction (il était \latin{pontifex maximus}) et il a voulu l'astreindre à fournir une véritable assistance à partir des temples, afin d'enlever aux chrétiens l'exclusivité d'un outil de séduction dont ils se servaient efficacement depuis leur apparition sur la scène religieuse antique.}% 
. 

\item[326] Le concubinage est interdit aux hommes \emph{mariés}. Cela n'aurait été qu'une décision symbolique sans grande portée s'il n'y avait eu à côté d'elle un ensemble de dispositions qui faisaient que désormais entretenir une concubine alors qu'on est marié présentait beaucoup moins d'intérêt et beaucoup plus d'inconvénients qu'auparavant : le principal de ces inconvénients était que les enfants nés des concubines des hommes mariés ne pouvaient plus être légitimés : ils étaient désormais considérés comme des enfants adultères. Ils ne pouvaient donc ni hériter de leur père ni lui succéder. Cela leur interdisait%
%[6] 
\footnote{… en théorie du moins, mais il y aura assez souvent des passe-droits au fil des siècles, avec des périodes très strictes et des périodes très tolérantes. Ceci dit cette règle va demeurer jusqu'au \siecle{20}.} 
de remplacer en cas de nécessité les héritiers légitimes, espérés en vain, ou décédés. 

% 326 
Les hommes qui n'ont pas d'épouse vivante ni d'enfants légitimes, mais qui ont une concubine ingénue (non esclave et non affranchie), de bonne réputation (non infâme, fidèle, non issue de la prostitution), et qui ont eu des enfants de cette femme, sont invités à l'épouser : s'ils le font les enfants qu'ils ont eu en commun avant le mariage seront reconnus comme légitimes. Dans tous les autres cas la légitimation des enfants illégitimes est interdite. Cette mesure a d'abord été prévue pour une période de transition d'une année, pour apurer le passé : dans un monde idéal il ne devait plus naître à partir de cette date aucun enfant illégitime. Face à la résistance des réalités elle sera renouvelée à plusieurs reprises jusqu'à ce qu'elle devienne permanente moins d'un siècle plus tard. C'est la \emph{légitimation par mariage subséquent}.

% 326 : 
Il n'était pas plus question pour Constantin, en dépit des pressions éventuelles de l'Église, que pour aucun de ses prédécesseurs de sanctionner les maris pour leurs propres infidélités, sauf quand ils avaient des relations avec la femme d'un autre. Par contre il abroge la loi d'Auguste contre l'adultère des femmes. Il décide qu'une femme adultère ne peut plus être dénoncée par son propre père (ce qui décharge ce dernier de l'obligation qui lui était faite de la dénoncer), ni par les étrangers (à qui les dénonciations rapportaient une part significative des fortunes confisquées aux deux coupables par le fisc impérial). Désormais seul le mari, et ses proches (cousin, beau-frère et frère), peuvent dénoncer les amants, mais ils n'y sont pas obligés, et en ce cas le mari règle l'affaire comme il règlerait n'importe quel autre conflit domestique. S'agissait-il pour Constantin de supprimer les dénonciations calomnieuses ? Ou de permettre au mari lésé de pardonner comme le demandait l'Église ? Par contre lorsque le mari choisissait de traîner sa femme en justice la peine maximale n'était plus comme auparavant l'infamie et l'exil, mais la mort des deux complices. En l'absence de données suffisantes on ne sait ni quel était le degré réel de répression des adultères avant Constantin, ni combien de coupables ont effectivement subi la peine qu'il avait prévue en cas de dénonciation par le mari. 

\item[331] La répudiation (rupture unilatérale du mariage, par opposition au divorce par consentement mutuel) devient un délit. L'épouse qui prend l'initiative de divorcer est condamnée à l'exil (assignée à résidence loin de chez elle), ce qui lui interdit le remariage, et elle perd une part substantielle de sa dot. Un homme qui répudie sa femme doit lui rendre l'intégralité de sa dot et ne peut plus se remarier non plus (mais il n'est pas exilé). La répudiation reste néanmoins autorisée, et l'époux(se) innocent(e) peut se remarier, dans les cas suivants :
\begin{enumerate}[leftmargin=*,itemsep=0pt]
% a)
\item si le mari est condamné à une peine infamante, ou s'il devient esclave, ou s'il est condamné comme homicide, violateur de sépulture ou empoisonneur ;
% b)
\item si la femme est convaincue d'être adultère, empoisonneuse ou entremetteuse.
 \end{enumerate}
Ces dispositions concernaient tous les citoyens, chrétiens ou non. Pendant ce temps-là les divorces par consentement mutuel restaient possibles, et dans ce cas les remariages l'étaient aussi. Les lois de l'Église ne concernaient pour le moment que les chrétiens, qui ne constituaient pas encore l'ensemble de la population. Moins d'un siècle plus tard un empereur ordonnera à tous les païens de se faire baptiser. Le remariage après divorce sera désormais \emph{en principe} interdit à tous les citoyens de l'Empire, sauf aux juifs%
%[7]
\footnote{Ceux-ci ne seront jamais interdits de remariage jusqu'à la Restauration (\siecle{19}).}% 
.

\item[334] Constantin interdit de séparer les membres d'une même famille pour les vendre comme esclaves, interdiction qui implique que de telles ventes se faisaient encore. Ce décret reprend à son compte des interdits déjà formulés au \siecle{3}. Son existence montre que les unions des esclaves, même si elles n'étaient pas reconnues comme de vrais mariages, étaient alors perçues de manière suffisamment positive pour créer des droits opposables aux maîtres qui les avaient autorisées. 

\item[336] L'Église avait toujours défendu la légitimité de tous les mariages librement voulus entre deux personnes non parentes, quel que soient leurs statuts légaux (esclaves, affranchis, citoyens, chevaliers, sénateurs...). Cela n'empêche pas Constantin de rappeler l'interdit de reconnaître les enfants nés des unions traditionnellement interdites par la loi : union d'un citoyen (tous les hommes libres depuis 212) avec un infâme, d'un sénateur avec une affranchie ou avec une esclave, d'un citoyen avec une esclave. Jusqu'à cette date les autorités pouvaient malgré tout les déclarer légitimes. Cette décision interdit en principe de le faire. 

\item[342] Les interdits de mariage traditionnels romains sont réaffirmés et le \emph{senatus-consulte} autorisant un oncle paternel à épouser sa nièce est abrogé.

 Si le Droit est l'expression des mœurs, alors il nous faut croire que Constantin a aligné le Droit romain sur les mœurs de son temps. Doit-on en déduire que celles-ci étaient déjà chrétiennes (ou christiano-stoïciennes) bien avant qu'il ne conquière le pouvoir ? Cela signifierait que les limites et interdits que Constantin a opposés à l'expression libre et spontanée des désirs sexuels et la canalisation de ces désirs sur le seul mariage monogame et fidèle faisaient déjà partie de l'idéal moral de son temps, païens et chrétiens confondus%
% [8]
\footnote{Aline \fsc{ROUSSELLE}, \emph{La contamination spirituelle, science, droit et religion dans l'Antiquité}, 1998.} 
 ? Nous avons aujourd'hui du mal à imaginer une telle situation tant nous paraît improbable un mouvement qui irait spontanément d'un niveau élevé de liberté sexuelle et matrimoniale vers un verrouillage du mariage et une limitation drastique des possibilités d'obtenir des enfants légitimes et des héritiers ... à moins que l'impression de grande liberté et d'aisance que suggère ce que l'on croit connaître de la vie des grecs et des romains de l'Antiquité ne corresponde pas à la réalité vécue, au moins par les dépendants (femmes, mineurs, esclaves)%
% [9]
\footnote{Cf. David \fsc{BROWN}, \emph{Le renoncement à la chair}, 2002} 
 ? 

 Mais une autre hypothèse est tout aussi vraisemblable. Les lois peuvent être une déclaration d'intention. Elles peuvent être chargées de désigner le bien, et le Droit peut être un instrument de normalisation des comportements et de remodelage des représentations. En ce sens les lois qu'ont édictées les empereurs chrétiens ont eu pour objectif d'orienter les comportements de leurs sujets dans le sens qui leur convenait, sans attendre qu'ils se soient tous convertis. 

 Au fil des deux siècles suivants les décisions fondatrices de Constantin ont été complétées par ses successeurs : 

\item[374] Valentinien décrète que les parents doivent subvenir aux besoins de tous leurs enfants, légitimes ou non. \emph{L'abandon est interdit}. Ce texte, le premier du genre, ne prévoit en fait aucune sanction. Il se contente d'affirmer un principe, de dénier aux pères le droit à l'abandon de leurs nouveaux-nés \emph{s'ils ont les moyens de l'élever}. Il semble n'avoir jamais été utilisé à l'encontre de parents incapables de nourrir leurs enfants, du moment que la vie de ceux-ci n'était pas mise en danger, et que leur découverte avait été facilitée (enfant placé en évidence, protégé des intempéries et surtout des animaux errants, exposé dans un lieu où passent beaucoup de personnes,~etc.). 

\item[384] $\!$ou \textbf{385}\,{} L'empereur Théodose condamne le mariage entre cousins germains. Vingt années plus tard, l'empereur d'Orient Arcadius lève cet interdit dans les territoires qui relèvent de son autorité. 

\item[390] \emph{L'empereur accorde aux veuves le droit d'exercer la tutelle de leurs propres enfants mineurs}. Par ce biais est pour la première fois reconnue à des femmes une pleine capacité à représenter autrui, même si les conditions de cette reconnaissance sont précises et limitées :
\begin{enumerate}[leftmargin=*,itemsep=0pt]
% a)
\item seuls sont concernés leurs propres enfants mineurs,
% b)
\item il leur faut avoir atteint cinquante ans, âge où elles ne pouvaient plus espérer d'autres grossesses, et
% c)
\item elles doivent promettre de ne pas se remarier : ce faisant elles retomberaient en effet dans la main, sous la coupe d'un homme à qui elles seraient obligées d'obéir, au risque de nuire aux enfants nés de leur union avec le conjoint décédé%
%[11]
\footnote{Il n'est pas sûr que cette mesure ait porté sur de très grands nombres de personnes, compte tenu du fait que les veuves âgées de plus de cinquante ans et ayant encore des enfants mineurs (moins de 25 ans) ne devaient pas être très nombreuses, puisque les femmes commençaient souvent d'avoir leurs enfants très tôt, bien avant leurs 20 ans. C'était néanmoins un pas important vers la reconnaissance du principe de l'égalité juridique des époux.}% 
.
\end{enumerate}

 Théodose II (empereur d'Orient de 408 à 450) ordonne qu'un procès-verbal de découverte soit rédigé pour chaque enfant trouvé. C'est la première fois que ces enfants reçoivent une reconnaissance administrative (et une forme minimale d'existence légale) avant même qu'une personne n'accepte de répondre d'eux et ne leur donne statut de citoyen en les déclarant comme libres. Cela signifie peut-être que désormais le souverain en prend possession même s'il les confie immédiatement à l'Église ? 

\item[438] Le Code Théodosien étend les interdits de mariage aux cousins germains. Cette règle ne sera pas acceptée et ne sera pas reprise par le Code de Justinien, mais elle préfigure la législation des siècles suivants. Le Code Théodosien interdit également les mariages entre beaux-frères et belles-sœurs.

\item[442] Le Concile de Vaison et celui d'Arles (\textbf{452})
%[12] 
décident%
\footnote{Comme il est de règle jusqu'à l'an mil, ces deux conciles ont été convoqués par les autorités civiles, et leurs décisions ont été promulguées par ces mêmes autorités.} 
(ou rappellent%
%[13]
\footnote{Probablement une fois de plus s'agissait-il avec ces décisions de généraliser des mesures déjà largement expérimentées.} 
 ?) que l'enfant exposé sera porté à l'église sur le territoire de laquelle il a été trouvé et qu'il y sera enregistré. \emph{Le dimanche suivant, le prêtre annoncera aux fidèles qu'un nouveau-né a été trouvé, et dix jours seront accordés aux parents pour reconnaître et réclamer leur enfant}. S'ils ne se manifestaient pas dans ce délai l'enfant devait être remis à titre onéreux, et non pas donné, à celui qui se proposait de le prendre en charge. À défaut d'un laïc volontaire pour le prendre (pour l'acheter ?) l'enfant pouvait (devait ?) être mis en nourrice aux frais de la communauté ecclésiale. 

 Le code de Justinien est une compilation du Droit romain réalisée au \siecle{6} sous la direction de cet empereur de Constantinople. Depuis Constantin, l'ancien droit de vie et de mort paternel \latin{(jus vitae necisque)} n'existait plus. Tout père qui tuait volontairement son enfant, même à sa naissance, même non encore né (avortement provoqué) était passible de la peine de mort. Dans le code de Justinien le père conserve le droit de correction paternelle, mais il n'a plus le droit d'infliger de graves châtiments corporels, de blesser ni d'estropier. S'il juge nécessaire de recourir à des châtiments sévères il doit s'adresser au gouverneur de la province ou au préfet de la ville%
% [14]
\footnote{Jean \fsc{IMBERT}, \emph{Le droit antique}, Que sais-je, 1961, p.93.} 
 : c'est la doctrine juridique qui va perdurer jusqu'au \siecle{20}. 

\item[529] Justinien décrète que deux amant adultères n'auront jamais le droit de s'épouser même si venait à décéder l'époux (ou les époux) qui leur faisait obstacle ;

% 529 :
% Il décrète 
Et qu'une femme convaincue d'adultère sera condamnée à vivre dans un couvent de femmes%
% [15] 
\footnote{Jusqu'à la Révolution, et en fait jusqu'au \siecle{20}, les incarcérations de femmes se feront dans des monastères pour femmes ou dans des lieux inspirés de ce modèle.} 
jusqu'à sa mort ... si du moins son époux porte plainte, mais rien n'oblige ce dernier à le faire. Compte tenu du fait que les textes antérieurs (ceux de Constantin) donnaient à l'époux droit de vie et de mort, c'est un adoucissement majeur. Il pourra se séparer de sa femme adultère, mais ne pourra pas se remarier. Cette règle de droit ne concerne que les chrétiens mais à cette date cela fait longtemps que tous les citoyens, à l'exception des juifs, ont reçu l'ordre de se faire chrétiens. Par conséquent s'il veut avoir des enfants légitimes, des héritiers, un époux bafoué n'a plus d'autre choix (en principe) que de se réconcilier avec sa femme et de reprendre la vie commune, ce qui eut été le comble de l'indécence ou de l'infamie deux siècles plus tôt. En dépit de l'infidélité passée de celle-ci il est même expressément invité à lui pardonner au bout d'un certain temps de réclusion dans un couvent pour femmes (2 ans au maximum ?). Autant dire que son intérêt n'est pas forcément de porter le cas de son épouse coupable devant les tribunaux. L'adultère féminin devient de plus en plus une affaire privée, même si le pouvoir civil continue et continuera jusqu'à la fin de l'ancien régime de prêter la main au mari pour soutenir son droit de correction marital. 

\item[533] Les \latin{institutes} de Justinien réaffirment la légitimité du mariage entre cousins, ou celui d'un veuf ou d'une veuve avec le frère ou la sœur de son conjoint décédé. Cela montre la résistance des autorités civiles à l'autorité morale de l'Église. Sur ce point précis l'Orient refusait la position intransigeante de l'église de Rome face à tout ce qui ressemble à l'inceste%
% [10]
\footnote{Cf. Jack \fsc{GOODY}, \emph{L'évolution de la famille et du mariage en Europe}, p. 66.}%
.

%\item[534] Le Code de Justinien décrète que les enfants adultérins et incestueux n'ont aucun droit, au contraire des autres enfants illégitimes : il leur dénie tout droit à des aliments, bien qu'il ne soit pas interdit à leurs
\item[534] Le Code de Justinien dénie tout droit aux enfants adultérins et incestueux, au contraire des autres enfants illégitimes. Ils n'ont pas droit à des aliments, bien qu'il ne soit pas interdit à leurs
géniteurs de leur en donner. Ils ne peuvent en aucun cas être légitimés. Au nom de la sauvegarde de l'institution familiale, cette décision retire à ces enfants les protections que leurs géniteurs pourraient vouloir leur donner. Ils sont réputés \emph{enfants trouvés}, et traités comme ces derniers par les institutions chargées de s'en occuper. 

 Par ailleurs le code de Justinien confirme \emph{l'interdiction de l'adrogation des enfants illégitimes}. Les « enfants du péché » (le péché des adultes contre l'institution familiale) sont désormais à écarter. Cela contraste avec la volonté de protection de tous les enfants qui se traduisait dans les autres décisions de l'époque, mais il est clair que l'objectif de ces lois était d'abord de prévenir la naissance de ces enfants. La suite de l'histoire montrera que cet objectif a été à peu près atteint en dépit d'un « reste » incompressible (seulement ... ou encore, un pour cent de naissances illégitimes dans diverses régions rurales françaises sous Louis~XIV). 

 Le code de Justinien confie les enfants trouvés à l'évêque du lieu de leur découverte. Il ordonne aux magistrats de s'en tenir au principe juridique que les enfants trouvés sont tous nés libres et non esclaves, même lorsque leur mère est une esclave : l'abandon donne donc aux enfants d'esclave une chance de vivre libres. 

 En l'absence d'enfants légitimes le code de Justinien autorise les enfants naturels nés d'un concubinage stable à hériter de leurs parents : il confirme ainsi le concubinage stable dans son statut de mariage à l'usage des pauvres et des humbles. 
\end{description}




