% Le 19 mars 2015 :
% ~etc.
% Moyen Âge
% Antiquité
% " --> « ou » ou \enquote{}
% même


\chapter[Démantèlement de la famille traditionnelle]{Démantèlement\\de la famille traditionnelle}


 

\section{Révolution dans le droit}

\begin{description}

\item[1961] Une mesure administrative qui sur le moment n'a pas frappé beaucoup d'esprits, mais dont l'importance symbolique n'en est pas moins significative (les grandes fractures commencent souvent par une fissure imperceptible à l'œil nu) : le ministère de l'éducation nationale supprime le caractère obligatoire de l'enseignement du droit romain dans le programme de la licence de droit, obligation qui datait de la création des études de droit dans les universités aux \crmieme{11} et \siecle{12}s. Ce corpus n'est plus qu'une option facultative parmi d'autres. 

\item[1965] La loi du 13 juillet lève les derniers obstacles à l'exercice d'une activité commerciale par les femmes mariées sans la tutelle de leurs maris. Ceux-ci ne gèrent plus de droit les biens propres de leurs épouses (dot,~etc.). Elles n'ont plus à obtenir leur autorisation pour exercer une profession séparée, quelle qu'elle soit.

\item[1966] La loi du 11 juillet sur l'adoption assimile les enfants adoptés aux enfants légitimes non adoptés (adoption plénière). 

%\item[1966] la (même)
Cette
loi du 11 juillet ouvre le droit à l'adoption plénière à une personne seule, qu'elle soit célibataire ou non et quelles que soient ses préférences sexuelles, d'au moins 28 ans.

\item[1967] La loi \fsc{NEUWIRTH} dépénalise la prévention des naissances : elle autorise la publicité concernant les méthodes anticonceptionnelles (interdite depuis les années 20), et elle autorise leur mise à disposition du public :
%la première visée et la principale était la « pilule » anticonceptionnelle 
la « pilule » anticonceptionnelle était principalement visée,
qui venait d'être mise au point. L'accord du mari n'est pas nécessaire, son refus n'a pas d'effet.

\item[1972] La loi du 3 janvier fait entrer les enfants naturels dans la famille du ou des parents qui les ont reconnus. À quelques restrictions près -- enfants adultérins, elle leur ouvre un droit à l'héritage égal à celui des enfants légitimes.

%\item[1972] 
La puissance paternelle est abolie au profit de l'autorité parentale. En cas de séparation, cette autorité est conservée à égalité par chacun des deux parents. 

%\item[1972] 
Une loi ordonne l'égalité des salaires féminins avec les salaires masculins.

\item[1974] L'âge de la majorité légale est abaissé de 21 à 18 ans. 

\item[1975] Loi du 30 juin relative aux institutions sociales et médicosociale : les usagers et les familles doivent être associés au fonctionnement de l'établissement qui les prend en charge (il doit les « prendre en compte » ).  Le pa(ma)ternalisme des institutions d'aide sociale est un peu affaibli.

%\item[1975] 
À côté du divorce pour faute, la loi du 11 juillet ouvre la possibilité de divorcer par consentement mutuel ou pour rupture de la vie commune. Par ailleurs, cette loi met les deux époux à égalité en matière de choix résidentiel et en matière de contribution aux charges du mariage.

%\item[1975] 
La loi \fsc{WEILL} dépénalise l'avortement (\emph{interruption volontaire de grossesse} ou IVG). La loi ne demande pas l'avis des maris éventuels.

%\item[1975] 
Les épouses ne sont plus tenues de faire usage du nom de leur mari dans la vie quotidienne et les relations avec l'administration.

%\item[1975] 
L'adultère féminin est dépénalisé. Ce n'est plus un délit qui concerne la société, ce n'est qu'un affront privé qui ne concerne que le mari.

\item[1976] Loi du 22 décembre relative aux conditions d'adoption : la présence d'enfants légitimes n'est plus un obstacle à l'adoption, même si leur avis est entendu. 

\item[1978] La loi du 6 janvier donne à tout individu majeur le droit de connaître le contenu de tout dossier administratif le concernant. Cela concerne notamment tous les enfants abandonnés.

\item[1983] Un arrêt de la cour de cassation du 21 mars 1983 admet la légalité de la garde conjointe de l'enfant après divorce.

%\item[1983] 
Loi sur l'égalité professionnelle entre femmes et hommes.

\item[1984] La loi du 6 juin relative aux \emph{droits des familles dans leurs rapports avec les services chargés de la protection de la famille et de l'enfance}, et au statut des pupilles de l'État, donne aux parents des droits plus étendus face à l'administration. L'autorité des parents sur leurs enfants placés à l'ASE est confortée dans tous les domaines (sauf limites définies expressément par un juge).

\item[1985] La loi du 23 décembre 1985 met les deux parents à égalité dans la gestion des biens de l'enfant : ils exercent cette tâche conjointement quand ils exercent en commun l'autorité parentale. Sinon l'un des deux l'exerce sous le contrôle du juge.

\item[1987] L'autorité parentale est redéfinie par la loi du 22 juillet (loi \fsc{MALHURET}) en termes de \emph{responsabilité parentale ordonnée à l'intérêt de l'enfant}. Elle est à égalité assumée par chacun des deux parents, qu'ils cohabitent ou pas.

\item[1989] \emph{Convention Internationale des Droits de l'Enfant} promulguée dans le cadre de l'ONU le 20 novembre : reconnaît le droit de tout mineur à une famille, et ses droits face à sa propre famille.

%\item[1989] 
Loi du 10 juillet \emph{relative à la prévention des mauvais traitements à l'égard des mineurs et à la protection de l'enfance}. Elle prévoit que le délai de prescription ne court qu'à partir de la majorité pour les mineurs victimes de violences.

\item[1993] L'autorité parentale conjointe devient la règle pour les couples de concubins comme pour les couples mariés.

\item[1996] Convention européenne du 25 janvier sur l'exercice des droits de l'enfant. Elle donne le droit aux enfants mineurs de donner leur avis sur les mesures qui les concernent lors du divorce de leurs parents.

\item[1999] Création du PACS : pacte civil de solidarité, ouvert aux couples hétérosexuels et aux couples homosexuels.

\item[2000] La pilule {\emph{du lendemain}} est en vente libre dans les pharmacies, et distribuée gratuitement aux mineures par les infirmières scolaires sur simple demande de la mineure, sans demander l'avis de ses parents, et sans qu'ils en soient informés.

\item[2002] Sur décision de la Cour Européenne de Justice les dernières discriminations juridiques que subissaient en matière d'héritage les enfants adultérins et incestueux sont effacées. Seuls sont distingués les enfants nés des incestes parent--enfant, qui ne peuvent être reconnus que par un seul de leurs deux parents. Ils doivent néanmoins être traités absolument en tout le reste, et d'abord en ce qui concerne l'héritage, comme leurs éventuels demi-frères ou sœurs. 

%\item[2002] 
Loi du 4 mars : {[...] \emph{les parents associent l'enfant aux décisions qui le concernent, selon son âge et son degré de maturité}}. L'administration de la famille par les deux parents doit être démocratique.

\item[2005] La loi autorise les femmes à donner à leurs enfants leur propre nom à égalité avec leur mari à compter de janvier 2005. 

%\item[2005] 
Interdiction du mariage des filles avant dix-huit ans (traditionnel âge au mariage des garçons).

\item[2013] Loi \fsc{Taubira} : ouverture du mariage aux couples de même sexe.

\item[2013] Remboursement à 100~\% de l'IVG.

\item[2014] Suppression de l'exigence d'une « détresse » pour reconnaître à une femme enceinte son droit à un avortement. 
\end{description}
 
 \section{Le corps des femmes est à elles}


 La pilule anticonceptionnelle ("la pilule") a été autorisée en France quelques années seulement après sa mise au point : en 1967. Et elle l'a été par une assemblée de députés dans laquelle il n'y avait pratiquement que des pères de famille, qui ont apparemment plus pensé aux intérêts de leurs épouses et de leurs filles qu'à la défense du patriarcat. Dès ce moment la pilule a été largement utilisée par toutes les femmes majeures, célibataires ou mariées, et par bien des mineures. Grâce à elle, les femmes pouvaient prendre l'initiative d'une rencontre sexuelle sans obérer leur avenir. Cela a permis de constater que si c'était sans risque de grossesse, bien des parents ne refusaient pas que leurs filles aient une vie sexuelle hors mariage. Il n'était plus nécessaire de donner une valeur à la virginité ou à la chasteté des femmes non mariées et les filles n'étaient plus contraintes par le risque de grossesse de fuir les garçons ni de nier, de réprimer ou de refouler leurs propres désirs sexuels
\footnote{En Janvier ou février 1968, les garçons de la cité universitaire de Nanterre réclament bruyamment le droit d'entrer dans les chambres des filles de la cité, jusque là sanctuaires (en principe) inviolés. Ce fait divers, grand-guignolesque à nos yeux d'aujourd'hui, n'en a pas moins servi de détonateur à la chaîne d'évènements mémorables qui ont culminé au mois de Mai de cette année-là. Il se trouve que la disponibilité (alors toute nouvelle, mais déjà répandue comme une trainée de poudre) de la pilule anticonceptionnelle, venait d'enlever à cette revendication une part du caractère scandaleux (à tout le moins angoissant pour les pères et mères des jeunes filles) qu'elle aurait eu peu de temps auparavant. Le sexe librement recherché pour lui-même devenait un jeu sans enjeu dramatique. Ce n'est que bien après ce printemps-là que le SIDA est venu lui rendre une gravité nouvelle.}% 
. Elles ont reçu une liberté égale à celle de leurs frères, et l'âge moyen de leurs premiers rapports sexuels (autant qu'on puisse le connaître) est rapidement passé de 21 à 17 ans (comme eux).

 La « pilule » a-t-elle été inventée dès que son emploi est apparu comme acceptable ? Ou bien est-ce plutôt le contraire ? On peut en effet se demander pourquoi les préservatifs, disponibles en vente libre en pharmacie depuis le début du \siecle{20} (officiellement en tant qu'outil de prévention contre les maladies vénériennes) n'ont pas été employés en France comme un outil de prévention des naissances, alors qu'ils l'étaient dans d'autres pays ? Et pourquoi la pilule ne s'est pas heurtée à la même réticence ? 

 C'est que ce sont les femmes qui ont la maîtrise de cet outil-là : les députés leur ont en effet accordé le droit de prendre la pilule anticonceptionnelle même en cas de désaccord avec leurs maris. Elles \emph{peuvent} la prendre sans le dire à leurs partenaires. Elles \emph{peuvent} aussi cesser de la prendre sans les prévenir. Et\emph{ ils n'y peuvent rien}. Avec l'appui de la législation et des pouvoirs publics, le Planning Familial, héritier des néo-malthusiens, s'emploie à rendre effective cette liberté pour toutes les femmes, mineures comme majeures. 

 C'est dans la foulée de cette première mesure que l'avortement a été autorisé par la loi. Il ne s'agissait plus de l'avortement à la romaine : celui de l'épouse ou de l'esclave sur l'ordre du \emph{pater familias}, ou avec son accord exprès. Désormais une femme peut prendre seule l'initiative d'un avortement, en dépit du refus de son compagnon, comme elle peut garder leur enfant même s'il lui demande d'avorter\footnote{Certes en accouchant « sous X",  mode d'accouchement aux origines très anciennes et ouvert aux femmes mariées autant qu'aux autres, puisqu'elles n'ont pas à donner leur identité, les femmes ont toujours pu priver de paternité, en toute légalité, le géniteur de l'enfant qu'elles portaient, mais il est peu vraisemblable que cela ait été leur objectif losqu'elles recouraient à cette procédure et l'immense majorité des accouchements sous X a concerné et concerne encore des femmes seules et sans soutien.}.
 
  \section{Personne n'est illégitime}
 
 Aujourd'hui tout se passe comme s'il n'existait plus que des enfants légitimes : tout enfant a vocation à faire partie de la famille de chacun de ses deux géniteurs quelles que soient les circonstances de sa conception. Tout enfant a vocation à hériter de ses deux parents à égalité avec ses éventuels demi-frères et demi-sœurs. Tout enfant est un « enfant de famille ». On peut aussi bien dire que tous les enfants sont devenus « naturels » et que la notion même de légitimité s'est évaporée, réduite à un mot sans épaisseur puisqu'il n'a plus de prise sur rien, puisque les effets concrets de la légitimité sont les mêmes que ceux de l'illégitimité, et inversement.
 \begin{displayquote}
\emph{"Cette distinction légitimité-illégitimité était totalement structurante de la société. Aujourd'hui il n'est pas un pays qui n'ait soit complètement aboli cette distinction, soit s'apprête à l'abolir. C'est un changement majeur des rapports entre famille et société qui montre que nous sommes face à des changements de la structure sociale elle-même"}.
\footnote{Irène \fsc{THERY}, \enquote{Peut-on parler d'une crise de la famille ? Un point de vue sociologique}, \emph{Neuropsychiatrie de l'enfance et de l'adolescence}, 2001, 49, 492-501, p. 403.}% } 
\end{displayquote}

La cheville qui depuis \nombre{1600} ans tenait ensemble tout le système de la famille constantinienne a été retirée, et cela se passe apparemment à la satisfaction de tous. Puisque ni les parents ni les enfants ne risquent plus aucun désagrément du fait d'une naissance illégitime, à quoi bon le mariage, surtout quand on est convaincu que le seul couple légitime c'est celui qui repose sur l'accord quotidien de deux volontés. Depuis plus d'une génération, le nombre d'enfants nés hors mariage a donc progressé en même temps que croissait leur assimilation aux enfants nés dans le mariage. Aujourd'hui ils représentent la moitié des naissances. 

 Dans la France d'aujourd'hui, l'illégitimité a cessé d'être honteuse et il n'est plus socialement utile que le nom que l'on porte atteste qu'on a été reconnu par un homme. Une loi de 2001 autorise (donc ?) les couples mariés à donner à leurs enfants le patronyme de la mère à la place de celui du père, ou bien à côté de lui (texte complété par la loi \fsc{Taubira} de 2013). Ce changement est significatif, puisque la pratique traditionnelle n'était pas celle-là (contrairement à l'Espagne, par exemple) tout comme est significatif le moment où il a été institué. 


 
 \section{Victoire du mariage d'amour}


 
 Aujourd'hui, à la condition de posséder une vraie qualification professionnelle (capital intellectuel), il n'est plus besoin de capitaux pour s'établir. Rien ne vaut un « bon » métier : un métier qui implique beaucoup de savoirs et de savoir-faire, dans un secteur d'activité porteur. Il n'est plus honteux de « servir ». Au fil du \siecle{20}, le salariat s'est souvent révélé plus sûr que la possession de capitaux ou d'outils de production, surtout au service de l'État. D'autre part la scolarité est désormais gratuite ou presque jusqu'aux niveaux les plus élevés (même si ce n'est souvent pas vrai aux niveaux les plus élevés). Les parents ont intégré cette logique : depuis la Libération, le taux de scolarisation n'a cessé de s'accroître bien au-delà de la fin de l'obligation scolaire, qui elle-même est passée de 14 (1936) à 16 ans (1959). Le nombre des diplômés de l'enseignement supérieur a explosé. Le nombre de bacheliers se situe actuellement entre 60~\% et 80~\% d'une classe d'âge, contre 5~\% à la Libération et 8~\% en 1960. Même si ce diplôme s'est largement dévalué et ne peut plus depuis longtemps procurer un emploi à lui seul, le niveau de culture moyen a indiscutablement progressé. 

 Aujourd'hui la rentabilité du travail domestique a été fortement réduite par les innovations techniques, commerciales et sociales du \siecle{20} : infrastructures collectives (électricité, tout à l'égout, eau courante) ; machines qui économisent le temps de travail (chauffage central, machine à laver le linge, la vaisselle, cuisines équipées électriques ou au gaz, aspirateur,~etc.) ; grande distribution qui rend non-compétitive l'auto production en couture, en jardinage vivrier, en préparation des aliments,~etc. Sans oublier les écoles maternelles et les garderies d'enfants. Si l'on peut dire que les ménagères ont été libérées d'une grande partie du poids des tâches domestiques, cela signifie aussi qu'elles été réduites au chômage technique, ce que traduit le fait que c'est au même moment qu'on observe la fin des « bonnes ». Pour contribuer significativement aux ressources de leur ménage les épouses doivent désormais travailler au dehors de leur foyer. Cela leur a ouvert la possibilité de se trouver un autre emploi que celui d'être la « maîtresse de maison » d'un homme mais on peut aussi bien dire que cela les  y a contraintes. 
 Elles y ont d'autant plus été contraintes que depuis la libéralisation du divorce elles ne peuvent plus être assurées, comme l'étaient leurs grand-mères, de leur position d'épouse titulaire d'un homme nommément désigné. Les filles de la bourgeoisie ont compris que leur avenir serait mieux assuré par un « bon » métier que par une « belle » dot, un « beau » parti, et par le « grand » mariage qui était jusqu'alors le point de focalisation de tous les désirs familiaux, le signe et le sommet de la réussite féminine (celle des filles et celle des mères). Toutes ont compris que grâce aux savoirs et aux diplômes elles seraient libres : indépendantes des désirs d'un homme et de sa bonne volonté, à l'abri des effets matériels des répudiations, en mesure de prendre l'initiative et de sortir des situations affectives ou familiales dans lesquelles elles ne trouveraient pas leur compte. Dans la course au diplôme les filles se sont (donc ?) montrées significativement plus déterminées que les garçons (quant aux décrochages scolaires de ces derniers, leurs causes n'ont probablement rien à voir avec les motifs de la détermination des filles). 

 S'il n'est plus nécessaire pour « s'établir » d'avoir l'appui financier ni des relations de ses parents, alors le mariage ne scelle plus l'alliance (économique surtout) de deux familles : alors rien n'exige plus que les jeunes gens subissent un mariage arrangé, un mariage d'argent et d'entregent. Ils peuvent sans risque \emph{matériel} s'offrir le luxe de n'être pas raisonnables et de baser leur couple sur la seule passion amoureuse : aujourd'hui les autres stratégies ne sont pas plus raisonnables que celle-là.

 C'est en tout cas tellement devenu notre logique que cela révolutionne notre compréhension du mariage lui-même. Si celui-ci se définit d'abord comme l'union de deux personnes qui s'aiment, alors la question de la durée perd de son sens. L'authenticité des désirs inscrits dans les actes posés ici et maintenant a plus d'importance que la fidélité à une promesse ancienne. L'infidélité conjugale n'est plus une offense à un ordre public qui ne se donne plus pour but de sanctuariser les familles. Ce n'est plus qu'une offense privée, le signe d'un désaccord entre deux associés. La séduction devient une obligation permanente. L'accord du conjoint à une relation charnelle ne peut plus être tenu pour acquis d'avance, par contrat. La notion de \emph{devoir conjugal} s'est vidée de son sens, et la loi ne le reconnaît plus. La notion de viol entre époux prend du sens, et comme tout viol c'est un délit punissable par la loi. 

 Si c'est l'amour mutuel qui fonde le couple, alors sa fécondité potentielle perd de son importance. Que le couple soit constitué d'un homme et d'une femme ne va plus sans le dire. La reconnaissance publique d'un couple de deux hommes ou de deux femmes n'est plus impossible à penser. 

 Mais si c'est l'enfant qui fait la famille, et s'il héritera de ses deux parents quoi qu'il arrive, alors à quoi bon se marier ?

 
 \section{Nouveaux jugements sur les violences sexuelles}

L'abondance actuelle, depuis les années 1985-1990, des discours sur les \emph{abus} sexuels intra familiaux 
\footnote{... comme s'il y avait un usage correct du sexe entre les générations différentes au sein des familles ?} 
signifie-t-elle qu'il s'en commet plus qu'autrefois ? Si l'on en croit le témoignage de Jeannine \fsc{NOEL} (1965) il est permis d'en douter : selon elle entre le quart et le tiers des adolescentes placées à l'Hôpital Hospice Saint Vincent de Paul
\footnote{À cette époque c'était encore le Foyer de l'Enfance de Paris (anciennement « dépôt de l'Assistance Publique ») recevant (souvent avant une orientation ailleurs) tous les jeunes dont les parents ne pouvaient pas s'occuper ou de l'autorité desquels ils avaient été soustraits par décision de justice. On plaçait et place toujours dans les foyers de l'enfance les jeunes qui n'ont pas d'autre lieu où aller, quelle que soit la raison qui les a mis dans cette situation.} 
 au cours des années cinquante du \siecle{20} avaient été confrontées à des problèmes de ce genre : la situation ne semble pas être pire aujourd'hui. 
 
Par contre depuis un demi-siècle toutes les formes de violences sexuelles, qu'elles soient extra ou intra familiales, ont été regardées avec un oeil nouveau. Meme si ce n'est que très progressivement que l'on a pris conscience de la gravité de leurs effets sur leurs victimes il nous est devenu moins difficile de nous identifier aux souffrances de celles-ci. Cela s'est traduit par la requalification de certaines actes délictueux, et surtout par une nouvelle façon d'écouter les plaignants et plaignantes et d'accorder crédit à leur parole.  
 

 Alors pourquoi n'est-ce qu'aujourd'hui que le caractère absolu des secrets professionnels imposé aux professionnels susceptibles de découvrir des violences sexuelles sur mineurs a été mis en question ? Pourquoi n'est-ce qu'aujourd'hui que l'évocation des sévices intra familiaux obtient un tel effet ? Avait-on peur d'ébranler l'autorité et la représentation d'une institution familiale sacralisée, et préférait-on lui sacrifier ses victimes ? Pensait-on que ces délits et ces crimes, aussi condamnables qu'ils étaient, ne pouvaient être traités pénalement, et qu'il était préférable de les recouvrir du \emph{manteau de Noé} ? 
 
 
 \section{Désarroi masculin}


 
S'il veut une femme et/ou des enfants un homme ne peut plus s'y prendre aujourd'hui comme naguère. Il ne lui sert plus à rien de demander à un futur beau-père la main de sa fille, de lui demander un transfert d'autorité, puisque ce dernier ne la détient plus et ne peut donc plus la donner. D'ailleurs lui-même n'a plus besoin d'un gendre pour légitimer les petits enfants que sa fille lui donnera et pour en faire des héritiers, puisqu'il n'y a plus de fonctions interdites aux enfants illégitimes et donc plus d'enfants illégitimes. Il n'y a donc plus d'intérêt commun entre beau-père et gendre, et le soupirant doit négocier seul et sans intermédiaire avec la femme dont il recherche les faveurs. Il n'aura d'elle des enfants que si elle le veut bien. Et elle pourra d'autant plus facilement le quitter en emmenant leurs enfants communs (ou le pousser hors du domicile familial) que l'absence d'un homme à côté d'une mère ne fait plus problème, tandis que la présence de celle-ci semble encore indispensable\footnote{Cela changera peut-être si on constate des aptitudes au "maternage" chez les pères célibataires ou les couples homosexuels masculins ?}. 
 
 Les ressources dont disposent les hommes (puissance économique, compétences culturelles et professionnelles, pouvoir politique,~etc.) ont la vertu de les rendre désirables. Ils se doivent comme toujours d'être « ceux qui peuvent », ceux qui ne sont pas marqués par le manque ou la défaillance. Plus ils sont intellectuellement et professionnellement qualifiés, plus ils ont de probabilités d'être mariés. C'est le contraire pour les femmes, ce qui suggère que dès qu'elles ont les moyens de leur indépendance elles n'ont plus intérêt à être mariées. Cela confirme la solidité de la répartition traditionnelle des rôles masculins et féminins.
 
 On a vu que « l'obligation de résultat », l'obligation de fécondité, qui pesait sur les seules femmes mariées a été supprimée par Constantin, qui a exclu la stérilité des motifs de divorce. La loi impériale romaine a ensuite confirmé l'interdit fait aux chrétiens de se remarier après divorce. À l'obligation de fécondité des épouses s'est substituée une obligation de moyens pour chacun des deux époux : ne pas mettre d'obstacle aux fécondations autre que l'abstinence 
\footnote{En France les relevés démographiques montrent l'érosion progressive du respect de cette obligation, et l'extension depuis trois siècles des pratiques anticonceptionnelles : ce que les anciens moralistes nommaient les « \emph{funestes secrets} ».}. Si les femmes mariées ont ainsi été protégées contre la répudiation et contre la privation de leurs enfants, par contre la loi ne les autorisait pas plus qu'avant à se dérober au « devoir conjugal » lorsque leur mari l'exigeait, ni aux grossesses qui en découleraient, et à leurs risques, sauf à demander une séparation. 

 Depuis 1967, même si leurs maris le désirent, les femmes mariées ne sont plus tenues par la loi de laisser libre cours à leur fécondité. Aujourd'hui le corps des femmes est à elles, \emph{y compris l'embryon ou le fœtus, qui juridiquement en fait partie depuis 1975}, comme c'était le cas dans le droit romain antique. La loi ne se soucie plus de soutenir le désir masculin en ce domaine. Même si elles sont leurs épouses, même s'ils sont les géniteurs de l'enfant qu'elles portent, même si elles avaient été d'accord pour le concevoir avec eux, les hommes n'ont plus le droit d'exiger des femmes qu'elles donnent naissance à cet enfant. Elles peuvent choisir d'avorter ou de l'abandonner à la naissance en dépit du désir du géniteur de l'enfant. On est au plus loin du droit du \emph{pater familias} romain de faire surveiller la grossesse et l'accouchement de son épouse (ou ex épouse), pour qu'elle ne puisse pas lui dérober un enfant né de ses œuvres.

 Dans le même temps ont été supprimées toutes les limites légales qui pouvaient interdire le rattachement d'un enfant naturel à un homme. Comme sous l'ancien régime une mère qui le demande recevra toujours l'appui de la justice pour rechercher le géniteur de son enfant (sauf insémination avec donneur), quelle que soit la situation personnelle de cet homme, mais désormais cela se fera avec une efficacité  imparable. Aucun père n'est plus « \emph{incertus} ». Vivant ou mort son ADN le désignera, sauf lorsque la mère veut cacher son identité à son enfant ou aux tiers. Si la mère le veut, le géniteur sera contraint d'assumer financièrement un enfant qui héritera de lui à part entière, contrairement à ce qui se passait jusqu'au \siecle{19}. Mais cela ne lui donnera pas forcément le moindre droit sur l'éducation de l'enfant : en ce sens cela n'en fera pas un père. Si une femme qui accouche « sous X » refuse de laisser à son enfant des renseignements sur sa propre identité, elle en a le droit. 

 Pour l'essentiel, on peut donc dire que la maîtrise de la génération est passée du côté des femmes. La famille monoparentale d'aujourd'hui, c'est le plus souvent la famille \emph{moins} le père. Dans la grande majorité des séparations (85~\%) ce sont les mères qui gardent les enfants. Est-ce pour ces raisons que l'initiative des divorces vient des femmes beaucoup plus souvent (trois fois sur quatre) que des hommes  ? 




  \section{Police des familles ?}
 
 
Selon Jacques \fsc{Donzelot} (\emph{La police des familles}, 1977), nous sommes passés du gouvernement « des » familles au gouvernement « par les » familles.  Aujourd'hui le pouvoir royal des pères sur leurs enfants est mort, et celui des mères en même temps et ils ne leur reste plus que celui que l'état leur concède, à la condition qu'ils se conforment aux modèles promus par ce dernier.

\begin{displayquote}
{\emph{Ce qui caractérise la loi de 1970 (qui substitue l'autorité parentale à la puissance paternelle) ce sont trois concepts au centre de la réforme, celui « d'égalité » des époux et parents, celui « d'intérêt de l'enfant » et enfin celui de « contrôle judiciaire » devenu nécessaire pour arbitrer d'éventuels conflits entre les parents, entre parents et enfants. On assiste à un recentrage des positions de chacun des membres de la famille. Au centre l'enfant, en face de lui, responsables de lui, ses parents. Entre les deux des médiateurs, les spécialistes judiciaires}%
\footnote{Françoise \fsc{HURSTEL}, \emph{La déchirure paternelle}, p. 117.}%. 
}.
\end{displayquote}

Nous avons assisté à la délégitimation de la justice domestique, du droit des deux parents à régler eux-mêmes sans tiers extérieur tous les conflits intra familiaux. Lorsqu'ils ne réussissent pas à se mettre d'accord entre eux ou avec leurs enfants, ils sont désormais contraints (par leur égalité elle-même) à recourir à un tiers extérieur pour arbitrer leur différend.  

De nouveaux personnages se sont imposés au sein des familles. Sous l'autorité des juges les travailleurs sociaux et les experts (psychologues, psychiatres, médiateurs,~etc.) sont entrés dans le champ, jusque là bien clos, des familles ordinaires, des familles non stigmatisées au préalable comme défaillantes (en ce qui concerne les familles reconnues officiellement comme incompétentes ou délinquantes, c'est depuis toujours que les représentants de la société y avaient leurs entrées). Ils font régner la bonne parole et les bonnes pratiques et vérifient que les familles adoptent les bonnes pratiques dans la prise en charge de leurs enfants, pratiques définies par les mêmes personnages. Quand ils l'estiment nécessaire ils ont l'appui des autorités pour faire passer le message.


 

 Assiste-t-on à la disparition de la sphère privée, cette sphère de la vie de chacun qui se définit par le fait que tant qu'il n'enfreint aucune loi, il n'a aucun compte à rendre sur ce qui s'y passe, et surtout pas à l'État et à ses représentants ? 

 Tout ce qui concerne les enfants est-il entré dans le domaine public, alignant le traitement de l'ensemble des familles sur celui qui était autrefois réservé aux seuls « cas sociaux », et mettant implicitement en cause l'aptitude des parents à défendre suffisamment bien (en \enquote{\emph{bons pères de famille}}) l'intérêt de leurs propres enfants ?



\section{Inertie des pratiques}


 Dans la réalité les changements ne sont pas (encore ?) aussi importants que dans l'idée que l'on s'en fait. Depuis une génération le nombre de mariage diminue innexorablement : en 1990, 90~\% des couples existants étaient mariés, en 1999, année où le Pacs est entré dans les pratiques ils n'étaient plus que 83~\% (\emph{Histoires de familles, histoires familiales}, INSEE, 1999). 
 Le Pacs, à l'origine pensé pour les couples homosexuels, est en réalité le plus souvent choisi par des couples mixtes (dix-neuf pacs sur vingt sont contractés par eux), dont près de la moitié finit par se marier, et la somme des Pacs et des mariages est plus élevée que le nombre des seuls mariages avant la création du Pacs. Le lien entre naissances et mariage semble solide : à la naissance du deuxième enfant 86~\% des couples sont mariés, et 93~\% au troisième. On est donc, avec le Pacs, tout près d'un mariage à l'essai.

 Le nombre des divorces se situe aujourd'hui entre le tiers et la moitié de celui des mariages. Ce nombre est élevé ou bas suivant le point de vue. Sur 29~millions d'adultes vivant en couple, mariés ou non, 26~millions (90~\%) en sont \emph{toujours} à leur première expérience de couple, et pour l'instant les recompositions de familles concernent \emph{seulement} 3~millions de personnes sur 29. C'est que le nombre de couples mariés de tous âges (le « stock ») est si important que les divorces n'en représentent pas plus de 1~\% par an : 99~\% des gens qui étaient mariés au premier janvier le sont encore au 31 décembre qui suit (mais qu'en sera-t-il de ces chiffres dans une génération ?). 

 En 2006, 1,2~millions de mineurs vivaient en famille recomposée, soit 9~\% de l'ensemble des mineurs. Parmi ces mineurs, \nombre{400000} sont nés des deux membres du nouveau couple. Ceux-là vivaient donc avec leurs deux parents, bien que dans une famille "recomposée". À la même date, 2,2~millions de mineurs vivaient au sein d'une famille monoparentale (six fois sur sept avec leur mère), tandis que 10,25~millions de mineurs
vivaient avec leur père et leur mère (dont les \nombre{400000} enfants vivant au sein de familles recomposées et nés du couple nouveau). 
 


\newlength{\lcol}
\setlength{\lcol}{0.16666667\textwidth}
\addtolength{\lcol}{-2\tabcolsep}


\begin{table}[!ht]% [!htb]
%\centering
\begin{minipage}{\textwidth} 
\caption[Cadre de vie des jeunes en 1999]%
{Cadre de vie des jeunes en 1999%
\footnote{Sources :
« Histoires de familles, histoires familiales », \emph{Les cahiers de l'INED}, \no 156 ;
\emph{Recensement de la population}, INSEE, 1999, p. 281.} }
\label{tableau-cadre-vie-1999}
\begin{tabular}{*{6}{>{\hspace{0pt}\centering\arraybackslash}b{\lcol}}}
Âge des jeunes (années) & Vivant avec les deux parents de naissance & Avec un parent seul%
\footnote{Familles monoparentales.}
 & Avec un parent et un beau-parent%
\footnote{Familles recomposées.}
 & Autres situations%
\footnote{En internat, en appartement, en chambre, chez un logeur, en placement ASE, en prison, en hôpital,~etc.}
 & Total\\
\hline
 0-4     & 85,0 & 11,1 & 1,8 & 2,2  & 100~\% \\
 5-9     & 77,7 & 15,6 & 5,2 & 1,5  & 100~\% \\
 10-14 & 72,7 & 17,5 & 8,4 & 1,5  & 100~\% \\
 15-19 & 68,5 & 18,7 & 8,6 & 4,1  & 100~\% \\
 20-24 & 43,5 & 11,5 & 4,3 & 40,6 & 100~\% \\
\hline
 0-17  & 76,5 & 15,7 & 6,0 & 1,8  & 100~\%
\end{tabular}
\end{minipage}
\end{table}

%CADRE DE VIE DES JEUNES EN 1999[6]
% 
%\emph{Age des jeunes}
%\emph{ (années)}
%\emph{Vivant avec ses deux parents de naissance}
%\emph{Avec un parent seul[7]}
%\emph{Avec un parent et un beau-parent[8]}
%\emph{Autres situations [9]}
%\emph{Total}
%\emph{0-4}

  

\makeatletter
\if@twoside
\begin{table}[t]% [!htb]
\else
\begin{table}[!t]% [!htb]
\fi
\makeatother
%\centering




\begin{minipage}{\textwidth} 
\caption[Cadre de vie des jeunes en 2004-2007]%
{Cadre de vie des jeunes en 2004-2007%
\footnote{Source : \emph{Moyenne annuelle des enquêtes emploi de 2004 à 2007}, INSEE.} }



\label{tableau-cadre-vie-2004-2007}

\begin{tabular}{*{6}{>{\hspace{0pt}\centering\arraybackslash}b{\lcol}}}
Âge des jeunes (années) & Vivant avec les deux parents de naissance & Avec un parent seul & Avec un parent et un beau-parent & Autres situations & Total\\
\hline
 0-6     & 82,2 & 10,1 & 7,2 & 0,5  & 100~\% \\
 7-13   & 72,8 & 16,6 & 9,9 & 0,7  & 100~\% \\
 14-17 & 66,9 & 19,0 & 9,8 & 4,4  & 100~\%
\end{tabular}

\end{minipage}

\end{table}

%CADRE DE VIE DES JEUNES EN 2004/2007[11]
% 
%\emph{Age des jeunes (années)}
%\emph{Vivant avec ses deux parents de naissance}
%\emph{Avec un parent seul[12]}
%\emph{Avec un parent et un beau-parent}
%\emph{Autres situations[13]}
%\emph{Total}
%\emph{0-6}
 
 
 
 
 L'évolution des comportements n'a rien de fulgurant. Vivre séparé de l'un de ses deux géniteurs reste une situation minoritaire : pour l'instant les trois quarts des mineurs vivent sous le même toit que leurs \emph{deux} parents \emph{de naissance} (dont les deux tiers des mineurs de 15 ans à 18 ans).
 
 Mais les évolutions actuelles sont aussi des évolutions symboliques : il n'y a peut-être (à vérifier) jamais eu autant d'enfants qu'aujourd'hui à vivre jusqu'à leur majorité avec leur père et leur mère de naissance et pourtant les familles ne sont plus pensées comme l'alliance irréversible de deux lignées, ni comme des institutions aux limites intangibles, mais comme des associations d'individus à géométrie variable. Les enfants d'aujourd'hui apprennent très tôt que les couples mixtes sont fragiles, qu'on rencontre aussi des couples mariés de même sexe, qu'amour ne rime pas avec toujours, que les princes et les princesses n'ont pas forcément beaucoup d'enfants, et qu'ils se séparent souvent avant la fin de leur histoire. Ils apprennent à dissocier parentalité et conjugalité, ou plutôt ils n'apprennent plus à les associer de manière indéfectible. À côté des scénarii traditionnels de leurs jeux d'imagination (le gendarme et le voleur, le client et la marchande, le malade et le docteur, l'indien et le cow-boy,~etc.) ils disposent maintenant du jeu du mariage et du divorce.

 Sous l'Ancien Régime c'était le contraire : en droit civil comme en droit canon, les mariages étaient indissolubles. Par contre, la mortalité d'alors, très élevée par rapport à celle d'aujourd'hui, faisait que plus de la moitié des époux étaient séparés par la mort avant même que leurs enfants n'aient atteint leurs vingt ans, et à cet âge il était normal d'être orphelin d'au moins un de ses deux parents. La durée moyenne effective des couples conjugaux était faible, environ quinze ans, comparée à celle d ceux des couples d'aujourd'hui qui n'ont pas divorcé, autour de cinquante ans. 

 La Révolution avait autorisé et facilité le divorce \emph{par consentement mutuel}, et à la suite de cette décision le taux de divorces observé dans les villes (mais \emph{seulement dans les villes}) avait rapidement atteint le niveau actuel. Mais contrairement à ce qui s'était passé dès l'an~III, aujourd'hui personne ne semble s'en inquiéter. Personne ne se donne plus pour objectif d'enrayer ce phénomène comme ce fut le cas avec le Code Napoléon, pendant la plus grande partie du \siecle{19} et sous le régime de Vichy (1940-45). Il ne s'agit plus de punir un coupable, ou deux, ni de chercher à prouver aux conjoints qu'ils peuvent respecter leurs engagements conjugaux au prix de quelques accommodements. Au contraire, les lois accompagnent ce mouvement de « \emph{démariage}
\footnote{Cf. Irène \fsc{THERY}, et son livre du même nom.}, et le divorce par consentement mutuel est devenu le modèle du bon divorce. 

 C'est en majeure partie du fait des divorces que les personnes seules avec enfants ont crû en nombre et en visibilité depuis 1970. En effet, le pourcentage de veufs et de veuves en leur sein a beaucoup baissé, au contraire de celui des divorcés : 9 fois sur 10 il s'agit de femmes seules avec enfants. 
 


 


 
\section{Problèmes de transmission}


 Les décideurs du passé n'avaient guère de difficultés à se mettre d'accord sur l'éducation des enfants et adolescents : jusqu'au milieu du \siecle{20} a régné dans le domaine éducatif un assez grand consensus autour de règles communes et de limites peu ou pas discutées. Cela se traduisait par exemple par le fait qu'au même moment les établissements éducatifs du \crmieme{19} ou du \siecle{20} présentaient partout à des nuances près les mêmes modes de fonctionnement, les mêmes limites, la même séparation des sexes, les mêmes styles de communication, qu'ils se réfèrent à un corps de doctrine religieux ou à une laïcité stricte. Les décideurs ne doutaient pas de leur droit à imposer leurs analyses aux parents qu'ils jugeaient négligents, délinquants, ou mal pensants. Ils valorisaient l'éducation au sein de la famille, mais au nom même de celle-ci ils plaçaient sans hésiter les enfants loin de leurs parents lorsque les conditions de leur éducation leur paraissaient compromises. C'est pourquoi l'époque actuelle est atypique en ce que depuis un bon demi-siècle elle hésite dans l'idée qu'elle se fait de l'intérêt de l'enfant, dans le choix de ce qu'elle veut lui transmettre, et dans les modalités de la transmission. 

 Il n'existe pas de savoir scientifique sur ce qu'est l'intérêt de l'enfant : il existe certes des connaissances scientifiques de plus en plus affinées sur les liens entre telle mesure éducative et tel résultat, telle performance, tel taux de morbidité,~etc. mais l'intérêt de l'enfant c'est bien autre chose. Il est lié aux fins que les hommes se donnent, qui ne sont pas scientifiques, mais politiques, philosophiques, religieuses, éthiques... 

 En l'absence de croyance partagée sur ce qu'est l'intérêt de l'enfant, l'accord minimal se fait sur l'idée qu'il faut avant tout ne pas lui nuire. Le reproche majeur qu'encourt un éducateur ce n'est plus de le « gâter » par sa complaisance et par son manque de fermeté. Le principal des risques actuels de son métier, c'est d'être accusé de le maltraiter par des exigences excessives. Parfois les enfants semblent assimilés à une population à libérer (par le droit) de l'arbitraire oppressif que les parents, les enseignants et les institutions d'assistance et de rééducation exerceraient sur eux%
% [1]
\footnote{Références : 
\\Gérard \fsc{MENDEL}, \emph{Pour décoloniser l'enfant, socio psychanalyse de l'autorité}, 1971.
\\Alain \fsc{RENAUT}, \emph{La libération des enfants, contribution philosophique à une histoire de l'enfance}, 2002.}% 
.

 Les « \emph{événements de mai 1968} » ont rendu visible et accéléré une remise en question des institutions, des hiérarchies et de l'argument d'autorité, mise en question qui avait commencé bien avant : avec les « maîtres du soupçon » ? Dès 1889 et la possibilité de déchoir les pères indignes ? Dès la Révolution française et l'exécution de Louis~XVI ? Avec les Lumières et Rousseau ? Avec la Renaissance et Rabelais,~etc. ? C'est aussi que bien des personnes et des institutions revêtues d'autorité avaient montré leurs propres limites, au nom de l'ordre et de l'obéissance, au cours des guerres et sous les régimes totalitaires dont le \siecle{20} a connu quelques beaux exemples, tandis que la psychanalyse soulignait la place centrale du désir du sujet dans son propre développement. D'autre part différentes recherches et expériences scientifiques avaient montré que les méthodes autoritaires d'éducation pouvaient être nocives, et que les méthodes non autoritaires de direction des groupes (Moren, Rogers) comme les pédagogies basées sur la découverte et l'initiative (Freinet, Montessori,~etc.) pouvaient être plus efficaces que les autres.

 Lorsqu'un anonyme inspiré a écrit sur un mur : « \emph{il est interdit d'interdire} », cette proposition jaillie d'on ne sait où a donc été reprise comme une évidence. Pour quelle raison ce paradoxe a-t-il à ce point fait vibrer la génération née au sortir de l'occupation ? ... sans doute parce qu'il était le corollaire d'un autre slogan aussi fameux de la même période : « \emph{jouissons sans entraves} ». 

 Les enfants se sont vus reconnaître le droit à la parole sur tout ce qui les concerne, notamment leurs orientations, mais aussi le droit à une vie sexuelle active dès l'âge de quinze ans (y compris le droit au secret médical, y compris pour les filles le droit si elles le désirent de mener à bien une grossesse ou de choisir une IVG,~etc.) tandis que leurs parents ont été sommés par la loi de les conduire démocratiquement vers une indépendance aussi précoce que possible (sauf dans le diomaine financier). 

 Depuis 1882, la durée de l'obligation scolaire s'est allongée et l'âge minimum de la mise au travail est passé de douze à seize ans afin de donner aux jeunes le maximum de chances d'insertion, de les protéger de toute exploitation au travail, et d'écrêter les différences entre milieux scolaires et sociaux différents. Depuis la création du Collège unique (\emph{Réforme \fsc{Haby}}, 1975), tous les enfants bénéficient de ce qui était un privilège jusqu'aux années soixante du \siecle{20}. Les études longues sont plus que jamais la voie royale vers la réussite personnelle. Grâce à leur quasi gratuité, tous ceux qui en ont les moyens intellectuels et le désir (et aussi des parents suffisamment aisés pour subvenir à leurs besoins matériels jusqu'à la fin de leurs études) ont des chances sérieuses de pouvoir en faire. Quant à savoir s'ils sont contents de l'extension de l'âge de 12 ans à 14 ans, puis à 16 ans, de leurs 5 ou 7 heures journalières de fréquentation des enseignants, il est assez évident que nombre d'entre eux, et notamment de garçons, ne la vivent pas bien et le font bruyamment savoir.

 Dans un contexte de concurrence scolaire généralisée, les richesses financières et culturelles des parents ne peuvent plus suppléer aussi massivement qu'autrefois à l'incompétence d'un jeune ou à son absence d'implication personnelle (même si elles jouent beaucoup). C'est pourquoi même s'ils sont toujours soucieux de l'avenir de leurs enfants, la pression qu'ils exercent a changé de lieu d'application : du contrôle rigoureux de leur sexualité pré conjugale, autrefois impératif pour leur futur établissement, et désormais sans importance, à l'exigence de performances scolaires aussi brillantes que possible, désormais sans alternative. C'est que rien n'a changé, bien au contraire, dans les règles du jeu qui permettent d'accéder aux meilleures sections des grands lycées et aux plus réputées des grandes écoles françaises et par là aux emplois les mieux payés, les plus attrayants ou les plus influents. Les jeunes n'ont donc plus guère à réprimer leurs désirs sexuels ni à supporter la culpabilité qui s'y attachait, devant un dieu ou devant leurs parents (sauf sans doute les jeunes aux tendances homosexuelles). Par contre il leur faut satisfaire à des normes exigeantes d'autonomie, de productivité intellectuelle et de compétitivité. Ceux qui n'y parviennent pas vivent une « honte » qui peut être au moins aussi insupportable que les anciennes culpabilités. Il n'y a sans doute pas moins de pression parentale aujourd'hui qu'autrefois, et il n'est peut-être pas plus agréable d'y être soumis, ni plus facile d'y satisfaire... 

 Sans parler de la responsabilité qui repose sur les épaules des enfants sur qui l'on compte, à défaut d'autre liens, pour donner sens à la vie de leurs parents :

\begin{displayquote}
\emph{Encore plus importante, naturellement, cette question : qu'est-ce qu'un enfant ? Le paradoxe est ici encore plus important car on n'a jamais autant prêté attention à l'enfant, on ne s'est jamais autant soucié de lui et on n'a jamais autant désenfantisé l'enfant.}
 
\emph{Désenfantiser l'enfant, comme s'il n'était possible de le concevoir comme notre égal qu'en le concevant comme notre semblable}[...]

\emph{L'enfant soutien de famille : ceci évoque un renversement tout à fait fondamental. \emph{[...]} la parentification des enfants dans les familles recomposées, c'est-à-dire un mouvement nouveau où, de façon tout à fait inattendue, la prise en compte de l'enfance aboutit à un déni d'enfance et où l'infantilisation du monde des adultes aboutit à une parentification du monde des enfants.}

 [... cette question encore paradoxale :] \emph{est-ce à l'enfant de dire qui appartient ou qui n'appartient pas à sa propre famille ?}%
% [2]
\footnote{Irène \fsc{THERY}, « Peut-on parler d'une crise de la famille ? un point de vue sociologique », \emph{Neuropsychiatrie de l'enfance et de l'adolescence}, 2001, 49, 492-501.} 
\end{displayquote}



 
 
 

 
 