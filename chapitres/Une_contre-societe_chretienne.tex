
\chapter{Une contre-société chrétienne}


\section{Laïcs et laïques consacrés}

 Si la fécondité et le mariage cessaient d'être des obligations morales, alors les jeunes gens avaient le droit moral de s'y refuser, non plus en raison du manque d'attrait du parti proposé par leur père, ce qui était parfaitement admis à Rome, mais en raison d'une préférence pour le célibat « en vue de Dieu » et pour la continence, chose qui jusque là n'avait pas de sens. C'était donner aux jeunes gens le droit de choisir une vie indépendante du désir de leur \emph{pater familias}. C'était aller directement contre l'intérêt des familles tel que celles-ci le concevaient : en droit les jeunes gens qui avaient un \emph{pater familias} étaient mariés par celui-ci. Si le consentement des futurs époux était obligatoire à Rome, en réalité il était présumé acquis à partir du moment où ils ne protestaient pas trop bruyamment. Les parents de ce temps n'hésitaient pas à déterminer eux-mêmes, en toute bonne conscience, le destin de leurs enfants. Au même moment et en dépit des interdictions légales des enfants de tous âges étaient vendus ou abandonnés aux prêteurs sur gage par des parents sans ressources, ce qui ne leur laissait guère de chances de se marier.

 Étant donné que quel que soit son âge une fille non mariée vivait sous l'autorité de son père ou de son tuteur, si elle voulait rester célibataire il fallait qu'elle le convainque de ne pas la marier. Or les représentations juives et romaines faisaient au contraire un devoir aux pères de marier leurs filles. Lorsqu'ils tardaient à s'acquitter de ce devoir, la rumeur publique était prompte à faire courir le bruit que la jeune personne manquait d'attraits physiques, ou qu'un problème de santé rendait dangereuse ou impossible pour elle la procréation, ou que des problèmes financiers empêchaient la constitution d'une dot, ou bien encore que l'amour du père pour sa fille était excessif, sinon coupable. 

 Comment comprendre alors que Paul de Tarse ait pu écrire que s'il le trouvait bon un père pouvait vouer sa fille au célibat au lieu de la marier ? En effet selon lui : \emph{Si pourtant quelqu'un estime qu'il n'est pas honorable pour sa fille de dépasser l'âge du mariage, et qu'il doit en être ainsi, qu'il fasse ce qu'il veut ; il ne pèche pas : qu'ils se marient. De même celui qui est fermement décidé en son cœur, à l'abri de toute contrainte et libre de son choix, s'il décide de garder sa fille, il agira bien. Ainsi celui qui marie sa fille fait bien, et celui qui ne la marie pas fait mieux}. (I Cor 7,36-38). À qui Paul apportait-il son appui ? ... aux pères contre leurs filles ? ... ou bien aux pères et aux filles contre le reste du monde ? Ceci dit ce texte a plus d'une fois servi au fil des siècles à imposer à des filles un choix de vie qui n'était pas le leur. 
 
 Le choix du célibat pouvait être financièrement coûteux : du fait des lois d'Auguste ceux qui le choisissaient étaient exclus des héritages et soumis à l'impôt sur les célibataires. Mais surtout il était reconnu aux pères le droit d'émanciper ou de déshériter les jeunes récalcitrants. Ces jeunes étaient alors sans ressources, ce qui les contraignait à travailler pour autrui comme salarié (mercenaire), ce qui rapportait peu, ou à se vendre comme esclaves. Les filles qui choisissaient le célibat se condamnaient en outre à demeurer toute leur vie des mineures légales, alors que celles qui donnaient le jour à trois enfants étaient dispensées de tutelle à partir du décès de leur \emph{pater familias} et pouvaient gérer elles-mêmes leur fortune personnelle, même lorsqu'elles étaient mariées : une fois veuves elles n'avaient plus de comptes à rendre à personne sur l'emploi de leurs biens. Quant aux pauvres, qui ne pouvaient espérer aucun héritage, ils avaient encore moins que les autres intérêt à demeurer célibataires s'ils voulaient préparer leurs vieux jours. 

 Mais même si le mariage cessait d'être une obligation, qu'est-ce que cela changeait pour les femmes dans un monde où les filles et les femmes honorables ne pouvaient travailler qu'à leur domicile ou sous le regard direct d'un père, d'un frère ou d'un époux ? Comment les veuves et les épouses répudiées sans père et sans fortune personnelle (et la pauvreté des concubines abandonnées était encore pire) pouvaient-elles subsister honorablement, sinon en se mettant sous la protection d'un concubin ou d'un époux ? À quel autre moyen non infamant pouvaient-elles recourir pour ne pas en être réduites par la faim à vendre leurs enfants et à se vendre elles-mêmes ? À quoi bon dire du mal du remariage ? Se remarier ne valait-il pas mieux que se prostituer ?

 Les problèmes matériels posés par le choix du célibat étaient identiques à ceux des candidats au baptême qui vivaient jusque là dans une situation personnelle désapprouvée par l'Église : personnel des temples païens, fabricants d'idoles et d'\emph{ex-voto} païens, prostitués des deux sexes, acteurs, danseurs et danseuses, gladiateurs, organisateurs de spectacles,~etc. Aucun d'eux n'était exclu du salut puisque le baptême lavait tous les péchés antérieurs, mais ils ne pouvaient être accueillis dans la communauté qu'à la condition de cesser de vivre comme ils l'avaient fait jusque là. Une prostituée qui renvoyait ses clients, une maquerelle, un sculpteur d'idoles, un acteur, un gladiateur qui cessaient leur activité mouraient de faim. Il n'était pas imaginable de les renvoyer en s'en lavant les mains. 

 Les premiers chrétiens ne pouvaient pas se contenter de proclamer la valeur du célibat et de la continence. S'ils y croyaient vraiment il leur fallait le rendre possible. Leurs propres enseignements leur interdisaient de se désintéresser de la situation de ceux et surtout de celles qui faisaient ce choix sur leurs conseils. Ils devaient leur procurer les moyens de mettre en œuvre l'aspiration qu'ils avaient suscitée ou se taire. Les évêques ont donc été conduits à prendre fait et cause pour tous ceux de leurs fidèles qui faisaient un choix de vie radical. Si nécessaire ils se sont mis en avant pour soutenir les jeunes dans leurs combats contre leurs familles. 

 Les \emph{vierges consacrées} (parfois nommées \emph{diaconesses} dans l'aire grecque) sont apparues très tôt, dès le deuxième siècle. A la différence des diacres, des prêtres et des évêques, elles n'ont jamais fait pas partie des clercs. C'étaient des laïques, à l'instar des futurs moines, dont elles étaient l'équivalent féminin. Elles n'étaient pas \emph{ordonnées} pour mener à bien une mission ni pour tenir une place dans la hiérarchie. Elles choisissaient le célibat pour se consacrer à la prière et au service de la communauté, notamment à la catéchisation des femmes et au service des malades. Leurs vœux étaient reçus par leur évêque au cours d'une cérémonie qui a très tôt été interprétée comme un mariage mystique avec le Christ, auquel elles se vouaient. Le fait qu'à compter de ce jour elles portaient le voile (d'où le nom de la cérémonie : « prise de voile ») affichait cette interprétation de leur choix de vie à leurs contemporains. Il s'agissait en effet du voile des matrones qui proclamait qu'elles avaient un \emph{dominus}, un seigneur et maître. Cela les rapprochait des vestales, vouées au célibat et à la virginité, qui elles aussi portaient un voile qui avertissait les passants que leur corps était consacré, et qu'y toucher était sacrilège.


\section{Clergé et continence}

 En ce qui concerne les desservants du culte%
% [2]
\footnote{Pour cette partie je me suis en particulier servi de \emph{Naissance d'une hiérarchie, les premières étapes du cursus clérical,} d'Alexandre \fsc{FAIVRE}, 1977.}
l'Église n'a pas adopté les pratiques des synagogues. Les rabbins sont des sages, des savants, des exégètes et des juges, et non des prêtres, des sacrificateurs. Ils ne président au culte qu'avec l'assentiment des fidèles. Jusqu'au milieu du Moyen-Âge au moins ils ne seront pas payés. Ils sont choisis par les fidèles eux-mêmes, et non reçus d'une autorité extérieure à la communauté. Ils ont moins un pouvoir normatif qu'un pouvoir d'influence, gagné par leur réputation de compétence, par leur savoir. Ils possèdent une autorité intellectuelle, morale et juridique acquise, reconnue par leurs pairs et non octroyée, même si la notion d'ordination par imposition des mains ne leur était pas étrangère. Cela signifie que le culte des synagogues était fondamentalement différent de celui des temples antiques. Il ne prétendait pas remplacer les sacrifices offerts au Temple de Jérusalem : bien au contraire si les prières faisaient référence aux sacrifices offerts au Temple, c'était pour rappeler que ces sacrifices n'étaient pas possibles, et qu'on ne pouvait qu'accepter la réalité de leur absence. 

 Au contraire le rite chrétien de la \emph{fraction du pain} se présentait comme le renouvellement littéral des gestes inaugurés par Jésus à la \emph{Cène}, selon le récit des Évangiles. Il n'était pas interprété comme un geste symbolique, destiné seulement à faire mémoire d'un événement historique une fois pour toutes réalisé. Il avait d'emblée été compris par les premiers chrétiens comme un geste efficace, performatif, comme un geste qui produisait ce qu'il énonçait, qui réalisait à la lettre ce qu'il disait : \emph{prenez et mangez : ceci est mon corps ... Buvez-en tous : car ceci est mon sang, le sang de l'alliance, qui va être répandu pour une multitude en rémission des péchés}. (Matthieu, 26, 28). L'Eucharistie a d'emblée été un rite surdéterminé. Pour Paul le « repas du Seigneur » est un repas pris en commun et c'est une anticipation du festin eschatologique, à la fin des temps (1 Co 11,17-34). C'est le rappel des repas pris avec Jésus avant sa mort, mais aussi après sa résurrection. C'est un repas où le Seigneur se donne, en personne, à son peuple comme nourriture. C'est « la Pâque du Seigneur » où Jésus est l'agneau pascal dont le sang consacre ceux qui le reçoivent (1 Co 5,7). C'est un renouvellement de l'alliance (1 Co 11,25) : dans la \emph{Tora} le sang de l'alliance scellait la communion entre le peuple et Dieu, le sang du Christ instaure une communion encore plus profonde. C'est un sacrifice de louange. C'est donc à plusieurs titres que Paul de Tarse faisait de l'Eucharistie un sacrifice bien avant la destruction du Temple, et alors qu'à Jérusalem le \emph{sacrifice perpétuel} continuait d'être tous les jours offert, et que selon les \emph{Actes des Apôtres} les membres de la communauté judéo-chrétienne locale continuaient d'y assister assidûment.

 Mais en 70 de notre ère la destruction du Temple a contraint les juifs à cesser tout sacrifice sanglant : cela a permis aux chrétiens de penser et de dire que le \emph{sacrifice perpétuel} continuait désormais dans un autre lieu et sous une autre forme, non sanglante. Le clergé chrétien s'est pensé comme le successeur du clergé du Temple et a pris modèle sur lui. L'évêque s'est identifié au grand prêtre, les prêtres chrétiens aux prêtres juifs, les diacres et les autres ministres du culte aux lévites, d'autant plus soucieux de se conformer à leurs modèles qu'ils prétendaient les remplacer et ne pouvaient ignorer que par ailleurs ils récusaient ces mêmes modèles.

 À ses débuts l'Église avait disqualifié la notion d'impureté rituelle, notamment par la voix de Paul qui en bon pharisien savait de quoi il parlait, et qui ne voyait pas d'inconvénient à manger des viandes consacrées aux idoles, du moment que cela ne choquait pas les esprits faibles (Rom, 14, 2-3; Cor I, 8, 4-13). Mais la notion d'impureté a fait un retour en force avec la constitution d'un personnel religieux permanent, spécialisé, consacré par l'imposition des mains et mis à part : un \emph{clergé}. Le clergé chrétien va en effet s'imposer de respecter scrupuleusement les règles de pureté qui s'appliquaient aux prêtres du Temple. Il va surenchérir par rapport à ces règles et cela va renforcer la sacralisation du sexe et la valorisation de la continence (même si cette imitation n'est que l'un des éléments qui y ont contribué). En effet ses membres étaient en permanence susceptibles de toucher les « vases sacrés ». S'ils prenaient modèle sur le Temple cela exigeait d'eux qu'ils soient en permanence dans l'état de pureté rituelle exigé du Grand Prêtre durant ses semaines de service. Il fallait donc qu'ils ne soient jamais souillés par « l'impureté » sexuelle. Les clercs vont donc très tôt être voués à la continence perpétuelle, à partir de leur ordination diaconale. 

 Avant qu'ils aient atteint l'âge où ils devaient être ordonnés diacres les clercs pouvaient se marier et cohabiter avec une épouse pour avoir des enfants. L'imposition des mains (ordination diaconale) ne se faisait en effet qu'à un âge relativement avancé, 30 ans, si ce n'est 40 (prêtre vient de presbytre, c'est-à-dire « ancien, âgé »). Cela laissait un temps suffisant à un clerc marié dès ses 18 ou 20 ans pour avoir des enfants. Il était donc normal de rencontrer des fils de diacres, de prêtres ou d'évêques. Mais si un enfant leur naissait après leur ordination, les membres du clergé devaient en principe être démis de leurs fonctions. Ceci dit bien des clercs mineurs demeuraient leur vie entière dans des fonctions qui n'exigeaient pas la continence. 


\section{Il n'y a plus ni esclave ni homme libre ?}


 
 
 En Ga 3, 28, Paul de Tarse écrit : \emph{Il n'y a ni juif ni grec, il n'y a ni esclave ni homme libre, il n'y a ni homme ni femme ; car tous vous ne faites qu'un dans le Christ Jésus}. Il développe la même idée en 1 Co 12, 13 : du point de vue de Dieu, il n'y a aucune différence entre les esclaves et les hommes libres. Ils sont égaux en dignité, en valeur spirituelle et promis au même salut après leur mort. Sur cette terre il en était autrement : les chrétiens ne condamnaient pas l'esclavage, ce en quoi ils étaient d'accord avec la totalité des peuples de l'antiquité. Ce n'étaient pas des révolutionnaires politiques et ils n'avaient qu'une confiance limitée dans les institutions et les pouvoirs humains. Ils croyaient que les changements de structure n'ont de chance d'atteindre leurs objectifs que si les cœurs ont d'abord été transformés. Avec Paul ils pensaient que le désir du mal est profondément inscrit en l'homme%
% [6]
\footnote{\emph{... en effet vouloir le bien est à ma portée, mais non pas l'accomplir : puisque je ne fais pas le bien que je veux et commets le mal que je ne veux pas.} (Rm, 7, 18-19).}%
. Pour lui comme pour les apôtres, pour ses maîtres pharisiens ou pour les stoïciens l'esclavage le plus grave était celui du péché%
%[7]
\footnote{\emph{En vérité, en vérité, je vous le dis, tout homme qui commet le péché est un esclave. Or l'esclave n'est pas pour toujours dans la maison, le fils y est pour toujours. Si donc le fils vous affranchit, vous serez réellement libres. \emph{(Jn 8,34-36).} Ne savez-vous pas qu'en vous offrant à quelqu'un comme esclaves pour obéir, vous devenez les esclaves du maître à qui vous obéissez, soit du péché pour la mort, soit de l'obéissance pour la justice ? Mais grâces soient rendues à Dieu ; jadis esclaves du péché, vous vous êtes soumis cordialement à la règle de doctrine à laquelle vous avez été confiés, et, affranchis du péché, vous avez été asservis à la justice (j'emploie une comparaison humaine en raison de votre faiblesse naturelle) car si vous avez jadis offert vos membres comme esclaves à l'impureté et au désordre de manière à vous désordonner, offrez-les de même aujourd'hui à la justice pour les sanctifier.} (Rm 6,16-19).}
(des désirs non contrôlés par la raison). La vie n'était qu'un bref passage%
%[8]
\footnote{\emph{J'estime en effet que les souffrances du temps présent ne sont pas à comparer à la gloire qui doit se révéler en nous.} (Rm, 8, 18).}% 
, et le salut religieux, la \emph{vie éternelle}, était plus important que le bonheur ou la réussite terrestre. Il n'était pas déterminant pour le salut d'être libre ou d'être esclave. L'essentiel était déjà gagné par la mort du Christ : \emph{Vous avez été achetés%
%[9] 
\footnote{La rédemption était le paiement d'un marché, d'une rançon, le rachat d'une créance par un rédempteur, c'est-à-dire un entrepreneur prêt à risquer ses capitaux dans une affaire, ou dans le rachat d'un esclave à son maître pour le libérer.} 
cher} (1 Co 6, 20). En conséquence Paul enseignait aux esclaves que Dieu lui-même avait accepté qu'ils soient placés là où ils étaient, et il leur recommandait la patience, l'honnêteté, le travail bien fait, indépendamment de toute récompense terrestre, non pour leur maître, mais pour Dieu ... et pour la bonne réputation des chrétiens. 

 Dans le même temps Paul enseignait%
% [10]
\footnote{Paul a maintes fois répété le même enseignement -- ce qui veut peut-être dire que la question lui a souvent été posée ?} 
aux maîtres que les esclaves ont la même valeur qu'eux%
%[11]
\footnote{\emph{Aussi bien est-ce en un seul esprit que nous tous avons été baptisés pour ne former qu'un seul corps, juifs ou grecs, esclaves ou hommes libres, et tous nous avons été abreuvés d'un seul esprit.} (I Cor 12, 13).}% 
, et il leur faisait un devoir de les traiter \emph{comme des collaborateurs} et \emph{comme des frères en Jésus-Christ} : « \emph{Esclaves, obéissez en tout à vos maîtres d'ici-bas, non d'une obéissance tout extérieure qui cherche à plaire aux hommes, mais en simplicité de cœur, par crainte du Maître. Quel que soit votre travail, faites-le avec âme, comme pour le Seigneur et non pour des hommes, sachant que le Seigneur vous récompensera en vous faisant ses héritiers. C'est le Seigneur Christ que vous servez; qui se montre injuste sera certes payé de son injustice, sans qu'il soit fait acception de personne. Maîtres, accordez à vos esclaves le juste et l'équitable, sachant que vous aussi, vous avez un maître au ciel}. » (Col 3, 22-25 ; 4,1). 

 Les épîtres de Pierre défendaient les mêmes thèses : \emph{Vous, les domestiques} (= esclaves) \emph{soyez soumis à vos maîtres, avec un profond respect, non seulement aux bons et aux bienveillants, mais aussi aux difficiles. Car c'est une grâce que de supporter par égard pour Dieu des peines que l'on souffre injustement. Quelle gloire en effet à supporter les coups si vous avez commis une faute, mais si faisant le bien vous supportez la souffrance c'est une grâce auprès de Dieu. Or c'est à cela que vous avez été appelés, car le Christ aussi a souffert pour vous, vous laissant un modèle afin que vous suiviez ses traces, lui qui n'a pas commis de faute. Et il ne s'est pas trouvé de fourberie dans sa bouche ; lui qui insulté ne rendait pas l'insulte, souffrant ne menaçait pas, mais s'en remettait à celui qui juge avec justice}. (Première Épître de Pierre 2,18-23). 

 Mais puisque la liberté est meilleure que la servitude, Paul rappelait aux maîtres qu'affranchir leurs esclaves était une bonne œuvre. Les communautés chrétiennes pouvaient racheter ceux de leurs membres qui étaient esclaves, notamment quand la nature de leur emploi, ou la dureté et l'injustice de leurs maîtres, mettaient en danger leur vie, leur santé, leur vertu ou leur foi. Il était cependant conseillé aux esclaves chrétiens de ne pas revendiquer trop bruyamment leur rachat par la communauté : ses ressources n'y auraient pas suffi, et on se serait vite demandé si leur piété n'était pas trop intéressée.

 Paul et ses successeurs auraient-ils pu s'engager plus loin ? Y ont-ils jamais pensé ?  On pourra approfondir ce sujet avec Peter \fsc{GARNSEY}, \emph{Conceptions de l'esclavage, d'Aristote à Saint Augustin}, Les belles lettres, Paris, 2004.



\section{Le service des pauvres}

 Comme n'importe quelle minorité de l'Empire les chrétiens se devaient de prendre en charge leurs pauvres, malades, et infirmes. Chaque église locale prenait en charge les siens. En cas de catastrophe elle pouvait demander des secours aux autres églises. Elle enregistrait ses pauvres, titulaires d'un droit à secours, sur la liste des personnes qu'elle entretenait \emph{(matricule)}, à côté des membres du clergé. Dans la Rome du \siecle{3}, alors que la religion chrétienne était encore illicite, c'était déjà par milliers que se comptaient les pauvres, les malades et les vieillards inscrits sur les listes de l'évêque. Chacun d'eux, pauvres et clercs entretenus, valides ou non, faisait en quelque sorte partie de la \emph{familia} de l'évêque. 

 Si toutes les églises de l'Empire faisaient de même, aucune ne concurrençait l'Annone sur son propre terrain. Les bénéficiaires n'étaient ni des hommes honorables ni des citoyens, c'étaient les indigents, les vieillards et les infirmes, beaucoup de femmes et de petits enfants, des esclaves et des concubines abandonnés, des étrangers sans ressources, exilés ou bannis... Leur sexe et leur statut ne comptaient pas. Ils étaient recrutés sur critères « sociaux », ce qui pouvait attirer les plus pauvres vers l'église de leur cité : c'est peut-être pour cela que lorsque l'empereur Julien a entrepris pendant son court règne de refonder les cultes traditionnels il a cherché à engager le clergé païen dans cette voie.

 Pourquoi les chrétiens donnaient-ils aux pauvres ? Selon la Bible YHWH préférait à tous les sacrifices l'assistance faite aux pauvres. Selon les évangiles Dieu regardait ce qui était fait aux pauvres, aux malades, aux étrangers, aux prisonniers… comme si c'est à lui-même que c'était fait : « \emph{Quand le Fils de l'homme viendra dans sa gloire, escorté de tous les anges, alors il prendra place sur son trône de gloire. Devant lui seront rassemblées toutes les nations, et il séparera les gens les uns des autres, tout comme le berger sépare les brebis des boucs. Il placera les brebis à sa droite, et les boucs à sa gauche. Alors le Roi dira à ceux de droite : » venez, les bénis de mon père, recevez en héritage le royaume qui vous a été préparé depuis la fondation du monde. Car j'ai eu faim et vous m'avez donné à manger, j'ai eu soif et vous m'avez donné à boire, j'étais un étranger et vous m'avez accueilli, nu et vous m'avez vêtu, malade et vous m'avez visité, prisonnier et vous êtes venus me voir. » Alors les justes lui répondront : « Seigneur, quand nous est-il arrivé de te voir affamé et de te nourrir, assoiffé et de te désaltérer, étranger et de t'accueillir, nu et de te vêtir, malade ou prisonnier et de venir te voir ? ». Et le roi leur fera cette réponse : » En vérité je vous le dis, dans la mesure où vous l'avez fait à l'un de ces plus petits de mes frères, c'est à moi que vous l'avez fait. » (Matthieu, 25, 31-40) et : « Alors il leur répondra : « en vérité je vous le dis, dans la mesure où vous ne l'avez pas fait à l'un de ces plus petits, à moi non plus vous ne l'avez pas fait. » Et ils s'en iront, ceux-ci à une peine éternelle, et les justes à la vie éternelle. » (Matthieu 25, 45-46)}"

 Le christianisme n'étant pas reconnu les églises n'étaient pas habilitées à recevoir des donations. Cela n'a pas empêché les dons d'avoir lieu ni un patrimoine de se constituer, sous couvert de prête-noms. Bien avant le quatrième siècle on constate l'existence d'une « zone grise », où les églises sont tolérées et où un patrimoine leur est tacitement reconnu par les autorités, même si une persécution pouvait tout remettre en question de temps en temps. 

 Les donateurs les plus généreux étaient les femmes de l'aristocratie romaine, veuves ou divorcées, libérées de toute tutelle, et non susceptibles d'être ordonnées à un \emph{honneur} (non susceptibles d'être contraintes à assumer les frais d'une \emph{évergésie}). La loi les disait \emph{sui juris} (autonomes juridiquement) si elles avaient donné le jour à trois enfants au moins, et si leur \emph{pater familias} était décédé. Elles préféraient se mettre sous la protection morale de l'évêque de leur choix plutôt que de se soumettre à un nouveau mari auquel il leur serait très difficile de s'opposer, sans que cela garantisse pour autant qu'il soit plus intéressé par leur personne que par leur dot%
% [5]
\footnote{La dot, contribution du père de l'épouse aux dépenses du ménage, appartenait à cette dernière, mais elle était gérée par le mari. Par contre une femme \emph{sui juris} pouvait gérer elle-même tout le reste de sa fortune personnelle. La coutume voulait que la totalité des biens d'une femme parvienne à ses seuls enfants, mais il fallait pour cela qu'elle teste activement en leur faveur. Cela lui laissait en théorie la possibilité de tester pour d'autres qu'eux.}%
. Leur refus de se remarier était le moyen de leur liberté. Il leur permettait d'échapper aux grossesses et aux accouchements. Il évitait à leurs enfants d'être mal traités par un beau-père et de rentrer en conflit pour leur héritage (maternel) avec ceux d'un nouveau lit. Et enfin il leur laissait la liberté de jouer un rôle social valorisé par leur communauté. Les palais de ces grandes dames pouvaient offrir aux femmes sans ressources un refuge contre un monde « machiste » : c'est ainsi que se sont agrégés les noyaux des premières communautés religieuses féminines. 

 Quelques hommes imitaient ces nombreuses donatrices dans la mesure où leurs responsabilités de chefs de famille (ce que les femmes n'étaient pas) leur en laissaient la possibilité. C'était d'abord le cas des hommes mariés \emph{sans enfant}.

 Les chrétiens aisés étaient donc invités à infléchir en ce sens leurs activités d'évergètes. Ils étaient les plus capables d'accueillir la communauté dans leurs vastes demeures, et les plus susceptibles de disposer des ressources nécessaires pour subvenir aux besoins des plus pauvres. Comme chaque église locale élisait ses clercs et son évêque, les chrétiens avaient tendance à les choisir parmi les plus riches d'entre eux, selon la tradition antique de confier les responsabilités civiques aux notables. 

 En contrepartie des moyens de vivre qu'il donnait à ses protégés, à ses « clients », l'évêque avait comme tout \emph{patron} romain le droit de leur demander des services. Sans charges de famille les célibataires étaient disponibles le jour et la nuit et ils n'avaient ni épou(x)se ni enfants à prendre en charge. Leur entretien pouvait donc être nettement plus économique que celui de personnes mariées de niveau social égal. Ils (elles) pouvaient se vouer à la pauvreté sans être irresponsables face à une descendance. Dans ce cas le coût de leur travail pouvait être compétitif avec celui des esclaves. Au premier rang des services que les laïcs voués à la continence pouvaient rendre il y avait les tâches d'assistance. La première initiative institutionnelle de la toute première communauté chrétienne (à Jérusalem) avait été de créer un corps de clercs spécialisés dans les tâches d'assistance, les \emph{diacres} (dont rien ne nous dit qu'ils étaient célibataires), dont la tâche était de visiter les malades à domicile, et de gérer l'assistance, et d'abord l'assistance aux veuves (\emph{Actes des Apôtres} 6,1-6). 

\subsection{Veuves}

Au nombre des pauvres figuraient d'abord les veuves. L'assistance aux veuves consistait en distribution d'argent, de vêtements, de nourriture,~etc. Dans un monde où les femmes ne pouvaient exercer une activité honnête que dans un cadre familial, une veuve pauvre qu'aucun parent ne recueillait était dans une détresse totale, surtout si elle avait des enfants à charge. 

 La situation était encore pire pour une ex-concubine. Le cas de la veuve qui prostitue sa fille parce que ni l'une ni l'autre n'ont aucun autre moyen de vivre est un poncif de la littérature, de l'antiquité au \siecle{19}. La mère qui en était réduite là n'avait le plus souvent aucun choix : si elle vendait sa fille comme esclave, celle-ci n'échapperait probablement pas à la prostitution, sauf à être vendue comme esclave-concubine (mais il fallait trouver un protecteur), et elle perdrait sa liberté, tandis que sa mère n'aurait aucune ressource. En la prostituant la mère lui faisait perdre \emph{seulement} sa réputation. 

 Les veuves sans enfants à charge secourues par l'Église pouvaient s'employer au service des œuvres de l'évêque. Elles pouvaient s'occuper des enfants orphelins de père et de mère, des malades, des infirmes, vieillards et insensés,~etc. Elles étaient chargées des fonctions que seule une femme pouvait remplir sans faire jaser : soins aux femmes et aux petits enfants, visites aux femmes à domicile (instruction religieuse, soutien psychologique, assistance matérielle),~etc. Pour remplir ces missions on choisissait des matrones d'expérience, de bonne réputation (c'est-à-dire non infâmes) et d'un âge suffisant pour inspirer le respect (âge « canonique »). 

 Au fil des trois premiers siècles de notre ère le \emph{rôle social des veuves} soutenues par l'Église s'est effiloché. Sauf exception elles ont peu à peu cessé d'assumer des tâches de service auprès des femmes et des enfants. On a cessé d'attendre d'elles autre chose qu'une vie réglée, et si possible pieuse et édifiante. Leurs fonctions étaient désormais assumées par les vierges consacrées, et notamment tout ce qui a trait à l'assistance, qui dans l'aire catholique a presque exclusivement reposé sur des religieuses (et des religieux) jusqu'au \siecle{19} inclus. 

\subsection{Orphelins}

On a vu que les juifs considéraient que c'était une œuvre pieuse que de prendre en charge l'éducation des orphelins pauvres. Les non-chrétiens en faisaient d'ailleurs autant. La loi romaine prévoyait que les mineurs orphelins de père (libres, ingénus ou affranchis) soient confiés à l'autorité d'un tuteur. La tutelle des orphelins était une charge et un devoir civiques, dont seules quelques professions étaient exemptées. Le tuteur était responsable de la gestion des biens du mineur et de son éducation, mais il n'était pas chargé de la financer lui-même. Les dépenses étaient supportées par l'héritage de l'enfant. Que se passait-il quand un orphelin ne possédait rien, quand ses parents ne lui avaient laissé aucun héritage, ce qui devait être fréquent ? 

 Quels étaient les orphelins assumés par les communautés dont les textes ecclésiaux les plus anciens parlent si souvent? Étaient-ce les enfants des martyrs ? Probablement pas : ce n'est qu'à quelques moments exceptionnels et brefs que le nombre de ceux-ci a été significativement élevé. Il s'agissait plus vraisemblablement des enfants des fidèles sans ressources décédés d'accident, de misère ou de maladie avant qu'eux-mêmes ne soient capables de gagner leur vie, ce qui était une situation banale à l'époque. Ceux dont les deux parents étaient morts n'étaient pas donnés en adoption. Les chrétiens s'y refusaient au même titre que les juifs : ces enfants appartenaient déjà à une famille connue même si celle-ci n'avait plus d'autres membres vivants qu'eux-mêmes. 

 Les orphelins secourus par l'Église pouvaient aussi être les enfants des veuves signalées plus haut : en effet les textes associent constamment \emph{les veuves et les orphelins}. Les uns et les autres étaient dans la misère et couraient un grave risque de tomber dans l'esclavage ou/et la prostitution (cela restera une constante : les plus pauvres sont toujours les femmes seules avec enfants). Selon la loi la tutelle d'un orphelin de père ne pouvait pas être exercée par sa mère : les femmes ne pouvaient exercer aucune autorité sur autrui, même sur leurs propres enfants, sinon par délégation (du père ou du tuteur).

 Sous quelle forme les communautés apportaient-elles aux enfants sans parents leur soutien matériel et moral ? On peut imaginer bien des formules, en fonction de l'âge des enfants et du contexte, depuis la nourrice, salariée jusqu'à ce que l'enfant puisse commencer à travailler (très tôt, comme tous les pauvres d'alors), ou la mise en apprentissage chez un professionnel aux frais de la communauté, jusqu'à la prise en charge collective de grands enfants dans la maison de l'évêque et sous son contrôle. 

 Certains de ces orphelins pouvaient être investis de manière spéciale : les plus vifs d'esprit, ou ceux dont les parents avaient laissé le meilleur souvenir. Certains garçons trouvaient une place dans le clergé : orientation naturelle s'ils n'avaient pas eu d'autre figure paternelle que des membres de ce même clergé.

\subsection{Malades}

Les malades et les infirmes de la communauté étaient visités à leur domicile. Ce service de proximité qui ne sépare pas le sujet de son milieu était l'un des premiers services assuré par les veuves auprès des femmes. Les diacres rendaient le même service aux hommes. Rien n'interdisait aux clercs ordonnés de pratiquer la médecine, du moment qu'ils ne versaient pas le sang. 

 Vers 260 Denys d'Alexandrie vante le comportement des chrétiens face à une des épidémies de son temps : \emph{La peste fondit sur la ville, objet d'épouvante. La plupart de nos frères ne s'écoutaient pas eux-mêmes, visitant sans précaution les malades, leur donnant leurs soins dans le Christ. Chez les païens il en était tout autrement ; ceux qui commençaient à être malades, on les chassait, on fuyait ceux qui étaient les plus chers, on jetait sur la route des gens à demi-morts}. L'aide de la communauté se bornait-elle à ses propres membres ? 

 Une prise en charge totale était inévitable quand il s'agissait de personnes sans domicile, de voyageurs malades, de vieillards sans famille, d'esclaves abandonnés, d'orphelins sans fortune et sans parents, d'insensés trop agités ou trop régressés... Comme les moyens de l'hospitalité individuelle n'étaient pas illimités, il était dans l'ordre des choses que des formes collectives de prise en charge aient progressivement été mises en place, au moins dans les grands centres, sur le modèle alors bien connu des hôpitaux de garnison. 

\subsection{Énergumènes}

Toute l'antiquité, chrétiens y compris, croyait aux démons et à leur capacité de nuisance sur le corps comme sur l'esprit. Les chrétiens incluaient souvent parmi les démons tous les dieux païens, auxquels ils croyaient eux aussi à leur façon. Ceux qui selon les conceptions du temps étaient possédés par un démon étaient le plus souvent des malades mentaux agités, des épileptiques atteints de grand mal, des autistes, des psychotiques délirants, des déments, de grands attardés mentaux et des hystériques en crise. Dès le \siecle{2} des énergumènes (littéralement : « possédés par un démon ») étaient hébergés à plein temps dans des locaux dépendant de l'Église (qui s'en chargeait auparavant ? Les familles les gardaient-elles ou les abandonnaient-elles à la rue ou bien dans un temple aux bons soins d'Esculape ou de tel ou tel autre dieu, c'est-à-dire à la charge de la charité des fidèles et de la bonne volonté des desservants du temple ?) 

 Dont le but de chasser leurs démons ces malades bénéficiaient d'exorcismes quotidiens dont étaient chargés leurs gardiens. La fonction d'exorciste ne s'est individualisée que lentement. Longtemps chaque chrétien a été jugé assez compétent pour chasser les démons en imposant les mains sur le front des possédés, sans avoir besoin d'une ordination spéciale. Selon les Constitutions Apostoliques (une compilation rédigée durant les années 380 de textes réglementaires ou liturgiques plus anciens : \fsc{FAIVRE}, \emph{Naissance d'une hiérarchie}, 1977, p. 118) : « L'exorciste n'est pas ordonné, car être exorciste dépend de la bienveillance, de la bonne volonté et de la grâce de Dieu donnée par le Christ par la venue du Saint-Esprit. Celui qui a reçu le don de guérison est manifesté par une révélation de Dieu et la grâce qui est en lui est visible pour tous. » 

 Comment cette fonction a-t-elle évolué par la suite ? Les \emph{statuta ecclesiae antiqua} sont une œuvre privée et non la transcription de décisions faisant autorité (d'un concile, du Pape, ou d'un évêque). Ils n'en sont pas moins un témoin de l'état des pratiques et des opinions du temps de leur rédaction : vers 476-485. Ils font aux exorcistes un devoir d'être patients et bons avec leurs protégés, et d'imposer les mains à chacun d'eux tous les jours. Ce geste s'accompagnait d'une prière. Ils avaient aussi le devoir de leur apporter leur nourriture quotidienne à l'heure prescrite, sans les faire attendre : « Les énergumènes qui séjournent dans la maison de Dieu doivent recevoir en temps voulu leur pitance quotidienne qui leur est apportée par les exorcistes. » (canon 64). Cette dernière remarque suggère que beaucoup des patients présentaient des traits de retard mental ou de régression massifs qu'on faisait travailler à la mesure de leurs capacités (exemple : \emph{pavimenta domorum dei energumeni everrant} : « les énergumènes balaieront le pavement des églises ») et que les exorcistes étaient plutôt des garde-malades vigoureux que des experts en pathologie mentale ou en théologie. 

 Comme les infirmiers psychiatriques de naguère les exorcistes n'ont jamais été placés bien haut dans la hiérarchie du clergé (clercs « mineurs »). Au fil du temps ils ont été relégués avec leurs patients de plus en plus loin de l'autel, au plus bas niveau, près de la porte. À partir du \siecle{6} ou du \siecle{7} les plus savants des théologiens cesseront de croire que le démon se cache derrière les états pathologiques ordinaires. Les actes d'exorcisme encore pratiqués, devenus tout à fait rares, seront réservés à des prêtres nommés pour cela. 

\subsection{Voyageurs, vagabonds, mendiants}

En raison de la valorisation évangélique de la pauvreté et de la souffrance, et en raison de leurs pratiques d'assistance, les chrétiens ne pouvaient qu'attirer tous les accidentés de l'existence : ceux et celles qui avaient tout perdu, qui ne pouvaient plus travailler, les vieillards sans enfants qui n'avaient plus la force de gagner leur pain, les esclaves abandonnés (« libérés ») par leurs maîtres parce que trop vieux, infirmes ou malades, les concubines abandonnées sans enfants pour les recueillir, les prostituées âgées, les exilés inconsolables, les bannis de leur cité,~etc. 

 Les premières Églises ont adopté la même règle que les synagogues face aux coreligionnaires (face aux « frères ») en déplacement, en voyage d'affaire, en mission pour leur communauté, et face aux vagabonds : les voyageurs valides étaient reçus comme des hôtes pendant trois jours, après quoi ils étaient invités à poursuivre leur chemin ou à gagner leur vie en se mettant au travail, conformément au mot de Saint-Paul : \emph{celui qui ne travaille pas n'a pas droit de manger}. 

 Quant à ceux qui étaient hors d'état de continuer la route ils étaient soignés durant le temps qu'il fallait pour qu'ils se remettent : cela pouvait durer des mois ou des années. Cela pouvait durer tout ce qui restait d'une vie.

\subsection{Captifs}

Les chrétiens de l'antiquité rencontraient de multiples occasions de s'occuper de captifs :
\begin{enumerate}
% 1°)
\item ils subissaient sporadiquement des persécutions qui n'étaient pas toujours évitables. À chaque retour de flamme de ces persécutions les plus convaincus, les plus téméraires ou les moins chanceux de ces fidèles se retrouvaient en prison ou étaient condamnés aux mines ou aux bêtes du cirque ;
% 2°)
\item d'autres parmi les fidèles étaient emprisonnés pour des crimes ou des délits de droit commun, pour faillite,~etc. Quel qu'ait pu être le motif de leur captivité, la communauté se devait de les visiter, si nécessaire sur le site lointain des mines où ils et elles purgeaient leur peine. Elle les soutenait moralement et les encourageait à tenir bon dans leur foi et la pratique religieuse. Elle essayait d'adoucir leur sort, par exemple en soudoyant les gardiens pour qu'ils leur procurent de meilleures conditions de vie ;
% 3°)
\item enfin les chrétiens étaient exposés au même titre que tous leurs contemporains au risque d'être enlevés, avec la perspective d'être vendus au loin comme esclaves. Ce risque était plus élevé pendant les périodes de trouble, et pendant tous les voyages. Mais les enlèvements étaient à craindre même en ville, notamment les enlèvements d'enfants. Pour revoir libres ceux qui avaient été enlevés il fallait négocier et rassembler une rançon. Les églises locales faisaient ce qu'elles pouvaient en fonction de leurs moyens. 
\end{enumerate}

\subsection{Morts sans sépulture}

En accord avec les différentes cultures antiques la Bible faisait un devoir à quiconque était présent de donner une sépulture décente à toute personne décédée, sans aucune exception. L'importance des rites funéraires
 était essentielle, et ne pas ensevelir un mort était un \emph{sacrilège} (cf. Antigone), en dépit de l'impureté qui touchait celui qui s'en chargeait. Le soin d'autrui n'était terminé que quand son corps avait été enseveli selon les règles. Même les plus pauvres cotisaient pour se payer une place dans un tombeau collectif et pour que leur soit rendu le culte mortuaire approprié. 
 Pour que tous les pauvres aient droit à une inhumation décente les églises finançaient un service collectif d'inhumation. Les synagogues en faisaient semble-t-il autant. Encore une fois les premières communautés chrétiennes se sont coulées dans le moule juif. Les fossoyeurs avaient le devoir d'enterrer tous les morts inconnus ou indigents trouvés dans les terrains vagues, au bord des routes ou sur les rivages. Le signe de l'importance symbolique de cette fonction, c'est qu'on s'est longtemps demandé s'ils faisaient ou non partie des clercs mineurs. 

\subsection{Enfants trouvés}

Quand les Écritures prescrivaient de prendre soin des orphelins, est-ce que ce mot recouvrait les enfants abandonnés anonymement, les enfants trouvés ? L'étymologie n'interdit pas de le penser. Le grec \emph{orphanos} désignait en effet l'enfant privé de l'un ou l'autre de ses deux parents, notamment de son père. En ce sens les enfants exposés étaient des orphelins, même quand leurs parents étaient bien vivants. Pas plus que les juifs les chrétiens n'avaient le droit moral d'abandonner leurs enfants, même si ceux-ci étaient trop nombreux ou mal formés. Les écrivains chrétiens soutenaient comme un fait d'observation quotidienne que les païens laissaient mourir beaucoup de leurs enfants. Mais les fidèles pouvaient avoir les mêmes raisons que les autres d'exposer des nouveaux-nés, sauf à supposer qu'ils aient tous et toujours vécu dans une rigueur morale impeccable, ce qui est peu vraisemblable%
%[16] 
\footnote{Les Épîtres de Paul contiennent nombre de passage où il sermonne vertement ses ouailles pour des fautes morales caractérisées (par exemple la \emph{Première épître aux Corinthiens}, chapitres 5 et 6).} 
: enfant né d'un adultère, d'un inceste, d'un viol, d'une rencontre sexuelle sans lendemain, difficulté d'accepter un enfant mal formé,~etc. Rien ne permet non plus d'assurer que l'assistance de l'église mettait tous les chrétiens à l'abri de la misère, et qu'aucun n'était écrasé sous le poids matériel ou psychologique de ses enfants. On ne saura jamais quel était le pourcentage de leurs enfants que les chrétiens abandonnaient : peut-être était-il faible dans les communautés petites et ferventes, où tout le monde connaissait tout le monde, et surtout si l'assistance mutuelle y fonctionnait correctement ? 

 Les apologistes antiques du christianisme craignaient que la plupart des enfants abandonnés ne soient prostitués (ce qui implique d'ailleurs qu'ils ne croyaient pas qu'ils étaient promis à la mort%
% [17]
\footnote{John \fsc{BOSWELL}, \emph{Au bon cœur des inconnus}, 1993.}%
). Il était louable de les recueillir pour les protéger de ce risque. Mais quel était le statut des enfants ainsi pris en charge ? En effet nul ne pouvait prouver qu'ils étaient nés de conceptions régulières, et les règles de pureté de la \emph{Tora} les classaient parmi les \emph{mamzerim}, les impurs de naissance. La situation de ceux dont on avait bien connu les parents de leur vivant (« orphelins pauvres ») paraissait autrement digne d'intérêt, et le restera jusqu'au \siecle{19}. 

 On peut formuler plusieurs hypothèses :
\begin{enumerate}
% 1°) 
\item On a vu que les juifs étaient opposés à l'adoption telle que la pratiquaient les païens. L'adoption proprement dite, celle qui d'un étranger fait le fils et l'héritier d'une famille et de ses ancêtres (adoption « plénière ») disparaîtra dès que les chrétiens seront en mesure d'orienter les décisions impériales : cela suggère que les chrétiens campaient sur la même position que les juifs. Il paraît donc difficile de croire que les enfants abandonnés aient pu être régulièrement adoptés par eux.
% 2°)
\item Par contre ils pouvaient donner à ces enfants le statut d'\emph{alumnii}, sans confondre l'entrée dans leur \emph{familia} avec l'entrée dans leur famille. Les païens et les juifs le faisaient bien ! Cela n'en faisait pas leurs enfants, sinon en un sens spirituel. Ils pouvaient tout de même les établir dans la vie. C'était une pratique socialement et religieusement valorisée.
% 3°)
\item Enfin rien ne leur interdisait non plus d'élever les enfants abandonnés dans le statut d'esclave, ce qui était une pratique traditionnelle. 
\end{enumerate} 

 En droit celui qui voulait prostituer un enfant trouvé était d'ailleurs obligé de lui donner le statut d'esclave, sans quoi son corps était protégé par la loi. On peut supposer que la plupart des fidèles ne s'autorisaient pas cette pratique, étant donné la véhémence avec laquelle les écrivains chrétiens de l'époque la dénoncent comme une abomination païenne. L'accueil des enfants trouvés pouvait être vu comme une bonne œuvre, dans le contexte de cette époque, du moment que l'accueillant s'interdisait d'exploiter le corps de l'enfant et qu'il s'efforçait de lui donner une bonne éducation. Celui qui prenait un enfant trouvé pour en faire son esclave était certes moins généreux que celui qui l'élevait en ingénu. Mais désormais ce nouveau-né sans droits n'était plus l'enfant de personne ni de nulle part, il faisait partie d'une \emph{familia} dont il recevait le nom et il trouvait une place non infamante, si petite fut-elle.

 Il n'y a pas de raison pour que ce type d'accueil ait laissé des traces dans les archives. Les enfants concernés n'entraient en effet dans les registres d'état civil que lorsqu'ils avaient reçu la liberté d'une personne libre. Sinon ils n'étaient pas enregistrés. La suite de l'histoire suggère pourtant que cette troisième solution a été fréquente, sinon la plus fréquente : dès Constantin les enfants trouvés ont en effet été mis à la charge des autorités civiles (cités, rois, seigneurs...), et ils ont aussi été comptés au nombre de leurs dépendants (« leurs hommes »), comme les futurs serfs et parmi eux. 
 












