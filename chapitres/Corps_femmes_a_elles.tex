% 28.02.2015 :
% ~etc.
% Moyen Âge
% _, --> ,
% Antiquité


\chapter{le corps des femmes est à elles}


 Jusqu'à ce qu'existent des méthodes fiables de contrôle des naissances, les grossesses pré conjugales étaient des catastrophes aux conséquences dévastatrices pour les jeunes filles et pour les stratégies d'alliance de leurs familles. Les parents s'en protégeaient en contrôlant étroitement leurs corps. Il leur fallait entretenir un lourd appareil répressif, d'où les grilles, parloirs et clôtures, les portiers, eunuques, duègnes et chaperons, les ouvroirs et les gynécées, les institutions d'éducation fermées, etc. A cela s'ajoutait la nécessité de la culture de « l'innocence », autrement dit la répression du désir. Cet appareil matériel et idéologique était source de violences, explicites et intériorisées, et de frustrations sans nombre. Il mettait à son service, outre les couvents et autres internats, toutes les représentations religieuses disponibles, de l'idéalisation de la pureté et de la virginité à la sanctification des souffrances et des frustrations subies. Il s'agissait que les filles « à marier » se maintiennent de manière ostensible et vérifiable dans une continence absolue, afin que personne ne puisse en douter. Il fallait que les débordements masculins ne puissent s'exercer à leurs dépens. Leur hymen servait de sceau et prouvait leur « pureté », leur chasteté, et donc leur « vertu », leur « honneur », c'est-à-dire leur capacité à s'identifier aux objectifs de leurs pères et mères, à ne pas se mettre en danger d'être « séduites », à contrôler et contenir elles-mêmes leurs pulsions sexuelles, et donc à montrer sur ce point leur capacité à gérer leur vie au lieu de la subir. 
 
 Même si les garçons étaient bien plus libres que leurs sœurs, il n'était pas question non plus qu'ils soient acculés à un mariage non voulu par un passage à l'acte inconsidéré dans les bras d'une jeune fille de leur monde. Chez eux non plus un certain degré d'inhibition ne nuisait pas, et un bon internat non mixte présentait bien des avantages, d'où la faveur dont jouissait cette formule éducative .

 Jusqu'aux années soixante du vingtième siècle, la législation, les mœurs, les discours dominants et le niveau élevé des aides matérielles à la famille et à la procréation, étaient conformes aux vœux des « populationnistes ». Les allocations familiales n'ont jamais été aussi fortes qu'alors par rapport aux salaires de base, ce qui a contribué à permettre à beaucoup de femmes de rester chez elles élever plus d'enfants qu'elles n'en auraient eu sans cela. Le marché de l'emploi aurait probablement permis à beaucoup d'entre elles de travailler au dehors de leur famille. Si les allocations étaient si substantielles, c'est bien parce que l'atmosphère nataliste d'alors était favorable à cette représentation des familles. Il s'agissait pour l'État de promouvoir les naissances, ce qui justifiait d'aider les familles prolifiques et de ne pas heurter de front les idées de ceux qui les représentaient. C'est dans cette atmosphère idéologique très favorable aux couples conjugaux, aux familles et aux associations qui les représentaient que les enfants du « baby-boom » ont été élevés. 

 Et pourtant la pilule anticonceptionnelle a été autorisée en France quelques années seulement après sa mise au point : en 1967. Et elle l'a été par une assemblée de députés dans laquelle il n'y avait pratiquement que des pères de famille, qui ont plus pensé aux intérêts de leurs épouses et de leur fille qu'à la défense du patriarcat. À peine accessible, elle a été aussitôt largement utilisée par toutes les femmes majeures, célibataires ou mariées, et par bien des mineures. Grâce à la pilule, les femmes pouvaient prendre l'initiative d'une rencontre sexuelle sans obérer leur avenir. Cela a permis de constater que si c'était sans risque de grossesse, bien des parents ne refusaient pas que leurs filles aient une vie sexuelle hors mariage. Il n'était plus nécessaire de donner une valeur à la virginité ou à la chasteté des femmes non mariées. Les filles n'étaient plus contraintes par le risque de grossesse de fuir les garçons ni de nier, de réprimer ou de refouler leurs propres désirs sexuels
\footnote{En Janvier ou février 1968, les garçons de la cité universitaire de Nanterre réclament bruyamment le droit d'entrer dans les chambres des filles de la cité, jusque là sanctuaires (en principe) inviolés. Ce fait divers, grand-guignolesque à nos yeux d'aujourd'hui, n'en a pas moins servi de détonateur à la chaîne d'évènements mémorables qui ont culminé au mois de Mai de cette année-là. Il se trouve que la disponibilité (alors toute nouvelle, mais déjà répandue comme une trainée de poudre) de la pilule anticonceptionnelle, venait d'enlever à cette revendication une part du caractère scandaleux (à tout le moins angoissant pour les pères et mères des jeunes filles) qu'elle aurait eu peu de temps auparavant. Le sexe librement recherché pour lui-même devenait un jeu sans enjeu dramatique. Ce n'est que bien après ce printemps-là que le SIDA est venu lui rendre une gravité nouvelle.}% 
. Elles ont reçu une liberté égale à celle de leurs frères, et l'âge moyen de leurs premiers rapports sexuels (autant qu'on puisse le connaître) est rapidement passé de 21 à 17 ans (comme eux).

 La « pilule » a-t-elle été inventée dès que son emploi est apparu comme acceptable ? Ou bien est-ce plutôt le contraire ? Est-ce son apparition qui a tranché ce qui jusque là était noué ? On peut en effet se demander pourquoi les préservatifs, disponibles en vente libre en pharmacie depuis le début du \siecle{20}, officiellement en tant qu'outil de prévention contre les maladies vénériennes, n'ont pas été employés en France plus largement comme un outil de prévention des naissances, alors qu'ils l'étaient dans d'autres pays ? Et pourquoi la pilule ne s'est pas heurtée à la même réticence ? 

 C'est que ce sont les femmes qui ont la maîtrise de cet outil-là : les députés leur ont en effet accordé le droit de prendre la pilule anticonceptionnelle même en cas de désaccord avec leurs maris. Elles \emph{peuvent} la prendre sans le dire à leurs partenaires. Elles \emph{peuvent} aussi cesser de la prendre sans les prévenir. Et\emph{ ils n'y peuvent rien}. Avec l'appui de la législation et des pouvoirs publics, le Planning Familial, héritier des néo-malthusiens, s'emploie à rendre effective cette liberté pour toutes les femmes, mineures comme majeures. 

 C'est dans la foulée de cette première mesure que l'avortement a été autorisé par la loi. Il ne s'agissait plus de l'avortement à la romaine : celui de l'épouse ou de l'esclave sur l'ordre du \emph{pater familias}, ou avec son accord exprès. Désormais une femme peut prendre seule l'initiative d'un avortement, en dépit du refus de son compagnon, comme elle peut garder leur enfant même s'il lui demande d'avorter. Certes en accouchant « sous X
\footnote{... mode d'accouchement ouvert aux femmes mariées autant qu'aux autres, puisqu'elles n'ont pas à donner leur identité, et qui a des origines très lointaine, dont témoigne la pratique obsessionnelle du secret qui entourait jusqu'au \siecle{20} les maternités des hôpitaux.}
 », les femmes ont toujours pu priver de paternité, en toute légalité, le géniteur de l'enfant qu'elles portaient, mais il n'est guère vraisemblable qu'elles aient souvent eu recours à ce moyen pour priver leur mari de la progéniture qu'il avait engendrée (l'immense majorité des accouchements sous X concernent des femmes seules).

\chapter{Nouveau regard sur les violences sexuelles}

Depuis un demi-siècle toutes les formes de violences sexuelles, qu'elles soient extra ou intra familiales, ont été regardées avec un oeil nouveau. Très progressivement la société a pris conscience de la gravité de leurs effets sur leurs victimes. Il nous est devenu moins difficile de nous identifier aux souffrances de celles-ci. Cela s'est traduit par la requalification de certaines actes délictueux, et surtout par une nouvelle façon d'écouter les plaignants et plaignantes et d'accorder crédit à leur parole.  
 


 L'abondance actuelle, depuis les années 1985-1990, des discours sur les \emph{abus} sexuels intra familiaux 
\footnote{... comme s'il y avait un usage correct du sexe entre les générations différentes au sein des familles ?} 
signifie-t-elle qu'il s'en commet plus qu'autrefois ? Si l'on en croit le témoignage de Jeannine \fsc{NOEL} (1965) il est permis d'en douter : selon elle entre le quart et le tiers des adolescentes placées à l'Hôpital Hospice Saint Vincent de Paul%
%[2]
\footnote{À cette époque c'était encore le Foyer de l'Enfance de Paris (anciennement « dépôt de l'Assistance Publique ») recevant (souvent avant une orientation ailleurs) tous les jeunes dont les parents ne pouvaient pas s'occuper ou de l'autorité desquels ils avaient été soustraits par décision de justice. On plaçait et place toujours dans les foyers de l'enfance les jeunes qui n'ont pas d'autre lieu où aller, quelle que soit la raison qui les a mis dans cette situation.} 
 au cours des années cinquante du \siecle{20} avaient été confrontées à des problèmes de ce genre : la situation ne semble pas être pire aujourd'hui. 

 Les travailleurs sociaux d'autrefois ont toujours su que même dans les « meilleures » familles il pouvait se passer des choses « pas très catholiques ». Durant tout le \siecle{19} les visiteurs des pauvres et les médecins n'ont pas arrêté de dénoncer la promiscuité des logements des pauvres et de plaider pour qu'à défaut d'une chambre par enfant il y ait au moins une chambre pour les garçons et une chambre pour les filles, avec un lit par enfant, et d'abord et avant tout une chambre pour le couple parental. Cela dépassait évidemment le seul souci d'hygiène. S'ils n'en disaient pas plus, tous comprenaient ce que ces euphémismes voilaient pudiquement, à savoir que l'excès de proximité, contraint ou choisi, conduisait souvent à la promiscuité sexuelle, et pas seulement dans les familles mal logées. 
 
Naguère la justice ne voulait rien entendre non plus des violences conjugales tant qu'il n'y avait pas de lésions physiques sérieuses
\footnote{\frquote{en vertu de l'adage\emph{"Entre l'arbre et l'écorce il ne faut pas mettre le doigt"}}.}. La notion de \emph{viol conjugal} paraissait absurde par définition.
 
 Les paniers de linge sale et les cadavres des placards familiaux étaient protégés par la rigueur du secret absolu auquel étaient tenus médecins, ecclésiastiques, infirmières visiteuses,~etc. Pourquoi n'est-ce qu'aujourd'hui que le caractère absolu des secrets professionnels a été mis en question ? Pourquoi n'est-ce qu'aujourd'hui que l'évocation des sévices intra familiaux obtient un tel effet ? Pensait-on que ces délits et ces crimes, aussi condamnables qu'ils étaient, ne pouvaient être traités pénalement, et qu'il était préférable de les recouvrir du \emph{manteau de Noé} ? Sans doute avait-on peur d'ébranler l'autorité et la représentation d'une institution familiale sacralisée, et préférait-on lui sacrifier ses victimes ? 
 
 



 
