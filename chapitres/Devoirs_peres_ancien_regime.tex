% Le 10 mars 2015 :
% Antiquité
% ~etc.
% Moyen Âge
% Droit

\chapter{L'évolution des pratiques éducatives}


 \section{Paternité et absolutisme}


Vécu comme un père par ses sujets, le roi s'identifiait à son tour à tous les pères de famille. Eux et lui étaient autant de représentants de Dieu « le Père » et ils se confortaient les uns les autres. Il en avait plus ou moins été ainsi depuis toujours, mais à la fin du Moyen Âge le fonctionnement des familles semble avoir eu tendance à se rigidifier dans un patriarcat de plus en plus rigoureux, en même temps que les états montaient en puissance et que les doctrines absolutistes gagnaient de l'audience. 

Qu'ils soient protestants ou catholiques les philosophes, les théoriciens du droit et les chantres de l'absolutisme (Jean Bodin, Omer Talon, Bossuet, Thomas Hobbes,~etc.) soutenaient la nécessité d'un pouvoir fort, incarné par un souverain aussi absolu que possible : absolu, c'est-à-dire sans contre-pouvoirs significatifs. On peut supposer que cela découlait pour une grande part de l'expérience des guerres, civiles ou entre états, dans lesquelles les européens se sont laissés entraîner par leurs divergences entre options religieuses. Cette expérience a révélé la violence mortelle que peuvent provoquer ces divergences. Elle a aussi révélé les limites de la capacité du pape et des évêques à réguler pacifiquement les conflits d'interprétation, ce qui les a délogés de leur position d'autorités millénaires et de partenaires autonomes et incontournables des autorités civiles. Bon gré mal gré les européens s'en sonr donc remis à "\emph{César}", quitte à s'accomoder du despotisme des souverains absolus ("\emph{cujus regio ejus religio}") et des injustices de la raison d'état. Tout valait mieux à leurs yeux que les désordres d'un monde où chacun serait un loup pour l'autre. 

Les ecclésiasriques eux-mêmes se sont ralliés à cette position : les protestants bien entendu, mais aussi les catholiques. Ainsi Pierre de Bérulle (fondateur de l'Ordre de l'Oratoire) écrivait en 1623, dans un discours\footnote{\emph{Discours de l'État et des grandeurs de Jésus}.} au Roi (Louis~XIII)  :
    \enquote{\emph{un monarque est un Dieu selon le langage de l'écriture : un Dieu non par essence mais par puissance ; un Dieu non par nature mais par grâce ; un Dieu non pour toujours mais pour un temps. Un Dieu non pour le Ciel mais pour la Terre. Un Dieu non subsistant, mais dépendant de celui qui est le subsistant par soi-même ; qui étant le Dieu des Dieux, fait les rois Dieux en ressemblance, en puissance et en qualité, Dieux visibles, images du Dieu invisible}}. Jusqu'au milieu du \siecle{18} l'image de l'autorité était globalement positive, et l'exercice que les rois et les pères (et avec eux les « pères spirituels » de tous ordres) faisaient de leurs pouvoirs était regardé comme légitime et bénéfique. Dans ce cadre de pensée s'opposer au souverain comme aux pères c'était faire preuve de présomption et peut-être s'opposer à Dieu lui-même. 
    
    Dans quelle mesure cette vision du pouvoir et de la paternité a-t-elle rejailli sur l'image que les gens d'alors se faisaient de Dieu ? Ils prêtaient à celui-ci une grande dureté et même de la cruauté : arbitraire de la grâce, arbitraire de la prédestination, dureté des exigences morales, poids de la culpabilisation, menace omniprésente de l'enfer,~etc. Mais on peut aussi bien se demander si ce n'est pas  dans les thèses des théologiens de la fin du Moyen Age que l'absolutisme des pères et des rois a pris sa source \footnote{Ils valorisaient en effet sans limites la toute-puissance divine en dépit des difficultés que ce parti-pris impliquait. Selon Jean-Claude Monod : "\emph{L'importance du nominalisme à la fin du Moyen Âge tient à ce que ce courant de pensée a mis en crise le système scolastique en voulant pousser l'homme à une capitulation sans condition dans l'acte de foi, et a retiré à la théologie toute tâche de médiation entre la connaissance et la foi. Ainsi en est-il de la souveraineté absolue de Dieu : volonté insaisissable et opaque "potentia absoluta", le Dieu du nominalisme et ses "décrets" se situent au-delà de toute tentative de compréhension par l'esprit humain. Tout ce qui est fait peut être défait, toute loi peut être suspendue, nulle garantie ne doit être attendue de Dieu, dont l'entendement est incommensurable au nôtre et dont dépend pourtant entièrement notre salut."} in \emph{La querelle de la sécularisation : théologie politique et philosophie de l'histoire de hegel à Blumenberg}, Paris, Vrin, 2002. Comment penser la liberté et la responsabilité des individus si Dieu connaît à l'avance tout leur avenir ? Comment imaginer Dieu comme bon s'il n'est lié par aucune exigence de justice ? etc. Ces difficultés ont contribué à l'éclatement de la chrétienté médiévale.}.       
    
    
    Puisqu'il n'était plus possible de faire confiance à Dieu ni à ses ministres pour contenir les désordres il fallait conforter les autres autorités. Les pères se voyaient donc confirmés dans leur devoir de maintenir leur maison en bon ordre, dans le respect des lois civiles et religieuses. On attendait d'eux qu'ils le fassent sans faiblir, quelles que soient les difficultés. Pour y parvenir il leur était reconnu une grande part de la puissance paternelle des romains. Dans les pays de Droit écrit, comme le sud de la France, revenus avant la fin du Moyen Âge à une application stricte du Droit romain, leur puissance ne cessait qu'avec leur mort. Partout leur mission éducative impliquait le \emph{droit de correction}. On considérait que c'était pour eux un devoir moral et social que de corriger les enfants \emph{et les épouses} indisciplinés. Jusqu'au \siecle{18} (au moins) il était admis qu'une tendresse excessive était plus dommageable, et donc plus coupable, qu'une sévérité excessive : « {qui aime bien châtie bien} ». Montaigne nous dit qu'il fut placé de sa naissance à l'âge de quatre ans chez des bûcherons, puis mis en pension en collège à partir de six ans. Il dit s'être trouvé mieux de cette enfance loin de sa famille... parce qu'il lui semblait que son père était « trop tendre%
% [1] 
\footnote{En justifiant la décision de son père par son « excès de tendresse » Montaigne nous fournit un bel exemple de ce qu'on désigne aujourd'hui sous le nom de « fidélité » ou de « loyauté » des enfants, et des trésors de compréhension dont ils sont prêts à faire preuve face à toutes les décisions, quelles qu'elles soient, que leurs parents ont pu prendre.} 
» !

 Le roi soutenait l'autorité des époux sur leur épouse et leurs enfants, et il leur prêtait main-forte s'ils le demandaient, entre autres moyens par les \emph{lettres de cachet} ordonnant sans jugement\footnote{...ancêtre des placements administratifs actuels, dont il faut reconnaître qu'ils sont mieux contrôlés qu'alors par les autorités judiciaires. Il est infiniment plus aisé aujourd'hui de mettre en question leur pertinence parce que nous n'idéalisons plus la parole des pères ni des autres autorités. Au contraire nous les tenons en suspicion.} l'incarcération de l'enfant récalcitrant, mineur ou majeur, ou de l'épouse indigne, volage ou frivole ou de mauvais caractère,~etc. S'il le jugeait nécessaire, il pouvait se substituer de sa propre initiative%
% [2] 
\footnote{De même que lorsqu'il s'agit de ses enfants un père n'attend pas d'être saisi : par définition il parle en leur nom et à leur place (et en Droit romain même quand ils sont adultes).} 
aux pères défaillants dans leur fonction de faire régner l'ordre dans leurs familles. 
 Mais il se devait aussi de contrôler qu'ils n'abusaient pas de leurs pouvoirs : leur droit de correction n'était pas un droit de vie ou de mort. Jamais les parents n'ont été autorisés à estropier leurs enfants, et l'appui donné par la force publique à leurs décisions n'était pas automatique.

 

\section{Les enseignements}

    
   Il n'est pas question ici de faire une histoire de l'enseignement, mais seulement d'en esquisser les traits qui ont rapport à notre sujet\footnote{\\\fsc{FURET} et \fsc{OZOUF}, \emph{Lire et écrire, l'alphabétisation des français de Calvin à Jules Ferry}, 1977.
\\Maurice \fsc{CAPUL}, \emph{Internat et internement sous l'ancien régime, contribution à l'histoire de l'éducation spéciale}, Thèse d'état, CTNERHI-PUF, Paris, 1983-1984.
\\Martine \fsc{SONNET}, {« Une fille à éduquer », in \emph{Histoire des femmes en Occident}, III, \siecles{16}{18}}, Collectif, sous la direction de Georges \fsc{DUBY} et Michelle \fsc{PERROT}, 2002, Chapitre 4, p. 131 à 168.
\\Sous la direction de Marie-Madeleine \fsc{COMPERE} et Philippe\fsc{SAVOIE}, \emph{L’établissement scolaire. Des collèges d'humanités à l'enseignement secondaire, XVIe-XXe siècles}, numéro spécial 90 de la revue \emph{Histoire de l’éducation}, mai 2001
\\ Marie-Madeleine \fsc{COMPERE}, \emph{Du collège au lycée. Généalogie de l'enseignement secondaire français (1500-1850)}
Collection Archives (n° 96), Gallimard,1985 
\\sous la dir. de Marie-Madeleine \fsc{COMPERE} et d'André \fsc{CHERVEL}, \emph{Les Humanités classiques}, Paris : Institut national de la recherche pédagogique, 1997
\\Marie-Madeleine \fsc{COMPERE},	\emph{L'histoire de l'éducation en Europe : essai comparatif sur la façon dont elle s'écrit} Paris : Institut national de recherche pédagogique ; Bern : P. Lang, c1995. }.

 
Jusqu'à la fin de l'ancien Régime il n'existait rien qui ressemblât à un enseignement public, et l'ensemble du domaine scolaire était sous le contrôle et à la charge des évêques que des décisions royales répétées confirmaient dans leur droit de contrôle mais aussi dans leurs obligations. L'enseignement était assumé en majeure partie par des prêtres et des religieux(ses) et les biens ecclésiastiques en finançaient la plus grande part. 
Le réseau des écoles autres que monastiques s'est développé depuis le début du Moyen Âge (cf. troisième Concile de Vaison, en 529) à partir des \emph{écoles cathédrales}, d'une part vers l'enseignement élémentaire avec les \emph{petites écoles} (écoles primaires), d'autre part vers l'enseignement supérieur (et secondaire) avec les \emph{universités} et leurs \emph{collèges}. À partir des derniers siècles du Moyen Âge des \emph{petites écoles} paroissiales existaient dans toutes les villes importantes, fondées par les curés, ou par les municipalités, et ordinairement par les deux à la fois. Elles étaient placées sous le contrôle du Chapitre de la Cathédrale. Au sein de celui-ci un chanoine exerçait cette responsabilité l'\emph{écolâtre},  le \emph{chantre} ou le \emph{chancelier}... C'est lui qui jusqu'à la fin de l'ancien régime agréera tous les candidats à l'enseignement, agrément sans lequel nul n'avait le droit d'enseigner sur le territoire sous sa juridiction.

Enseignement primaire

Dans les petites écoles à côté des connaissances profanes (d'abord la lecture, le calcul, souvent l'écriture, mais pas toujours) on enseignait aussi la religion, les disciplines du corps et de l'esprit, les bonnes manières de se conduire. Leur mission était en effet d'éduquer autant que d'enseigner. L'instruction, une fois entendu qu'elle se devait d'inclure la religion, était considérée comme la meilleure défense contre l'envie de mal faire. Curés ou pasteurs protestants, parents et autorités locales étaient d'accord sur ce point. D'autre part les citadins voyaient aussi en elle la meilleure arme pour trouver et pour garder un travail, ce qui avait à la fois un intérêt économique et un intérêt social. À partir de la Réforme et du Concile de Trente cette foi en l'enseignement s'est exprimée en un véritable apostolat. C'est pourquoi divers ordres enseignants ont été créés au fil des siècles :
%\begin{description}
%\item[
\siecle{12} : religieuses bernardines,
\siecle{15} : Frères de la vie commune, Minimes,
\siecle{16} : Jésuites, Ursulines, Bénédictins de Saint-Maur, Prêtres de la Doctrine chrétienne,
\siecle{17} : Oratoriens, Dames de Saint-Maur, Piaristes, Religieuses de Notre-Dame, Dames de la Providence, Frères des écoles chrétiennes,~etc. (pour ne citer que les principaux).
%\end{description}

 Les petites écoles s'adressaient aux « enfants des pauvres », c'est-à-dire, dans le langage d'alors, de tous ceux dont les ressources étaient précaires, ceux qui n'avaient pas de rentes, de quelque nature qu'elles soient, et qui devaient gagner leur vie en travaillant. Il s'agissait donc de l'essentiel de la population des villes. Mais les petites écoles ne pouvaient pas toujours être complètement gratuites (pas plus que les universités). Elles étaient donc à la portée des bourgeois aisés, des commerçants et artisans, mais pas toujours à celle des autres. Lorsque les paroisses ne pouvaient pas exempter ces derniers des frais de scolarité, ce qui était le cas lorsque l'ensemble de leurs paroissiens étaient réellement pauvres, seuls de généreux donateurs et surtout des ordres religieux pouvaient les prendre en charge (cf.  les « écoles de charité »). Les religieux bénéficiaient en effet d'une sécurité financière, d'une surface sociale et d'un entregent que ne pouvaient avoir des particuliers ou des communes pauvres. Certains ordres avaient d'ailleurs explicitement pour vocation d'assurer gratuitement l'enseignement des indigents.

Beaucoup d'enfants n'étaient pourtant pas scolarisés, même dans les villes où l'enseignement était gratuit : leurs parents avaient trop besoin du produit de leur travail, ou bien ils ne voyaient aucune utilité à un apprentissage scolaire. Même aux yeux de ceux qui envoyaient leurs enfants à l'école il n'était pas toujours évident qu'il faille que ceux-ci soient scolarisés avec assiduité pendant plusieurs années. Beaucoup, et peut-être meme la plupart, se contentaient des quelques mois ou années nécessaires pour apprendre à lire et/ou à écrire. Les enfants de famille aisée étaient traditionnellement confiés à un précepteur.  

Quant à l'instruction des paysans, jusqu'à la fin de l'ancien régime elle n'était pas jugée nécessaire. L'illettrisme n'avait pas d'incidence sur leur productivité étant donné le niveau des techniques alors en usage. D'autre part les maîtres et seigneurs craignaient qu'une instruction même minime ne les rende « raisonneurs » et « arrogants »\footnote{Voltaire, représentatif de sa classe et de son temps, sera lui aussi de cet avis.}.

Enseignement secondaire 

Les écoles cathédrales et les écoles monastiques ont été créées dès la fin de l'Antiquité pour fournir l'Église en clercs, mais elles ont toujours reçu un petit contingent d'élèves promis à la vie civile. À partir du \siecle{10} la croissance des villes a provoqué la demande d'une instruction de niveau universitaire (c'est-à-dire à l'époque secondaire et supérieur). À partir du \siecle{12} les universités se sont créées comme des corporations autogérées de professeurs, sur l'impulsion de ces derniers et avec l'appui des autorités civiles. Elles étaient sous l'autorité conjointe de l'évêque du lieu et du pape (arbitre des conflits avec l'évêque), et leur personnel comme leurs étudiants bénéficiaient des avantages et exemptions attachés aux clercs. En lien avec les universités ont été créés des collèges avec internat pour les boursiers pauvres, sur le modèle des écoles monastique (ex. : le collège qui deviendra la Sorbonne). Peu à peu ces collèges d'universités sont devenus des lieux d'enseignement appréciés. 

La formule du collège s'est généralisée à partir de la Renaissance tout en se transformant profondément (externat, élèves promis à la vie laïque, etc.). A partir du XVIème siècle de nombreux nouveaux collèges ont été créés à la demande des municipalités et/ou des évêques. Les initiatives étaient très décentralisées et les créations partaient le plus souvent des besoins et des demandes locales. Au fil du temps beaucoup de ces nouveaux collèges ont été confiés à des ordres religieux comme les jésuites ou les oratoriens. Tous les collèges n'étaient pas de "plein exercice", avec des classes de tous niveaux, philosophie comprise. En fait beaucoup d'entre eux (les "petits collèges") se contentaient de quelques classes, le cursus secondaire devant s'achever dans un autre établissement\footnote{..quand la commune ne se bornait pas à entretenir un seul professeur de latin (une \emph{régence latine}) pour les quelques élèves concernés.}. 

Au fil du temps la scolarité secondaire a eu tendance à se décentraliser, sans que pour autant le nombre d'élèves du secondaire n'ait augmenté. En fait il est resté stable et comparativement bas pendant très longtemps\footnote{Selon un rapport établi en 1843 par A. F. Villemain  et cité par Antoine LEON (\emph{Histoire de l'enseignement en France}, Que Sais-je ?, PUF, Paris, 1967) il existait à la veille de la Révolution 562 collèges avec 73000 élèves, dont 40000 boursiers : 178 collèges congréganistes et 384 collèges dépendant des universités ou gérés par des communes ou des particuliers. En 1812 il y avait 36 lycées et 337 collèges publics avec 44000 élèves, et 1000 autres institutions et pensionnats privés pour 27000 élèves, soit 71000 élèves au total. En 1880 il y avait environ 150000 élèves dans les lycées et collèges, et 500000 en 1940.}), sous la pression de municipalités poussées par les familles désireuses de garder leurs fils chez elles. Il faut aussi souligner que beaucoup se contentaient d'une scolarité secondaire réduite à quelques années. 
 
 Pour les parents des collégiens l'éducation était un investissement familial, et cela même quand ils se destinaient à devenir des clercs (ce qui a été le cas d'une minorité significative jusqu'au \siecle{18}). Cela justifiait qu'ils soient improductifs pendant leurs années de scolarité. Quant aux jeunes garçons sans ressources mais doués pour les études et en particulier ceux qui se destinaient à entrer dans les ordres (soit par amour pour la vie religieuse soit par désir de promotion sociale) ils pouvaient bénéficier de bourses : le pourcentage de boursiers semble avoir été assez important.


Beaucoup de pédagogues avaient une image positive de l'internat car il mettait les jeunes dans des conditions éducatives plus favorables à leurs yeux que l'externat. À partir de 1640 la diffusion de la doctrine janséniste, en dépit de la résistance des jésuites et du roi, mais avec la sympathie des oratoriens et des parlements, a encore accru la méfiance traditionnelle face aux tendances spontanées des adolescents. Elle a conforté la répression du désir, et du désir sexuel en particulier. Dans un internat le jeune interne était coupé de ses parents d'une manière rigoureuse dans un cadre totalement maîtrisé (lieux, communications, temps). Il était contenu fermement (murs, grilles, portier, clôture, clés,~etc.) à l'abri du monde extérieur. Il s'agissait de donner de saines habitudes à son corps et à son esprit.  L'internat protégeait les « enfants de famille » contre les « mauvaises influences » qui les « pervertissaient », tout en préservant les « filles honnêtes », la « paix des ménages », et la « tranquillité publique » des « débordements de la jeunesse ». Il était pour les parents une garantie contre les « erreurs de jeunesse » qui réduisent à néant les meilleures stratégies familiales. Les pensionnaires ne perturbaient pas la vie de la cité où ils étudiaient comme étaient capables de le faire des écoliers externes de tous âges vivant loin de leur famille et sur lesquels les logeurs comme les maîtres n'avaient guère d'autorité. 

Quant aux jeunes filles de bonne famille, la clôture des couvents leur interdisait toute rencontre avec les jeunes gens de leur âge. Elle protégeait leur « vertu » et leur réputation en attendant que leurs parents les marient. Il semble que la durée de leurs séjours dans les couvents ait été très inférieure à celle de leurs frères dans les collèges. L'enseignement qui leur était dispensé était habituellement moins poussé (exclusion du latin,~etc.). Par contre celles qui se destinaient à la vie religieuses pouvaient bénéficier d'un enseignement qui en faisait des soeurs "de choeur", des lettrées capables de chanter en latin les offices en comprenant ce qu'elles chantaient, et d'enseigner aux jeunes filles pensionnaires.

Mais entre l'externat des collèges, souvent gratuit (ex. les collèges jésuites) ou presque, et la pension des internats l'écart des coûts était énorme%
%[5]
\footnote{Selon Martine \fsc{SONNET}, la pension d'un seul enfant, garçon ou fille, représentait presque la totalité du salaire d'un ouvrier (« une fille à éduquer », Chapitre 4 de \emph{l'Histoire des femmes en Occident}, III, \siecles{16}{18}, p. 146). C'est pourquoi en 1760 les internats parisiens n'accueillaient que 13~\% de la population scolaire de la ville, et il semble qu'il en était de même ailleurs.}% 
. Aux familles qui ne pouvaient payer les frais d'une pension, c'est-à-dire la plupart, seul l'externat était accessible, en vivant en ville chez ses parents (d'où la pression des municipalités pour créer un collège, un petit collège, ou au minimum une classe de latin une"régence latine", pour gagner quelques années de scolarité sans recourir à la pension) ou chez un parent, ou chez un logeur peu exigeant. Dans les collèges il y avait donc ordinairement beaucoup plus d'externes que d'internes et souvent il n'y avait pas d'internat du tout (ainsi la grande majorité des collèges jésuites n'avait pas d'internat). Les professeurs de collège n'avaient guère envie d'assumer les contraintes et les soucis d'un internat. Mais le prestige théorique de la formule et la nécessité pratique de regrouper une grande part des collégiens originaires des campagnes et des bourgs loin de leur domicile a fait qu'ils ont grossi peu à peu mais \emph{surtout à partir du milieu du \siecle{18}}\footnote{Leur croissance s'est poursuivie longtemps et leur réseau n'a été achevé qu'au début du \siecle{20}. Le nombre de places d'internat semble s'être maintenu ensuite sans grands changements jusqu'aux années soixante du \siecle{20} où il a commencé à baisser.}. 

 
 

Les collégiens étaient confiés soit à l'un des ordres religieux spécialisés à partir de la Renaissance dans l'enseignement (Jésuites surtout, mais aussi Oratoriens, Dominicains,~etc.) soit à des collèges dépendant des universités, soit à des collèges dépendant des municipalitésoù enseignaient des clercs recrutés sur place, de niveau universitaire inégal et aux motivations fluctuantes (beaucoup parmi ces derniers gagnaient leur vie en enseignant en attendant d'obtenir un bénéfice ecclésiastique plus intéressant et plus lucratif). 

Les collèges étaient ouverts à la ville dans les murs de laquelle ils étaient établis. Ils en formaient souvent l'un des fleurons les plus prestigieux. Ils y entretenaient une vie intellectuelle et mondaine active et d'autant plus valorisée que les autres sources de distraction étaient rares. Ils proposaient aux collégiens de s'investir dans la découverte du savoir, et celui-ci était ressenti par leurs enseignants et leurs parents comme quelque chose qui en valait la peine. Ils entraient dans une aristocratie de l'esprit. À l'époque dans toute l'Europe l'enseignement secondaire et supérieur se faisait en latin sans lequel on savait peut-être lire, mais on n'en demeurait pas moins un \emph{illettré}%
% [7]
\footnote{Dans toute l'Europe les thèses seront encore soutenues en latin durant la plus grande partie du \siecle{19}.}% 
. C'était la langue vivante, la langue de communication des communautés intellectuelles du temps. Mais depuis l'\emph{ordonnance de Villers-Cotterêts} (1539) qui imposait le français comme langue administrative du Royaume, il n'était plus possible de tenir un \emph{office} public si on ne le maîtrisait pas suffisamment. La langue française n'était encore que le parler de l'Île-de-France, domaine du roi. Partout ailleurs c'était une langue étrangère qui allait mettre très longtemps à déloger les langues locales des places et des marchés. L'enseignement du français reçu dans les collèges (meme si l'accent était mis sur le latin) était donc incontournable pour entrer dans les professions libérales, la fonction publique ou le clergé. 

 Le collège était par conséquent un moyen sûr de promotion individuelle et familiale. Pour les gens ordinaires c'était la seule voie d'accès aux emplois prestigieux et qualifiés. Le fils de famille confronté à l'épreuve de la vie loin de ses parents continuait de dépendre d'eux. Ils payaient sa pension : il continuait de manger leur pain. Il continuait de correspondre avec eux. Il les retrouverait aux prochaines vacances s'ils ne venaient pas le voir avant. Quand la discipline et le travail intellectuel lui pesaient trop il pouvait se dire avec assez de vraisemblance qu'il était en train d'acquérir à ce prix les moyens d'atteindre un statut personnel valorisant, et qu'il s'inscrivait dans le projet de ses parents. En acceptant de se soumettre à cette exigence il pouvait espérer devenir un membre puissant et respecté de sa communauté d'origine : cela présentait l'allure d'une épreuve initiatique \footnote{Au fil du XVIIIème siècle un enseignement de niveau secondaire \emph{sans latin} a été mis en place à l'initiative en particulier des \emph{frères des écoles chrétiennes} à l'intention de jeunes qui se destinaient à des métiers techniques. De la meme façon ont été créées des écoles militaires, avec internat obligatoire, à destination des jeunes nobles, qui n'accordaient au latin qu'une place mesurée, au bénéfice de l'histoire ou d'autres matières plus adaptées. De meme encore ont été créées des écoles techniques comme l'école des Ponts et Chaussées. Mais ces nouveautés sont déjà le signe du changement d'époque de la fin deuxième moitié du XVIIIème siècle.} .

\section{La correction paternelle}

 Tous les jeunes \emph{de famille} n'entraient pas docilement dans les projets parentaux. Certains d'entre eux entraient en conflit ouvert avec leurs parents au-delà des normes reçues (éminemment variables suivant les siècles et les lieux) : vagabonds, fugueurs, jeunes aux fréquentations suspectes, exclus pour indiscipline de collèges successifs, fauteurs de vols domestiques ou d'actes « d'inconduite sexuelle », d'insultes et de voies de faits, « libertins », c'est-à-dire jeunes rétifs à toute mesure éducative,~etc. 

 À la demande de leurs parents, ces jeunes peuvent être traités en tout comme les délinquants con\-dam\-nés. Pour les enfants difficiles des familles aisées il y avait des solutions payantes dans les sections des collèges et internats contemporains affectés à la « correction ». Ceux qui n'en avaient pas les moyens étaient internés avec les délinquants condamnés, dans les sections « de force » des hôpitaux, où s'effectuaient les peines de prison. Leurs parents payaient une pension qui tenait compte de leurs ressources. 

 À partir de la fin du \siecle{17} et de plus en plus souvent au fil du \crmieme{18}, les \emph{enfants de famille}, garçons et filles mineurs \emph{et majeurs}, qui avaient commis de vrais actes de délinquance, mais aussi ceux qui donnaient simplement du mécontentement à leurs parents par leurs fréquentations, leur mauvaise conduite, leur indocilité, leur violence aveugle ou leur absence de sens commun (« insensés »), leurs dépenses inconsidérées, ou leurs dettes de jeu, pouvaient, sur la demande de ces derniers qui exerçaient ainsi leur droit de correction, faire l'objet d'une \emph{lettre de cachet}, c'est-à-dire d'une \emph{décision administrative d'internement} dans un hôpital, une prison, une forteresse, un couvent, un collège, ou même leur déportation aux colonies. Les lettres de cachet, qui ont une origine très ancienne, bien antérieure au \siecle{17}, pouvaient aussi être accordées à l'encontre de conjoints aux comportements répréhensibles (cette mesure a beaucoup plus souvent frappé les épouses que les époux). 

 L'autorité publique n'était pas obligée d'accorder satisfaction aux demandes qui lui était faites, et restait seule juge de l'opportunité de la mesure. Elle était surtout sollicitée à Paris, notamment par les couches populaires, contrairement aux provinces où l'internement administratif était moins facile à obtenir et où les couches populaires n'y avaient guère recours. Même si au fil du temps les lettres de cachet ont fait l'objet de critiques de plus en plus virulentes et si les autorités publiques y répugnaient de plus en plus, les demandes se sont faites de plus en plus nombreuses au fil du \siecle{18}. 

 En effet les familles sollicitaient ces lettres comme une grâce : cela leur évitait la honte causée par la publicité du recours à la justice, le coût d'un procès, et aussi la publicité de la mesure d'enfermement. La réputation du jeune (ou de l'adulte) ainsi placé pouvait s'en relever plus facilement. Cela évitait le contrôle par la justice de la nature exacte des faits et de la proportionnalité des sanctions aux dommages et délits constatés, ce qui permettait à l'occasion à d'authentiques délinquants bien nés d'échapper à peu de frais aux conséquences normales de leurs actes. 

Mais cela permettait aussi aux parents abusifs d'exercer des pressions sur leurs enfants rétifs à leurs projets (ce qui expliquait les critiques de plus en plus virulentes des lettres de cachet au fil du \siecle{18}), à une époque où le consentement des parents était exigé à tout âge et pour tout mariage sous peine d'exhérédation, et où bien des entrées en religion étaient imposées par eux sans tenir compte des désirs du ou de la jeune concerné. 

\section{Enfants « adoptifs »}

 On a vu que dans le but de défendre le mariage monogame et indissoluble, l'Église a tout fait depuis l'Antiquité pour que les enfants illégitimes ne puissent pas devenir des héritiers de plein exercice. C'est pour cette raison que l'adoption était interdite, et pourtant... De l'Antiquité à la fin de l'ancien régime, on peut observer en nombre non négligeable des situations plus ou moins proches d'une adoption, où une personne, souvent un ecclésiastique (cf. \hbox{Villon}, adopté par un chanoine), souvent aussi un couple sans enfants, exerçaient la puissance paternelle sur un enfant qui n'était pas né d'eux et qu'ils élevaient jusqu'à sa majorité. C'était par exemple le cas à Lyon, où les recteurs de l'Hôtel-Dieu « adoptaient » ainsi des orphelins. 

 Ces situations d'\latin{alumnii} (adoptions simples) étaient parfois sanctionnées par des actes juridiques où les nourriciers faisaient un legs à l'enfant devant un procureur fiscal, et où ils s'engageaient à l'élever, instruire et établir matériellement à leurs frais comme leur propre enfant. Pour autant cela ne faisait pas de lui un membre de leur famille ni un héritier. 

 En principe seul un enfant légitime sans parents pouvait bénéficier de ce dispositif. Souvent, probablement le plus souvent, il était orphelin, mais des enfants légitimes pouvaient aussi être abandonnés solennellement par leurs parents, qui reconnaissaient par écrit qu'ils renonçaient à leur puissance paternelle, et à l'héritage de leur enfant s'il décédait. Pour autant ce dernier ne changeait ni de parenté ni de nom. Quand il possédait des biens, l'adoptant, tel un tuteur, les gérait jusqu'à sa majorité et il était responsable sur ses propres biens de sa gestion. 

 Les enfants abandonnés, nés de parents inconnus, ont très longtemps été exclus de ce genre de prise en charge%
% [8]
\footnote{À Lyon jusqu'en 1765. Ensuite ils y ont été traités comme les autres. Ce n'est que dans les dernières années avant la Révolution que les idées ont changé sur ce point : un signe de l'évolution qui s'est faite dans les esprits au fil du \siecle{18} et qui est apparue au grand jour à partir des années 1760-1770.}%
. Pourtant il était courant que des personnes accueillent pour l'élever un enfant abandonné à eux confié par un hôpital ou par une paroisse, qu'elles refusent d'être rémunérées pour l'élever, qu'elles le gardent jusqu'à sa majorité et qu'elles l'établissent dans la vie, ce qui en fait ressemblait beaucoup à la situation des enfants nés légitimes et juridiquement « adoptés ». Si aucun de leurs héritiers légitimes ne s'y opposait, elles faisaient de lui l'un de leurs héritiers. Mais il n'était pas question pour cet enfant d'hériter d'une fonction impliquant l'exercice public du pouvoir. 

 Derrière les mots employés il n'est pas toujours facile de reconnaître les situations réelles : adoption simple ? tutelle ? parrainage ?%
% [9]
\footnote{Cf. Jean-Pierre \fsc{GUTTON}, \emph{Histoire de l'adoption en France}, 1993.} 


