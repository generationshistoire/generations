% Le 10 mars 2015 :
% Antiquité
% ~etc.
% Moyen Âge
% Droit

\chapter{Les éducations}


 \section{Paternité et pouvoir}
 
 Plus ou moins vécu par ses sujets comme un père, le roi s'identifiait à son tour à tous les pères de famille. Ils étaient tous des représentants de Dieu "le Père ". Ces images se confortaient les unes les autres. Il en était ainsi depuis l'antiquité, mais de la fin du Moyen-Age au XVIIIème siècle le fonctionnement des familles semble s'être particulièrement rigidifié, pour des raisons probablement liées à la naissance des états modernes. Qu'ils soient protestants ou catholiques les théoriciens et les chantres de l'absolutisme (Jean Bodin, Omer Talon, Bossuet, Hobbes, etc.) soutenaient la nécessité, la légitimité et la bonté d'un pouvoir fort, incarné par un souverain aussi absolu que possible. Pierre de Bérulle (fondateur de l'Ordre de l'Oratoire) écrivait ainsi en 1623, dans un discours \footnote{ \emph{Discours de l'État et des grandeurs de Jésus}} adressé au Roi Louis XIII  :
    \emph{« un monarque est un Dieu selon le langage de l'écriture : un Dieu non par essence mais par puissance ; un Dieu non par nature mais par grâce ; un Dieu non pour toujours mais pour un temps. Un Dieu non pour le Ciel mais pour la Terre. Un Dieu non subsistant, mais dépendant de celui qui est le subsistant par soi-même ; qui étant le Dieu des Dieux, fait les rois Dieux en ressemblance, en puissance et en qualité, Dieux visibles, images du Dieu invisible. »}. On peut supposer que l'expérience des guerres civiles, pour des motifs religieux ou non, que les européens venaient de subir les conduisait à préférer le poids du despotisme des souverains absolus et les injustices de la raison d'état aux désordres sanglants de l'anarchie et du "chacun contre tous"\footnote{On peut se demander dans quelle mesure cela ne rejaillissait pas sur l'image qu'ils se faisaient de Dieu le Père lui-même et les duretés et l'arbitraire qu'ils lui prêtaient : arbitraire de la grâce, arbitraire de la prédestination, dureté des jugements, importance de l'enfer, etc.  ? Mais ceci est une autre histoire, dont les racines sont anciennes, et les causalités en jeu peuvent fort bien avoir été circulaires.}. Jusqu'au milieu du \siecle{18} l'image de l'autorité était globalement positive, et on jugeait que l'exercice que les rois et les pères (et aussi les "pères spirituels" de tous ordres) faisaient de leurs pouvoirs était bénéfique. Dans ce cadre de pensée s'opposer au souverain ou aux pères c'était s'opposer à Dieu lui-même, et au minimum faire preuve d'une présomption coupable. 
    
    Les pères de famille avaient le devoir de maintenir leur maison en bon ordre, dans le respect des lois civiles et religieuses, et la société attendait d'eux qu'ils le fassent sans faiblir, quelles que soient les difficultés. Pour y parvenir il leur était reconnu une grande part de la puissance que détenaient les citoyens romains sur leurs enfants et leurs épouses. Dans les pays de Droit écrit, comme le sud de la France, revenus à la fin du Moyen Âge à une application stricte du Droit romain, leur puissance ne cessait qu'avec leur mort. Partout leur mission éducative impliquait le \emph{droit de correction}. On considérait que c'était pour eux un devoir moral et social que de corriger les enfants \emph{et les épouses indisciplinées}. Jusqu'au \siecle{18} (au moins) il était admis qu'une tendresse excessive était plus dommageable pour l'enfant, et donc plus coupable, qu'une sévérité excessive : « {qui aime bien châtie bien} ». Montaigne nous dit qu'il fut placé de sa naissance à l'âge de quatre ans chez des bûcherons, puis mis en pension en collège à partir de six ans. Il dit s'être trouvé mieux de cette enfance loin de sa famille... parce qu'il lui semblait que son père était « trop tendre%
% [1] 
\footnote{En tous cas en justifiant la décision de son père par son « excès de tendresse » Montaigne nous fournit un bel exemple de ce qu'on désigne aujourd'hui sous le nom de « fidélité » ou de « loyauté » des enfants, et des trésors de compréhension dont ils sont prêts à faire preuve face à toutes les décisions, quelles qu'elles soient, que leurs parents ont pu prendre.} 
» !

 Le roi soutenait l'autorité des pères sur leurs épouses et sur leurs enfants, et leur prêtait main-forte s'ils le demandaient, entre autres moyens par les \emph{lettres de cachet} ordonnant sans jugement l'incarcération de l'enfant récalcitrant, mineur ou majeur, ou de l'épouse indigne, volage ou frivole ou de mauvais caractère,~etc. S'il le jugeait nécessaire, il pouvait se substituer de sa propre initiative%
% [2] 
\footnote{De même que lorsqu'il s'agit de ses enfants un père n'attend pas d'être saisi : par définition il parle en leur nom et à leur place (et en Droit romain même quand ils sont adultes).} 
aux pères défaillants dans leur fonction de faire régner l'ordre dans leurs familles. 
 Mais il se devait aussi de contrôler qu'ils n'abusaient pas de leurs pouvoirs : leur droit de correction n'était pas un droit de vie ou de mort. Jamais les parents n'ont été autorisés à estropier leurs enfants, et l'appui donné par la force publique à leurs décisions n'était pas automatique.

 

\section{Les enseignements}

    
   Il n'est pas question ici de faire une histoire de l'enseignement, mais seulement d'en esquisser les quelques grands traits qui ont rapport à notre sujet. Jusqu'à la fin de l'ancien Régime il n'existait rien qui ressemble à un enseignement public, et l'ensemble du domaine scolaire était placé sous le contrôle des évêques. Le plus souvent l'enseignement était assumé par les prêtres et les religieux(ses) et les biens ecclésiastiques (au sens large) supportaient une grande part du financement des établissements scolaires\footnote{\\\fsc{FURET} et \fsc{OZOUF}, \emph{Lire et écrire, l'alphabétisation des français de Calvin à Jules Ferry}, 1977.
\\Maurice \fsc{CAPUL}, \emph{Internat et internement sous l'ancien régime, contribution à l'histoire de l'éducation spéciale}, Thèse d'état, CTNERHI-PUF, Paris, 1983-1984.
\\Martine \fsc{SONNET}, {« Une fille à éduquer », in \emph{Histoire des femmes en Occident}, III, \siecles{16}{18}}, Collectif, sous la direction de Georges \fsc{DUBY} et Michelle \fsc{PERROT}, 2002, Chapitre 4, p. 131 à 168.
\\Georges \fsc{MINOIS}, \emph{Les religieux en Bretagne sous l'Ancien Régime}, 1989 ...}%. 
. Le réseau des établissements autres que les écoles monastiques s'est développé au fil des siècles à partir des écoles cathédrales, vers le bas avec les \emph{petites écoles}, et vers le haut avec les universités et les collèges. À partir des derniers siècles du Moyen Âge des \emph{petites écoles} (écoles primaires) existaient dans toutes les villes importantes. Elles étaient placées sous l'autorité du Chapitre de la Cathédrale. Au sein de celui-ci c'était le \emph{Chantre} de la Cathédrale qui avait la responsabilité de leur contrôle. C'est lui qui jusqu'à la fin de l'ancien régime agréera tous les candidats à l'enseignement, agrément sans lequel nul n'avait le droit d'enseigner. 

 Dans ces petites écoles à côté des connaissances profanes (d'abord la lecture, le calcul, souvent l'écriture) on enseignait aussi la religion, les disciplines du corps et de l'esprit, les bonnes manières de se conduire. Leur mission était en effet d'éduquer autant que d'enseigner. L'instruction, une fois entendu qu'elle se devait d'inclure la religion, était considérée comme la meilleure défense contre l'envie de mal faire. Curés (ou pasteurs protestants), parents et autorités locales étaient d'accord sur ce point. D'autre part les citadins voyaient aussi en elle la meilleure arme pour trouver et pour garder un travail, ce qui avait à la fois un intérêt économique et un intérêt social. À partir de la Réforme et du Concile de Trente cette foi en l'enseignement s'est traduite par un véritable apostolat, dont témoignent la création de nombreux ordres enseignants et d'innombrables écoles. C'est pourquoi divers ordres enseignants ont été créés au fil des siècles : XIIème siècle, religieuses bernardines, XVème siècle, Frères de la vie commune, XVème siècle, Minimes, XVIème siècle, Jésuites, Ursulines, XVIème siècle, Prêtres de la Doctrine chrétienne, XVIIème siècle, Oratoriens, Dames de Saint-Maur, Piaristes, Religieuses de Notre-Dame, Dames de la Providence, Frères des écoles chrétiennes, etc. (cette liste n'est pas exhaustive). 

 Les petites écoles s'adressaient aux « enfants des pauvres », c'est-à-dire, dans le langage d'alors, de tous ceux dont les ressources étaient précaires, de ceux qui n'avaient pas de rentes, de quelque nature qu'elles soient, et qui devaient gagner leur vie en travaillant. Il s'agissait donc de l'essentiel de la population des villes. Mais les petites écoles étaient ordinairement payantes, comme les universités d'alors, qui étaient elles aussi sous le contrôle du Chapitre cathédral. Elles étaient donc sans doute à la portée des bourgeois aisés, des commerçants et artisans, mais pas forcément à celle des autres. Lorsque les paroisses ne pouvaient pas exempter ces derniers des frais de scolarité, ce qui était le cas lorsque l'ensemble de leurs paroissiens étaient réellement pauvres, seuls de généreux donateurs et surtout des ordres religieux pouvaient les prendre en charge (cf.  les "écoles de charité"). Les religieux bénéficiaient d'une sécurité financière et d'une surface sociale que ne pouvaient avoir des particuliers ou des communes pauvres. Certains ordres avaient d'ailleurs explicitement pour but d'assurer gratuitement l'enseignement des indigents. 

 Beaucoup d'enfants n'étaient pourtant pas scolarisés, même dans les villes : leurs parents avaient trop besoin du produit de leur travail, ou bien ils ne voyaient aucune utilité à un apprentissage scolaire. Même aux yeux de ceux qui envoyaient leurs enfants à l'école il n'était pas toujours évident  qu'il faille que ceux-ci soient scolarisés avec assiduité pendant plusieurs années. Beaucoup, ou peut-être la plupart, se contentaient des quelques mois ou années nécessaires pour apprendre à lire et/ou à écrire.

 Quant à l'instruction des paysans, jusqu'à la fin de l'ancien régime elle n'était pas jugée nécessaire. L'illettrisme n'avait pas d'incidence sur leur productivité étant donné le niveau des techniques alors en usage. D'autre part les maîtres et seigneurs craignaient qu'une instruction même minime ne les rende « raisonneurs » et « arrogants » (Voltaire, représentatif de sa classe et de son temps, sera lui aussi de cet avis).

 Les écoles cathédrales comme les écoles monastiques ont été créées à partir de la fin de l'Antiquité pour fournir l'Église en clercs, mais elles ont toujours reçu un certain contingent d'élèves promis à la vie civile. A partir du Xème siècle la croissance des villes a provoqué la demande d'une instruction de niveau universitaire (secondaire ou supérieur). Les universités et leurs collèges ont été créés sous l'autorité épiscopale (mais aussi et en même temps sur l'impulsion des autorités civiles), durant les derniers siècles du Moyen Âge, sur le modèle monastique (comme le collège qui deviendra la Sorbonne) et se sont généralisés à partir de la Renaissance. A la différence des petites écoles les écoles monastiques, les universités et les collèges qui en dépendaient, et tous ceux qui ont été créés, entre autres par les jésuites ou les oratoriens, enseignaient le latin sans lequel, selon les conceptions du temps, on restait un illettré, c'est-à-dire un non lettré. 

Pour les parents des collégiens l'éducation était un investissement familial du moins quand il s'agissait des garçons, et même quand ils se destinaient à devenir des clercs (ce qui a été le cas de beaucoup ou même de la plupart jusqu'au \siecle{18}). Cela justifiait la perte de leur salaire pendant leurs années de scolarité. Quant aux jeunes garçons sans ressources mais doués pour les études et en particulier ceux qui se destinaient à entrer dans les ordres (soit par amour pour la vie religieuse soit par désir de promotion sociale) ils pouvaient bénéficier de bourses, dans la limite des places fondées par de généreux donateurs.
 
Beaucoup de pédagogues du passé ont eu une image positive de l'internat car à leurs yeux il mettait les jeunes dans des conditions éducatives plus favorables que l'externat. A partir de 1640 la diffusion de la doctrine janséniste, en dépit de la résistance des jésuites et du roi, mais avec l'appui des parlements, a encore accru la méfiance traditionnelle face aux tendances spontanées des adolescents. Elle a conforté la répression des désirs en général et des désirs sexuels en particulier. Dans un internat le (la) jeune interne était coupé de ses parents d'une manière rigoureuse dans un cadre totalement maîtrisé (lieux, communications, temps). Il était contenu fermement (murs, grilles, portier, clôture, clés, etc.) à l'abri du monde extérieur. Il s'agissait de donner de saines habitudes à son corps et à son esprit.  L'internat protégeait les « enfants de famille » contre les « mauvaises influences » qui les « pervertissaient », tout en préservant les « filles honnêtes », la « paix des ménages », et la « tranquillité publique » des « débordements de la jeunesse ». Il était pour les parents une garantie contre les "erreurs de jeunesse" qui réduisent à néant les meilleures stratégies familiales. Les pensionnaires ne perturbaient pas la vie de la cité où ils étudiaient comme étaient capables de le faire des écoliers externes de tous âges vivant loin de leur famille et sur lesquels les logeurs comme les maîtres n'avaient guère d'autorité. 

Quant aux jeunes filles de bonne famille, la clôture des couvents leur interdisait toute rencontre avec les jeunes gens de leur âge. Elle protégeait leur « vertu », leur réputation (et leur imagination ?) en attendant que leurs parents les marient. Il semble que la durée de leurs séjours dans les couvents ait été très inférieure à celle de leurs frères dans les collèges, sauf quand il s'agissait d'un prélude à un noviciat. L'enseignement qui leur était dispensé était habituellement bien moins poussé (exclusion du latin, etc.).  

Mais entre l'externat des collèges et autres établissements secondaires, souvent gratuit ou presque, et la pension des internats l'écart des coûts était énorme%
%[5]
\footnote{Selon Martine \fsc{SONNET}, la pension d'un seul enfant représentait presque la totalité du salaire d'un ouvrier (« une fille à éduquer », Chapitre 4 de \emph{l'Histoire des femmes en Occident}, III, \siecles{16}{18}, p. 146). C'est pourquoi en 1760 les internats parisiens n'accueillaient que 13~\% de la population scolaire de la ville, et il semble qu'il en était de même ailleurs.}% 
. Aux familles qui ne pouvaient payer les frais d'une pension, c'est-à-dire la plupart, seul l'externat était accessible, en vivant en ville chez ses parents ou chez un membre de sa famille ou chez un logeur peu exigeant. Dans les collèges il y avait donc ordinairement beaucoup plus d'externes que d'internes et souvent il n'y avait pas d'internat du tout (par exemple chez les Jésuites). Mais le prestige des internats et la nécessité pratique de regrouper une grande part des collégiens loin de leur domicile a fait qu'ils se sont répandus et ont grossi peu à peu, surtout à partir du milieu du XVIIIème siècle\footnote{Leur croissance s'est poursuivie longtemps et leur réseau n'a été achevé qu'au début du \siecle{20}. Le nombre de places d'internat semble s'être maintenu ensuite sans grands changements jusqu'aux années soixante du XXème siècle où il a commencé à baisser.}. 

 
 

Les collégiens étaient confiés soit à l'un des ordres religieux spécialisés à partir de la Renaissance dans l'enseignement (Jésuites surtout, mais aussi Oratoriens, Dominicains,~etc.) soit à des clercs recrutés sur place, de niveau universitaire inégal et parfois médiocre : souvent ces derniers gagnaient leur vie en enseignant en attendant de se voir attribuer un bénéfice ecclésiastique plus lucratif. Les collèges étaient ouverts à la ville dans les murs de laquelle ils étaient établis. S'ils étaient bien tenus et de qualité ils en formaient ordinairement l'un des fleurons les plus prestigieux. Ils y entretenaient une vie intellectuelle et mondaine active et d'autant plus valorisée que les autres sources de distraction étaient rares. Pour prix de leur liberté perdue, il était proposé aux collégiens de s'investir dans la découverte du savoir, et celui-ci était ressenti par leurs enseignants et leurs parents comme quelque chose qui en valait la peine. Ils entraient dans une aristocratie de l'esprit. La première chose que réclamaient les parents était qu'on enseigne le latin à leurs fils (pas à leurs filles). À l'époque l'enseignement secondaire et supérieur se faisait en latin dans toute l'Europe%
% [7]
\footnote{Dans toute l'Europe les thèses seront encore soutenues en latin durant la plus grande partie du \siecle{19}.}% 
. C'était la langue vivante, la langue de communication des communautés intellectuelles du temps, celle dont la connaissance faisait le lettré. Mais depuis l'\emph{ordonnance de Villers-Cotterêts} (1539) qui imposait le français comme langue administrative du Royaume, il n'était plus possible de tenir un \emph{office} public si on ne le maîtrisait pas suffisamment. La langue française n'était encore que le parler de l'Île-de-France, domaine du roi. Partout ailleurs c'était une langue étrangère qui allait mettre très longtemps à déloger les langues locales des places et des marchés. L'enseignement du français reçu dans les collèges était donc incontournable pour entrer dans les professions libérales, la fonction publique ou le clergé. 

 Le collège était par conséquent un moyen sûr de promotion individuelle et familiale. Pour les gens ordinaires c'était la seule voie d'accès aux emplois prestigieux et qualifiés. Le fils de famille confronté à l'épreuve de l'internat ou de la vie loin de ses parents continuait de dépendre d'eux. Ils payaient sa pension : il continuait de manger leur pain. Il continuait de correspondre avec eux. Il les retrouverait aux prochaines vacances s'ils ne venaient pas le voir avant. Quand la discipline et le travail intellectuel lui pesaient trop il pouvait se dire avec assez de vraisemblance qu'il était en train d'acquérir à ce prix les moyens d'atteindre un statut personnel valorisant, et qu'il s'inscrivait dans le projet de ses parents. En acceptant de se soumettre à cette exigence il pouvait espérer devenir un membre puissant et respecté de sa communauté d'origine : cela présentait l'allure d'une épreuve initiatique. 

\section{La correction paternelle}

 Tous les jeunes \emph{de famille} n'entraient pas docilement dans les projets parentaux. Certains d'entre eux entraient en conflit ouvert avec leurs parents au-delà des normes reçues (éminemment variables suivant les siècles et les lieux) : vagabonds, fugueurs, jeunes aux fréquentations suspectes, exclus pour indiscipline de collèges successifs, fauteurs de vols domestiques ou d'actes « d'inconduite sexuelle », d'insultes et de voies de faits, « libertins », c'est-à-dire jeunes rétifs à toute mesure éducative,~etc. 

 À la demande de leurs parents, ces jeunes peuvent être traités en tout comme les délinquants con\-dam\-nés. Pour les enfants difficiles des familles aisées il y avait des solutions payantes dans les sections des collèges et internats contemporains affectés à la « correction ». Ceux qui n'en avaient pas les moyens étaient internés avec les délinquants condamnés, dans les sections « de force » des hôpitaux, où s'effectuaient les peines de prison. Leurs parents payaient une pension qui tenait compte de leurs ressources. 

 À partir de la fin du \siecle{17} et de plus en plus souvent au fil du \crmieme{18}, les \emph{enfants de famille}, garçons et filles mineurs \emph{et majeurs}, qui ont commis de vrais actes de délinquance, mais aussi ceux qui donnent simplement du mécontentement à leurs parents par leurs fréquentations, leur mauvaise conduite, leur indocilité, leur violence aveugle ou leur absence de sens commun (« insensés »), leurs dépenses inconsidérées, ou leurs dettes de jeu, peuvent, sur la demande de ces derniers exerçant ainsi leur droit de correction, faire l'objet d'une \emph{lettre de cachet}, c'est-à-dire d'une \emph{décision administrative d'internement} dans un hôpital, une prison, une forteresse, un couvent, un collège, ou même leur déportation aux colonies. Les lettres de cachet, qui ont une origine très ancienne, bien antérieure au \siecle{17}, peuvent aussi être accordées à l'encontre de conjoints aux comportements répréhensibles (cette mesure a beaucoup plus souvent frappé les femmes que les hommes). 

 L'autorité publique n'est pas obligée d'accorder satisfaction aux demandes qui lui sont faites, et reste seule juge de l'opportunité de la mesure. Elle est surtout sollicitée à Paris, notamment par les couches populaires, contrairement aux provinces où l'internement administratif est moins facile à obtenir et où les couches populaires n'y ont guère recours. Même si au fil du temps les lettres de cachet font l'objet de critiques de plus en plus virulentes et si les autorités publiques y répugnent de plus en plus, les demandes se font de plus en plus nombreuses au fil du \siecle{18}. 

 En effet les familles sollicitent ces lettres comme une grâce : cela leur évite la honte causée par la publicité du recours à la justice, le coût d'un procès, et aussi la publicité de la mesure d'enfermement. La réputation du jeune (ou de l'adulte) ainsi placé peut s'en relever plus facilement. Cela évite le contrôle par la justice de la nature exacte des faits et de la proportionnalité des sanctions aux dommages et délits constatés, ce qui permet à d'authentiques délinquants bien nés d'échapper à peu de frais aux conséquences normales de leurs actes. 

 Mais cela permet aussi aux parents abusifs d'exercer des pressions sur leurs enfants rétifs à leurs projets (ce qui explique les critiques de plus en plus virulentes des lettres de cachet au fil du XVIIIème siècle), à une époque où le consentement des parents est exigé à tout âge et pour tout mariage sous peine d'exhérédation, et où bien des entrées en religion sont imposées par eux sans tenir compte des désirs du ou de la jeune concerné. 

\section{Enfants « adoptifs »}

 On a vu que dans le but de défendre le mariage monogame et indissoluble, l'Église a tout fait depuis l'Antiquité pour que les enfants illégitimes ne puissent pas devenir des héritiers de plein exercice. C'est pour cette raison que l'adoption était interdite, et pourtant... De l'Antiquité à la fin de l'ancien régime, on peut observer en nombre non négligeable des situations plus ou moins proches d'une adoption, où une personne, souvent un ecclésiastique (cf. \hbox{Villon}, adopté par un chanoine), souvent aussi un couple sans enfants, exerçaient la puissance paternelle sur un enfant qui n'était pas né d'eux et l'élevaient jusqu'à sa majorité. C'était par exemple le cas à Lyon, où les recteurs de l'Hôtel-Dieu « adoptaient » ainsi des orphelins. 

 Ces situations d'\latin{alumnii} (adoptions simples) étaient parfois sanctionnées par des actes juridiques où les nourriciers faisaient un legs à l'enfant devant un procureur fiscal, et où ils s'engageaient à l'élever, instruire et établir matériellement à leurs frais comme leur propre enfant. Pour autant cela ne faisait pas de lui un membre de leur famille ni un héritier. 

 En principe seul un enfant légitime sans parents pouvait bénéficier de ce dispositif. Souvent, probablement le plus souvent, il était orphelin, mais des enfants légitimes pouvaient aussi être abandonnés solennellement par leurs parents, qui reconnaissaient par écrit qu'ils renonçaient à leur puissance paternelle, et à l'héritage de leur enfant s'il décédait. Pour autant ce dernier ne changeait ni de parenté ni de nom. Quand il possédait des biens, l'adoptant, tel un tuteur, les gérait jusqu'à sa majorité et il était responsable sur ses propres biens de sa gestion. 

 Les enfants abandonnés, nés de parents inconnus, ont très longtemps été exclus de ce genre de prise en charge%
% [8]
\footnote{À Lyon jusqu'en 1765. Ensuite ils y ont été traités comme les autres. Ce n'est que dans les dernières années avant la Révolution que les idées ont changé sur ce point : un signe de l'évolution qui s'est faite dans les esprits au fil du \siecle{18} et qui est apparue au grand jour à partir des années 1760-1770.}%
. Pourtant il était courant que des personnes accueillent pour l'élever un enfant abandonné à eux confié par un hôpital ou par une paroisse, qu'elles refusent d'être rémunérées pour l'élever, qu'elles le gardent jusqu'à sa majorité et qu'elles l'établissent dans la vie, ce qui en fait ressemblait beaucoup à la situation des enfants nés légitimes et juridiquement « adoptés ». Si aucun de leurs héritiers légitimes ne s'y opposait, elles faisaient de lui l'un de leurs héritiers. Mais il n'était pas question pour cet enfant d'hériter d'une fonction impliquant l'exercice public du pouvoir. 

 Derrière les mots employés il n'est pas toujours facile de reconnaître les situations réelles : adoption simple ? tutelle ? parrainage ?%
% [9]
\footnote{Cf. Jean-Pierre \fsc{GUTTON}, \emph{Histoire de l'adoption en France}, 1993.} 


