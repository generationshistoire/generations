% 28.02.2015 :
% haut Moyen Âge
% _, --> ,
% ~etc.
% Antiquité


\chapter{La Révolution française et les familles}


 La première démarche des représentants de la nation réunis en 1789 a été de rédiger une \emph{Déclaration des droits de l'homme et du citoyen}. Ils ont commencé par refuser les privilèges \emph{et les désavantages} fondés sur les circonstances de la conception, de la naissance, sur le statut des parents, sur la religion ou l'absence de religion. Selon l'article~1 de la Déclaration de 1789, \emph{les hommes naissent et demeurent libres et égaux en droit. Les distinctions sociales ne peuvent être fondées que sur l'utilité commune}. L'article~6 de la même Déclaration précise que \emph{tous les citoyens étant égaux aux yeux de la loi sont également admissibles à toutes les dignités, places et emplois publics, selon leur capacité et sans autres distinctions que celles de leurs vertus et de leurs talents}. En conséquence, personne ne naît plus esclave ni serf, et aucun nouveau-né ne doit être traité différemment des autres, quoi qu'aient pu commettre ses parents et quelles qu'aient pu être les circonstances de sa naissance : conception hors mariage, adultère, inceste,~etc. 


\section{Limitation de la puissance paternelle}

 La législation révolutionnaire sur la famille a une histoire complexe mais ses acteurs étaient d'accord sur l'essentiel. Ils avaient d'abord en ligne de mire la puissance paternelle\footnote{Cf. \emph{l'Histoire des pères et de la paternité}, voir en particulier le chapitre XI (p. 289 à 328) : « La volonté d'un homme » écrit par Jacques \fsc{MULLIEZ}.}. 
Certains d'entre eux allaient jusqu'à affirmer que les enfants appartenaient à l'état avant d'appartenir à leurs parents. Dans le même ordre d'idée les plus radicaux auraient voulu que tous les jeunes soient pris en charge en internat dès l'âge de 5 ans, pour les préserver de l'influence néfaste de leurs parents, suspects d'être « contre-révolutionnaires » ou « obscurantistes », et pour en faire des citoyens conformes à leurs désirs : répéter en somme pour tous les français ce que Louis~XIV avait cherché en vain à faire avec les protestants et autres « déviants ». En fait ces extrémistes étaient peu nombreux. La majorité tenait à ce que la nation contrôle l'éducation de sa jeunesse, mais elle était ouverte à une large liberté de l'enseignement, et en dépit des péripéties plus ou moins chaotiques vécues par certains l'essentiel du corps enseignant en place à la fin de l'ancien régime (en grande partie constitué d'ecclésiastiques) a formé l'armature des écoles privées ou publiques et des collèges de la Révolution.  

 En 1790 l'Assemblée constituante avait aboli les \emph{lettres de cachet}, dont la plus grande part était octroyée par les autorités civiles dans l'intérêt des chefs de famille. Ceci dit, le père, ou la mère si elle était seule, ou le tuteur (et eux seuls) pouvaient demander à un juge d'emprisonner pour un temps un enfant qui leur créait des "sujets de mécontentement". Mais cet internement n'était renouvelable qu'une seule fois pour un jeune de moins de 16 ans, et un jeune récalcitrant ne pouvait être interné plus d'une année entre 16 et 21 ans pour ce seul motif. D'autre part les parents devaient d'abord obtenir l'accord des \emph{tribunaux de la famille}, qui délibéraient \emph{sous l'autorité d'un juge professionnel}, même si leurs membres étaient recrutés au sein de la famille élargie (et à défaut dans le voisinage immédiat). Ces tribunaux étaient par ailleurs chargés de rétablir la concorde dans les foyers en conflit. 


\section{Privatisation des vœux perpétuels et ouverture du droit au divorce}

 La Constitution de 1791 refusait de reconnaître une valeur juridique aux vœux prononcés par les religieux et fermait tous les couvents. Dans la même logique, les révolutionnaires refusaient de reconnaître tout aspect religieux au mariage et d'y voir autre chose qu'un contrat civil, révocable comme tout autre contrat. En conséquence, la loi du 20 septembre 1792 supprimait la \emph{séparation de corps}, qui sentait trop le catholicisme ...

 ... tandis qu'elle autorisait le divorce par \emph{consentement mutuel} et le divorce \emph{sur demande d'un seul époux}, demande qu'elle autorisait de manière très large et d'abord pour \emph{convenance personnelle}. 

 Le divorce est à ce moment-là devenu aussi facile et plus rapide qu'aujourd'hui (2018). Jusqu'à l'an VII on observe \emph{en ville} un divorce pour 5 mariages ; ensuite un divorce pour 3 mariages. L'inflation du nombre des divorces, non anticipée par la plupart de ceux qui les avaient facilités, a choqué bien des sensibilités. Si les citadins ont recouru très largement du nouveau droit, les habitants des campagnes ne l'ont guère utilisé, d'où l'on conclura sans risque de se tromper qu'on ne divorce pas d'une terre obtenue par mariage aussi facilement que du conjoint qui l'a procurée. C'est une constante de l'histoire : comme le dit le Talmud : \emph{" malheur à celui qui est mal marié et ne peut rembourser la dot de son épouse "}. Ceci dit les campagnes n'avaient pas été gagnées au même degré que les villes par les critiques des philosophes contre l'indissolubilité du mariage, que ce soit pour des raisons religieuses ou parce que celle-ci allait en réalité dans le sens de l'intérêt des familles. 

 


\section{Autonomisation des enfants majeurs}

 La Révolution affirmait l'égalité entre les héritiers et elle la défendait contre tout droit d'aînesse. Pour ce motif et pour empêcher les parents d'exercer une pression indirecte sur les actes de leurs enfants majeurs, la liberté des testateurs était très limitée.
 
 En 1792 l'âge de la majorité a été abaissé de 25 à 21 ans, et surtout les enfants majeurs ont été totalement déliés de la puissance paternelle. Leur capacité juridique a été reconnue comme pleine et entière, qu'il s'agisse d'aliéner leurs biens ou de s'engager dans n'importe quel contrat. Ils pouvaient notamment se marier ou divorcer librement, si nécessaire en passant outre à l'opposition de leurs parents, sans risquer d'être déshérités pour autant. 

 


\section{Nul ne peut être parent contre son gré}

 Les révolutionnaires assimilaient les enfants illégitimes aux enfants légitimes, qu'ils soient adultérins, incestueux ou nés hors mariage de personnes libres de tout engagement ou empêchement, \emph{à la condition expresse qu'ils aient été reconnus par au moins l'un de leurs deux géniteurs}. 
 Mais ils affirmaient aussi que \emph{nul ne peut être parent contre son gré}. Nul, ni femme ni homme, ne devait être contraint à reconnaître un enfant pour sien. L'enfant ne devait être reconnu que volontairement et librement. En cas de naissance hors mariage, une décision libre de chacun des géniteurs était nécessaire pour qu'il devienne parent. La seule exception était le viol avec enlèvement, auquel cas le coupable perdait son droit de ne pas reconnaître l'enfant et de ne pas assumer de responsabilité financière vis à vis de lui. 
De là découlait que dans le temps même où les révolutionnaires accordaient aux enfants illégitimes le droit à entrer dans la famille du parent qui les reconnaissait, et d'hériter de lui à égalité avec un enfant légitime, ils écartaient toute possibilité de \emph{recherche de paternité naturelle}, même pour l'allocation de simples \emph{aliments}. 
 
 On peut s'étonner de cette rigueur à l'encontre des enfants nés hors mariage, comparée à la propension des tribunaux d'ancien régime à traiter favorablement toutes les accusations des mères célibataires portées contre leurs amants réels ou supposés : ils ne reconnaissaient pas la liberté de \emph{ne pas reconnaître} l'enfant dont on ne veut pas. Si les tribunaux "d'avant" écoutaient d'une oreille complaisante les accusations des mères célibataires, et si celles-ci avaient objectivement intérêt à accuser des hommes riches, ceux-ci s'en tiraient ordinairement sans trop de dommages. En effet ils ne risquaient pas de voir entrer les enfants concernés dans leur famille, ni de devoir les compter au nombre de leurs héritiers. Même s'ils l'avaient voulu les lois de l'ancien régime le leur interdisaient. Jusqu'à la Révolution une recherche en paternité se soldait dans le pire des cas, que l’homme condamné soit vraiment le géniteur de l’enfant concerné ou qu'il ait échoué à prouver le contraire, par l’obligation de verser des frais de \emph{gésine} puis \emph{d'aliments} proportionnés au statut social de la mère, jusqu'à ce que l'enfant puisse gagner son pain, à douze ans au plus tard. Par contre à partir du moment où les lois nouvelles ne faisaient plus de différence entre les enfants naturels et les enfants légitimes une recherche en paternité naturelle entraînait de tout autres conséquences sur les familles. 

 Dans le même esprit, l'adultère féminin ne posait pas problème aux révolutionnaires tant que la légitimité de l'enfant conçu n'était pas dénoncée par le mari, suivant le vieux principe du droit romain qui voulait que l'époux de la mère était le père de tous les enfants de celle-ci nés pendant leur union. Le mari d'une femme adultère avait le pouvoir de dénoncer sa paternité et il était le seul dans ce cas : ni l'épouse, ni son amant ne pouvaient le faire, même s'ils le voulaient. 

 Dans l'esprit des hommes de la Révolution, la contrepartie du droit de ne pas être parent si et de refuser de reconnaître un enfant né de ses œuvres, était l'ouverture aux enfants non reconnus d'un large droit à être adoptés dès leur plus jeune âge. En donnant aux personnes sans enfants le droit de se faire ainsi des successeurs et des héritiers ils espéraient résoudre le problème posé par le grand nombre d'enfants abandonnés de cette période. 

 


\section{Démembrement de l'Hôpital Général}

 Le 19 avril 1801 (an IX), le \emph{Conseil général des hospices} réorganisait administrativement les établissements hospitaliers. Les différentes fonctions assurées indistinctement jusque là étaient administrativement démembrées et réparties entre des institutions indépendantes et spécialisées%
% [4]
\footnote{En fait toutes les populations contrôlées par l'ancien Hôpital Général (hôpitaux, hospices, prisons, nourrices et même services d'assistance au domicile) et toutes les institutions nées de son éclatement, sont restées jusqu'au début du \siecle{20} sous la tutelle du Ministère de l'Intérieur, chargé par ailleurs de la police et des cultes. Les personnes incarcérées, quel que soit leur âge, dépendront du ministère de l'intérieur jusqu'en 1911, date où elles passeront sous l'autorité du ministère de la Justice.}%
, dont les ressources allaient être de plus en plus exclusivement assurées par l'impôt. Le classement des établissements établi en 1801 est à l'origine de celui qui a cours aujourd'hui, même si à l'intérieur de chacune de ses catégories de profondes évolutions ont depuis lors transformé le traitement des problèmes des personnes prises en charge :
%\begin{enumerate}[leftmargin=*,itemsep=0pt]
\begin{itemize}
%1) 
\item pour les prévenus, pour les hommes condamnés à de courtes peines, et pour toutes les femmes condamnées : les prisons ; 
\item pour les hommes condamnés à de longues peines, les bagnes ;
\item pour les malades mentaux (c'est-à-dire ceux désignés comme tels par leurs familles ou les autorités civiles, après confirmation du diagnostic par les médecins aliénistes) : les hôpitaux psychiatriques (dont l'architecture, l'organisation interne et le personnel présentaient une grande proximité avec ceux des prisons) ;
\item pour les malades pauvres : les hôpitaux\footnote{Le même jour, le 16 avril 1801, le Conseil général des hospices supprimait les lits de plus d'une personne. C'était la survivance d'un archaïsme que Tenon tenait dès 1788 pour une aberration nuisible à la cure des malades. Mais à Paris cette situation perdurait partout, et surtout dans les sections des indigents des hôpitaux généraux.}, réservés aux malades pauvres ou sans famille, aux personnes en voyage loin de chez elles, et aux militaires éloignés de toute infirmerie de garnison. Les riches préféraient se faire soigner à domicile ou dans des "cliniques" privées payantes, comme toujours ;
\item pour les personnes incapables de gagner leur vie : vieillards sans ressources, infirmes et enfants non abandonnés dont on connaît les parents (qu'ils soient ou non vivants) : les hospices ;
\item pour tous les nourrissons, d'une part, et pour les enfants abandonnés (pupilles) jusqu'à leur majorité d'autre part : placements en nourrice sous l'autorité des hospices mentionnés ci-dessus ;
\item les mineurs vagabonds ou délinquants étaient emprisonnés en tant que délinquants comme les adultes, et avec les adultes. 
 \end{itemize}

 


