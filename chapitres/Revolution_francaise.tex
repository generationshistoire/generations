% 28.02.2015 :
% haut Moyen Âge
% _, --> ,
% ~etc.
% Antiquité


\chapter{La Révolution française et les familles}


 La première démarche des représentants de la nation réunis en 1789 a été de rédiger une \emph{Déclaration des droits de l'homme et du citoyen}. Les représentants du peuple ont commencé par refuser les privilèges \emph{et les désavantages} fondés sur les circonstances de la conception, de la naissance, sur le statut des parents, sur la religion ou l'absence de religion. Selon l'article~1 de la Déclaration de 1789, \emph{les hommes naissent et demeurent libres et égaux en droit. Les distinctions sociales ne peuvent être fondées que sur l'utilité commune}. L'article~6 de la même Déclaration précise que \emph{tous les citoyens étant égaux aux yeux de la loi sont également admissibles à toutes les dignités, places et emplois publics, selon leur capacité et sans autres distinctions que celles de leurs vertus et de leurs talents}. En conséquence, personne ne naît plus esclave ni serf, et aucun nouveau-né ne doit être traité différemment des autres, quoi qu'aient pu commettre ses parents et quelles qu'aient pu être les circonstances de sa naissance : conception hors mariage, adultère, inceste,~etc. 


\section{Limitation de la puissance paternelle}

 La législation révolutionnaire sur la famille a une histoire complexe mais ses acteurs étaient d'accord sur l'essentiel. Ils avaient d'abord en ligne de mire la puissance paternelle%
% [1]
\footnote{Cf. \emph{l'Histoire des pères et de la paternité}, voir en particulier le chapitre XI (p. 289 à 328) : « La volonté d'un homme » écrit par Jacques \fsc{MULLIEZ}.}% 
. Certains d'entre eux allaient jusqu'à affirmer que les enfants appartenaient à l'état avant d'appartenir à leurs parents%
%[2]
\footnote{... laïcisant ainsi d'une façon le discours religieux qui faisait des enfants un don de Dieu confié aux parents pour qu'ils les élèvent pour lui, et non pour eux-mêmes.}%
. Dans le même ordre d'idée les plus radicaux voulaient que tous les jeunes soient pris en charge en internat dès l'âge de 5 ans, pour les préserver de l'influence néfaste de leurs parents, suspects d'être {« contre-révolutionnaires »} ou « {obscurantistes}%
%[3] 
\footnote{... c'est-à-dire attachés à une religion.} 
», et pour en faire des citoyens conformes à leurs désirs : répéter en somme pour tous les français ce que Louis~XIV avait cherché en vain à faire avec les protestants et autres « déviants ».

 En 1790 l'Assemblée constituante avait aboli les \emph{lettres de cachet}, dont la plus grande part était octroyée par les autorités civiles dans l'intérêt des familles (des chefs de famille). Ceci dit, le père, ou la mère si elle était seule, ou le tuteur, avaient toujours le droit de demander à un juge (à l'exclusion des autres autorités civiles, ce qui en soi était un changement important) d'emprisonner pour un temps un enfant qui leur créait des sujets de mécontentement. Mais cet internement n'était renouvelable qu'une seule fois pour un jeune de moins de 16 ans, et un jeune récalcitrant ne pouvait être interné plus d'une année entre 16 et 21 ans pour ce seul motif. D'autre part les parents devaient d'abord obtenir l'accord des \emph{tribunaux de la famille}, qui délibéraient \emph{sous l'autorité d'un juge professionnel}, même si leurs membres étaient recrutés au sein de la famille élargie (et à défaut dans le voisinage immédiat). Ces tribunaux étaient par ailleurs chargés de rétablir la concorde dans les foyers en conflit. 


\section{Privatisation des vœux perpétuels et droit au divorce}

 La Constitution de 1791 refusait toute valeur juridique aux vœux prononcés par les religieux et fermait tous les couvents. Dans la même logique, les révolutionnaires refusaient de reconnaître les aspects religieux du mariage et d'y voir autre chose qu'un contrat civil, révocable comme tout autre contrat. En conséquence, la loi du 20 septembre 1792 supprimait la \emph{séparation de corps}, qui sentait trop le catholicisme ...

 ... tandis qu'elle autorisait le divorce par \emph{consentement mutuel} et le divorce \emph{sur demande d'un seul époux}, demande qu'elle autorisait de manière très large et d'abord pour \emph{convenance personnelle}. 

 Le divorce est alors devenu aussi facile et plus rapide qu'aujourd'hui (2012). Jusqu'à l'an VII on observe \emph{en ville} un divorce pour 5 mariages ; ensuite un divorce pour 3 mariages. Si les citadins ont recouru très largement à ce droit, les habitants des campagnes ne l'ont guère utilisé, d'où l'on conclura sans risque de se tromper de beaucoup qu'on ne divorce pas d'une terre acquise par mariage aussi sereinement que du conjoint qui l'a amenée. C'est une constante de l'histoire : la nécessité de rendre la dot en cas de divorce fait beaucoup réfléchir, et comme dit le Talmud, \emph{malheur à celui qui est mal marié et ne peut rembourser la dot de son épouse}. Ceci dit les campagnes n'avaient pas été gagnées au même degré que les villes par les critiques des philosophes contre l'indissolubilité du mariage, que ce soit pour des raisons religieuses ou parce que cela allait dans le sens des désirs des chefs de famille. 

 L'inflation du nombre des divorces, non anticipée par la plupart de ceux qui les avaient facilités, a choqué bien des sensibilités.


\section{Libération des enfants majeurs des tutelles parentales}

 En 1792 l'âge de la majorité a été abaissé de 25 à 21 ans, et surtout les enfants majeurs ont été totalement déliés de la puissance paternelle. Leur capacité juridique a été reconnue comme pleine et entière, qu'il s'agisse d'aliéner leurs biens ou de s'engager dans n'importe quel contrat. Ils pouvaient notamment se marier ou divorcer librement, si nécessaire en passant outre à l'opposition de leurs parents, sans risquer d'être déshérités pour autant. 

 Par ailleurs la révolution affirmait l'égalité entre les héritiers et elle la défendait contre tout droit d'aînesse. Pour ce motif et pour empêcher les parents d'exercer une pression indirecte sur les actes de leurs enfants majeurs, la liberté des testateurs était très limitée.


\section{Nul ne peut être parent contre son gré}

 Les révolutionnaires assimilaient les enfants illégitimes aux enfants légitimes, qu'ils soient adultérins, incestueux ou nés hors mariage de personnes libres de tout engagement ou empêchement, \emph{à la condition expresse qu'ils aient été reconnus par au moins l'un de leurs deux géniteurs}. 

 Mais la révolution affirmait aussi que \emph{nul ne peut être parent contre son gré}. Nul, ni femme ni homme, ne devait être contraint à reconnaître un enfant pour sien. L'enfant ne devait être reconnu que volontairement et librement. La seule exception était le viol avec enlèvement, auquel cas le coupable perdait toute liberté de ne pas reconnaître l'enfant et de ne pas assumer sa responsabilité financière vis à vis de lui. En cas de naissance hors mariage, une décision libre de chacun des géniteurs était nécessaire pour qu'il devienne parent. 

 De là découlait que dans le temps même où les révolutionnaires accordaient aux enfants illégitimes le droit à entrer dans la famille du parent qui les reconnaissait, et d'hériter de lui à égalité avec un enfant légitime, ils écartaient toute possibilité de recherche de paternité naturelle, même pour l'allocation de simples \emph{aliments}. 

 On peut s'étonner de cette rigueur, comparée à la façon dont l'ancien régime encourageait les actions de recherche en paternité. C'est que sous l'ancien régime, au pire, du point de vue de l'homme condamné, une recherche en paternité aboutie se soldait seulement par l'obligation de verser des frais de \emph{gésine} (proportionnés au statut social de la mère, en général fort humble) et d'aliments jusqu'à ce que l'enfant puisse gagner son pain, à douze ans au plus tard. Si les tribunaux écoutaient d'une oreille complaisante les accusations des mères célibataires, et si celles-ci avaient objectivement intérêt à accuser des hommes riches, ceux-ci s'en tiraient ordinairement sans trop de dommages, qu'ils soient vraiment les géniteurs des enfants concernés ou qu'ils aient échoué à prouver le contraire. En effet ils ne risquaient pas de voir entrer ces enfants dans leur famille, ni de devoir les compter au nombre de leurs héritiers. Même s'ils l'avaient voulu c'était impossible. Par contre à partir du moment où la loi ne faisait plus de différence entre les enfants naturels et les enfants légitimes une recherche en paternité naturelle entraînait de tout autres conséquences. La relative complaisance des tribunaux d'ancien régime à accueillir et à traiter les accusations des mères célibataires portées contre leurs amants réels ou supposés ne paraissait plus de mise. 

 Dans l'esprit des hommes de la révolution, la contrepartie du droit de chaque géniteur de ne pas être parent s'il ne le voulait pas, de ne pas reconnaître l'enfant né de ses œuvres, était l'ouverture d'un large droit à être adoptés dès leur plus jeune âge aux enfants non reconnus. Ils espéraient résoudre ainsi le problème posé par le grand nombre d'enfants abandonnés de cette période. 

 Dans le même esprit, l'adultère féminin ne posait pas problème aux révolutionnaires tant que la légitimité de l'enfant conçu n'était pas dénoncée par le mari, suivant le vieux principe du droit romain qui voulait que l'époux de la mère était le père de tous les enfants de celle-ci nés pendant leur union. \emph{Le mari d'une femme adultère était le seul à pouvoir dénoncer sa paternité} : ni l'épouse, ni son amant ne pouvaient le faire, même s'ils le voulaient. 


\section{Démembrement de l'Hôpital Général}

 Le 19 avril 1801 (an IX), le \emph{Conseil général des hospices} réorganisait administrativement les établissements hospitaliers. Les différentes fonctions assurées indistinctement jusque là étaient administrativement démembrées et réparties entre des institutions indépendantes et spécialisées%
% [4]
\footnote{... ce qui n'a pas empêché toutes les populations anciennement contrôlées par l'ancien Hôpital Général (hôpitaux, hospices, prisons, nourrices et même services d'assistance au domicile) et toutes les institutions nées de son éclatement, de rester jusqu'au début du \siecle{20} sous la tutelle du Ministère de l'Intérieur, chargé par ailleurs de la police (et des cultes). Les personnes incarcérées, quel que soit leur âge, dépendront du ministère de l'intérieur jusqu'en 1911, date où elles passeront sous l'autorité du ministère de la Justice.}%
, dont les ressources allaient être de plus en plus exclusivement assurées par l'impôt. 

 Le classement des établissements établi en 1801 est pour l'essentiel celui qui a cours aujourd'hui :
%\begin{enumerate}[leftmargin=*,itemsep=0pt]
\begin{itemize}
%1) 
\item pour les délinquants et les prévenus : les prisons ; pour les hommes condamnés à de longues peines, les bagnes ;
% 2)
\item pour les malades pauvres : les hôpitaux%
%[5] 
\footnote{Le même jour, le 16 avril 1801, le Conseil général des hospices supprimait les lits de plus d'une personne. C'était la survivance d'un archaïsme que Tenon tenait dès 1788 pour une aberration nuisible à la cure des malades. Mais à Paris cette situation perdurait partout, et surtout dans les sections des indigents des hôpitaux généraux.} 
;
% 3)
\item pour les malades mentaux : les hôpitaux psychiatriques ;
% 4)
\item pour les indigents, les vieillards sans ressources, les enfants non abandonnés encore incapables de gagner leur vie et les infirmes : les hospices ;
% 5)
\item pour tous les nourrissons, d'une part, et pour les enfants sans parents jusqu'à leur majorité d'autre part : placements en nourrice sous l'autorité des hospices mentionnés ci-dessus ;
% 6)
\item pour les enfants qui ont des parents, vivants ou morts, placement en institution.
\end{itemize}
%\end{enumerate}

 Les mineurs vagabonds ou délinquants du début du \siecle{19} étaient emprisonnés en tant que délinquants comme les adultes, et au début avec les adultes. 

 L'Hôpital restait réservé aux malades pauvres, sans famille, aux personnes en voyage loin de chez elles, et aux militaires éloignés de toute infirmerie de garnison. Les riches se faisaient soigner à domicile, comme toujours.


