% Le 28.02.2015 :
% Antiquité
% Moyen Âge
% _, --> ,
% Le 24.02.2015 :
% ~etc.
% Moyen-Âge
% ~\%


\chapter[Le mouvement des Lumières]{Le mouvement des Lumières et les autorités établies}


 Durant le XVIIIème siècle les "philosophes\footnote{\emph{"Tandis que le philosophe au sens traditionnel est avant tout un spécialiste de la théorie et de l'abstraction, le philosophe du XVIIIème siècle, c'est-à-dire, en général, tout homme éclairé, pénétré d'esprit critique, est donc d'abord un homme pratique et soucieux de la réalité quotidienne. Tout ceci se fait non au gré du hasard mais sous l'égide de la Raison et va concerner des auteurs aussi différents que Crébillon (le dramaturge), Voltaire le polémiste et l'ironiste, ou d'Alembert, le mathématicien et le collaborateur de l'Encyclopédie."} in \emph{la littérature française de A à Z}}" se sont attaqués à l'absolutisme, à l'argument d'autorité et à « l'obscurantisme ». Leur audience est allée croissant, surtout à partir des années-charnières 1760-1765 et un grand nombre d'usages et de croyances millénaires se sont mis à poser problème, du moins au sein de la fraction de la population « éclairée ». A leurs yeux des usages séculaires sont devenus insupportables et des réformes jusque là inenvisageables sont devenues évidentes. Parallèlement à ces phénomènes on assistait au décollage économique d'une France industrielle qui se développait à grands pas%
% [1]
\footnote{Sources principales :
\\José \fsc{CUBERO}, \emph{Histoire du vagabondage du Moyen Âge à nos jours},1998.
\\Jacques \fsc{DONZELOT}, \emph{La police des familles}, 1977.
\\Bronislaw \fsc{GEREMEK}, \emph{La potence ou la pitié, l'Europe et les pauvres du Moyen Âge à nos jours}, 1987.
\\Jack \fsc{GOODY}, \emph{L'évolution de la famille et du mariage en Europe}, 1985.
\\Jean-Philippe \fsc{LÉVY} et André \fsc{CASTALDO}, \emph{Histoire du droit civil}, 2002.}% 
.

 
\section{Contestation de la position de l'Église}

 Outre le service du culte dans les quarante mille paroisses du pays, l'Église avait  depuis la fin de l'antiquité la haute main sur l'assistance (hôpitaux, hospices, enfants des hôpitaux placés en nourrice, etc.) et l'enseignement (petites écoles, collèges et universités). Les prêtres, religieux et religieuses fournissaient la majeure partie du personnel des hôpitaux, des collèges et des universités et les dirigeaient comme ils dirigeaient couvents et monastères. Dans la plupart de cas ces institutions avaient été fondés sur la base d'initiatives individuelles. Elles vivaient du revenu des biens reçus de leurs bienfaiteurs (le plus souvent des legs faits à l'article de la mort par des personnes privées). Ces biens ressortissaient du régime juridique et fiscal des biens ecclésiastiques (biens de \emph{mainmorte} inaliénables). En complément elles pouvaient recevoir des subventions ou le produit de taxes affectées ou encore de quêtes . 
 
 Aux institutions d'Eglise appartenaient donc%
% [2] 
\footnote{Cf. François \fsc{BLUCHE}, \emph{L'ancien régime, institutions et sociétés}, 1993, p. 71-72.} 
les églises, les cures, et les bâtiments nécessaires à l'activité des hôpitaux, collèges et universités. Elles possédaient par ailleurs des biens de toute nature (7~\% des terres du royaume, des maisons de rapport,~etc.) dont les revenus subvenaient aux besoins de ces diverses institutions. 

 Au fil du \siecle{18} ce modèle économique a fait l'objet de critiques de plus en plus sévères. L'opinion publique considérait que la gestion des biens ecclésiastiques était négligente et entachée d'amateurisme. D'autre part il existait de grandes disparités de revenus entre communautés religieuses : certaines étaient plus riches que nécessaire, et parfois même très riches, pendant que beaucoup d'autres vivaient dans une gêne extrême. L'opinion dénonçait le tribut prélevé sur les revenus de ces biens par de nombreuses rentes de situation, par divers emplois fictifs, et d'abord par le plus criant des abus, c'est-à-dire le système de la \emph{commende}
\footnote{François Ier avait obtenu du Pape le droit d'attribuer aux clercs de son choix (le \emph{bénéficier}) les revenus des abbayes ou des couvents (rarement ceux d'un hôpital) au même titre que les autres \emph{bénéfices} ecclésiastiques (évêchés et cures). C'était une façon pour lui de récupérer les revenus des biens qui avaient été donnés par les autorités civiles aux ordres religieux et une façon déguisée de soumettre les religieux à une imposition, alors qu'ils étaient théoriquement non imposables. Cela les empêchait de devenir financièrement trop puissants. C'était un moyen de récompenser les familles qui lui avaient rendu des services en casant ainsi l'un de leurs cadets. Cela lui permettait aussi de récompenser les clercs qui lui avaient plu ou qui se mettaient à son service : un bénéficier pouvait en effet recevoir plusieurs commendes en même temps (c'est l'un des outils qui ont permis à Richelieu ou à Mazarin de s'enrichir). Cf. \fsc{MINOIS}, 1989. Il n'était pas nécessaire qu'il appartienne à la maison concernée, ni qu'il y réside, et il n'y exerçait aucune autorité spirituelle. Il fallait et il suffisait qu'il soit homme et tonsuré, ce qui ne l'engageait à rien en termes de vie religieuse, à part le port de la tenue ecclésiastique et l'interdiction de se marier (qui n'impliquait pas le voeu de chasteté). Par définition le bénéficiaire de cette nomination ne faisait pas non plus vœu de pauvreté. Il suffisait qu'il laisse aux moines de quoi vivre, après quoi il pouvait consommer tout le reste. Les abbés commendataires qu'on rencontre dans bien des pages des romans et mémoires du XVIIIème, promenant de salon en salon leur ennui distingué avaient financièrement intérêt à ce qu'il y ait le moins de religieux possible, et à réduire au minimum les dépenses d'entretien des couvents, les aumônes aux pauvres et autres dépenses improductives. De leur côté les religieux avaient intérêt à tirer de leurs biens le maximum de revenus pour qu'il leur en reste assez après les prélèvements des commendataires, ce qui pouvait les encourager à se montrer plus durs en affaire etplus exigeants face à leurs fermiers et locataires.}. 

 On reprochait surtout aux institutions religieuses d'être trop nombreuses, de continuer d'être fondées sans cesse et sans concertation avec les autorités civiles et religieuses, sans analyse des besoins et des ressources, et de stériliser au détriment de l'intérêt commun une part sans cesse croissante de la richesse nationale (puisqu'elles n'avaient pas d'héritiers). Les économistes de cette époque pensaient en effet que le total des richesses existantes était fixe, inextensible. La croissance d'une famille nouvelle (charnelle ou spirituelle) impliquait donc à leurs yeux l'appauvrissement de toutes les autres : \emph{"un des principaux objets de notre attention, ce sont les inconvénients de la multiplication des établissements des gens de mainmorte et la facilité qu'ils trouvent à acquérir des fonds naturellement destinés à la subsistance et à la conservation des familles, \emph{[...]} qui ont souvent le déplaisir de s'en voir privées \emph{[...]} en sorte qu'une très grande part des fonds de notre moyenne\footnote{classe moyenne} se trouve actuellement possédés par eux ..."} (Édit du 25 Août 1749). 

 Ces problèmes patrimoniaux et ces abus financiers n'avaient rien de nouveau : ils existaient dès le moyen-âge (y compris la commende, en dépit des efforts des papes). S'ils ont été mis en avant durant la seconde moitié du \siecle{18}, c'est que le monopole de l'Église sur les fonctions d'assistance et d'enseignement n'allait plus de soi : son autorité morale comme le principe même de la vie religieuse ont été mis en cause. La tradition d'utiliser les monastères pour la régulation des familles était en train sinon de tomber en désuétude, du moins d'être mise en question (ainsi Diderot, dont une soeur était entrée au couvent, écrit \emph{la religieuse 
\footnote{Le roman \emph{La religieuse} de \fsc{DIDEROT} paraît en 1782 : il exprime cet état d'esprit. Cette \emph{effrayante satire des couvents}, selon l'auteur lui-même, raconte l'histoire d'une fille contrainte par sa famille à prendre le voile et à vivre dans un couvent aux modes de vie et de penser terrifiants.}}, pamphlet contre la vie claustrale). Les pères de famille n'avaient plus aussi fortement besoin de l'Église pour caser leurs enfants surnuméraires, soit que les pratiques de limitation du nombre des héritiers qui se répandaient rapidement à cette époque%
% [5] 
\footnote{Banalisation des abandons. Recours aux méthodes de prévention des naissances disponibles alors : \emph{coïtus interruptus} certainement, douches vaginales,~etc.} 
aient diminué le nombre des enfants à établir, soit que le développement économique du XVIIIème siècle ait offert aux cadets de famille ou aux jeunes ambitieux sans fortune des perspectives plus alléchantes que l'entrée en religion. L'état général des esprits de l'époque de Louis XV et de Louis XVI ne portait ni à la vie contemplative ni au mysticisme et sans qu'on puisse en faire une règle générale il semble que beaucoup de monastères aient été à cette époque dans un état pitoyable : effectifs squelettiques, ferveur discutable ou absente, indiscipline, non respect des règles de l'ordre... 

 Est-ce pour cela que commençait de paraître scandaleuse
l'idée qu'on puisse à vingt ans aliéner sa liberté de manière définitive en prononçant des vœux perpétuels, sans tenir compte des évolutions psychologiques et intellectuelles qu'une vie peut entraîner ? Les \emph{philosophes} n'épargnaient pas davantage l'indissolubilité du mariage, qui leur paraissait une oppression du même ordre. Mais dans le cas des religieux cela leur paraissait d'autant plus monstrueux que depuis la fin de l'Antiquité le droit privait ceux-ci de tous leurs droits familiaux et en faisait des morts civils. 

 Le premier juin 1739, sur requête du Parlement de Metz, Louis~XV publiait un édit spécial à destination de la Lorraine {[...] \emph{pour empêcher que par des voies indirectes on ne fasse de nouveaux établissements sans autorisation, soit pour empêcher les communautés autorisées de faire sans permission de nouveaux acquêts}}. Il interdisait qu'aucune donation, qu'aucun legs et qu'aucune rente ne soit plus acceptés à l'avenir par une communauté religieuse sans permission, et qu'aucune acquisition ne soit faite par une communauté, qu'aucune communauté nouvelle ne soit créée sans une enquête préalable \emph{de commodo et incommodo}. Il défendait à quiconque de se faire prête-nom pour des religieux. Il donnait droit aux gens lésés par des dons ou des ventes non autorisés de réclamer leurs biens aux communautés concernées,~etc. En 1749, Louis~XV étendait par édit ces règles à l'ensemble du royaume : \emph{il ne sera fait aucun nouvel établissement, chapitre, séminaire, communauté religieuse quelconque même sous prétexte d'hospice, de quelque qualité que ce soit, sans permission expresse par lettres patentes enregistrées}. Il s'agissait de distinguer entre les fondations d'intérêt public (dont les hôpitaux et les hospices, mais seulement au cas par cas, et s'ils étaient approuvés par les autorités civiles) et les autres (couvents et monastères divers). 

 Les autorités civiles contestaient à l'Église le droit de s'opposer à l'autorité de l'État, garant de \emph{l'intérêt général} et de la tranquillité publique, ainsi que le montre l'arrêt du conseil du 24 mai 1766 : [considérant que] \emph{s'il appartient à l'autorité spirituelle d'examiner et d'approuver les instituts religieux dans l'ordre de la religion ; si elle seule peut consacrer les vœux, en dispenser ou en relever dans le for intérieur, la puissance temporelle a le droit de déclarer abusifs et non véritablement émis les vœux qui n'auraient pas été formés suivant les règles canoniques et civiles, comme aussi d'admettre ou de ne pas admettre les ordres religieux suivant qu'ils peuvent être utiles ou dangereux dans l'État, même d'exclure ceux qui s'y seraient établis contre lesdites règles ou qui deviendraient nuisibles à la tranquillité publique,~etc.}
 
\section{Vers une Education Nationale}
 À une période où sous la pression des parents et des municipalités il s'est peut-être créé plus de collèges et d'internats éducatifs qu'à aucune autre, la société civile, par l'intermédiaire des Parlements, entendait exercer un contrôle sur les contenus de l'enseignement, et ne plus laisser les mains libres à des religieux qui se situeraient au-dessus de la nation ou de l'État (ou du roi), même au nom d'une légitimité spirituelle supranationale. 
 La Chalotais, procureur-général du roi au Parlement de Bretagne, avait en 1761 contribué plus que quiconque à l'expulsion des Jésuites du Royaume de France\footnote{Il avait été chargé en 1761 d'instruire l'affaire qui a conduit à cette expulsion : cf. le \emph{"Compte-rendu des constitutions des Jésuites",} édition de 1762 (disponible sur Gallica).}. 
Pour lui il était du domaine de l'Etat et de lui seul de régir l'Education :  \emph{"Je prétends revendiquer pour la Nation une éducation qui ne dépende que de l'Etat, parce qu'elle lui appartient essentiellement ; parce que toute Nation a un droit inaliénable et imprescriptible d'instruire ses membres ; parce qu'enfin les enfans de l'Etat doivent etre élevés par des membres de l'Etat."}(p. 27 de son \emph{Essai d'Education Nationale, ou plan d'études pour la jeunesse}, publié en 1763, disponible sur Gallica).

Il est symptomatique que le vœu d'obéissance à une autorité étrangère (le Pape) ait été l'un des principaux griefs formulés contre les jésuites :
  \emph{"Comment a-t-on pu penser que des hommes qui ne tiennent point à l'Etat, qui sont accoutumés à mettre un religieux au-dessus des chefs des Etats, leur Ordre au-dessus de la Patrie, leur Institut et des Constitutions au-dessus des Loix, seraient capables d'élever et d'instruire la jeunesse d'un Royaume. L'enthousiasme et les prestiges de la dévotion avaient livré les François à de pareils instituteurs, livrés eux-mêmes à un maître étranger. Ainsi l'enseignement de la Nation entière, cette portion de la législation qui est la base et le fondement des Etats, était resté sous la direction immédiate d'un Régime Ultramontain, nécessairement ennemi de nos Loix. "} (page 23 de l'\emph{Essai d'Education Nationale..}).  Il était en effet beaucoup plus difficile d'imposer quelque chose aux Jésuites, adossés à une structure supranationale et à l'autorité des papes, qu'à des prêtres séculiers soumis à des évêques nommés par le roi. Pourtant l'autre critique faite aux religieux était sans doute plus grave : on leur reprochait de préparer leurs élèves à entrer dans les ordres (à travers l'importance des exercices religieux et la prédominance du latin, des lettres et de la philosophie au détriment des autres matières) et non à prendre une place dans les emplois laïcs qu'ils étaient appelés à exercer et dont avait besoin l'Etat : \emph{"doit-on s'étonner que le vice de la Monasticité ait infecté toute notre éducation ? Un Etranger à qui on en expliqueroit les détails, s'imagineroit que la France veut peupler les Séminaires, les Cloîtres et des Colonies Latines. Comment pouroit-il supposer que l'étude d'une langue étrangère, des pratiques de Cloître, fussent des moyens destinés à former des Militaires, des Magistrats, des chefs de famille propres à remplir des différentes professions, dont l'ensemble constitue la force de l'Etat."}
 
 À partir de leur expulsion en 1761, les nombreux collèges tenus par les Jésuites ont donc été pris en main par les représentants du pouvoir civil, qui en ont confié la gestion quotidienne au personnel de leur choix, et qui sont entrés jusque dans les détails de leur fonctionnement (et un parlementaire comme La Chalotais avait beaucoup d'idées sur ce qu'il fallait enseigner et comment ainsi que l'annonce le titre de son essai). Centrés sur les contenus et la pédagogie ils n'avaient rien ni contre les laïcs mariés ni contre les pretres pourvu qu'ils fussent compétents, comme \emph{"...les Prêtres de l'Oratoire, qui sont dégagés des préjugés de l'Ecole et du Cloître, et qui sont citoyens."} (p. 27), qui partageaient leur sympathie pour le jansénisme gallican. Ils entendaient que les professeurs des collèges préparent les jeunes « de famille » à vivre dans le siècle. Ils entendaient que le monde, avec ses langues vivantes (dont le français), sa géographie et son histoire, ses réalités matérielles, ses sciences et ses techniques profanes, soient introduits dans la formation des futures élites. 



Le désir de réformer l'enseignement des élites  et de l'adapter aux besoins du temps n'allait pourtant pas jusqu'à les vouloir plus nombreuses : cela ne concernait qu'environ 1\% de la population scolarisable totale et au plus 20\% de la population scolarisable des villes. Au contraire et toujours selon La Chalotais : \emph{"Il n'y a jamais eu tant d'Etudians dans un Royaume où tout le monde se plaint de la dépopulation : le Peuple même veut étudier ; des Laboureurs, des Artisans envoient leurs enfans dans les Colleges des petites Villes, où il en coûte peu pour vivre ; et quand ils ont fait de mauvaises études qui ne leur ont appris qu'à dédaigner la profession de leurs pères, ils se jettent deans les Cloîtres, dans l'Etat Ecclésiastique ; ils prennent des Offices de Justice, et deviennent souvent des Sujets nuisibles à la Société. [...] Les Freres de la Doctrine Chretienne, qu'on appelle Ignorantins, sont survenus pour achever de tout perdre ; ils aprennent à lire et à écrire à des gens qui n'eussent dû apprendre qu'à dessiner et à manier le rabot et la lime, mais qui ne le veulent plus faire. Ce sont les rivaux ou les succeffeurs des Jésuites. Le bien de la Société demande que les connoissances du Peuple ne s'étendent pas plus loin que ses occupations. Tout homme qui voit au-delà de son triste métier, ne s'en acquittera jamais avec courage et avec patience."} 

A la même époque un enseignement de niveau secondaire \emph{sans latin} a été mis en place à l'initiative en particulier des \emph{frères des écoles chrétiennes}\footnote{..à qui les statuts donnés par leur fondateur interdisaient l'enseignement du latin afin qu'ils ne soient pas détournés de leur vocation qui était d'enseigner aux pauvres.} à l'intention des jeunes (citadins surtout) qui se destinaient à des métiers techniques. C'est aussi à cette époque que l'on crée les premières grandes écoles, puisque l'université refusait de développer en son sein certains des enseignements techniques de niveau supérieur dont la société d'alors commençait à éprouver un besoin impérieux (exemples : Chirurgie, Ponts et Chaussées, Mines,~etc.). Dans le même esprit dès le début du XVIIIème siècle ont été créées des écoles militaires avec internat obligatoire, à destination des jeunes nobles, qui n'accordaient au latin qu'une place mesurée, au bénéfice de l'histoire ou d'autres matières plus adaptées. 





\section{La Commission Royale des Réguliers}


 Le souci du contrôle des religieux par l'Etat à conduit à la création d'une \emph{Commission royale des Réguliers} qui s'est donné le droit de supprimer, entre 1766 et 1784, 458 monastères et couvents%
% [9] 
\footnote{François \fsc{BLUCHE}, idem, p. 68.} 
sans en référer à Rome et en dépit de l'opposition d'une partie du clergé français. Sur son instigation, un édit royal de 1773 supprimait \emph{l'exemption} qui depuis le haut Moyen Âge interdisait aux évêques (nommés par le Roi, contrôlés par lui, et qui donc le représentaient) d'exercer leur autorité sur tous les couvents et monastères de leurs diocèses. Il leur confiait la mission de les contrôler. 

 En 1768 la même Commission repoussait à 21 ans pour les garçons et 18 ans pour les filles l'âge à partir duquel ils avaient la capacité de prononcer des vœux solennels. L'objectif était de protéger la liberté des jeunes gens contre toutes les formes d'oppression : celle des pères était sans doute visée au moins plus que celle des couvents qui n'avaient rien à gagner à s'encombrer de membres malheureux, aigris ou révoltés.

 Par ailleurs la part d'héritage donnée aux enfants destinés à entrer en religion était limitée par les autorités civiles pour empêcher toute surenchère entre familles, et pour ne pas immobiliser trop d'argent dans des institutions en principe vouées à la pauvreté et servant, en fait sinon en droit, à protéger les familles d'un éparpillement de leur patrimoine. Elles pouvaient aussi se borner à payer une pension viagère pour leur fils ou fille religieux, sans qu'il soit plus question d'accroître le capital du couvent.

 Cela étant dit les relations entre les autorités civiles et le monde des clercs et religieux ne se résumaient absolument pas à ces antagonismes. C'était bien plus complexe, et les prêtres et les religieux ne venaient pas d'un autre monde que leurs contemporains. Au contraire ils étaient en majorité issus des couches bien insérées (sinon aisées) de la population. En 1761, c'est à des ecclésiastiques que les parlementaires confient les collèges dont ils venaient d'expulser les jésuites. Les clercs soumis aux évêques, soumis eux-mêmes au roi, avaient toute leur confiance. Les uns et les autres partageaient la même sympathie pour les thèses jansénistes et gallicanes et souscrivaient aux mêmes volumes de l'Encyclopédie. La Chalotais comme ses collègues comptaient sur les prêtres et les religieux pour l'éducation morale des élèves. Ils attendaient d'eux qu'ils confortent le droit naturel par les sanctions promises par la Révélation chrétienne. 
 
 De la même façon ce sont les juges du tribunal d'Abbeville qui en 1766 condamnent à mort le jeune chevalier François-Jean Lefebvre de la Barre pour blasphème et sacrilège, et ceux de la Grande Chambre du Parlement de Paris qui la même année confirment la sentence. Cette sentence sera exécutée en dépit de l'intervention de l'évêque dAmiens auprès de Louis XV pour obtenir sa grâce, demande refusée par le roi !
 
 De la même façon Tenon, chirurgien réputé, exprimait en 1788 toute son estime pour les religieuses que sa carrière l'amenait à côtoyer jour après jour dans les hôpitaux de Paris. Alors que de sa place il participait avec brio aux mouvements de réflexion, de rationalisation et de modernisation des Lumières, il n'imaginait pas un instant se passer de leurs services. 


\section{Nouvelles pratiques procréatives et éducatives}

 En conséquence du retour au droit romain à partir du \siecle{12}, la puissance paternelle avait été restaurée dans toute sa force à partir de la fin du Moyen Âge, dans les pays de droit écrit surtout, mais aussi dans le reste de la France%
% [10]
\footnote{Cf. \emph{Histoire des pères et de la paternité}, Collectif, 1990, édition 2000.}% 
. Aussi longtemps que le père vivait il conservait sa puissance de décision dans les domaines essentiels de la vie de ses enfants, même devenus adultes (mariage, achats et ventes de pièces du patrimoine...). 

 Mais le roi soleil, qui avait donné un éclat incomparable à la monarchie de droit divin, avait également révoqué l'Edit de Nantes, et c'est lui qui avait ordonné les persécutions qui s'en étaient suivies pendant des générations contre les membres de la « {religion prétendue réformée} » (RPR). C'est donc lui qui avait attaqué la fonction paternelle dans la personne de ceux qu'il avait disqualifiés aux yeux de leurs propres enfants en ne reconnaissant pas leurs unions conjugales comme légitimes. Cela faisait de ces enfants des bâtards et les empêchait d'hériter des biens de leurs parents, ce qui gênait beaucoup leur établissement dans la vie. C'est lui qui avait enlevé aux parents réformés le droit d'élever leurs enfants en émancipant ces derniers dès l'âge de raison (7 ans). En faisant tout cela il avait placé les représentations idéologiques et (surtout ?) le pouvoir de l'État au-dessus des pères qu'il prétendait pourtant défendre : il s'était conduit comme un père abusif%
% [11]
\footnote{Maurice \fsc{CAPUL}, \emph{Infirmités et hérésies, les enfants placés sous l'ancien régime} (tome II), 1989, 1990.}% 
.


 Au contraire les auteurs des Lumières voulaient que les pères soient au service de l'épanouissement de leurs enfants, et que leur autorité ne s'exerce que durant le temps où ces derniers étaient incapables de se conduire seuls. Ils voulaient qu'ils appuient leur autorité sur l'affection plutôt que sur la crainte. En 1762 paraissait \emph{L'Émile} et son succès était immédiat, ce qui prouve combien ce roman était en accord avec l'air de son temps. Jean-Jacques \fsc{ROUSSEAU}%
% [12] 
\footnote{Compte tenu de leurs expériences personnelles, ni Rousseau ni Voltaire ni D'Alembert ne pouvaient supporter que puisse exister quelque chose comme un droit supérieur, divin, des pères. De même plusieurs des personnages emblématiques de la Révolution Française ont eu maille à partir avec leur père et avec le droit de correction paternelle tel qu'il pouvait s'exercer sous l'Ancien Régime : Mirabeau, Sade...} 
y proposait une nouvelle image de l'enfance et des rapports parents--enfants. 

 




 À cette époque, le poids des interdits religieux avait diminué, au moins dans les villes et dans certaines campagnes, dont celles du bassin parisien, ce qui facilitait à la fois les relations sexuelles hors mariage et le refus des géniteurs masculins de « réparer » en cas de grossesse, comme c'était l'usage jusque là quand un mariage, des vœux religieux ou l'inégalité des conditions ne s'y opposaient pas. La croissance des villes et des fabriques, ateliers, mines et autres industries nouvelles concourait au relâchement de la pression sociale sur les comportements individuels. Le nombre des naissances hors mariage non légitimées par mariage subséquent avait donc augmenté. 

 Mais même dans les couples stables, concubins ou mariés, le recours à l'abandon s'était généralisé. Les scrupules religieux avaient cessé de le freiner. On s'était mis à pratiquer l'abandon des nouveaux-nés dans tous les milieux et de plus en plus souvent à visage découvert. L'abandon était devenu un droit pour tous, exercé sans honte, sans qu'il provoque de questions, d'enquête, de formalités, ou de poursuites, même en dehors des cas de nécessité vitale, et c'est ce qui était nouveau. C'était devenu un moyen comme un autre de régulation des familles. 
 
  Si l'on décomptait \nombre{312} abandons à Paris en 1670, du temps de Monsieur Vincent de Paul, on en dénombrait \nombre{5842} en 1790, pour environ \nombre{600000} habitants. Ils représentaient 40~\% des naissances parisiennes en 1772, et 33 à 34~\% à la veille de la révolution%
% [13]
\footnote{Pour l'ensemble de la France de 2010 de tels taux donneraient un nombre d'abandon de l'ordre de \nombre{600000} enfants (six cent mille), soit trois fois plus que le nombre actuel d'IVG, et au bas mot 600 fois plus que le nombre d'abandons actuels.}%
. À la fin du \siecle{18} les mœurs ont donc beaucoup changé.

 Pour être juste il faut dire aussi que le nombre des abandons dans les villes était artificiellement gonflé et cela d'autant plus qu'elles étaient grandes. À Paris c'était clairement le cas. On y envoyait des enfants de plusieurs \emph{centaines} de kilomètres à la ronde. Néanmoins la croissance du nombre et du pourcentage des abandons depuis l'époque de Monsieur Vincent était indiscutable et massive. La grande période de l'abandon d'enfant, la période où il a été utilisé de la façon la plus massive et la moins contestée, se situe entre 1760 et 1860%
% [14]
\footnote{... en 1810 : \nombre{55700} abandons sur toute la France ; en 1833 : \nombre{164000} abandons.}%
.

 Le grand public culpabilisait d'autant moins l'abandon qu'il croyait en \emph{la bonté de l'éducation donnée par les hôpitaux}. \fsc{Rousseau} explique dans ses \emph{Confessions} que s'il a abandonné ses cinq enfants, c'est parce que \emph{tout pesé, je choisis pour mes enfants le mieux ou ce que je crus l'être. J'aurais voulu, je voudrais encore avoir été élevé et nourri comme ils l'ont été}. Il croyait que les nouveaux-nés abandonnés avaient de réelles chances de survie. Cette croyance semble à l'époque avoir été très largement partagée. Les administrateurs des hôpitaux étaient presque les seuls à savoir combien la réalité était loin de cet idéal. 

 Les enfants pouvaient être abandonnés à tout âge, et le lien était évident entre le nombre des abandons et les crises économiques. Le manque de ressources des parents, leur maladie, le chômage, ou encore le veuvage, expliquent que des enfants n'étaient pas abandonnés à la naissance, mais après un certain nombre de mois ou d'années. Les mères seules étaient dans une situation économique particulièrement fragile : à travail égal les femmes étaient \emph{beaucoup} moins bien payées que les hommes. 



 Parmi les nouveaux-nés des villes placés en nourrice%
% [15] 
\footnote{La grande majorité des enfants des villes, même non abandonnés, vivaient alors leurs premières années en placement nourricier rural : \emph{1780 : Le lieutenant de police Lenoir constate, non sans amertume, que sur les \nombre{21000} enfants qui naissent annuellement à Paris, \nombre{1000} à peine sont nourris par leur mère. \nombre{1000} autres, des privilégiés, sont allaités par des nourrices à demeure. Tous les autres quittent le sein maternel pour le domicile plus ou moins lointain d'une nourrice mercenaire. Nombreux sont les enfants qui mourront sans avoir jamais connu le regard de leur mère. Ceux qui reviendront quelques années plus tard sous le toit familial découvriront une étrangère : celle qui leur a donné le jour.} (cité par Élisabeth \fsc{BADINTER}). Les propos du lieutenant de police montrent aussi qu'en 1780, les bébés et l'allaitement maternel font désormais partie des sujets de préoccupation légitimes d'un haut fonctionnaire.} 
à la campagne par leurs parents, un certain nombre entraient dans la catégorie des enfants abandonnés si leurs parents ne payaient plus les gages de la nourrice. Lorsque celle-ci n'obtenait pas de réponse à ses réclamations, il allait de soi qu'elle remettait l'enfant à l'hôpital le plus proche : elle n'avait pas reçu de mandat pour faire autre chose, et elle avait besoin de son salaire. 




 Traditionnellement les enfants placés en nourrice par les hôpitaux y revenaient quand leur petite enfance était achevée. Mais dès 1696 le \emph{bureau de l'Hôpital} observait que : [...] \emph{les enfants qu'on ramène à 4 ans à Paris s'accoutument mal à l'air de la capitale et qu'il en meurt beaucoup. On pense qu'il serait bon de les laisser un an de plus à la campagne...} C'est pourquoi au fil du \siecle{18} leur séjour à la campagne s'est prolongé.

 De nouveaux règlements sont édictés en 1761 par l'Hôpital des Enfants Trouvés de Paris%
% [16]
\footnote{Comme celui-ci exerce un rôle de modèle national puisqu'il reçoit le tiers des indigents du royaume, et que le roi suit de très près ce qui se passe dans la ville dont il est le seigneur, ces règlements vont avoir une postérité importante.}% 
. En ce qui concerne ces enfants-là, le placement dans des familles nourricières est désormais mis sur le même pied que l'internat de l'hôpital. On lui reconnaît une valeur au moins égale, au nom de la vie qu'il permet de sauver. 

 À cette époque le sort ordinaire des enfants ordinaires était de commencer très tôt à travailler chez leurs parents ou chez le maître où ceux-ci les avaient placés : souvent dès l'âge de 6 ans. Dans les villes l'entrée au travail attendait dans les meilleurs cas l'issue de la scolarité dans une petite école, scolarité qui durait fort peu de temps. Si le jeune était placé chez un maître, celui-ci avait une large délégation de l'autorité parentale. Il en était ainsi depuis l'Antiquité pour la majorité de la population, pour tous les humbles. Le sort des enfants placés en nourrice paraissait donc naturel et normal, à défaut d'être désirable.

 Il s'agissait d'insérer l'enfant dans un milieu naturel rural ou artisanal, et si possible de lui donner une famille, même si l'adoption demeurait impensable et impossible. Comme l'observera en 1790 \fsc{LA ROCHEFOUCAULT-LIANCOURT}, du Comité de Salut Public, une génération après la prise de décision de ne pas ramener ces jeunes à l'hôpital : \emph{presque tous les enfants conservés par les nourrices sont gardés dans leurs maisons jusqu'à ce qu'ils se marient, y sont traités comme leurs propres enfants, le plus grand nombre tourne bien et ils deviennent de bons habitants des campagnes}. Les enfants placés en nourrice restaient les enfants de l'hôpital, employeur des nourrices, qui avait pleine autorité sur eux, et exerçait l'autorité paternelle jusqu'à leur majorité (25 ans) ou leur mariage (pour les filles). 

 D'emblée ce système a été jugé satisfaisant, mis à part le fait que beaucoup de garçons avaient tendance à s'en aller avant d'avoir eu 25 ans, pour gagner de l'argent. D'autre part un certain nombre de garçons étaient \emph{... renvoyés par le nourricier}. Comme toujours les filles posaient nettement moins de problèmes de discipline que les garçons. 

 Désormais l'objectif était de faire grandir un futur sujet pour le service du roi et de l'État. Il était encore moins question qu'auparavant d'enlever systématiquement leurs enfants aux indigents pour les placer dans un Hôpital coûteux et à la valeur éducative douteuse. À un moment de crise économique (1770) le ministre Turgot a ordonné qu'on mette en place dans chaque paroisse un \emph{bureau d'aumône}, ou \emph{bureau de charité}, à l'intention des pauvres domiciliés, et d'eux seuls, avec pour mission de redistribuer des taxes levées sur les propriétaires aisés de la paroisse%
% [17]
\footnote{Ces institutions n'existaient alors que dans certaines paroisses, même si selon des décisions vieilles de plusieurs siècles, et jamais abrogées, elles auraient dû exister partout. Inutiles durant les périodes de bonne santé économique, elles étaient de celles qu'il fallait refonder constamment.}% 
. Pour secourir les pauvres dociles, les bons pauvres, les femmes seules chargées de famille, les journaliers au chômage, pour protéger les jeunes filles pauvres et en danger de « se perdre » dans la prostitution, sans pour autant les héberger ni les prendre en charge totalement, il a fait ouvrir, ou plutôt rouvrir, des \emph{ateliers de charité}. Les pauvres y travaillaient comme ils l'auraient fait à l'hôpital, et ils continuaient à vivre à leur domicile, dans leur communauté. Ils n'étaient pas déracinés, ni désocialisés, et cela coûtait moins cher. 

 Pour éviter les abandons, les hôpitaux (ou du moins certains hôpitaux) aidaient financièrement les mères indigentes à nourrir chez elles leur propre enfant. Cela ne leur coûtait pas plus cher que de mettre un enfant abandonné en nourrice, mais en termes de survie c'était bien plus efficace : ainsi la mortalité des bébés de Rouen vivant avec leur mère, alors que celles-ci étaient secourues à domicile par l'Hôpital Général (c'étaient donc des indigentes) ne dépassait pas 18,7~\% entre 1777 et 1789%
% [18]
\footnote{Selon Élisabeth \fsc{BADINTER}.}% 
. À la même période ceux des enfants qui étaient mis en nourrice par leurs parents, avec l'aide matérielle du même Hôpital (c'étaient donc des indigents eux aussi), subissaient une mortalité de 38,1~\%. Quant à ceux qui étaient abandonnés à ce même hôpital, il en mourait plus de 90~\%. À Lyon il en était de même à la même période : les bébés nourris par les mères qui ont été secourues à domicile par le bureau de bienfaisance maternelle n'ont subi de 1785 à 1788 qu'un taux de mortalité de 16~\% avant l'âge d'un an. Ces taux étaient très bons pour l'époque, même comparés à ceux des familles non indigentes dont les mères nourrissaient elles-mêmes : il est vrai qu'il s'agissait d'enfants uniques (sans quoi leurs mères n'étaient plus jugées dignes, « méritantes », de bénéficier d'une telle mesure), qu'elles avaient voulu les garder et les élever, et qu'elles avaient le temps de s'en occuper.

 Des recherches véritablement scientifiques ont été menées afin de diminuer la mortalité infantile en collectivité. À partir de 1784 une expérience a été conduite dans l'une des salles de la Couche de Paris sous l'impulsion et le contrôle du corps médical. Elle avait pour principe d'augmenter le taux d'encadrement et l'intensité des relations des bébés avec les soignants. Cette expérience s'est poursuivie pendant 4 ans (1784-1788) sous le contrôle de l'Académie de Médecine. Elle a conduit à une baisse significative du taux de la mortalité. Aussi avec l'approbation de la même faculté de médecine (1788) ces pratiques ont-elles connu un début de généralisation timide : un tel dispositif était en effet fort coûteux, et la Révolution a suspendu sa mise en œuvre.

 Voici ce qu'écrivait en l'an XI \fsc{CAMUS}, membre du Conseil qui avait dans ses attributions les maisons d'enfants trouvés, dans son \emph{Rapport au Conseil général des hospices sur les hôpitaux et hospices, les secours à domicile, la direction des nourrices} : \emph{"Peut-être est-il beaucoup plus difficile de suppléer aux soins de la mère et de la nourrice qu'à leur lait. On est assez avancé dans les connaissances chimiques pour composer une boisson qui ait la qualité du lait de femme, même avec les variations que le lait éprouve pendant la durée de l'allaitement%
% [19] 
\footnote{En réalité à cette date aucun essai n'avait réussi. Tout au plus savait-on à peu près compléter un allaitement insuffisant par des bouillies, et cela ressortait de l'art des mères et des nourrices plus que de l'expertise des médecins.} 
; mais ces tendres soins d'une femme pour l'enfant auquel elle donne une partie de sa substance, cette gestation entre les bras, ces embrassements continus, ces baisers fréquents : en un mot, cette espèce d'incubation qui doit suivre la sortie du sein de la mère, voilà ce qu'on n'obtient ni avec des combinaisons chimiques, ni avec des règlements, ni avec des gages."}%
%[20] 
\footnote{Cité par \fsc{DUPOUX}, idem, p. 136 et 181, 1958.} 

Le regard que les « {philosophes} » portaient sur l'enfance avait-il pour autant radicalement changé ? Même s'ils ne parlaient plus en termes de péché, les pédagogues et philosophes du \siecle{18} ne montraient pas beaucoup plus de vraie confiance en la bonté \emph{naturelle} des enfants que ceux des siècles précédents. Pour \fsc{ROUSSEAU} l'enfant nouveau-né ne portait plus la marque d'un quelconque péché originel, mais il le jugeait sans défenses face aux tentations et trop aisément corruptible par la société, c'est-à-dire d'abord par son entourage immédiat. Cela se traduisait dans \emph{l'Émile} par une pédagogie aussi peu spontanée et naturelle que l'internat le plus contrôlé. 
 
  À partir de 1640 la diffusion de la doctrine janséniste, en dépit de la résistance des jésuites et du roi, mais avec la sympathie des oratoriens et des parlements, avait accru la méfiance traditionnelle face aux tendances spontanées des adolescents. Elle confortait la répression du désir, et du désir sexuel en particulier. Mais la croisade anti-masturbatoire qui a commencé au début du \siecle{18} et qui a continué jusqu'au début du \crmieme{20} ne se fondait pas sur des raisons religieuses. Elle était promue par des médecins réputés qui comme Tissot présentaient les garanties de scientificité les plus sérieuses et qui ont été unanimes pendant deux siècles dans leur foi dans la dangerosité mortelle de la pratique du "plaisir solitaire" des garçons et des filles. Ils fournissaient aux parents et aux éducateurs les arguments nécessaires pour légitimer une surveillance intrusive (et dont il était entendu que c'était "pour leur bien") sur le corps des enfants et adolescent(e)s. 
  
  Le contrôle de la vie d'un jeune était encore plus rigoureux quand il était coupé de ses parents dans le cadre d'un internat totalement maîtrisé (lieux, communications, temps). Il était contenu fermement (murs, grilles, portier, clôture, clés,~etc.) à l'abri du monde extérieur. Il s'agissait de donner de saines habitudes à son corps et à son esprit.  L'internat protégeait les « enfants de famille » contre les « mauvaises influences » qui les « pervertissaient », tout en préservant les « filles honnêtes », la « paix des ménages », et la « tranquillité publique » des « débordements de la jeunesse ». Il était pour les parents une garantie contre les « erreurs de jeunesse » qui réduisent à néant les meilleures stratégies familiales. Les pensionnaires ne perturbaient pas la vie de la cité où ils étudiaient comme étaient capables de le faire des écoliers externes de tous âges vivant loin de leur famille et sur lesquels les logeurs comme les maîtres n'avaient guère d'autorité. 
  
  Si les professeurs de collège n'avaient guère envie d'assumer les contraintes, les soucis pratiques et les responsabilités d'un internat, le prestige grandissant de la formule et la nécessité pratique de regrouper loin de leur domicile une grande part des collégiens originaires des campagnes et des bourgs  ont fait qu'ils ont grossi peu à peu, mais \emph{surtout à partir du milieu du \siecle{18}}. C'est à cette époque que les internats éducatifs ont connu leur taux de créations le plus élevé. Leur croissance s'est poursuivie longtemps et leur réseau n'a été achevé qu'au début du \siecle{20}. Le nombre de places d'internat semble s'être maintenu ensuite sans grands changements jusqu'aux années soixante du \siecle{20} où il a commencé à baisser.
 
 
 
  




