
\chapter{Un enfant pour quoi ? Pour qui ?}


Est-il vrai, si l'on en croit \fsc{COLUCHE}, que {\emph{y a des gens qui ont des enfants parce qu'ils n'ont pas les moyens de s'offrir un chien}} ? Il posait à sa manière une question essentielle et relativement nouvelle : pourquoi fait-on des enfants ? Pourquoi veut-on des enfants ?

 Si jusqu'à ces dernières décennies le mariage alliait deux lignées en associant un homme et une femme dans le cadre d'une division sexuelle du travail indiscutée, la première de ses fonctions, ressentie comme incontournable et justifiée par la survie des individus aussi bien que celle de l'espèce, était de donner des enfants aux hommes%
% [6]
\footnote{{\emph{La fonction principale du mariage était d'ailleurs de fabriquer du père} [...]}, Irène \fsc{THERY}, idem.}% 
. Ils ne pouvaient en effet donner le jour à des semblables, mais ils n'en avaient pas moins un impérieux besoin pour s'occuper d'eux jusqu'à leur mort même quand ils ne pourraient plus subvenir à leurs propres besoins et pour leur succéder. Cela impliquait de donner de la valeur au fait que les enfants aient un père et non un géniteur anonyme. 
 
 Tout était (donc ?) fait pour décourager les femmes de concevoir des enfants sans en passer par un homme publiquement désigné (jusqu'à l'infériorité des salaires féminins à travail égal ?). « À cause des enfants », (grâce aux enfants ?) dont l'avenir, le statut et l'installation dans l'existence dépendaient plus d'eux que d'elles, les hommes tenaient les femmes en leur « main ». Le mariage permettait à presque tous ceux qui le désiraient (c'est-à-dire la plupart des hommes) d'avoir des enfants bien à eux et qui ne leur seraient contestés par personne et d'abord par leur mère. Il leur permettait aussi de s'attacher une femme et les services de tous ordres que seule une femme pouvait alors fournir. 
 
 Mais la réciproque était vraie aussi : le mariage permettait aux femmes d'avoir des enfants sans être obligées de les élever seules, dans la pauvreté et l'illégitimité. Quant à celles qui y attachaient du prix, il leur permettait de s'attacher solidement un homme%
% [7]
\footnote{... ce que symbolisaient depuis l'antiquité les anneaux que s'échangeaient les conjoints, et ce qu'exprimait sur le mode burlesque des expressions comme {\emph{se laisser mettre le grappin dessus}}, ou {\emph{se passer la corde au cou}}.}%
. 
 
 Un homme qui désire des enfants ne peut plus s'y prendre aujourd'hui comme naguère. Il ne lui sert plus à rien de demander à un futur beau-père la main de sa fille, de lui demander un transfert d'autorité, puisque ce dernier ne la détient plus et ne peut donc plus la donner. D'ailleurs lui-même n'a plus besoin d'un gendre pour légitimer les petits enfants que sa fille lui donnera et pour en faire des héritiers, puisqu'il n'y a plus de fonctions interdites aux enfants illégitimes et donc plus d'enfants illégitimes. Il n'y a donc plus d'intérêt commun entre beau-père et gendre, et le soupirant doit négocier seul et sans intermédiaire avec la femme dont il recherche les faveurs. Il n'aura d'elle des enfants que si elle le veut bien. Et elle pourra d'autant plus facilement le quitter en emmenant leurs enfants communs (ou le pousser hors du domicile familial) que l'absence d'un homme à côté d'une mère ne fait plus problème, tandis que la présence de celle-ci semble encore presque indispensable (mais cela changera peut-être si on constate des compétences maternantes au sein des couples masculins ?). 
 
 En ce qui concerne les hommes, les ressources dont ils disposent (puissance économique, puissance militaire, compétences professionnelles,~etc.) ont la vertu de les rendre désirables autant que leurs qualités physiques et psychologiques, et on l'a toujours su. Plus ils sont intellectuellement et professionnellement qualifiés, plus ils ont de chances d'être mariés. C'est le contraire pour les femmes. C'est peut-être une preuve que celles-ci n'ont pas intérêt au mariage dès qu'elles ont les moyens de leur indépendance ? au contraire des hommes ? Le fait que dans certains pays d'Europe un nombre significatif de femmes aux ressources au dessus de la moyenne semble choisir aujourd'hui de ne pas avoir d'enfants montre que pour elles en tout cas la famille et le mariage n'ont pas d'attraits. 
 
 Pour le moment, ces évolutions n'infirment pas la répartition traditionnelle des rôles érotiques masculins et féminins : les hommes se doivent encore d'être « ceux qui peuvent », ceux qui ne sont pas marqués par le manque ou la défaillance (pouvoir politique, financier, intellectuel, militaire, puissance sexuelle...) à défaut de quoi ils n'exercent guère d'attrait sur la plupart des femmes, tandis que lorsque celles-ci ne s'éprouvent pas, au moins un peu, comme « celles qui n'ont pas » (pas tout), comme celles qui « ne peuvent pas » (pas toutes seules), elles n'ont pas besoin des hommes (mais peut-être n'exercent-elles pas non plus d'attrait sur eux ?).
 
 Même si tous les habitants des pays dotés d'un système d'assistance sociale et de retraite suffisamment efficace ont besoin qu'il naisse des enfants pour financer leurs moments d'invalidité et leurs vieux jours, nul n'a \emph{besoin} que ce soient ses enfants à lui. D'un point de vue strictement comptable et sauf dispositifs de compensation \emph{très}  généreux des frais qu'ils entrainent, l'intérêt des hommes et des femmes \emph{des états providence} d'aujourd'hui (et seulement d'eux) est de ne pas avoir d'enfant. Leur niveau de vie et leur crédit auprès des banques sont plus élevés s'ils évitent d'investir dans une progéniture%
% [1]
\footnote{... en dehors des impôts versés pour financer l'enseignement, l'assistance sociale en direction de tous les mineurs, la santé infantile, et les allocations servies aux familles...}% 
. Dans les pays les plus socialement développés, seule la collectivité a \emph{besoin} d'enfants. Il ne faut sans doute pas chercher plus loin la faiblesse des taux de natalité de leurs membres, taux qui ne sont que la résultante des stratégies individuelles de leurs citoyens, stratégies en grande partie rationnelles d'autant plus qu'on ne voit plus aujourd'hui au nom de quel argument on pourrait les leur reprocher. 
 
 Face à la désaffection du mariage et de la procréation qui menaçait l'Empire Romain dans sa survie, l'empereur Auguste a réagi en pénalisant les célibataires et ceux qui n'avaient pas d'enfants, et ses lois ont été appliquées sans faillir pendant au moins trois siècles. Si les taux de natalité baissaient dangereusement, verrions-nous à l'avenir de pareilles incitations légales%
% [2] 
\footnote{... encouragements (ou pénalisation) par l'impôt ou le calcul des retraites ? Allocations couvrant le montant des dépenses d'éducation ? Crèches, internats et autres formes de prises en charge gratuites des enfants et adolescents par la collectivité ? Soutien matériel direct aux jeunes majeurs \emph{sans} conditions de ressources parentales ?~etc.} 
à procréer ? Il existe déjà des éléments de politiques de soutien à la natalité et aux familles, auxquels est attribué le taux de natalité français, comparativement élevé pour un pays développé.
 
 Mais les malthusiens et les écologistes pensent que les problèmes de santé de notre planète ont pour origine le fait que les humains sont trop nombreux. Il faudrait en effet que le nombre de ces derniers diminue drastiquement s'ils voulaient tous consommer comme les citoyens des pays développés actuels sans épuiser les ressources disponibles et sans mettre en danger les équilibres de la nature. Cela impliquerait non pas une croissance démographique zéro, mais une décroissance très énergique. L'intérêt commun de l'humanité serait-il sa décroissance numérique et l'évitement de la reproduction jusqu'au retour à un effectif écologiquement optimal ? 
 
 ... même si le désir de serrer des bébés de chair dans ses bras -- aussi irrépressible qu'irrationnel -- est vraisemblablement aussi fort aujourd'hui que par le passé ...
 
 
 