%C2 EVOLUTION DES DROITS PERSONNELS SOUS L'EMPIRE

%DES DROITS POUR LES FEMMES
\section{Des droits pour les femmes}

 On a vu plus haut qu'Auguste avait décidé que les citoyennes qui avaient donné le jour à trois enfants ingénus (quatre pour les affranchies), qui avaient 25 ans ou plus, et qui n'avaient plus de \emph{pater familias}, étaient dispensées de tutelle. Elles pouvaient gérer elles-mêmes leurs affaires financières ou commerciales, et se marier ou se remarier à leur gré. Pour la première fois à Rome des citoyennes pouvaient ne plus être mineures à vie et accéder à la majorité légale. Cela concernait les citoyennes sans \emph{pater familias}, les orphelines de père, celles dont le père ne possédait pas ou plus les droits d'un \emph{pater familias} (esclave affranchi, citoyen déchu de ses droits par une condamnation, vendu comme esclave, etc.), et les affranchies dont le patron était décédé ou condamné à l'infamie. 

 Mais elles n'accédaient à la majorité légale que par leur utérus, et cela n'en faisait pas pour autant les égales des hommes. Cela ne leur donnait en effet aucun droit sur la conduite d'autrui, même pas sur leurs propres enfants mineurs, qui recevaient un tuteur. Cela ne leur donnait pas non plus la plénitude des droits civiques : elles ne pouvaient ni voter dans les assemblées politiques, ni porter plainte pour autrui ni plaider devant les tribunaux, ni donner la plénitude du statut de citoyen à ceux de leurs esclaves qu'elles affranchissaient, ni adopter un enfant, etc.
 
%DES DROITS POUR LES ENFANTS
\section{Des droits pour les enfants}

 Au cours du premier siècle de notre ère une loi a autorisé un fils ou une fille en puissance de \emph{pater familias} à porter plainte contre ce dernier si celui-ci abusait de son pouvoir. Le magistrat pouvait décider d'émanciper cet « enfant » sous puissance (qui pouvait avoir quarante ans et plus). Si nécessaire il pouvait condamner le père à subvenir aux besoins de cet « enfant » qu'il venait d'émanciper mais qui ne possédait rien en propre (sauf sa solde lorsqu'il était militaire, ou son salaire s'il était fonctionnaire). 
 
 C'est à cette époque qu'a été mise au point la notion de \emph{prestation alimentaire}, distincte des problèmes de transmission de pouvoir, de succession et d'héritage. Cette prestation était fondée sur la seule \emph{pietas}, la piété (familiale), c'est-à-dire l'affection \emph{naturelle} qui doit régner entre tous les membres d'une même famille \emph{biologique}, liés par le sang, indépendamment des liens juridiques. La prestation était due entre ascendants et descendants, dans les deux sens, sans tenir compte du côté paternel ou maternel de la parenté ni de la légitimité des unions ni des filiations. Les mères y étaient astreintes au même titre que les pères. L'obligation alimentaire n'était supprimée ni par le refus de reconnaître un enfant, ni par son émancipation : « \emph{qui fait l'enfant doit le nourrir} ». 

 Peu à peu l'émancipation a (donc ?) cessé d'impliquer l'exclusion hors de la famille, et l'enfant émancipé a progressivement obtenu de la justice un droit à l'héritage de son père. De ce fait, alors qu'elle avait commencé par être une sanction, l'émancipation est devenue un cadeau qu'un père faisait à son fils.

%DES DROITS POUR LES ESCLAVES
\section{Des droits pour les esclaves}

 Face à une vieille noblesse républicaine, très réticente devant le nouveau régime, les premiers empereurs romains ont organisé l'administration sur la base de leur propre domesticité (qu'ils se sont transmise de l'un à l'autre par héritage). Ils ont fait gérer l'Empire par leurs milliers d'esclaves et d'affranchis, qui par nécessité leur étaient tout dévoués. Les plus qualifiés de ces employés ont pu aider le droit à s'infléchir en faveur des esclaves. 

 Chacun des empereurs successifs savait comment il avait lui-même conquis le pouvoir, souvent dans un climat de guerre civile. En effet en l'absence de mécanismes clairs et acceptés de dévolution du pouvoir, à côté des légions qui soutenaient la candidature de leur général (dans l'espoir de se partager les bénéfices de sa victoire), les postulants à l'empire avaient recours à leurs armées d'esclaves pour gagner leur bataille plus ou moins sanglante contre leurs rivaux. Ils savaient de quel secours leur avaient été ces hommes qui n'avaient aucun lien juridique avec la cité et que la loi contraignait à ne rendre de comptes qu'à leur maître. Ils ne pouvaient donc pas accepter que leurs rivaux potentiels, c'est-à-dire beaucoup de monde, acquièrent trop de puissance. Ils avaient un intérêt direct et urgent à s'immiscer dans la relation entre les maîtres et les esclaves, et à faire en sorte que ces derniers deviennent des sujets de leur empire comme les autres. 

 Les lois de l'empire ont essayé de manière répétitive d'empêcher les ventes de citoyens ingénus mineurs à quelque âge que ce soit. Elles n'y sont jamais totalement parvenues, mais elles les déclaraient invalides, ce qui permettait à l'esclave qui prétendait être né libre et n'avoir pas donné son accord à son asservissement de contester en justice son statut d'esclave sans être torturé comme il l'était à chaque fois qu'il formulait toute autre plainte. Il lui fallait évidemment apporter ensuite la preuve de son ingénuité : chaque cité tenait ses registres d'état civil pour les enfants de naissance libre et reconnus.

 Claude, empereur de 41 à 54 de notre ère, a interdit la mise à mort des esclaves vieux ou malades et décidé que les esclaves qui avaient été abandonnés par leurs maîtres (pour impotence, maladie, chômage, etc.) seraient définitivement libres, et que leurs anciens maîtres ne pourraient plus mettre en avant leurs droits antérieurs s'ils survivaient à leur abandon. 

 Néron (fils adoptif de Claude) a donné au Préfet de la ville de Rome le pouvoir de recevoir et d'instruire les plaintes des esclaves contre les injustices de leurs maîtres. Jusque là ils n'avaient aucun droit de porter plainte pour eux-mêmes. À partir de ce moment ils ont eu un recours contre les mauvais traitements.

 La \emph{lex petronia} a subordonné à l'accord du Préfet la vente des esclaves aux proxénètes ou aux \emph{lanistes} (organisateurs des jeux du cirque). En effet lorsque les esclaves concernés n'étaient pas volontaires ces ventes étaient des sanctions très sévères. En ce qui concerne la vente au laniste il s'agissait en pratique d'une quasi-condamnation à mort.

 En 83, sous Domitien, un sénatus-consulte a interdit la castration des esclaves et ordonné de confisquer la moitié des biens des maîtres qui se livreraient à cet exercice. En 138 sous Hadrien les sanctions pour le même délit ont été aggravées : cela montre à la fois la persistance de la politique impériale, et celle des pratiques interdites par ces décisions. 

 À cette époque cela faisait très longtemps que l'assassin d'un esclave était poursuivi comme homicide, sauf quand il s'agissait de son maître lui-même. Au milieu du \siecle{2} de notre ère l'empereur Antonin le Pieux a qualifié d'homicide la mise à mort d'un esclave sur l'ordre de son maître, quand celui-ci n'avait pas d'abord obtenu l'accord d'un magistrat qualifié (Claude l'avait également interdit, ce qui tendrait à prouver là aussi que ce crime n'avait pas disparu). Désormais un maître pouvait être condamné pour ce motif : l'exercice de la justice domestique dans le cadre de la \emph{familia} était placé sous le contrôle de l'État.

 D'autre part les esclaves n'étaient plus tenus aussi rigoureusement que par le passé à l'écart du mariage, ou du moins de la vie en couple mixte stable. On ne regardait plus comme une absurdité de parler d'eux en termes de mari et de femme, même si pour un juriste pointilleux il ne s'agissait que d'une cohabitation%
% [1]
\footnote{\emph{contubernium} = le fait de partager la même « chambrée » ou le même « poste d'équipage ».}
sans effets légaux : en effet les droits du maître sur les deux partenaires et sur leurs enfants restaient entiers. Cette tolérance était-elle due à une diminution de l'approvisionnement du marché aux esclaves, qui donnait de la valeur aux enfants que ces derniers pouvaient concevoir et élever (et le moins cher et le plus sûr moyen pour que ces enfants soient aussi bien soignés qu'il était possible, était que leurs géniteurs les élèvent eux-mêmes) ? Ou bien était-ce un effet du stoïcisme, dont la morale conjugale triomphait sous le Haut Empire, et qui soulignait l'appartenance des esclaves à l'humanité commune ({Épictète} lui-même avait été esclave) ? Ou les deux ? 
 
%TOUS ROMAINS, TOUS EGAUX ?
\section{Tous romains, tous égaux ?}

 En 212 de notre ère l'Empereur Caracalla accorde à tous les habitants de l'Empire la citoyenneté romaine. Ce faisant il leur impose aussi l'institution du \emph{pater familias} et les lois romaines sur le mariage (bien mitigées par rapport à ce qui se passait sous la République). 

 Cela aurait pu entraîner qu'il n'existe plus que deux statuts, celui de citoyen libre et celui d'esclave, si l'on n'avait pas constaté le renforcement de l'aristocratie et de ses privilèges, et s'il n'y avait pas eu de plus en plus de barbares étrangers, de peuples allogènes à l'intérieur même des frontières de l'Empire, immigrés qui obéissaient à leurs propres droits et qui assumaient les activités militaires que ne voulaient plus faire les citoyens romains : ce qu'on désigne sous le nom de « grandes invasions », et qui ont commencé dès le \siecle{3} de notre ère. 

 La distinction entre les nobles, « \emph{clarissimi} » (« les plus beaux, les plus illustres ») et les autres, tous qualifiés de « \emph{humili} » (« humbles ») ou de « \emph{pauperes} » (pauvres), même quand ils étaient financièrement à l'aise, mais sans pouvoir civique, a été affirmée de plus en plus fortement. C'étaient les sénateurs et les chevaliers ainsi que leurs familles, et leurs descendants pendant les trois premières générations qui suivaient l'exercice de la magistrature qui leur avait accordé leur titre : cette noblesse était héréditaire mais se perdait si on ne la retrempait pas périodiquement dans l'exercice des responsabilités (comme dans la Chine ancienne).

 Les nobles étaient les seuls citoyens romains qui n'avaient pas reçu d'Auguste le droit d'épouser des affranchies : la protection de leur pureté familiale était la seule qui comptait vraiment pour le sort de la Cité. À partir de Caracalla ils sont aussi les seuls à conserver la plénitude des droits traditionnels des citoyens romains. Jusque là aucun de ces derniers ne pouvait être soumis à la torture judiciaire, à la « \emph{question} ». Désormais cette protection n'était plus reconnue qu'aux seuls aristocrates, comme s'ils étaient devenus les seuls authentiques représentants des vieux romains : les nobles du moyen-âge hériteront de cette protection. Ce que la citoyenneté romaine avait gagné en extension elle l'avait donc perdu en consistance, et sur un point très significatif le statut des citoyens ordinaires se rapprochait désormais de celui des esclaves.

 Caracalla n'en étendait pas moins à tous les \emph{pères} de l'Empire l'interdiction faite depuis fort longtemps aux citoyens romains de vendre leurs enfants nés libres. Dioclétien a interdit à nouveau aux pères \emph{et aux mères} de vendre leurs enfants ... sans plus de succès que Caracalla. Car il ne suffisait pas d'interdire ces ventes, il aurait fallu en supprimer les causes. En effet la demande d'esclaves n'avait pas disparu, pas plus que la misère. Pour les prolétaires la vente de leurs enfants demeurait l'ultime moyen de survivre libre.
 
 