% Le 10 mars 2015 :
% Antiquité
% Moyen Âge
% ~etc.
% Droit


\section{Constantin et le droit des personnes}


\begin{description}
\item[313] Constantin :
 \begin{enumerate}[leftmargin=*,itemsep=0pt]
%  a)
\item contrairement aux décisions de ses prédécesseurs, reconnaît aux citoyens indigents le droit de vendre leurs enfants, mais seulement à leur naissance ; 
% b)
\item décide que cette vente suspend la puissance paternelle. Le père garde le droit de récupérer son enfant, même contre le gré du possesseur, mais à la condition de fournir en échange un esclave de valeur équivalente, ou la somme correspondante ; 
% c)
\item reconnaît à celui qui recueille un nouveau-né exposé ou qui l'achète à son parent le droit d'en faire son esclave. Cette décision aggravait indiscutablement le sort juridique des enfants puisque le statut d'esclave ne pouvait plus être contesté, ni par les enfants concernés ni par leurs parents : s'ils parvenaient à les racheter, ces enfants étaient des affranchis, non des ingénus, avec toutes les limitations juridiques que cela entraînait. Il paraît plausible que cette décision ait été prise pour favoriser l'accueil de tous les enfants abandonnés sans exception. En effet jusque là les pères qui avaient abandonné, et non vendu, leurs enfants, pouvaient les récupérer à tout moment sans avoir à verser aucune contrepartie financière, puisqu'il était interdit d'asservir un citoyen né libre. Ils pouvaient donc être tentés de réclamer l'enfant qu'ils avaient exposé dès qu'il était capable de leur rapporter de l'argent, quitte à le revendre l'instant d'après : ce n'était pas impossible dans un monde esclavagiste. Cela pouvait suffire à dissuader les bonnes volontés et les spéculateurs de recueillir les enfants abandonnés ? 
\end{enumerate}
 
\item[315] Il décide (ou plutôt il rappelle ce qui était le cas jusque là) que la vente d'un ingénu est illégale dès qu'il ne s'agit plus d'un nouveau-né, et qu'elle ne peut effacer le statut initial d'un ingénu : l'intéressé peut donc en tout temps revendiquer sa liberté devant les tribunaux (à charge de fournir des preuves suffisantes).

% 315 
Il décide que les biens d'une mère défunte sont de droit la propriété de ses enfants. C'était d'ores et déjà l'usage établi, mais il fallait jusque là un testament maternel en bonne et due forme.

% 315 
Le mari devient le \emph{curateur} de sa femme, c'est lui et non plus le père de celle-ci qui gère ses affaires financières tant qu'elle n'a pas eu trois enfants et ne peut donc selon le Droit le faire elle-même. Était-ce un progrès ? Sûrement pour le mari, peut-être moins pour l'épouse qui ne pouvait plus se retourner vers son père ou son tuteur en cas de conflit avec son mari.

\item[316] Constantin revoit l'arsenal des peines prévues contre ceux qui enlèvent les enfants pour les vendre. Il leur promet la condamnation au travail forcé dans les mines%
%[1]
\footnote{Constantin a interdit la crucifixion, la condamnation aux jeux du cirque et celle des femmes aux lupanars, remplacés par la condamnation aux mines. En fait le travail des mines était très dur et particulièrement malsain (ex. emploi du feu en front d'exploitation pour désagréger la roche...) et on y mourait vite, ce n'était donc qu'un adoucissement relatif. L'interdit de Constantin avait l'avantage connexe de tarir l'une des sources des distractions offertes au public dans les jeux du cirque, condamnés depuis toujours par les chrétiens. Constantin interdisait aussi les mutilations du visage, ce qui ne l'empêchait pas de prévoir des peines terribles pour divers délits.}% 
. Cette décision n'était pas nouvelle, et ces rapts continueront aussi longtemps qu'il sera permis de vendre et d'acheter des esclaves.

\item[318] Il décide d'étendre la notion juridique de parricide à tous ceux, parents ou enfants, qui tuent un de leurs parents. Les parents infanticides seront désormais passibles de la peine de mort. 

 Dans le même mouvement \emph{il assimile l'avortement à un infanticide.}

 Par ailleurs à partir de Constantin, la puissance paternelle est retirée non seulement aux pères qui exposent leurs enfants, comme on l'a vu plus haut, mais aussi à ceux qui les prostituent et à ceux qui ont avec eux des relations sexuelles. Les pères déchus de leurs droits continuent de devoir assumer financièrement leurs enfants, mais ces derniers sont confiés à un tuteur.

\item[320] Il abroge les lois d'Auguste contre le célibat : dès l'âge de 25 ans, un homme ou une femme \emph{sans enfants}, célibataire ou non, \latin{sui juris}, peut recevoir tous les héritages venant de personnes extérieures à sa famille. Il n'est plus non plus question de sanctionner financièrement le célibat par un impôt spécial. Le refus du mariage ou du remariage n'est plus pénalisé.

\item[321] Constantin accorde à l'Église le droit de recevoir des legs et des successions, même par une simple déclaration orale. 

\item[324] Il dispense de tutelle les jeunes gens \latin{sui juris} (orphelins de père) qui ne sont pas infâmes%
%[2]
\footnote{... c'est-à-dire qu'il émancipe presque tous les jeunes citoyens : la plupart d'entre eux ne sont en effet pas infâmes, à l'exception des prostitué(e)s.}% 
, les filles dès 18 ans et les garçons dès 20 ans. Jusque là les garçons \latin{sui juris} avaient un tuteur jusqu'à leurs 25 ans, et seules les mères \latin{sui juris} de 3 enfants pouvaient être exemptées de tuteur à partir de leurs 25 ans. Les filles continuent de devoir obtenir l'accord de leurs parents (mère, oncles, grand-pères, frères (?)) pour se marier, \emph{mais elles reçoivent le droit de s'en passer pour entrer en religion}. Ce choix de vie est donc protégé contre les pressions des parents, alors que le choix du conjoint (et de la famille avec laquelle faire alliance) ne l'est pas : \emph{le mariage reste une alliance de deux familles}.

\item[325] Le concile de Nicée, approuvé par l'empereur, ordonne la création auprès de chaque évêque de « maisons de charité », pour les malades, les pauvres, les vieillards, les voyageurs, les pèlerins et les infirmes. Il confie la gestion de ces maisons à un « religieux du désert » c'est-à-dire à un moine, explicitement invité à considérer son travail quotidien au service des indigents et des malades comme une prière. En confirmant les décisions du concile Constantin donnait aux évêques la mission de réaliser en grand, à l'échelle de l'empire, ce qu'ils avaient expérimenté depuis le premier siècle et qu'ils affirmaient être au cœur de leur mission religieuse. Il faisait ainsi de l'assistance un service \emph{public} exercé par l'Église, ce qui donnait aux évêques le droit de réclamer aux autorités civiles des moyens à la mesure de leur mission de protecteurs des pauvres : dotations en argent, en domaines%
%[3] 
\footnote{... sur le statut juridique et fiscal desquels les discussions ne semblent pas terminées (cf. Jean \fsc{DURLIAT}, \emph{De l'Antiquité au Moyen Âge, l'Occident de 313 à 800}, 2002).} 
et en bâtiments%
%[4]
\footnote{En dépit du respect manifesté aux responsables ecclésiastiques, il s'agissait de moyens fournis par les autorités pour une mission et non de dons sans contrepartie, et il n'en sera jamais autrement. Les autorités civiles se sentiront toujours un droit de regard sur les moyens alloués aux évêques, de la même façon que les autorités des cités antiques n'ont jamais hésité à dépouiller les temples civiques de leurs trésors en cas de nécessité. Si Constantin a transféré une grande partie des biens des temples païens aux églises, c'est parce qu'il y voyait l'intérêt de son État. En dépit des protestations des clercs, les autorités civiles ne renonceront jamais longtemps à reprendre les moyens à eux confiés pour les affecter à d'autres fins lorsque cela leur paraitra aller dans le sens de l'intérêt général dont elles sont comptables.}% 
. Cela leur donnait un outil de conquête des esprits%
%[5]
\footnote{Lorsque Julien (empereur de 361 à 363) a tenté de restaurer les religions traditionnelles de l'empire, il a cherché à mettre en place un clergé païen hiérarchisé sous sa direction (il était \latin{pontifex maximus}) et il a voulu l'astreindre à fournir une véritable assistance à partir des temples, afin d'enlever aux chrétiens l'exclusivité d'un outil de séduction dont ils se servaient efficacement depuis leur apparition sur la scène religieuse antique.}% 
. 

\item[326] Le concubinage est interdit aux hommes \emph{mariés}. Cela n'aurait été qu'une décision symbolique sans grande portée s'il n'y avait eu à côté d'elle un ensemble de dispositions qui faisaient que désormais entretenir une concubine alors qu'on est marié présentait beaucoup moins d'intérêt et beaucoup plus d'inconvénients qu'auparavant : le principal de ces inconvénients était que les enfants nés des concubines des hommes mariés ne pouvaient plus être légitimés : ils étaient désormais considérés comme des enfants adultères. Ils ne pouvaient donc ni hériter de leur père ni lui succéder. Cela leur interdisait%
%[6] 
\footnote{… en théorie du moins, mais il y aura assez souvent des passe-droits au fil des siècles, avec des périodes très strictes et des périodes très tolérantes. Ceci dit cette règle va demeurer jusqu'au \siecle{20}.} 
de remplacer en cas de nécessité les héritiers légitimes, espérés en vain, ou décédés. 

% 326 
Les hommes qui n'ont pas d'épouse vivante ni d'enfants légitimes, mais qui ont une concubine ingénue (non esclave et non affranchie), de bonne réputation (non infâme, fidèle, non issue de la prostitution), et qui ont eu des enfants de cette femme, sont invités à l'épouser : s'ils le font les enfants qu'ils ont eu en commun avant le mariage seront reconnus comme légitimes. Dans tous les autres cas la légitimation des enfants illégitimes est interdite. Cette mesure a d'abord été prévue pour une période de transition d'une année, pour apurer le passé : dans un monde idéal il ne devait plus naître à partir de cette date aucun enfant illégitime. Face à la résistance des réalités elle sera renouvelée à plusieurs reprises jusqu'à ce qu'elle devienne permanente moins d'un siècle plus tard. C'est la \emph{légitimation par mariage subséquent}.

% 326 : 
Il n'était pas plus question pour Constantin, en dépit des pressions éventuelles de l'Église, que pour aucun de ses prédécesseurs de sanctionner les maris pour leurs propres infidélités, sauf quand ils avaient des relations avec la femme d'un autre. Par contre il abroge la loi d'Auguste contre l'adultère des femmes. Il décide qu'une femme adultère ne peut plus être dénoncée par son propre père (ce qui décharge ce dernier de l'obligation qui lui était faite de la dénoncer), ni par les étrangers (à qui les dénonciations rapportaient une part significative des fortunes confisquées aux deux coupables par le fisc impérial). Désormais seul le mari, et ses proches (cousin, beau-frère et frère), peuvent dénoncer les amants, mais ils n'y sont pas obligés, et en ce cas le mari règle l'affaire comme il règlerait n'importe quel autre conflit domestique. S'agissait-il pour Constantin de supprimer les dénonciations calomnieuses ? Ou de permettre au mari lésé de pardonner comme le demandait l'Église ? Par contre lorsque le mari choisissait de traîner sa femme en justice la peine maximale n'était plus comme auparavant l'infamie et l'exil, mais la mort des deux complices. En l'absence de données suffisantes on ne sait ni quel était le degré réel de répression des adultères avant Constantin, ni combien de coupables ont effectivement subi la peine qu'il avait prévue en cas de dénonciation par le mari. 

\item[331] La répudiation (rupture unilatérale du mariage, par opposition au divorce par consentement mutuel) devient un délit. L'épouse qui prend l'initiative de divorcer est condamnée à l'exil (assignée à résidence loin de chez elle), ce qui lui interdit le remariage, et elle perd une part substantielle de sa dot. Un homme qui répudie sa femme doit lui rendre l'intégralité de sa dot et ne peut plus se remarier non plus (mais il n'est pas exilé). La répudiation reste néanmoins autorisée, et l'époux(se) innocent(e) peut se remarier, dans les cas suivants :
\begin{enumerate}[leftmargin=*,itemsep=0pt]
% a)
\item si le mari est condamné à une peine infamante, ou s'il devient esclave, ou s'il est condamné comme homicide, violateur de sépulture ou empoisonneur ;
% b)
\item si la femme est convaincue d'être adultère, empoisonneuse ou entremetteuse.
 \end{enumerate}
Ces dispositions concernaient tous les citoyens, chrétiens ou non. Pendant ce temps-là les divorces par consentement mutuel restaient possibles, et dans ce cas les remariages l'étaient aussi. Les lois de l'Église ne concernaient pour le moment que les chrétiens, qui ne constituaient pas encore l'ensemble de la population. Moins d'un siècle plus tard un empereur ordonnera à tous les païens de se faire baptiser. Le remariage après divorce sera désormais \emph{en principe} interdit à tous les citoyens de l'Empire, sauf aux juifs%
%[7]
\footnote{Ceux-ci ne seront jamais interdits de remariage jusqu'à la Restauration (\siecle{19}).}% 
.

\item[334] Constantin interdit de séparer les membres d'une même famille pour les vendre comme esclaves, interdiction qui implique que de telles ventes se faisaient encore. Ce décret reprend à son compte des interdits déjà formulés au \siecle{3}. Son existence montre que les unions des esclaves, même si elles n'étaient pas reconnues comme de vrais mariages, étaient alors perçues de manière suffisamment positive pour créer des droits opposables aux maîtres qui les avaient autorisées. 

\item[336] L'Église avait toujours défendu la légitimité de tous les mariages librement voulus entre deux personnes non parentes, quel que soient leurs statuts légaux (esclaves, affranchis, citoyens, chevaliers, sénateurs...). Cela n'empêche pas Constantin de rappeler l'interdit de reconnaître les enfants nés des unions traditionnellement interdites par la loi : union d'un citoyen (tous les hommes libres depuis 212) avec un infâme, d'un sénateur avec une affranchie ou avec une esclave, d'un citoyen avec une esclave. Jusqu'à cette date les autorités pouvaient malgré tout les déclarer légitimes. Cette décision interdit en principe de le faire. 

\item[342] Les interdits de mariage traditionnels romains sont réaffirmés et le \emph{senatus-consulte} autorisant un oncle paternel à épouser sa nièce est abrogé.

 Si le Droit est l'expression des mœurs, alors il nous faut croire que Constantin a aligné le Droit romain sur les mœurs de son temps. Doit-on en déduire que celles-ci étaient déjà chrétiennes (ou christiano-stoïciennes) bien avant qu'il ne conquière le pouvoir ? Cela signifierait que les limites et interdits que Constantin a opposés à l'expression libre et spontanée des désirs sexuels et la canalisation de ces désirs sur le seul mariage monogame et fidèle faisaient déjà partie de l'idéal moral de son temps, païens et chrétiens confondus%
% [8]
\footnote{Aline \fsc{ROUSSELLE}, \emph{La contamination spirituelle, science, droit et religion dans l'Antiquité}, 1998.} 
 ? Nous avons aujourd'hui du mal à imaginer une telle situation tant nous paraît improbable un mouvement qui irait spontanément d'un niveau élevé de liberté sexuelle et matrimoniale vers un verrouillage du mariage et une limitation drastique des possibilités d'obtenir des enfants légitimes et des héritiers ... à moins que l'impression de grande liberté et d'aisance que suggère ce que l'on croit connaître de la vie des grecs et des romains de l'Antiquité ne corresponde pas à la réalité vécue, au moins par les dépendants (femmes, mineurs, esclaves)%
% [9]
\footnote{Cf. David \fsc{BROWN}, \emph{Le renoncement à la chair}, 2002} 
 ? 

 Mais une autre hypothèse est tout aussi vraisemblable. Les lois peuvent être une déclaration d'intention. Elles peuvent être chargées de désigner le bien, et le Droit peut être un instrument de normalisation des comportements et de remodelage des représentations. En ce sens les lois qu'ont édictées les empereurs chrétiens ont eu pour objectif d'orienter les comportements de leurs sujets dans le sens qui leur convenait, sans attendre qu'ils se soient tous convertis. 

 Au fil des deux siècles suivants les décisions fondatrices de Constantin ont été complétées par ses successeurs : 

\item[374] Valentinien décrète que les parents doivent subvenir aux besoins de tous leurs enfants, légitimes ou non. \emph{L'abandon est interdit}. Ce texte, le premier du genre, ne prévoit en fait aucune sanction. Il se contente d'affirmer un principe, de dénier aux pères le droit à l'abandon de leurs nouveaux-nés \emph{s'ils ont les moyens de l'élever}. Il semble n'avoir jamais été utilisé à l'encontre de parents incapables de nourrir leurs enfants, du moment que la vie de ceux-ci n'était pas mise en danger, et que leur découverte avait été facilitée (enfant placé en évidence, protégé des intempéries et surtout des animaux errants, exposé dans un lieu où passent beaucoup de personnes,~etc.). 

\item[384] $\!$ou \textbf{385}\,{} L'empereur Théodose condamne le mariage entre cousins germains. Vingt années plus tard, l'empereur d'Orient Arcadius lève cet interdit dans les territoires qui relèvent de son autorité. 

\item[390] \emph{L'empereur accorde aux veuves le droit d'exercer la tutelle de leurs propres enfants mineurs}. Par ce biais est pour la première fois reconnue à des femmes une pleine capacité à représenter autrui, même si les conditions de cette reconnaissance sont précises et limitées :
\begin{enumerate}[leftmargin=*,itemsep=0pt]
% a)
\item seuls sont concernés leurs propres enfants mineurs,
% b)
\item il leur faut avoir atteint cinquante ans, âge où elles ne pouvaient plus espérer d'autres grossesses, et
% c)
\item elles doivent promettre de ne pas se remarier : ce faisant elles retomberaient en effet dans la main, sous la coupe d'un homme à qui elles seraient obligées d'obéir, au risque de nuire aux enfants nés de leur union avec le conjoint décédé%
%[11]
\footnote{Il n'est pas sûr que cette mesure ait porté sur de très grands nombres de personnes, compte tenu du fait que les veuves âgées de plus de cinquante ans et ayant encore des enfants mineurs (moins de 25 ans) ne devaient pas être très nombreuses, puisque les femmes commençaient souvent d'avoir leurs enfants très tôt, bien avant leurs 20 ans. C'était néanmoins un pas important vers la reconnaissance du principe de l'égalité juridique des époux.}% 
.
\end{enumerate}

 Théodose II (empereur d'Orient de 408 à 450) ordonne qu'un procès-verbal de découverte soit rédigé pour chaque enfant trouvé. C'est la première fois que ces enfants reçoivent une reconnaissance administrative (et une forme minimale d'existence légale) avant même qu'une personne n'accepte de répondre d'eux et ne leur donne statut de citoyen en les déclarant comme libres. Cela signifie peut-être que désormais le souverain en prend possession même s'il les confie immédiatement à l'Église ? 

\item[438] Le Code Théodosien étend les interdits de mariage aux cousins germains. Cette règle ne sera pas acceptée et ne sera pas reprise par le Code de Justinien, mais elle préfigure la législation des siècles suivants. Le Code Théodosien interdit également les mariages entre beaux-frères et belles-sœurs.

\item[442] Le Concile de Vaison et celui d'Arles (\textbf{452})
%[12] 
décident%
\footnote{Comme il est de règle jusqu'à l'an mil, ces deux conciles ont été convoqués par les autorités civiles, et leurs décisions ont été promulguées par ces mêmes autorités.} 
(ou rappellent%
%[13]
\footnote{Probablement une fois de plus s'agissait-il avec ces décisions de généraliser des mesures déjà largement expérimentées.} 
 ?) que l'enfant exposé sera porté à l'église sur le territoire de laquelle il a été trouvé et qu'il y sera enregistré. \emph{Le dimanche suivant, le prêtre annoncera aux fidèles qu'un nouveau-né a été trouvé, et dix jours seront accordés aux parents pour reconnaître et réclamer leur enfant}. S'ils ne se manifestaient pas dans ce délai l'enfant devait être remis à titre onéreux, et non pas donné, à celui qui se proposait de le prendre en charge. À défaut d'un laïc volontaire pour le prendre (pour l'acheter ?) l'enfant pouvait (devait ?) être mis en nourrice aux frais de la communauté ecclésiale. 

 Le code de Justinien est une compilation du Droit romain réalisée au \siecle{6} sous la direction de cet empereur de Constantinople. Depuis Constantin, l'ancien droit de vie et de mort paternel \latin{(jus vitae necisque)} n'existait plus. Tout père qui tuait volontairement son enfant, même à sa naissance, même non encore né (avortement provoqué) était passible de la peine de mort. Dans le code de Justinien le père conserve le droit de correction paternelle, mais il n'a plus le droit d'infliger de graves châtiments corporels, de blesser ni d'estropier. S'il juge nécessaire de recourir à des châtiments sévères il doit s'adresser au gouverneur de la province ou au préfet de la ville%
% [14]
\footnote{Jean \fsc{IMBERT}, \emph{Le droit antique}, Que sais-je, 1961, p.93.} 
 : c'est la doctrine juridique qui va perdurer jusqu'au \siecle{20}. 

\item[529] Justinien décrète que deux amant adultères n'auront jamais le droit de s'épouser même si venait à décéder l'époux (ou les époux) qui leur faisait obstacle ;

% 529 :
% Il décrète 
Et qu'une femme convaincue d'adultère sera condamnée à vivre dans un couvent de femmes%
% [15] 
\footnote{Jusqu'à la Révolution, et en fait jusqu'au \siecle{20}, les incarcérations de femmes se feront dans des monastères pour femmes ou dans des lieux inspirés de ce modèle.} 
jusqu'à sa mort ... si du moins son époux porte plainte, mais rien n'oblige ce dernier à le faire. Compte tenu du fait que les textes antérieurs (ceux de Constantin) donnaient à l'époux droit de vie et de mort, c'est un adoucissement majeur. Il pourra se séparer de sa femme adultère, mais ne pourra pas se remarier. Cette règle de droit ne concerne que les chrétiens mais à cette date cela fait longtemps que tous les citoyens, à l'exception des juifs, ont reçu l'ordre de se faire chrétiens. Par conséquent s'il veut avoir des enfants légitimes, des héritiers, un époux bafoué n'a plus d'autre choix (en principe) que de se réconcilier avec sa femme et de reprendre la vie commune, ce qui eut été le comble de l'indécence ou de l'infamie deux siècles plus tôt. En dépit de l'infidélité passée de celle-ci il est même expressément invité à lui pardonner au bout d'un certain temps de réclusion dans un couvent pour femmes (2 ans au maximum ?). Autant dire que son intérêt n'est pas forcément de porter le cas de son épouse coupable devant les tribunaux. L'adultère féminin devient de plus en plus une affaire privée, même si le pouvoir civil continue et continuera jusqu'à la fin de l'ancien régime de prêter la main au mari pour soutenir son droit de correction marital. 

\item[533] Les \latin{institutes} de Justinien réaffirment la légitimité du mariage entre cousins, ou celui d'un veuf ou d'une veuve avec le frère ou la sœur de son conjoint décédé. Cela montre la résistance des autorités civiles à l'autorité morale de l'Église. Sur ce point précis l'Orient refusait la position intransigeante de l'église de Rome face à tout ce qui ressemble à l'inceste%
% [10]
\footnote{Cf. Jack \fsc{GOODY}, \emph{L'évolution de la famille et du mariage en Europe}, p. 66.}%
.

%\item[534] Le Code de Justinien décrète que les enfants adultérins et incestueux n'ont aucun droit, au contraire des autres enfants illégitimes : il leur dénie tout droit à des aliments, bien qu'il ne soit pas interdit à leurs
\item[534] Le Code de Justinien dénie tout droit aux enfants adultérins et incestueux, au contraire des autres enfants illégitimes. Ils n'ont pas droit à des aliments, bien qu'il ne soit pas interdit à leurs
géniteurs de leur en donner. Ils ne peuvent en aucun cas être légitimés. Au nom de la sauvegarde de l'institution familiale, cette décision retire à ces enfants les protections que leurs géniteurs pourraient vouloir leur donner. Ils sont réputés \emph{enfants trouvés}, et traités comme ces derniers par les institutions chargées de s'en occuper. 

 Par ailleurs le code de Justinien confirme \emph{l'interdiction de l'adrogation des enfants illégitimes}. Les « enfants du péché » (le péché des adultes contre l'institution familiale) sont désormais à écarter. Cela contraste avec la volonté de protection de tous les enfants qui se traduisait dans les autres décisions de l'époque, mais il est clair que l'objectif de ces lois était d'abord de prévenir la naissance de ces enfants. La suite de l'histoire montrera que cet objectif a été à peu près atteint en dépit d'un « reste » incompressible (seulement ... ou encore, un pour cent de naissances illégitimes dans diverses régions rurales françaises sous Louis~XIV). 

 Le code de Justinien confie les enfants trouvés à l'évêque du lieu de leur découverte. Il ordonne aux magistrats de s'en tenir au principe juridique que les enfants trouvés sont tous nés libres et non esclaves, même lorsque leur mère est une esclave : l'abandon donne donc aux enfants d'esclave une chance de vivre libres. 

 En l'absence d'enfants légitimes le code de Justinien autorise les enfants naturels nés d'un concubinage stable à hériter de leurs parents : il confirme ainsi le concubinage stable dans son statut de mariage à l'usage des pauvres et des humbles. 
\end{description}

