%K1 démantèlement de la famille traditionnelle
%K2 Victoire du mariage d'inclination
%K3 "Le corps des femmes est à elles"
%K4 Incestes et paradoxes
%K5 Perplexités éducatives
%K6 Désarrois masculins
%K7 Inertie des pratiques
%L1 Un enfant pour quoi ? pour qui ?
%L2 Qui pour accueillir l'enfant ?
%L3 Droit à l'enfant ?
%L4 Progrès ou régressions ? 

\part{Depuis 1960, révolution dans la reproduction}



\chapter[Démantèlement de la famille traditionnelle]{Démantèlement\\de la famille traditionnelle}


 En ce qui concerne le sexe, la conjugalité et la procréation, les évolutions du droit et des mœurs depuis 1960 ont été fulgurantes comparées aux rythmes antérieurs de l'histoire.


\section{Lois principales}

\begin{description}

\item[1961] Une mesure administrative qui sur le moment n'a pas frappé beaucoup d'esprits, mais dont l'importance symbolique n'en est pas moins significative (les grandes fractures commencent souvent par une fissure imperceptible à l'œil nu) : le ministère de l'éducation nationale supprime le caractère obligatoire de l'enseignement du droit romain dans le programme de la licence de droit, obligation qui datait de la création des études de droit dans les universités aux \crmieme{11} et \siecle{12}s. Ce corpus n'est plus qu'une option facultative parmi d'autres. 

\item[1965] La loi du 13 juillet lève les derniers obstacles à l'exercice d'une activité commerciale par les femmes mariées sans la tutelle de leurs maris. Ceux-ci ne gèrent plus de droit les biens propres de leurs épouses (dot, etc.). Elles n'ont plus à obtenir leur autorisation pour exercer une profession séparée, quelle qu'elle soit.

\item[1966] La loi du 11 juillet sur l'adoption assimile les enfants adoptés aux enfants légitimes non adoptés (adoption plénière). 

%\item[1966] la (même)
Cette
loi du 11 juillet ouvre le droit à l'adoption plénière à une personne seule, qu'elle soit célibataire ou non et quelles que soient ses préférences sexuelles, d'au moins 28 ans.

\item[1967] La loi \fsc{NEUWIRTH} dépénalise la prévention des naissances : elle autorise la publicité concernant les méthodes anticonceptionnelles (interdite depuis les années 20), et elle autorise leur mise à disposition du public :
%la première visée et la principale était la « pilule » anticonceptionnelle 
la « pilule » anticonceptionnelle était principalement visée,
qui venait d'être mise au point. L'accord du mari n'est pas nécessaire, son refus n'a pas d'effet.

\item[1972] La loi du 3 janvier fait entrer les enfants naturels dans la famille du ou des parents qui les ont reconnus. À quelques restrictions près -- enfants adultérins, elle leur ouvre un droit à l'héritage égal à celui des enfants légitimes.

%\item[1972] 
La puissance paternelle est abolie au profit de l'autorité parentale. En cas de séparation, cette autorité est conservée à égalité par chacun des deux parents. 

%\item[1972] 
Une loi ordonne l'égalité des salaires féminins avec les salaires masculins.

\item[1974] L'âge de la majorité légale est abaissé de 21 à 18 ans. 

\item[1975] Loi du 30 juin relative aux institutions sociales et médicosociale : les usagers et les familles doivent être associés au fonctionnement de l'établissement qui les prend en charge (il doit les « prendre en compte » ).  Le pa(ma)ternalisme des institutions d'aide sociale est un peu affaibli.

%\item[1975] 
À côté du divorce pour faute, la loi du 11 juillet ouvre la possibilité de divorcer par consentement mutuel ou pour rupture de la vie commune. Par ailleurs, cette loi met les deux époux à égalité en matière de choix résidentiel et en matière de contribution aux charges du mariage.

%\item[1975] 
La loi \fsc{WEILL} dépénalise l'avortement (\emph{interruption volontaire de grossesse} ou IVG). La loi ne demande pas l'avis des maris éventuels.

%\item[1975] 
Les épouses ne sont plus tenues de faire usage du nom de leur mari dans la vie quotidienne et les relations avec l'administration.

%\item[1975] 
L'adultère féminin est dépénalisé. Ce n'est plus un délit qui concerne la société, ce n'est qu'un affront privé qui ne concerne que le mari.

\item[1976] Loi du 22 décembre relative aux conditions d'adoption : la présence d'enfants légitimes n'est plus un obstacle à l'adoption, même si leur avis est entendu. 

\item[1978] La loi du 6 janvier donne à tout individu majeur le droit de connaître le contenu de tout dossier administratif le concernant. Cela concerne notamment tous les enfants abandonnés.

\item[1983] Un arrêt de la cour de cassation du 21 mars 1983 admet la légalité de la garde conjointe de l'enfant après divorce.

%\item[1983] 
Loi sur l'égalité professionnelle entre femmes et hommes.

\item[1984] La loi du 6 juin relative aux \emph{droits des familles dans leurs rapports avec les services chargés de la protection de la famille et de l'enfance}, et au statut des pupilles de l'État, donne aux parents des droits plus étendus face à l'administration. L'autorité des parents sur leurs enfants placés à l'ASE est confortée dans tous les domaines (sauf limites définies expressément par un juge).

\item[1985] La loi du 23 décembre 1985 met les deux parents à égalité dans la gestion des biens de l'enfant : ils exercent cette tâche conjointement quand ils exercent en commun l'autorité parentale. Sinon l'un des deux l'exerce sous le contrôle du juge.

\item[1987] L'autorité parentale est redéfinie par la loi du 22 juillet (loi \fsc{MALHURET}) en termes de \emph{responsabilité parentale ordonnée à l'intérêt de l'enfant}. Elle est à égalité assumée par chacun des deux parents, qu'ils cohabitent ou pas.

\item[1989] \emph{Convention Internationale des Droits de l'Enfant} promulguée dans le cadre de l'ONU le 20 novembre : reconnaît le droit de tout mineur à une famille, et ses droits face à sa propre famille.

%\item[1989] 
Loi du 10 juillet \emph{relative à la prévention des mauvais traitements à l'égard des mineurs et à la protection de l'enfance}. Elle prévoit que le délai de prescription ne court qu'à partir de la majorité pour les mineurs victimes de violences.

\item[1993] L'autorité parentale conjointe devient la règle pour les couples de concubins comme pour les couples mariés.

\item[1996] Convention européenne du 25 janvier sur l'exercice des droits de l'enfant. Elle donne le droit aux enfants mineurs de donner leur avis sur les mesures qui les concernent lors du divorce de leurs parents.

\item[1999] Création du PACS : pacte civil de solidarité, ouvert aux couples hétérosexuels et aux couples homosexuels.

\item[2000] La pilule {\emph{du lendemain}} est en vente libre dans les pharmacies, et distribuée gratuitement aux mineures par les infirmières scolaires sur simple demande de la mineure, sans demander l'avis de ses parents, et sans qu'ils en soient informés.

\item[2002] Sur décision de la Cour Européenne de Justice les dernières discriminations juridiques que subissaient en matière d'héritage les enfants adultérins et incestueux sont effacées. Seuls sont distingués les enfants nés des incestes parent--enfant, qui ne peuvent être reconnus que par un seul de leurs deux parents. Ils doivent néanmoins être traités absolument en tout le reste, et d'abord en ce qui concerne l'héritage, comme leurs éventuels demi-frères ou sœurs. 

%\item[2002] 
Loi du 4 mars : {[...] \emph{les parents associent l'enfant aux décisions qui le concernent, selon son âge et son degré de maturité}}. L'administration de la famille par les deux parents doit être démocratique.

\item[2005] La loi autorise les femmes à donner à leurs enfants leur propre nom à égalité avec leur mari à compter de janvier 2005. 

%\item[2005] 
Interdiction du mariage des filles avant dix-huit ans (traditionnel âge au mariage des garçons).

\item[2013] Loi \fsc{Taubira} : ouverture du mariage aux couples de même sexe.

\item[2013] Remboursement à 100~\% de l'IVG.

\item[2014] Suppression de l'exigence d'une « détresse » pour reconnaître à une femme enceinte son droit à un avortement. 
\end{description}
 
 
\section{Le sens des évolutions}

 Jusqu'à notre présent, la prééminence du masculin allait de soi. Le mâle \emph{(vir)} était le modèle accompli du genre humain \emph{(homo)}, l'homme véritable. La femme était son exception et n'était en principe que cela. Pleine de charme et de mystère et belle à troubler les plus chastes, elle n'en était pas moins caractérisée par le manque. Toutes les sociétés faisaient d'elle un être comme de second rang (le {\emph{deuxième sexe}}), presque toujours limité dans ses droits et dans son autonomie, un peu à la manière d'une enfant, et toujours exclue des postes de pouvoir. Les hommes remerciaient leurs dieux de n'être pas nés femmes. Celles-ci se montraient autrement comblées par la naissance d'un garçon que par celle d'une fille, et exerçaient une pression redoutablement efficace sur leurs filles pour qu'elles ne s'écartent pas du rôle que société et familles attendaient d'elles.

 Jusqu'au milieu du \siecle{20}, le statut légal des femmes françaises était plus proche de celui des femmes de l'antiquité tardive que de celui de leurs petites-filles de l'an 2010. Jusqu'aux années soixante du \siecle{20}, tout mari était le chef de sa famille et avait à ce titre autorité, sur ses enfants mineurs certes, mais aussi sur sa femme, puisque celle-ci avait abdiqué une bonne part de sa capacité juridique en se mariant. C'est lui qui détenait ces droits, et même s'il l'autorisait à les exercer, c'était sous sa propre responsabilité. Il donnait son nom à leurs enfants communs? Sauf contrat de mariage particulier c'est lui qui gérait tous leurs biens. Il signait seul la déclaration de revenus, et pouvait laisser son épouse dans l'ignorance sur le montant des ressources du couple. Il était censé être le principal pourvoyeur financier même quand par son travail ou par sa dot son épouse contribuait autant ou plus que lui aux dépenses du ménage. Au nom de l'unité de commandement nécessaire à toute entreprise il pouvait lui interdire d'exercer un emploi salarié, d'ouvrir un commerce, de prendre une gérance en son nom propre, et même de posséder un compte en banque personnel. Lorsqu'elle travaillait avec lui elle était censée lui être subordonnée, et son travail était rarement reconnu et individualisé. 

 Lorsque les deux époux s'entendaient bien, la réalité de leurs rapports et l'influence de chacun sur les décisions communes pouvaient être fort différentes du modèle que la loi définissait, et il en a toujours été ainsi, mais en cas de conflit c'est à ce modèle que se référaient les juges.

 Si l'on en croit les observateurs du \siecle{19}, de nombreux prolétaires chefs de famille dilapidaient les gains des uns et des autres, même si dans la plupart des couples populaires c'étaient en réalité les épouses qui tenaient les cordons de la bourse. En tout cas c'est en principe au nom de ces abus, et non pour favoriser la parité homme--femme, que les femmes mariées ont reçu dès le début du \siecle{20} le droit de percevoir et de gérer elles-mêmes leurs gains et leurs économies personnelles.

 Certes depuis Constantin (au plus tard), les femmes ne sont plus considérées comme incapables ou déficientes par nature. Quand leur « seigneur et maître » venait à mourir les veuves  retrouvaient le plein exercice de leurs droits personnels, ceux qu'elles lui avaient confiés en se donnant à lui par le mariage. Une veuve non remariée remplaçait donc de plein droit son époux dans tous les actes juridiques ou commerciaux, comme dans l'exercice de l'autorité éducative sur leurs enfants communs. Mais jusqu'au \siecle{20} le masculin avait la préséance alors qu'aujourd'hui c'est l'idée même d'une préséance du masculin qui n'a plus de sens. Elle n'est plus défendable dans les pays développés, quelles que soient les difficultés que l'on peut rencontrer dans la recherche de l'égalité absolue des deux sexes.

 Les brimades, interdits et humiliations subis par les « mal nés » du passé avaient pour but ultime de prévenir la conception d'enfants en dehors du cadre du mariage légitime, en dehors de l'alliance de la mère avec un homme nommément désigné. Pour défendre l'institution familiale et la société comprise comme un réseau de familles alliées (un filet de relations noué par les mariages) il fallait décourager les femmes de faire naître des enfants sans pères, et interdire aux hommes de se procurer des héritiers en dehors d'une alliance avec une autre famille. Aujourd'hui tout se passe comme s'il n'existait plus que des enfants légitimes : tout enfant a vocation à faire partie de la famille de chacun de ses deux géniteurs quelles que soient les circonstances de sa conception. Tout enfant a vocation à hériter de ses deux parents à égalité avec ses éventuels demi-frères et demi-sœurs. Tout enfant est un « enfant de famille ». On peut aussi bien dire que tous les enfants sont devenus « naturels » et que la notion même de légitimité s'est évaporée, réduite à un mot sans épaisseur puisqu'il n'a plus de prise sur rien, puisque les effets concrets de la légitimité sont les mêmes que ceux de l'illégitimité, et inversement.
 
 {\emph{Cette distinction légitimité--illégitimité était totalement structurante de la société. Aujourd'hui il n'est pas un pays qui n'ait soit complètement aboli cette distinction, soit s'apprête à l'abolir. C'est un changement majeur des rapports entre famille et société qui montre que nous sommes face à des changements de la structure sociale elle-même}.%
% [2]
\footnote{Irène \fsc{THERY}, « peut-on parler d'une crise de la famille ? Un point de vue sociologique » , \emph{Neuropsychiatrie de l'enfance et de l'adolescence}, 2001, 49, 492-501, p. 403.}% 
} La cheville qui depuis \nombre{1600} ans tenait ensemble tout le système de la famille constantinienne a été retirée, et cela se passe apparemment à la satisfaction de tous. Puisque ni les parents ni les enfants ne risquent plus aucun désagrément du fait d'une naissance illégitime, à quoi bon le mariage, surtout quand on est convaincu que le seul couple légitime c'est celui qui repose sur l'accord quotidien de deux volontés. Depuis plus d'une génération, le nombre d'enfants nés hors mariage a donc progressé en même temps que croissait leur assimilation aux enfants nés dans le mariage. Aujourd'hui ils représentent la moitié des naissances. 

 Depuis la fin du Moyen-Âge, la coutume en France était de nommer les enfants du nom de leur père. Dans ce contexte porter le nom de sa mère signifiait que l'on était né hors mariage et qu'on n'avait pas de père légitime. Dans la France d'aujourd'hui, l'illégitimité a cessé d'être honteuse et il n'est plus socialement utile que le nom que l'on porte atteste qu'on a été reconnu par un homme. Une loi de 2001 autorise (donc ?) les couples mariés à donner à leurs enfants le patronyme de la mère à la place de celui du père, ou bien à côté de lui (texte complété par la loi \fsc{Taubira} de 2013). Ce changement est significatif, puisque la pratique traditionnelle n'était pas celle-là (contrairement à l'Espagne, par exemple) tout comme est significatif le moment où il a été institué. 


\section{Séparation des familles et de l'État}

 Sauf crime caractérisé les membres dépendants d'une famille antique (enfants, serviteurs libres ou esclaves) ne pouvaient faire appel d'aucune des décisions des parents devant un magistrat extérieur à la famille. Chez les grecs et les romains l'unité de commandement était considérée comme au moins aussi nécessaire à la famille qu'à toute autre institution : c'est donc le mari qui tranchait en cas de conflit entre les deux époux, comme dans toutes les sociétés patriarcales. Mais selon Aristote c'est comme un couple royal que les deux époux régnaient sur leur maison, et qu'ils exerçaient sur leurs enfants et leurs esclaves ce qu'il appelle la \emph{justice domestique}. 
 
 Cette conception royale du couple parental a trouvé son achèvement lorsqu'à la fin du siècle de Constantin, en 390, les veuves non remariées \emph{(sui juris)} se sont vu reconnaître le droit d'exercer elles-mêmes la tutelle de leurs enfants mineurs, en lieu et place de leur mari défunt. Quand Thomas d'Aquin a réintroduit Aristote dans les universités du Moyen-Âge, il a repris cette doctrine du pouvoir royal des parents sur leur maison. 

 Du haut Moyen-Âge aux Lumières, la continuité entre le pouvoir du roi et celui des pères allait sans dire et n'avait pas à être démontrée. Au début du \siecle{19} le Code Civil reprenait encore à son compte cette conception monarchique du rôle parental. Il n'imaginait pas un instant un fonctionnement familial démocratique associant les enfants aux décisions qui les concernaient. Jusqu'à leur majorité leurs parents avaient pleine autorité sur eux, parlaient pour eux, et si nécessaire le père, \emph{chef de la famille}, tranchait en dernier ressort.
 
 
Selon Jacques Donzelot (in \emph{La police des familles}1977)} nous sommes passés du gouvernement "des" familles au gouvernement "par les" familles. Sous l'Ancien Régime chaque père était responsable des membres de sa famille et avait autorité sur eux pour éviter tout désordre, avec l'appui des institutions : ce qui se passait au sein de la famille ne regardait que lui. L'ensemble des chefs de famille tenait la société en leurs main. Aujourd'hui le pouvoir royal des pères sur leurs enfants est mort, et celui des mères en meme temps.  {\emph{Ce qui caractérise la loi de 1970 (qui substitue l'autorité parentale à la puissance paternelle) ce sont trois concepts au centre de la réforme, celui « d'égalité » des époux et parents, celui « d'intérêt de l'enfant » et enfin celui de « contrôle judiciaire » devenu nécessaire pour arbitrer d'éventuels conflits entre les parents, entre parents et enfants. On assiste à un recentrage des positions de chacun des membres de la famille. Au centre l'enfant, en face de lui, responsables de lui, ses parents. Entre les deux des médiateurs, les spécialistes judiciaires}%
\footnote{Françoise \fsc{HURSTEL}, \emph{La déchirure paternelle}, p. 117.}%. 
}. Nous avons assisté à la délégitimation de la justice domestique, du droit des deux parents à régler eux-mêmes sans tiers extérieur tous les conflits intra familiaux. Lorsqu'ils ne réussissent pas à se mettre d'accord entre eux ou avec leurs enfants, ils sont désormais contraints (par leur égalité elle-même) à recourir à un tiers extérieur pour arbitrer leur différend.  

De nouveaux personnages se sont imposés au sein des familles. Sous l'autorité des juges les travailleurs sociaux et les experts (psychologues, psychiatres, médiateurs, etc.) sont entrés dans le champ, jusque là bien clos, des familles ordinaires, des familles non stigmatisées au préalable comme défaillantes (en ce qui concerne les familles reconnues officiellement comme incompétentes ou délinquantes, c'est depuis toujours que les représentants de la société y avaient leurs entrées). Ils font régner la bonne parole et les bonnes pratiques et vérifient que les familles adoptent les bonnes pratiques dans la prise en charge de leurs enfants, pratiques définies par les mêmes personnages. Quand ils l'estiment nécessaire ils ont l'appui des autorités pour faire passer le message.

Quelqu'un a dit que nous assistions à la \frquote{séparation de la famille et de l'État%
\footnote{J'ai malheureusement égaré le nom du l'auteur de cette formule choc, qui pose si bien le problème. Qu'il ou elle me le pardonne.}%
} ? 

 

 Assiste-t-on à la dissolution de la famille en tant qu'institution juridique ? 

 Assiste-t-on à la disparition de la sphère privée, cette sphère de la vie de chacun qui se définit par le fait que tant qu'il n'enfreint aucune loi, il n'a aucun compte à rendre sur ce qui s'y passe, et surtout pas à l'État et à ses représentants ? 

 Tout ce qui concerne les enfants est-il entré dans le domaine public, alignant le traitement de l'ensemble des familles sur celui qui était autrefois réservé aux seuls « cas sociaux », et mettant en cause l'aptitude de tous les parents à défendre suffisamment bien l'intérêt de leurs propres enfants ?







 \chapter[Et après ?]{Et après ?}


\section{"Il est interdit d'interdire"} 

  Depuis "Mai 68" nous ne tolérons plus aucune contrainte dont l'inéluctabilité n’ait été prouvée de manière indiscutable. Nous sommes prêts à tout essayer et à repousser toutes les limites. Dans le domaine de la génération, tout est aujourd'hui remis en question, jusqu'à la nécessité de la rencontre de deux cellules germinales mâle et femelle pour concevoir un enfant. Les historiens, les ethnologues et les sociologues nous fournissent tous les exemples de pratiques alternatives dont nous pourrions avoir besoin pour « déconstruire » et délégitimer les représentations traditionnelles et les idées que nous avons reçues du passé. Dans un article écrit en 1999 (\emph{"Enfin au-delà du Pacs"}) Jeanne Favret-Saada observe que :
  \begin{displayquote}
  "\emph{L'anthropologie  de  la  parenté  propose  le  répertoire
des arrangements culturels imaginés et pratiqués dans
les sociétés qui ont eu la chance d'être visitées par un
ethnologue. L'existence  même  de  cet  herbier  montre
l'immense  plasticité  des  sociétés  humaines  en  matière
de filiation, d'alliance et d'organisation familiale. Bien sûr,
le mariage homosexuel et l'homoparentalité au sens où
nous en parlons aujourd'hui, en France, n'ont été expérimentés  dans  aucune  des  populations  exotiques  que
nous  citons  quand  on  nous  oppose  l'impossibilité
anthropologique de nos revendications, et qu'on nous
accuse  de  vouloir  ruiner  les  fondements  de  la  culture
occidentale. Ni les Azandé, ni les berdaches, ni les çi, et
moins encore les ça n'ont eu notre projet politique.
La  seule  conclusion  qu'autorise  l'examen  de  l'herbier,
c'est donc que l'homoparentalité et le mariage homo-
sexuel  ne  sont  pas  une  impossibilité  anthropologique.}

\emph{...L'anthropologie  considère comme possible tout ce qui s'est déjà fait quelque part - du moins tout ce qu'un ethnologue dit avoir observé. Or  plusieurs  formes  d'homoparentalité  et  quelques
formes de mariage homosexuel existent déjà dans des
sociétés  comme  les  nôtres, avec  ou  sans  statut  légal,
selon  le  cas  : des  ethnologues  parfois (...), des
reporters  aussi  l'affirment. Dès  lors, le  problème  est
donc  moins  celui  de  la  possibilité  de  ces  nouvelles
formes de parenté que celui de leur passage au droit.}

\emph{...On voit donc qu'il est déjà nécessaire d'agrandir l'herbier  anthropologique, puisque  des  formes  inédites  de
parenté  sont  déjà  pratiquées  dans  nos  sociétés ; mais,
plus  encore, de questionner  les  concepts  utilisés  pour
fabriquer cet herbier.}" 
 \end{displayquote}
 




Les lois qui autorisent le divorce par consentement mutuel, la contraception et l'avortement ont dénaturalisé le modèle de famille traditionnel. Même si rien n'empêche personne de croire à la valeur religieuse, éthique, éducative ou civique de la famille fondée sur un couple hétérosexuel, monogame et indissoluble, élevant lui-même les enfants nés de ses œuvres, ce modèle n'est plus qu'un parmi les autres, comme c'était le cas avant les décisions de Constantin. Dans la mesure où ce modèle n'est plus étayé par la loi et par la puissance publique, comme il l'était sous l'Ancien Régime ou du temps du Code Napoléon, la famille traditionnelle, ce que j'ai appelé la famille constantinienne, n'existe plus. Ce qu'on nomme aujourd'hui de ce nom est une création des volontés individuelles des protagonistes au même titre que les autres façons "post constantiniennes" d'organiser la reproduction humaine et la vie en famille (ou pas), une parmi d'autres, dont la paternité, la maternité ou l'adoption célibataires, le concubinage, le PACS ou le mariage homosexuel, etc.

On n'a jamais tant parlé d'inceste que depuis que les enfants des couples incestueux ne sont plus pénalisés par le droit. Sans jamais le désigner par son nom le Code Napoléon avait refusé de criminaliser l'inceste entre \emph{majeurs} consentants, mais il n'en refusait pas moins toute reconnaissance aux enfants nés de relations incestueuses. Au contraire ces enfants ne sont plus discriminés depuis une génération. S'ils ne peuvent pour l'instant être reconnus que par un seul de leurs géniteurs, ils peuvent du moins l'être par l'un d'eux, et surtout ils peuvent hériter \emph{simultanément}
 de \emph{chacun} des deux, ce qui a toujours été la forme la plus substantielle de la reconnaissance et qui le reste. Même si c'est de façon indirecte que cela leur est reconnu ils ont donc bien  \emph{deux parents légaux}, et ne sont plus hors famille. 
 
 
 
 Cela veut-il dire qu'il n'y a plus aucun enfant mal né ? 
 
 Cela veut-il dire qu'il n'y a plus de partenaires sexuels avec qui concevoir un enfant est interdit ? 
 
 
 
 \begin{displayquote} 
\emph{[...] il convient de se poser la question : notre société est-elle toujours fondée sur le principe de prohibition de l'inceste ? On ne s'est jamais autant soucié de l'inceste, de lever le voile sur la réalisation de l'inceste, de punir l'inceste. Cependant la prohibition de l'inceste est devenue floue dès lors que l'interdit sexuel s'est délié de la question matrimoniale... N'est-ce pas aujourd'hui une autre catégorie, celle du viol, qui devient le cadre à l'intérieur duquel vient s'inscrire l'inceste ? L'inceste n'est-il pas considéré comme un viol sur mineur ?}
 (Irène \fsc{THERY}, idem p. 499-501)
 \end{displayquote} 
 
 \section{Suffit-il d'aimer ?} 

Le vingtième siècle a cru\footnote{La valorisation du sentiment amoureux a pu s'appuyer sur les sources religieuses les plus anciennes et les plus vénérées de notre aire culturelle. En effet lorsque YVWH dit \emph{« Il n'est pas bon que l'homme soit seul. Il faut que je lui fasse une aide qui lui soit assortie."} (Gn 2, 18) c'est amoureusement qu'Adam reçoit et investit celle qui lui est donnée : \emph{"à ce coup c'est [Eve] l'os de mes os et la chair de ma chair !"} (Gn, 2, 23). Dans ce modèle (à la fois égalitaire et inégalitaire) la relation de l’homme et de la femme est si intime qu'elle les rend au moins aussi proches que le sont les parents les plus proches : \emph{"C'est pourquoi l'homme quitte son père et sa mère et s'attache à sa femme, et ils deviennent une seule chair."} (Gn, 2, 24). Mais ce texte est lu et commenté depuis 25 siècles, et ce n'est que depuis la fin du XXème siècle que le sentiment amoureux est placé au-dessus de toutes les autres considérations.} de toutes ses forces à l’amour de l'homme et de la femme et il l'a sacralisé, posant en règle qu'il sanctifie tout ce qui est fait en son nom.
Mais de quelle nature est cet amour que l'on porte aux nues aujourd'hui ? En effet ce mot possède plusieurs acceptions. 

On a vu qu'autrefois nul ne pensait que le désir amoureux (ou passionnel, ou sexuel : \emph{Eros}) soit suffisant pour former un couple conjugal, pour fonder une famille, même pas ceux qui plaidaient pour que les jeunes aient la liberté de choisir leurs conjoints. On disait que la passion est aveugle, qu'elle est le lieu de toutes les illusions, qu'elle doit plus aux représentations du passionné qu’à la réalité de l’objet de sa passion. De l'Antiquité à la Belle Époque les moralistes se sont méfiés du désir sexuel, y voyant une entrave à l'exercice de la raison, une force aveugle, inconstante, décevante et potentiellement destructrice de tous les liens (dont les liens conjugaux) et de tous les principes sur lesquels repose la société. C'est pourquoi ils insistaient sur la maîtrise des passions : contrôle des pulsions, valorisation du jugement et de la raison, valorisation du sentiment du devoir, entraînement à résister à la frustration, etc. Pour eux un amour ne méritait d'être qualifié de conjugal que lorsque s'étaient éteintes les (éventuelles) flambées du désir charnel. 

Lorsque Paul de Tarse évoque l'amour conjugal il ne s'agit pas pour lui d'un sentiment ou d'une passion, mais d'une tâche à accomplir (\emph{agapè}). Il s'agit de prendre soin de son conjoint comme de son propre corps et de chercher activement à lui plaire. Il s'agit de se vouer à son bien-être physique et mental, de pardonner ses défaillances et ses limites, et enfin « last but not the least » de lui reconnaître un droit exclusif sur son propre corps. 

Erasme qui n'était ni pudibond ni ascète et qui tenait le mariage pour un choix de vie aussi méritoire que le célibat des moines et des clercs, ce qui était en son temps le signe d'une grande ouverture d'esprit, n'en écrivait pas moins dans son \emph{En-commium matrimonii christani} (Eloge du mariage chrétien, 1526) : 

\begin{displayquote}
\emph{"... Les poètes appellent l'ardeur des amants une fureur et non point un amour. Car enfin, où la raison est éteinte, peut-il y avoir autre chose que de la folie ? ...ceux qui épousent des femmes imprudemment et sans jugement ont coutume de se repentir, mais quelquefois trop tard, de ces malheureux mariages. Il arrive rarement qu'on regrette de s'être marié par l'avis des parents, et d'avoir pris par un choix mûr et délibéré des femmes qu'on puisse aimer toujours. Car tout ce qui se fait en ne consultant que nos passions n'a qu'un temps. Ce qui se fait au contraire par raison et par jugement est stable et dure longtemps"}.  
\end{displayquote}

 C'est un constat similaire que fait Maurice Godelier, cinq siècles plus tard, à l'issue de ses enquêtes ethnologiques. Selon lui le désir sexuel est fondamentalement asocial et même destructeur : 
 \begin{displayquote}
 \emph{« La permissivité en matière de sexe s'arrête [...], dans toutes les sociétés, soit là où la formule d'alliance serait menacée, soit là où les rapports de coopération et d'autorité entre consanguins risqueraient de s'effondrer et, glissant les uns dans les autres, de disparaître (Na). Mais cette fois, ce n'est plus de la sexualité-reproduction qu'il est question, mais de la sexualité-désir qui, nous l'avons vu, est dans son fond asociale. Elle n'est jamais la base d'une coopération durable entre les individus, tant au sein du groupe où ils sont nés qu'entre lui et les groupes avec lesquels il est allié. Et ce n'est pas seulement le désir hétérosexuel qui unit et divise. [...] C'est, nous l'avons dit, parce que le désir sexuel en lui-même est asocial qu'aucune société ne peut permettre que tout soit permis.} 
 
 \emph{Et ce travail d'auto-domestication est toujours à recommencer, alors que le processus de domestication des plantes et des animaux semble avoir atteint ses limites. [...] Partout la spontanéité du désir a dû être sacrifiée pour produire un ordre social qui est toujours en même temps un ordre entre les sexes et un ordre sexuel. Partout a dû être éliminé le caractère asocial de la sexualité, sacrifié le polymorphisme du désir, interdite la permissivité sexuelle généralisée pour que la société puisse s'organiser et se reproduire.}
 
\emph{[...] Cependant, sacrifier le caractère asocial de la sexualité n'est pas seulement un acte d'amputation.  C'est en même temps une sorte de création. C'est agir sur soi pour continuer non seulement à vivre en société, mais à produire de la société pour vivre, ce qui est le propre de l'homme et le séparera toujours davantage, chaque jour qui passe, des primates, ses lointains cousins. (p. 632-636)}
\end{displayquote}
Malgré la réhabilitation du désir, et notamment du désir sexuel, que l’oeuvre de Freud a initiée un ethnopsychiâtre d'aujourd'hui comme Tobie Nathan  ne dit pas autre chose : 
\begin{displayquote}
\emph{"Qu'est-ce qui différencie la passion de l'amour, notamment conjugal ? La passion, ce n'est pas l'amour. D'ailleurs, les Grecs avaient deux mots distincts. Philia signifie l'amour raisonnable - comme l'amour conjugal - ou l'amitié, tandis qu'Eros désigne le désir, la passion amoureuse. Platon la caractérisait par le manque. Observation exacte, mais insuffisante. Il s'agit d'une exacerbation du manque - dans la passion, l'autre me manque quand il n'est pas là, il me manque quand il est là, car il n'y est jamais suffisamment ; dans la relation sexuelle, et même au moment de l'orgasme, il me manque encore. Recherche d'une fusion impossible, pulsion à offrir à l'autre tout votre espace intérieur - la passion amoureuse est une folie. La seconde caractéristique est qu'elle produit du changement, un bouleversement radical, et ce mouvement n'est pas maîtrisable."}
\end{displayquote}
Si le désir sexuel ne suffit pas pour unir les amants au-delà de quelques mois ou années, est-ce qu'on pourrait au moins défendre l'idée qu'il faut s'en remettre à lui pour choisir celui ou celle avec qui fonder une communauté de vie durable et accueillir des enfants ? Mais il est de notoriété publique que les unions n'ont jamais été aussi fragiles que depuis que le mariage d’amour est devenu le modèle et que les parents ont perdu la capacité d’arranger (ou d'empêcher) les mariages de leurs enfants ! 
Mais est-ce que nos sociétés ont encore pour but de favoriser la création de couples conjugaux durables ? Nietzsche affirmait dès 1888 que la dénaturation du mariage était en cours. Il stigmatisait l'importance donnée dès son époque au mariage d'amour mais il mettait en cause d’abord et avant tout l'égalité des époux :
\begin{displayquote}
\emph{« On vit pour aujourd'hui, on vit très vite -- on vit sans aucune responsabilité : c'est précisément ce que l'on appelle « liberté ». Tout ce qui fait que les institutions sont des institutions est méprisé, haï, écarté : on se croit de nouveau en danger d'esclavage dès que le mot « autorité » se fait seulement entendre [...] Témoin : le mariage moderne. Apparemment toute raison s'en est retirée : pourtant cela n'est pas une objection contre le mariage, mais contre la modernité. La raison du mariage -- elle résidait dans la responsabilité juridique exclusive de l'homme : de cette façon le mariage avait un élément prépondérant, tandis qu'aujourd'hui il boite sur deux jambes. La raison du mariage -- elle résidait dans le principe de son indissolution : cela lui donnait un accent qui, en face du hasard des sentiments et des passions, des impulsions du moment, savait se faire écouter. Elle résidait de même dans la responsabilité des familles quant au choix des époux. Avec cette indulgence croissante pour le mariage d'amour on a éliminé les bases mêmes du mariage, tout ce qui en faisait une institution. Jamais, au grand jamais, on ne fonde une institution sur une idiosyncrasie ; je le répète, on ne fonde pas le mariage sur « l'amour », -- on le fonde sur l'instinct de l'espèce, sur l'instinct de propriété (la femme et les enfants étant des propriétés), sur l'instinct de la domination qui sans cesse s'organise dans la famille en petite souveraineté, qui a besoin des enfants et des héritiers pour maintenir, physiologiquement aussi, en mesure acquise de puissance, d'influence, de richesse, pour préparer de longues tâ-ches, une solidarité d'instinct entre les siècles. Le mariage, en tant qu'institution, com-prend déjà l'affirmation de la forme d'organisation la plus grande et la plus durable : si la société prise comme un tout ne peut porter caution d'elle même jusque dans les générations les plus éloignées, le mariage est complètement dépourvu de sens. -- Le mariage moderne a perdu sa signification -- par conséquent on le supprime. »} (\emph{Le Crépuscule des idoles}, 1888) 
\end{displayquote}

Que pensait Nietzsche de cet état de fait ? Le dénonçait-il ou se bornait-il à le constater de manière lucide et un peu provocatrice ?

En tout cas les femmes ne retourneront plus dans des gynécées, sinon contraintes et forcées (et par qui à part elles-mêmes ?). L'exigence d'égalité absolue (de dignité, de pouvoir, de salaire, de promotion, etc.) entre hommes et femmes placés dans les mêmes situations est une évidence de notre temps sur laquelle il est peu probable que l'on revienne. Avec quels arguments pourrait-on défendre de manière \emph{convaincante} une inégalité fondée sur le sexe ou sur le genre, que le bénéficiaire de cette inégalité soit mâle ou femelle ? Cette exigence d'égalité est d'ailleurs loin d'avoir encore produit tous ses effets, directs et indirects. 


 
 En même temps que le mariage d’amour triomphait dans les représentations s’effondraient l’un après l’autre la quasi-totalité des contreforts qui naguère étayaient le lien conjugal, que ce soient les lois (l'indissolubilité du mariage « constantinien » n’a été a peu près respectée que sous la pression d’un encadrement juridique patiemment construit et vigoureusement défendu pendant plus d'un millénaire), l'intérêt matériel (le lien conjugal est significativement plus fragile lorsque l'épouse peut se procurer des ressources propres hors du foyer), la pression familiale ou sociale (les divorcés et leurs enfants ne sont plus stigmatisés et leurs nouveaux partenaires sont de plus en plus souvent reçus par leurs parents à égalité avec les précédents), l’impossibilité d’obtenir des héritiers hors mariage régulier (sauf inceste tous les enfants peuvent aujourd’hui être reconnus par leurs deux parents quel que soit le statut matrimonial de ces derniers, et tous peuvent hériter à égalité) les croyances religieuses (les morales traditionnelles d'inspiration religieuses  paraissent désuètes et inadaptées, ou même incompréhensibles, et ne sont plus respectées). Lorsqu'il ne reste plus pour relier les conjoints que le souci d’élever leurs enfants c'est trop souvent insuffisant pour donner du sens à une vie en commun.


 
 Depuis qu'il n'y a plus de différences entre les enfants nés dans le mariage et les autres, il n'est plus nécessaire d'épouser pour avoir des héritiers légitimes et l'intérêt du mariage diminue au fur et à mesure qu'au nom de l'égalité les lois étendent aux non-mariés les droits accordés aux mariés. L’institution du mariage servait à créer des différences, et comme elle n’en crée plus elle s'étiole.   Certes les unions libres existaient déjà autrefois (depuis toujours en fait) mais dans des groupes sociaux peu ou pas concernés par les questions d’héritage. Elles se sont multipliées de façon exponentielle. 
 
Mais pourquoi vouloir que les couples durent ? Après tout il n'est même plus nécessaire que les sexes se rencontrent pour faire un enfant, ni d'être deux pour l'élever. Pourquoi pas une succession d'amours éphémères ?  ...ou pas d'amours du tout ? Ce n'est plus que que par habitude que la loi prescrit encore aux conjoints d'être fidèles, mais elle ne prévoit plus de sanctions à l'encontre des infidèles. Elle ne se sent plus concernée par ce qui ne relève désormais que des vies privées. 


En commentaire des polémiques suscitées par le projet de loi ouvrant le mariage aux personnes de même sexe, Jacques ATTALI esquisse l'avenir qui, compte tenu des évolutions récentes dans les pratiques familiales, reproductives et sexuelles, lui paraît le plus probable   : 
\begin{displayquote}
\emph{« Comme toujours, quand s'annonce une réforme majeure, il faut comprendre dans quelle évolution de long terme elle s'inscrit.
Et la légalisation, en France après d'autres pays, du mariage entre deux adultes homosexuels, s'inscrit comme une anecdote sans importance, dans une évolution commencée depuis très longtemps, et dont on débat trop peu : après avoir connu d'innombrables formes d'organisations sociales, dont la famille nucléaire n'est qu'un des avatars les plus récents, et tout aussi provisoire que ceux qui l'ont précédé, nous allons lentement vers une humanité unisexe, où les hommes et les femmes seront égaux sur tous les plans, y compris celui de la procréation, qui ne sera plus le privilège, ou le fardeau, des femmes.}

\emph{1. La demande d'égalité. D'abord entre les hommes et les femmes. Puis entre les hétérosexuels et les homosexuels. Chacun veut, et c'est naturel, avoir les mêmes droits : travailler, voter, se marier, avoir des enfants. Et rien ne résistera, à juste titre, à cette tendance multiséculaire. Mais cette égalité ne conduit pas nécessairement à l'uniformité : les hommes et les femmes restent différents, quelles que soient leurs préférences sexuelles.}

 \emph{2. La demande de liberté. Elle a conduit à l'émergence des droits de l'homme et de la démocratie. Elle pousse à refuser toute contrainte ; elle implique, au-delà du droit au mariage, les mêmes droits au divorce. Et au-delà, elle conduira les hommes et les femmes, quelles que soient leurs orientations sexuelles, à vouloir vivre leurs relations amoureuses et sexuelles libres de toute contrainte, de tout engagement. La sexualité se séparera de plus en plus de la procréation et sera de plus en plus un plaisir en soi, une source de découverte de soi, et de l'autre. Plus généralement, l'apologie de la liberté individuelle conduira inévitablement à celle de la précarité ; y compris celle des contrats. Et donc à l'apologie de la déloyauté, au nom même de la loyauté : rompre pour ne pas tromper l'autre. Telle est l'ironie des temps présents : pendant qu'on glorifie le devoir de fidélité, on généralise le droit à la déloyauté. Pendant qu'on se bat pour le mariage pour tous, c'est en fait le mariage de personne qui se généralise.}
 
\emph{3. La demande d'immortalité, qui pousse à accepter toutes mutations sociales ou scientifiques permettant de lutter contre la mort, ou au moins de la retarder.}

\emph{4. Les progrès techniques découlent en effet de ces valeurs, et s'orientent dans le sens qu'elles exigent : en matière de sexualité, cela a commencé par la pilule, puis la procréation médicalement assistée, puis la gestation pour autrui. Ces questions de bioéthique ne découlent évidemment pas des demandes d'égalité venant des couples homosexuels et concernent toutes les formes de reproduction, y compris -- et surtout -- « hétérosexuelles ». Le vrai danger viendra si l'on n'y prend garde, du clonage et de la matrice artificielle, qui permettra de concevoir et de faire naitre des enfants hors de toute matrice maternelle. Et il sera très difficile de l'empêcher, puisque cela sera toujours au service de l'égalité, de la liberté, ou de l'immortalité.}

\emph{5. La convergence de ces trois tendances est claire : nous allons inexorablement vers une humanité unisexe, sinon qu'une moitié aura des ovocytes et l'autre des spermatozoïdes, qu'ils mettront en commun pour faire naitre des enfants, seul ou à plusieurs, sans relation physique, et sans même que nul ne les porte. Sans même que nul ne les conçoive si on se laisse aller au vertige du clonage.}

\emph{6. Accessoirement, cela résoudrait un problème majeur qui freine l'évolution de l'humanité: l'accumulation de connaissances et des capacités cognitives est limitée par la taille du cerveau, elle-même limitée par le mode de naissance: si l'enfant naissait d'une matrice artificielle, la taille de son cerveau n'aurait plus de limite. Après le passage à la station verticale, qui a permis à l'humanité de surgir, ce serait une autre évolution radicale, à laquelle tout ce qui se passe aujourd'hui nous prépare. Telle est l'humanité que nous préparons, indépendamment de notre sexualité, par l'addition implicite de nos désirs individuels...}

\emph{Alors, au lieu de s'opposer à une évolution banale et naturelle du mariage laïc, qui ne les concerne pas, les Eglises devraient plutôt se préoccuper de réfléchir, avec les laïcs, à ces sujets bien plus importants : comment permettre à l'humanité de définir et de protéger le sanctuaire de son identité ?}

\emph{Comment poser les barrières qui lui permettront de ne pas se transformer en une collection d'artefacts producteurs d'artefacts ?}

\emph{Comment faire de l'amour et de l'altruisme le vrai moteur de l'histoire ? »}
\end{displayquote}

On aura remarqué que Jacques ATTALI ne décrit pas tant ce qu'il désire que ce qu'il prévoit, dans l'hypothèse où les dynamiques en cours se prolongeraient sans changement. Et il n'exclut pas formellement l'idée que l'avenir qu'il décrit, s'il se réalisait, pourrait n'être pas totalement radieux. 



 Ainsi l'exigence de liberté individuelle absolue, à tout prix et quelles qu'en soient les conséquences est grosse des problèmes pointés par Jacques Attali. Peut-on croire qu'il pourrait exister un domaine de l'existence dans lequel aucun engagement n'aurait d'importance, où aucune parole ne vaudrait rien, et que cela n'entraînerait pas de répercussions significatives dans les autres domaines ? …dans les autres conversations ? …dans les autres relations ? D’autant plus qu'il s'agit d'un domaine charnellement lié à la construction par chacun de son identité ? Peut-on croire que cela n'aurait aucun effet en termes de « lien social » ? Si entre les individus aucune promesse ne vaut, alors il n’est pas impossible qu’il ne reste que la sauvagerie des rapports de force (« brutalisation » des rapports interpersonnels ?). 
 
 Quant à la valorisation de l'immortalité individuelle, c'est une autre forme de l'individualisme. Contrairement au refus de la mort d'autrui, le refus à n'importe quel prix de la mort de soi met l'individu au-dessus de tous les autres et des liens avec eux. Dans ce contexte \emph{« Comment faire de l'amour et de l'altruisme le vrai moteur de l'histoire ? »} Reste-t-il même une place pour l'amour et l'altruisme ? Le sacrifice de soi pour un autre ou pour une noble cause (la justice, la vérité, le bien, etc..) garde-t-il encore un sens ? 



 


\section{Qui désire les enfants ?} 

 
 A qui appartiennent les enfants ?  Ils ne s'appartiennent pas à eux-mêmes, sauf à supprimer le statut de mineur. On ne peut pas non plus dire qu'ils n'appartiennent à personne. Du point de vue des enfants, n'appartenir à personne (ou appartenir à une institution d'assistance publique) c'est être abandonné. 
 Depuis l’antiquité tardive les enfants n'appartiennent plus seulement à leurs pères. Est-ce qu'ils appartiennent aux deux parents, comme dit la loi ? ...ou bien plus à leur mère qu’à leur père ? …ou bien à l'ensemble de ceux qui les élèvent en leur donnant leur argent et leur temps, dont les beaux-pères et belles-mères ? ...ou bien encore à l'État ? 
 
Si l'on en croit Coluche, \emph{« y a des gens qui ont des enfants parce qu'ils n'ont pas les moyens de s'offrir un chien »}. Il posait à sa manière une question essentielle et relativement nouvelle : pourquoi fait-on des enfants ? Pour quoi veut-on des enfants ? À quoi servent les enfants ?
Le désir de serrer un bébé de chair dans ses bras est d'autant plus irrépressible que ses motivations les plus vraies sont inconscientes. Il est sans doute aussi répandu et aussi fort aujourd'hui que par le passé et il ne concerne pas seulement les femmes. 



 Si tous les citoyens des pays dotés d'un bon système d'assistance sociale et de caisses de retraite suffisantes ont besoin qu'il naisse des enfants pour financer leurs périodes d'invalidité et leurs vieux jours, aucun d'eux n'a besoin que ce soient ses propres enfants : c'est précisément pour cela que ces systèmes ont été mis en place. Dans les pays les plus socialement développés, seule la collectivité a besoin d'enfants. D'un point de vue strictement comptable et sauf dispositifs de compensation très généreux des frais qu'entraînent ces derniers l'intérêt financier bien compris des citoyens des États providence est de ne pas en avoir. Leur niveau de vie et leur crédit auprès de leur banquier seront plus élevés s'ils évitent d'investir dans une progéniture. Il ne faut sans doute pas chercher plus loin la faiblesse des taux de natalité de leurs membres, taux qui ne sont que la résultante des stratégies individuelles de leurs citoyens, stratégies d'autant plus rationnelles qu'on ne voit plus aujourd'hui au nom de quelle exigence morale on pourrait les leur reprocher. 
 
Face à la désaffection du mariage et de la procréation qui menaçait la survie de l'Empire Romain, l'empereur Auguste avait réagi en pénalisant les célibataires et ceux qui n'avaient pas d'enfants, et ses lois ont été appliquées sans faillir pendant au moins trois siècles. Si nos taux de natalité baissaient dangereusement, verrions-nous à l'avenir de pareilles incitations légales  à procréer ? 

 Mais les malthusiens et avec eux bien des écologistes pensent que les problèmes de santé de notre planète ont pour origine le fait que les humains sont trop nombreux. Il faudrait en effet que le nombre de ces derniers diminue drastiquement s'ils voulaient tous consommer comme les citoyens des pays développés actuels sans épuiser les ressources disponibles et sans mettre en danger les équilibres de la nature. Cela impliquerait non pas une croissance démographique zéro, mais une décroissance très énergique. L'intérêt commun de l'humanité serait-il sa décroissance numérique et donc l'évitement de la reproduction jusqu'au retour à un effectif écologiquement optimal ? 
 


Chez les juifs et les chrétiens l'accueil de toute naissance est un devoir\footnote{l'avortement a toujours été strictement condamné chez les chrétiens, et les juifs ne le toléraient qu'en cas de force majeure et/ou dans les premières semaines de la grossesse.}. Dans ce cadre à celui qui demande pourquoi il est né il est possible de répondre que Dieu a voulu qu'il vive (en passant à l’occasion par le truchement d’erreurs humaines). Des générations d'enfants ont trouvé cette explication satisfaisante. Un droit inconditionnel à l'existence leur était reconnu quoi qu'il arrive, même s'ils ne correspondaient pas totalement, ou pas du tout, aux attentes de leurs parents. Leur narcissisme en était suffisamment étayé.

La légalisation du droit à l'avortement a changé la donne. Dans des circonstances précisées par la loi l'embryon ou le fœtus a perdu la protection que le texte de la loi (à défaut des pratiques réelles) lui accordait inconditionnellement depuis Constantin. Devenir un jour la personne qu'il est en potentiel, capable de discernement et de réciprocité avec autrui  n'est plus son droit. 

L'argument de fond c'est que tant qu'il n'a pas un nombre de semaines fixé par la loi il n'est qu'une partie du corps de sa mère, qui détient la maîtrise sur cette partie-là comme sur tout le reste. Jusqu'à sa naissance il n'a pas d’existence reconnue, même quand il existe bel et bien aux yeux de ses parents et de leur entourage (d’où parfois des demandes de réparation juridiquement irrecevables en cas d’avortement provoqué par un accident).

Les avortements dans les cas où la santé physique de la mère est sérieusement menacée par la grossesse ne posent guère de problème éthiques, pas plus que ceux où le fœtus est atteint de troubles interdisant sa survie ou son accès à un minimum de communication, quelle que soit la douleur ressentie par les protagonistes. Les médecins sont amenés de temps en temps à abréger sans souffrance la vie de nouveaux-nés reconnus non viables : la Hollande l'a reconnu dans le cadre du protocole de Groeningen. La Belgique s'est également engagée dans cette voie. 

 Par contre lorsque c'est d'abord ou seulement le bien-être de la mère ou celui de sa famille qui sont visés par un avortement, les enfants conscients de ces situations peuvent comprendre qu'on attend d'eux de n'être pas une gêne et de ne pas coûter d'efforts excessifs. Ils peuvent croire que c'est dans la réalité, et non dans leurs fantasmes les plus archaïques, que leurs parents ont eu sur eux pendant un temps droit de vie ou de mort.
 
 Les opposants « pro-vie » à l'avortement se scandalisent qu'on tue des embryons ou des fœtus puisque selon eux il n'y a rien qui les différencie radicalement des nouveaux-nés. Pendant ce temps-là quelques moralistes s'appuient sur  le même constat pour demander au contraire que soit reconnu aux parents le droit de supprimer les \emph{nouveaux-nés} dont ils ne veulent pas, notamment ceux qui présentent des problèmes biologiques (non léthaux) non détectés au cours de la grossesse (ex : trisomie 21). Les memes vont encore plus loin : dans un article du 2 mars 2012 publié dans le \emph{Journal of Medical ethics}, Alberto Giubilini et Francesca Minerva proposent, à la suite de Peter Singer, d'étendre le droit à l'avortement au-delà de la naissance (ce qu'ils nomment avortement post-natal). Voici un extrait de cet article (traduction personnelle) :
 \begin{displayquote}
\emph{« Le droit prétendu des individus (tels que fœtus et nouveaux-nés) de déve-lopper leurs potentialités, droit que certains défendent, cède devant l'intérêt de ceux qui sont actuellement (dès aujourd'hui) des personnes (parents, famille, société) de rechercher leur propre bien-être, parce que, comme nous venons de le démontrer, ceux qui sont seulement des personnes potentielles ne peuvent pas être lésés par le fait de ne pas être introduits dans l'existence. Le bien-être des personnes actuelles c'est-à-dire le bien-être actuel des humains parvenus au stade de personnes en acte, de plein exercice  pourrait être affecté par de nouveaux enfants (même en bonne santé), réclamant de l'énergie, de l'argent et des soins, toutes choses dont la famille peut manquer. Parfois cette situation peut être évitée par un avortement, mais parfois cela n'est pas possible. Dans ces cas du moment que les non-personnes n'ont pas de droit moral à vivre, il n'y a pas de raisons de refuser l'avortement post-natal. Nous avons certes un devoir moral envers les futures générations alors qu'elles n'existent pas encore. Parce que nous tenons pour garanti que ces personnes existeront (quelles qu'elles soient) nous devons les traiter comme des personnes actuelles du futur. Cet argument, cependant, ne s'appli-que pas à tel ou tel nouveau-né en particulier, parce que nous ne pouvons pas tenir pour garanti qu'il deviendra une personne un jour. Est-ce qu'il existera  en tant que personne en acte  dépend en fait de nous et de notre choix.}

\emph{L'adoption peut-elle être une alternative à l'avortement post-natal ?}

\emph{On pourrait nous objecter que l'avortement post-natal ne devrait être pratiqué que sur les personnes potentielles qui ne pourront jamais avoir une vie digne d'être vécue. Dans cette hypothèse les individus en bonne santé et capables d'être heureux devraient être donnés à l'adoption lorsque leur famille ne peut pas les élever. Pourquoi devrions-nous tuer un nouveau-né en bonne santé alors que le confier à l'adoption ne grèverait les droits de personne mais au contraire accroîtrait le bonheur des personnes impliquées (adoptant et adopté) ?}

\emph{Notre réponse est la suivante : nous avons précédemment examiné l'argument de la potentialité (potentialité des êtres de devenir une personne) et montré qu'il n'est pas suffisamment puissant pour contrebalancer l'intérêt de ceux qui sont actuellement des personnes. En réalité combien minces puissent être les intérêts d'une personne actuelle, ils seront toujours supérieurs à l'intérêt (hypothétique) d'une personne en puissance de devenir une personne réelle, parce que ce dernier est égal à zéro. Dans cette perspective ce sont les intérêts des personnes actuelles qui ont de l'importance, et parmi ces intérêts nous devons en particulier considérer les intérêts de la mère qui peut souffrir psychologiquement si elle donne son enfant en adoption. On observe souvent que les mères de naissance rencontrent des problèmes psychologiques sérieux à cause de leur incapacité à élaborer leur perte et à surmonter leur chagrin. Il est vrai que le chagrin et le sentiment de perte peuvent accompagner l'avortement et l'avortement post-natal aussi bien que l'adoption, mais nous ne pouvons pas affirmer que pour la mère de naissance celle-ci est la moins traumatique. Par exemple, ceux qui pleurent un décès doivent accepter l'irréversibilité de la perte, mais souvent les mères naturelles rêvent que leur enfant va revenir vers elles. Cela rend difficile pour elles d'accepter la réalité de la perte parce qu'elles ne peuvent jamais être tout à fait certaines que cette perte est irréversible.
Nous ne cherchons pas à suggérer que ce sont des arguments décisifs contre la validité de l'adoption comme alternative à l'avortement post-natal. Cela dépend beaucoup des circonstances et des réactions psychologiques. Ce que nous sommes en train de suggérer c'est que si l'intérêt des personnes actuelles doit prévaloir, alors l'avortement post-natal doit être considéré comme une option permise aux femmes qui pourraient souffrir de donner leur nouveau-né à adopter. »}
\end{displayquote}

 Pour Alberto Giubilini et Francesca Minerva  il  s'agit donc de promouvoir le droit à l'infanticide, très largement répandu dans le monde entier, mais supprimé par Constantin. Cette demande fait penser à Jonathan Swift et à son \emph{« Humble proposition pour empêcher les enfants des pauvres en Irlande d'être à la charge de leurs parents ou de leur pays et pour les rendre utiles au public »} (1729), mais cette proposition-ci  est formulée sans le moindre humour. Elle provoque un mouvement de refus horrifié. Mais combien de temps durera ce refus ? Ne peut-on imaginer qu'à force de jouer avec elle on finira par en valoriser les avantages et par en accepter les aspects déplaisants  ? On finira peut-etre meme par défendre l'idée que cette proposition va dans le sens de l'intéret de l'enfant ?

 \section{Filiation adoptive ou filiation « biologique » ?} 
 
 Le "mariage constantinien" tel que je l'ai défini télescopait sur le couple des seuls géniteurs (unis de manière socialement reconnue) toutes les dimensions de la conjugalité et de la parentalité (juridique, biologique, affective et éducative) et frappait tout le reste d'illégitimité et notamment la filiation élective, volontaire, et même adoptive. C'est au nom de cette dernière que l'hégémonie de notre tradition juridique est aujourd’hui théoriquement contestée, et cette remise en question est, comme on peut s'y attendre, consubstantiellement liée à la promotion de nouvelles formes de conjugalités. Selon Daniel BORRILLO  les possibilités nouvelles de dissociation entre sexualité et reproduction qui se sont ouvertes grâce aux progrès de la biologie en à peine une génération ont provoqué une panique morale, qui a conduit les théoriciens et les praticiens du Droit à survaloriser les liens biologiques géni-teurs-enfants :
  \begin{displayquote}
\emph{« La biologie commença à devenir ainsi le soubassement réel ou symbolique du système de parenté, à rebours d'une science juridique qui avait plutôt instauré la volonté au cœur de ce système [...] À partir des années 1990, l'expertise biologique  s'est imposée dans les procès en contestation de paternité, la recherche des origines est revendiquée socialement comme droit fondamental de la personne, la différence de sexe est devenue une valeur [...] La nouvelle place prépondérante de la vérité biologi-que dans l'établissement du lien filial fut confirmée en France par la Cour de cassation  ...Par là, la distinction traditionnelle entre reproduction (fait biologique) et filiation (fait culturel), fondement du droit civil moderne, se trouvait questionnée... non pas à partir d'arguments classiques provenant du droit canonique, mais par une rhé-torique qui, d'une part, fera de la différence des sexes une condition sine qua non de la filiation, et, d'autre part, placera l'expertise sanguine et la preuve d'ADN au cœur du dispositif juridique de la parenté.}

\emph{[...] La filiation peut certes tenir compte du fait naturel, mais, en tant que dispositif d'agencement parental, elle répond à des règles propres, affranchies de la nature [...] Elle n'existe que lorsqu'elle est établie dans les conditions et selon les modes prévus par la loi. Autrement dit, la filiation est déterminée par la norme juridique et non par la nature. Ce lien juridique se tisse à partir de quatre fils principaux : la biologie (filiation par le sang), la volonté (adoption), la présomption (paternité suppo-sée du mari de la mère) et le vécu (appelé en droit « possession d'état »).}

\emph{Ce qui compte ce ne sont plus tant les racines naturelles ou surnaturelles d'institutions intangibles que l'efficacité et la plasticité d'instruments juridiques procurant tel ou tel résultat (par exemple la paix des familles ou la solidarité des générations).  [...] fondée sur la volonté, l'adoption est une institution plus apte que la vérité biologique à assurer la stabilité des liens familiaux.}

 \emph{[...]La contestation actuelle de l'ordre familial « naturel » n'est en définitive que la radicalisation de l'idéologie individualiste moderne, selon laquelle la volonté et non la différence des sexes constitue la base de l'institution matrimoniale et parentale. Une filiation dissociée de la reproduction permettra de justifier un système juridique fondé non pas sur la vérité biologique, mais sur le projet parental responsable. De ce point de vue, peu importe l'agencement familial (traditionnel, monoparental, homo-parental, recomposé...), si les prémisses du contrat (égalité dans l'alliance et dans la filiation) sont respectées jusque dans leurs moindres effets. L'État devrait donc traiter sur un plan d'égalité l'ensemble des familles et les autres formes d'intimité.}
 
\emph{Contrairement à la filiation charnelle, la filiation choisie trouve son principe dans la liberté non seulement d'accueillir les enfants des autres, mais également d'abandonner ses propres enfants biologiques, ce qui est pour l'heure uniquement pos-sible pour les femmes (accouchement sous X), mais devrait pouvoir s'élargir aussi aux hommes à travers une déclaration formelle de renoncement à la paternité. La généralisation de la filiation adoptive (y compris pour ses propres enfants biologiques) per-mettrait aussi de mettre la volonté au cœur du dispositif parental. Celui-ci reposerait exclusivement sur la volonté du ou des géniteurs qui donnent l'enfant et celle du ou des adoptants qui l'accueillent. De surcroît, l'adoption est une institution conçue à partir du droit de l'enfant à avoir une famille, contrairement à la filiation biologique qui apparaît plutôt comme un dispositif du droit à l'enfant ». }
\end{displayquote}

 Que tous les systèmes juridiques soient des constructions humaines et non des faits de nature, qu'ils reposent sur des idéologies, sur des prises de positions morales et des croyances plus ou moins partagées, et qu'ils fassent tous des choix entre des possibles, acceptant les uns et refusant les autres, cela est évident. Mais sur quels arguments se fonde l'affirmation que \emph{"L'adoption est une institution plus apte que la vérité biologique à assurer la stabilité des liens familiaux"} ?  On peut tout aussi bien affirmer le contraire. Il n'est pas nécessaire que la filiation adoptive soit meilleure que la filiation ordinaire pour etre reconnue par le droit. A devoir choisir entre le sang ou la volonté pour fonder le droit de la filiation il y a quelque chose qui paraît artificiel et forcé. Tout faire reposer sur la biologie est certes méconnaître qu'elle n'a jamais suffit dans aucune société pour légitimer une naissance, et oublier combien le lien entre un adulte et son enfant est un lien co-créé dans le cadre de leur relation, à l'instar d'une adoption réciproque qui déborde de tous côtés la proximité biologique. Mais ne reconnaitre que la volonté en déniant les corps et leurs dialogues est une fiction juridique, héritée des romains, qui comme toutes les fictions juridiques fait plus ou moins violence aux réalités telles qu'elles sont vécues au jour le jour, dans la complexité et l’ambivalence.
 
 \emph{"L'adoption est une institution conçue à partir du droit de l'enfant à avoir une famille, contrairement à la filiation biologique qui apparaît plutôt comme un dispositif du droit à l'enfant."} Il est vrai qu'en France l'adoption des jeunes enfants, formellement interdite depuis la fin de l'antiquité, a été ressuscitée après la Grande Guerre pour donner des parents à des enfants abandonnés. Il est vrai aussi que pour chaque enfant adoptable il y a actuellement plusieurs candidats à l'adoption. Au plan mondial il en est de plus en plus de même : l'enfant naturel devient de plus en plus rare. Il n’est donc pas nécessaire de promouvoir l’adoption : du point de vue des enfants adoptables elle se porte plutôt bien. La valorisation actuelle de l'adoption vient non de la prise en compte de l'intéret des enfants sans parents mais de la prise en compte du désir d'enfant de ceux qui ne peuvent ou ne veulent pas recourir aux relations hétérosexuelles, désir d'enfant qui en soi n'est ni plus ni moins légitime que celui des autres.  
 
 Mais dès que les enfants sont sortis de la petite enfance, leur adoption n'est pas simple et elle peut être terriblement éprouvante pour le narcissisme des adoptants. Les adoptés courent bien plus que les autres enfants le risque d'être rejetés à cause des difficultés de tous ordres qu'ils ont rencontrées du fait de leur histoire et auxquelles ils se sont adaptés comme ils ont pu. Leurs attentes ne s'engrènent pas toujours harmonieusement avec celles de ceux qui se proposent de devenir leurs parents adoptifs : le pourcentage de ces enfants qui sont abandonnés une deuxième fois après une adoption n'est malheureusement pas négligeable. Tous les adultes ne sont pas prêts à prendre de pareils risques. 
Jusqu'ici il n'a pas été permis par la loi de concevoir des enfants pour les donner, mais si l'on voulait répondre à toutes les demandes d'adoption il faudrait s'y résoudre. 

Tant que dureront les énormes inégalités de revenu observables sur cette planète, les plus fortunés pourront toujours louer le ventre des plus belles et des plus saines des filles des pauvres, de la même façon que les riches romains achetaient les plus jolies des jeunes esclaves afin qu'elles leur fassent des enfants bien à eux qu'ils n'auraient à partager ni avec un partenaire égal à eux en dignité, ni avec une belle famille aussi puissante que la leur. Le recours à des mères porteuses est dans la logique des évolutions libérales actuelles. Il est d'ores et déjà légalement possible dans plusieurs pays développés. Est-il appelé à se généraliser ? Comment refuser ce recours aux hommes homosexuels si l'on accorde l'assistance médicale à la procréation (PMA) aux femmes homosexuelles, et comment le refuser à tous les autres, hommes et femmes célibataires, si on l'accorde aux hommes homosexuels ? Et comment le refuser à qui que ce soit si des femmes (ordinairement de condition modeste et vivant souvent dans des pays sous-développés) sont volontaires pour prêter leur ventre et abandonner leur enfant nouveau-né contre une indemnité suffisante. 

 C'est le seul moyen de mettre les hommes à égalité avec les femmes dans l'accès à l'enfant, ou plutôt de corriger l'inégalité que leur corps leur impose dans ce domaine, mis à part bien sûr le mariage traditionnel, monogame et indissoluble, dont c'était l'une des finalités. Lorsque leur mariage était rompu les pères romains gardaient leurs enfants : ils n'avaient donc pas spécialement intérêt à ce que les unions soient indissolubles. Par contre leurs épouses avaient de bonnes raisons de craindre d'être répudiées et \emph{ipso facto} séparées de leurs enfants. Elles ont trouvé bon à partir du IVème siècle d'être mieux protégées contre ce risque et d'avoir leur vie durant l'exclusivité de la fécondité légitime de leur mari. Aujourd'hui où leur autonomie financière et les lois leur permettent de prendre l'initiative de quitter leurs maris sans risque de devoir lui laisser leurs enfants la situation se retourne et ce sont les hommes qui peuvent commencer de craindre d'être séduits puis abandonnés. 
 
L’instauration d'une déclaration formelle de renoncement à la paternité, proposée par Borillo en miroir du droit reconnu aux femmes à l'accouchement sous X, peut paraître provocante. Pourtant une telle disposition ne ferait que rejoindre le point de vue des révolutionnaires de 1789 : pas plus de contrainte en paternité qu’en maternité. Plutôt que de se retrouver un jour contraints de continuer de payer pour leurs enfants sans plus les avoir auprès d'eux, tandis que parfois un autre qu'eux les éduque, les hommes pourraient choisir, quelle que soient par ailleurs leurs préférences sexuelles, de commencer par payer pour les posséder sans partage afin que leur génitrice ne puisse jamais les leur contester. Sur quels arguments fonder le refus d'une pareille évolution ? Elle ne serait au fond que le miroir de celle qui voit des femmes choisir en toute connaissance de cause de faire un enfant toutes seules. Si les humains ne diffèrent en rien de significatif en dehors de leurs caractéristiques biologiques, si les femmes n'ont pas besoin d'un homme pour élever un enfant, alors les hommes n'ont pas non plus besoin d'une femme pour élever leurs propres enfants.
 
 Le recours aux mères porteuses ne pourrait être interdit, en dépit de la pression des demandes individuelles et du modèle fourni par les pays où cette pratique est autorisée, que s'il était d'abord admis qu'il implique la réduction d'un humain au statut d'instrument de la volonté d'un tiers jusque dans son corps, ce qui est la définition de l'esclave, et s'il était reconnu que c'est inacceptable, même si cette personne a donné son accord. Une deuxième raison serait que cette pratique fait de l'enfant à naître le produit d'un contrat commercial (sauf à ce que se généralise le don d'enfants par des femmes qu'aucune nécessité matérielle y pousserait, mais rien ne montre qu'on aille vers là).
  
  Mais refuser le recours à des mères porteuses impliquerait aussi d'accepter l'idée qu'il n'existe pas de droit à l'enfant, c'est-à-dire que tout un chacun peut être irrémédiablement privé d'enfant en dépit de ses désirs les plus authentiques et les plus légitimes sans avoir pour autant droit à la répara-tion de cette injustice. Le mouvement des pratiques depuis un demi-siècle ne va pas dans ce sens.

Quant à espérer sortir de ces contradictions en recourant à un utérus artificiel, c'est encore et pour longtemps de l'utopie.

Est-ce que le recours à une adoption ou à une mère porteuse est aussi satisfaisant pour les enfants concernés que pour leur(s) parent(s) ? On aimerait que ce soit le cas, mais beaucoup d'adultes nés d'une insémination artificielle avec donneur (IAD) n’en expriment pas moins le désir de connaître leurs « origines ». On pourrait postuler que si la filiation adoptive était instituée comme le modèle de la filiation la réalité des parents de naissance perdrait de son importance, mais ce n'est qu'une hypothèse. Beaucoup parmi les jeunes et les adultes nés sous X veulent connaître au moins leur génitrice, et ceux à qui cela est refusé disent souffrir d'une peine inguérissable. A défaut de pouvoir exiger d'être élevés par leurs deux parents de naissance, les enfants concernés (beaucoup d'entre eux) veulent au moins les connaître et même si l'on ne voit pas tou-jours à quoi cela pourrait leur servir, eux le voient et ils s'obstinent. Même si on le leur refuse ils continuent de le vouloir. Et au nom de quoi pourrait-on les en empêcher ? Ils ont le droit pour eux au moins autant que les adultes ont le droit de vouloir un enfant.
 
 Certes tous les jeunes nés sous X ou d'une IAD ne sont pas tourmentés par ces interrogations, mais cela ne peut que rendre dubitatif. Même s'il est assez facilement accepté par les parents légaux (et on peut humainement le comprendre) l'oubli des parents de naissance, des géniteurs, n'est pas possible. Même si c'est seulement de façon imaginaire ces personnes font irrémédiablement partie de la relation entre les parents réels, légaux, et leurs enfants,  même lorsqu'ils s'interdisent d'en parler. 



\section{Il n'est pas bon que l'homme soit seul ?} 

Face à l'ensemble de ces évolutions sur quoi fonder des jugements ? 

Les préceptes religieux ne convainquent que ceux qui y croient. 

Les morales laïques sont variées et contradictoires : d'un côté (en perte de vitesse) on valorise l'engagement et la fidélité à la parole donnée, de l'autre (qui a le vent en poupe) on plaide pour l'authenticité et la liberté. 

Le critère le plus objectif et le moins discutable serait peut-être l'utilité sociale ? D’un point de vue collectif qu'est-ce qui est préférable ? la fidélité aux engagements ou la liberté d'être à chaque instant en accord avec ses désirs du moment ?  

A défaut d'enquêtes de satisfaction il serait peut-être possible de chiffrer l'ensemble des coûts et des bénéfices directs et indirects comparés de la fidélité et de l'authenticité.

Dans l’un de ses récits la Bible définit la femme comme une aide pour l’homme. Selon les traducteurselle est \emph{à côté de lui} ou \emph{contre lui} , mais elle est de toute façon décrite comme complémentaire : Eve complète Adam et là est la justification de son existence. 

Avec ou sans la Bible les hommes ont toujours été d'accord avec cette façon de penser et ils ont partout trouvé normal de se situer en chefs de famille. Ils ont toujours trouvé \emph{naturel}, de classer les tâches en anoblissantes et viles, de s'attribuer les nobles et de vouer les femmes aux autres. Ils ont partout bridé l’efficacité, la productivité de ces dernières en ne leur accordant pas l’accès aux meilleurs outils, en les tenant à l’écart des apprentissages et des savoirs. Ils ne leur ont reconnu de place qu’au sein du foyer, auprès des petits enfants, des malades et des grands vieillards, et dans les activités ménagères\footnote{Les Mains, les outils, les armes [article], Paola Tabet
in \emph{L'Homme, Année 1979, Volume 19, Numéro 3}, pp. 5-61.
Paola Tabet, \emph{La Construction sociale de l’inégalité des sexes. Des outils et des corps}
Paris-Montréal, L’Harmattan, 1998, 206 p.}. 


Ce constat n'interdit pas de se demander en quoi il pourrait être bon, utile, de vivre en couple, même avec un(e) autre strictement égal(e) à soi  ? 
 

A côté des avantages subjectifs (être soutenu, trouver à qui parler, vivre avec un partenaire sexuel, etc...), qui peuvent prendre plus ou moins d'importance suivant les circonstances, il y a des avantages matériels certains à vivre en couple : un seul loyer, un seul équipement domestique, une seule voiture familiale,  etc.. et c'est d'autant plus vrai qu'on est moins à l’aise financièrement, tandis que les séparations font perdre cet avantage et sont parfois ruineuses (cf. tous les dépenses engendrées par le « démariage » sur le marché immobilier, sur celui de l'équipement domestique ou sur celui de l'assistance juridique, etc). La traduction concrète de tous ces coûts c'est la pauvreté qui frappe beaucoup de divorcés ou de parents célibataires : femmes seules avec enfants, mais aussi hommes seuls dont les ressources sont insuffisantes pour retrouver un logement, etc. Il faut y ajouter les coûts psychologiques des séparations, qui sont en partie inséparables des coûts matériels, ainsi que le note Gérard Neyrand  :  
\begin{displayquote}
\emph{« En effet, les nouvelles valeurs familiales sont portées par les couches moyennes cultivées et sont devenues système de référence global. Leur confrontation aux habitus des couches populaires en la matière ne s'effectue pas sans conflits (Commaille, Martin, Les enjeux politiques de la famille, Paris Bayard, 1998). L'une des issues des contradictions entre ces systèmes différents de références, qui traversent différemment les individus selon leur sexe et leur position sociale, réside dans la fréquence des séparations conjugales conflictuelles, la mono-parentalisation maternelle qui s'en suit et la précarisation des foyers mono-parentaux ainsi définis. Leur caractéristique est bien d'être soumis à un double système de contraintes croisées, socio-économiques et psycho-relationnelles.
La montée du chômage et la précarisation des emplois les moins qualifiés\footnote{ Boltanski, Chiapello, Le nouvel esprit du capitalisme, Paris Gallimard, 1999}, contribuent à une fragilisation globale des situations familiales des plus démunis, qui risque d'autant plus de déstabiliser les familles que ces familles populaires se pensent de façon unitaire, quasi-symbiotique.
Elles sont basées sur un couple conçu comme une entité indissoluble, un « couple unité organique » selon l'expression d'Irène Théry\footnote{Le couple occidental et son évolution sociale : du couple « chainon » au couple « duo », Dialogue, \no 150, 4e trimestre, 2000}, et sont loin d'adhérer sans réserve au nouveau modèle moderne du « couple duo ». La séparation, dès lors, constituera une catastrophe identitaire dont beaucoup auront du mal à se relever, en particulier les pères.
On conçoit alors l'importance des difficultés que des séparations dans un tel contexte peuvent générer :
- difficultés relationnelles entre les ex-conjoints et dans le rapport des pères à leurs enfants,
- et difficultés socio-économiques des mères confrontées aux nécessités d'une survie familiale qu'elles doivent bien souvent affronter seules.
Mono-parentalisation et précarisation s'avèrent alors intimement liées. »}
\end{displayquote}


La séparation des amants dont la passion s'est refroidie ne va pas dans le sens du renforcement de leurs capacités éducatives. La dissolution du couple parental multiplie le nombre des situations où la fonction éducative de l'un ou de l'autre est plus ou moins disqualifiée ou à tout le moins entravée, tandis que son remplacement au quotidien par le ou les partenaire sexuels et affectifs de l'autre n'est pas forcément bien accepté par les enfants concernés et ne présente pas toujours l'efficacité éducative souhaitable. Le nombre s'élève donc des parents qui face aux inévitables problèmes éducatifs que posent tous les mineurs sont seuls et/ou en difficulté. Lorsque leurs enfants leur posent problème, notamment à l'adolescence, des aides extérieures sont souhaitables, mais rien ne garantit que l'efficacité des diverses aides éducatives ainsi apportées soit supérieure à ce qu'en d'autres circonstances les parents auraient pu assumer eux-mêmes : ce serait déjà bien si on pouvait être assuré qu'elle n’est pas trop souvent inférieure. D'autre part même en tenant pour négligeable la disqualification et la mise en dépendance des parents par les spécialistes du contrôle social des familles et les experts de l'éducation, les soutiens que propose la collectivité ne sont pas gratuits (ex. :  internats scolaires, assistance éducative, placement en famille d'accueil,~etc.). On passe de l'auto production à l’externalisation et à la professionnalisation des tâches éducatives. Comme c'est un domaine où il n'y a à espérer aucun gain de productivité cela accroît les coûts éducatifs de manière très sensible. Jusqu'où peut-on aller dans cette voie avant que les électeurs n'estiment que c'est trop cher payé ?
 
En conclusion il serait peut-être difficile de prouver chiffres en main que le mariage traditionnel est supérieur à toutes les autres manières d'organiser la vie des individus, mais il est au moins aussi difficile de croire que du seul point de vue de la société la stabilité des unions fécondes et une suffisante fidélité des parents l’un à l’autre n'ont aucun intérêt  et sont définitivement dépassées. 

Par contre du point de vue des entreprises et des administrations la disponibilité d'un(e) employé(e) est plus grande lorsqu'il n'est plus nécessaire de tenir compte de son désir de vivre avec un(e) partenaire lui-même (elle-même) bien inséré(e) professionnellement et qui lui (elle) aussi s'attend à être significativement investi(e), de même qu'un(e) célibataire sans enfants et sans intention d'en avoir est précieux(se) pour son employeur et possède un atout significatif pour réussir une belle "carrière" professionnelle.


\section{"Papaoutai"} 


 
Dans \emph{Quelle alternative au patriarcat ? Valoriser un modèle social non conjugal} (2004), Agnès \fsc{echene} accuse le couple hétérosexué d'être le lieu privilégié d'expression et de transmission de la violence masculine, et cela trop souvent avec la complicité (masochiste) féminine. Elle en tire la conclusion qu'il faut éliminer la paternité en tant que telle :
\begin{displayquote}
\emph{« ...ce n'est qu'en valorisant le modèle social non conjugal qu'une société peut se défaire du patriarcat. Il importe donc de favoriser une sexualité libre et variée, tout en étant discrète et protégée, surtout chez nos propres enfants ; peu importe dès lors qu'elle soit ardente ou paisible, monotone ou changeante, homophile ou hétérophile, dès l'instant qu'elle reste une affaire personnelle dont nul ne se mêle. Une telle évolution nécessite également une reconsidération du modèle familial qui doit se re-fonder sur des liens d'appartenance utérine et non pas consanguine ; cela remet en cause dès lors la paternité génitale qui doit laisser place à une paternité germaine : il faut en effet que ce soit les frères, oncles et cousins [de la mère] qui assument les enfants des femmes ; de nombreux signes avant-coureurs montrent qu'ils sont prêts à le faire et qu'il ne manque qu'un déclic. Mais il faut aussi que les femmes renoncent à obliger les géniteurs à être pères ; il faut qu'elles abandonnent toute velléité de recherche de paternité, de pension, partage, alternance,~etc. et se tournent résolument vers leurs frères, oncles et cousins pour « donner » des pères à leurs enfants, qui ne s'en porteront pas plus mal\footnote{Puisqu'elle le dit.}. »}
\end{displayquote}

On est là à très peu de choses près dans le monde de l'ethnie Na et des autres groupes, chinois de l’ouest ou thibétains, qui ne connaissent pas de pères, et où les hommes de chaque famille sont les amants librement choisis pour une nuit ou pour plusieurs des femmes des familles voisines. Ces familles reposent toutes sur un principe matriarcal, puisque les enfants appartiennent exclusivement à la famille de leur mère. Ce sont donc les oncles maternels qui "paternent" les enfants de leurs sœurs. Mais il n'a pas suffit que les femmes Na aient le choix du géniteur de leurs enfants pour que l'autorité dans le groupe familial leur soit dévolu. 

D'autres imaginent des constellations d'une tout autre espèce, des associations basées sur des contrats de solidarité privés ne se référant plus au couple, mais plutôt aux communautés créées dans les années 60-70 du XXème siècle. Selon Marcella Jakub :
\begin{displayquote}  
\emph{"A l'époque de la discussion sur le pacs, certains avaient proposé de créer des liens de solidarité entre plusieurs individus, et pas uniquement au sein du couple, qu'il soit hétérosexuel ou homosexuel. Le pacs aurait pu permettre, par exemple, d'associer des personnes au moyen de liens juridiques alternatifs qui ne soient pas forcément fondés sur la famille. Voilà une proposition sociale intéressante, qui aurait permis d'inventer des formes de vie à plusieurs. Mais nous sommes loin d'une telle réflexion ».}
\end{displayquote}
 Va-t-on de manière plus banale vers des foyers constitués d'une femme et des enfants qu'elle a mis au monde, autour desquels graviterait la nébuleuse de ses amants et ex-amants ? Dans cette hypothèse, les hommes de demain auraient des enfants de plusieurs femmes, enfants vivant ordinairement chez leurs mères, si bien que le poids de leur parole auprès de chacun d'eux serait à peu près nul ? Serait-ce en quelque sorte l'inverse de la situation du pater familias romain, qui pouvait demander à plusieurs femmes des enfants sur lesquels lui seul avait autorité et qui vivaient tous chez lui s’il le voulait ?
 
 
 C'est ainsi que fonctionne le modèle matrifocal « antillais » ou « caraïbe » dont l'origine se situe dans l'histoire du peuplement des Antilles. On a vu que les esclaves n'ont par définition aucun des attributs juridiques d'un père ou d'une mère sur les enfants dont ils sont les géniteurs : seuls les propriétaires des génitrices possèdent des droits sur les enfants de celles-ci. C'étaient ces propriétaires qui faisaient d'elles des mères lorsqu'ils leur confiaient la garde des enfants qu'elles avaient portés, quel qu'en ait été le géniteur. Il y avait une espèce d'alliance de fait (alliance sous contrainte, perverse) entre les génitrices et leurs maîtres pour élever les enfants qu'elles avaient mis au monde, tandis que leurs partenaires sexuels n'avaient pas droit à la parole et étaient réduits, quel qu'ait pu être leur désir,  à n'être que des donneurs de sperme. 
 
 



 
Autrefois (jusqu'aux années 60 du siècle dernier ?) c'est l'excès de présence et de poids des pères qui faisait problème. Aujourd'hui nous ne pouvons plus investir les pères comme idéaux : ils ne sont plus pensés comme les relais du pouvoir d'un Dieu, de la Cité, de la République, de l'Empereur, du Roi ou de l'État. Aujourd'hui ils semblent n'être jamais assez présents, jamais là où il faut. Dans l'effritement de leur image, Françoise \fsc{HURSTEL} pointe trois moments clé : la loi de 1889 contre les « \emph{parents indignes} », la loi de 1935 abolissant le droit de « \emph{correction paternelle} » et la loi de 1938 abolissant la « \emph{puissance maritale} ». Ont été abolies toutes les dispositions juridiques sur lesquelles était fondé dans le passé l'exercice masculin d'un rôle patriarcal. Le résultat est que « [...] \emph{nous ne savons plus ce qu'est la place d'un père et ce que sont ses fonctions} », et que « \emph{ce ne sont pas des petits bouts de la paternité qui ont changé, mais l'ensemble du système a muté avec la mort du \emph{pater familias}.} »%
% [3]
\footnote{Françoise \fsc{HURSTEL}, « Penser la paternité contemporaine dans le monde occidental : quelles places et quelles fonctions du père pour le devenir humain, sujet et citoyen des enfants ? », in \emph{Neuropsychiatrie de l'enfance et de l'adolescence}, 53 (2005) 224-230.} 


 
  Si les lois suivaient l'évolution des mœurs, alors la promulgation d'une loi serait le signe que les esprits sont prêts à l'accueillir. Dans cette hypothèse, pendant les années précédant la promulgation de chacune des lois ci-dessus, on aurait dû observer un mouvement de l'opinion publique stigmatisant les parents indignes, le recours abusif au droit de correction paternelle, ou le scandale que constitue l'existence d'une puissance maritale. Selon Françoise \fsc{HURSTEL} ce n'est pas ainsi que cela s'est passé, au contraire. Ce n'est qu'à partir de la promulgation de la loi de 1889 que la presse aurait commencé de dénoncer les carences des pères « indignes%
% [4]
\footnote{« \emph{alcoolique, pauvre, inculte et violent} », Françoise \fsc{HURSTEL}, \emph{la déchirure paternelle}, p. 113.} 
 ». Et de même ce n'est que vers 1942 que les spécialistes de l'éducation auraient commencé de dénoncer les pères sans autorité, tandis que la notion de carence n'aurait envahi les écrits qu'à partir de 1950 :
 
\begin{displayquote}
\emph{"C'est donc quelques années après la promulgation de ces lois faisant disparaître des textes juridiques les termes de puissance (maritale) et ceux de correction paternelle tout en maintenant ceux de chef et d'autorité (paternelle), qu'est décrite cette figure d'un père manquant d'autorité et de sévérité ; et que les spécialistes admonestent les pères d'une position qui est bien celle de chef de famille."}
\end{displayquote}

Elle soutient que ces discours sont l'effet de ces changements, et non leur cause.  Selon elle, l'opinion publique n'aurait appelé aucune de ces lois de ses vœux. Ces réformes n'auraient été imaginées, réclamées, et parfois discrètement expérimentées que par les seuls experts, médecins, administrateurs, juges et travailleurs sociaux directement intéressés à leur mise en œuvre. Pour Françoise \fsc{HURSTEL}, tous les discours sur les déficiences des pères actuels ne sont que des productions imaginaires qui coexistent avec des réalités qui n'ont pas grand-chose à voir avec elles. En effet, les enquêtes sur le terrain ne montrent rien qui permette de croire que les pères d'aujourd'hui seraient dans l'ensemble moins attentifs et moins présents que ne l'étaient ceux du passé%
% [5]
\footnote{... mais cela exige d'éviter les biais méthodologiques. Il faut notamment que ces enquêtes ne se placent pas consciemment ou inconsciemment du seul point de vue des mères. Cf. Germain \fsc{DULAC}, « La configuration du champ de la paternité : politiques, acteurs et enjeux », in \emph{Politiques du père, numéro spécial de Lien social et politiques}, (n° 37) 1997, p. 133-142.}%
. Certes il y a des pères qui sont incompétents, irresponsables ou délinquants, mais cela n'a rien de nouveau, et rien ne permet d'affirmer qu'il y en ait plus qu'autrefois. Les discours ne portent pas tant sur ce que font réellement les pères que sur ce qu'ils devraient faire dans l'idéal pour être de bons pères. Il s'agirait, à l'aide de ces discours, d'asseoir l'autorité de ceux qui prétendent savoir ce qu'est un bon père et qui sont les bons pères. En somme la mise en acte de l'utopie décrite par Ernest \fsc{TARBOURIECH}  : {\emph{"le père et la mère n'auront sur leur progéniture aucun droit d'aucune sorte, mais seulement des devoirs\footnote{in \emph{La cité future}, 1902}"} ? Et comme il le préconisait ce sont des experts de l'éducation (il mettait à cette place les médecins) qui apprécient la conformité ou non des parents à leurs devoirs et en prennent acte. En conclusion :
 
\begin{displayquote} 
\emph{"Du point de vue de la paternité les hommes de la période contemporaine n'auront pas été gâtés. Je propose une image pour illustrer ce que peut être la notion de carence : lorsqu'un homme devient père, il endosse un pardessus plein de trous et de soupçons..., plus précisément une image de plus en plus dévalorisée, et cela quelle que soit la valeur personnelle de l'homme qui assume une telle fonction. Et ce qui les caractérise est un discours dévalorisant des spécialistes ; tellement dévalorisant qu'il apparaît, en fait, comme un discours de l'exclusion des pères... au profit du super père spécialiste. Si les pères peuvent être dits carents \emph{[en Droit, le père « carent » est celui qui ne laisse rien à ses enfants, qui ne leur laisse aucun héritage]}, c'est parce qu'ils sont relégués à cette place par ceux-là mêmes qui normalisent les pratiques autour de l'enfant. Nous dirons que ces pères carents sont en fait d'abord des pères exclus par les théoriciens de l'éducation.}%
% [6]
\footnote{Françoise \fsc{HURSTEL}, \emph{la déchirure paternelle}, p. 112-113.} 

[...] \emph{Ainsi les signifiants inscrits dans la loi produisent des effets imaginaires qui se repèrent dans les représentations collectives, les modèles normatifs du père et les pratiques sociales.} 

\emph{Je ferai ici un pas de plus et avancerai ceci : non seulement les signifiants des lois produisent des effets imaginaires, mais encore les lois elles-mêmes ne sont connues que par le biais de ces productions...}

\emph{Les figures du père carent semblent bien avoir une fonction sociale et idéologique importante, celle d'être l'une de ces fonctions sociales qui rendent compte et qu'il y a du père dans notre société (au sens du père symbolique et de la fonction paternelle) et qu'il y a du changement dans les montages qui instituent le père... bref, elles seraient un mode d'historicisation d'une structure.}

\emph{Mais en retour cet imaginaire du père marquera chaque homme ayant à assumer la fonction paternelle, chaque mère appelée à reconnaître qu'il y a du père pour son enfant."}%
% [7]
\footnote{Idem, p. 113-115.}
\end{displayquote}

 Les mères et la fonction maternelle ont toujours été valorisées dans les représentations communes : elles sont traditionnellement du côté de l'accueil de la vie et de son entretien, de l'intime, de la tendresse, du cœur (du care). Mais aujourd'hui cette idéalisation n'est plus contrebalancée par celle qui entourait les pères et la fonction paternelle des siècles classiques. Aujourd'hui la déploration des déficiences des pères, de leurs fragilités et de leur immaturité est un passage obligé de tout discours sur la famille, tandis que l'idée qu'ils puissent mettre en oeuvre leur force ou leur puissance dans une relation à des enfants suscite quasi-mécaniquement des représentations de violence et de maltraitance. Quand on parle sans les spécifier des violences conjugales, il va de soi qu'il s'agit des violences masculines, alors que les enquêtes et recherches rigoureuses montrent que les femmes sont très capables de concurrencer les hommes de manière significative dans ce domaine \emph{aussi} et qu'elles n'ont jamais été sans défenses.
 
 \begin{displayquote}
 \emph{"Viols et violences, mépris et humiliation des femmes et des hommes dévalorisés qui leur sont assimilés, cynisme, manque de pensée et appauvrissement affectif : la représentation des hommes qui exsude d'une lecture attentive des recherches qui leur sont consacrées est suffocante. Quels que soient les champs disciplinaires et les orientations théoriques, la virilité désigne l'expression collective et individuelle de la domination masculine et ne saurait donc constituer une définition positive du masculin\footnote{Molinier Pascale \emph{Virilité défensive, masculinité créatrice}, in  \emph{Travail, genre et société}, n° 3, mars 2000.}".}
 \end{displayquote}
 
D'ailleurs maintenant que le capital le plus utile c'est le capital intellectuel, maintenant que l'avenir des enfants se prépare à coup d'études longues, financées en partie par la collectivité, sous la houlette de professionnels de l'enseignement et sous le contrôle de l'État, qu'est-ce qu'un père pourrait bien transmettre à ses enfants, à part ses biens, sans menacer leur autonomie ? Dans un environnement allergique à tout ce qui ressemble à du paternalisme, qu'est-ce qu'un homme est autorisé à désirer concernant des enfants ? Des points de vue et des désirs spécifiquement masculins sur les enfants sont-ils acceptables ? Il est ordinaire que les adolescents traversent des états d'incertitude identitaire, avec les malaises que cela implique, étant donnés toutes les métamorphoses par où ils passent. Mais encore plus déstabilisantes pour eux sont les incertitudes identitaires de leurs adultes de référence. Si même les adultes ne savent plus ce qu'est un homme, à quel modèle les garçons peuvent-ils se mesurer ? les discours qu'ils peuvent entendre sur ce que c'est qu'un homme sont souvent insupportables. C'est pourquoi ce n'est pas par hasard si aujourd'hui ce sont eux qui plus que les filles expriment bruyament leur désarroi : violences contre eux-mêmes, contre les personnes et contre les biens, prises de risques inconsidérées, désinvestissement scolaire, etc.
 
 



