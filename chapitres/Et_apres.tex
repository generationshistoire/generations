% Le 25.03.2015 :
% Â_ --> À_
% ~etc.
% même
% Le 22.03.2015 :
% A_ --> Â_
% Le 16.03.2015 :
% Moyen Âge
% Le 24.02.2015 :
% ~etc.
% Moyen-Âge
% ~\%
% fœtus


\part{Et après ?}



 Tout est aujourd'hui remis en question dans le domaine de la génération, jusqu'à la nécessité pour concevoir un enfant de deux cellules germinales, mâle et femelle. Nous ne tolérons plus aucune contrainte dont la nécessité ou l'inéluctabilité n’aient été prouvées de manière indiscutable. Nous sommes prêts à tout essayer et à repousser toutes les limites. Les historiens, les ethnologues et les sociologues nous fournissent à l'envi les exemples de pratiques alternatives dont nous pourrions avoir besoin pour « déconstruire » les représentations traditionnelles et les idées reçues du passé. 
Tout semble donc désormais possible. Et pourtant tout n’arrivera pas : tous les possibles ne seront pas compatibles entre eux et certains se révéleront indésirables. 
A quoi ressemblera le résultat final de ces évolutions ? 
Ce chapitre ne prétend pas prédire l’avenir mais présenter les problèmes que demain devra résoudre, ceux du moins qu’il est possible de percevoir aujourd'hui.

 \chapter{Le sentiment amoureux peut-il à lui seul unir durablement les amants ?} 

Le vingtième siècle a cru de toutes ses forces à l’amour et il a tout fait pour le promouvoir au sein des couples. Il s’y est peut-être senti encouragé par les autorités les plus vénérables. En effet lorsque dans la genèse YVWH dit \emph{« Il n'est pas bon que l'homme soit seul. Il faut que je lui fasse une aide qui lui soit assortie."} (Gn 2, 18) c'est amoureusement qu'Adam investit celle qui lui est donnée : \emph{"à ce coup c'est [Eve] l'os de mes os et la chair de ma chair !"} (Gn, 2, 23). Dans ce modèle la relation de l’homme et de la femme est si intime qu'elle les rend aussi proches que le sont les parents les plus proches : \emph{"C'est pourquoi l'homme quitte son père et sa mère et s'attache à sa femme, et ils deviennent une seule chair."} (Gn, 2, 24). 

Mais il faut se demander de quel amour il s’agit. En effet ce mot possède plusieurs acceptions. Autrefois nul ne pensait que le désir amoureux ou sexuel (Eros) soit suffisant pour former un couple, même pas ceux qui plaidaient pour que les jeunes à marier aient la liberté de choisir leurs conjoints. On disait que la passion est aveugle, qu'elle est le lieu de toutes les illusions, qu'elle doit plus aux représentations du passionné qu’à la réalité de l’objet de sa passion. De l'Antiquité à la Belle Époque les moralistes se sont méfiés du désir sexuel, y voyant une entrave à l'exercice de la raison, une force aveugle, inconstante, décevante et potentiellement destructrice de tous les liens (dont les liens conjugaux) et de tous les principes sur lesquels repose la société, d'où leur insistance sur la maîtrise des passions : contrôle des instincts, valorisation du jugement et de la raison, valorisation du sentiment du devoir, entraîne-ment à résister à la frustration, etc. Pour eux un amour ne méritait d'être qualifié de conjugal que lorsque s'étaient éteintes les éventuelles premières flambées du désir charnel. 

Erasme qui n'est suspect ni de pudibonderie ni d'ascétisme et qui tenait le mariage pour un choix de vie aussi méritoire que le célibat des moines et des clercs, ce qui était plutôt nouveau, n'en écrivait pas moins dans son \emph{En-commium matrimonii christani} (Eloge du mariage chrétien, 1526) : \emph{"... Les poètes appellent l'ardeur des amants une fureur et non point un amour. Car enfin, où la raison est éteinte, peut-il y avoir autre chose que de la folie ? ...ceux qui épousent des femmes imprudemment et sans jugement ont coutume de se repentir, mais quelquefois trop tard, de ces malheureux mariages. Il arrive rarement qu'on regrette de s'être marié par l'avis des parents, et d'avoir pris par un choix mûr et délibéré des femmes qu'on puisse aimer toujours. Car tout ce qui se fait en ne consultant que nos passions n'a qu'un temps. Ce qui se fait au contraire par raison et par jugement est stable et dure longtemps"}.  

 C'est un constat similaire que fait cinq siècles plus tard Maurice Godelier à l'issue de ses enquêtes ethnologiques. Selon lui le désir sexuel est fondamentalement asocial et même destructeur : \emph{« La permissivité en matière de sexe s'arrête [...], dans toutes les sociétés, soit là où la formule d'alliance serait menacée, soit là où les rapports de coopération et d'autorité entre consanguins risqueraient de s'effondrer et, glissant les uns dans les autres, de disparaître (Na). Mais cette fois, ce n'est plus de la sexualité-reproduction qu'il est question, mais de la sexualité-désir qui, nous l'avons vu, est dans son fond asociale. Elle n'est jamais la base d'une coopération durable entre les individus, tant au sein du groupe où ils sont nés qu'entre lui et les groupes avec lesquels il est allié. Et ce n'est pas seulement le désir hétérosexuel qui unit et divise. [...] C'est, nous l'avons dit, parce que le désir sexuel en lui-même est asocial qu'aucune société ne peut permettre que tout soit permis.} 
 
 \emph{Et ce travail d'auto-domestication est toujours à recommencer, alors que le processus de domestication des plantes et des animaux semble avoir atteint ses limites. [...] Partout la spontanéité du désir a dû être sacrifiée pour produire un ordre social qui est toujours en même temps un ordre entre les sexes et un ordre sexuel. Partout a dû être éliminé le caractère asocial de la sexualité, sacrifié le polymorphisme du désir, interdite la permissivité sexuelle généralisée pour que la société puisse s'organiser et se reproduire.}
 
\emph{[...] Cependant, sacrifier le caractère asocial de la sexualité n'est pas seulement un acte d'amputation.  C'est en même temps une sorte de création. C'est agir sur soi pour continuer non seulement à vivre en société, mais à produire de la société pour vivre, ce qui est le propre de l'homme et le séparera toujours davantage, chaque jour qui passe, des primates, ses lointains cousins. (p. 632-636)}

Malgré la réhabilitation du désir que l’oeuvre de Freud a initiée un ethnopsychiâtre d'aujourd'hui comme Tobie Nathan  ne dit pas autre chose : 
\emph{"Qu'est-ce qui différencie la passion de l'amour, notamment conjugal ? La passion, ce n'est pas l'amour. D'ailleurs, les Grecs avaient deux mots distincts. Philia signifie l'amour raisonnable - comme l'amour conjugal - ou l'amitié, tandis qu'Eros désigne le désir, la passion amoureuse. Platon la caractérisait par le manque. Obser-vation exacte, mais insuffisante. Il s'agit d'une exacerbation du manque - dans la passion, l'autre me manque quand il n'est pas là, il me manque quand il est là, car il n'y est jamais suffisamment ; dans la relation sexuelle, et même au moment de l'orgasme, il me manque encore. Recherche d'une fusion impossible, pulsion à offrir à l'autre tout votre espace intérieur - la passion amoureuse est une folie. La seconde caractéristique est qu'elle produit du changement, un bouleversement radical, et ce mou-vement n'est pas maîtrisable."}
 
 Lorsque Paul de Tarse évoque l'amour conjugal il ne pense pas à un sentiment mais à une tâche à réaliser (agape). Il s'agit de s'occuper de son conjoint comme de son propre corps et de chercher activement à lui plaire. Il s'agit de se dévouer pour son bien-être physique et mental, de pardonner ses défaillances et ses limites, et enfin « last but not the least » de lui reconnaître un droit exclusif sur son propre corps. 

Si le désir sexuel ne suffit pas pour attacher les amants au-delà de quelques années, est-ce qu'on pourrait au moins défendre l'idée qu'il faut s'en remettre à lui pour choisir celui ou celle avec qui fonder une communauté de vie durable et accueillir des enfants ? Mais il est de notoriété publique que les unions n'ont jamais été aussi fragiles que depuis que le mariage d’amour est proposé comme modèle et depuis que le pouvoir des parents d’arranger les mariages de leurs enfants a été brisé ! 


\chapter{La femme est-elle une aide pour l’homme (et réciproquement) ?} 


Si dans l’un de ses deux récits de création la bible ne pose pas de différence entre les deux sexes, l’autre définit la femme comme une aide pour l’homme, (à côté de lui, ou contre lui, selon les traducteurs), mais de toute façon comme un être complémentaire. C’est une chose que toutes les civilisations  ont compri-se de la même façon : partout les hommes ont trouvé normal d’exercer le pou-voir au sein des couples, et de jouer les chefs de famille. Partout ils ont trouvé  normal d’assumer les tâches nobles ou anoblissantes et de vouer les femmes aux autres.  Ils ont partout eu tendance  à  brider l’efficacité des femmes en ne leur accordant pas l’accès aux meilleurs outils et en les tenant à l’écart du savoir, en ne leur reconnaissant une place qu’au sein du foyer, auprès des enfants et dans les activités ménagères.

Nietzsche affirmait dès 1888 que la dénaturation du mariage était en cours. Il stigmatisait l'importance donnée (dès son époque) au mariage d'amour mais il incriminait d’abord et avant tout l'égalité des époux :
\emph{« On vit pour aujourd'hui, on vit très vite -- on vit sans aucune responsabilité : c'est précisément ce que l'on appelle « liberté ». Tout ce qui fait que les institutions sont des institutions est méprisé, haï, écarté : on se croit de nouveau en danger d'esclavage dès que le mot « autorité » se fait seulement entendre [...] Témoin : le mariage moderne. Apparemment toute raison s'en est retirée : pourtant cela n'est pas une objection contre le mariage, mais contre la modernité. La raison du mariage -- elle résidait dans la responsabilité juridique exclusive de l'homme : de cette façon le mariage avait un élément prépondérant, tandis qu'aujourd'hui il boite sur deux jambes. La raison du mariage -- elle résidait dans le principe de son indissolution : cela lui donnait un accent qui, en face du hasard des sentiments et des passions, des impulsions du moment, savait se faire écouter. Elle résidait de même dans la responsabilité des familles quant au choix des époux. Avec cette indulgence croissante pour le mariage d'amour on a éliminé les bases mêmes du mariage, tout ce qui en faisait une institution. Jamais, au grand jamais, on ne fonde une institution sur une idiosyncrasie ; je le répète, on ne fonde pas le mariage sur « l'amour », -- on le fonde sur l'instinct de l'espèce, sur l'instinct de propriété (la femme et les enfants étant des propriétés), sur l'instinct de la domination qui sans cesse s'organise dans la famille en petite souveraineté, qui a besoin des enfants et des héritiers pour maintenir, physiologiquement aussi, en mesure acquise de puissance, d'influence, de richesse, pour préparer de longues tâ-ches, une solidarité d'instinct entre les siècles. Le mariage, en tant qu'institution, com-prend déjà l'affirmation de la forme d'organisation la plus grande et la plus durable : si la société prise comme un tout ne peut porter caution d'elle même jusque dans les générations les plus éloignées, le mariage est complètement dépourvu de sens. -- Le mariage moderne a perdu sa signification -- par conséquent on le supprime. »} (Le Crépuscule des idoles, 1888) 

La hiérarchie comme ciment du mariage ? Mais qu’est ce qui empêche la subordonnée de s’en aller (ce serait aussi vrai dans l’autre sens) ? Qu’est ce qui la contient ? Le manque de ressources personnelles ? La pression sociale et la peur de l’exclusion ? La peur de perdre ses enfants ? 

En même temps que le mariage d’amour triomphait dans les représentations s’effondraient l’un après l’autre la quasi-totalité des contreforts qui naguère étayaient le lien conjugal, que ce soient les lois (l'indissolubilité du mariage « constantinien » n’a été a peu près respectée que sous la pression d’un encadrement juridique patiemment construit et vigoureusement défendu pendant plus d'un millénaire), l'intérêt matériel (le lien conjugal est plus fragi-le lorsque l'épouse peut se procurer des ressources propres hors du foyer), la pression familiale ou sociale (les divorcés et leurs enfants ne sont plus stigma-tisés et leurs nouveaux partenaires sont de plus en plus souvent reçus par leurs parents à égalité avec les précédents), l’impossibilité d’obtenir des héritiers hors mariage régulier (tous les enfants peuvent aujourd’hui être reconnus par leurs deux parents quel que soit le statut matrimonial de ces derniers, sauf inceste, et tous, même ces derniers, peuvent hériter à égalité) les croyances religieuses (les morales religieuses traditionnelles paraissent désuètes et inadaptées). Lorsqu'il ne reste plus pour relier les conjoints que le souci d’élever leurs enfants c'est trop souvent insuffisant pour donner du sens à une vie en commun.

Les femmes ne retourneront plus dans des gynécées, sinon contraintes et forcées (par qui ? À  part elles-mêmes ?) et l'égalité entre les hommes et les femmes s'impose comme un principe de base indiscutable. Il faudra donc inventer (ou découvrir, ou redécouvrir) pour les hommes une place dans les familles qui soit sinon aussi désirable que celle des femmes, au moins suffi-samment désirable pour qu'ils s'y impliquent. 

Dans un environnement allergique à tout ce qui ressemble à du paternalisme, qu'est-ce qu'un homme est autorisé à désirer concernant des enfants ? Des points de vue et des désirs spécifiquement masculins sur les enfants sont-ils acceptables ? Il faudrait sans doute commencer  par accepter l’idée qu'il peut exister quelque chose comme des valeurs masculines, ou une manière masculine de faire vivre les valeurs universelles. Certes \emph{« ...dans la perspective proféministe, on ne peut vouloir à la fois que le genre disparaisse comme système hiérarchique et que les catégories du masculin et du féminin continuent d'exister. Mais pour d'autres auteurs, le terme de masculinité marque la volonté d'analyser s'il est possible d'être un homme sans coller aux stéréotypes de la virilité, d'une part ; sans devenir une femme, d'autre part . »} Virilité défensive, masculinité créatrice, in Travail, genre et sociétés
2000/1 (N° 3), parPascale Molinier

Il est possible que la présence de couples d’hommes avec enfants renouvelle l'abord de ces problèmes ? De même la coexistence de couples d’hommes, de femmes et de couples mixtes permettra une approche moins stéréotypée des dynamiques au sein des couples.

\chapter{Qui veut des enfants ?} 

 
 A qui appartiennent les enfants ?  Ils ne s'appartiennent pas à eux-mêmes, sauf à supprimer le statut de mineur. On ne peut pas non plus dire qu'ils n'appartiennent à personne. Du point de vue des enfants, n'appartenir à personne (ou appartenir à une institution d'assistance publique) c'est être abandonné. 
 
 Depuis l’antiquité tardive les enfants n'appartiennent plus seulement à leurs pères. Est-ce qu'ils appartiennent aux deux parents, comme dit la loi ? ...ou bien plus à leur mère qu’à leur père ? …ou bien à l'ensemble de ceux qui les élèvent en leur don-nant leur argent et leur temps, dont les beaux-pères et belles-mères ? ...ou bien encore à l'État ? 
 
Le désir de serrer un bébé de chair dans ses bras est d'autant plus ir-répressible que ses motivations les plus vraies sont inconscientes. Il est sans doute aussi répandu et aussi fort aujourd'hui que par le passé et il ne concerne pas seulement les femmes. Pourtant, si l'on en croit Coluche, « y a des gens qui ont des enfants parce qu'ils n'ont pas les moyens de s'offrir un chien ». Il posait à sa manière une question essentielle et relativement nouvelle : pourquoi fait-on des enfants ? Pour quoi veut-on des enfants ? À quoi servent les enfants ?

 Si ous les citoyens des pays dotés d'un bon système d'assistance sociale et de caisses de retraite suffisantes ont besoin qu'il naisse des enfants pour financer leurs périodes d'invalidité et leurs vieux jours, aucun d'eux n'a besoin que ce soient ses propres enfants : c'est précisément pour cela que ces systèmes ont été mis en place. Dans les pays les plus socialement développés, seule la collectivité a besoin d'enfants. D'un point de vue strictement comptable et sauf dispositifs de compensation très généreux des frais qu'entraînent ces derniers l'intérêt financier bien compris des citoyens des États providence est de ne pas en avoir. Leur niveau de vie et leur crédit auprès de leur banquier seront plus élevés s'ils évitent d'investir dans une progéniture. Il ne faut sans doute pas chercher plus loin la faiblesse des taux de natalité de leurs membres, taux qui ne sont que la résultante des stratégies individuelles de leurs citoyens, stratégies d'autant plus rationnelles qu'on ne voit plus aujour-d'hui au nom de quelle exigence morale on pourrait les leur reprocher. 
 
Face à la désaffection du mariage et de la procréation qui menaçait la survie de l'Empire Romain, l'empereur Auguste avait réagi en pénalisant les célibataires et ceux qui n'avaient pas d'enfants, et ses lois ont été appliquées sans faillir pendant au moins trois siècles. Si nos taux de natalité baissaient dangereusement, verrions-nous à l'avenir de pareilles incitations légales  à procréer ? 

 Mais les malthusiens et avec eux bien des écologistes pensent que les problèmes de santé de notre planète ont pour origine le fait que les humains sont trop nombreux. Il faudrait en effet que le nombre de ces derniers diminue drastiquement s'ils voulaient tous consommer comme les citoyens des pays développés actuels sans épuiser les ressources disponibles et sans mettre en danger les équilibres de la nature. Cela impliquerait non pas une croissance démographique zéro, mais une décroissance très énergique. L'intérêt commun de l'humanité serait-il sa décroissance numérique et donc l'évitement de la reproduction jusqu'au retour à un effectif écologiquement optimal ? 

 \chapter{Ringardisation de la fidélité ?} 
 
 Depuis qu'il n'y a plus de différences entre les enfants nés dans le mariage et les autres, les unions libres se sont multipliées de façon exponentielle. Elles existaient déjà auparavant, mais surtout dans des groupes sociaux peu ou pas concernés par les questions d’héritage. Il n'est plus nécessaire d'épouser pour avoir des héritiers, et l'intérêt du mariage diminue au fur et à mesure qu'au nom de l'égalité les lois étendent aux non-mariés les droits accordés aux mariés. Autrement dit l’institution du mariage servait à créer des différences, et si elle n’en crée  plus elle s'étiole.
 
Et pourquoi vouloir que les couples soient durables ? pourquoi pas une succession d'amours éphémères ?  ...ou pas d'amours du tout ? Ce n'est plus que que par habitude que la loi prescrit aux conjoints d'être fidèles, mais elle ne prévoit plus guère de sanction à l'encontre des infidèles. Après tout il n'est même plus nécessaire que les sexes se rencontrent pour faire un enfant, ni d'être deux pour l'élever. 

En commentaire des polémiques suscitées par le projet de loi ouvrant le mariage aux personnes de même sexe, Jacques ATTALI esquisse l'avenir qui, compte tenu des évolutions récentes dans les pratiques familiales, repro-ductives et sexuelles, lui paraît le plus probable   : 
\emph{« Comme toujours, quand s'annonce une réforme majeure, il faut compren-dre dans quelle évolution de long terme elle s'inscrit.
Et la légalisation, en France après d'autres pays, du mariage entre deux adultes homosexuels, s'inscrit comme une anecdote sans importance, dans une évolu-tion commencée depuis très longtemps, et dont on débat trop peu : après avoir connu d'innombrables formes d'organisations sociales, dont la famille nucléaire n'est qu'un des avatars les plus récents, et tout aussi provisoire que ceux qui l'ont précédé, nous allons lentement vers une humanité unisexe, où les hommes et les femmes seront égaux sur tous les plans, y compris celui de la procréation, qui ne sera plus le privilège, ou le fardeau, des femmes.}

\emph{1. La demande d'égalité. D'abord entre les hommes et les femmes. Puis entre les hétérosexuels et les homosexuels. Chacun veut, et c'est naturel, avoir les mêmes droits: travailler, voter, se marier, avoir des enfants. Et rien ne résistera, à juste titre, à cette tendance multi-séculaire. Mais cette égalité ne conduit pas nécessairement à l'uniformité : les hommes et les femmes restent différents, quelles que soient leurs pré-férences sexuelles.}

 \emph{2. La demande de liberté. Elle a conduit à l'émergence des droits de l'hom-me et de la démocratie. Elle pousse à refuser toute contrainte ; elle implique, au-delà du droit au mariage, les mêmes droits au divorce. Et au-delà, elle conduira les hom-mes et les femmes, quelles que soient leurs orientations sexuelles, à vouloir vivre leurs relations amoureuses et sexuelles libres de toute contrainte, de tout engagement. La sexualité se séparera de plus en plus de la procréation et sera de plus en plus un plai-sir en soi, une source de découverte de soi, et de l'autre. Plus généralement, l'apologie de la liberté individuelle conduira inévitablement à celle de la précarité ; y compris celle des contrats. Et donc à l'apologie de la déloyauté, au nom même de la loyauté : rompre pour ne pas tromper l'autre. Telle est l'ironie des temps présents : pendant qu'on glorifie le devoir de fidélité, on généralise le droit à la déloyauté. Pendant qu'on se bat pour le mariage pour tous, c'est en fait le mariage de personne qui se gé-néralise.}
 
\emph{3. La demande d'immortalité, qui pousse à accepter toutes mutations sociales ou scientifiques permettant de lutter contre la mort, ou au moins de la retarder.}

\emph{4. Les progrès techniques découlent en effet de ces valeurs, et s'orientent dans le sens qu'elles exigent : en matière de sexualité, cela a commencé par la pilule, puis la procréation médicalement assistée, puis la gestation pour autrui. Ces questions de bioéthique ne découlent évidemment pas des demandes d'égalité venant des couples homosexuels et concernent toutes les formes de reproduction, y compris -- et surtout -- « hétérosexuelles ». Le vrai danger viendra si l'on n'y prend garde, du clonage et de la matrice artificielle, qui permettra de concevoir et de faire naitre des enfants hors de toute matrice maternelle. Et il sera très difficile de l'empêcher, puisque cela sera tou-jours au service de l'égalité, de la liberté, ou de l'immortalité.}

\emph{5. La convergence de ces trois tendances est claire : nous allons inexorable-ment vers une humanité unisexe, sinon qu'une moitié aura des ovocytes et l'autre des spermatozoïdes, qu'ils mettront en commun pour faire naitre des enfants, seul ou à plusieurs, sans relation physique, et sans même que nul ne les porte. Sans même que nul ne les conçoive si on se laisse aller au vertige du clonage.}

\emph{6. Accessoirement, cela résoudrait un problème majeur qui freine l'évolution de l'humanité: l'accumulation de connaissances et des capacités cognitives est limitée par la taille du cerveau, elle-même limitée par le mode de naissance: si l'enfant naissait d'une matrice artificielle, la taille de son cerveau n'aurait plus de limite. Après le passage à la station verticale, qui a permis à l'humanité de surgir, ce serait une autre évolution radicale, à laquelle tout ce qui se passe aujourd'hui nous prépare. Telle est l'humanité que nous préparons, indépendamment de notre sexualité, par l'addition implicite de nos désirs individuels...}

\emph{Alors, au lieu de s'opposer à une évolution banale et naturelle du mariage laïc, qui ne les concerne pas, les Eglises devraient plutôt se préoccuper de réfléchir, avec les laïcs, à ces sujets bien plus importants : comment permettre à l'humanité de définir et de protéger le sanctuaire de son identité ?}

\emph{Comment poser les barrières qui lui permettront de ne pas se transformer en une collection d'artefacts producteurs d'artefacts ?}

\emph{Comment faire de l'amour et de l'altruisme le vrai moteur de l'histoire ? »}

Cela n'a rien d'évident en effet. On aura remarqué que Jacques ATTALI ne décrit pas tant ce qu'il désire que ce qu'il prévoit dans l'hypothèse où les dynamiques en cours se prolongeraient sans changement, et il n'exclut pas formellement l'idée que l'avenir qu'il décrit, s'il se réalisait, pourrait n'être pas totalement radieux. 

Malgré tout le mal que semble en penser Nietzsche, l'exigence d'égalité absolue entre hommes et femmes est un acquêt de notre temps sur lequel il est infiniment peu probable que l'on revienne. Avec quels arguments pourrait-on en effet défendre de manière convaincante (désirable pour tous) une inégalité fondée sur le sexe ou sur le genre, que le bénéficiaire de cette inéga-lité soit né mâle ou femelle ? Cette exigence d'égalité est d'ailleurs encore loin d'avoir produit tous ses effets, directs et indirects. 

 Par contre l'exigence de liberté individuelle absolue, à tout prix et quelles qu'en soient les conséquences est grosse des problèmes pointés par Jacques Attali. Peut-on croire qu'il pourrait exister un domaine de l'existence dans lequel aucun engagement n'aurait d'importance, où aucune parole ne vaudrait rien, et que cela n'entraînerait pas de répercussions significatives dans les autres domaines ? …dans les autres conversations ? …dans les autres relations ? D’autant plus qu'il s'agit d'un domaine charnellement lié à la construction par chacun de son identité ? Peut-on croire que cela n'aurait aucun effet en termes de « lien social » ? Si entre les individus aucune promesse ne vaut, alors il n’est pas impossible qu’il ne reste que la sauvagerie des rapports de force (« brutalisation » des rapports interpersonnels ?). 
 
 Quant à la valorisation de l'immortalité individuelle, c'est une autre forme de l'individualisme. Contrairement au refus de la mort d'autrui, le refus à n'importe quel prix de la mort de soi met l'individu au-dessus de tout, au-dessus de tous les autres et des liens avec eux.
 
Dans ce contexte \emph{« Comment faire de l'amour et de l'altruisme le vrai moteur de l'histoire ? »} Reste-t-il même une place pour l'amour et l'altruisme ? Est-ce que le sacrifice de soi pour un autre ou pour une cause (la justice, la vérité, etc..) garde encore un sens ? 

Face à ces évolutions sur quoi fonder des jugements ? Les préceptes religieux (ex. : \emph{« Il n'y a pas de plus grand amour que de donner sa vie pour ceux qu'on aime. »} Jn 15, 13, etc..) ne convainquent que ceux qui y croient. 
Les morales laïques sont variées et contradictoires : d'un côté (en perte de vitesse) on valorise l'engagement et la fidélité à la parole donnée, de l'autre (qui a le vent en poupe) on plaide pour l'authenticité et la liberté. 
Le critère le plus objectif et le moins discutable serait peut-être l'utilité sociale ? D’un point de vue collectif qu'est-ce qui est préférable ? la fidélité aux engagements ou la liberté d'être à chaque instant en accord avec ses désirs du moment ?  

 A défaut d'enquêtes de satisfaction il serait peut-être possible de chiffrer l'ensemble des coûts et des bénéfices directs et indirects comparés de la fidélité et de l'authenticité : coûts financier, incidence sur les santés physique et mentale, problèmes d'ordre public, etc... 
On pense entre autres à tous les dépenses engendrées par le « démariage » sur le marché immobilier, sur celui de l'équipement domestique ou sur celui de l'assistance juridique, etc. La traduction concrète de ces coûts c'est la pauvreté qui frappe beaucoup de divorcés ou de parents célibataires : femmes seules avec enfants, mais aussi hommes seuls dont les ressources sont insuffi-santes pour retrouver un logement, etc. 
Il y a des avantages certains à vivre en couple (un seul loyer, un seul équipement domestique, une seule voiture familiale,  etc..) et c'est d'autant plus vrai qu'on est moins à  l’aise, tandis que les séparations font perdre cet avantage et sont parfois ruineuses. Il faut y ajouter les coûts psychologiques, qui sont en partie inséparables des coûts matériels, ainsi que le note Gérard Neyrand  :  
\emph{« En effet, les nouvelles valeurs familiales sont portées par les couches moyennes cultivées et sont devenues système de référence global. Leur confrontation aux habitus des couches populaires en la matière ne s'effectue pas sans conflits (Commaille, Martin, Les enjeux politiques de la famille, Paris Bayard, 1998). L'une des issues des contradictions entre ces systèmes différents de références, qui tra-versent différemment les individus selon leur sexe et leur position sociale, réside dans la fréquence des séparations conjugales conflictuelles, la mono-parentalisation maternelle qui s'en suit et la précarisation des foyers mono-parentaux ainsi définis. Leur caractéristique est bien d'être soumis à un double système de contraintes croisées, socio-économiques et psycho-relationnelles.
La montée du chômage et la précarisation des emplois les moins qualifiés\footnote{ Boltanski, Chiapello, Le nouvel esprit du capitalisme, Paris Gallimard, 1999}, contribuent à une fragilisation globale des situations familiales des plus démunis, qui risque d'autant plus de déstabiliser les familles que ces familles populaires se pensent de façon unitaire, quasi-symbiotique.
Elles sont basées sur un couple conçu comme une entité indissoluble, un « couple unité organique » selon l'expression d'Irène Théry\footnote{Le couple occidental et son évolution sociale : du couple « chainon » au couple « duo », Dialogue, \no 150, 4e trimestre, 2000}, et sont loin d'adhérer sans réserve au nouveau modèle mo-derne du « couple duo ». La séparation, dès lors, constituera une catastrophe identitaire dont beaucoup auront du mal à se relever, en particulier les pères.
On conçoit alors l'importance des difficultés que des séparations dans un tel contexte peuvent générer :
- difficultés relationnelles entre les ex-conjoints et dans le rapport des pères à leurs enfants,
- et difficultés socio-économiques des mères confrontées aux nécessités d'une survie familiale qu'elles doivent bien souvent affronter seules.
Mono-parentalisation et précarisation s'avèrent alors intimement liées. »}


la dissolution des couples d'amants ne va évidemment pas dans le sens du renforcement des capacités éducatives des couples de parents, et il est un peu forcé sinon tout à fait artificiel de les distinguer. Cette dissolution multiplie le nombre des situations où la fonction éducative de l'un ou de l'autre des parents est plus ou moins disqualifiée ou entravée, tandis que son remplacement au quotidien par le ou les partenaire sexuels et affectifs de l'autre n'est pas forcément bien accepté par les enfants concernés et ne présente pas toujours l'efficacité éducative nécessaire. Le nombre s'élève donc des parents qui face aux inévitables problèmes éducatifs que posent tous les mineurs sont seuls et/ou en difficulté. Lorsque leurs enfants leur posent problème, notamment à l'adolescence, des aides extérieures sont souhaitables, mais rien ne garantit que l'efficacité des diverses aides éducatives ainsi apportées aux parents soit supérieure à ce qu'en d'autres circonstances ils auraient pu assumer eux-mêmes : ce serait déjà bien si on pouvait être assuré qu'elle n’est pas trop souvent inférieure. D'autre part même en tenant pour négligeable la disqualification et la mise en dépendance des parents par les spécialistes du contrôle social des familles et les experts de l'éducation, les soutiens que propose la collectivité ne sont pas gratuits (ex. :  internats scolaires, assistance éducative, placement en famille d'accueil,~etc.). On passe de l'auto production à l’externalisation et à la professionnalisation des tâches éducatives. Comme c'est un domaine où il n'y a à espérer aucun gain de productivité cela accroît les coûts éducatifs de manière très sensible. Jusqu'où peut-on aller dans cette voie avant que les électeurs n'estiment que c'est trop cher payé ?
 
En conclusion il serait peut-être difficile de prouver chiffres en main que le mariage traditionnel est supérieur à toutes les autres manières d'organiser la vie des individus, mais il est au moins aussi difficile de croire que la stabilité des unions fécondes et la fidélité des parents l’un à l’autre n'ont aucun intérêt du seul point de vue de la société et sont définitivement dépassés. 

Par contre du point de vue des entreprises et des administrations la disponibilité d'un(e) employé(e) est plus grande lorsqu'il n'est plus nécessaire de tenir compte de son désir de vivre avec un(e) partenaire lui-même (elle-même) bien inséré(e) professionnellement et qui lui (elle) aussi s'attend à être significativement investi(e), de même qu'un(e) célibataire sans enfants et sans intention d'en avoir est précieux(se) pour son employeur et possède un atout significatif pour réussir une belle "carrière" professionnelle.
 
\chapter{Nouvelles familles ?} 

Les lois qui autorisent le divorce par consentement mutuel, la contraception et l'avortement ont dénaturalisé le modèle de famille traditionnel. Même si rien ne pourra jamais empêcher personne de croire à la valeur éthique, éducative ou civique de la sainte famille, fondée sur un couple hétérosexuel, monogame et indissoluble, élevant lui-même les enfants nés de ses œuvres, ce modèle n'est plus qu'un parmi d'autres également légitimes, comme c'était le cas avant les décisions de Constantin. Dans la mesure où ce modèle n'est plus étayé par la loi et la puissance publique (comme il l'était par l'Ancien Régime ou le Code Napoléon) la famille traditionnelle ne correspond plus à ce que j'ai appelé la famille constantinienne. Il ne s’agit plus que de l'une des façons "post constantiniennes" d'organiser la reproduction humaine et la vie en commun, une parmi d'autres, dont la paternité, la maternité ou l'adoption célibataires, le concubinage, le PACS ou le mariage homosexuel, etc.

Dans \emph{Quelle alternative au patriarcat ? Valoriser un modèle social non conjugal} (2004), Agnès ECHENE accuse le couple hétérosexué d'être le lieu privilégié d'expression et de transmission de la violence machiste, et cela trop souvent avec la complicité (masochiste) féminine. Elle en tire la conclusion qu'il faut éliminer la paternité en tant que telle :
\emph{« ...ce n'est qu'en valorisant le modèle social non conjugal qu'une société peut se défaire du patriarcat. Il importe donc de favoriser une sexualité libre et variée, tout en étant discrète et protégée, surtout chez nos propres enfants ; peu importe dès lors qu'elle soit ardente ou paisible, monotone ou changeante, homophile ou hétérophile, dès l'instant qu'elle reste une affaire personnelle dont nul ne se mêle. Une telle évolution nécessite également une reconsidération du modèle familial qui doit se re-fonder sur des liens d'appartenance utérine et non pas consanguine ; cela remet en cause dès lors la paternité génitale qui doit laisser place à une paternité germaine : il faut en effet que ce soit les frères, oncles et cousins [de la mère] qui assument les enfants des femmes ; de nombreux signes avant-coureurs montrent qu'ils sont prêts à le faire et qu'il ne manque qu'un déclic. Mais il faut aussi que les femmes renoncent à obliger les géniteurs à être pères ; il faut qu'elles abandonnent toute velléité de recher-che de paternité, de pension, partage, alternance,~etc. et se tournent résolument vers leurs frères, oncles et cousins pour « donner » des pères à leurs enfants, qui ne s'en porteront pas plus mal. »}
On est là à très peu de choses près dans le monde de l'ethnie Na et des autres groupes, chinois de l’ouest ou thibétains, qui ne connaissent pas de pères, et où les hommes de chaque famille sont les amants librement choisis pour une nuit ou pour plusieurs des femmes des familles voisines. Ces familles reposent toutes sur un principe matriarcal, puisque les enfants appartiennent exclusivement à la famille de leur mère. Ce sont donc les oncles maternels qui "paternent" les enfants de leurs sœurs. Ceci dit il ne suffit pas que les femmes aient le choix du géniteur de leurs enfants pour que l'autorité dans le groupe familial leur appartienne, même si c'est une brèche dans le pouvoir symbolique des hommes. 

D'autres imaginent des constellations d'une tout autre espèce, des associations basées sur des contrats de solidarité privés ne se référant plus au couple, mais plutôt aux communautés créées dans les années 60-70 du XXème siècle. Selon Marcella Jakub  \emph{"A l'époque de la discussion sur le pacs, certains avaient proposé de créer des liens de solidarité entre plusieurs individus, et pas uni-quement au sein du couple, qu'il soit hétérosexuel ou homosexuel. Le pacs aurait pu permettre, par exemple, d'associer des personnes au moyen de liens juridiques alter-natifs qui ne soient pas forcément fondés sur la famille. Voilà une proposition sociale intéressante, qui aurait permis d'inventer des formes de vie à plusieurs. Mais nous sommes loin d'une telle réflexion ».}

 Va-t-on de manière plus banale vers des foyers constitués d'une femme et des enfants qu'elle a mis au monde, autour desquels graviterait la nébuleuse de ses amants et ex-amants ? Dans cette hypothèse, les hommes de demain auraient des enfants de plusieurs femmes, enfants vivant ordinaire-ment chez leurs mères, si bien que leur autorité sur chacun d'eux serait prati-quement nulle ? Serait-ce en quelque sorte l'inverse de la situation du pater familias romain, qui pouvait demander à plusieurs femmes des enfants sur lesquels lui seul avait autorité et qui vivaient tous chez lui s’il le voulait ?
  
 On reconnait là le modèle matrifocal « antillais » ou « caraïbe » dont l'origine se situe dans l'histoire du peuplement des Antilles. On a vu que les esclaves n'ont par définition aucun des attributs juridiques d'un père ou d'une mère sur les enfants dont ils sont les géniteurs : seuls les propriétaires des génitrices possèdent des droits sur les enfants de celles-ci. C'étaient ces proprié-taires qui faisaient d'elles des mères lorsqu'ils leur confiaient la garde des enfants qu'elles avaient portés, quel qu'en ait été le géniteur. Il y avait une espèce d'alliance de fait (alliance sous contrainte, perverse) entre les génitrices et leurs maîtres pour élever les enfants qu'elles avaient mis au monde, tandis que leurs partenaires sexuels n'avaient pas droit à la parole et étaient réduits, quel qu'ait pu être leur désir,  à n'être que des donneurs de sperme. 
Le rôle traditionnel des pères était de soutenir matériellement et psy-chologiquement les mères dans leur tâche éducative, d’introduire au sein de la dyade mère-enfant les exigences du monde extérieur et d’aider leurs enfants à  y trouver une place. Faut-il comprendre (faut-il craindre ?) qu'à l'avenir ce pourrait être l'État qui, à l'aide de ses services judiciaires et médico-socio-educatifs et de toutes les prestations qu'il peut fournir, assumerait ce rôle ?
 
 \chapter{Filiation adoptive ou « biologique » ?} 
 
 Le "mariage constantinien" tel que je l'ai défini télescopait sur le couple des seuls géniteurs (unis de manière socialement reconnue) toutes les dimensions de la conjugalité et de la parentalité (juridique, biologique, affective et éducative) et frappait tout le reste d'illégitimité et notamment la filiation élective, volontaire, et même adoptive. C'est au nom de cette dernière que l'hégémonie de notre tradition juridique est aujourd’hui théoriquement contestée, et cette remise en question est, comme on peut s'y attendre, consubstantiellement liée à la promotion de nouvelles formes de conjugalités. Selon Daniel BORRILLO  les possibilités nouvelles de dissociation entre sexualité et reproduction qui se sont ouvertes grâce aux progrès de la biologie en à peine une génération ont provoqué une panique morale, qui a conduit les théoriciens et les praticiens du Droit à survaloriser les liens biologiques géni-teurs-enfants :
  
\emph{« La biologie commença à devenir ainsi le soubassement réel ou symbolique du système de parenté, à rebours d'une science juridique qui avait plutôt instauré la volonté au cœur de ce système [...] À partir des années 1990, l'expertise biologique  s'est imposée dans les procès en contestation de paternité, la recherche des origines est revendiquée socialement comme droit fondamental de la personne, la différence de sexe est devenue une valeur [...] La nouvelle place prépondérante de la vérité biologi-que dans l'établissement du lien filial fut confirmée en France par la Cour de cassa-tion  ...Par là, la distinction traditionnelle entre reproduction (fait biologique) et fi-liation (fait culturel), fondement du droit civil moderne, se trouvait questionnée... non pas à partir d'arguments classiques provenant du droit canonique, mais par une rhé-torique qui, d'une part, fera de la différence des sexes une condition sine qua non de la filiation, et, d'autre part, placera l'expertise sanguine et la preuve d'ADN au cœur du dispositif juridique de la parenté.}

\emph{[...] La filiation peut certes tenir compte du fait naturel, mais, en tant que dispositif d'agencement parental, elle répond à des règles propres, affranchies de la nature [...] Elle n'existe que lorsqu'elle est établie dans les conditions et selon les mo-des prévus par la loi. Autrement dit, la filiation est déterminée par la norme juridique et non par la nature. Ce lien juridique se tisse à partir de quatre fils principaux : la biologie (filiation par le sang), la volonté (adoption), la présomption (paternité suppo-sée du mari de la mère) et le vécu (appelé en droit « possession d'état »).}

\emph{Ce qui compte ce ne sont plus tant les racines naturelles ou surnaturelles d'institutions intangibles que l'efficacité et la plasticité d'instruments juridiques pro-curant tel ou tel résultat (par exemple la paix des familles ou la solidarité des générations).  [...] fondée sur la volonté, l'adoption est une institution plus apte que la vérité biologique à assurer la stabilité des liens familiaux.}

 \emph{[...]La contestation actuelle de l'ordre familial « naturel » n'est en définitive que la radicalisation de l'idéologie individualiste moderne, selon laquelle la volonté et non la différence des sexes constitue la base de l'institution matrimoniale et parentale. Une filiation dissociée de la reproduction permettra de justifier un système juridique fondé non pas sur la vérité biologique, mais sur le projet parental responsable. De ce point de vue, peu importe l'agencement familial (traditionnel, monoparental, homo-parental, recomposé...), si les prémisses du contrat (égalité dans l'alliance et dans la filiation) sont respectées jusque dans leurs moindres effets. L'État devrait donc traiter sur un plan d'égalité l'ensemble des familles et les autres formes d'intimité.}
 
\emph{Contrairement à la filiation charnelle, la filiation choisie trouve son principe dans la liberté non seulement d'accueillir les enfants des autres, mais également d'abandonner ses propres enfants biologiques, ce qui est pour l'heure uniquement pos-sible pour les femmes (accouchement sous X), mais devrait pouvoir s'élargir aussi aux hommes à travers une déclaration formelle de renoncement à la paternité. La généralisation de la filiation adoptive (y compris pour ses propres enfants biologiques) per-mettrait aussi de mettre la volonté au cœur du dispositif parental. Celui-ci reposerait exclusivement sur la volonté du ou des géniteurs qui donnent l'enfant et celle du ou des adoptants qui l'accueillent. De surcroît, l'adoption est une institution conçue à partir du droit de l'enfant à avoir une famille, contrairement à la filiation biologique qui apparaît plutôt comme un dispositif du droit à l'enfant ». }

 Que tous les systèmes juridiques soient des constructions humaines et non des faits de nature, qu'ils reposent sur des idéologies, sur des prises de positions morales et des croyances plus ou moins partagées, et qu'ils fassent tous des choix entre des possibles, acceptant les uns et refusant les autres, cela est évident. Mais sur quels arguments se fonde l'affirmation que \emph{"L'adoption est une institution plus apte que la vérité biologique à assurer la stabilité des liens familiaux"} ?  On peut tout aussi bien affirmer le contraire. Il n'est pas nécessaire que la filiation adoptive soit meilleure que la filiation ordinaire pour etre reconnue par le droit. A devoir choisir entre le sang ou la volonté pour fonder le droit de la filiation il y a quelque chose qui paraît artificiel et forcé. Tout faire reposer sur la biologie est certes méconnaître qu'elle n'a jamais suffit dans aucune société pour légitimer une naissance, et oublier combien le lien entre un adulte et son enfant est un lien co-créé dans le cadre de leur relation, à l'instar d'une adoption réciproque qui déborde de tous côtés la proximité biologique. Mais ne reconnaitre que la volonté en déniant les corps et leurs dialogues est une fiction juridique, héritée des romains, qui comme toutes les fictions juridiques fait plus ou moins violence aux réalités telles qu'elles sont vécues au jour le jour, dans la complexité et l’ambivalence.
 
 \emph{"L'adoption est une institution conçue à partir du droit de l'enfant à avoir une famille, contrairement à la filiation biologique qui apparaît plutôt comme un dispositif du droit à l'enfant."} Il est vrai qu'en France l'adoption des jeunes enfants, formellement interdite depuis la fin de l'antiquité, a été ressuscitée après la Grande Guerre pour donner des parents à des enfants abandonnés. Il est vrai aussi que pour chaque enfant adoptable il y a actuellement plusieurs candidats à l'adoption. Au plan mondial il en est de plus en plus de même : l'enfant naturel devient de plus en plus rare. Il n’est donc pas nécessaire de promouvoir l’adoption : du point de vue des enfants adoptables elle se porte plutôt bien. La valorisation actuelle de l'adoption vient non de la prise en compte de l'intéret des enfants sans parents mais de la prise en compte du désir d'enfant de ceux qui ne peuvent ou ne veulent pas recourir aux relations hétérosexuelles, désir d'enfant qui en soi n'est ni plus ni moins légitime que celui des autres.  
 
 Mais dès que les enfants sont sortis de la petite enfance, leur adoption n'est pas simple et elle peut être terriblement éprouvante pour le narcissisme des adoptants. Les adoptés courent bien plus que les autres enfants le risque d'être rejetés à cause des difficultés de tous ordres qu'ils ont rencontrées du fait de leur histoire et auxquelles ils se sont adaptés comme ils ont pu. Leurs attentes ne s'engrènent pas toujours harmonieusement avec celles de ceux qui se proposent de devenir leurs parents adoptifs : le pourcentage de ces enfants qui sont abandonnés une deuxième fois après une adoption n'est malheureusement pas négligeable. Tous les adultes ne sont pas prêts à prendre de pareils risques. 
Jusqu'ici il n'a pas été permis par la loi de concevoir des enfants pour les donner, mais si l'on voulait répondre à toutes les demandes d'adoption il faudrait s'y résoudre. 

Tant que dureront les énormes inégalités de revenu ob-servables sur cette planète, les plus fortunés pourront toujours louer le ventre des plus belles et des plus saines des filles des pauvres, de la même façon que les riches romains achetaient les plus jolies des jeunes esclaves afin qu'elles leur fassent des enfants bien à eux qu'ils n'auraient à partager ni avec un partenaire égal à eux en dignité, ni avec une belle famille aussi puissante que la leur. Le recours à des mères porteuses est dans la logique des évolutions libérales actuelles. Il est d'ores et déjà légalement possible dans plusieurs pays développés. Est-il appelé à se généraliser ? Comment refuser ce recours aux hommes homosexuels si l'on accorde l'assistance médicale à la procréation (PMA) aux femmes homosexuelles, et comment le refuser à tous les autres, hommes et femmes célibataires, si on l'accorde aux hommes homosexuels ? Et comment le refuser à qui que ce soit si des femmes (ordinairement de condition modeste et vivant souvent dans des pays sous-développés) sont volontaires pour prêter leur ventre et abandonner leur enfant nouveau-né contre une indemnité suffisante. 

 C'est le seul moyen de mettre les hommes à égalité avec les femmes dans l'accès à l'enfant, ou plutôt de corriger l'inégalité que leur corps leur impose dans ce domaine, mis à part bien sûr le mariage traditionnel, monogame et indissoluble, dont c'était l'une des finalités. Lorsque leur mariage était rompu les pères romains gardaient leurs enfants : ils n'avaient donc pas spécialement intérêt à ce que les unions soient indissolubles. Par contre leurs épouses avaient de bonnes raisons de craindre d'être répudiées et \emph{ipso facto} séparées de leurs enfants. Elles ont trouvé bon à partir du IVème siècle d'être mieux protégées contre ce risque et d'avoir leur vie durant l'exclusivité de la fécondité légitime de leur mari. Aujourd'hui où leur autonomie financière et les lois leur permettent de prendre l'initiative de quitter leurs maris sans risque de devoir lui laisser leurs enfants la situation se retourne et ce sont les hommes qui peuvent commencer de craindre d'être séduits puis abandonnés. 
L’instauration d'une déclaration formelle de renoncement à la paternité, proposée par Borillo en miroir du droit reconnu aux femmes à l'accouchement sous X, peut paraître provocante. Pourtant une telle disposition ne ferait que rejoindre le point de vue des révolutionnaires de 1789 : pas plus de contrainte en paternité qu’en maternité. 

 Plutôt que de se retrouver un jour contraints de continuer de payer pour leurs enfants sans plus les avoir auprès d'eux, tandis que parfois un autre qu'eux les éduque, les hommes pourraient choisir, quelle que soient par ailleurs leurs préférences sexuelles, de commencer par payer pour les posséder sans partage afin que leur génitrice ne puisse jamais les leur contester. Sur quels arguments fonder le refus d'une pareille évolution ? Elle ne serait au fond que le miroir de celle qui voit des femmes choisir en toute connaissance de cause de faire un enfant toutes seules. Si les humains ne diffèrent en rien de significatif en dehors de leurs caractéristiques biologiques, si les femmes n'ont pas besoin d'un homme pour élever un enfant, alors les hommes n'ont pas non plus besoin d'une femme pour élever leurs propres enfants.
 
 Le recours aux mères porteuses ne pourrait être interdit, en dépit de la pression des demandes individuelles et du modèle fourni par les pays où cette pratique est autorisée, que s'il était d'abord admis qu'il implique la réduction d'un humain au statut d'instrument de la volonté d'un tiers jusque dans son corps, ce qui est la définition de l'esclave, et s'il était reconnu que c'est inacceptable, même si cette personne a donné son accord. Une deuxième raison serait que cette pratique fait de l'enfant à naître le produit d'un contrat commercial (sauf à ce que se généralise le don d'enfants par des femmes qu'aucune nécessité matérielle y pousserait, mais rien ne montre qu'on aille vers là).
  
  Mais refuser le recours à des mères porteuses impliquerait aussi d'accepter l'idée qu'il n'existe pas de droit à l'enfant, c'est-à-dire que tout un chacun peut être irrémédiablement privé d'enfant en dépit de ses désirs les plus authentiques et les plus légitimes sans avoir pour autant droit à la répara-tion de cette injustice. Le mouvement des pratiques depuis un demi-siècle ne va pas dans ce sens.

Quant à espérer sortir de ces contradictions en recourant à un utérus artificiel, c'est encore et pour longtemps de l'utopie.

 \chapter{La fin des exclusions ?} 
 
 Dans L'avenir d'une illusion (1927), FREUD se demande jusqu'où une société humaine peut se permettre d'être souple et tolérante étant donnée la violence des pulsions, désirs et angoisses qu'elle a pour tâche d'humaniser. Sceptique et pessimiste il soutient qu'une grande dose de répression est inévi-table, et que c'est même une des conditions de l'élaboration de sociétés viva-bles et d'œuvres culturelles de valeur.
 
 Dans une période donnée peuvent être invisibles, inconscients, ou plutôt innommables et innommés (déniés) les traits de dureté qu'elle n'a pas suffisamment élaborés, les blocs de sauvagerie qu'elle n'a pas su penser. Ce sont des points aveugles dans la représentation que cette période se donne d'elle-même. Ils sont involontaires et bien évidemment personne ne les a voulus en connaissance de cause... par contre ils sauteront aux yeux des générations suivantes qui ne comprendront pas comment il a été possible de ne pas les voir. 
 Le plus bel exemple est que depuis des milliers d'années on a admis comme un fait établi et ne souffrant pas la discussion que les femmes étaient inférieures aux hommes, faites pour leur obéir et les servir, qu'il était donc in-dispensable qu'une part plus ou moins grande de leurs droits soient détenus et exercés par des membres masculins de leur entourage, et qu’il allait de soi qu’elles étaient  exclues de l’enseignement des apprentissages les plus presti-gieux et les plus efficaces.
 
L'histoire des enfants sans parents est elle aussi marquée par plusieurs de ces points aveugles, à commencer par la dureté du sort fait partout, depuis toujours et jusqu'aujourd'hui en toute bonne conscience aux enfants de naissance illégitime, quelles qu'aient été les manières successives de définir en quoi leur naissance était illégitime, c'est-à-dire inopportune. Quoi de plus barbare que la croyance en une impureté ou une infamie de naissance ? Quoi de plus arbitraire et déraisonnable que l'idée qu'être né d'un ou d'une esclave ou encore de ne pas être reconnu par un homme, interdisait irrévocablement de prétendre à des postes à responsabilité ? Quoi de plus étrange pour nous que la valeur religieuse du sang, ou la « pureté » d'une généalogie ? Quoi de plus absurde que de disqualifier moralement les « enfants du péché » tout en absolvant ceux qui auraient commis le « péché » dont ils étaient nés ? 

 Tout se passe comme si les conceptions archaïques du pur et de l'impur avaient continué d'être tenues pour vraies jusqu'à nos jours alors que le caractère moralement inadéquat de ces représentations avait été dénoncé il y a deux mille ans par les stoïciens aussi bien que par les évangiles, dont les thèses ont pourtant été méditées sans interruption depuis lors. Jusqu'au début du 20e siècle chacune de ces propositions a été tenue pratiquement pour vraie par tous ou presque tous, ou par chacun presque tout le temps, alors qu'elles étaient théoriquement insoutenables du point de vue même de ceux qui s'y conformaient. 
 
 Jusqu'à Vincent de Paul on n'appelait pas négligence le sort mortifère qui était fait aux nouveaux-nés abandonnés, parce que les ex-clure du monde des familles légitimes paraissait être la façon correcte de les traiter et qu'on n'en imaginait pas d'autre. Lui a su le premier ou l'un des premiers, voir en eux autre chose que des êtres impurs qu'il était moralement indifférent de laisser mourir du moment qu'ils étaient baptisés. C'est sur les représentations de ses contemporains qu'il a travaillé et non sur l'art d'accommoder les bébés séparés de leur mère (cet art ne posait pas plus de problèmes aux femmes de son époque qu'à celles d'aujourd'hui). 
 
 Il y a moins d'un siècle les mineurs vagabonds étaient encore considérés et traités comme des délinquants : la criminalisation de leurs errances avait commencé à la fin du Moyen Âge : auparavant on les assimilait aux pèlerins et on se recommandait à leurs prières. 
 
 De même il n'y a guère plus d'un demi-siècle qu'on regardait encore avec méfiance les rencontres entre les enfants placés en institution et leurs parents. 
 Et il n'y a guère plus de trente ans qu'on a vraiment pris la mesure de la gravité des dégâts psychologiques produits par les "abus" sexuels perpétrés par les adultes sur les enfants, surtout quand ils ont autorité sur eux, et qu'on a accepté de voir que les "abuseurs" font le plus souvent partie de l'entourage immédiat de leurs victimes, et d’abord de leur famille et de leurs éducateurs. Ces points aveugles étaient visibles par tous, mais ils n'étaient pas vus, mais ces violences et ces cruautés faisaient d'autant moins problème qu'elles paraissaient aussi inexorables que le jour et la nuit, aussi naturelles et nécessaires que le soleil et la pluie. 
 
 Il n'est donc pas impossible, il est même probable qu'aujourd'hui aussi s'étalent sous nos yeux des malheurs et des souffrances que nous ne voyons pas, des maltraitances que nous produisons en toute bonne (in)conscience. Si c'est réellement le cas, alors dans un siècle, ou dans dix, on nous reprochera de les avoir méconnus, sans comprendre que nous ne pouvions pas les voir, aveuglés que nous sommes par nos théories, nos croyances, nos désirs ou nos intérêts inconscients, de la même façon que nous sommes scandalisés par la brutalité, l'insensibilité et les aberrations logiques de nos prédécesseurs. 
 En l'absence d'observateurs venus d'un autre monde seules des recherches scientifiques rigoureuses sont en mesure d'apporter des éléments de réponse à de telles cécités, mais elles le font toujours trop lentement.
 
 Est-ce que les lois et les pratiques qui encadreront à l'avenir la conception des enfants et l'art de les accommoder produiront moins de souffrances et de troubles que celles du passé ? On ne voit pas bien en quoi les enfances organisées par les manipulations de la biologie et des relations inter-personnelles évoquées plus haut par Jacques Attali seraient un progrès du point de vue des enfants. Il est vrai que ce n'est pas leur objectif. 
 
 Le recours à la prévention des naissances, à la pilule anticonceptionnelle, à la pilule « du lendemain » et à l'avortement permet en principe qu'il ne naisse plus d'enfants non désirés. Mais suffit-il que ceux qui naissent aient été originellement désirés par leurs géniteurs ou par leurs parents adoptifs pour que disparaissent les problèmes qu'ils posent ou ceux qu'ils rencontrent ? Les enfants ne sont pas sans influence, pour le meilleur et pour le pire, sur la relation que leurs parents construisent avec eux. Volontairement ou sans pouvoir s'en empêcher ils peuvent déplaire à leurs parents sur des points auxquels ces derniers sont viscéralement attachés. Nul ne peut donc garantir qu'à l'avenir il y aura moins d'enfants mal assumés que par le passé. 
 
 Si la pauvreté matérielle n'est plus depuis longtemps un motif suffisant à lui seul pour séparer les enfants de leurs parents, est-on assuré pour autant qu'il n'existe et n'existera plus jamais d'enfants privés de l'un ou de l'autre de leurs parents alors que ceux-ci sont disponibles, volontaires pour les élever et suffisamment compétents ? L'absence de l'un des deux parents pour d'autres raisons que la maladie ou la mort devient au contraire quelque chose de plus en plus fréquent. 
 
 Est-ce que le recours à une adoption ou à une mère porteuse est aussi satisfaisant pour les enfants concernés que pour leur(s) parent(s) ? On aimerait que ce soit le cas, mais beaucoup d'adultes nés d'une insémination artificielle avec donneur (IAD) n’en expriment pas moins le désir de connaître leurs « origines ». On pourrait postuler que si la filiation adoptive était instituée comme le modèle de la filiation la réalité des parents de naissance perdrait de son importance, mais ce n'est qu'une hypothèse. Beaucoup parmi les jeunes et les adultes nés sous X veulent connaître au moins leur génitrice, et ceux à qui cela est refusé disent souffrir d'une peine inguérissable. A défaut de pouvoir exiger d'être élevés par leurs deux parents de naissance, les enfants concernés (beaucoup d'entre eux) veulent au moins les connaître et même si l'on ne voit pas tou-jours à quoi cela pourrait leur servir, eux le voient et ils s'obstinent. Même si on le leur refuse ils continuent de le vouloir. Et au nom de quoi pourrait-on les en empêcher ? Ils ont le droit pour eux au moins autant que les adultes ont le droit de vouloir un enfant.
 
 Certes tous les jeunes nés sous X ou d'une IAD ne sont pas tourmentés par ces interrogations, mais cela ne peut que rendre dubitatif. Même s'il est assez facilement accepté par les parents légaux (et on peut humainement le comprendre) l'oubli des parents de naissance, des géniteurs, n'est pas possible. Même si c'est seulement de façon imaginaire ces personnes font irrémédiablement partie de la relation entre les parents réels, légaux, et leurs enfants,  même lorsqu'ils s'interdisent d'en parler. 

 \chapter{Des valeurs irréconciliables ?} 

Chez les juifs et les chrétiens l'accueil de toute naissance est un devoir\footnote{l'avortement a toujours été strictement condamné chez les chrétiens, Les juifs le toléraient en cas de force majeure dans les premières semaines de la grossesse.}. Dans ce cadre à celui qui demande pourquoi il est né il est possible de répondre que Dieu a voulu qu'il vive (en passant à l’occasion par le truchement d’erreurs humaines). Des générations d'enfants ont trouvé cette explication satisfaisante : leur narcissisme en était suffisamment étayé. Un droit absolu à l'existence leur était reconnu quoi qu'il arrive, même s'ils ne correspondaient pas totalement, ou pas du tout, aux attentes de leurs parents.

La légalisation du droit à l'avortement a changé la donne. Dans des circonstances précisées par la loi l'embryon ou le fœtus a perdu la protection que le texte de la loi (à défaut des mœurs) lui accordait inconditionnellement depuis Constantin. Devenir un jour la personne qu'il est en potentiel, capable de discernement et de réciprocité avec autrui  n'est plus son droit. 

L'argument de fond c'est que tant qu'il n'a pas un nombre de semaines fixé par la loi il n'est qu'une partie du corps de sa mère, qui détient la maîtrise sur cette partie comme sur tout le reste. Jusque là il n'a pas d’existence reconnue, même quand il existe bel et bien pour ses parents et leur entourage (d’où parfois des demandes de réparation juridiquement irrecevables en cas d’avortement provoqué par un accident).

Les avortements dans les cas où la santé physique de la mère est sérieusement menacée par la grossesse ne posent guère de problème, pas plus que ceux où le fœtus est atteint de troubles interdisant sa survie ou son accès à un minimum de com-munication. Les médecins sont amenés de temps en temps à abréger sans souffrance la vie de nouveaux-nés reconnus non viables : la Hollande l'a re-connu dans le cadre du protocole de Groeningen. La Belgique s'est également engagée dans cette voie. 

 Par contre lorsque c'est d'abord ou seulement le bien-être de la mère ou celui de sa famille qui sont visés par un avortement, les enfants conscients de ces situations peuvent comprendre qu'on attend d'eux de n'être pas une gêne et de ne pas coûter trop d'efforts. Ils peuvent croire que c'est dans la réalité, et non dans leurs fantasmes les plus archaïques, que leurs parents ont eu sur eux pendant un temps droit de vie ou de mort.
 
 Les opposants « pro-vie » à l'avortement se scandalisent qu'on tue des embryons ou des fœtus puisque selon eux il n'y a rien qui les différencie radi-calement des nouveaux-nés. Pendant ce temps-là quelques moralistes s'ap-puient sur  le même constat pour demander au contraire que soit reconnu aux parents le droit de supprimer les \emph{nouveaux-nés} dont ils ne veulent pas, notamment ceux qui présentent des problèmes biologiques non détectés au cours de la grossesse (ex : trisomie 21). Les memes vont encore plus loin : dans un article du 2 mars 2012 publié dans le \emph{Journal of Medical ethics}, Alberto Giubilini et Francesca Minerva proposent, à la suite de Peter Singer, d'étendre le droit à l'avortement au-delà de la naissance (ce qu'ils nomment avortement post-natal). Voici un extrait de cet article (traduction personnelle) :
\emph{« Le droit prétendu des individus (tels que fœtus et nouveaux-nés) de déve-lopper leurs potentialités, droit que certains défendent, cède devant l'intérêt de ceux qui sont actuellement (dès aujourd'hui) des personnes (parents, famille, société) de rechercher leur propre bien-être, parce que, comme nous venons de le démontrer, ceux qui sont seulement des personnes potentielles ne peuvent pas être lésés par le fait de ne pas être introduits dans l'existence. Le bien-être des personnes actuelles c'est-à-dire le bien-être actuel des humains parvenus au stade de personnes en acte, de plein exercice  pourrait être affecté par de nouveaux enfants (même en bonne santé), réclamant de l'énergie, de l'argent et des soins, toutes choses dont la famille peut manquer. Parfois cette situation peut être évitée par un avortement, mais parfois cela n'est pas possible. Dans ces cas du moment que les non-personnes n'ont pas de droit moral à vivre, il n'y a pas de raisons de refuser l'avortement post-natal. Nous avons certes un devoir moral envers les futures générations alors qu'elles n'existent pas encore. Parce que nous tenons pour garanti que ces personnes existeront (quelles qu'elles soient) nous devons les traiter comme des personnes actuelles du futur. Cet argument, cependant, ne s'appli-que pas à tel ou tel nouveau-né en particulier, parce que nous ne pouvons pas tenir pour garanti qu'il deviendra une personne un jour. Est-ce qu'il existera  en tant que personne en acte  dépend en fait de nous et de notre choix.}

\emph{L'adoption peut-elle être une alternative à l'avortement post-natal ?}

\emph{On pourrait nous objecter que l'avortement post-natal ne devrait être pratiqué que sur les personnes potentielles qui ne pourront jamais avoir une vie digne d'être vécue. Dans cette hypothèse les individus en bonne santé et capables d'être heureux devraient être donnés à l'adoption lorsque leur famille ne peut pas les élever. Pourquoi devrions-nous tuer un nouveau-né en bonne santé alors que le confier à l'adoption ne grèverait les droits de personne mais au contraire accroîtrait le bonheur des personnes impliquées (adoptant et adopté) ?}

\emph{Notre réponse est la suivante : nous avons précédemment examiné l'argument de la potentialité (potentialité des êtres de devenir une personne) et montré qu'il n'est pas suffisamment puissant pour contrebalancer l'intérêt de ceux qui sont actuellement des personnes. En réalité combien minces puissent être les intérêts d'une personne actuelle, ils seront toujours supérieurs à l'intérêt (hypothétique) d'une personne en puissance de devenir une personne réelle, parce que ce dernier est égal à zéro. Dans cette perspective ce sont les intérêts des personnes actuelles qui ont de l'importance, et parmi ces intérêts nous devons en particulier considérer les intérêts de la mère qui peut souffrir psychologiquement si elle donne son enfant en adoption. On observe souvent que les mères de naissance rencontrent des problèmes psychologiques sérieux à cause de leur incapacité à élaborer leur perte et à surmonter leur chagrin. Il est vrai que le chagrin et le sentiment de perte peuvent accompagner l'avortement et l'avortement post-natal aussi bien que l'adoption, mais nous ne pouvons pas affirmer que pour la mère de naissance celle-ci est la moins traumatique. Par exemple, ceux qui pleurent un décès doivent accepter l'irréversibilité de la perte, mais souvent les mères naturelles rêvent que leur enfant va revenir vers elles. Cela rend difficile pour elles d'accepter la réalité de la perte parce qu'elles ne peuvent jamais être tout à fait certaines que cette perte est irréversible.
Nous ne cherchons pas à suggérer que ce sont des arguments décisifs contre la validité de l'adoption comme alternative à l'avortement post-natal. Cela dépend beaucoup des circonstances et des réactions psychologiques. Ce que nous sommes en train de suggérer c'est que si l'intérêt des personnes actuelles doit prévaloir, alors l'avortement post-natal doit être considéré comme une option permise aux femmes qui pourraient souffrir de donner leur nouveau-né à adopter. »}

 Pour Alberto Giubilini et Francesca Minerva  il  s'agit donc de promouvoir le droit à l'infanticide, très largement répandu dans le monde entier, mais supprimé par Constantin. Cette demande fait penser à Jonathan Swift et à son \emph{« Humble proposition pour empêcher les enfants des pauvres en Irlande d'être à la charge de leurs parents ou de leur pays et pour les rendre utiles au public »} (1729), mais cette proposition-ci  est formulée sans le moindre humour. Elle provoque un mouvement de refus horrifié. Mais combien de temps durera ce refus ? Ne peut-on imaginer qu'à force de jouer avec elle on finira par en valoriser les avantages et par en accepter les aspects déplaisants  ? 
 
 On finira peut-etre meme par défendre l'idée que cette proposition va dans le sens de l'intéret de l'enfant ?
 
 