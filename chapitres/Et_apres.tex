% Le 25.03.2015 :
% Â_ --> À_
% ~etc.
% même
% Le 22.03.2015 :
% A_ --> Â_
% Le 16.03.2015 :
% Moyen Âge
% Le 24.02.2015 :
% ~etc.
% Moyen-Âge
% ~\%
% fœtus



 \chapter[Et après ?]{Et après ?}


\section{"Interdit d'interdire ?"} 

  Depuis "Mai 68" nous savons que nous devons récuser toute contrainte dont l'intérêt ou la nécessité ne nous aient été prouvés de manière indiscutable. Nous sommes prêts à repousser toutes les limites et à tout essayer. Nous nous en faisons une gloire. Dans le domaine de la génération tout est donc remis en question. Avec l'aide de la médecine on en est presque arrivé à passer outre la nécessité de la rencontre d'un spermatozoïde et d'un ovule pour concevoir un enfant. Les historiens, les ethnologues et les sociologues nous fournissent tous les exemples de pratiques alternatives dont nous pourrions avoir besoin pour « déconstruire » (en bon français pour délégitimer) les représentations traditionnelles qui nous encombrent. Dans un article écrit en 1999 dans le cadre de la création du Pacs en faveur des couples homosexuels (\emph{"Enfin au-delà du Pacs"}) Jeanne Favret-Saada observe que :
  \begin{displayquote}
  "\emph{L'anthropologie  de  la  parenté  propose  le  répertoire
des arrangements culturels imaginés et pratiqués dans
les sociétés qui ont eu la chance d'être visitées par un
ethnologue. L'existence  même  de  cet  herbier  montre
l'immense  plasticité  des  sociétés  humaines  en  matière
de filiation, d'alliance et d'organisation familiale. Bien sûr,
le mariage homosexuel et l'homoparentalité au sens où
nous en parlons aujourd'hui, en France, n'ont été expérimentés  dans  aucune  des  populations  exotiques  que
nous  citons  quand  on  nous  oppose  l'impossibilité
anthropologique de nos revendications, et qu'on nous
accuse  de  vouloir  ruiner  les  fondements  de  la  culture
occidentale. Ni les Azandé, ni les berdaches, ni les çi, et
moins encore les ça n'ont eu notre projet politique.
La  seule  conclusion  qu'autorise  l'examen  de  l'herbier,
c'est donc que l'homoparentalité et le mariage homo-
sexuel  ne  sont  pas  une  impossibilité  anthropologique.}

\emph{...L'anthropologie  considère comme possible tout ce qui s'est déjà fait quelque part - du moins tout ce qu'un ethnologue dit avoir observé. Or  plusieurs  formes  d'homoparentalité  et  quelques
formes de mariage homosexuel existent déjà dans des
sociétés  comme  les  nôtres, avec  ou  sans  statut  légal,
selon  le  cas  : des  ethnologues  parfois (...), des
reporters  aussi  l'affirment. Dès  lors, le  problème  est
donc  moins  celui  de  la  possibilité  de  ces  nouvelles
formes de parenté que celui de leur passage au droit.}

\emph{...On voit donc qu'il est déjà nécessaire d'agrandir l'herbier  anthropologique, puisque  des  formes  inédites  de
parenté  sont  déjà  pratiquées  dans  nos  sociétés ; mais,
plus  encore, de questionner  les  concepts  utilisés  pour
fabriquer cet herbier.}" 
 \end{displayquote}
 




Les lois qui autorisent le divorce par consentement mutuel, la contraception, l'avortement et les fécondations artificielles ont dénaturalisé le modèle de famille traditionnel. Même si rien n'empêche personne de croire à la valeur religieuse, éthique, éducative ou civique de la famille fondée sur un couple hétérosexuel, monogame et indissoluble, élevant lui-même les enfants nés de ses œuvres, ce modèle n'est plus qu'un parmi les autres, comme c'était le cas avant les décisions de Constantin. Dans la mesure où ce modèle n'est plus étayé par la loi et par la puissance publique, comme il l'était sous l'Ancien Régime ou du temps du Code Napoléon, ce que j'ai appelé la famille constantinienne n'existe plus. Celle qu'on qualifie aujourd'hui de traditionnelle est, au même titre que les autres façons "post constantiniennes" d'organiser la reproduction humaine, une création des volontés individuelles des protagonistes, une parmi d'autres, dont la paternité, la maternité ou l'adoption célibataires, le concubinage, le PACS ou le mariage homosexuel, etc.

Sans jamais le désigner par son nom le Code Napoléon avait refusé de criminaliser l'inceste entre \emph{majeurs} consentants, mais il n'en refusait pas moins toute reconnaissance aux enfants nés de relations incestueuses. Depuis une génération ces enfants ne sont plus discriminés : s'ils ne peuvent pour l'instant être reconnus que par un seul de leurs géniteurs, ils peuvent hériter \emph{simultanément}
 de \emph{chacun} des deux, ce qui a toujours été la forme la plus substantielle de la reconnaissance. Même si ce n'est que de façon indirecte que cela leur est reconnu ils ont donc bien  \emph{deux parents légaux}, et ne sont plus hors famille. 
 
 
 
 Cela veut-il dire qu'il n'y a plus aucun enfant mal né ? 
 
 Cela veut-il dire qu'il n'y a plus de partenaires sexuels avec qui concevoir un enfant est interdit ? 
 
 
 
 \begin{displayquote} 
\emph{[...] il convient de se poser la question : notre société est-elle toujours fondée sur le principe de prohibition de l'inceste ? On ne s'est jamais autant soucié de l'inceste, de lever le voile sur la réalisation de l'inceste, de punir l'inceste. Cependant la prohibition de l'inceste est devenue floue dès lors que l'interdit sexuel s'est délié de la question matrimoniale... N'est-ce pas aujourd'hui une autre catégorie, celle du viol, qui devient le cadre à l'intérieur duquel vient s'inscrire l'inceste ? L'inceste n'est-il pas considéré comme un viol sur mineur ?}
 (Irène \fsc{THERY}, idem p. 499-501)
 \end{displayquote} 
 
 \section{Suffit-il d'aimer ?} 

Le vingtième siècle a sacralisé l’amour de l'homme et de la femme, posant en règle qu'il sanctifie tout ce qui est fait en son nom et que tout doit céder devant lui. Cette  valorisation moderne du sentiment amoureux a pu s'appuyer sur la Bible. En effet lorsque YVWH dit \emph{« Il n'est pas bon que l'homme soit seul. Il faut que je lui fasse une aide qui lui soit assortie."} (Gn 2, 18) c'est amoureusement qu'Adam reconnaît celle qui lui est donnée : \emph{"à ce coup c'est [Eve] l'os de mes os et la chair de ma chair !"} (Gn, 2, 23). Ce texte pluri-millénaire définit la relation de l’homme et de la femme comme si intime qu'elle les rend plus proches que ne le sont les parents les plus proches : \emph{"C'est pourquoi l'homme quitte son père et sa mère et s'attache à sa femme, et ils deviennent une seule chair."} (Gn, 2, 24). Pourtant la sacralisation actuelle du sentiment amoureux aurait bien étonné nos ancêtres, même ceux qui défendaient la liberté de choix conjugal des jeunes gens. Ils n'ont jamais cru  que le désir amoureux (ou passionnel, ou sexuel : \emph{Eros}) suffisait pour former un couple et fonder une famille. Lorsqu'ils parlaient d'amour conjugal il ne s'agissait pas d'un sentiment et encore moins d'une passion, mais d'une tâche à accomplir. Même si de façon très traditionnelle Paul de Tarse demandait aux épouses d'obéir à leur mari\footnote{exemple : Epître aux éphésiens, 5, 21-33}, il n'en faisait pas moins un devoir à \emph{chacun} des époux de prendre soin de son conjoint comme de son propre corps et de chercher activement à lui plaire. Il s'agissait de se vouer à son bien-être physique et mental, de pardonner ses défaillances et ses limites, et enfin « last but not the least » de lui reconnaître dans la durée un droit exclusif sur son propre corps. 
 
 Jusqu'au XXème siècle on croyait que la passion est aveugle, qu'elle est le lieu de toutes les illusions, qu'elle doit plus aux représentations de celui qui l'éprouve qu’à la réalité de l’objet de sa passion. De l'Antiquité à la Belle Époque les moralistes se sont méfiés du désir sexuel, y voyant une entrave à l'exercice de la raison, une force aveugle, inconstante, décevante et potentiellement destructrice de tous les liens et de tous les principes sur lesquels repose la société. C'est pourquoi ils insistaient sur la maîtrise des passions : contrôle des pulsions, valorisation du jugement et de la raison, valorisation du sentiment du devoir, entraînement à résister à la frustration, etc. Pour eux un amour ne méritait d'être qualifié de conjugal que lorsque s'étaient éteintes les éventuelles flambées du désir charnel. 



Erasme qui n'était ni pudibond ni ascète et qui tenait le mariage pour un choix de vie aussi méritoire que le célibat des moines et des clercs, ce qui était en son temps le signe d'une grande ouverture d'esprit, n'en écrivait pas moins dans son \emph{Encommium matrimonii christani} (Eloge du mariage chrétien, 1526) : 

\begin{displayquote}
\emph{"... Les poètes appellent l'ardeur des amants une fureur et non point un amour. Car enfin, où la raison est éteinte, peut-il y avoir autre chose que de la folie ? ...ceux qui épousent des femmes imprudemment et sans jugement ont coutume de se repentir, mais quelquefois trop tard, de ces malheureux mariages. Il arrive rarement qu'on regrette de s'être marié par l'avis des parents, et d'avoir pris par un choix mûr et délibéré des femmes qu'on puisse aimer toujours. Car tout ce qui se fait en ne consultant que nos passions n'a qu'un temps. Ce qui se fait au contraire par raison et par jugement est stable et dure longtemps"}.  
\end{displayquote}

 C'est un constat similaire que fait Maurice Godelier, cinq siècles plus tard, à l'issue de ses enquêtes ethnologiques. Selon lui le désir sexuel est fondamentalement asocial et même destructeur : 
 \begin{displayquote}
 \emph{« La permissivité en matière de sexe s'arrête [...], dans toutes les sociétés, soit là où la formule d'alliance serait menacée, soit là où les rapports de coopération et d'autorité entre consanguins risqueraient de s'effondrer et, glissant les uns dans les autres, de disparaître (Na). Mais cette fois, ce n'est plus de la sexualité-reproduction qu'il est question, mais de la sexualité-désir qui, nous l'avons vu, est dans son fond asociale. Elle n'est jamais la base d'une coopération durable entre les individus, tant au sein du groupe où ils sont nés qu'entre lui et les groupes avec lesquels il est allié. Et ce n'est pas seulement le désir hétérosexuel qui unit et divise. [...] C'est, nous l'avons dit, parce que le désir sexuel en lui-même est asocial qu'aucune société ne peut permettre que tout soit permis.} 
 
 \emph{Et ce travail d'auto-domestication est toujours à recommencer, alors que le processus de domestication des plantes et des animaux semble avoir atteint ses limites. [...] Partout la spontanéité du désir a dû être sacrifiée pour produire un ordre social qui est toujours en même temps un ordre entre les sexes et un ordre sexuel. Partout a dû être éliminé le caractère asocial de la sexualité, sacrifié le polymorphisme du désir, interdite la permissivité sexuelle généralisée pour que la société puisse s'organiser et se reproduire.}
 
\emph{[...] Cependant, sacrifier le caractère asocial de la sexualité n'est pas seulement un acte d'amputation.  C'est en même temps une sorte de création. C'est agir sur soi pour continuer non seulement à vivre en société, mais à produire de la société pour vivre, ce qui est le propre de l'homme et le séparera toujours davantage, chaque jour qui passe, des primates, ses lointains cousins. (p. 632-636)}
\end{displayquote}
Malgré la réhabilitation du désir, et notamment du désir sexuel, que l’oeuvre de Freud a initiée un ethnopsychiâtre d'aujourd'hui comme Tobie Nathan ne dit pas autre chose : 
\begin{displayquote}
\emph{"Qu'est-ce qui différencie la passion de l'amour, notamment conjugal ? La passion, ce n'est pas l'amour. D'ailleurs, les Grecs avaient deux mots distincts. Philia signifie l'amour raisonnable - comme l'amour conjugal - ou l'amitié, tandis qu'Eros désigne le désir, la passion amoureuse. Platon la caractérisait par le manque. Observation exacte, mais insuffisante. Il s'agit d'une exacerbation du manque - dans la passion, l'autre me manque quand il n'est pas là, il me manque quand il est là, car il n'y est jamais suffisamment ; dans la relation sexuelle, et même au moment de l'orgasme, il me manque encore. Recherche d'une fusion impossible, pulsion à offrir à l'autre tout votre espace intérieur - la passion amoureuse est une folie. La seconde caractéristique est qu'elle produit du changement, un bouleversement radical, et ce mouvement n'est pas maîtrisable."}
\end{displayquote}
Si le désir sexuel ne suffit pas pour unir les amants au-delà de quelques mois ou années, est-ce qu'on pourrait au moins défendre l'idée qu'il faut s'en remettre à lui pour choisir celui ou celle avec qui fonder une communauté de vie durable et accueillir des enfants ? Mais il est de notoriété publique que les unions n'ont jamais été aussi fragiles que depuis que le mariage d’amour est devenu le modèle et que les parents ont perdu la capacité d’arranger (ou d'empêcher) les mariages de leurs enfants, ...ce qui ne prouve pas que leur fragilité trouve là son origine.


Mais est-ce que nos sociétés ont encore pour but de favoriser la création de couples conjugaux durables ? Nietzsche affirmait dès 1888 que la dénaturation du mariage était en cours. Il stigmatisait l'importance donnée au mariage d'amour et à l'égalité des époux (pourtant fort limitées àl'époque à côté de nos moeurss actuelles) :
\begin{displayquote}
\emph{« On vit pour aujourd'hui, on vit très vite -- on vit sans aucune responsabilité : c'est précisément ce que l'on appelle « liberté ». Tout ce qui fait que les institutions sont des institutions est méprisé, haï, écarté : on se croit de nouveau en danger d'esclavage dès que le mot « autorité » se fait seulement entendre [...] Témoin : le mariage moderne. Apparemment toute raison s'en est retirée : pourtant cela n'est pas une objection contre le mariage, mais contre la modernité. La raison du mariage -- elle résidait dans la responsabilité juridique exclusive de l'homme : de cette façon le mariage avait un élément prépondérant, tandis qu'aujourd'hui il boite sur deux jambes. La raison du mariage -- elle résidait dans le principe de son indissolution : cela lui donnait un accent qui, en face du hasard des sentiments et des passions, des impulsions du moment, savait se faire écouter. Elle résidait de même dans la responsabilité des familles quant au choix des époux. Avec cette indulgence croissante pour le mariage d'amour on a éliminé les bases mêmes du mariage, tout ce qui en faisait une institution. Jamais, au grand jamais, on ne fonde une institution sur une idiosyncrasie ; je le répète, on ne fonde pas le mariage sur « l'amour », -- on le fonde sur l'instinct de l'espèce, sur l'instinct de propriété (la femme et les enfants étant des propriétés), sur l'instinct de la domination qui sans cesse s'organise dans la famille en petite souveraineté, qui a besoin des enfants et des héritiers pour maintenir, physiologiquement aussi, en mesure acquise de puissance, d'influence, de richesse, pour préparer de longues tâ-ches, une solidarité d'instinct entre les siècles. Le mariage, en tant qu'institution, comprend déjà l'affirmation de la forme d'organisation la plus grande et la plus durable : si la société prise comme un tout ne peut porter caution d'elle même jusque dans les générations les plus éloignées, le mariage est complètement dépourvu de sens. -- Le mariage moderne a perdu sa signification -- par conséquent on le supprime. »} (\emph{Le Crépuscule des idoles}, 1888) 
\end{displayquote}

Que pensait Nietzsche de cet état de fait ? Le dénonçait-il ? ou se bornait-il à le constater avec une ironie un peu cynique ?

En tout cas les femmes ne retourneront plus dans des gynécées, sinon contraintes et forcées (et par qui à part elles-mêmes ?). L'exigence d'égalité absolue (de dignité, de pouvoir, de salaire, de promotion, etc.) entre hommes et femmes placés dans les mêmes situations est une évidence de notre temps sur laquelle il est peu probable que l'on revienne. Avec quels arguments pourrait-on défendre de manière \emph{convaincante} une inégalité fondée sur le sexe ou sur le genre, que le bénéficiaire de cette inégalité soit mâle ou femelle ? Cette exigence d'égalité est d'ailleurs loin d'avoir encore produit tous ses effets, directs et indirects. 


 
 En même temps que le mariage d’amour triomphait dans les représentations s’effondraient l’un après l’autre la quasi-totalité des contreforts qui naguère étayaient le lien conjugal, que ce soient les lois (l'indissolubilité du mariage « constantinien » n’a été a peu près respectée que sous la pression d’un encadrement juridique patiemment construit et vigoureusement défendu pendant plus d'un millénaire), l'intérêt matériel (le lien conjugal est significativement plus fragile lorsque l'épouse peut se procurer des ressources propres hors du foyer), la pression familiale ou sociale (les divorcés et leurs enfants ne sont plus stigmatisés et leurs nouveaux partenaires sont de plus en plus souvent reçus par leurs parents à égalité avec les précédents), l’impossibilité d’obtenir des héritiers hors mariage régulier (sauf inceste tous les enfants peuvent aujourd’hui être reconnus par leurs deux parents quel que soit le statut matrimonial de ces derniers, et tous peuvent hériter à égalité) les croyances religieuses (les morales traditionnelles d'inspiration religieuses  paraissent désuètes et inadaptées, ou même incompréhensibles, et ne sont plus respectées). Lorsqu'il ne reste plus pour relier les conjoints que le souci d’élever leurs enfants on observe que c'est souvent insuffisant pour donner du sens à une vie en commun.

 
 Depuis qu'il n'y a plus de différences entre les enfants nés dans le mariage et les autres, il n'est plus nécessaire d'épouser pour avoir des héritiers légitimes et l'intérêt du mariage diminue au fur et à mesure qu'au nom de l'égalité les lois étendent aux non-mariés les droits accordés aux mariés. L’institution du mariage servait à créer des différences, et comme elle n’en crée plus elle s'étiole.   Certes les unions libres existaient déjà autrefois (depuis toujours en fait) mais dans des groupes sociaux peu ou pas concernés par les questions d’héritage. Elles se sont multipliées de façon exponentielle. 
 
Mais pourquoi vouloir que les couples durent ? Après tout il n'est même plus nécessaire que les sexes se rencontrent pour faire un enfant, ni d'être deux pour l'élever. Pourquoi pas une succession d'amours éphémères ?  ...ou pas d'amours du tout ? Ce n'est plus que que par habitude que la loi prescrit encore aux conjoints mariés d'être fidèles, mais elle ne prévoit plus de sanctions à l'encontre des infidèles (même le nombre de divorce pour fautes s'effondre). Elle ne se sent plus concernée par ce qui ne relève désormais que des vies privées. 

\section{Les risques du désir} 

En commentaire des polémiques suscitées par le projet de loi ouvrant le mariage aux personnes de même sexe, Jacques ATTALI esquisse l'avenir qui, compte tenu des évolutions récentes dans les pratiques familiales, reproductives et sexuelles, lui paraît le plus probable   : 
\begin{displayquote}
\emph{« Comme toujours, quand s'annonce une réforme majeure, il faut comprendre dans quelle évolution de long terme elle s'inscrit.
Et la légalisation, en France après d'autres pays, du mariage entre deux adultes homosexuels, s'inscrit comme une anecdote sans importance, dans une évolution commencée depuis très longtemps, et dont on débat trop peu : après avoir connu d'innombrables formes d'organisations sociales, dont la famille nucléaire n'est qu'un des avatars les plus récents, et tout aussi provisoire que ceux qui l'ont précédé, nous allons lentement vers une humanité unisexe, où les hommes et les femmes seront égaux sur tous les plans, y compris celui de la procréation, qui ne sera plus le privilège, ou le fardeau, des femmes.}

\emph{1. La demande d'égalité. D'abord entre les hommes et les femmes. Puis entre les hétérosexuels et les homosexuels. Chacun veut, et c'est naturel, avoir les mêmes droits : travailler, voter, se marier, avoir des enfants. Et rien ne résistera, à juste titre, à cette tendance multiséculaire. Mais cette égalité ne conduit pas nécessairement à l'uniformité : les hommes et les femmes restent différents, quelles que soient leurs préférences sexuelles.}

 \emph{2. La demande de liberté. Elle a conduit à l'émergence des droits de l'homme et de la démocratie. Elle pousse à refuser toute contrainte ; elle implique, au-delà du droit au mariage, les mêmes droits au divorce. Et au-delà, elle conduira les hommes et les femmes, quelles que soient leurs orientations sexuelles, à vouloir vivre leurs relations amoureuses et sexuelles libres de toute contrainte, de tout engagement. La sexualité se séparera de plus en plus de la procréation et sera de plus en plus un plaisir en soi, une source de découverte de soi, et de l'autre. Plus généralement, l'apologie de la liberté individuelle conduira inévitablement à celle de la précarité ; y compris celle des contrats. Et donc à l'apologie de la déloyauté, au nom même de la loyauté : rompre pour ne pas tromper l'autre. Telle est l'ironie des temps présents : pendant qu'on glorifie le devoir de fidélité, on généralise le droit à la déloyauté. Pendant qu'on se bat pour le mariage pour tous, c'est en fait le mariage de personne qui se généralise.}
 
\emph{3. La demande d'immortalité, qui pousse à accepter toutes mutations sociales ou scientifiques permettant de lutter contre la mort, ou au moins de la retarder.}

\emph{4. Les progrès techniques découlent en effet de ces valeurs, et s'orientent dans le sens qu'elles exigent : en matière de sexualité, cela a commencé par la pilule, puis la procréation médicalement assistée, puis la gestation pour autrui. Ces questions de bioéthique ne découlent évidemment pas des demandes d'égalité venant des couples homosexuels et concernent toutes les formes de reproduction, y compris -- et surtout -- « hétérosexuelles ». Le vrai danger viendra si l'on n'y prend garde, du clonage et de la matrice artificielle, qui permettra de concevoir et de faire naitre des enfants hors de toute matrice maternelle. Et il sera très difficile de l'empêcher, puisque cela sera toujours au service de l'égalité, de la liberté, ou de l'immortalité.}

\emph{5. La convergence de ces trois tendances est claire : nous allons inexorablement vers une humanité unisexe, sinon qu'une moitié aura des ovocytes et l'autre des spermatozoïdes, qu'ils mettront en commun pour faire naitre des enfants, seul ou à plusieurs, sans relation physique, et sans même que nul ne les porte. Sans même que nul ne les conçoive si on se laisse aller au vertige du clonage.}

\emph{6. Accessoirement, cela résoudrait un problème majeur qui freine l'évolution de l'humanité: l'accumulation de connaissances et des capacités cognitives est limitée par la taille du cerveau, elle-même limitée par le mode de naissance: si l'enfant naissait d'une matrice artificielle, la taille de son cerveau n'aurait plus de limite. Après le passage à la station verticale, qui a permis à l'humanité de surgir, ce serait une autre évolution radicale, à laquelle tout ce qui se passe aujourd'hui nous prépare. Telle est l'humanité que nous préparons, indépendamment de notre sexualité, par l'addition implicite de nos désirs individuels...}

\emph{Alors, au lieu de s'opposer à une évolution banale et naturelle du mariage laïc, qui ne les concerne pas, les Eglises devraient plutôt se préoccuper de réfléchir, avec les laïcs, à ces sujets bien plus importants : comment permettre à l'humanité de définir et de protéger le sanctuaire de son identité ?}

\emph{Comment poser les barrières qui lui permettront de ne pas se transformer en une collection d'artefacts producteurs d'artefacts ?}

\emph{Comment faire de l'amour et de l'altruisme le vrai moteur de l'histoire ? »}
\end{displayquote}

On aura remarqué que Jacques ATTALI ne décrit pas tant ce qu'il désire que ce qu'il prévoit, dans l'hypothèse où les dynamiques en cours se prolongeraient sans changement. Et il n'exclut pas formellement l'idée que l'avenir qu'il décrit, s'il se réalisait, pourrait n'être pas totalement radieux. 



 Ainsi l'exigence de liberté individuelle absolue en matière amoureuse, à tout prix et quelles qu'en soient les conséquences est grosse des problèmes pointés par Jacques Attali. Peut-on croire qu'il pourrait exister un domaine de l'existence dans lequel aucun engagement n'aurait d'importance, où aucune parole ne vaudrait rien, et que cela n'entraînerait pas de répercussions significatives dans les autres domaines ? …dans les autres conversations ? …dans les autres relations ? D’autant plus qu'il s'agit d'un domaine charnellement lié à la construction par chacun de son identité ? Peut-on croire que cela n'aurait aucun effet en termes de « lien social » ? Si  aucune promesse ne vaut, sauf les contrats commerciaux, alors il n’est pas impossible qu’il ne reste entre les individus que la sauvagerie, la « brutalisation » et les rapports de force. 
 
 Quant à la valorisation de l'immortalité individuelle, c'est une forme de l'individualisme. Refuser de mourir, si c'est \emph{à n'importe quel prix}, c'est se placer au-dessus de tous les autres et faire fi des liens avec eux. Dans ce contexte \emph{« Comment faire de l'amour et de l'altruisme le vrai moteur de l'histoire ? »} Comment pourrait-il rester même une place pour l'amour d'un autre que soi, pour l'altruisme ? Le sacrifice de soi pour un (pour une) autre ou pour une noble cause (la justice, la vérité, le bien, la paix, la liberté, la nation, la démocratie, etc..) n'a plus de sens 



 


\section{Qui désire les enfants ?} 

 
 A qui appartiennent les enfants ?  Ils ne s'appartiennent pas à eux-mêmes, sauf à supprimer le statut de mineur. On ne peut pas non plus dire qu'ils n'appartiennent à personne. Du point de vue des enfants, n'appartenir à personne (ou appartenir à une institution d'assistance publique) c'est être abandonné. 
 Depuis l’antiquité tardive les enfants n'appartiennent plus seulement à leurs pères. Est-ce qu'ils appartiennent aux deux parents, comme dit la loi ? ...ou bien plus à leur mère qu’à leur père ? …ou bien à l'ensemble de ceux qui les élèvent en leur donnant leur argent et leur temps, dont les beaux-pères et belles-mères ? ...ou bien encore à l'État ? 
 
Si l'on en croit Coluche, \emph{« y a des gens qui ont des enfants parce qu'ils n'ont pas les moyens de s'offrir un chien »}. Il posait à sa manière une question essentielle et relativement nouvelle : pourquoi fait-on des enfants ? Pour quoi veut-on des enfants ? À quoi servent les enfants ?
Le désir de serrer un bébé de chair dans ses bras est d'autant plus irrépressible que ses motivations les plus vraies sont inconscientes. Il est sans doute aussi répandu et aussi fort aujourd'hui que par le passé et il ne concerne pas seulement les femmes. 



 Si tous les citoyens des pays dotés d'un bon système d'assistance sociale et de caisses de retraite suffisantes ont besoin qu'il naisse des enfants pour financer leurs périodes d'invalidité et leurs vieux jours, aucun d'eux n'a besoin que ce soient ses propres enfants : c'est précisément pour cela que ces systèmes ont été mis en place. Dans les pays les plus socialement développés, seule la collectivité a besoin d'enfants. D'un point de vue strictement comptable et sauf dispositifs de compensation très généreux des frais qu'entraînent ces derniers l'intérêt financier bien compris des citoyens des États providence est de ne pas en avoir. Leur niveau de vie et leur crédit auprès de leur banquier seront plus élevés s'ils évitent d'investir dans une progéniture. Il ne faut sans doute pas chercher plus loin la faiblesse des taux de natalité de leurs membres, taux qui ne sont que la résultante des stratégies individuelles de leurs citoyens, stratégies d'autant plus rationnelles qu'on ne voit plus aujourd'hui au nom de quelle exigence morale on pourrait les leur reprocher. 
 
Face à la désaffection du mariage et de la procréation qui menaçait la survie de l'Empire Romain, l'empereur Auguste avait réagi en pénalisant les célibataires et ceux qui n'avaient pas d'enfants, et ses lois ont été appliquées sans faillir pendant au moins trois siècles. Si nos taux de natalité baissaient dangereusement, verrions-nous à l'avenir de pareilles incitations légales  à procréer ? 

 Mais les malthusiens et avec eux bien des écologistes pensent que les problèmes de santé de notre planète ont pour origine le fait que les humains sont trop nombreux. Il faudrait en effet que le nombre de ces derniers diminue drastiquement s'ils voulaient tous consommer comme les citoyens des pays développés actuels sans épuiser les ressources disponibles et sans mettre en danger les équilibres de la nature. Cela impliquerait non pas une croissance démographique zéro, mais une décroissance très énergique. L'intérêt commun de l'humanité serait-il sa décroissance numérique et donc l'évitement de la reproduction jusqu'au retour à un effectif écologiquement optimal ? 
 


Chez les juifs et les chrétiens l'accueil de toute naissance est un devoir\footnote{l'avortement a toujours été strictement condamné chez les chrétiens, et les juifs ne le toléraient qu'en cas de force majeure et/ou dans les premières semaines de la grossesse.}. Dans ce cadre à celui qui demande pourquoi il est né il est possible de répondre que Dieu a voulu qu'il vive (en passant à l’occasion par le truchement d’erreurs humaines). Des générations d'enfants ont trouvé cette explication satisfaisante. Un droit inconditionnel à l'existence leur était reconnu quoi qu'il arrive, même s'ils ne correspondaient pas totalement, ou pas du tout, aux attentes de leurs parents. Leur narcissisme en était suffisamment étayé.

La légalisation du droit à l'avortement a changé la donne. Dans des circonstances précisées par la loi l'embryon ou le fœtus a perdu la protection que le texte de la loi (à défaut des pratiques réelles) lui accordait inconditionnellement depuis Constantin. Devenir un jour la personne qu'il est en potentiel, capable de discernement et de réciprocité avec autrui  n'est plus son droit. 

L'argument de fond c'est que tant qu'il n'a pas un nombre de semaines fixé par la loi il n'est qu'une partie du corps de sa mère, qui détient la maîtrise sur cette partie-là comme sur tout le reste. Jusqu'à sa naissance il n'a pas d’existence reconnue, même quand il existe bel et bien aux yeux de ses parents et de leur entourage (d’où parfois des demandes de réparation juridiquement irrecevables en cas d’avortement provoqué par un accident).

Les avortements dans les cas où la santé physique de la mère est sérieusement menacée par la grossesse ne posent guère de problème éthiques, pas plus que ceux où le fœtus est atteint de troubles interdisant sa survie ou son accès à un minimum de communication, quelle que soit la douleur ressentie par les protagonistes. Les médecins sont amenés de temps en temps à abréger sans souffrance la vie de nouveaux-nés reconnus non viables : la Hollande l'a reconnu dans le cadre du protocole de Groeningen. La Belgique s'est également engagée dans cette voie. 

 Par contre lorsque c'est d'abord ou seulement le bien-être de la mère ou celui de sa famille qui sont visés par un avortement, les enfants conscients de ces situations peuvent comprendre qu'on attend d'eux de n'être pas une gêne et de ne pas coûter d'efforts excessifs. Ils peuvent croire que c'est dans la réalité, et non dans leurs fantasmes les plus archaïques, que leurs parents ont eu sur eux pendant un temps droit de vie ou de mort.
 
 Les opposants « pro-vie » à l'avortement se scandalisent qu'on tue des embryons ou des fœtus puisque selon eux il n'y a rien qui les différencie radicalement des nouveaux-nés. Pendant ce temps-là quelques moralistes s'appuient sur  le même constat pour demander au contraire que soit reconnu aux parents le droit de supprimer les \emph{nouveaux-nés} dont ils ne veulent pas, notamment ceux qui présentent des problèmes biologiques (non léthaux) non détectés au cours de la grossesse (ex : trisomie 21). Les memes vont encore plus loin : dans un article du 2 mars 2012 publié dans le \emph{Journal of Medical ethics}, Alberto Giubilini et Francesca Minerva proposent, à la suite de Peter Singer, d'étendre le droit à l'avortement au-delà de la naissance (ce qu'ils nomment avortement post-natal). Voici un extrait de cet article (traduction personnelle) :
 \begin{displayquote}
\emph{« Le droit prétendu des individus (tels que fœtus et nouveaux-nés) de déve-lopper leurs potentialités, droit que certains défendent, cède devant l'intérêt de ceux qui sont actuellement (dès aujourd'hui) des personnes (parents, famille, société) de rechercher leur propre bien-être, parce que, comme nous venons de le démontrer, ceux qui sont seulement des personnes potentielles ne peuvent pas être lésés par le fait de ne pas être introduits dans l'existence. Le bien-être des personnes actuelles c'est-à-dire le bien-être actuel des humains parvenus au stade de personnes en acte, de plein exercice  pourrait être affecté par de nouveaux enfants (même en bonne santé), réclamant de l'énergie, de l'argent et des soins, toutes choses dont la famille peut manquer. Parfois cette situation peut être évitée par un avortement, mais parfois cela n'est pas possible. Dans ces cas du moment que les non-personnes n'ont pas de droit moral à vivre, il n'y a pas de raisons de refuser l'avortement post-natal. Nous avons certes un devoir moral envers les futures générations alors qu'elles n'existent pas encore. Parce que nous tenons pour garanti que ces personnes existeront (quelles qu'elles soient) nous devons les traiter comme des personnes actuelles du futur. Cet argument, cependant, ne s'appli-que pas à tel ou tel nouveau-né en particulier, parce que nous ne pouvons pas tenir pour garanti qu'il deviendra une personne un jour. Est-ce qu'il existera  en tant que personne en acte  dépend en fait de nous et de notre choix.}

\emph{L'adoption peut-elle être une alternative à l'avortement post-natal ?}

\emph{On pourrait nous objecter que l'avortement post-natal ne devrait être pratiqué que sur les personnes potentielles qui ne pourront jamais avoir une vie digne d'être vécue. Dans cette hypothèse les individus en bonne santé et capables d'être heureux devraient être donnés à l'adoption lorsque leur famille ne peut pas les élever. Pourquoi devrions-nous tuer un nouveau-né en bonne santé alors que le confier à l'adoption ne grèverait les droits de personne mais au contraire accroîtrait le bonheur des personnes impliquées (adoptant et adopté) ?}

\emph{Notre réponse est la suivante : nous avons précédemment examiné l'argument de la potentialité (potentialité des êtres de devenir une personne) et montré qu'il n'est pas suffisamment puissant pour contrebalancer l'intérêt de ceux qui sont actuellement des personnes. En réalité combien minces puissent être les intérêts d'une personne actuelle, ils seront toujours supérieurs à l'intérêt (hypothétique) d'une personne en puissance de devenir une personne réelle, parce que ce dernier est égal à zéro. Dans cette perspective ce sont les intérêts des personnes actuelles qui ont de l'importance, et parmi ces intérêts nous devons en particulier considérer les intérêts de la mère qui peut souffrir psychologiquement si elle donne son enfant en adoption. On observe souvent que les mères de naissance rencontrent des problèmes psychologiques sérieux à cause de leur incapacité à élaborer leur perte et à surmonter leur chagrin. Il est vrai que le chagrin et le sentiment de perte peuvent accompagner l'avortement et l'avortement post-natal aussi bien que l'adoption, mais nous ne pouvons pas affirmer que pour la mère de naissance celle-ci est la moins traumatique. Par exemple, ceux qui pleurent un décès doivent accepter l'irréversibilité de la perte, mais souvent les mères naturelles rêvent que leur enfant va revenir vers elles. Cela rend difficile pour elles d'accepter la réalité de la perte parce qu'elles ne peuvent jamais être tout à fait certaines que cette perte est irréversible.
Nous ne cherchons pas à suggérer que ce sont des arguments décisifs contre la validité de l'adoption comme alternative à l'avortement post-natal. Cela dépend beaucoup des circonstances et des réactions psychologiques. Ce que nous sommes en train de suggérer c'est que si l'intérêt des personnes actuelles doit prévaloir, alors l'avortement post-natal doit être considéré comme une option permise aux femmes qui pourraient souffrir de donner leur nouveau-né à adopter. »}
\end{displayquote}

 Pour Alberto Giubilini et Francesca Minerva  il  s'agit donc de promouvoir le droit à l'infanticide, très largement répandu dans le monde entier, mais supprimé par Constantin. Cette demande fait penser à Jonathan Swift et à son \emph{« Humble proposition pour empêcher les enfants des pauvres en Irlande d'être à la charge de leurs parents ou de leur pays et pour les rendre utiles au public »} (1729), mais cette proposition-ci  est formulée sans le moindre humour. Elle provoque un mouvement de refus horrifié. Mais combien de temps durera ce refus ? Ne peut-on imaginer qu'à force de jouer avec elle on finira par en valoriser les avantages et par en accepter les aspects déplaisants  ? On finira peut-etre meme par défendre l'idée que cette proposition va dans le sens de l'intéret de l'enfant ?

 \section{Filiation adoptive ou filiation « biologique » ?} 
 
 Le "mariage constantinien" tel que je l'ai défini télescopait sur le couple des seuls géniteurs (unis de manière socialement reconnue) toutes les dimensions de la conjugalité et de la parentalité (juridique, biologique, affective et éducative) et frappait tout le reste d'illégitimité et notamment la filiation élective, volontaire, et même adoptive. C'est au nom de cette dernière que l'hégémonie de notre tradition juridique est aujourd’hui théoriquement contestée, et cette remise en question est, comme on peut s'y attendre, consubstantiellement liée à la promotion de nouvelles formes de conjugalités. Selon Daniel BORRILLO  les possibilités nouvelles de dissociation entre sexualité et reproduction qui se sont ouvertes grâce aux progrès de la biologie en à peine une génération ont provoqué une panique morale, qui a conduit les théoriciens et les praticiens du Droit à survaloriser les liens biologiques géni-teurs-enfants :
  \begin{displayquote}
\emph{« La biologie commença à devenir ainsi le soubassement réel ou symbolique du système de parenté, à rebours d'une science juridique qui avait plutôt instauré la volonté au cœur de ce système [...] À partir des années 1990, l'expertise biologique  s'est imposée dans les procès en contestation de paternité, la recherche des origines est revendiquée socialement comme droit fondamental de la personne, la différence de sexe est devenue une valeur [...] La nouvelle place prépondérante de la vérité biologi-que dans l'établissement du lien filial fut confirmée en France par la Cour de cassation  ...Par là, la distinction traditionnelle entre reproduction (fait biologique) et filiation (fait culturel), fondement du droit civil moderne, se trouvait questionnée... non pas à partir d'arguments classiques provenant du droit canonique, mais par une rhé-torique qui, d'une part, fera de la différence des sexes une condition sine qua non de la filiation, et, d'autre part, placera l'expertise sanguine et la preuve d'ADN au cœur du dispositif juridique de la parenté.}

\emph{[...] La filiation peut certes tenir compte du fait naturel, mais, en tant que dispositif d'agencement parental, elle répond à des règles propres, affranchies de la nature [...] Elle n'existe que lorsqu'elle est établie dans les conditions et selon les modes prévus par la loi. Autrement dit, la filiation est déterminée par la norme juridique et non par la nature. Ce lien juridique se tisse à partir de quatre fils principaux : la biologie (filiation par le sang), la volonté (adoption), la présomption (paternité suppo-sée du mari de la mère) et le vécu (appelé en droit « possession d'état »).}

\emph{Ce qui compte ce ne sont plus tant les racines naturelles ou surnaturelles d'institutions intangibles que l'efficacité et la plasticité d'instruments juridiques procurant tel ou tel résultat (par exemple la paix des familles ou la solidarité des générations).  [...] fondée sur la volonté, l'adoption est une institution plus apte que la vérité biologique à assurer la stabilité des liens familiaux.}

 \emph{[...]La contestation actuelle de l'ordre familial « naturel » n'est en définitive que la radicalisation de l'idéologie individualiste moderne, selon laquelle la volonté et non la différence des sexes constitue la base de l'institution matrimoniale et parentale. Une filiation dissociée de la reproduction permettra de justifier un système juridique fondé non pas sur la vérité biologique, mais sur le projet parental responsable. De ce point de vue, peu importe l'agencement familial (traditionnel, monoparental, homo-parental, recomposé...), si les prémisses du contrat (égalité dans l'alliance et dans la filiation) sont respectées jusque dans leurs moindres effets. L'État devrait donc traiter sur un plan d'égalité l'ensemble des familles et les autres formes d'intimité.}
 
\emph{Contrairement à la filiation charnelle, la filiation choisie trouve son principe dans la liberté non seulement d'accueillir les enfants des autres, mais également d'abandonner ses propres enfants biologiques, ce qui est pour l'heure uniquement pos-sible pour les femmes (accouchement sous X), mais devrait pouvoir s'élargir aussi aux hommes à travers une déclaration formelle de renoncement à la paternité. La généralisation de la filiation adoptive (y compris pour ses propres enfants biologiques) per-mettrait aussi de mettre la volonté au cœur du dispositif parental. Celui-ci reposerait exclusivement sur la volonté du ou des géniteurs qui donnent l'enfant et celle du ou des adoptants qui l'accueillent. De surcroît, l'adoption est une institution conçue à partir du droit de l'enfant à avoir une famille, contrairement à la filiation biologique qui apparaît plutôt comme un dispositif du droit à l'enfant ». }
\end{displayquote}

 Que tous les systèmes juridiques soient des constructions humaines et non des faits de nature, qu'ils reposent sur des idéologies, sur des prises de positions morales et des croyances plus ou moins partagées, et qu'ils fassent tous des choix entre des possibles, acceptant les uns et refusant les autres, cela est évident. Mais sur quels arguments se fonde l'affirmation que \emph{"L'adoption est une institution plus apte que la vérité biologique à assurer la stabilité des liens familiaux"} ?  On peut tout aussi bien affirmer le contraire. Il n'est pas nécessaire que la filiation adoptive soit meilleure que la filiation ordinaire pour etre reconnue par le droit. A devoir choisir entre le sang ou la volonté pour fonder le droit de la filiation il y a quelque chose qui paraît artificiel et forcé. Tout faire reposer sur la biologie est certes méconnaître qu'elle n'a jamais suffit dans aucune société pour légitimer une naissance, et oublier combien le lien entre un adulte et son enfant est un lien co-créé dans le cadre de leur relation, à l'instar d'une adoption réciproque qui déborde de tous côtés la proximité biologique. Mais ne reconnaitre que la volonté en déniant les corps et leurs dialogues est une fiction juridique, héritée des romains, qui comme toutes les fictions juridiques fait plus ou moins violence aux réalités telles qu'elles sont vécues au jour le jour, dans la complexité et l’ambivalence.
 
 \emph{"L'adoption est une institution conçue à partir du droit de l'enfant à avoir une famille, contrairement à la filiation biologique qui apparaît plutôt comme un dispositif du droit à l'enfant."} Il est vrai qu'en France l'adoption des jeunes enfants, formellement interdite depuis la fin de l'antiquité, a été ressuscitée après la Grande Guerre pour donner des parents à des enfants abandonnés. Il est vrai aussi que pour chaque enfant adoptable il y a actuellement plusieurs candidats à l'adoption. Au plan mondial il en est de plus en plus de même : l'enfant naturel devient de plus en plus rare. Il n’est donc pas nécessaire de promouvoir l’adoption : du point de vue des enfants adoptables elle se porte plutôt bien. La valorisation actuelle de l'adoption vient non de la prise en compte de l'intéret des enfants sans parents mais de la prise en compte du désir d'enfant de ceux qui ne peuvent ou ne veulent pas recourir aux relations hétérosexuelles, désir d'enfant qui en soi n'est ni plus ni moins légitime que celui des autres.  
 
 Mais dès que les enfants sont sortis de la petite enfance, leur adoption n'est pas simple et elle peut être terriblement éprouvante pour le narcissisme des adoptants. Les adoptés courent bien plus que les autres enfants le risque d'être rejetés à cause des difficultés de tous ordres qu'ils ont rencontrées du fait de leur histoire et auxquelles ils se sont adaptés comme ils ont pu. Leurs attentes ne s'engrènent pas toujours harmonieusement avec celles de ceux qui se proposent de devenir leurs parents adoptifs : le pourcentage de ces enfants qui sont abandonnés une deuxième fois après une adoption n'est malheureusement pas négligeable. Tous les adultes ne sont pas prêts à prendre de pareils risques. 
Jusqu'ici il n'a pas été permis par la loi de concevoir des enfants pour les donner, mais si l'on voulait répondre à toutes les demandes d'adoption il faudrait s'y résoudre. 

Tant que dureront les énormes inégalités de revenu observables sur cette planète, les plus fortunés pourront toujours louer le ventre des plus belles et des plus saines des filles des pauvres, de la même façon que les riches romains achetaient les plus jolies des jeunes esclaves afin qu'elles leur fassent des enfants bien à eux qu'ils n'auraient à partager ni avec un partenaire égal à eux en dignité, ni avec une belle famille aussi puissante que la leur. Le recours à des mères porteuses est dans la logique des évolutions libérales actuelles. Il est d'ores et déjà légalement possible dans plusieurs pays développés. Est-il appelé à se généraliser ? Comment refuser ce recours aux hommes homosexuels si l'on accorde l'assistance médicale à la procréation (PMA) aux femmes homosexuelles, et comment le refuser à tous les autres, hommes et femmes célibataires, si on l'accorde aux hommes homosexuels ? Et comment le refuser à qui que ce soit si des femmes (ordinairement de condition modeste et vivant souvent dans des pays sous-développés) sont volontaires pour prêter leur ventre et abandonner leur enfant nouveau-né contre une indemnité suffisante. 

 C'est le seul moyen de mettre les hommes à égalité avec les femmes dans l'accès à l'enfant, ou plutôt de corriger l'inégalité que leur corps leur impose dans ce domaine, mis à part bien sûr le mariage traditionnel, monogame et indissoluble, dont c'était l'une des finalités. Lorsque leur mariage était rompu les pères romains gardaient leurs enfants : ils n'avaient donc pas spécialement intérêt à ce que les unions soient indissolubles. Par contre leurs épouses avaient de bonnes raisons de craindre d'être répudiées et \emph{ipso facto} séparées de leurs enfants. Elles ont trouvé bon à partir du IVème siècle d'être mieux protégées contre ce risque et d'avoir leur vie durant l'exclusivité de la fécondité légitime de leur mari. Aujourd'hui où leur autonomie financière et les lois leur permettent de prendre l'initiative de quitter leurs maris sans risque de devoir lui laisser leurs enfants la situation se retourne et ce sont les hommes qui peuvent commencer de craindre d'être séduits puis abandonnés. 
 
L’instauration d'une déclaration formelle de renoncement à la paternité, proposée par Borillo en miroir du droit reconnu aux femmes à l'accouchement sous X, peut paraître provocante. Pourtant une telle disposition ne ferait que rejoindre le point de vue des révolutionnaires de 1789 : pas plus de contrainte en paternité qu’en maternité. Plutôt que de se retrouver un jour contraints de continuer de payer pour leurs enfants sans plus les avoir auprès d'eux, tandis que parfois un autre qu'eux les éduque, les hommes pourraient choisir, quelle que soient par ailleurs leurs préférences sexuelles, de commencer par payer pour les posséder sans partage afin que leur génitrice ne puisse jamais les leur contester. Sur quels arguments fonder le refus d'une pareille évolution ? Elle ne serait au fond que le miroir de celle qui voit des femmes choisir en toute connaissance de cause de faire un enfant toutes seules. Si les humains ne diffèrent en rien de significatif en dehors de leurs caractéristiques biologiques, si les femmes n'ont pas besoin d'un homme pour élever un enfant, alors les hommes n'ont pas non plus besoin d'une femme pour élever leurs propres enfants.
 
 Le recours aux mères porteuses ne pourrait être interdit, en dépit de la pression des demandes individuelles et du modèle fourni par les pays où cette pratique est autorisée, que s'il était d'abord admis qu'il implique la réduction d'un humain au statut d'instrument de la volonté d'un tiers jusque dans son corps, ce qui est la définition de l'esclave, et s'il était reconnu que c'est inacceptable, même si cette personne a donné son accord. Une deuxième raison serait que cette pratique fait de l'enfant à naître le produit d'un contrat commercial (sauf à ce que se généralise le don d'enfants par des femmes qu'aucune nécessité matérielle y pousserait, mais rien ne montre qu'on aille vers là).
  
  Mais refuser le recours à des mères porteuses impliquerait aussi d'accepter l'idée qu'il n'existe pas de droit à l'enfant, c'est-à-dire que tout un chacun peut être irrémédiablement privé d'enfant en dépit de ses désirs les plus authentiques et les plus légitimes sans avoir pour autant droit à la répara-tion de cette injustice. Le mouvement des pratiques depuis un demi-siècle ne va pas dans ce sens.

Quant à espérer sortir de ces contradictions en recourant à un utérus artificiel, c'est encore et pour longtemps de l'utopie.

Est-ce que le recours à une adoption ou à une mère porteuse est aussi satisfaisant pour les enfants concernés que pour leur(s) parent(s) ? On aimerait que ce soit le cas, mais beaucoup d'adultes nés d'une insémination artificielle avec donneur (IAD) n’en expriment pas moins le désir de connaître leurs « origines ». On pourrait postuler que si la filiation adoptive était instituée comme le modèle de la filiation la réalité des parents de naissance perdrait de son importance, mais ce n'est qu'une hypothèse. Beaucoup parmi les jeunes et les adultes nés sous X veulent connaître au moins leur génitrice, et ceux à qui cela est refusé disent souffrir d'une peine inguérissable. A défaut de pouvoir exiger d'être élevés par leurs deux parents de naissance, les enfants concernés (beaucoup d'entre eux) veulent au moins les connaître et même si l'on ne voit pas tou-jours à quoi cela pourrait leur servir, eux le voient et ils s'obstinent. Même si on le leur refuse ils continuent de le vouloir. Et au nom de quoi pourrait-on les en empêcher ? Ils ont le droit pour eux au moins autant que les adultes ont le droit de vouloir un enfant.
 
 Certes tous les jeunes nés sous X ou d'une IAD ne sont pas tourmentés par ces interrogations, mais cela ne peut que rendre dubitatif. Même si la tentation d'escamoter les géniteurs est acceptée par les adoptants, elle n'est pas soutenable dans la durée.  Même lorsque les parents légaux s'interdisent d'en parler, les parents de naissance font irrémédiablement partie de la relation entre eux et leurs enfants adoptés. 

\section{« Papaoutai »} 


 
Dans \emph{Quelle alternative au patriarcat ? Valoriser un modèle social non conjugal} (2004), Agnès \fsc{echene} accuse le couple hétérosexué d'être le lieu privilégié d'expression et de transmission de la violence masculine, et cela trop souvent avec la complicité (masochiste) féminine. Elle en tire la conclusion qu'il faut éliminer la paternité en tant que telle :
\begin{displayquote}
\emph{« ...ce n'est qu'en valorisant le modèle social non conjugal qu'une société peut se défaire du patriarcat. Il importe donc de favoriser une sexualité libre et variée, tout en étant discrète et protégée, surtout chez nos propres enfants ; peu importe dès lors qu'elle soit ardente ou paisible, monotone ou changeante, homophile ou hétérophile, dès l'instant qu'elle reste une affaire personnelle dont nul ne se mêle. Une telle évolution nécessite également une reconsidération du modèle familial qui doit se re-fonder sur des liens d'appartenance utérine et non pas consanguine ; cela remet en cause dès lors la paternité génitale qui doit laisser place à une paternité germaine : il faut en effet que ce soit les frères, oncles et cousins [de la mère] qui assument les enfants des femmes ; de nombreux signes avant-coureurs montrent qu'ils sont prêts à le faire et qu'il ne manque qu'un déclic. Mais il faut aussi que les femmes renoncent à obliger les géniteurs à être pères ; il faut qu'elles abandonnent toute velléité de recherche de paternité, de pension, partage, alternance,~etc. et se tournent résolument vers leurs frères, oncles et cousins pour « donner » des pères à leurs enfants, qui ne s'en porteront pas plus mal. »}
\end{displayquote}

On est là apparemment à peu près dans le monde de l'ethnie Na (groupe chinois de l’Ouest aux confins du Tibet) dont les membres ne reconnaissent pas de pères, et où les hommes de chaque famille sont les amants librement choisis, pour une nuit ou pour plusieurs, des femmes des familles voisines. Ces familles reposent sur un principe matriarcal, puisque les enfants appartiennent exclusivement à la famille de leur mère. Les oncles maternels "paternent" les enfants de leurs sœurs et ne font rien pour ceux de leurs amantes. Mais il ne suffit pas que les femmes Na aient le choix des géniteurs de leurs enfants pour que l'autorité dans le groupe familial leur soit dévolue. La répartition sexuée des rôles dans les familles Na est rigoureuse et les hommes y ont leurs domaines réservés, notamment les relations avec le monde extérieur.

D'autres imaginent des constellations d'une tout autre espèce, des associations de personnes basées sur des contrats de solidarité privés ne se référant plus au couple, mais plutôt aux communautés créées dans les années 60-70 du XXème siècle. Selon Marcella Jakub :
\begin{displayquote}  
\emph{"A l'époque de la discussion sur le pacs, certains avaient proposé de créer des liens de solidarité entre plusieurs individus, et pas uniquement au sein du couple, qu'il soit hétérosexuel ou homosexuel. Le pacs aurait pu permettre, par exemple, d'associer des personnes au moyen de liens juridiques alternatifs qui ne soient pas forcément fondés sur la famille. Voilà une proposition sociale intéressante, qui aurait permis d'inventer des formes de vie à plusieurs. Mais nous sommes loin d'une telle réflexion ».}
\end{displayquote}
 
 Va-t-on de manière plus banale vers des foyers constitués d'une femme et des enfants qu'elle a mis au monde, autour desquels graviterait la nébuleuse de ses amants et ex-amants ? Dans cette hypothèse, les hommes de demain auraient des enfants de plusieurs femmes, enfants vivant ordinairement chez leurs mères, si bien que le poids de leur parole auprès de chacun d'eux serait à peu près nul ? Serait-ce en quelque sorte l'inverse de la situation du \emph{pater familias} romain, qui pouvait demander à plusieurs femmes des enfants sur lesquels lui seul aurait autorité et qui vivraient tous chez lui s’il le voulait ?
 
 
 C'est ainsi que fonctionne le modèle matrifocal « antillais » ou « caraïbe » dont l'origine se situe dans l'histoire du peuplement des Antilles. On a vu que les esclaves n'ont par définition aucun des attributs juridiques d'un père ou d'une mère sur les enfants dont ils sont les géniteurs : seuls les propriétaires des génitrices possèdent des droits sur les enfants de celles-ci. C'étaient ces propriétaires qui faisaient d'elles des mères \emph{lorsqu'ils leur confiaient la garde} des enfants qu'elles avaient portés, quel qu'en ait été le géniteur \footnote{Qu'ils aient été eux-memes ce fécondateur ou que ç'ait été un esclave ne changeait rien à leur statut de propriétaire de la fécondité de leurs esclaves femmes.}. Il y avait une espèce d'alliance de fait (alliance sous contrainte, perverse) entre les génitrices et leurs maîtres pour élever les enfants qu'elles avaient mis au monde, tandis que leurs (autres) partenaires sexuels n'avaient pas droit à la parole et étaient réduits, quel qu'ait pu être leur désir,  à n'être que des donneurs de sperme. 
 
 



 
Jusqu'aux années 60 du XXème siècle c'est l'excès de présence et de poids des pères qui était présenté comme un problème. Aujourd'hui ils semblent n'être jamais assez présents, jamais là où il faut. Il ne semble plus possible de les penser comme les relais, les représentants d'un Dieu, de la Cité, de la République, de l'Empereur, du Roi ou de l'État. Dans l'effritement de leur image, Françoise \fsc{HURSTEL} pointe trois moments clé : la loi de 1889 contre les « \emph{parents indignes} », la loi de 1935 abolissant le droit de « \emph{correction paternelle} » et la loi de 1938 abolissant la « \emph{puissance maritale} ». Ont été abolies toutes les dispositions juridiques sur lesquelles était fondé dans le passé l'exercice masculin d'un rôle patriarcal. Le résultat est que « [...] \emph{nous ne savons plus ce qu'est la place d'un père et ce que sont ses fonctions} », et que « \emph{ce ne sont pas des petits bouts de la paternité qui ont changé, mais l'ensemble du système a muté avec la mort du \emph{pater familias}.} »%
% [3]
\footnote{Françoise \fsc{HURSTEL}, « Penser la paternité contemporaine dans le monde occidental : quelles places et quelles fonctions du père pour le devenir humain, sujet et citoyen des enfants ? », in \emph{Neuropsychiatrie de l'enfance et de l'adolescence}, 53 (2005) 224-230.} 


 
  Si les lois suivaient l'évolution des mœurs, alors la promulgation d'une loi serait le signe que les esprits sont prêts à l'accueillir. Si cela était vrai, alors on aurait dû observer durant les années précédant la promulgation de chacune des lois ci-dessus, un mouvement de l'opinion publique stigmatisant les parents indignes, le recours abusif au droit de correction paternelle, ou le scandale que constitue l'existence d'une puissance maritale. Selon Françoise \fsc{HURSTEL} ce n'est pas ainsi que cela s'est passé. Au contraire ce n'est qu'à partir de la promulgation de la loi de 1889 que la presse aurait commencé de dénoncer les carences des pères « indignes%
% [4]
\footnote{« \emph{alcoolique, pauvre, inculte et violent} », Françoise \fsc{HURSTEL}, \emph{la déchirure paternelle}, p. 113.} 
 ». Et de même ce n'est que vers 1942 que les spécialistes de l'éducation auraient commencé de dénoncer les pères sans autorité, tandis que la notion de carence n'aurait envahi les écrits qu'à partir de 1950 :
 
\begin{displayquote}
\emph{"C'est donc quelques années après la promulgation de ces lois faisant disparaître des textes juridiques les termes de puissance (maritale) et ceux de correction paternelle tout en maintenant ceux de chef et d'autorité (paternelle), qu'est décrite cette figure d'un père manquant d'autorité et de sévérité ; et que les spécialistes admonestent les pères d'une position qui est bien celle de chef de famille."}
\end{displayquote}

Françoise \fsc{HURSTEL} soutient que ces discours sont l'effet de ces changements législatifs et non leur cause.  Selon elle, l'opinion publique n'aurait appelé aucune de ces lois de ses vœux. Ces réformes n'auraient été imaginées, réclamées, et parfois discrètement expérimentées que par les seuls experts, médecins, administrateurs, juges et travailleurs sociaux directement intéressés à leur mise en œuvre. Pour elle, tous les discours sur les déficiences des pères actuels ne sont que des productions imaginaires qui coexistent avec des réalités qui n'ont pas grand-chose à voir avec elles. En effet, les enquêtes sur le terrain ne montrent rien qui permette de croire que les pères d'aujourd'hui seraient dans l'ensemble moins attentifs et moins présents que ne l'étaient ceux du passé
\footnote{... mais cela exige d'éviter les biais méthodologiques. Il faut notamment que ces enquêtes ne se placent pas consciemment ou inconsciemment du seul point de vue des mères. Cf. Germain \fsc{DULAC}, « La configuration du champ de la paternité : politiques, acteurs et enjeux », in \emph{Politiques du père, numéro spécial de Lien social et politiques}, (n° 37) 1997, p. 133-142.}%
. Certes il y a des pères qui sont incompétents, irresponsables ou délinquants, mais cela n'a rien de nouveau, et rien ne permet d'affirmer qu'il y en ait plus qu'autrefois. Les propos tenus ne portent pas tant sur ce que font réellement les pères que sur ce qu'ils devraient faire dans l'idéal pour être de bons pères. Pour Françoise \fsc{HURSTEL} il s'agirait, à l'aide de ces discours, d'asseoir l'autorité de ceux qui prétendent savoir ce qu'est un bon père et qui sont les bons pères. En somme ce seraient les experts de l'éducation qui prétendraient apprécier la conformité des parents à leurs devoirs, ainsi que le préconisait Ernest \fsc{TARBOURIECH}\footnote{Il allait de soi pour lui que les médecins étaient seuls qualifiés pour la place d'experts en dernière instance (il était lui-même médecin).} de telle manière qu'à la fin du processus : {\emph{"le père et la mère n'auront sur leur progéniture aucun droit d'aucune sorte, mais seulement des devoirs\footnote{in \emph{La cité future}, 1902}"}. 


 
\begin{displayquote} En conclusion :
\emph{"Du point de vue de la paternité les hommes de la période contemporaine n'auront pas été gâtés. Je propose une image pour illustrer ce que peut être la notion de carence : lorsqu'un homme devient père, il endosse un pardessus plein de trous et de soupçons..., plus précisément une image de plus en plus dévalorisée, et cela quelle que soit la valeur personnelle de l'homme qui assume une telle fonction. Et ce qui les caractérise est un discours dévalorisant des spécialistes ; tellement dévalorisant qu'il apparaît, en fait, comme un discours de l'exclusion des pères... au profit du super père spécialiste. Si les pères peuvent être dits carents \emph{[en Droit, le père « carent » est celui qui ne laisse rien à ses enfants, qui ne leur laisse aucun héritage]}, c'est parce qu'ils sont relégués à cette place par ceux-là mêmes qui normalisent les pratiques autour de l'enfant. Nous dirons que ces pères carents sont en fait d'abord des pères exclus par les théoriciens de l'éducation.}%
% [6]
\footnote{Françoise \fsc{HURSTEL}, \emph{la déchirure paternelle}, p. 112-113.} 

[...] \emph{Ainsi les signifiants inscrits dans la loi produisent des effets imaginaires qui se repèrent dans les représentations collectives, les modèles normatifs du père et les pratiques sociales.} 

\emph{Je ferai ici un pas de plus et avancerai ceci : non seulement les signifiants des lois produisent des effets imaginaires, mais encore les lois elles-mêmes ne sont connues que par le biais de ces productions...}

\emph{Les figures du père carent semblent bien avoir une fonction sociale et idéologique importante, celle d'être l'une de ces fonctions sociales qui rendent compte et qu'il y a du père dans notre société (au sens du père symbolique et de la fonction paternelle) et qu'il y a du changement dans les montages qui instituent le père... bref, elles seraient un mode d'historicisation d'une structure.}

\emph{Mais en retour cet imaginaire du père marquera chaque homme ayant à assumer la fonction paternelle, chaque mère appelée à reconnaître qu'il y a du père pour son enfant."}%
% [7]
\footnote{Idem, p. 113-115.}
\end{displayquote}

 Les mères  sont traditionnellement placées du côté de l'accueil de la vie et de son entretien, de l'intime, de la tendresse, du cœur (du \emph{care}). La fonction maternelle a toujours été valorisée et presque sacralisée, mais aujourd'hui cette idéalisation n'est plus contrebalancée par celle qui auréolait les pères et la fonction paternelle des siècles classiques. D'ailleurs maintenant que le capital le plus utile c'est le capital intellectuel, maintenant que l'avenir des enfants se prépare à coup d'études longues, financées en grande partie par la collectivité, sous la houlette de professionnels de l'enseignement et sous le contrôle de l'État, qu'est-ce qu'un père pourrait bien transmettre à ses enfants, à part ses biens, sans menacer leur autonomie ? Dans un environnement allergique à tout ce qui ressemble à du paternalisme, qu'est-ce qu'un homme est autorisé à désirer concernant des enfants ? Des points de vue et des désirs spécifiquement masculins sur les enfants sont-ils même acceptables ? 
 
 La déploration des déficiences des pères, de leurs fragilités et de leur propension à fuir devant les responsabilités est un passage obligé de tout discours sur les familles, tandis que l'idée qu'ils puissent mettre en oeuvre leurs forces ou leur puissance dans une relation avec des enfants suscite des représentations de violence et de maltraitance. Quand on parle sans les spécifier des violences conjugales, il va de soi qu'il s'agit des violences masculines ...alors que toutes les enquêtes et recherches rigoureuses montrent que les femmes sont très capables de concurrencer les hommes de manière significative dans le domaine de la violence \emph{aussi} et qu'elles sont loin d'être sans défenses. Il est de fait qu'il y a une très grande différence entre les hommes et les femmes en ce qui concerne la violence \emph{physique} et les comportements sanctionnés par les lois. L'immense majorité des violeurs sont des hommes. De même la majorité des actes de violence (sexuelle ou non) visant les corps (des femmes, des enfants ou des autres hommes) sont posés par eux.  
 
 

\begin{displayquote}
 \emph{"Viols et violences, mépris et humiliation des femmes et des hommes dévalorisés qui leur sont assimilés, cynisme, manque de pensée et appauvrissement affectif : la représentation des hommes qui exsude d'une lecture attentive des recherches qui leur sont consacrées est suffocante. Quels que soient les champs disciplinaires et les orientations théoriques, la virilité désigne l'expression collective et individuelle de la domination masculine et ne saurait donc constituer une définition positive du masculin\footnote{Molinier Pascale \emph{Virilité défensive, masculinité créatrice}, in  \emph{Travail, genre et société}, n° 3, mars 2000.}".}
 \end{displayquote}
 
 
 
 C'est entre l'entrée dans l'adolescence et l'age de 25 ans que le risque de poser des actes sanctionnables par la loi est le plus grand et les jeunes hommes de ces ages représentent l'écrasante majorité des personnes condamnées. 
Il est fréquent que les adolescents traversent des états d'incertitude identitaire, avec les malaises que cela implique, étant données toutes les métamorphoses par où ils passent. Mais encore plus déstabilisantes pour eux sont les incertitudes identitaires de leurs adultes de référence. Si même les adultes d'aujourd'hui ne savent plus ce qu'est un homme, à quoi les garçons peuvent-ils se mesurer ? les discours qu'ils entendent sur ce que c'est qu'un homme sont souvent insupportables.
 C'est pourquoi ce n'est pas par hasard si aujourd'hui ce sont eux qui plus que les filles expriment bruyament leur désarroi : violences contre eux-mêmes, contre les personnes et contre les biens, prises de risques inconsidérées, désinvestissement scolaire, etc.
  Tous les hommes sont concernés par la puissance physique, la leur et celle des autres, et il faut que chacun se débrouille pour trouver comment gérer celle qui lui est échue. La plupart y parviennent vaille que vaille et s'abstiennent de tout acte de violence à l'encontre d'autrui, que ce soit parce qu'ils n'en a pas envie ou parce qu'ils se l'interdisent. Combien y parviennent seulement en inhibant leurs forces et sans trouver à les canaliser de manière féconde ? Comment favoriser chez les hommes une meilleure gestion de leur puissance ? C'est une question d'éducation, bien évidemment, mais cela ne s'enseigne pas. 
 Il faudrait sans doute commencer  par accepter l’idée qu'il peut exister quelque chose comme des valeurs masculines, ou une manière masculine de faire vivre les valeurs universelles. 


\begin{displayquote} \emph{« ...dans la perspective proféministe, on ne peut vouloir à la fois que le genre disparaisse comme système hiérarchique et que les catégories du masculin et du féminin continuent d'exister. Mais pour d'autres auteurs, le terme de masculinité marque la volonté d'analyser s'il est possible d'être un homme sans coller aux stéréotypes de la virilité, d'une part ; sans devenir une femme, d'autre part\footnote{Virilité défensive, masculinité créatrice, in \emph{Travail, genre et sociétés
2000/1 (N° 3)}, parPascale Molinier.}."} 
\end{displayquote}

 
 
 Il est possible que la banalisation de couples d’hommes avec bébés renouvelle l'abord de ces représentations ? Dans le même sens la coexistence de couples de femmes, d’hommes et de couples mixtes, avec et sans enfants, permettra des approches moins stéréotypées des dynamiques conjugales.
 







\section{A quoi sert une famille ?} 




 
 
 Dans le second récit de création de la Genèse (Gn 2, 18-24) la relation d'Adam et Eve est inégalitaire. Selon les traducteurs de la Bible Eve est \emph{à côté d'Adam} ou bien \emph{contre lui} (ou les deux à la fois) mais de toute façon elle \emph{lui} est donnée pour que \emph{lui} ne soit pas seul, pour le \emph{compléter}. Elle semble n'être qu'un \emph{complément}. Avec ou sans Bible les hommes ont toujours été d'accord avec cette façon de penser et ont partout trouvé normal de se définir comme chefs de famille. Ils ont toujours trouvé \emph{naturel} de classer les tâches en anoblissantes et viles, de s'attribuer celles qu'ils considéraient comme nobles et de vouer les femmes aux autres. Ils ont partout bridé l’efficacité et la productivité de ces dernières en ne leur accordant pas l’accès aux meilleurs outils, en les tenant à l’écart des apprentissages et des savoirs\footnote{Les Mains, les outils, les armes [article], Paola Tabet
in \emph{L'Homme, Année 1979, Volume 19, Numéro 3}, pp. 5-61.
Paola Tabet, \emph{La Construction sociale de l’inégalité des sexes. Des outils et des corps}
Paris-Montréal, L’Harmattan, 1998, 206 p.}. Ils ne leur ont reconnu une légitimité qu’au sein du foyer, auprès des petits enfants, des malades et des grands vieillards, et dans les activités ménagères. 

Cela n'interdit pas de se demander en quoi il pourrait être bon de vivre avec un(e) autre strictement égal(e) à soi ? En effet il y a des avantages à vivre en couple. Les avantages subjectifs sont évidents (avoir sous la main un appui, un autre avec qui parler, un partenaire sexuel, etc...) mais ils sont subjectifs, et chacun les appréciera à sa façon. 

Mais il y a aussi des avantages matériels à se marier et ceux-là sont tout à fait chiffrables : partager un même logement, un seul loyer, un seul crédit immobilier, un seul jeu d'équipement domestique, une seule voiture familiale, les mêmes avantages fiscaux, etc.. Si ce qui évite la pauvreté c'est d'abord l'emploi, la vie en couple n'en est pas moins un moyen efficace d'optimiser sa situation. Si l'on en croit l'INSEE\footnote{\emph{Couples et familles}, édition 2015, Insee références} dans les familles avec au moins un enfant mineur où les deux adultes travaillent le taux de pauvreté est de 4 \% contre 22 \% dans les familles monoparentales (85~\% de mères seules) où le parent travaille, et 76~\% quand elle (il) ne travaille pas. Les familles monoparentales ont un revenu par individu massivement inférieur à celui des couples avec enfants : en 2011, 1240 euros en moyenne contre 1880. 7 familles "traditionnelles" sur 10 sont propriétaires de leur logement, contre 6 familles recomposées sur 10, 4 personnes seules sur 10, et seulement 3 familles monoparentales sur 10. 

La perte de revenus après une rupture (mariage, pacs ou union libre) est significative  : en 2010 après impôts, allocations diverses et pensions alimentaires, les hommes perdent en moyenne 3 pour cent de revenus et les femmes 23 pour cent \footnote{idem, page 59.}. Il en est de même en ce qui concerne les couples \emph{sans enfants à charge} au moment de la rupture  :  les hommes subissent une perte de 10,5~\% en moyenne et les femmes de 23~\%. L'écart entre les pertes moyennes des hommes et celles des femmes vient de l'écart entre leurs revenus d'activité : Les trois-quarts des hommes en couple gagnent plus que leurs compagnes (et à peine plus de 10~\% gagnent moins), soit parce que celles-ci ne travaillent pas, ou pas à plein temps, ou parce qu'en moyenne les qualifications professionnelles des hommes sont un peu plus élevées que celles de leurs compagnes, ou bien parce que une préférence est accordée aux hommes dans les promotions à compétence et implication égales. Comme sept fois sur huit aucune prestation compensatoire n'est accordée après une rupture le niveau de vie moyen des femmes baisse significativement. Il n'y a que les hommes dont les revenus d'activité sont nettement plus élevés que ceux de leurs ex-conjointes qui en moyenne \emph{gagnent} à une rupture, et ce d'autant plus qu'ils ont plus d'enfants à charge ! Autrement dit presque tout le monde gagne \emph{financièrement} à vivre en couple. 

Il n'est donc pas difficile de comprendre que la vie en couple (marié, pacsé ou en union libre) soit largement plébiscitée : à 55 ans seuls 7 ou 8~\% des hommes et des femmes n'ont jamais vécu en couple. Ce mode de vie est si valorisé que dès qu'ils le peuvent ceux qui ont rompu une relation s'engagent dans une nouvelle relation. En 2013 19~\% des personnes âgées de 26 à 65 ans ont vécu deux relations de couple, 5~\% ont vécu trois relations. 

Au marché du mariage (du Pacs, de l"union libre et du remariage...) les femmes jeunes sans enfants et les plus diplômés (hommes et femmes) ont la plus forte cote. Les hommes qui ont les diplômes et les revenus les plus élevés et qui ne perdraient rien à vivre en célibataires s'engagent encore plus que les autres dans un couple. Les femmes les plus diplômées sont à la fois les plus nombreuses à vivre en couple et les plus nombreuses à vivre seules (après 45 ans) : elles peuvent plus facilement que les autres se passer de l'aide d'un conjoint ? ou bien elles sacrifient leur vie de famille à leur carrière ? A contrario ce sont les femmes qui n'appartiennent pas aux catégories professionnelles les plus élevées qui perdent le plus à une séparation, surtout si elles ont des enfants (même si les prestations sociales ont un effet significatif sur les écarts de revenus). Les jeunes femmes s'engagent un peu plus vite que les jeunes hommes dans une vie de couple, mais au fil des années la situation s'inverse. Le délai entre la fin d'une relation et l'engagement dans une autre est deux fois plus bref pour les hommes, même avec enfants mineurs, et surtout pour les plus diplômés, que pour les femmes avec enfants mineurs. Après 60 ans le nombre d'hommes vivant en couple est beaucoup plus élevé que celui des femmes et à 80 ans ils vivent en couple trois fois plus souvent qu'elles. Dès 76 ans la majorité des femmes vivent seules, ce qui n'est le cas des hommes qu'à partir de 90 ans (le petit nombre de ceux qui ont survécu).

 Depuis qu'elle existe et quelles que soient les formes qu'elle a pu revetir la famille a toujours été une institution protectrice pour ses membres. Depuis la fin du XIX ème siècle la montée en puissance des états-providence s'était traduite par un desserrement progressif  des contraintes familiales, desserrement accompagné par l'évolution législative\footnote{l'un des deux facteurs a-t-il été plus important que l'autre ?}. La crise économique interminable qui depuis la fin des "trente glorieuses" frappe ces memes états\footnote{...à moins qu'on ne soit simplement sortis d'une période exceptionnelle et retournés à la médiocrité habituelle de l'histoire économique ?} les contraint à faire des économies et à se montrer plus regardants. Elle a permis de vérifier que plus les temps sont difficiles plus l'assistance intra-familiale est indispensable et aussi que l'on peut le plus souvent compter sur elle ...encore faut-il faire partie d'une famille ! Preuve en est le pourcentage scandaleusement élevé des anciens enfants de l'Aide Sociale à l'Enfance qui se retrouvent à la rue. Celle-ci les abandonne à leur sort dès leurs 18 ans, sauf s'ils acceptent de se comporter de manière exemplaire comme de bons jeunes gens appliqués, dociles et studieux. Ils ont une histoire qui ne les y porte pas aussi naturellement que les enfants de famille, mais l'état est pour eux infiniment moins patient que leurs parents ne le sont pour ces derniers et on ne leur pardonne aucun écart de conduite. 

Les avantages fournis par la vie en couple sont d'autant plus importants qu'on est plus pauvre, moins bien inséré dans le monde professionnel et/ou moins bien équipé mentalement pour faire face aux situations difficiles. Les séparations des couples ne sont jamais anodines et elles peuvent être financièrement ruineuses. La pauvreté frappe d'abord les femmes seules avec enfants, mais aussi les hommes seuls dont les ressources sont insuffisantes pour trouver un logement, etc. Il faut y ajouter les coûts psychologiques des séparations, en partie inséparables des coûts matériels, ainsi que le note Gérard Neyrand  :  
\begin{displayquote}
\emph{« En effet, les nouvelles valeurs familiales sont portées par les couches moyennes cultivées et sont devenues système de référence global. Leur confrontation aux habitus des couches populaires en la matière ne s'effectue pas sans conflits (Commaille, Martin, Les enjeux politiques de la famille, Paris Bayard, 1998). L'une des issues des contradictions entre ces systèmes différents de références, qui traversent différemment les individus selon leur sexe et leur position sociale, réside dans la fréquence des séparations conjugales conflictuelles, la monoparentalisation maternelle qui s'en suit et la précarisation des foyers monoparentaux ainsi définis. Leur caractéristique est bien d'être soumis à un double système de contraintes croisées, socioéconomiques et psychorelationnelles.
La montée du chômage et la précarisation des emplois les moins qualifiés\footnote{ Boltanski, Chiapello, Le nouvel esprit du capitalisme, Paris Gallimard, 1999}, contribuent à une fragilisation globale des situations familiales des plus démunis, qui risque d'autant plus de déstabiliser les familles que ces familles populaires se pensent de façon unitaire, quasi-symbiotique.
Elles sont basées sur un couple conçu comme une entité indissoluble, un « couple unité organique » selon l'expression d'Irène Théry\footnote{Le couple occidental et son évolution sociale : du couple « chainon » au couple « duo », \emph{Dialogue}, \no 150, 4e trimestre, 2000}, et sont loin d'adhérer sans réserve au nouveau modèle moderne du « couple duo ». La séparation, dès lors, constituera une catastrophe identitaire dont beaucoup auront du mal à se relever, en particulier les pères.
On conçoit alors l'importance des difficultés que des séparations dans un tel contexte peuvent générer :
- difficultés relationnelles entre les ex-conjoints et dans le rapport des pères à leurs enfants,
- et difficultés socio-économiques des mères confrontées aux nécessités d'une survie familiale qu'elles doivent bien souvent affronter seules.
Mono-parentalisation et précarisation s'avèrent alors intimement liées. »}
\end{displayquote}


La séparation des amants dont la passion s'est refroidie ne va pas dans le sens du renforcement de leurs capacités éducatives. La dissolution du couple parental multiplie le nombre des situations où la fonction éducative de l'un ou de l'autre est plus ou moins disqualifiée ou à tout le moins entravée, tandis que son remplacement au quotidien par le ou les partenaire sexuels et affectifs de l'autre n'est pas forcément bien accepté par les enfants concernés et ne présente pas toujours l'efficacité éducative souhaitable. Le nombre s'élève donc des parents qui face aux problèmes éducatifs que posent un jour ou l'autre presque tous les mineurs sont seuls et/ou en difficulté. 

Lorsque les mineurs posent problème, notamment à l'adolescence, à des parents trop envahis par leurs problèmes personnels ou trop éloignés pour intervenir utilement, des aides extérieures sont souhaitables, mais rien ne garantit que l'efficacité des prises en charge éducatives ainsi apportées soit supérieure à ce qu'en d'autres circonstances les parents auraient pu assumer eux-mêmes : ce serait déjà bien si on pouvait être assuré qu'elle n’est pas inférieure. D'autre part même en tenant pour négligeable la disqualification et la mise en dépendance des parents par les spécialistes du contrôle social des familles et les experts de l'éducation, les soutiens que propose la collectivité ne sont pas gratuits (ex. :  internats scolaires, assistance éducative, placement en famille d'accueil,~etc.). On passe de l'auto production des tâches éducatives à leur externalisation et à leur professionnalisation. Comme c'est un domaine où il n'y a aucun gain de productivité à espérer cela accroît les coûts éducatifs de manière très sensible. Jusqu'où peut-on aller dans cette voie avant que les électeurs n'estiment que c'est trop cher payé ?


 
En conclusion il serait peut-être difficile de prouver chiffres en main que la famille traditionnelle est supérieure à toutes les autres manières d'organiser la vie des individus, mais il est au moins aussi difficile de croire que du seul point de vue de la société la stabilité des unions "productrices" d'enfants n'a aucun intérêt et est définitivement dépassée. 

Par contre du point de vue des entreprises et des administrations la disponibilité d'un(e) employé(e) est plus grande lorsqu'il n'est plus nécessaire de tenir compte de son désir de vivre avec un(e) partenaire lui-même (elle-même) bien inséré(e) professionnellement et qui lui (elle) aussi s'attend à être significativement investi(e), de même qu'un(e) célibataire sans enfants et sans intention d'en avoir est précieux(se) pour son employeur et possède un atout significatif pour réussir une belle "carrière" professionnelle.





 