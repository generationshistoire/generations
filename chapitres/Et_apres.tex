% Le 25.03.2015 :
% Â_ --> À_
% ~etc.
% même
% Le 22.03.2015 :
% A_ --> Â_
% Le 16.03.2015 :
% Moyen Âge
% Le 24.02.2015 :
% ~etc.
% Moyen-Âge
% ~\%
% fœtus


\part{Et après ?}

 \chapter{Fragilité du mariage d'amour ?}
 

 La Bible pose la relation de l'homme et de la femme comme le modèle, l'archétype de la relation humaine, plus exemplaire que toute autre relation, même fraternelle : \emph{" YHWH Dieu dit : "il n'est pas bon que l'homme soit seul. Il faut que je lui fasse une aide qui lui soit assortie.}" (Gn 2, 18). Et c'est joyeusement, pour ne pas dire amoureusement, qu'Adam investit celle que Dieu lui a donnée : \emph{"Alors celui-ci s'écria : "à ce coup c'est [Eve] l'os de mes os et la chair de ma chair !}" (Gn, 2, 23). Dans ce modèle la relation de toute femme et de tout homme est si intime qu'elle les rend au moins aussi proches que ne le sont les parents les plus proches :\emph{"C'est pourquoi l'homme quitte son père et sa mère et s'attache à sa femme, et ils deviennent une seule chair.}" (Gn, 2, 24). A la suite de Jésus (ex.: Mc, 10, 6-8) les chrétiens ont fait de cette parole le fondement de leur doctrine du mariage. 
 
 Mais quand Paul de Tarse parle de l'amour conjugal il ne le définit pas comme un sentiment à ressentir mais comme une tâche à réaliser. Il s'agit de chercher à plaire par ses actes à son conjoint (quel que soit son sexe) et de s'occuper de lui comme de son corps à soi. Il s'agit de lui reconnaître un droit exclusif sur sa propre sexualité, de pardonner ses défaillances, de se livrer pour lui, etc. 
 
Dans le passé nul n'a jamais cru que le désir (amoureux ou sexuel) était suffisant pour maintenir un couple dans la durée, même pas ceux qui ont plaidé pour que les jeunes gens aient la liberté de choisir leur conjoint. On disait que la passion est aveugle, qu'elle doit plus aux représentation du passionné qu'aux caractéristiques véritables de l'objet de sa passion, qu'elle est d'abord l'expression de son narcissisme et qu'elle est le lieu de toutes les illusions. De l'Antiquité à la Belle Époque les moralistes se sont méfiés du désir sexuel, y voyant une entrave à l'exercice de la raison, une force aveugle, inconstante, décevante et potentiellement destructrice de tous les liens et de tous les principes sur lesquels repose la société, d'où leur insistance sur l'éducation à la maîtrise des passions : contrôle des instincts, valorisation du jugement et de la raison, culture du devoir, entrainement à résister à la frustration, etc. Pour eux un amour ne méritait d'être qualifié de conjugal que lorsque s'étaient éteintes les premières flambées du désir charnel. 

Erasme, qui n'est suspect ni de pudibonderie ni d'ascétisme, et qui tenait le mariage pour un choix de vie aussi méritoire que le célibat consacré, n'en écrivait pas moins dans son \emph{Encommium matrimonii christani (Eloge du mariage chrétien},1526) : \begin{displayquote}{\emph{"... Les poètes appellent l'ardeur des amants une fureur et non point un amour. Car enfin, où la raison est éteinte, peut-il y avoir autre chose que de la folie ? ...ceux qui épousent des femmes imprudemment et sans jugement ont coutume de se repentir, mais quelquefois trop tard de ces malheureux mariages. Il arrive rarement qu'on regrette de s'être marié par l'avis des parents, et d'avoir pris par un choix mûr et délibéré des femmes qu'on puisse aimer toujours. Car tout ce qui se fait en ne consultant que nos passions n'a qu'un temps. Ce qui se fait au contraire par raison et par jugement est stable et dure longtemps". }}\end{displayquote}  

 C'est un constat similaire que fait Maurice Godelier à l'issue de ses enquêtes ethnologiques\footnote{\emph{Métamorphoses de la parenté}, librairie Arthème Fayard, 2004, réédition par Flammarion, collection Champs, Essais, 2010.}.  Selon lui le désir sexuel est fondamentalement asocial et même destructeur :
 \begin{displayquote}{\emph{La permissivité en matière de sexe s'arrête \emph{[...]}, dans toutes les sociétés, soit là où la formule d'alliance serait menacée, soit là où les rapports de coopération et d'autorité entre consanguins risqueraient de s'effondrer et, glissant les uns dans les autres, de disparaître (Na). Mais cette fois, ce n'est plus de la sexualité-reproduction qu'il est question, mais de la sexualité-désir qui, nous l'avons vu, est dans son fond asociale. Elle n'est jamais la base d'une coopération durable entre les individus, tant au sein du groupe où ils sont nés qu'entre lui et les groupes avec lesquels il est allié. Et ce n'est pas seulement le désir hétérosexuel qui unit et divise. \emph{[...]} C'est, nous l'avons dit, parce que le désir sexuel en lui-même est asocial qu'aucune société ne peut permettre que tout soit permis.}
 
 \emph{Et ce travail d'auto-domestication est toujours à recommencer, alors que le processus de domestication des plantes et des animaux semble avoir atteint ses limites. \emph{[...]} Partout la spontanéité du désir a dû être sacrifiée pour produire un ordre social qui est toujours en même temps un ordre entre les sexes et un ordre sexuel. Partout a dû être éliminé le caractère asocial de la sexualité, sacrifié le polymorphisme du désir, interdite la permissivité sexuelle généralisée pour que la société puisse s'organiser et se reproduire.}
 
[...] \emph{Cependant, sacrifier le caractère asocial de la sexualité n'est pas seulement un acte d'amputation.  C'est en même temps une sorte de création. C'est agir sur soi pour continuer non seulement à vivre en société, mais à produire de la société pour vivre, ce qui est le propre de l'homme et le séparera toujours davantage, chaque jour qui passe, des primates, ses lointains cousins.} (p. 632-636)}
\end{displayquote}

Même après Freud et la réhabilitation du désir que son oeuvre a initiée un ethnopsychiâtre d'aujourd'hui  comme Tobie Nathan\footnote{Interview faite par Cécile DAUMAS et Anastasia VÉCRIN (paru le 15 août 2014 dans Libération)} ne dit pas autre chose :

 \begin{displayquote}\emph{Qu'est-ce qui différencie la passion de l'amour, notamment conjugal ?} 
 
 \emph{- La passion, ce n'est pas l'amour. D'ailleurs, les Grecs avaient deux mots distincts. Philia signifie l'amour raisonnable - comme l'amour conjugal - ou l'amitié, tandis qu'Eros désigne le désir, la passion amoureuse. Platon la caractérisait par le manque. Observation exacte, mais insuffisante. Il s'agit d'une exacerbation du manque - dans la passion, l'autre me manque quand il n'est pas là, il me manque quand il est là, car il n'y est jamais suffisamment ; dans la relation sexuelle, et même au moment de l'orgasme, il me manque encore. Recherche d'une fusion impossible, pulsion à offrir à l'autre tout votre espace intérieur - la passion amoureuse est une folie. La seconde caractéristique est qu'elle produit du changement, un bouleversement radical, et ce mouvement n'est pas maîtrisable.} \end{displayquote}

 
 Si le désir charnel ne suffit pas pour fonder un couple, est-ce qu'on pourrait au moins défendre l'idée qu'il est un élément déterminant lorsqu'il s'agit de choisir celui ou celle avec qui fonder une communauté de vie durable et accueillir des enfants ? Mais il est de notoriété publique que les unions n'ont jamais été aussi fragiles que depuis que le désir amoureux est censé commander le choix du partenaire ! Mais il faut aussi reconnaitre qu'ont été enlevés la plupart des contreforts qui naguère étayaient le lien conjugal, que ce soient les lois (l'indissolubilité du mariage « constantinien » avait été imposée par l'entremise d'un cadre juridique patiemment construit et vigoureusement défendu pendant plus d'un millénaire), l'intérêt matériel (le lien conjugal est moins solide lorsque la famille n'est plus une unité de production et lorsque chacun des deux partenaires se procure des ressources propres hors du foyer), la pression familiale ou sociale (les divorcés et leurs enfants ne sont plus stigmatisés, leurs nouveaux partenaires sont reçus par leurs parents), les croyances religieuses (la doctrine chrétienne du mariage n'est plus comprise). Lorsqu'il ne reste plus pour relier les conjoints que le souci des enfants conçus en commun, l'expérience montre que c'est souvent insuffisant même si Godelier fait l'hypothèse que les familles nucléaires sont nées dans les groupes de grands singes à partir du moment où les mâles se sont mis à coopérer activement et durablement avec les femelles pour élever les enfants de celles-ci. 
 
  \fsc{Nietzsche} affirmait dès 1888 que la dénaturation du mariage était en cours à travers l'importance donnée (dès son époque) au mariage d'amour et à l'égalité des époux, et qu'elle était le prélude à sa disparition.

\begin{displayquote}[\emph{Le Crépuscule des idoles}, 1888.]
%
\emph{On vit pour aujourd'hui, on vit très vite -- on vit sans aucune responsabilité : c'est précisément ce que l'on appelle « liberté ». Tout ce qui fait que les institutions sont des institutions est méprisé, haï, écarté : on se croit de nouveau en danger d'esclavage dès que le mot « autorité » se fait seulement entendre \emph{[...]} Témoin : \emph{le mariage moderne}. Apparemment toute raison s'en est retirée : pourtant cela n'est pas une objection contre le mariage, mais contre la modernité. La raison du mariage -- elle résidait dans la responsabilité juridique exclusive de l'homme : de cette façon le mariage avait un élément prépondérant, tandis qu'aujourd'hui il boite sur deux jambes. La raison du mariage -- elle résidait dans le principe de son indissolution : cela lui donnait un accent qui, en face du hasard des sentiments et des passions, des impulsions du moment, \emph{savait se faire écouter}. Elle résidait de même dans la responsabilité des familles quant au choix des époux. Avec cette indulgence croissante pour le mariage \emph{d'amour} on a éliminé les bases mêmes du mariage, tout ce qui en faisait une institution. Jamais, au grand jamais, on ne fonde une institution sur une idiosyncrasie ; je le répète, on ne fonde pas le mariage sur « l'amour », -- on le fonde sur l'instinct de l'espèce, sur l'instinct de propriété (la femme et les enfants étant des propriétés), sur \emph{l'instinct de la domination} qui sans cesse s'organise dans la famille en petite souveraineté, qui a \emph{besoin} des enfants et des héritiers pour maintenir, physiologiquement aussi, en mesure acquise de puissance, d'influence, de richesse, pour préparer de longues tâches, une solidarité d'instinct entre les siècles. Le mariage, en tant qu'institution, comprend déjà l'affirmation de la forme d'organisation la plus grande et la plus durable : si la société prise comme un tout ne peut \emph{porter caution} d'elle-même jusque dans les générations les plus éloignées, le mariage est complètement dépourvu de sens. -- Le mariage moderne a perdu sa signification -- par conséquent on le supprime.}  
%
\end{displayquote}

 \chapter{Un enfant pour quoi ? pour qui ?}




Puisque le patriarcat est mort et que les femmes ne retourneront plus dans des gynécées, sinon contraintes et forcées, et puisque le droit à l'égalité s'impose comme un principe de base indiscutable, il faudra inventer (ou découvrir, ou redécouvrir) pour les hommes une place dans le domaine familial  qui soit sinon aussi désirable que celle des femmes, au moins suffisamment désirable pour qu'ils s'y impliquent. Dans un environnement allergique à tout ce qui ressemble à du paternalisme, qu'est-ce qu'un homme est autorisé à désirer concernant des enfants ? Des points de vue et des désirs spécifiquement masculins sur les enfants sont-ils acceptables ? Il faudra sans doute commencer par admettre qu'il existe des valeurs masculines, ou une manière masculine de faire vivre les valeurs universelles.  Certes \emph{...dans la perspective proféministe, on ne peut vouloir à la fois que le genre disparaisse comme système hiérarchique et que les catégories du masculin et du féminin continuent d'exister. Mais pour d'autres auteurs, le terme de masculinité marque la volonté d'analyser s'il est possible d'être un homme sans coller aux stéréotypes de la virilité, d'une part ; sans devenir une femme, d'autre part\footnote{Molinier Pascale \emph{Virilité défensive, masculinité créatrice}, in  \emph{Travail, genre et société}, n° 3, mars 2000}}.

 Derrière ce problème se profile la question « à qui appartient l'enfant ? » : depuis très longtemps les enfants n'appartiennent plus à leurs pères seulement. Ils ne s'appartiennent pas à eux-mêmes, sauf à supprimer le statut de mineur. On ne peut pas non plus dire qu'ils n'appartiennent à personne. Du point de vue des enfants, n'appartenir à personne (ou appartenir à une institution d'assistance publique ou à un foyer d'accueil) c'est être abandonné. Est-ce qu'ils appartiennent désormais aux seules mères ? ... ou bien aux deux parents, comme le dit la loi ? ... ou bien à l'ensemble de ceux qui les élèvent en leur donnant leur argent et leur temps, dont les beaux-pères et belles-mères ? ... ou bien encore à l'État ?
  
Le désir de serrer un bébé de chair dans ses bras est d'autant plus irrépressible que ses motivations les plus profondes sont inconscientes. Il est sans doute aussi répandu et aussi fort aujourd'hui que par le passé et il ne concerne pas seulement les femmes. Pourtant, si l'on en croit \fsc{Coluche}, \frquote{\emph{y a des gens qui ont des enfants parce qu'ils n'ont pas les moyens de s'offrir un chien}}. Il posait à sa manière une question essentielle et relativement nouvelle : pourquoi fait-on des enfants ? Pour quoi veut-on des enfants ?

 
 Tous les citoyens des pays dotés d'un système d'assistance sociale et de retraite suffisamment efficace ont besoin qu'il naisse des enfants pour financer leurs moments d'invalidité et leurs vieux jours, mais aucun d'eux n'a \emph{besoin} que ce soient ses propres enfants : c'est précisément pour cela que ces systèmes ont été mis en place. Mais dans les pays les plus socialement développés, seule la collectivité a \emph{besoin} d'enfants. D'un point de vue strictement comptable et sauf dispositifs de compensation \emph{très}  généreux des frais qu'ils entrainent, l'intérêt bien compris de chacun des des citoyens \emph{des États providence} d'aujourd'hui est de ne pas avoir d'enfant. Leur niveau de vie et leur crédit auprès des banques sont plus élevés s'ils évitent d'investir dans une progéniture
\footnote{... en dehors des investissements collectifs qu'ils font, contraints et forcés, à travers leurs impôts pour financer l'enseignement, l'assistance sociale en direction de tous les mineurs, la santé infantile, et les allocations servies aux familles...}. Il ne faut sans doute pas chercher plus loin la faiblesse des taux de natalité de leurs membres, taux qui ne sont que la résultante des stratégies individuelles de leurs citoyens, stratégies d'autant plus rationnelles qu'on ne voit plus aujourd'hui au nom de quel argument on pourrait les leur reprocher. 
 
 Face à la désaffection du mariage et de la procréation qui menaçait la survie de l'Empire Romain, l'empereur Auguste a réagi en pénalisant les célibataires et ceux qui n'avaient pas d'enfants, et ses lois ont été appliquées sans faillir pendant au moins trois siècles. Si les taux de natalité baissaient dangereusement, verrions-nous à l'avenir de pareilles incitations légales
\footnote{... encouragements (ou pénalisation) par l'impôt ou le calcul des retraites ? Allocations couvrant le montant des dépenses d'éducation ? Crèches, internats et autres formes de prises en charge gratuites des enfants et adolescents par la collectivité ? Soutien matériel direct aux jeunes majeurs \emph{sans} conditions de ressources parentales ?~etc.} 
à procréer ? Il existe déjà des éléments de politiques de soutien à la natalité et aux familles, auxquels est attribué le taux de natalité français, comparativement élevé pour un pays développé.
 
 Mais les malthusiens et bien des écologistes pensent que les problèmes de santé de notre planète ont pour origine le fait que les humains sont trop nombreux. Il faudrait en effet que le nombre de ces derniers diminue drastiquement s'ils voulaient tous consommer comme les citoyens des pays développés actuels sans épuiser les ressources disponibles et sans mettre en danger les équilibres de la nature. Cela impliquerait non pas une croissance démographique zéro, mais une décroissance très énergique. L'intérêt commun de l'humanité serait-il sa décroissance numérique et l'évitement de la reproduction jusqu'au retour à un effectif écologiquement optimal ? 
 



 \chapter{Le célibat pour tous ?}
 

 
 
 Depuis que le droit a traité les enfants nés hors mariage comme les enfants nés dans le mariage, depuis qu'il n'y a plus de différences entre les enfants nés dans le mariage et les autres, les unions libres se sont multipliées de façon exponentielle. En effet il n'est plus nécessaire que les couples soient mariés pour que les hommes aient des héritiers, et l'intérêt de se marier diminue au fur et à mesure qu'au nom de l'égalité les lois étendent aux non-mariés les droits accordés aux gens mariés. Pourquoi même vouloir que les couples soient durables ? Ce n'est plus que que par habitude et du bout des lèvres que la loi prescrit aux conjoints d'être fidèles, puisqu'elle n'exerce plus aucune sanction à l'encontre des infidèles. Alors pourquoi pas une succession d'amours éphémères ?  ...ou pas d'amours du tout ? Après tout il n'est même plus nécessaire que les sexes se rencontrent pour faire un enfant, ni d'être deux pour l'élever. 



Dans un texte publié le 29 janvier 2013 sur \href{http://www.slate.fr}{Slate.fr}, en commentaire des polémiques suscitées par le projet de loi ouvrant le mariage aux personnes de même sexe, Jacques \fsc{ATTALI} esquisse l'avenir qui, compte tenu des évolutions récentes dans les pratiques familiales, reproductives et sexuelles, lui paraît le plus probable  : 

\begin{displayquote}
\emph{Comme toujours, quand s'annonce une réforme majeure, il faut comprendre dans quelle évolution de long terme elle s'inscrit.}
 
\emph{Et la légalisation, en France après d'autres pays, du mariage entre deux adultes homosexuels, s'inscrit comme une anecdote sans importance, dans une évolution commencée depuis très longtemps, et dont on débat trop peu : après avoir connu d'innombrables formes d'organisations sociales, dont la famille nucléaire n'est qu'un des avatars les plus récents, et tout aussi provisoire que ceux qui l'ont précédé, nous allons lentement vers une humanité unisexe, où les hommes et les femmes seront égaux sur tous les plans, y compris celui de la procréation, qui ne sera plus le privilège, ou le fardeau, des femmes.} 
 
\emph{\primo La demande d'égalité. D'abord entre les hommes et les femmes. Puis entre les hétérosexuels et les homosexuels. Chacun veut, et c'est naturel, avoir les mêmes droits: travailler, voter, se marier, avoir des enfants. Et rien ne résistera, à juste titre, à cette tendance multi-séculaire. Mais cette égalité ne conduit pas nécessairement à l'uniformité: les hommes et les femmes restent différents, quelles que soient leurs préférences sexuelles.}
 
\emph{\secundo La demande de liberté. Elle a conduit à l'émergence des droits de l'homme et de la démocratie. Elle pousse à refuser toute contrainte ; elle implique, au-delà du droit au mariage, les mêmes droits au divorce. Et au-delà, elle conduira les hommes et les femmes, quelles que soient leurs orientations sexuelles, à vouloir vivre leurs relations amoureuses et sexuelles libres de toute contrainte, de tout engagement. La sexualité se séparera de plus en plus de la procréation et sera de plus en plus un plaisir en soi, une source de découverte de soi, et de l'autre. Plus généralement, l'apologie de la liberté individuelle conduira inévitablement à celle de la précarité ; y compris celle des contrats. Et donc à l'apologie de la déloyauté, au nom même de la loyauté : rompre pour ne pas tromper l'autre. Telle est l'ironie des temps présents : pendant qu'on glorifie le devoir de fidélité, on généralise le droit à la déloyauté. Pendant qu'on se bat pour le mariage pour tous, c'est en fait le mariage de personne qui se généralise.}
 
\emph{\tertio La demande d'immortalité, qui pousse à accepter toutes mutations sociales ou scientifiques permettant de lutter contre la mort, ou au moins de la retarder.}
 
\emph{\quarto Les progrès techniques découlent en effet de ces valeurs, et s'orientent dans le sens qu'elles exigent : en matière de sexualité, cela a commencé par la pilule, puis la procréation médicalement assistée, puis la gestation pour autrui. Ces questions de bioéthique ne découlent évidemment pas des demandes d'égalité venant des couples homosexuels et concernent toutes les formes de reproduction, y compris -- et surtout -- « hétérosexuelles ». Le vrai danger viendra si l'on n'y prend garde, du clonage et de la matrice artificielle, qui permettra de concevoir et de faire naitre des enfants hors de toute matrice maternelle. Et il sera très difficile de l'empêcher, puisque cela sera toujours au service de l'égalité, de la liberté, ou de l'immortalité.}
 
\emph{\FrenchEnumerate{5} La convergence de ces trois tendances est claire: nous allons inexorablement vers une humanité unisexe, sinon qu'une moitié aura des ovocytes et l'autre des spermatozoïdes, qu'ils mettront en commun pour faire naitre des enfants, seul ou à plusieurs, sans relation physique, et sans même que nul ne les porte. Sans même que nul ne les conçoive si on se laisse aller au vertige du clonage.}
 
\emph{\FrenchEnumerate{6} Accessoirement, cela résoudrait un problème majeur qui freine l'évolution de l'humanité: l'accumulation de connaissances et des capacités cognitives est limitée par la taille du cerveau, elle-même limitée par le mode de naissance: si l'enfant naissait d'une matrice artificielle, la taille de son cerveau n'aurait plus de limite. Après le passage à la station verticale, qui a permis à l'humanité de surgir, ce serait une autre évolution radicale, à laquelle tout ce qui se passe aujourd'hui nous prépare. Telle est l'humanité que nous préparons, indépendamment de notre sexualité, par l'addition implicite de nos désirs individuels...}


\emph{Alors, au lieu de s'opposer à une évolution banale et naturelle du mariage laïc, qui ne les concerne pas, les Eglises devraient plutôt se préoccuper de réfléchir, avec les laïcs, à ces sujets bien plus importants : comment permettre à l'humanité de définir et de protéger le sanctuaire de son identité ?}

\emph{Comment poser les barrières qui lui permettront de ne pas se transformer en une collection d'artefacts producteurs d'artefacts ?}

\emph{Comment faire de l'amour et de l'altruisme le vrai moteur de l'histoire ?}


\end{displayquote}


Cela n'a rien d'évident en effet, et il n'est pas habituel d'associer l'amour et l'altruisme au refus de tout engagement et à la déloyauté. Jacques \fsc{ATTALI} ne décrit d'ailleurs pas ce qu'il désire, mais ce qu'il prévoit (dans l'hypothèse où les dynamiques en cours se prolongeraient sans changement) et il n'exclut pas l'idée que l'avenir qu'il décrit, s'il se réalisait, pourrait n'être pas totalement radieux. Sa question, \emph{"Comment faire de l'amour et de l'altruisme le vrai moteur de l'histoire ?"}, est une vraie question.

 Malgré tout le mal que semble en penser \fsc{Nietzsche}, l'exigence d'égalité absolue entre hommes et femmes est un acquêt de notre temps sur lequel il est très peu probable que l'on revienne. Avec quels arguments pourrait-on en effet défendre de manière convaincante (c'est-à-dire désirable pour tous) une inégalité fondée sur le sexe ou sur le genre, que le bénéficiaire de cette inégalité soit né fille ou garçon ? Cette exigence d'égalité est d'ailleurs encore loin d'avoir produit tous ses effets, directs et indirects. 
 
 Par contre l'exigence de liberté individuelle absolue, à tout prix et quelles qu'en soient les conséquences (ce qui est un autre nom pour l'individualisme), est grosse des problèmes pointés par Jacques \fsc{Attali}. Peut-on croire qu'il pourrait exister un domaine de l'existence dans lequel aucun engagement n'aurait d'importance, où aucune parole ne vaudrait plus rien, et que cela n'entraînerait pas de répercussions significatives sur les autres domaines de la vie ? sur les autres conversations ? sur les autres relations ? Surtout lorsqu'il s'agit d'un domaine aussi charnellement lié à la construction par chaque sujet de son identité ? Peut-on croire que cela n'aurait aucun effet en termes de « lien social » ? Si entre les personnes aucune promesse ne vaut, alors la première place a toutes les probabilité de revenir aux actes, et même aux passages à l'acte. On est rejeté sur le terrain de la sauvagerie et des rapports de force. 
 
 Quant à la valorisation de l'immortalité individuelle, si c'est à n'importe quel prix c'est aussi une autre facette de l'individualisme. Contrairement au refus de la mort \emph{d'autrui}, le refus \emph{à n'importe quel prix} de la mort \emph{de soi} met l'individu au-dessus de tout, et en particulier au-dessus de tous les autres et des liens avec eux. Ce choix est foncièrement asocial. \enquote{\emph{Après moi le déluge}} : difficile de fonder une société viable là-dessus !

 Dans ce contexte \enquote{\emph{Comment faire de l'amour et de l'altruisme le vrai moteur de l'histoire ?}} Reste-t-il même une place pour l'amour et l'altruisme ? Est-ce que le sacrifice de soi pour un autre garde un sens? C'est pourtant un sacrifice qu'ont fait d'innombrables individus qui au fil de l'histoire ont choisi librement de mettre en péril leur bonheur, leur santé ou leur sécurité pour leur conjoint ou leurs enfants, pour lutter contre des souffrances et des injustices individuelles ou collectives, qui ont résisté aux oppresseurs, aux pervers et aux injustes, et qui se sont sacrifiés en toute connaissance de cause pour que les autres vivent. 
 
 Comment s'orienter face à ces contradictions ? Les positions religieuses ne convainquent que ceux qui y croient\footnote{\emph{Il n'y a pas de plus grand amour que de donner sa vie pour ceux qu'on aime.} (Jn 15, 13)}. Les positions morales sont variées et contradictoires : d'un côté (en perte de vitesse) on valorise l'engagement et la fidélité à la parole donnée, de l'autre (qui a le vent en poupe) on plaide pour l'authenticité et la liberté. Le critère le moins contesté serait peut-être l'utilité : pour notre société dans son ensemble qu'est-ce qui est préférable ? la fidélité aux engagements ou la liberté d'être à chaque instant authentique ?  

 A défaut d'enquêtes de satisfaction il serait au moins possible de chiffrer l'ensemble des coûts et des bénéfices directs et indirects de la fidélité et de l'authenticité  : coûts financier, incidence sur les santés physique et mentale, problèmes d'ordre public, etc... On pense entre autres à tous les dépenses engendrées par le « dé-mariage » sur le marché immobilier, sur celui de l'équipement domestique ou sur celui de l'assistance juridique, etc. La traduction concrète de ces coûts c'est la pauvreté qui frappe beaucoup de divorcés ou de parents célibataires : femmes seules avec enfants, mais aussi hommes seuls dont les ressources sont insuffisantes pour retrouver un logement, etc. Il y a matériellement avantage à vivre en couple, et c'est d'autant plus vrai qu'on est plus démuni, et les séparations font perdre cet avantage. Il faut y ajouter les coûts psychologiques, qui sont en partie inséparables des coûts matériels, ainsi que le note Gérard Neyrand \footnote{Gérard Neyrand, journée d'étude « la famille dans tous ses états », IRTS et Université de Metz, 25-11-2010} :  
\begin{displayquote}{\emph{En effet, les nouvelles valeurs familiales sont portées par les couches moyennes cultivées et sont devenues système de référence global. Leur confrontation aux habitus des couches populaires en la matière ne s'effectue pas sans conflits (Commaille, Martin, \emph{Les enjeux politiques de la famille}, Paris Bayard, 1998). L'une des issues des contradictions entre ces systèmes différents de références, qui traversent différemment les individus selon leur sexe et leur position sociale, réside dans la fréquence des séparations conjugales conflictuelles, la mono-parentalisation maternelle qui s'en suit et la précarisation des foyers mono-parentaux ainsi définis. Leur caractéristique est bien d'être soumis à un double système de contraintes croisées, socio-économiques et psycho-relationnelles.
La montée du chômage et la précarisation des emplois les moins qualifiés (Boltanski, Chiapello, \emph{Le nouvel esprit du capitalisme}, Paris Gallimard, 1999), contribuent à une fragilisation globale des situations familiales des plus démunis, qui risque d'autant plus de déstabiliser les familles que ces familles populaires se pensent de façon unitaire, quasi-symbiotique.
Elles sont basées sur un couple conçu comme une entité indissoluble, un « couple unité organique » selon l'expression d'Irène Théry (\enquote{\emph{Le couple occidental et son évolution sociale : du couple « chainon » au couple « duo »}}, Dialogue, \no 150, 4\ieme trimestre, 2000), et sont loin d'adhérer sans réserve au nouveau modèle moderne du « couple duo ». La séparation, dès lors, constituera une catastrophe identitaire dont beaucoup auront du mal à se relever, en particulier les pères.
On conçoit alors l'importance des difficultés que des séparations dans un tel contexte peuvent générer :
- difficultés relationnelles entre les ex-conjoints et dans le rapport des pères à leurs enfants,
- et difficultés socio-économiques des mères confrontées aux nécessités d'une survie familiale qu'elles doivent bien souvent affronter seules.
Mono-parentalisation et précarisation s'avèrent alors intimement liées.}}

\end{displayquote}



Par ailleurs l'éclatement des couples ne va pas dans le sens du renforcement des capacités éducatives des parents. Il multiplie le nombre des situations où la fonction éducative de l'un ou de l'autre des parents est plus ou moins disqualifiée ou empêchée, tandis que son remplacement au quotidien par un partenaire sexuel et affectif de l'autre n'est pas toujours bien accepté par les enfants concernés et ne présente pas forcément l'efficacité éducative nécessaire. Le nombre s'élève donc des parents qui face aux inévitables problèmes éducatifs sont plus ou moins seuls et/ou en difficulté. Lorsque leurs enfants leur font problème, notamment à l'adolescence, des aides extérieures sont souhaitables, mais même en tenant pour négligeable la disqualification et la mise en dépendance des parents par les « spécialistes » et les « experts » de l'éducation, les soutiens que propose la collectivité ne sont pas gratuits (ex. internats scolaires, assistance éducative, placement en famille d'accueil,~etc.). On passe de « l'auto production » familiale des activités éducatives à leur « externalisation » et à leur « professionnalisation ». Comme c'est un domaine où il n'y a pas à espérer de gain de productivité cela accroît les coûts éducatifs de manière très sensible. Jusqu'où peut-on aller dans cette voie avant que les électeurs n'estiment que c'est trop cher payé ? Enfin rien ne garantit que l'efficacité des diverses aides éducatives ainsi apportées aux parents soit en règle générale supérieure à ce qu'en d'autres circonstances ils auraient pu assumer eux-mêmes : ce serait déjà bien si on pouvait être assuré qu'elle ne soit pas moindre. 

En conclusion il serait peut-être difficile de prouver chiffres en main que le mariage traditionnel est \emph{supérieur} à toutes les autres manières d'organiser la vie des individus\footnote{Pour savoir ce qu'il en est vraiment, par delà les représentations que chacun peut s'en faire, il faudrait commencer par établir tous les chiffres, par faire toutes les études nécessaires. Pour l'instant le sujet semble chargé de beaucoup trop de passions pour être abordé, beaucoup trop politiquement connoté pour que des chercheurs s'y risquent ?}, mais il est au moins aussi difficile de croire que la stabilité des unions fécondes et la fidélité des partenaires qui sont aussi des parents n'auraient aucun intérêt du point de vue de la société et seraient définitivement dépassés. 

Par contre du point de vue des entreprises ou des administrations la disponibilité d'un(e) employé(e) est plus grande lorsqu'il n'est plus nécessaire de tenir compte de son désir de vivre avec un(e) partenaire lui-même (elle-même) bien inséré(e) professionnellement et qui lui (elle) aussi s'attend à être significativement investi(e), de même qu'un(e) célibataire sans enfants et sans intention d'en avoir est précieux pour son employeur et a des atouts significatifs pour réussir une belle "carrière" professionnelle.




\chapter{Vers quelles constellations familiales allons-nous ? }

Les lois autorisant le divorce par consentement mutuel, la contraception et l'avortement ont dénaturalisé le modèle de famille traditionnel si bien que même si rien ne pourra jamais empêcher personne de croire à la valeur éthique, éducative ou civique de la {\emph{sainte famille}}, fondée sur un couple hétérosexuel, monogame et indissoluble, élevant lui-même les enfants nés de ses œuvres, ce n'est plus qu'un choix parmi d'autres également légitimes, comme c'était le cas avant les décisions de Constantin. Dans la mesure où ce modèle n'est plus étayé par la loi et la puissance publique, comme il l'était par l'Ancien Régime ou le Code Napoléon, la « sainte famille », la famille « traditionnelle » ne correspond plus à ce que j'ai appelé la « famille constantinienne ». Il s'agit seulement de l'une des façons ("post constantiniennes") d'organiser la reproduction humaine, une parmi d'autres, dont la paternité, la maternité ou l'adoption célibataires, le concubinage, le PACS ou le mariage hétéro ou homosexuel,~etc.

Dans \enquote{\emph{Quelle alternative au patriarcat ? Valoriser un modèle social non conjugal}} (2004), Agnès \fsc{ECHENE} accuse le couple hétérosexué d'être le lieu privilégié d'expression et de transmission de la violence machiste, et cela trop souvent avec la complicité (masochiste) féminine. Elle en tire la conclusion qu'il faut éliminer la paternité en tant que telle :

\begin{displayquote}
{[...] \emph{ce n'est qu'en valorisant le modèle social non conjugal qu'une société peut se défaire du patriarcat. Il importe donc de favoriser une sexualité libre et variée, tout en étant discrète et protégée, surtout chez nos propres enfants ; peu importe dès lors qu'elle soit ardente ou paisible, monotone ou changeante, homophile ou hétérophile, dès l'instant qu'elle reste une affaire personnelle dont nul ne se mêle. Une telle évolution nécessite également une reconsidération du modèle familial qui doit se refonder sur des liens d'appartenance utérine et non pas consanguine ; cela remet en cause dès lors la paternité génitale qui doit laisser place à une paternité germaine : il faut en effet que ce soit les frères, oncles et cousins \emph{[de la mère]} qui assument les enfants des femmes ; de nombreux signes avant-coureurs montrent qu'ils sont prêts à le faire et qu'il ne manque qu'un déclic. Mais il faut aussi que les femmes renoncent à obliger les géniteurs à être pères ; il faut qu'elles abandonnent toute velléité de recherche de paternité, de pension, partage, alternance,~etc. et se tournent résolument vers leurs frères, oncles et cousins pour « donner » des pères à leurs enfants, qui ne s'en porteront pas plus mal.}}
\end{displayquote}

On est là à très peu près dans le monde de l'ethnie Na et des autres groupes chinois ou thibétains qui ne connaissent pas de pères, et où les hommes de chaque famille sont les amants librement choisis, pour une nuit ou pour plusieurs, des femmes des familles voisines, qui reposent toutes sur un principe matriarcal, puisque les enfants appartiennent exclusivement à la famille de leur mère. Ce sont donc les oncles maternels qui "paternent" les enfants de leurs sœurs, dans le cadre d'une répartition sexuée des tâches et des pouvoirs qui par ailleurs est identique à celle qu'on rencontre dans les familles patriarcales chinoises environnantes (il ne suffit pas que les femmes aient le choix du géniteur de leurs enfants pour que l'autorité dans le groupe familial leur appartienne, même si c'est évidemment une brèche dans le pouvoir des hommes). 

D'autres imaginent des "familles" d'une tout autre espèce, des associations basées sur des contrats de solidarité privés ne se référant plus au couple, mais plutôt aux communautés (« post-soixantehuitardes » ?) : 

\begin{displayquote}
{\emph{"A l'époque de la discussion sur le pacs, certains avaient proposé de créer des liens de solidarité entre plusieurs individus, et pas uniquement au sein du couple, qu'il soit hétérosexuel ou homosexuel. Le pacs aurait pu permettre, par exemple, d'associer des personnes au moyen de liens juridiques alternatifs qui ne soient pas forcément fondés sur la famille. Voilà une proposition sociale intéressante, qui aurait permis d'inventer des formes de vie à plusieurs. Mais nous sommes loin d'une telle réflexion.."}\footnote{(Marcela Iacub \emph{"Critique du mariage gay et de la famille nucléaire. Vers la Famille Choisie Associative ?"} 
Publié par Uncas Sachem le 11 novembre 2012 dans \emph{Actualité, Les matriciens qui s'ignorent.)}}} 
 \end{displayquote}
 
 Va-t-on de manière plus banale vers des foyers constitués d'une femme et des enfants qu'elle a mis au monde, autour desquels graviterait la nébuleuse de ses amants et ex-amants ? Dans cette hypothèse, les hommes de demain auraient des enfants de plusieurs femmes, enfants vivant ordinairement chez leurs mères, si bien que leur autorité sur chacun d'eux serait pratiquement nulle ? Serait-ce en quelque sorte l'inverse de la situation du \latin{pater familias} romain, qui pouvait demander à plusieurs femmes des enfants sur lesquels lui seul avait autorité ? 
 
 On reconnait là le modèle matrifocal « antillais » ou « caraïbe » dont l'origine se situe dans l'histoire du peuplement des Antilles. On a vu que les esclaves n'ont par définition aucun des attributs juridiques d'un père ou d'une mère sur les enfants dont ils sont les géniteurs et génitrices : seuls les propriétaires des génitrices possèdent des droits sur les enfants de celles-ci. C'étaient ces propriétaires qui faisaient d'elles des mères lorsqu'ils leur confiaient la garde des enfants qu'elles avaient portés, quel qu'en était le géniteur. Il y avait une espèce d'alliance de fait (sous contrainte et perverse) entre les génitrices et leurs maîtres pour élever les enfants qu'elles avaient mis au monde, tandis que leurs partenaires sexuels n'avaient pas droit à la parole et étaient réduits, quel qu'ait pu être leur désir,  à n'être que des donneurs de sperme. 
 

 
 Faut-il comprendre (craindre ?) qu'à l'avenir ce pourrait être l'État qui, à l'aide de ses services judiciaires et médoco-socio-éducatifs et de toutes les prestations qu'il sait fournir, assumerait le rôle de tiers traditionnellement dévolu aux pères ? Tiers qui soutient matériellement et psychologiquement l'autre (parent) dans sa tâche éducative, et qui introduit au sein de la dyade parent-enfant les exigences du monde extérieur .
 
 \chapter{Filiation adoptive contre filiation biologique ? }
 
 Le "mariage constantinien" tel que je l'ai défini télescopait sur le couple des seuls géniteurs (légitimes) toutes les dimensions de la conjugalité et de la parentalité (juridique, biologique, affective et éducative) et frappait tout le reste d'illégitimité et notamment la filiation élective, volontaire, adoptive. C'est au nom de cette dernière que l'hégémonie de notre tradition juridique est contestée, et cette remise en question est comme on peut s'y attendre consubstantiellement liée à la promotion de nouvelles formes de conjugalités. Daniel \fsc{BORRILLO}
\footnote{\href{http://www.universalis.fr/encyclopedie/famille-les-enjeux-de-la-parentalite/}{\frquote{Famille - Les enjeux de la parentalité}}, Daniel \fsc{Borrillo}, \latin{Encyclopædia Universalis}.} est un bon interprète de cette contestation globale. Selon lui les possibilités vertigineuses de dissociation entre sexualité et reproduction qui se sont ouvertes grâce aux progrès de la biologie en à peine une génération ont provoqué une\emph{ panique morale}, qui a conduit les théoriciens et les praticiens du Droit à survaloriser les liens biologiques géniteurs-enfants : 

\begin{displayquote}
{\emph{La biologie commença à devenir ainsi le soubassement réel ou symbolique%
% [2] 
\footnote{Le \emph{biologique} comme fondement \emph{symbolique} du système de parenté ?} 
du système de parenté, à rebours d'une science juridique qui avait plutôt instauré la volonté au cœur de ce système \emph{[...]} À partir des années 1990, l'expertise biologique\footnote{elle-même permise désormais grâce aux mêmes progrès de la biologie…} s'est imposée dans les procès en contestation de paternité, la recherche des origines est revendiquée socialement comme droit fondamental de la personne, la différence de sexe est devenue une valeur \emph{[...]} La nouvelle place prépondérante de la vérité biologique dans l'établissement du lien filial fut confirmée en France par la Cour de cassation%
%[3]
\footnote{... dans un arrêt du 28 mars 2000 établissant que {\emph{l'expertise biologique est de droit en matière de filiation, sauf s'il existe un motif légitime de ne pas l'ordonner}}. Civ. 1\iere, 28 mars 2000, Bull. \no 103 ; \hbox{Defrénois}, \hbox{30.06.2000}, \no~12, p. 769, note J.~\fsc{Massip} ; \hbox{Dalloz}, \hbox{12.10.2000}, \no 35, p. 731, note T.~\fsc{Garé} ; JCP \hbox{25.10.2000}, \nos 43-44, conclusions C.~\fsc{Petit} et note M.C.~\fsc{Monsallier-Saint-Mieu}.}% 
... Par là, la distinction traditionnelle entre reproduction (fait biologique) et filiation (fait culturel), fondement du droit civil moderne, se trouvait questionnée... non pas à partir d'arguments classiques provenant du droit canonique, mais par une rhétorique qui, d'une part, fera de la différence des sexes une condition \latin{sine qua non} de la filiation, et, d'autre part, placera l'expertise sanguine et la preuve d'ADN au cœur du dispositif juridique de la parenté.}}

[...]\emph{La filiation peut certes tenir compte du fait naturel, mais, en tant que dispositif d'agencement parental, elle répond à des règles propres, affranchies de la nature \emph{[...]} Elle n'existe que lorsqu'elle est établie dans les conditions et selon les modes prévus par la loi. Autrement dit, la filiation est déterminée par la norme juridique et non par la nature. Ce lien juridique se tisse à partir de quatre fils principaux : la biologie (filiation par le sang), la volonté (adoption), la présomption (paternité supposée du mari de la mère) et le vécu (appelé en droit « possession d'état »).}

\emph{Ce qui compte ce ne sont plus tant les racines naturelles ou surnaturelles d'institutions intangibles que l'efficacité et la plasticité d'instruments juridiques procurant tel ou tel résultat (par exemple la paix des familles ou la solidarité des générations).}  [...] \emph{fondée sur la volonté, l'adoption est une institution plus apte que la vérité biologique à assurer la stabilité des liens familiaux.}
 
{\emph{[...]La contestation actuelle de l'ordre familial « naturel » n'est en définitive que la radicalisation de l'idéologie individualiste moderne, selon laquelle la volonté et non la différence des sexes constitue la base de l'institution matrimoniale et parentale. Une filiation dissociée de la reproduction permettra de justifier un système juridique fondé non pas sur la vérité biologique, mais sur le projet parental responsable. De ce point de vue, peu importe l'agencement familial (traditionnel, monoparental, homoparental, recomposé...), si les prémisses du contrat (égalité dans l'alliance et dans la filiation) sont respectées jusque dans leurs moindres effets. L'État devrait donc traiter sur un plan d'égalité l'ensemble des familles et les autres formes d'intimité.}

\emph{Contrairement à la filiation charnelle, la filiation choisie trouve son principe dans la liberté non seulement d'accueillir les enfants des autres, mais également d'abandonner ses propres enfants biologiques, ce qui est pour l'heure uniquement possible pour les femmes (accouchement sous X), mais devrait pouvoir s'élargir aussi aux hommes à travers une déclaration formelle de renoncement à la paternité. La généralisation de la filiation adoptive (y compris pour ses propres enfants biologiques) permettrait aussi de mettre la volonté au cœur du dispositif parental. Celui-ci reposerait exclusivement sur la volonté du ou des géniteurs qui donnent l'enfant et celle du ou des adoptants qui l'accueillent. De surcroît, l'adoption est une institution conçue à partir du droit de l'enfant à avoir une famille, contrairement à la filiation biologique qui apparaît plutôt comme un dispositif du droit à l'enfant}. (Daniel \fsc{BORRILLO})}\end{displayquote}


 
 Que tous les systèmes juridiques soient des constructions humaines et non des faits de nature, qu'ils reposent sur des "idéologies", au sens de prises de positions et/ou de croyances plus ou moins partagées, et qu'ils fassent tous des choix entre des possibles, acceptant les uns et refusant les autres, cela est évident. Mais sur quels arguments se fonde l'affirmation que
 \emph{"L'adoption est une institution plus apte que la vérité biologique à assurer la stabilité des liens familiaux"} ?  On peut tout aussi bien affirmer le contraire. Il n'est pas nécessaire que la filiation adoptive soit \emph{meilleure} que la filiation \emph{ordinaire} pour lui faire une place. A devoir choisir entre le \emph{sang} et la \emph{volonté} pour fonder le droit de la filiation il y a quelque chose qui parait artificiel et forcé. Tout faire reposer sur la biologie est méconnaître qu'elle n'a jamais suffit dans aucune société (et surtout pas chez les juifs ou les chrétiens) pour légitimer une naissance, et oublier combien le lien entre un adulte et son enfant est un lien co-créé dans le cadre de leur relation, à l'instar d'une adoption réciproque (psychologique ou affective) et qui déborde de tous côtés la proximité biologique. Mais privilégier la volonté sur tout le reste est une fiction juridique, héritée des romains, qui comme toutes les fictions juridiques fait d'une certaine manière violence aux réalités telles qu'elles sont vécues au jour le jour.
 
 \emph{"l'adoption est une institution conçue à partir du droit de l'enfant à avoir une famille, contrairement à la filiation biologique qui apparaît plutôt comme un dispositif du droit à l'enfant."} Cette affirmation paraît arbitraire, même s'il est vrai qu'en France l'adoption des jeunes enfants a été ressucitée après la Grande Guerre pour donner des parents à des enfants abandonnés. En effet il ne s'agit pas de donner des parents à des enfants qui en manqueraient : il y a actuellement plusieurs candidats à l'adoption pour chaque enfant adoptable né en France. 

La valorisation théorique de l'adoption vient pour partie de la prise en compte du désir d'enfant de ceux qui ne peuvent ou ne veulent pas recourir aux relations hétérosexuelles, désir d'enfant qui en soi n'est ni plus ni moins légitime que celui des autres.  Mais dès que les enfants sont sortis de la petite enfance, leur adoption n'est pas simple et elle peut être terriblement éprouvante pour le narcissisme des adoptants. Les adoptés courent plus que les autres enfants le risque d'être rejetés à cause des difficultés de tous ordres qu'ils ont rencontrées du fait de leur histoire et auxquelles ils se sont adaptés comme ils ont pu. Leurs attentes ne s'engrènent pas toujours harmonieusement avec celles de leurs parents adoptifs : le pourcentage de ces enfants qui sont abandonnés une deuxième fois après adoption n'est malheureusement pas négligeable. Tous les adultes ne sont pas prêts à prendre de pareils risques. 

Jusqu'ici il n'a jamais été permis par la loi de concevoir des enfants pour les donner en adoption, mais si l'on voulait répondre à toutes les demandes d'adoption il faudrait s'y résoudre. Tant que dureront les énormes inégalités de revenu observables sur cette planète, les plus fortunés pourront toujours louer le ventre des plus belles et des plus saines des filles des pauvres, de la même façon que les riches romains achetaient les plus jolies des jeunes esclaves afin qu'elles leur fassent des enfants bien à eux qu'ils n'auraient à partager ni avec un partenaire égal à eux en dignité, ni avec une belle famille aussi puissante que la leur. Le recours à des « mères porteuses » est dans la logique des évolutions libérales actuelles. Il est d'ores et déjà légalement possible dans plusieurs pays développés. Est-il appelé à se généraliser ? Comment refuser ce recours aux hommes homosexuels si l'on accorde l'assistance médicale à la procréation (PMA) aux femmes homosexuelles, et comment le refuser à tous les autres, hommes et femmes, si on l'accorde aux hommes homosexuels ? Et comment le refuser à qui que ce soit si des femmes (souvent pauvres et vivant dans des pays sous-développés) sont volontaires pour prêter leur ventre et abandonner leur enfant nouveau-né contre une indemnité suffisante. 

 C'est le seul moyen de mettre les hommes à égalité avec les femmes dans l'accès à l'enfant, ou plutôt de corriger l'inégalité que leur corps leur impose dans ce domaine, mis à part bien sûr le mariage traditionnel, monogame et indissoluble, dont c'était l'une des finalités. Lorsque leur mariage était rompu les pères romains gardaient leurs enfants : ils n'avaient donc pas particulièrement intérêt à ce que les unions soient indissolubles. Par contre leurs épouses avaient de bonnes raisons de craindre d'être répudiées et séparées de leurs enfants. Elles ont peut-être trouvé bon à partir du IVème siècle d'être mieux protégées de ce risque et d'avoir leur vie durant l'exclusivité de la sexualité légitime de leur mari. Aujourd'hui où leur autonomie financière et les lois leur permettent de prendre l'initiative de quitter leurs maris sans quitter leurs enfants la situation se retourne et ce sont les hommes qui peuvent commencer de craindre d'être séduits puis abandonnés. 
 
 L'idée d'une {\emph{déclaration formelle de renoncement à la paternité}}, en miroir du droit reconnu aux femmes à \emph{l'accouchement sous X}, peut paraître provocante. Pourtant une telle disposition ne ferait que rejoindre le point de vue des révolutionnaires de 1789 : pas de contrainte en parentalité, donc pas de contrainte en paternité. On pourrait considérer qu'il s'agit d'une simple mesure de justice pour rendre aux hommes un peu de la liberté face à la parentalité qu'ils ont perdue depuis longtemps.

 Plutôt que de se retrouver un jour contraints de continuer de payer pour leurs enfants sans plus les avoir auprès d'eux, tandis que souvent un autre qu'eux les éduque, les hommes pourraient choisir, quelle que soient par ailleurs leurs préférences sexuelles, de commencer par payer pour les posséder sans partage afin que personne ne puisse jamais les leur contester. Sur quels arguments fonder le refus d'une pareille évolution ? Elle ne serait au fond que le miroir de celle qui voit des femmes choisir en toute connaissance de cause de faire un enfant toutes seules. Si les humains ne diffèrent en rien de significatif en dehors de leurs caractéristiques biologiques, si les femmes n'ont pas besoin d'un homme pour élever un enfant, alors les hommes n'ont pas non plus besoin d'une femme pour élever leurs propres enfants.
 
 La pratique des mères porteuses fait l'impasse sur le droit des enfants à être élevés par leurs géniteurs. On pourrait cependant considérer que ceux qui seraient ainsi adoptés auraient tort d'attacher de l'importance à ces derniers, et que leur adoption serait sûrement ce qui leur serait arrivé de mieux (sans elle ils n'auraient pas même été conçus), mais beaucoup d'entre ceux qui ont effectivement été adoptés jusqu'ici n'en veulent pas moins qu'on leur explique pourquoi cela leur est arrivé à eux. Comment se situeront, une fois qu'ils seront devenus adultes, les enfants obtenus par recours à une mère porteuse ou à l'un des divers autres "bricolages" auxquels l'actualité médiatique fait référence ? L'avenir nous le dira.   

 Le recours aux mères porteuses ne pourrait être interdit, en dépit de la pression des demandes individuelles et du modèle fourni par les pays où cette pratique est autorisée, que s'il était d'abord admis qu'il implique la réduction d'un humain au statut d'instrument de la volonté d'un tiers jusque dans son corps, ce qui est la définition de l'esclave, et s'il était reconnu que c'est inacceptable, même si cette personne a donné son accord. Une deuxième raison est que cette pratique fait de l'enfant à naître le produit d'un contrat commercial, sauf à ce que se généralise le don d'enfants par des femmes qu'aucune nécessité matérielle y pousserait, mais rien ne montre qu'on aille vers là.
  
  Mais refuser le recours à des mères porteuses impliquerait aussi d'accepter l'idée qu'il n'existe pas de droit à l'enfant, c'est-à-dire que tout un chacun peut être irrémédiablement privé d'enfant en dépit de ses désirs les plus authentiques et les plus légitimes sans avoir pour autant droit à la réparation de cette injustice. 
Le mouvement des pratiques depuis un demi-siècle ne va pas dans ce sens.

Quant à espérer sortir de ces contradictions en recourant à un utérus artifiel, c'est encore de l'utopie.
 
 
 
 
  
 
 

 \chapter{Une société peut-elle se passer d'exclure ?}
 
 Dans \emph{L'avenir d'une illusion} (1927), \fsc{FREUD} se demande jusqu'où une société humaine peut se permettre d'être souple et tolérante étant donnée la violence des pulsions, désirs et angoisses qu'elle a pour tâche d'humaniser. Sceptique et pessimiste il soutient qu'une grande dose de répression est inévitable, et que c'est même une des conditions de l'élaboration de sociétés vivables et d'œuvres culturelles de valeur.

 Dans une période donnée peuvent être inapparents, inconscients, ou plutôt innommables et innommés (déniés) les traits de dureté qu'elle n'a pas suffisamment élaborés, les blocs de sauvagerie qu'elle n'a pas su penser. Ce sont des \emph{points aveugles} dans la représentation que cette période se donne d'elle-même. Ils sont involontaires et bien évidemment personne ne les a voulus en connaissance de cause... par contre ils sauteront aux yeux des générations suivantes qui ne comprendront pas comment il a été possible de ne pas les voir. 
 
 Le plus bel exemple est que depuis des milliers d'années on a admis comme un fait établi et ne souffrant pas la discussion que les femmes étaient inférieures aux hommes, faites pour leur obéir et les servir, et qu'il était donc indispensable qu'une part plus ou moins grande de leurs droits soient détenus et exercés par des membres masculins de leur entourage. 

 L'histoire des enfants sans parents est elle aussi marquée par plusieurs de ces points aveugles, à commencer par la dureté du sort fait partout, depuis toujours et jusqu'aujourd'hui en toute bonne conscience aux enfants de naissance illégitime, quelles qu'aient été les manières successives de définir en quoi leur naissance était illégitime, c'est-à-dire inopportune. Quoi de plus barbare que la croyance en une impureté ou une infamie de naissance ? Quoi de plus arbitraire et déraisonnable que l'idée qu'être né d'un ou d'une esclave interdisait irrévocablement de prétendre à des postes à responsabilité ? Quoi de plus étrange pour nous que la valeur religieuse du sang, ou la « pureté » d'une généalogie ? Quoi de plus absurde que de disqualifier moralement les « enfants du péché » tout en absolvant ceux qui auraient commis le « péché » dont ils étaient nés ? 

 Tout se passe comme si les conceptions archaïques du pur et de l'impur avaient continué d'être tenues pour vraies jusqu'à nos jours alors que le caractère moralement inadéquat de ces représentations avait été dénoncé il y a deux mille ans par les stoïciens aussi bien que par les évangiles, dont les thèses ont pourtant été méditées sans interruption depuis lors. Jusqu'au début du \siecle{20} chacune de ces propositions a été tenue pratiquement pour vraie par tous ou presque tous, ou par chacun presque tout le temps, alors qu'elles étaient théoriquement insoutenables du point de vue même de ceux qui s'y conformaient. Jusqu'à Vincent de Paul on n'appelait pas négligence le sort mortel qui était fait aux nouveaux-nés abandonnés, parce que les exclure du monde des familles légitimes paraissait être la façon correcte de les traiter et qu'on n'en imaginait pas d'autre. Lui a su le premier ou l'un des premiers, voir en eux autre chose que des êtres impurs qu'il était moralement indifférent de laisser mourir du moment qu'ils étaient baptisés. C'est sur les représentations de ses contemporains qu'il a travaillé et non sur l'art d'accommoder les bébés séparés de leur mère (cet art ne posait pas plus de problèmes aux femmes de son époque qu'à celles d'aujourd'hui). 

 Il y a moins d'un siècle les mineurs vagabonds étaient encore considérés et traités comme des délinquants : la criminalisation de leurs errances avait commencé à la fin du Moyen Âge : auparavant on les assimilait aux pèlerins et on se recommandait à leurs prières. 

 De même il n'y a guère plus d'un demi-siècle qu'on regardait encore avec méfiance les rencontres entre les enfants placés en institution et leurs parents. 

 Et il n'y a guère plus de trente ans qu'on a vraiment pris la mesure de la gravité des dégâts psychologiques produits par les \emph{"abus" sexuels} perpétrés par les adultes sur les enfants, surtout quand ils ont autorité sur eux, et qu'on a accepté de voir que les \emph{"abuseurs"} font le plus souvent partie de l'entourage immédiat de leurs victimes. 
 
 Ces \emph{points aveugles} étaient visibles par tous, mais ils n'étaient pas vus, mais les violences et les cruautés faisaient d'autant moins problème qu'elles paraissaient aussi inexorables que le jour et la nuit, aussi naturelles et nécessaires que le soleil et la pluie (bien évidemment ce sont les personnes qui étaient aveugles). 
 
 Il n'est donc pas impossible, il est même probable qu'aujourd'hui aussi s'étalent sous nos yeux des malheurs et des souffrances que nous ne voyons pas, des maltraitances que nous produisons en toute bonne (in)conscience. Si c'est réellement le cas, alors dans un siècle, ou dans dix, on nous reprochera de les avoir méconnus, sans comprendre que nous ne pouvions pas les voir, aveuglés que nous sommes par nos théories, nos croyances, nos désirs ou nos intérêts inconscients, de la même façon que nous sommes scandalisés par la brutalité, l'insensibilité et les aberrations logiques de nos prédécesseurs. 
 
 En l'absence d'observateurs venus d'un autre monde seules des recherches scientifiques rigoureuses sont en mesure d'apporter des éléments de réponse à de telles cécités, mais elles le font toujours trop lentement.


 
 
 Est-ce que les lois et les pratiques qui encadreront à l'avenir la conception des enfants et l'art de les accommoder produiront moins de souffrances et de troubles que celles du passé chez les enfants et chez leurs parents ? On ne voit pas bien en quoi les enfances organisées par les manipulations de la biologie et des relations interpersonnelles évoquées plus haut par Jacques Attali seraient un progrès du point de vue des enfants. Il est vrai que ce n'est pas leur objectif. 

 Le recours à la prévention des naissances, à la pilule anticonceptionnelle, à la pilule « du lendemain » et à l'avortement permet en principe qu'il ne naisse plus d'enfants non désirés. Mais suffit-il que ceux qui naissent aient été désirés par leurs géniteurs ou par leurs parents adoptifs pour que disparaissent les problèmes qu'ils posent ou ceux qu'ils rencontrent ? Les enfants ne sont pas sans influence, pour le meilleur et pour le pire, sur la relation que leurs parents construisent avec eux. Volontairement ou sans pouvoir s'en empecher ils peuvent déplaire à leurs parents sur des points auxquels ces derniers sont viscéralement attachés. Nul ne peut donc garantir qu'à l'avenir il y aura moins d'enfants mal assumés que par le passé. 

 Si la pauvreté matérielle n'est plus depuis longtemps un motif suffisant à lui seul pour séparer les enfants de leurs parents, est-on assuré pour autant qu'il n'existe et n'existera plus jamais d'enfants privés de l'un ou de l'autre de leurs parents alors que ceux-ci sont disponibles, volontaires pour les élever et suffisamment compétents ? L'absence de l'un des deux parents pour d'autres raisons que la maladie ou la mort devient au contraire quelque chose de plus en plus fréquent. 

 Est-ce que le recours à une adoption ou à une mère porteuse est aussi satisfaisant pour les enfants concernés que pour leur(s) parent(s) ? On aimerait que ce soit le cas, mais beaucoup d'adultes nés d'une insémination artificielle avec donneur (IAD) expriment le désir de connaître leurs « origines". Et beaucoup parmi les jeunes et les adultes nés sous X veulent connaître leur génitrice, et ceux à qui cela est refusé disent souffrir d'une peine inguérissable. Certes tous les jeunes nés sous X ou d'une IAD ne se sentent pas tourmentés par ces interrogations, mais cela ne peut que rendre dubitatif. Même si elle est assez ordinairement souhaitée par les parents légaux, et on peut humainement comprendre ce souhait, l'évacuation des parents de naissance ou des géniteurs n'est pas possible. Ces personnes font irrémédiablement partie de la relation entre les parents réels, c'est-à-dire légaux, et leurs enfants, même si c'est seulement de façon imaginaire, et même si on s'interdit d'en parler. Â défaut de pouvoir exiger d'être élevés par leurs deux parents de naissance, les enfants concernés (beaucoup d'entre eux) veulent au moins les connaître.  même si l'on ne voit pas à quoi cela pourrait leur servir, ils s'obstinent. Même si on le leur refuse ils continuent de le vouloir. Et au nom de quel droit supérieur pourrait-on les en empêcher ? Ils ont le droit pour eux au moins autant que les adultes ont le droit de vouloir un enfant. On pourrait imaginer que si la filiation adoptive était instituée comme le modèle de la filiation la réalité des parents de naissance (géniteurs) perdrait de son importance, mais ce n'est qu'une hypothèse invérifiée.

Dans le modèle de famille judéo-chrétien l'accueil de tout enfant est un devoir dès sa conception. Dans ce cadre à celui qui demande pourquoi il est né il est possible de répondre que Dieu a voulu qu'il vive (en passant si nécessaire par des maladresses humaines). Des générations d'enfants ont trouvé cette explication satisfaisante : leur narcissisme en était suffisamment étayé. Un droit absolu à l'existence leur était ainsi reconnu quoi qu'il ait pu arriver, et cela même s'ils ne correspondaient pas totalement, ou pas du tout, aux attentes de leurs parents.

Le droit à l'interruption de grossesse (IVG) a changé la donne. Dans certaines circonstances précisées par la loi l'embryon ou le fœtus a perdu la protection que le texte de la loi lui accordait inconditionnellement depuis Constantin. Devenir un jour la personne qu'il est \emph{en potentiel}, capable de discernement et de réciprocité avec autrui  n'est plus un droit. L'argument de fond c'est qu'un individu qui n'est une personne qu'en puissance a moins de droits que celui qui est d'ores et déjà une personne accomplie : le fœtus n'est pas une personne accomplie, contrairement à sa mère. Tant qu'il n'est pas né il n'est en quelque sorte qu'une partie du corps de sa mère (cf. le Droit romain). Il n'acquiert de personnalité juridique qu'à la naissance. Jusque là il n'est qu'un \emph{objet} juridique. 

 Les cas où la santé physique de la mère est sérieusement menacée par la grossesse ne posent guère de problème moraux, pas plus que ceux où le fœtus est atteint de troubles interdisant sa survie ou l'accession à un minimum de communication. Les médecins sont amenés de temps en temps à abréger sans souffrance la vie des nouveaux-nés non viables : la Hollande l'a reconnu dans le cadre du \emph{protocole de Groeningen}. La Belgique s'est également engagée dans cette voie. 
 
 Par contre lorsque c'est à première vue d'abord ou seulement le bien-être de la mère ou celui de sa famille qui sont visés par un avortement, les enfants conscients de ces situations pourraient comprendre qu'on attend d'eux de n'être pas une gêne et de ne pas coûter trop d'efforts. Ils pourraient traduire que c'est dans la réalité, et non dans leurs fantasmes les plus archaïques, que leurs parents ont sur eux droit de vie ou de mort.

 Les opposants (« pro-vie ») à l'avortement se scandalisent qu'on tue des enfants non nés puisque selon eux il n'y a rien qui les différencie radicalement des nouveaux-nés. D'autres moralistes s'appuient sur  \emph{le même constat} pour demander au contraire que soit reconnu aux parents le droit de supprimer les nouveaux-nés dont ils ne veulent pas, même viables, et notamment ceux qui présentent des problèmes biologiques non détectés au cours de la grossesse (ex : trisomie 21,~etc.). Dans un article du 2 mars 2012 publié dans le \anglais{Journal of Medical ethics}, Alberto \fsc{Giubilini} et Francesca \fsc{Minerva} proposent, à la suite de Peter \fsc{Singer}, d'étendre le droit à l'avortement au-delà de la naissance (ce qu'ils nomment \emph{avortement post-natal}). Voici un extrait de cet article (traduction personnelle) :

\begin{displayquote}
\emph{Le droit prétendu des individus (tels que fœtus et nouveaux-nés) de développer leurs potentialités, droit que certains défendent, cède devant l'intérêt de ceux qui sont actuellement (dès aujourd'hui) des personnes (parents, famille, société) de rechercher leur propre bien-être, parce que, comme nous venons de le démontrer, ceux qui sont seulement des personnes potentielles ne peuvent pas être lésés par le fait de ne pas être introduits dans l'existence. Le bien-être des personnes actuelles \emph{[c'est-à-dire le bien-être actuel des humains parvenus au stade de personnes en acte, de plein exercice]} pourrait être affecté par de nouveaux enfants (même en bonne santé), réclamant de l'énergie, de l'argent et des soins, toutes choses dont la famille peut manquer. Parfois cette situation peut être évitée par un avortement, mais parfois cela n'est pas possible. Dans ces cas du moment que les non-personnes n'ont pas de droit moral à vivre, il n'y a pas de raisons de refuser l'avortement post-natal. Nous avons certes un devoir moral envers les futures générations alors qu'elles n'existent pas encore. Parce que nous tenons pour garanti que ces personnes existeront (quelles qu'elles soient) nous devons les traiter comme des personnes actuelles du futur. Cet argument, cependant, ne s'applique pas à tel ou tel nouveau-né en particulier, parce que nous ne pouvons pas tenir pour garanti qu'il deviendra une personne un jour. Est-ce qu'il existera \emph{[en tant que personne en acte]} dépend en fait de nous et de notre choix}.

\emph{L'adoption peut-elle être une alternative à l'avortement post-natal ?}

\emph{On pourrait nous objecter que l'avortement post-natal ne devrait être pratiqué que sur les personnes potentielles qui ne pourront jamais avoir une vie digne d'être vécue. Dans cette hypothèse les individus en bonne santé et capables d'être heureux devraient être donnés à l'adoption lorsque leur famille ne peut pas les élever. Pourquoi devrions-nous tuer un nouveau-né en bonne santé alors que le confier à l'adoption ne grèverait les droits de personne mais au contraire accroîtrait le bonheur des personnes impliquées (adoptant et adopté) ?}

\emph{Notre réponse est la suivante : nous avons précédemment examiné l'argument de la potentialité (potentialité des êtres de devenir une personne) et montré qu'il n'est pas suffisamment puissant pour contrebalancer l'intérêt de ceux qui sont actuellement des personnes. En réalité combien minces puissent être les intérêts d'une personne actuelle, ils seront toujours supérieurs à l'intérêt (hypothétique) d'une personne en puissance de devenir une personne réelle, parce que ce dernier est égal à zéro. Dans cette perspective ce sont les intérêts des personnes actuelles qui ont de l'importance, et parmi ces intérêts nous devons en particulier considérer les intérêts de la mère qui peut souffrir psychologiquement si elle donne son enfant en adoption. On observe souvent que les mères de naissance rencontrent des problèmes psychologiques sérieux à cause de leur incapacité à élaborer leur perte et à surmonter leur chagrin. Il est vrai que le chagrin et le sentiment de perte peuvent accompagner l'avortement et l'avortement post-natal aussi bien que l'adoption, mais nous ne pouvons pas affirmer que pour la mère de naissance celle-ci est la moins traumatique. Par exemple, ceux qui pleurent un décès doivent accepter l'irréversibilité de la perte, mais souvent les mères naturelles rêvent que leur enfant va revenir vers elles. Cela rend difficile pour elles d'accepter la réalité de la perte parce qu'elles ne peuvent jamais être tout à fait certaines que cette perte est irréversible.}

\emph{Nous ne cherchons pas à suggérer que ce sont des arguments décisifs contre la validité de l'adoption comme alternative à l'avortement post-natal. Cela dépend beaucoup des circonstances et des réactions psychologiques. Ce que nous sommes en train de suggérer c'est que si l'intérêt des personnes actuelles doit prévaloir, alors l'avortement post-natal doit être considéré comme une option permise aux femmes qui pourraient souffrir de donner leur nouveau-né à adopter.}
\end{displayquote}

 Pour Alberto \fsc{Giubilini} et Francesca \fsc{Minerva}  il  s'agit donc de promouvoir le \emph{droit à l'infanticide}, très largement répandu dans le monde entier, mais supprimé en Europe par Constantin. Cette demande fait penser à Jonathan \fsc{Swift} et à son « \emph{Humble proposition pour empêcher les enfants des pauvres en Irlande d'être à la charge de leurs parents ou de leur pays et pour les rendre utiles au public} » (1729), mais elle est formulée avec un terrible sérieux et sans le moindre humour. Elle provoque actuellement un mouvement de refus général et horrifié. Mais combien de temps durera ce refus ? Ne peut-on imaginer qu'à force de jouer avec cette proposition on finira par en valoriser les avantages incontestables et par en accepter les aspects déplaisants\footnote{On peut se poser les mêmes questions (droit des « personnes » opposé à celui des individus qualifiés de « non-personnes ») pour les grands vieillards trop dépendants (démences séniles...) et pour tous les malades physiques ou mentaux aux capacités de relation irréversiblement dégradées. Leur euthanasie active soulagerait bien évidemment leurs familles et les systèmes d'assistance médicale et sociale des multiples problèmes qu'entraîne leur prise en charge.}? 
 
 
 

 
