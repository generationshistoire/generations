% Le 24.02.2015 :
% ~etc.
% Moyen-Âge
% ~\%
% fœtus


\part{Et après ?}


\chapter{Un enfant pour quoi ? Pour qui ?}


Le désir de serrer un bébé de chair dans ses bras -- aussi irrépressible qu'irrationnel -- est sans doute aussi répandu et aussi fort aujourd'hui que par le passé. Pourtant, si l'on en croit \fsc{Coluche},{\emph{y a des gens qui ont des enfants parce qu'ils n'ont pas les moyens de s'offrir un chien}}. Il posait à sa manière une question essentielle et relativement nouvelle : pourquoi fait-on des enfants ? Pour quoi veut-on des enfants ?

 Tous les citoyenss des pays dotés d'un système d'assistance sociale et de retraite suffisamment efficace ont besoin qu'il naisse des enfants pour financer leurs moments d'invalidité et leurs vieux jours, mais aucun n'a \emph{besoin} que ce soient ses propres enfants. D'un point de vue strictement comptable et sauf dispositifs de compensation \emph{très}  généreux des frais qu'ils entrainent, l'intérêt des hommes et des femmes \emph{des états providence} d'aujourd'hui (et seulement d'eux) est de ne pas avoir d'enfant. Leur niveau de vie et leur crédit auprès des banques sont plus élevés s'ils évitent d'investir dans une progéniture%
% [1]
\footnote{... en dehors des impôts versés pour financer l'enseignement, l'assistance sociale en direction de tous les mineurs, la santé infantile, et les allocations servies aux familles...}% 
. Dans les pays les plus socialement développés, seule la collectivité a \emph{besoin} d'enfants. Il ne faut sans doute pas chercher plus loin la faiblesse des taux de natalité de leurs membres, taux qui ne sont que la résultante des stratégies individuelles de leurs citoyens, stratégies d'autant plus rationnelles qu'on ne voit plus aujourd'hui au nom de quel argument on pourrait les leur reprocher. 
 
 Face à la désaffection du mariage et de la procréation qui menaçait l'Empire Romain dans sa survie, l'empereur Auguste a réagi en pénalisant les célibataires et ceux qui n'avaient pas d'enfants, et ses lois ont été appliquées sans faillir pendant au moins trois siècles. Si les taux de natalité baissaient dangereusement, verrions-nous à l'avenir de pareilles incitations légales%
% [2] 
\footnote{... encouragements (ou pénalisation) par l'impôt ou le calcul des retraites ? Allocations couvrant le montant des dépenses d'éducation ? Crèches, internats et autres formes de prises en charge gratuites des enfants et adolescents par la collectivité ? Soutien matériel direct aux jeunes majeurs \emph{sans} conditions de ressources parentales ?~etc.} 
à procréer ? Il existe déjà des éléments de politiques de soutien à la natalité et aux familles, auxquels est attribué le taux de natalité français, comparativement élevé pour un pays développé.
 
 Mais les malthusiens et les écologistes pensent que les problèmes de santé de notre planète ont pour origine le fait que les humains sont trop nombreux. Il faudrait en effet que le nombre de ces derniers diminue drastiquement s'ils voulaient tous consommer comme les citoyens des pays développés actuels sans épuiser les ressources disponibles et sans mettre en danger les équilibres de la nature. Cela impliquerait non pas une croissance démographique zéro, mais une décroissance très énergique. L'intérêt commun de l'humanité serait-il sa décroissance numérique et l'évitement de la reproduction jusqu'au retour à un effectif écologiquement optimal ? 
 
 
 \chapter{Vers la mort du mariage ?}

 De même que chaque période historique est caractérisée par ses représentations sur le bien, le mal, le désirable et l'insupportable, chaque période secrète aussi sa propre vision de l'avenir. Celle-ci éclaire en général beaucoup mieux sur ceux qui l'ont produite que sur l'avenir. Dans un texte publié le 29 janvier 2013 sur \href{http://www.slate.fr}{Slate.fr}, dans le cadre des polémiques préalables au vote ouvrant le mariage aux personnes de même sexe, Jacques \fsc{ATTALI} met en perspective ces polémiques. Sans prendre formellement parti il dessine l'avenir des pratiques familiales, reproductives et sexuelles, qui est selon lui le plus probable  : 

« \emph{Comme toujours, quand s'annonce une réforme majeure, il faut comprendre dans quelle évolution de long terme elle s'inscrit.}
 
« \emph{Et la légalisation, en France après d'autres pays, du mariage entre deux adultes homosexuels, s'inscrit comme une anecdote sans importance, dans une évolution commencée depuis très longtemps, et dont on débat trop peu : après avoir connu d'innombrables formes d'organisations sociales, dont la famille nucléaire n'est qu'un des avatars les plus récents, et tout aussi provisoire que ceux qui l'ont précédé, nous allons lentement vers une humanité unisexe, où les hommes et les femmes seront égaux sur tous les plans, y compris celui de la procréation, qui ne sera plus le privilège, ou le fardeau, des femmes.} 
 
« \emph{\primo La demande d'égalité. D'abord entre les hommes et les femmes. Puis entre les hétérosexuels et les homosexuels. Chacun veut, et c'est naturel, avoir les mêmes droits: travailler, voter, se marier, avoir des enfants. Et rien ne résistera, à juste titre, à cette tendance multiséculaire. Mais cette égalité ne conduit pas nécessairement à l'uniformité: les hommes et les femmes restent différents, quelles que soient leurs préférences sexuelles.}
 
« \emph{\secundo La demande de liberté. Elle a conduit à l'émergence des droits de l'homme et de la démocratie. Elle pousse à refuser toute contrainte ; elle implique, au-delà du droit au mariage, les mêmes droits au divorce. Et au-delà, elle conduira les hommes et les femmes, quelles que soient leurs orientations sexuelles, à vouloir vivre leurs relations amoureuses et sexuelles libres de toute contrainte, de tout engagement. La sexualité se séparera de plus en plus de la procréation et sera de plus en plus un plaisir en soi, une source de découverte de soi, et de l'autre. Plus généralement, l'apologie de la liberté individuelle conduira inévitablement à celle de la précarité ; y compris celle des contrats. Et donc à l'apologie de la déloyauté, au nom même de la loyauté : rompre pour ne pas tromper l'autre. Telle est l'ironie des temps présents : pendant qu'on glorifie le devoir de fidélité, on généralise le droit à la déloyauté. Pendant qu'on se bat pour le mariage pour tous, c'est en fait le mariage de personne qui se généralise.}
 
« \emph{\tertio La demande d'immortalité, qui pousse à accepter toutes mutations sociales ou scientifiques permettant de lutter contre la mort, ou au moins de la retarder.}
 
« \emph{\quarto Les progrès techniques découlent en effet de ces valeurs et s'orientent dans le sens qu'elles exigent: en matière de sexualité, cela a commencé par la pilule, puis la procréation médicalement assistée, puis la gestation pour autrui. Ces questions de bioéthique ne découlent évidemment pas des demandes d'égalité venant des couples homosexuels et concernent toutes les formes de reproduction, y compris -- et surtout -- « hétérosexuelles ». Le vrai danger viendra si l'on n'y prend garde, du clonage et de la matrice artificielle, qui permettra de concevoir et de faire naitre des enfants hors de toute matrice maternelle. Et il sera très difficile de l'empêcher, puisque cela sera toujours au service de l'égalité, de la liberté, ou de l'immortalité.}
 
« \emph{\FrenchEnumerate{5} La convergence de ces trois tendances est claire: nous allons inexorablement vers une humanité unisexe, sinon qu'une moitié aura des ovocytes et l'autre des spermatozoïdes, qu'ils mettront en commun pour faire naitre des enfants, seul ou à plusieurs, sans relation physique, et sans même que nul ne les porte. Sans même que nul ne les conçoive si on se laisse aller au vertige du clonage.}
 
« \emph{\FrenchEnumerate{6} Accessoirement, cela résoudrait un problème majeur qui freine l'évolution de l'humanité: l'accumulation de connaissances et des capacités cognitives est limitée par la taille du cerveau, elle-même limitée par le mode de naissance: si l'enfant naissait d'une matrice artificielle, la taille de son cerveau n'aurait plus de limite. Après le passage à la station verticale, qui a permis à l'humanité de surgir, ce serait une autre évolution radicale, à laquelle tout ce qui se passe aujourd'hui nous prépare. Telle est l'humanité que nous préparons, indépendamment de notre sexualité, par l'addition implicite de nos désirs individuels... » }

Cet article est un bon résumé des idées que l'on se fait couramment aujourd'hui sur les mœurs à venir, et ces perspectives font rêver beaucoup de gens. Il garde pourtant une certaine distance critique, et meme une légère ironie. Il aboutit par tout un autre chemin aux memes conclusions que Nietszche sur le mariage  : \emph{On vit pour aujourd'hui, on vit très vite -- on vit sans aucune responsabilité : c'est précisément ce que l'on appelle « liberté ». Tout ce qui fait que les institutions sont des institutions est méprisé, haï, écarté : on se croit de nouveau en danger d'esclavage dès que le mot « autorité » se fait seulement entendre.[...] Témoin : \emph{le mariage moderne}. Apparemment toute raison s'en est retirée : pourtant cela n'est pas une objection contre le mariage, mais contre la modernité. La raison du mariage -- elle résidait dans la responsabilité juridique exclusive de l'homme : de cette façon le mariage avait un élément prépondérant, tandis qu'aujourd'hui il boite sur deux jambes. La raison du mariage -- elle résidait dans le principe de son indissolution : cela lui donnait un accent qui, en face du hasard des sentiments et des passions, des impulsions du moment, \emph{savait se faire écouter}. Elle résidait de même dans la responsabilité des familles quant au choix des époux. Avec cette indulgence croissante pour le mariage \emph{d'amour} on a éliminé les bases mêmes du mariage, tout ce qui en faisait une institution. Jamais, au grand jamais, on ne fonde une institution sur une idiosyncrasie ; je le répète, on ne fonde pas le mariage sur « l'amour », -- on le fonde sur l'instinct de l'espèce, sur l'instinct de propriété (la femme et les enfants étant des propriétés), sur \emph{l'instinct de la domination} qui sans cesse s'organise dans la famille en petite souveraineté, qui a \emph{besoin} des enfants et des héritiers pour maintenir, physiologiquement aussi, en mesure acquise de puissance, d'influence, de richesse, pour préparer de longues tâches, une solidarité d'instinct entre les siècles. Le mariage, en tant qu'institution, comprend déjà l'affirmation de la forme d'organisation la plus grande et la plus durable : si la société prise comme un tout ne peut \emph{porter caution} d'elle-même jusque dans les générations les plus éloignées, le mariage est complètement dépourvu de sens. -- Le mariage moderne a perdu sa signification -- par conséquent on le supprime.}  (\emph{Le Crépuscule des idoles} 1888)


 Malgré tout le mal qu'en pense Nietszche, l'exigence d'égalité absolue entre hommes et femmes est un acquis de notre temps sur lequel il est peu probable que l'on revienne. Cette exigence d'égalité est d'ailleurs encore loin d'avoir produit tous ses effets. Sur quel argumentaire serait-il possible de fonder de manière convaincante un retour vers une inégalité fondée sur le sexe de naissance ou sur le genre ?
 
 Par contre l'exigence de liberté absolue, à tout prix et quelles qu'en soient les conséquences, est grosse de problèmes que pointe Jacques \fsc{ATTALI} : \frquote{\emph{l'apologie de la liberté individuelle conduira inévitablement à celle de la précarité ; y compris celle des contrats. Et donc à l'apologie de la déloyauté, au nom même de la loyauté : rompre pour ne pas tromper l'autre. Telle est l'ironie des temps présents : pendant qu'on glorifie le devoir de fidélité, on généralise le droit à la déloyauté.}} Peut-on croire qu'il pourrait exister un domaine  dans lequel aucun engagement, aucune parole donnée n'aurait de valeur, et que cela n'aurait pas de répercussions importantes sur les autres domaines de la vie humaine ? Une telle chose est-elle possible lorsqu'il s'agit d'un domaine aussi charnellement lié à la construction de chaque sujet ? Peut-on croire qu'il n'y aurait aucun effet en termes de « lien social » ?
 
 Quant à la valorisation de l'immortalité individuelle, qu'est-elle sinon une des facettes de l'individualisme ou de l'égocentrisme le plus narcissique, ce qui revient \emph{in fine} à mettre en question l'intéret des liens avec autrui : \emph{Il n'y a pas de plus grand amour que de donner sa vie pour ceux qu'on aime} (Jn 15, 13).
 
  \chapter{Pourquoi faire lien ?}
 
 A part durant le dernier siècle, et encore, personne n'a jamais cru que la passion amoureuse était suffisante pour se marier. Aveugle elle doit plus aux représentation du sujet passionné qu'aux caractéristiques véritables de son objet d'élection. Elle est une expression de son narcissisme, et source d'illusions. Depuis l'Antiquité jusqu'au dix-neuvième siècle oinclus les moralistes s'en sont toujours méfié, la voyant comme une entrave à l'exercice de la raison, comme une force aveugle, inconstante, décevante et potentiellement destructrice de tous les liens et de tous les principes sur lesquels repose la société, d'où leur insistance sur l'éducation à la maîtrise des passions : répression des instincts, valorisation de l'intellect, culture du devoir, apprentissage de la frustration, etc... 
 
 Est-ce qu'on pourrait au moins penser que la passion amoureuse est utile pour construire un couple durable ? Constatons que les couples n'ont jamais été aussi fragiles que depuis que le choix amoureux est censé règner en maître sur la nuptialité. Il est vrai aussi que la plupart des étais qui confortaient le lien conjugal ont été enlevés. Parmi ceux-ci en dehors de l'intéret matériel, ne reste plus que la préoccupation pour les enfants communs, ce qui, on est contraint de le constater,  ne crée pas un lien aussi solide que des étais juridiques \emph{ad hoc}.
 
 Mais la vraie question n'est-elle pas : pourquoi chercher à construire un couple durable ? Pourquoi pas plutôt une succession d'amours éphémères et d'histoires (d'amour) successives ? ...
 
 ou pas d'amours du tout, mais plutôt une concentration sur des projets moins communs ? collections de livres rares ou de montres anciennes, recherche scientifique ou création d'une oeuvre, etc...
 
Comment construire une réponse à cette question ?

Les positions religieuses ne convainquent que ceux qui y croient. Les positions morales sont variées et contradictoires : d'un côté (en perte de vitesse, comme le pointent Nietszche et Jacques Attali) on valorise l'engagement et la fidélité à la parole donnée, de l'autre (qui a le vent en poupe) on plaide pour l'authenticité et la liberté. Aujourd'hui le recours à la morale ne peut plus emporter l'adhésion commune. Ne reste comme point de départ incontestable que l'utilité : pour notre société dans son ensemble qu'est-ce qui est préférable ? la fidélité ou l'authenticité ?

On peut débattre des avantages respectifs de la fidélité et de l'authenticité pour le bonheur et pour la santé physique et morale des membres des familles, enfants et parents. Mais en l'absence d'observations précises le risque est qu'on s'enlise dans des dialogues de sourds et qu'on constate que les positions en compétition sont inconciliables. Comment construire des indicateurs fiables et reconnus par (presque) tous ? A défaut de données plus pertinentes on peut au minimum chiffrer l'ensemble des coûts et des avantages directs et indirects de chacune de ces deux positions (domaines financier, de santé, d'ordre public...). 

On pense d'abord à tous les surcoûts engendrés par le "démariage" sur le marché immobilier et celui de l'équipement domestique. 

Mais il y en a bien d'autres : l'évolution de la législation et des mœurs ne va pas dans le sens du renforcement des capacités éducatives des familles. Elle multiplie le nombre des situations où la fonction éducative de l'un ou de l'autre des parents est plus ou moins disqualifiée ou empêchée, tandis que le remplacement au quotidien de l'un des parents de naissance par un partenaire sexuel et affectif de l'autre n'est pas toujours accepté par les jeunes concernés et ne présente pas forcément l'efficacité éducative nécessaire. Le nombre s'élève donc des parents qui face à leurs enfants sont plus ou moins seuls et/ou en difficulté. Lorsque la prise en charge éducative de ceux-ci fait problème, notamment à l'adolescence, des étais extérieurs sont souhaitables, mais même en tenant pour négligeable la mise en dépendance de parents plus ou moins disqualifiés par les "spécialistes" et "experts" de l'éducation, les soutiens que propose la collectivité ne sont pas gratuits (ex. internats scolaires, assistance éducative, placement en famille d'accueil,~etc.). On passe de « l'auto production » familiale des activités éducatives à leur « externalisation » et à leur « professionnalisation ». Comme c'est un domaine où il n'y a pas à espérer de gain de productivité cela accroît les coûts éducatifs de manière très sensible. Jusqu'où peut-on aller dans cette voie avant que la collectivité n'estime que c'est trop cher payé ? 

Enfin rien ne garantit que l'efficacité des diverses aides éducatives apportées aux parents soit en règle générale supérieure à ce qu'en d'autres circonstances ils auraient pu assumer eux-mêmes : ce serait déjà bien si on pouvait être assuré qu'elle ne soit pas moindre. 

 

\chapter{des conjugalités aux modes de filiation}


Les lois autorisant le divorce par consentement mutuel, la contraception et l'avortement ont dénaturalisé le modèle de famille traditionnel si bien que même si rien ne pourra jamais empêcher personne de croire à la valeur éthique, éducative ou civique de la {\emph{sainte famille}}, fondée sur un couple hétérosexuel, monogame et indissoluble, élevant lui-même les enfants nés de ses œuvres, ce n'est plus qu'un choix parmi d'autres également légitimes. Dans la mesure où ce n'est plus qu'une possibilité parmi d'autres, comme c'était le cas avant les décisions de Constantin, et dans la mesure où cette possibilité n'est plus étayée par la loi et la puissance publique, comme elle l'était par l'Ancien Régime ou le Code Napoléon, la « sainte famille », la famille « traditionnelle » ne peut plus être identifiée avec ce que j'ai appelé la « famille constantinienne ». Il s'agit seulement de l'une des façons (post constantiniennes) d'organiser la reproduction humaine, à côté de diverses autres, dont la paternité, la maternité et l'adoption célibataires, le concubinage, le PACS ou le mariage hétéro ou homosexuel,~etc.

On pourrait imaginer d'autres configurations, notamment la parentalité célibataire. Un nombre grandissant de femmes, cadres surtout, mais pas seulement, élèvent désormais leurs enfants toutes seules, sans reconnaître un père. 

Dans {\emph{Quelle alternative au patriarcat ? Valoriser un modèle social non conjugal} (2004)} Agnès \fsc{ECHENE} accuse le couple hétérosexué d'etre le lieu privilégié d'expression et de transmission de la violence machiste, et cela trop souvent avec la complicité (masochiste) féminine. C'est pourquoi {\emph{ce n'est qu'en valorisant le modèle social non conjugal qu'une société peut se défaire du patriarcat. Il importe donc de favoriser une sexualité libre et variée, tout en étant discrète et protégée, surtout chez nos propres enfants ; peu importe dès lors qu'elle soit ardente ou paisible, monotone ou changeante, homophile ou hétérophile, dès l'instant qu'elle reste une affaire personnelle dont nul ne se mêle. Une telle évolution nécessite également une reconsidération du modèle familial qui doit se refonder sur des liens d'appartenance utérine et non pas consanguine ; cela remet en cause dès lors la paternité génitale qui doit laisser place à une paternité germaine : il faut en effet que ce soit les frères, oncles et cousins \emph{[de la mère]} qui assument les enfants des femmes ; de nombreux signes avant-coureurs montrent qu'ils sont prêts à le faire et qu'il ne manque qu'un déclic. Mais il faut aussi que les femmes renoncent à obliger les géniteurs à être pères ; il faut qu'elles abandonnent toute velléité de recherche de paternité, de pension, partage, alternance,~etc. et se tournent résolument vers leurs frères, oncles et cousins pour « donner » des pères à leurs enfants, qui ne s'en porteront pas plus mal.}}
 
 
 Va-t-on vers des foyers constitués d'une femme et des enfants qu'elle a mis au monde, autour desquels graviterait la nébuleuse de ses amants et ex-amants ? Dans cette hypothèse, les hommes de demain auraient des enfants de plusieurs femmes, enfants vivant ordinairement chez leurs mères, si bien que leur autorité sur chacun d'eux serait pratiquement nulle ? 
 
 Serait-ce l'inverse de la situation du \emph{pater familias} romain, qui pouvait demander à plusieurs femmes des enfants sur lesquels lui seul avait autorité ? Cela se rapprocherait surtout de la configuration familiale matrifocale du modèle « antillais » ou « caraïbe », dont l'origine se situe dans l'histoire du peuplement des Antilles. On a vu que les esclaves n'ont par définition aucun des attributs juridiques d'un père ou d'une mère sur les enfants dont ils sont les géniteurs et génitrices : seuls les propriétaires des génitrices possèdent des droits sur les enfants de celles-ci. C'étaient ces propriétaires qui faisaient d'elles des mères lorsqu'ils leur confiaient la garde des enfants qu'elles avaient portés, quel qu'en soit le géniteur. Il y avait une espèce d'alliance de fait entre les génitrices et leurs maîtres pour élever les enfants qu'elles avaient mis au monde, tandis que leurs partenaires sexuels n'avaient pas droit à la parole et étaient réduits à n'être que des donneurs de sperme. 

 
 Face à ces parentalités célibataires faut-il comprendre qu'à l'avenir ce pourrait être l'État qui assumerait le rôle de tiers traditionnellement dévolu aux pères, à l'aide de ses services judiciaires et socioéducatifs ? Tiers qui soutient matériellement (cf. toutes les prestations « sociales » pour "parent isolé") et psychologiquement le parent (homme ou femme) dans sa tâche éducative, et qui introduit les exigences du monde extérieur au sein de la dyade parent-enfant.
 
 À rebours du modèle constantinien qui télescope sur le couple des seuls géniteurs toutes les dimensions de la conjugalité et de la parentalité (juridique, biologique, affective et éducative) et qui frappe tout le reste d'illégitimité, le modèle de la filiation élective, volontaire, adoptive, est aujourd'hui valorisé. 
 
 Ainsi \fsc{BORRILLO}%
% [1] 
\footnote{\frquote{Les enjeux de la parentalité}, Daniel \fsc{Borrillo}, \emph{Encyclopedia Universalis}.} 
écrit que {[...] \emph{la filiation peut certes tenir compte du fait naturel, mais, en tant que dispositif d'agencement parental, elle répond à des règles propres, affranchies de la nature... \emph{[La filiation]} n'existe que lorsqu'elle est établie dans les conditions et selon les modes prévus par la loi. Autrement dit, la filiation est déterminée par la norme juridique et non par la nature. Ce lien juridique se tisse à partir de quatre fils principaux : la biologie (filiation par le sang), la volonté (adoption), la présomption (paternité supposée du mari de la mère) et le vécu (appelé en droit « possession d'état »).}}

 {\emph{En Occident, la coutume des barbares (ordalies) et le droit canonique (\emph{copula carnalis}, coït charnel) ont eu en commun la vérité du corps comme fondement du lien juridique, contrairement à la civilité romaine pour laquelle la volonté constituait la clé de voûte du système juridique. Le droit moderne des personnes physiques qui s'esquisse à partir du \siecle{18} va opérer un retour aux règles du droit civil romain, en accordant à l'autonomie de la volonté une place centrale dans l'établissement des liens de filiation et une place éminente à la fiction juridique \emph{(fictio legis)}, créatrice de droits.}}
 
 {\emph{Ce qui compte ce ne sont plus tant les racines naturelles ou surnaturelles d'institutions intangibles que l'efficacité et la plasticité d'instruments juridiques procurant tel ou tel résultat (par exemple la paix des familles ou la solidarité des générations).}}
 
 Selon lui :  {\emph{Fondée sur la volonté, l'adoption est une institution plus apte que la vérité biologique à assurer la stabilité des liens familiaux.}} 
  
 {\emph{Si l'adoption, et non la capacité reproductrice, était retenue comme modèle universel de la filiation, cela permettrait de fonder la parenté sur la responsabilité et sur un projet parental réfléchi, et d'éviter ainsi toutes les figures malheureuses des maternités ou des paternités non désirées.}} 
 
 {\emph{Ainsi la pure contractualisation des liens familiaux permettrait de laisser dans les mains des principaux intéressés la limitation et le contenu de cette communauté des affects, de volontés désirantes qui est l'essence de la famille.}}
 
 {\emph{Ainsi, la vie de couple cesserait d'être limitée à deux personnes de sexe différent, et l'ordre générationnel ne serait plus borné aux lignées masculine et féminine mais serait ouvert à la parenté unisexuée.}}
 
 {\emph{La contestation actuelle de l'ordre familial « naturel » n'est en définitive que la radicalisation de l'idéologie individualiste moderne, selon laquelle la volonté et non la différence des sexes constitue la base de l'institution matrimoniale et parentale. Une filiation dissociée de la reproduction permettra de justifier un système juridique fondé non pas sur la vérité biologique, mais sur le projet parental responsable. De ce point de vue, peu importe l'agencement familial (traditionnel, monoparental, homoparental, recomposé...), si les prémisses du contrat (égalité dans l'alliance et dans la filiation) sont respectées jusque dans leurs moindres effets. L'État devrait donc traiter sur un plan d'égalité l'ensemble des familles et les autres formes d'intimité.}}
 
 {\emph{La coexistence du mariage, du Pacs et du concubinage pour tous les couples répondrait à cette exigence tout comme l'ouverture du droit à l'adoption, à l'A.M.P. et aux maternités de substitution au-delà des cas de stérilité.}}
 
 {\emph{Contrairement à la filiation charnelle, la filiation choisie trouve son principe dans la liberté non seulement d'accueillir les enfants des autres, mais également d'abandonner ses propres enfants biologiques, ce qui est pour l'heure uniquement possible pour les femmes (accouchement sous X), mais devrait pouvoir s'élargir aussi aux hommes à travers une déclaration formelle de renoncement à la paternité. La généralisation de la filiation adoptive (y compris pour ses propres enfants biologiques) permettrait aussi de mettre la volonté au cœur du dispositif parental. Celui-ci reposerait exclusivement sur la volonté du ou des géniteurs qui donnent l'enfant et celle du ou des adoptants qui l'accueillent. De surcroît, l'adoption est une institution conçue à partir du droit de l'enfant à avoir une famille, contrairement à la filiation biologique qui apparaît plutôt comme un dispositif du droit à l'enfant.}}
 
Selon Daniel \fsc{BORRILLO} {\emph{les progrès scientifiques -- congélation de sperme, d'ovules ou d'embryons, insémination artificielle, fécondation in vitro, identification génétique des parents -- ont provoqué \emph{[...]} une \frquote{panique morale}}}  face aux possibilités vertigineuses de dissociation entre sexualité et reproduction qui se sont ouvertes en à peine une génération. 

 D'où une tendance des théoriciens et des praticiens du droit à valoriser de manière à ses yeux excessive, sinon exclusive, les liens biologiques parents-enfants : {\emph{la biologie commença à devenir ainsi le soubassement réel ou symbolique%
% [2] 
\footnote{Le \emph{biologique} comme fondement \emph{symbolique} du système de parenté ?!?} 
du système de parenté, à rebours d'une science juridique qui avait plutôt instauré la volonté au cœur de ce système \emph{[...]} À partir des années 1990, l'expertise biologique s'est imposée dans les procès en contestation de paternité, la recherche des origines est revendiquée socialement comme droit fondamental de la personne, la différence de sexe est devenue une valeur \emph{[...]} La nouvelle place prépondérante de la vérité biologique dans l'établissement du lien filial fut confirmée en France par la Cour de cassation%
%[3]
\footnote{... dans un arrêt du 28 mars 2000 établissant que {\emph{l'expertise biologique est de droit en matière de filiation, sauf s'il existe un motif légitime de ne pas l'ordonner}}. Civ. 1\iere, 28 mars 2000, Bull. \no 103 ; \hbox{Defrénois}, \hbox{30.06.2000}, \no~12, p. 769, note J.~\fsc{Massip} ; \hbox{Dalloz}, \hbox{12.10.2000}, \no 35, p. 731, note T.~\fsc{Garé} ; JCP \hbox{25.10.2000}, \nos 43-44, conclusions C.~\fsc{Petit} et note M.C.~\fsc{Monsallier-Saint-Mieu}.}% 
... Par là, la distinction traditionnelle entre reproduction (fait biologique) et filiation (fait culturel), fondement du droit civil moderne, se trouvait questionnée... non pas à partir d'arguments classiques provenant du droit canonique, mais par une rhétorique qui, d'une part, fera de la différence des sexes une condition \emph{sine qua non} de la filiation, et, d'autre part, placera l'expertise sanguine et la preuve d'ADN au cœur du dispositif juridique de la parenté.}}
 
 {\emph{La vérité biologique apparaît comme l'argument fondamental non seulement pour s'opposer à la filiation homoparentale, mais aussi pour créer une sorte de hiérarchie des filiations par référence à la procréation naturelle, et finalement pour désigner les familles monoparentales ou recomposées comme cause de dysfonctionnements individuels et sociaux.}}
 
 On peut souligner le caractère provocateur de l'idée d'une {\emph{déclaration formelle de renoncement à la paternité}}, en miroir du droit reconnu aux femmes à \emph{l'accouchement sous X}. Pourtant une telle disposition ne ferait que rejoindre le point de vue révolutionnaire : pas de contrainte en parentalité. Pas de contrainte en paternité.
 
 
 


\chapter{Avenir du droit à l'enfant ?}


 À l'exception des orphelins, il n'existe pas d'enfant adoptable qui n'ait d'abord été abandonné. Du fait de la généralisation des recours aux procédés anticonceptionnels et aux interruptions volontaires de grossesse, les abandons de nouveaux-nés sont de plus en plus rares dans les pays développés : depuis bien longtemps il y en a beaucoup moins que de demandeurs. L'adoption des enfants plus âgés n'est pas simple et elle peut être terriblement éprouvante pour le narcissisme des adoptants : le nombre des enfants qui ont été adoptés et qui sont par la suite abandonnés par leurs parents adoptifs n'est pas négligeable. En fait les enfants adoptés risquent malheureusement plus que les autres enfants d'être abandonnés à cause des difficultés de tous ordres rencontrées avant et après l'acte d'adoption, par eux-mêmes, par leurs parents de naissance et par leurs parents adoptifs. Tout le monde n'est pas prêt à prendre de pareils risques.

 Dans ces conditions, comment ceux et celles qui ne peuvent ou ne veulent pas procréer, et qui ne renoncent pas pour autant à leur désir d'enfant, qui n'est après tout ni plus ni moins légitime que celui des autres, se procureront à l'avenir les enfants qu'ils désirent ? Pour le moment il demeure possible d'adopter les bébés des pays sous-développés qu'abandonnent celles des pauvres qui ne peuvent prévenir leurs grossesses autant qu'elles le voudraient, mais cela ne durera qu'un temps : et après ? Comment s'y prendra-t-on ?

 On pourrait dire cyniquement qu'il est toujours possible, dans les zones de non-droit, de faire disparaître des parents pour prendre leurs petits enfants, ou d'enlever les enfants qui sont les moins bien surveillés. C'est même très lucratif. Mais ces procédés criminels, dont les exemples contemporains ne manquent pas, ne peuvent donner à long terme de bons résultats : que répondre lorsque les anciens bébés enlevés demandent d'où ils viennent ? Si on leur dit la vérité, ils ne peuvent que prendre parti pour leurs parents de naissance, poussés qu'ils sont par la nécessité vitale de sauvegarder leur propre narcissisme. Et même si on la leur cache ils flairent le mensonge avec une sûreté (inconsciente) imparable, et ce mensonge empoisonnera toutes leurs relations jusqu'à sa levée (au moins).

 Mais il n'est pas nécessaire de recourir à des méthodes criminelles : tant que dureront les énormes inégalités de revenu observables sur cette planète, les plus fortunés pourront toujours louer le ventre des plus belles et des plus saines des filles des pauvres, de la même façon que les riches romains achetaient les plus jolies des jeunes esclaves afin qu'elles leur fassent des enfants bien à eux qu'ils n'auraient à partager ni avec un partenaire égal à eux en dignité, ni avec une belle famille aussi puissante que la leur. Le recours à des « mères porteuses » est dans la logique des évolutions libérales actuelles. Il est d'ores et déjà légalement possible dans plusieurs pays développés. Est-il appelé à se généraliser ? Comment refuser ce recours aux hommes homosexuels si l'on accorde l'assistance médicale à la procréation (PMA) aux femmes homosexuelles, et comment le refuser à tous les autres, hommes et femmes, si on l'accorde aux hommes homosexuels ? Et comment le refuser à qui que ce soit si des femmes (souvent pauvres et vivant dans des pays sous-développés) sont volontaires pour prêter leur ventre et abandonner leur enfant nouveau-né contre une indemnité suffisante. 

 C'est le seul moyen de mettre les hommes à égalité avec les femmes dans l'accès à l'enfant, ou plutôt de corriger l'inégalité que leur corps leur impose dans ce domaine, mis à part bien sûr le mariage traditionnel, monogame et indissoluble, dont c'était l'une des finalités. Lorsque leur mariage était rompu les pères romains gardaient leurs enfants : ils n'avaient donc pas particulièrement intérêt à ce que les unions soient indissolubles. Par contre leurs épouses avaient de bonnes raisons de craindre d'être répudiées et séparées de leurs enfants. Elles ont peut-être trouvé bon d'être mieux protégées de ce risque à partir du IVème siècle. Aujourd'hui où leur autonomie financière et les lois leur permettent de prendre l'initiative de quitter leurs maris sans quitter leurs enfants la situation se retourne et ce sont les hommes qui peuvent commencer de craindre d'être séduits puis abandonnés. 

 Plutôt que de se retrouver un jour contraints de continuer de payer pour leurs enfants sans plus les avoir auprès d'eux, tandis que souvent un autre qu'eux les éduque, les hommes pourraient choisir, quelle que soient par ailleurs leurs préférences sexuelles, de commencer par payer pour les posséder sans partage afin que personne ne puisse jamais les leur contester. Sur quels arguments fonder le refus d'une pareille évolution ? Elle ne serait au fond que le miroir de celle qui voit des femmes choisir en toute connaissance de cause de faire un enfant toutes seules. Si les humains ne diffèrent en rien de significatif en dehors de leurs caractéristiques biologiques, si les femmes n'ont pas besoin d'un homme pour élever un enfant, alors les hommes n'ont pas non plus plus besoin d'une femme pour assumer l'éducation de leurs propres enfants.

 En dépit de la pression des demandes individuelles et du modèle fourni par les pays où cette pratique est autorisée, le recours aux mères porteuses pourrait être interdit s'il était admis qu'il implique la réduction d'un humain au statut d'instrument de la volonté d'un tiers jusque dans son corps, s'il était reconnu que c'est inacceptable, même si cette personne a donné son accord, parce que cela fait de l'enfant à naître le produit d'un contrat commercial, toutes choses qui sont au cœur de l'esclavage. Mais refuser ce recours impliquerait aussi d'accepter l'idée qu'il n'existe pas de droit à l'enfant, c'est-à-dire que chacun peut être irrémédiablement privé d'enfant en dépit de ses désirs les plus authentiques et les plus légitimes. 

 Le mouvement des pratiques depuis un demi-siècle ne va pas dans ce sens.
 
 
  
 
 
 \chapter{Peut-on éclairer ses propres points aveugles ?} 
 
 Dans \emph{L'avenir d'une illusion} (1927), \fsc{FREUD} se demande jusqu'où une société humaine peut se permettre d'être souple et tolérante étant donnée la violence des pulsions, désirs et angoisses qu'elle a pour tâche d'humaniser. Il répond qu'une grande dose de répression est inévitable, et que c'est même une des conditions de l'élaboration d'œuvres culturelles de valeur.

 Dans une période donnée peuvent être inapparents, inconscients, ou plutôt innommables et innommés, déniés, les traits de dureté qu'elle n'a pas suffisamment élaborés, les blocs de sauvagerie qu'elle n'a pas su penser. Ce sont des points aveugles dans la représentation que cette période se donne d'elle-même. Ils sont involontaires et personne ne les a voulus en connaissance de cause... par contre ils sauteront aux yeux des générations suivantes qui ne comprendront pas comment il a été possible de ne pas les voir. 
 
 Le plus bel exemple est que depuis des milliers d'années on a admis comme un fait établi et ne souffrant pas la discussion que les femmes étaient inférieures aux hommes, faites pour leur obéir et les servir, et qu'il était donc indispensable qu'une part plus ou moins grande de leurs droits soient détenus et exercés par des membres de leur entourage. 

 L'histoire des enfants sans parents est elle aussi marquée par plusieurs de ces points aveugles, à commencer par la dureté du sort fait partout, depuis toujours et jusqu'aujourd'hui en toute bonne conscience aux enfants de naissance illégitime, quelles qu'aient été les manières successives de définir en quoi leur naissance était illégitime, c'est-à-dire inopportune. Quoi de plus barbare que la croyance en une impureté ou une infamie de naissance ? Quoi de plus arbitraire et déraisonnable que l'idée qu'être né d'un ou d'une esclave interdisait irrévocablement de prétendre à des postes à responsabilité ? Quoi de plus étrange pour nous que la valeur religieuse du sang, ou la « pureté » d'une généalogie ? Quoi de plus absurde que de disqualifier moralement les « enfants du péché » tout en absolvant ceux qui avaient commis le « péché » dont ils étaient nés ? 

 Tout se passe comme si les conceptions archaïques du pur et de l'impur avaient continué d'être tenues pour vraies jusqu'à nos jours alors que le caractère moralement insatisfaisant de ces représentations avait été dénoncé il y a deux mille ans par les stoïciens aussi bien que par les évangiles, dont pourtant les thèses ont été méditées sans interruption depuis lors. Jusqu'au début du \siecle{20} chacune de ces propositions, en théorie insoutenables, du point de vue même de ceux qui s'y conformaient, a été tenue pratiquement pour vraie par tous ou presque tous, ou par chacun presque tout le temps. Jusqu'à Vincent de Paul on n'appelait pas négligence le sort qui était fait aux nouveaux-nés abandonnés, parce que les exclure du monde des familles légitimes paraissait être la façon correcte de les traiter et qu'on n'en imaginait pas d'autre. Lui a su le premier ou l'un des premiers, voir en eux autre chose que des êtres religieusement impurs qu'il était moralement indifférent de laisser mourir du moment qu'ils étaient baptisés. C'est sur les représentations de ses contemporains qu'il a travaillé et non sur l'art d'accommoder les bébés séparés de leur mère (cet art ne posait pas plus de problèmes à la majorité des femmes de son époque qu'à celles d'aujourd'hui). 

 Il y a moins d'un siècle les mineurs vagabonds étaient encore considérés et traités comme des délinquants : la criminalisation de leurs errances avait commencé à la fin du Moyen-Âge : auparavant on les assimilait aux pèlerins et on se recommandait à leurs prières. 

 De même il n'y a guère plus d'un demi-siècle on regardait encore avec méfiance les rencontres entre les enfants placés en institution et leurs parents. 

 Et il n'y a pas beaucoup plus de trente ans qu'on a pris la mesure de la gravité des dégâts psychologiques produits par les sévices sexuels perpétrés par les adultes sur les enfants, surtout quand ils ont autorité sur eux. 
 
 Tous ces points "aveugles" étaient en évidence aux yeux de tous, mais les violences et les cruautés commises faisaient d'autant moins problème qu'elles paraissaient aussi inexorables que le jour et la nuit, aussi normales et naturelles que le soleil et la pluie (bien évidemment ce sont les personnes qui étaient aveugles). Il n'est donc pas impossible qu'aujourd'hui même s'étalent sous nos yeux des malheurs et des souffrances que nous ne voyons pas, des maltraitances que nous tolérons ou que nous produisons en toute bonne conscience. Si c'est réellement le cas, alors dans un siècle, ou dans dix, on nous reprochera de les avoir méconnus, sans comprendre que nous ne pouvions pas les voir, aveuglés que nous étions par nos théories, nos croyances ou nos intérêts inconscients, de la même façon que nous sommes scandalisés par la brutalité, l'insensibilité, l'aveuglement et les aberrations des logiques de nos prédécesseurs. 
 
 En l'absence d'observateurs venus d'un autre monde seules des recherches scientifiques sont en mesure d'apporter peu à peu des éléments de réponse à de tels aveuglements, et toujours trop lentement.
 
 \chapter{Quels droits pour l'enfant ?}
 
 Est-ce que les lois et les pratiques qui encadreront à l'avenir la conception des enfants et l'art de les accommoder produiront moins de souffrances et de troubles que celles du passé chez les enfants et chez leurs parents ? On ne voit pas bien en quoi les enfances organisées par les manipulations de la biologie et des relations interpersonnelles évoquées plus haut par Jacques Attali seraient un progrès du point de vue des enfants. Il est vrai que ce n'est pas leur objectif. 
 

 Le recours à la prévention des naissances, à la pilule anticonceptionnelle, à la pilule » du lendemain » et à l'avortement permet en principe qu'il ne naisse plus d'enfants non désirés. Mais suffit-il que ceux qui naissent aient été désirés par leurs géniteurs ou par leurs parents adoptifs pour que disparaissent les problèmes qu'ils posent ou ceux qu'ils rencontrent ? Les enfants ne sont pas sans influence, pour le meilleur et pour le pire, sur la relation que leurs parents construisent avec eux. Ils peuvent déplaire à leurs parents (volontairement ou non) sur des points auxquels ces derniers sont viscéralement attachés. Nul ne peut garantir qu'à l'avenir il y aura moins d'enfants mal assumés que par le passé. 

 Si la pauvreté matérielle n'est plus depuis longtemps un motif suffisant à lui seul pour séparer les enfants de leurs parents, est-on assuré pour autant qu'il n'existe et n'existera plus jamais d'enfants privés de l'un ou de l'autre de leurs parents alors que ceux-ci sont disponibles, volontaires pour les élever et suffisamment compétents ? L'absence de l'un des deux parents pour d'autres raisons que la maladie ou la mort devient au contraire quelque chose de plus en plus fréquent. 


 Est-ce que le recours à une adoption ou à une mère porteuse est aussi satisfaisant du point de vue des enfants que du point de vue de leur(s) parent(s) ? Le désir de connaître leurs « origines » est de plus en plus fermement affirmé par beaucoup d'adultes nés d'une insémination artificielle avec donneur (IAD), tout comme celui des jeunes et des adultes nés sous X de connaître leur génitrice. Cela ne peut que rendre dubitatif. Même si elle est assez ordinairement souhaitée (et on peut humainement comprendre ce souhait), l'évacuation par les parents légaux des parents de naissance, des géniteurs, n'est pas possible. Les parents de naissance font irrémédiablement partie de la relation entre les parents légaux et leurs enfants, même si c'est seulement de façon imaginaire. A défaut de pouvoir exiger d'être élevés par leurs deux parents de naissance, les enfants veulent au moins les connaître. Même si on le leur refuse ils continuent de le vouloir. Et au nom de quel droit supérieur pourrait-on les en empêcher ? Ils ont le droit pour eux au moins autant que tous les adultes ont le droit de vouloir un enfant. 
 
 Dans le modèle de famille judéo-chrétien  l'accueil de tout enfant est un devoir aussitot qu'il est conçu. Dans ce cadre à celui qui demande pourquoi il est né il est possible de répondre que Dieu l'a voulu. Des générations d'enfants ont trouvé cette explication suffisante : leur narcissisme en était suffisamment étayé. Un droit absolu à l'existence leur était ainsi reconnu quoi qu'il arrive, et cela même s'ils ne correspondaient pas totalement, ou pas du tout, aux attentes de leurs parents.
  
 Le droit à l'interruption de grossesse (IVG) a changé la donne. Dans certaines circonstances précisées par la loi l'embryon ou le fœtus a perdu la protection que la loi lui accordait inconditionnellement depuis Constantin. Devenir un jour la personne qu'il est \emph{en potentiel}, capable de discernement et de réciprocité avec autrui  n'est plus un droit. L'argument de fond c'est qu'un individu qui n'est une personne qu'en puissance a moins de droits que celui qui est d'ores et déjà une personne accomplie : le fœtus n'est pas une personne accomplie, contrairement à sa mère. Tant qu'il n'est pas né il n'est en quelque sorte qu'une partie du corps de sa mère (cf. le droit romain). Il n'acquiert de personnalité juridique qu'à la naissance. Jusque là il n'est qu'un \emph{objet} juridique. 

 Les cas où la santé physique de la mère est sérieusement menacée par la grossesse ne posent guère de problème moraux, pas plus que ceux où le fœtus est atteint de troubles interdisant sa survie ou l'accession à un minimum de communication. Les médecins sont amenés de temps en temps à abréger sans souffrance la vie des nouveaux-nés non viables : la Hollande l'a reconnu dans le cadre du \emph{protocole de Groeningen}. La Belgique s'est également engagée dans cette voie. 
 
 Par contre lorsque c'est à première vue le bien-être de la mère ou celui de sa famille qui sont visés par un avortement, les enfants conscients de ces situations pourraient entendre qu'on attend d'eux de n'être pas une gêne et de ne pas coûter d'efforts. Ils pourraient comprendre que c'est dans la réalité, et non dans leurs fantasmes les plus archaïques, que leurs parents ont sur eux droit de vie ou de mort.

 Les opposants ("pro-vie") à l'avortement se scandalisent qu'on tue des enfants non nés puisque selon eux il n'y a rien qui les différencie radicalement des nouveaux-nés. D'autres moralistes \emph{s'appuient sur le même constat} pour demander au contraire que soit reconnu aux parents le droit de supprimer les nouveaux-nés dont ils ne veulent pas, même viables, et notamment ceux qui présentent des problèmes biologiques non détectés au cours de la grossesse (ex : trisomie 21,~etc.). D'autres vont encore plus loin. Dans un article du 2 mars 2012 publié dans le \emph{Journal of Medical ethics}, Alberto \fsc{Giubilini} et Francesca \fsc{Minerva} proposent, à la suite de Peter \fsc{Singer}, d'étendre le droit à l'avortement au-delà de la naissance (ce qu'ils nomment \emph{avortement post-natal}). Voici un extrait de cet article (traduction personnelle) :

« \emph{Le droit prétendu des individus (tels que fœtus et nouveaux-nés) de développer leurs potentialités, droit que certains défendent, cède devant l'intérêt de ceux qui sont actuellement des personnes (parents, famille, société) de rechercher leur propre bien-être, parce que, comme nous venons de le démontrer, ceux qui sont seulement des personnes potentielles ne peuvent pas être lésés par le fait de ne pas être introduits dans l'existence. Le bien-être des personnes actuelles \emph{[c'est-à-dire le bien-être actuel des humains parvenus au stade de personnes en acte, de plein exercice]} pourrait être affecté par de nouveaux enfants (même en bonne santé), réclamant de l'énergie, de l'argent et des soins, toutes choses dont la famille peut manquer. Parfois cette situation peut être évitée par un avortement, mais parfois cela n'est pas possible. Dans ces cas du moment que les non-personnes n'ont pas de droit moral à vivre, il n'y a pas de raisons de refuser l'avortement post-natal. Nous avons certes un devoir moral envers les futures générations alors qu'elles n'existent pas encore. Parce que nous tenons pour garanti que ces personnes existeront (quelles qu'elles soient) nous devons les traiter comme des personnes actuelles du futur. Cet argument, cependant, ne s'applique pas à tel ou tel nouveau-né en particulier, parce que nous ne pouvons pas tenir pour garanti qu'il deviendra une personne un jour. Est-ce qu'il existera \emph{[en tant que personne en acte]} dépend en fait de nous et de notre choix}.

« \emph{L'adoption peut-elle être une alternative à l'avortement post-natal ?}

« \emph{On pourrait nous objecter que l'avortement post-natal ne devrait être pratiqué que sur les personnes potentielles qui ne pourront jamais avoir une vie digne d'être vécue. Dans cette hypothèse les individus en bonne santé et capables d'être heureux devraient être donnés à l'adoption lorsque leur famille ne peut pas les élever. Pourquoi devrions-nous tuer un nouveau-né en bonne santé alors que le confier à l'adoption ne grèverait les droits de personne mais au contraire accroîtrait le bonheur des personnes impliquées (adoptant et adopté) ?}

« \emph{Notre réponse est la suivante : nous avons précédemment examiné l'argument de la potentialité (potentialité des êtres de devenir une personne) et montré qu'il n'est pas suffisamment puissant pour contrebalancer l'intérêt de ceux qui sont actuellement des personnes. En réalité combien minces puissent être les intérêts d'une personne actuelle, ils seront toujours supérieurs à l'intérêt (hypothétique) d'une personne en puissance de devenir une personne réelle, parce que ce dernier est égal à zéro. Dans cette perspective ce sont les intérêts des personnes actuelles qui ont de l'importance, et parmi ces intérêts nous devons en particulier considérer les intérêts de la mère qui peut souffrir psychologiquement si elle donne son enfant en adoption. On observe souvent que les mères de naissance rencontrent des problèmes psychologiques sérieux à cause de leur incapacité à élaborer leur perte et à surmonter leur chagrin. Il est vrai que le chagrin et le sentiment de perte peuvent accompagner l'avortement et l'avortement post-natal aussi bien que l'adoption, mais nous ne pouvons pas affirmer que pour la mère de naissance celle-ci est la moins traumatique. Par exemple, ceux qui pleurent un décès doivent accepter l'irréversibilité de la perte, mais souvent les mères naturelles rêvent que leur enfant va revenir vers elles. Cela rend difficile pour elles d'accepter la réalité de la perte parce qu'elles ne peuvent jamais être tout à fait certaines que cette perte est irréversible.}

" \emph{Nous ne cherchons pas à suggérer que ce sont des arguments décisifs contre la validité de l'adoption comme alternative à l'avortement post-natal. Cela dépend beaucoup des circonstances et des réactions psychologiques. Ce que nous sommes en train de suggérer c'est que si l'intérêt des personnes actuelles doit prévaloir, alors l'avortement post-natal doit être considéré comme une option permise aux femmes qui pourraient souffrir de donner leur nouveau-né à adopter.} »

 Pour Alberto \fsc{Giubilini} et Francesca \fsc{Minerva}  il  s'agit donc de promouvoir le \emph{droit à l'infanticide}, très largement répandu dans le monde entier, mais supprimé par Constantin. Cette demande fait penser à Jonathan \fsc{Swift} et à son « \emph{Humble proposition pour empêcher les enfants des pauvres en Irlande d'être à la charge de leurs parents ou de leur pays et pour les rendre utiles au public} » (1729), mais elle est formulée avec un terrible sérieux et sans le moindre humour. Elle provoque actuellement un mouvement de refus général et horrifié. Mais combien de temps durera ce refus ? Ne peut-on imaginer qu'à force de jouer avec cette proposition on finira par en valoriser les avantages incontestables et par en accepter les aspects déplaisants ? 
  

 On peut se poser les mêmes questions (droit des « personnes » opposé à celui des individus qualifiés de « non-personnes ») pour les grands vieillards trop dépendants (démences séniles...) et pour tous les malades physiques ou mentaux aux capacités de relation irréversiblement dégradées. Leur euthanasie active soulagerait bien évidemment leurs familles et les systèmes d'assistance médicale et sociale des multiples problèmes qu'entraîne leur prise en charge.
 

 
