%E1 Le tournant Constantinien 
%E2 Constantin et le droit des personnes 
%F1 entrée en scène des barbares 
%F2 Les sociétés du Bas-Empire et du Haut Moyen Âge 
%F3 l'esclavage chez les chrétiens de l'antiquité tardive et du haut moyen-âge 
%F4 clercs et religieux 
%F5 Le "mariage constantinien" 
%F6 Familles de chair 
%F7 Familles spirituelles 
%G1 Les familles de l'Ancien Régime entre autorités civiles et religieuses 
%G2 Les devoirs des pères de l'Ancien Régime 
%G3 Création d'une police des familles (XIVème-XVIIIème siècles)

\part[Les familles de l'ancien régime]{Les familles de l'ancien régime}

\chapter[Entre autorités civiles et religieuses]{Entre autorités civiles et religieuses}



\section{Le mariage constantinien}

Ce que j'appelle le mariage constantinien, c'est en première approximation le mariage tel que le conçoit l'église, avec ses obligations et ses contraintes, \emph{plus}  l'appui des autorités civiles. Au terme d'une longue évolution commencée par Constantin, il n'y a plus de droit civil du mariage à la fin du Moyen-Âge dans une Europe totalement christianisée où les seuls dissidents religieux tolérés sont les juifs. L'église a gagné suffisamment d'ascendant moral et de prestige intellectuel pour faire prévaloir une grande partie de sa doctrine. Depuis la réforme Grégorienne le mariage et la famille sont de son ressort exclusif. Les autorités civiles se soumettent, bon gré mal gré, à ses décisions : c'est celle-ci qui reconnaît les mariages, qui ordonne les séparations, qui reconnaît les nullités de mariage, qui décide des interdits de mariage, et qui accorde des dispenses, etc...

Le « mariage constantinien » c'est le mariage des romains de l'antiquité tardive tel qu'il a été modifié par Constantin et ses successeurs pour l'accommoder aux conceptions chrétiennes. Il télescope plusieurs fonctions distinctes sur une seule personne : l'époux est à la fois le détenteur des droits juridiques de son épouse (son curateur), son amant (à qui on recommande la mesure), le géniteur de ses enfants, le détenteur des droits de ces enfants mineurs, et le responsable de leur éducation, c'est-à-dire leur père légal. Symétriquement, une épouse est la seule femme capable de donner à son époux des enfants légitimes, des héritiers, quel que soit le nombre de ses concubines. Pour être légitime chaque enfant doit être l'enfant biologique de ses parents légaux (leur enfant « naturel » au sens antique du terme). Et par définition seuls les enfants légitimes ont droit à une part d'héritage et à succéder à leurs parents.



Les religieux sont voués au célibat par choix : leurs voeux de pauvreté, d'obéissance à un supérieur et de chasteté sont incompatibles avec une vie de famille. Pour les pretres et les éveques qui ne font pas de voeux de chasteté leur célibat a d'abord pour objectif de les empecher d'avoir des fils légitimes, de transmettre à ceux-ci leurs bénéfices (cure, éveché, etc...) et de créer des dynasties cléricales. 
 
En dehors de ces exceptions le célibat est licite, mais chez les jeunes gens sans enfants, en bonne santé et disposant de moyens matériels suffisants, il est suspect d'égoïsme, de libertinage ou de désirs « contraires à la nature » (homosexualité dont la mise en acte a toujours été condamnée moralement). Quels que soient les préférences individuelles la copulation n'est légitime que dans l'état de mariage monogame, qui est le seul moyen acceptable de répondre à l'ordre divin (le \emph{"croissez et multipliez"} de la Genèse).  Comme la fin première du mariage est la procréation d'enfants légitimes les remariages sont déconseillés (quoique autorisés) quand cette fin n'est pas atteignable étant donné l'âge ou l'état de santé des conjoints. 
 
 

 Pour l'Église, \emph{c'est le mariage qui fonde la famille, et non la présence d'enfants}, même si elle met l'accueil des enfants au premier rang des « fins du mariage ». De son point de vue, il crée \emph{dès sa célébration} une parenté nouvelle, \emph{une seule chair}, entre les époux, \emph{qu'ils soient féconds ou non}. Cette parenté « par alliance » a des effets directs et immédiats sur les membres des parentèles des époux (frères, sœurs,~etc.) : elle étend le cercle des partenaires qui leur sont désormais définitivement interdits, même si l'un des époux décède.

 Selon la doctrine chrétienne, identique sur ce point au droit romain, ce sont les époux qui s'unissent l'un à l'autre : cela implique qu'ils soient capables de discernement (âge suffisant, santé mentale) et libres de leur personne : célibataires ou veufs, non esclaves, non engagés par contrat dans une entreprise qui empêcherait la vie commune, à l'abri de toute pression, libres de tout vœux religieux, sexuellement aptes au mariage. L'Église a toujours soutenu contre les parents que les jeunes gens ont le pouvoir de se marier validement sans leur accord, même si elle admettait qu'en leur désobéissant ces jeunes gens les déliaient de leur devoir de les établir dans la vie. Contrairement au droit romain elle en est progressivement venue à ne reconnaître la réalité juridique d'un mariage que lorsqu'il a été consommé. 
 
 Contrairement au droit romain l'Eglise pose depuis sa fondation que le "oui" que se donnent les époux est irrévocable et elle enseigne que le mariage est indissoluble. Le divorce est interdit\footnote{...pour les chrétiens, mais pas pour les juifs soumis aux tribunaux rabbiniques qui l'autorisent.} quel que soit le motif. Seules sont possibles les actions en nullité de mariage, ou les actions en séparation (\emph{divortium}) avec interdiction de se remarier du vivant du conjoint. C'est sans doute l'un des points où les frictions entre clercs et laics ont été les plus vives (cf. Henri VIII, etc...).



 Les deux époux se doivent réciproquement fidélité, et même si les infidélités du mari ne sont pas sanctionnées par la loi, au contraire de celles de l'épouse, tout est fait pour qu'il n'ait aucun intérêt à entretenir des maîtresses. Cela ne l'empêche évidemment pas d'avoir des rapports avec des prostitué(e)s, rapports qui par nature ne s'inscrivent pas dans la durée et sont moins menaçants pour l'épouse. 
  
  Chacun des deux époux reconnaît à l'autre un droit sur son propre corps et a l'obligation de satisfaire autant qu'il est en son pouvoir ses désirs sexuels, ce qui veut dire que l'épouse doit accepter les étreintes de son mari, quoi qu'elle puisse penser des risques de grossesse et de santé à quoi cela l'expose, et quels que soient ses propres désirs. Ceci dit la modération est prêchée aux maris, qui se voient prescrire la continence de nombreux jours par an. Le \emph{devoir conjugal} n'est exempt de faute morale que si aucun obstacle n'est mis à la fécondation (éjaculation ailleurs que dans le vagin, pessaire, douche intime, ~etc.). 

Il n'est pas permis de se débarrasser des enfants non voulus par l'avortement ou par l'infanticide. Sauf indigence extrême il n'est pas non plus permis de s'en débarrasser par l'exposition ni la vente. Une femme n'a donc pas à craindre qu'on l'oblige contre son gré à avorter ou à abandonner son nouveau-né. Mais sa fécondité ne lui appartient pas plus qu'elle n'appartient à son mari, et pas plus que son mari elle n'a droit de vie ou de mort sur l'enfant qu'elle porte.


 Il n'est pas possible de répudier une épouse présumée stérile (en cas de stérilité dans un couple, c'est celle de la femme qui est toujours suspectée en premier) au motif de sa stérilité supposée. 
Les couples stériles, dont le nombre n'est pas du tout négligeable jusqu'à l'avènement de la médecine moderne, 20 % environ, et ceux dont aucun enfant n'ont atteint vivant l'âge adulte, sont invités à consacrer aux bonnes œuvres, aux pauvres et à l'Église les ressources qu'ils auraient transmises à leurs héritiers s'ils en avaient eus.

En cas de décès du père, c'est la mère qui, chez les Romains et à partir de 390, exerce la tutelle de ses enfants mineurs (et d'eux seuls) si elle a cinquante ans et plus, et du moins tant qu'elle ne se remarie pas, ce qui est le cas général des veuves dotées d'enfants\footnote{les femmes avec enfants trouvent un compagnon moins facilement que les autres. C'est encore vrai aujourd'hui.}. À partir de 390 une femme n'est donc plus considérée comme incapable par nature de représenter juridiquement une autre personne qu'elle-même. 

 Tous les enfants nés hors mariage sont pénalisés. Il n'est guère ou pas du tout possible pour un homme de se faire des héritiers sans se marier, même si la prise en charge d'\latin{alumnii} et leur installation dans l'existence est une bonne œuvre. La \emph{légitimation par mariage subséquent} est la seule exception de plein droit\footnote{... jusqu'au \siecle{20}. Les enfants irréguliers légitimés par les empereurs, les rois ou les papes, ne l'étaient pas de plein droit mais à la faveur d'une grâce, qui pouvait être refusée sans justification, et qui n'allait pas sans contreparties coûteuses.} 
à la pénalisation des enfants nés hors mariage, et ses conditions sont strictes. Chacun, quelque puissant qu'il soit, doit savoir que s'il a l'imprudence de faire un enfant hors mariage ou dans un mariage contesté par son curé, par son évêque, par son seigneur, par le roi ou par sa propre parentèle, il ne pourra pas le faire reconnaître comme un de ses héritiers sans combat ou sans procès. Cet enfant ne pourra probablement pas lui succéder. L'exhérédation totale ou partielle des enfants illégitimes est restée jusqu'à la fin du \siecle{20} le premier frein apporté au désir des hommes de se procurer une descendance autrement qu'avec leur épouse légitime.

 Et la perspective de se retrouver avec un enfant à charge, seule, sans aucun espoir d'une légitimation (ni même d'une aide significative venant du père de l'enfant lorsqu'il était déjà marié puisque aucune donation au-delà des frais d'éducation n'était autorisée depuis Constantin) a été un obstacle majeur à l'exercice d'une sexualité féminine en dehors du mariage ou avant le mariage. 

 Les épouses savent qu'il est, sinon impossible, du moins difficile de les chasser de leur maison ou de leur imposer de cohabiter avec une concubine
\footnote{... même si pour ceux dont la puissance excède de beaucoup celle du commun des mortels, aristocrates, rois, la question peut se présenter différemment, et si les amours ancillaires sont de tous les temps.}% 
. Elles sont à peu près assurées que les infidélités de leur époux n'entraîneront de conséquences graves ni sur elles, ni sur leurs enfants, ni sur le futur héritage de ceux-ci. Tout au plus des « aliments » devront-ils être versés aux enfants nés de leurs maîtresses, mais cela ne portera que sur d'assez petites sommes et seulement jusqu'à ce qu'ils soient mis au travail : 8-10 ans. Il n'est pas question de financer leur établissement dans la vie. 



Contrairement au droit romain de l'antiquité le concubinage n'est pas traité comme une union de second ordre, propre aux gens qui n'ont pas d'héritage à transmettre, mais comme un état de fornication durable. Il n'est possible à un homme de légitimer les enfants qu'il a obtenus d'une concubine qu'en l'épousant (\emph{mariage subséquent}), et à la condition de ne pas etre déjà marié. 



La première des méthode de limitation des naissances autorisées est la continence. C'est une méthode dont les résultats sont immédiats. A plus long terme on peut y ajouter  le retard de l'âge au mariage des filles : si au lieu de se marier à 16 ans comme beaucoup de filles bien dotées elles attendent 24 ans en économisant pour se constituer une petite dot elles "gagnent" huit années sans grossesses\footnote{...soit 4 enfants environ en moyenne, sans contraception et avec allaitement.}. 

Les parents peuvent aussi utiliser l'entrée en religion de certains de leurs enfants pour diminuer à terme le nombre de leurs descendants et concentrer leurs ressources sur l'établissement des autres. Meme s'il faut ordinairement une "dot" pour entrer dans les couvents et monastères son montant dépend de la notoriété de l'établissement, et il suffit de ne pas viser trop haut pour faire des économies sérieuses. 
Les voeux des religieux et religieuses sont reconnus par les autorités civiles. Le voeu de pauvreté les enlève définitivement tout droit à l'héritage de leurs parents. Leur voeu d'obéissance à leur supérieur les délie face à eux de leur devoir d'obéissance. Ils ont en quelque sorte un statut de \emph{morts civils}. Il est possible d'etre relevé de ces voeux, mais cela demande une procédure lourde et aléatoire. Une fois un enfant placé au couvent il lui sera difficile de brouiller les stratégies familiales.  
   



 



Les laïcs n'ont jamais totalement épousé les points de vue des clercs et jusqu'à la Réforme Grégorienne (\siecle{11}), l'Église n'avait pas le monopole du droit familial. Les autorités civiles ne se sentaient pas obligées d'appuyer toutes ses prétentions dans un domaine aussi critique pour la transmission du pouvoir. Les écarts entre le droit religieux (droit \emph{canon}) et les lois civiles n'ont jamais été nuls, pour ne pas parler de \emph{l'à peu près} avec lequel ces lois étaient respectées. Ce que j'appelle le mariage constantinien est donc un modèle qui n'a jamais été pleinement réalisé, et surtout pas sous Constantin. Pourtant il tendra peu à peu à s'incarner dans les pratiques et les représentations, et il ne sera peut-être jamais aussi bien respecté que durant les derniers siècles de notre ancien régime.

Malgré sa rigueur, et même si elle n'a souvent été respectée que de manière très lâche la morale familiale et sexuelle enseignée par l'Église est devenue au fil des siècles la norme. Si du \crmieme{11} au \siecle{16} des mouvements de contestation religieuse se succèdent, qui culmineront avec la réforme protestante, rares sont ceux qui à cette période mettent vraiment en question la morale familiale et sexuelle enseignée par l'Église. Au contraire les opposants s'appuient sur elle pour critiquer les écarts des clercs avec leurs propres principes. Cette morale a été formulée dès les premiers temps de l'Église, mais les règles de droit qui en découlent ont mis au moins un millénaire à s'imposer comme le droit commun. Ceci dit jamais elles n'ont réussi à le faire partout ni parfaitement. Certaines régions s'y sont conformées avec beaucoup de rigueur tandis que d'autres ont été beaucoup plus tolérantes avec les irrégularités.  Le christianisme a certes contribué à façonner les sociétés d'ancien régime, mais il en a été lui-même fortement influencé. Il lui a été demandé de bénir, et même de sacraliser, leurs mécanismes et leurs logiques, et de les conforter dans leur fonctionnement, et c'est ce qu'il a souvent fait. 




 
 
 
 
 \section{traitement des naissances illégitimes}

 Les contraintes et limites imposées à la reproduction par le roi et par l'Église n'ont jamais été acceptées totalement ni par tous. C'était notamment le cas dans la noblesse. Depuis l'Antiquité tardive et durant tout le Moyen Âge elle était tenue, par elle-même et par les autres ordres de la société, pour une \emph{race} supérieure qui transmettait ses vertus par son sang. Cette antique croyance n'accordait pas d'importance au statut juridique ou religieux de l'union des parents. Elle coexistait sereinement dans les têtes avec le modèle canonique judéo-chrétien. 

 Les membres les plus puissants de la noblesse, et d'abord les rois eux-mêmes, n'ont jamais cessé de pratiquer une polygamie de fait qui leur donnait de nombreux enfants, de second rang certes, mais parfois bien utiles à défaut ou en complément d'enfants légitimes et valeureux. Jusqu'au \siecle{11} les différences faites dans les familles puissantes entre enfants légitimes et bâtards nés du chef de famille étaient faibles (capacité d'hériter, de succéder,~etc.). En effet le sang du père ennoblissait. Cette conception était un héritage des mœurs d'inspiration germanique du haut Moyen Âge. Le nombre des bâtards nobles semble même avoir crû au \siecle{15}. Par comparaison les bourgeois reconnaissaient beaucoup moins d'enfants illégitimes. De 1400 à 1649 les rois de France ont reconnu 24~\% d'enfants naturels tandis que les grands officiers, employés roturiers de la maison du roi, n'en avouaient que 10,3~\%. 

 Alors que le mariage des grands seigneurs ne répondait habituellement qu'à des critères politiques, les enfants qu'ils avaient conçus avec leurs maîtresses, \emph{enfants de l'amour}, étaient fréquemment \emph{réussis}. S'ils étaient légitimés, ces enfants pouvaient leur permettre des alliances profitables. Or les rois d'Europe, héritiers en cela aussi de l'empereur de Rome, pouvaient légitimer les « bâtards » par \emph{lettres royaux}. Les bâtards des familles aristocratiques ont donc souvent été légitimés par le roi ou par mariage subséquent. Même s'ils ne l'étaient pas, cela n'a pas fait problème pendant longtemps. Souvent ils n'ont été légitimés qu'après la mort de leur père, \latin{ad honores}, c'est-à-dire pour accéder aux honneurs qu'ils détenaient, c'est-à-dire pour leur succéder dans les emplois d'intérêt public, les charges qu'ils exerçaient. Par contre aucune reine, princesse du sang ou femme d'officier n'a pu légitimer d'enfant naturel autrement que par un mariage subséquent : c'est l'homme qui ennoblissait, c'est également lui qui légitimait. 

 Les cas d'illégitimité susceptibles de bénéficier d'une légitimation par mariage subséquent avaient été étendus par les rois au-delà des critères de Constantin, de façon à inclure les enfants nés d'une relation passagère, et ceux conçus dans le cadre d'un enlèvement suivi d'un viol (enfants dont le géniteur n'était pas forcément le futur mari). Par ce biais la fiction retrouvait une place dans la filiation : l'adoption par l'époux de la mère redevenait possible. 

 Et pourtant les déclarations officielles de l'Église stigmatisaient tous les enfants illégitimes. Le Concile de Bourges (1031) confirmait les jugements des conciles antérieurs (« semence maudite »). Et l'Église continuait d'écarter de la prêtrise les fils de prêtres, sauf dispense (à vrai dire facilement accordée). 



 Ceux que les gens du Moyen Âge nommaient « bâtards » étaient d'abord les enfants abandonnés dont on ne connaissait pas les parents. À partir du \siecle{11}, le mot « bâtard » devient un terme de mépris, une injure. Dès le \siecle{12}, avec la renaissance du droit romain, le bâtard n'appartient plus à aucune famille même dans les pays de droit coutumier\footnote{En gros les pays situés au nord de la Loire, opposés aux pays de droit (romain) écrit, situés au sud de la Loire.} 
: pas même celle de sa mère. Être un « bâtard » était une tare, et semble avoir été de plus en plus pénalisant du Moyen Âge au \siecle{18} : est-ce vraiment pour des raisons religieuses ? ou parce que la société reposait sur l'alliance des familles, alliance que protégeait la mise hors jeu des enfants nés en dehors de ce cadre ?

 Dans une société où l'on n'était fils ou fille de quelqu'un que si l'on était né de deux parents légitimement mariés, un jeune de naissance illégitime ou né de parents inconnus portait une tare indélébile. Il était considéré comme fils de personne, hors parenté, même s'il vivait au foyer de l'un de ses deux parents (la mère en général). Un enfant non légitime ne pouvait ni succéder à un membre de sa parentèle dans un office, ni en hériter, sauf si aucun autre héritier légitime n'y trouvait ombrage. Personne ne se portait caution pour lui. Sans père il ne pouvait pas apporter sa contrepartie dans un système d'alliance. Il était condamné à une position marginale, du moins par rapport à celle de ses éventuels demi-frères et sœurs. En contrepartie de ces exclusions il n'était pas non plus contraint de se porter caution pour un parent. Il n'avait aucune autorisation parentale à demander pour convoler : ses géniteurs comme ses tuteurs ne pouvaient pas le lui interdire.

 L'Hôpital du Saint Esprit de Paris était initialement un hospice destiné à toutes les personnes démunies. À la fin du Moyen Âge il était devenu l'orphelinat de Paris. Vers le milieu du \siecle{15} le roi lui a demandé de prendre en charge les enfants abandonnés de Paris. Tout roi qu'il fût ses demandes réitérées ont été récusées par les maîtres de l'un des hôpitaux de sa propre ville : \emph{en faisant prévaloir les statuts de fondation et la bonne réputation dont jouissent les enfants qu'il \emph{[l'hôpital]} entretient et éduque déjà. Si tous les orphelins d'origine inconnue lui étaient conduits, les gens de métier qui viennent chercher un apprenti, ou les jeunes compagnons qui y prennent femme ne seraient plus assurés de la légitimité, et partant de la moralité de l'adolescent}%
% [7]
\footnote{\fsc{SAUNIER}, \emph{Le « pauvre malade » dans le cadre hospitalier médiéval, France du Nord}, vers 1300-1500, 1993, p. 53.}% 
.

 Réellement convaincu par ces arguments, ou bien de guerre lasse, le roi a confirmé les maîtres de l'Hôpital du Saint-Esprit dans l'idée que leurs statuts et eux-mêmes se faisaient de leur mission : en 1445 il a donc accepté qu'on n'y admet que les \emph{orphelins et orphelines nés en loyal mariage} et à qui on ne peut reprocher \emph{la tache de bâtardise}, car, selon un autre argument fourni par les maîtres de l'hôpital, \enquote{[...] \emph{ſy on y admettoit des baſtards, il ſeroit à craindre que la division ne ſe miſt bientôt dans cette maiſon par les reproches continuels que les enfants légitimes feroient aux baſtards}%
% [8]
\footnote{Lettres patentes de Charles~VII du 7 août 1445, A.A.P. Saint-Esprit, L, II, p. 32 ; cité par \fsc{SAUNIER}, id. p. 212.}% 
}.

 À cette époque les maîtres du Saint-Esprit faisaient remettre tous les enfants exposés qu'on leur présentait aux paroisses sur le territoire desquelles on les avait trouvés, alors qu'ils acceptaient de prendre en charge les adolescents légitimes (orphelins pauvres) qui sortaient convalescents de l'Hôtel-Dieu. Même les léproseries excluaient les bâtards \enquote{\emph{parce que les gardes des maladreries diſaient que les bâtards n'avaient pas de lignage, ni n'étaient à hériter de nul droit par quoy ils ne ſe pouvaient aider de leur maiſon, pas plus qu'un étranger qui ſerait venu d'Eſpagne}%
% [10] 
\footnote{\fsc{SAUNIER}, id. p. 213.}%
}. Mais ces mêmes léproseries admettaient sans réserves les lépreux sans ressources s'ils étaient de naissance légitime : l'indigence leur faisait encore moins peur que l'illégitimité.

 Jusqu'à la fin de l'ancien régime les « bâtards » étaient exclus de nombreux métiers. En règle générale les corporations les refusaient, de la même façon et pour les mêmes raisons que la prêtrise leur était interdite. Maître Jacques \fsc{Ducros}, avocat au Parlement de Bordeaux, et premier Consul d'Agen en 1659, écrit dans ses \href{http://www.babordnum.fr/files/original/859d36685f2d7b2f871c648ea08bd103.pdf}{\emph{Réflexions singulières sur l'ancienne coutume d'Agen}}  :
%
\begin{displayquote}

{[...] \emph{les batards n'ont pas le bonheur de paſſer pour des domeſtiques%
% [11] 
\footnote{Ici « domestiques » signifie « appartenant à une maison », pas forcément comme salarié au service du maître. Cela inclut aussi tous les enfants et parents vivant dans la maison.} 
ny d'auoir rien en propre dans les maiſons. Ils ſont les productions du vice \& les enfans d'iniquité. Les peres les forment dans les tenebres \emph{[et]} les meres en cachent la conception. A meſme qu'ils sont nez , ces infames producteurs les deſauoüent. Les enfans legitimes cherchent le iour \& la lumiere, les illegitimes la nuit \& l'obſcurité. A proprement parler ce ſont des excremens, deſquels à meſme que la nature les chaſſe \& les pouſſe dehors, on couure d'ordure \& de ſaleté : ils n'ont ny nom ny race ny famille , c'eſt pourquoy ils ne peuuent eſtre admis au nombre des proches de ſang de conſanguinité ny d'allience}}%
%[12]
\footnote{\fsc{CAPUL}, Thèse, tome II, p. 111.%
\label{notecapul111}}%
.

\end{displayquote}


 \fsc{CAPUL} rapporte que lors des États généraux de 1614, le Tiers-état d'Agenais demande au roi : \enquote{\emph{que toutes lettres de légitimation ſeront deſnyees a tous enfens nez d'inceſte, d'adultère ou filz de prebſtres, et qu'on n'y aura aucun eſgard, ſoit pour ſucceſion, dignites, offices, bennefices (eccléſiaſtiques) et tous autres droitz}}%
% [13]
\footnote{%\fsc{CAPUL}, Thèse, tome II, p. 111.}%
Voir note \ref{notecapul111}.}%
. Les places désirables étaient trop peu nombreuses pour se montrer généreux. Ceci dit leur démarche conforte l'idée que les interdits qui pesaient sur les « fruits du péché » pouvaient assez aisément être tournés avec de l'argent et de l'entregent, mais cela ne concernait que les rares enfants illégitimes qui étaient investis par des parents puissants ou fortunés. Ainsi Erasme, (1469-1536), « prince des humanistes », âme de la « république des lettres » de son temps, était-il fils de prêtre. Fils d'un père cultivé il reçut une instruction soignée dans les écoles monastiques de son temps et entra en 1688 chez les chanoines de Saint-Augustin, où il fut ordonné prêtre en 1492. S'il ne fit pas une brillante carrière dans les allées du pouvoir temporel, comme évêque ou cardinal, c'est parce qu'il refusa les offres qu'on lui en fit au profit de la recherche intellectuelle et théologique, où il réussit il est vrai de manière exceptionnelle. Sa bâtardise et le statut ecclésiastique de son père ne semblent avoir fait problème à personne.

 Les bourgeois prospères qui représentaient leurs concitoyens de l'Agenais, et qui exprimaient probablement l'opinion publique de leur époque, ne s'identifiaient en aucune façon aux enfants nés des unions sexuelles illicites, pourtant innocents des actes de leurs pères et mères. Ils ne toléraient pas que les « bâtards » soient confondus avec les enfants légitimes, et surtout pas avec les orphelins. Ces députés tenaient fermement à ce qu'aucun passe-droit ne puisse désavantager leurs propres fils dans la course aux honneurs, et leurs filles dans la chasse aux maris. C'était la défense la plus intransigeante de la morale conjugale qui servait leurs intérêts, puisqu'elle leur permettait d'écarter une partie des concurrents nobles ou bourgeois de leurs propres enfants.  
 
 

\section{Conflits de juridictions}

 Le droit romain n'a jamais totalement disparu dans les pays de droit écrit, au Sud de la Loire ou en Italie, mais à partir de sa redécouverte au \siecle{12} il a connu une nouvelle faveur en tant que modèle et outil de réflexion. La Renaissance a vu le triomphe du droit tel que les empereurs chrétiens l'ont mis en forme%
% [4]
\footnote{Justinien~I\ier{} (483 - 565) ou Justinien le Grand, empereur de Byzance de 527 à 565, essaya de restaurer l'unité de l'empire romain. Il a ordonné et dirigé une compilation du droit romain, le \latin{Corpus iuris civilis}, qui est l'une des bases du droit civil de divers pays européens.}% 
. Si bien qu'au bout de plus d'un millénaire, c'étaient encore les choix de Constantin et de ses successeurs immédiats qui modelaient en profondeur les mœurs familiales européennes : celles-ci n'ont jamais été aussi conformes à ses décrets qu'aux \crmieme{16} et \siecle{17}. À partir de la Renaissance et jusqu'au \siecle{20} les femmes \emph{mariées} ont été pratiquement réduites à un statut de mineures. Par rapport au Moyen Âge le retour en faveur du droit romain a appesanti l'autorité des pères sur les enfants, et contribué à enlever aux femmes, et surtout aux épouses, une part significative des libertés économiques et juridiques que le Moyen Âge leur avait reconnues.

 Du \crmieme{13} au \siecle{18} les autorités civiles reprennent peu à peu une grande partie du terrain qu'elles avaient concédé aux autorités religieuses au fil du haut Moyen Âge. Le \emph{Concordat de Bologne} (1516) accorde au roi de France le droit de nommer les titulaires des principaux bénéfices (évêques et abbés et abbesses), en dépit la règle qui depuis l'Antiquité voulait qu'ils soient désignés (élus) par leur communauté. 

 À partir du \siecle{14} un petit nombre de curés ont commencé de tenir des \emph{registres de catholicité} où ils enregistraient les baptêmes (autant dire les naissances dans un monde où tous sont baptisés) et parfois aussi les décès et les mariages. L'intérêt de ces registres était de faire foi dans les procès éventuels, même si manquaient les témoins capables de répondre à des questions concernant par exemple l'âge des personnes, ou leurs liens de parenté,~etc. En raison de cet intérêt quelques évêques ont ordonné à tous leurs curés d'en faire autant. \emph{L'Ordonnance de Villers-Cotterêts} (1539) a généralisé à tous les curés du royaume de France l'obligation d'enregistrer par écrit tous les baptêmes. \emph{L'Ordonnance de Blois} (1579) la complète en ordonnant que soient également notés sur ces registres tous les mariages et tous les décès, ce qui permettait de lutter contre les bigames. À cela s'ajoute l'obligation légale d'une publication des bans préalable au mariage, préconisée depuis longtemps par les conciles, mais appliquée de manière irrégulière, afin que ceux qui connaîtraient un empêchement au mariage projeté puissent le déclarer en temps utile. Par ces diverses initiatives et par d'autres les autorités civiles ont repris pied dans le domaine matrimonial. Le contrôle des unions importait en effet au moins autant aux rois et aux parents qu'à l'Église, et les mariages avaient un effet déterminant sur le bon fonctionnement de la société civile, sur la paix des familles et sur l'organisation économique. Les juges royaux ont cherché et trouvé, ou reçu du roi, des moyens de contester certaines des décisions des juges ecclésiastiques, si bien qu'on a pu observer un retour progressif du contentieux des mariages devant les tribunaux civils. 

 Le conflit le plus rude entre les autorités civiles et les autorités religieuses a porté sur la place à donner à l'autorité des parents sur les unions. Selon l'Église catholique les conjoints s'unissaient irrévocablement l'un à l'autre par leur « oui », devant le prêtre qui n'était qu'un témoin représentant l'Église, un témoin privilégié à partir du moment où il tenait les registres d'état civil (le curé de la paroisse de l'un ou l'autre des époux sauf dispense). La position traditionnelle de l'Église était que l'autorisation parentale n'était pas nécessaire pour la validité du mariage, même si elle déconseillait aux jeunes gens de s'en passer et si elle ne contestait pas aux parents le droit de déshériter les contrevenants. Les autorités civiles et les familles pensaient au contraire qu'un mariage, alliance entre deux familles et contrat civil, ne pouvait pas être valide, quel que soit l'âge des conjoints, sans l'accord formel de leurs parents. Du point de vue de ces derniers (et des clercs eux-mêmes lorsqu'ils ne parlaient pas en tant que représentants de l'Église, mais en tant que membres d'une famille particulière) l'absence de cet accord était une preuve du manque de bon sens, de l'immaturité des deux jeunes concernés, ou de la perversité de l'un d'eux (cf. la réaction du chanoine Fulbert, oncle d'Héloïse, face au mariage secret, et néanmoins canoniquement valide, de sa nièce avec Abélard). 

 Malgré la pression du roi de France, et le besoin qu'ils avaient de son appui et de ses armées, les évêques rassemblés en concile à Trente ont refusé de modifier la doctrine traditionnelle. Le roi a donc promulgué \emph{l'édit de 1556} qui ne contestait pas la validité religieuse des mariages célébrés sans l'accord des parents \emph{(mariages clandestins)} mais qui les déclarait civilement illégaux. Il confirmait le droit traditionnellement accordé aux parents de déshériter les enfants qui se rendaient coupables de tels mariages. Il décidait surtout que l'instigateur ou l'instigatrice d'un tel mariage (c'est-à-dire celui qui avait à y gagner, en principe le plus pauvre) pouvait être condamné à la peine de mort pour \emph{rapt}, ce qui réglait radicalement la question de l'indissolubilité du mariage. Cet édit a été en vigueur jusqu'à la Révolution, et semble avoir réglé le problème à la satisfaction des pères et des mères de familles. 
 
 \section{Les protestant, le mariage et le sexe}

 La position des protestants était très proche de celle du roi de France. Pour eux le mariage n'était pas un sacrement mais seulement un contrat entre deux personnes, par nature révocable, et du ressort des seules autorités civiles. Selon Luther (Traubüchlein, 1529) : « \emph{il faut laisser à chaque ville et à chaque pays ses us et coutumes tels qu'ils sont pratiqués \emph{[... tous ces usages]} c'est aux princes et aux magistrats qu'il appartient de les établir et de les régler} ». Le roi et les pères et mères étaient d'accord sur l'idée que le choix d'un conjoint était trop important pour être laissé à la discrétion des futurs époux. Mais le fait de dénier au mariage le poids d'un sacrement et de n'y voir qu'un contrat ne lui enlevait pas une certaine forme d'indissolubilité. Même Henri~VIII n'avait pas rompu le lien entre l'Église d'Angleterre et Rome pour divorcer, mais pour faire reconnaître par les évêques de son royaume la nullité de son premier mariage. 
 
 À partir du moment où le mariage était invalide lorsque les parents des futurs conjoints ne lui apportaient pas leur approbation, comme pour tout autre contrat, il était du devoir de ceux-ci de se soumettre à la volonté de ceux-là dans ce domaine comme dans tous les autres. Une fois leurs parents d'accords, au terme de négociations plus ou moins âpres, et après que le père de la mariée ait remis celle-ci en mains propres à son futur gendre, les époux se devaient de respecter la volonté de leurs auteurs et de rester ensemble en dépit des difficultés éventuelles. C'est pour cela que tout en reconnaissant aux époux le droit de divorcer, les protestants leur imposaient tant de conditions (en Angleterre il y fallait entre autres un acte du parlement) que leurs divorces étaient en réalité difficiles à obtenir et coûteux (800 livres au \siecle{19} en Angleterre) et donc rares : en Angleterre 184 divorces entre 1715 et 1852, pour 9 millions d'habitants environ ; au Massachusetts, état américain bien plus libéral, 143 divorces entre 1692 et 1786 pour \nombre{300000} habitants environ (un et demi par an !) 
 \footnote{Indépendamment des doctrines il serait extrêmement intéressant de comparer la réalité des pratiques chez les catholiques et chez les protestants : pays par pays et dénomination par dénomination : tâche immense !}. 

\chapter{Evolution des pratiques éducatives}


 \section{Des pères et des rois}


À partir du \siecle{4} dans l'empire romain, ce n'est plus d'abord et avant tout par la relation de pouvoir qu'il exerce sur les membres de sa maison que le père est juridiquement défini. En effet, il est soumis au devoir de \emph{piété} 
\footnote{La piété était l'affection réciproque et le respect mutuel entre les divers membres de la famille nucléaire, y compris le devoir d'assistance.} 
à l'égard de ses enfants au même titre qu'ils le sont à son égard, et autant qu'eux. L'accent se déplace sur sa responsabilité envers eux. 

La paternité est exaltée en liaison avec la paternité divine, mais la valorisation des parentalités spirituelles à travers le culte de Saint Joseph (père adoptif de Jésus selon les évangiles) affirme la prééminence de la relation éducative sur la reproduction biologique. Au même moment la maternité est très fortement idéalisée à travers le culte de la Vierge Marie. Au total c'est la famille nucléaire, le couple et ses enfants légitimes, qui est sacralisée. Cela s'exprime entre autres dans le culte de la « sainte famille » qui prend son essor au \siecle{17} : c'est en 1665 qu'est fondée la \emph{confrérie de la Sainte Famille} (la fête religieuse de la sainte famille n'a été instaurée qu'en 1893). 


Vécu comme un père par ses sujets, le roi s'identifiait à son tour à tous les pères de famille. Eux et lui étaient autant de représentants de Dieu « le Père » et ils se confortaient les uns les autres. Il en avait plus ou moins été ainsi depuis toujours (est-ce d'ailleurs si différent avec les dirigeants politiques de l'heure actuelle ?) mais à la fin du Moyen Âge le fonctionnement des familles semble avoir eu tendance à se rigidifier dans un patriarcat de plus en plus rigoureux, en même temps que les états montaient en puissance et que les doctrines absolutistes gagnaient de l'audience. 

Qu'ils soient protestants ou catholiques les philosophes, les théoriciens du droit et les chantres de l'absolutisme (Jean Bodin, Omer Talon, Bossuet, Thomas Hobbes,~etc.) soutenaient la nécessité d'un pouvoir fort, incarné par un souverain aussi absolu que possible : absolu, c'est-à-dire sans contre-pouvoirs significatifs. On peut supposer que cela découlait pour une part de l'expérience des guerres, civiles ou entre états, dans lesquelles les européens se sont laissés entraîner par les divergences entre options religieuses coexistantes. Cette expérience a révélé la violence mortelle que peuvent provoquer les identités, notamment religieuses. Elle a aussi révélé les limites de la capacité du pape et des évêques à réguler pacifiquement les conflits d'interprétation, ce qui les a délogés de leur millénaire position d'autorité et de partenaires autonomes et incontournables des pouvoirs civils. Bon gré mal gré les européens s'en sonr donc remis à "\emph{César}", quitte à s'accomoder de son despotisme ("\emph{cujus regio ejus religio}") et des injustices de la raison d'état. Tout valait mieux à leurs yeux que les désordres d'un monde où chacun serait un loup pour l'autre. 

Les ecclésiastiques eux-mêmes se sont ralliés à cette position : les protestants bien entendu, qui déniaient toute autorité particulière à l'éveque de Rome et conforté les princes dans leurs désirs de contrôler les cultes, mais aussi les catholiques. Ainsi Pierre de Bérulle (fondateur de l'Ordre de l'Oratoire) écrivait en 1623, dans un discours\footnote{\emph{Discours de l'État et des grandeurs de Jésus}.} au Roi (Louis~XIII)  :
    \enquote{\emph{un monarque est un Dieu selon le langage de l'écriture : un Dieu non par essence mais par puissance ; un Dieu non par nature mais par grâce ; un Dieu non pour toujours mais pour un temps. Un Dieu non pour le Ciel mais pour la Terre. Un Dieu non subsistant, mais dépendant de celui qui est le subsistant par soi-même ; qui étant le Dieu des Dieux, fait les rois Dieux en ressemblance, en puissance et en qualité, Dieux visibles, images du Dieu invisible}}. Jusqu'au milieu du \siecle{18} l'image de l'autorité était globalement positive, et l'exercice que les rois et les pères (et avec eux les « pères spirituels » de tous ordres) faisaient de leurs pouvoirs était regardé comme légitime et bénéfique. Dans ce cadre de pensée s'opposer au souverain comme aux pères c'était faire preuve de présomption et peut-être s'opposer à Dieu lui-même. 
    
    
    Dans quelle mesure cette vision du pouvoir et de la paternité a-t-elle rejailli sur l'image que les gens d'alors se faisaient de Dieu ? Ils prêtaient en effet à celui-ci une dureté ou même une cruauté impitoyable : exigences morales inflexibles, poids de la culpabilisation, arbitraire de la grâce, prédestination, terrorisme de la damnation, rareté des élus,~etc. Mais on peut aussi bien se demander si ce ne sont pas les thèses des théologiens de la fin du Moyen Age qui ont renforcé l'absolutisme des pères et des rois. Ils valorisaient en effet sans limites la toute-puissance divine. Selon Jean-Claude Monod\footnote{ in \emph{La querelle de la sécularisation : théologie politique et philosophie de l'histoire de hegel à Blumenberg}, Jean-Claude Monod, Paris, Vrin, 2002.} : "\emph{L'importance du nominalisme à la fin du Moyen Âge tient à ce que ce courant de pensée a mis en crise le système scolastique en voulant pousser l'homme à une capitulation sans condition dans l'acte de foi, et a retiré à la théologie toute tâche de médiation entre la connaissance et la foi. Ainsi en est-il de la souveraineté absolue de Dieu : volonté insaisissable et opaque "potentia absoluta", le Dieu du nominalisme et ses "décrets" se situent au-delà de toute tentative de compréhension par l'esprit humain. Tout ce qui est fait peut être défait, toute loi peut être suspendue, nulle garantie ne doit être attendue de Dieu, dont l'entendement est incommensurable au nôtre et dont dépend pourtant entièrement notre salut."} Comment penser la liberté des individus si Dieu connaît à l'avance tout leur avenir ? Comment peuvent-ils être responsables de leurs actes ? Comment imaginer un Dieu bon s'il n'est lié par aucune exigence de justice ? etc. Ces difficultés théoriciennes ont contribué à l'éclatement de la chrétienté médiévale.       
    
    
    S'il n'était plus possible de faire confiance à Dieu ni à ses ministres pour contenir les désordres il fallait conforter les autres autorités. Les pères se voyaient donc rappeler leur devoir de maintenir leur maison en bon ordre, dans le respect des lois civiles et religieuses. On attendait d'eux qu'ils le fassent sans faiblir, et pour y parvenir il leur était reconnu une grande part de la puissance paternelle des romains. Dans les pays de Droit écrit, comme le sud de la France, revenus avant la fin du Moyen Âge à une application stricte du Droit romain, leur puissance ne cessait qu'avec leur mort. Leur mission éducative impliquait le \emph{droit de correction}. On considérait que c'était pour eux un devoir moral et social que de corriger les enfants \emph{et les épouses} indisciplinés. Jusqu'au \siecle{18} (au moins) il était admis qu'une tendresse excessive était plus dommageable, et donc plus coupable, qu'une sévérité excessive : « {qui aime bien châtie bien} ». Montaigne nous dit qu'il fut placé de sa naissance à l'âge de quatre ans chez des bûcherons, puis mis en pension en collège à partir de six ans. Il dit s'être trouvé mieux de cette enfance loin de sa famille... parce qu'il lui semblait que son père était « trop tendre%
% [1] 
\footnote{En justifiant la décision de son père par son « excès de tendresse » Montaigne nous fournit un bel exemple de ce qu'on désigne aujourd'hui sous le nom de « fidélité » ou de « loyauté » des enfants, et des trésors de compréhension dont ils sont prêts à faire preuve face à toutes les décisions, quelles qu'elles soient, que leurs parents ont pu prendre.} 
» !

 Le roi soutenait l'autorité des époux sur leur épouse et leurs enfants, et leur prêtait main-forte s'ils le demandaient, entre autres moyens par les \emph{lettres de cachet} ordonnant sans jugement\footnote{...ancêtre des placements administratifs actuels, dont il faut reconnaître qu'ils sont mieux contrôlés qu'alors par les autorités judiciaires. Il est infiniment plus aisé aujourd'hui de mettre en question leur pertinence parce que nous n'idéalisons plus la parole des pères ni des autres autorités. Au contraire nous les tenons en suspicion.} l'incarcération de l'enfant récalcitrant, mineur ou majeur, ou de l'épouse indigne, volage ou frivole ou de mauvais caractère,~etc. S'il le jugeait nécessaire, il pouvait se substituer de sa propre initiative%
% [2] 
\footnote{De même que lorsqu'il s'agit de ses enfants un père n'attend pas d'être saisi : par définition il parle en leur nom et à leur place (et le Droit romain lui donne ce droit même quand ils sont adultes).} 
aux pères défaillants dans leur fonction de faire régner l'ordre dans leurs familles. 
 Mais il se devait aussi de contrôler qu'ils n'abusaient pas de leurs pouvoirs : leur droit de correction n'était pas un droit de vie ou de mort. Jamais les parents n'ont été autorisés à estropier leurs enfants, et l'appui donné par la force publique à leurs décisions n'était pas automatique.

 

\section{Les enseignements}

    
   Il n'est pas question ici de faire une histoire de l'enseignement, mais seulement d'en esquisser les traits qui ont directement rapport à notre sujet\footnote{\\\fsc{FURET} et \fsc{OZOUF}, \emph{Lire et écrire, l'alphabétisation des français de Calvin à Jules Ferry}, 1977.
\\Maurice \fsc{CAPUL}, \emph{Internat et internement sous l'ancien régime, contribution à l'histoire de l'éducation spéciale}, Thèse d'état, CTNERHI-PUF, Paris, 1983-1984.
\\Martine \fsc{SONNET}, {« Une fille à éduquer », in \emph{Histoire des femmes en Occident}, III, \siecles{16}{18}}, Collectif, sous la direction de Georges \fsc{DUBY} et Michelle \fsc{PERROT}, 2002, Chapitre 4, p. 131 à 168.
\\Sous la direction de Marie-Madeleine \fsc{COMPERE} et Philippe\fsc{SAVOIE}, \emph{L’établissement scolaire. Des collèges d'humanités à l'enseignement secondaire, XVIe-XXe siècles}, numéro spécial 90 de la revue \emph{Histoire de l’éducation}, mai 2001
\\ Marie-Madeleine \fsc{COMPERE}, \emph{Du collège au lycée. Généalogie de l'enseignement secondaire français (1500-1850)}
Collection Archives (n° 96), Gallimard, 1985.
\\sous la dir. de Marie-Madeleine \fsc{COMPERE} et d'André \fsc{CHERVEL}, \emph{Les Humanités classiques}, Paris : Institut national de la recherche pédagogique, 1997.
\\Marie-Madeleine \fsc{COMPERE},	\emph{L'histoire de l'éducation en Europe : essai comparatif sur la façon dont elle s'écrit} Paris : Institut national de recherche pédagogique ; Bern : P. Lang, 1995. }.

 
Il n'existait rien qui ressemblât à un enseignement public, et l'ensemble du domaine scolaire était sous le contrôle et à la charge des évêques. Des décisions royales répétées au fil des siècles (à commencer par Charlemagne) confirmaient ceux-ci dans leurs droits et aussi dans leurs obligations : ils remplissaient une mission de service public en accord avec les autorités civiles, ce qui n'excluait pas la possibilité de frictions. 

 
En dehors des monastères qui ont toujours formé leurs propres novices, le réseau des écoles s'est développé depuis le début du Moyen Âge (cf. troisième \emph{Concile de Vaison}, 529) à partir des \emph{écoles cathédrales}, d'une part vers l'enseignement élémentaire avec les \emph{petites écoles} (écoles primaires), d'autre part vers l'enseignement supérieur (incluant à cette époque ce qu'à partir du \siecle{19} on appellera l'enseignement \emph{secondaire}) avec les \emph{universités} et leurs \emph{collèges}.  Elles étaient placées sous le contrôle du Chapitre de la Cathédrale. L'un des chanoines exerçait cette responsabilité : l'\emph{écolâtre},  le \emph{chantre} ou le \emph{chancelier}... C'est lui qui jusqu'à la fin de l'ancien régime agréera tous les candidats à l'enseignement, agrément sans lequel nul n'avait le droit d'enseigner sur le territoire sous sa juridiction.

L'enseignement était assuré en majeure partie par des prêtres et des religieux(ses), même si des laïcs y collaboraient parfois. En principe il était gratuit. Le financement venait de subventions, de dons, ou de taxes affectées, ou du produit de fondations qui faisaient partie des biens ecclésiastiques.

\subsection{Enseignement primaire}

À partir des derniers siècles du Moyen Âge des \emph{petites écoles} paroissiales existaient dans toutes les villes importantes, fondées par les curés, ou par les municipalités, et ordinairement par les deux à la fois. A côté des connaissances profanes (d'abord la lecture, le calcul, souvent l'écriture, mais pas toujours) on enseignait aussi la religion, les disciplines du corps et de l'esprit, les bonnes manières de se conduire. Leur mission était en effet d'éduquer autant que d'enseigner. L'instruction, une fois entendu qu'elle se devait d'inclure la religion, était considérée comme la meilleure défense contre l'envie de mal faire. Curés ou pasteurs protestants, parents et autorités locales étaient d'accord sur ce point. D'autre part les citadins voyaient aussi en elle la meilleure arme pour trouver et pour garder un travail, ce qui avait à la fois un intérêt économique et un intérêt social. À partir de la Réforme et du Concile de Trente cette foi en l'enseignement s'est exprimée en un véritable apostolat. C'est pourquoi divers ordres enseignants ont été créés au fil des siècles :
%\begin{description}
%\item[
\siecle{12}, religieuses bernardines ;
\siecle{15}, Frères de la vie commune, Minimes ;
\siecle{16}, Jésuites, Ursulines, Bénédictins de Saint-Maur, Prêtres de la Doctrine chrétienne ;
\siecle{17}, Oratoriens, Dames de Saint-Maur, Piaristes, Religieuses de Notre-Dame, Dames de la Providence, Frères des écoles chrétiennes,~etc. (pour ne citer que les principaux).
%\end{description}

 Les petites écoles s'adressaient aux « enfants des pauvres », c'est-à-dire, dans le langage d'alors, à tous ceux dont les ressources étaient précaires, ceux qui n'avaient pas de rentes, de quelque nature qu'elles soient, et qui devaient gagner leur vie en travaillant. Il s'agissait donc de l'essentiel de la population des villes. Mais les petites écoles ne pouvaient pas toujours être complètement gratuites (pas plus que les universités). Elles étaient donc à la portée des bourgeois aisés, des commerçants et artisans, mais pas toujours à celle des autres. Lorsque les paroisses ne pouvaient pas exempter ces derniers des frais de scolarité, ce qui était le cas lorsque l'ensemble de leurs paroissiens étaient réellement pauvres, seuls de généreux donateurs et surtout des ordres religieux pouvaient les prendre en charge (cf.  les « écoles de charité »). Les religieux bénéficiaient en effet d'une sécurité financière, d'une surface sociale et d'un entregent que ne pouvaient avoir des particuliers ou des communes pauvres. Certains ordres avaient d'ailleurs explicitement été créés pour assurer gratuitement l'enseignement des indigents.

Beaucoup d'enfants n'étaient pourtant pas scolarisés, même dans les villes où l'enseignement était gratuit : leurs parents avaient trop besoin du produit de leur travail, ou bien ils ne voyaient aucune utilité à un apprentissage scolaire. Même aux yeux de ceux qui envoyaient leurs enfants à l'école il n'était pas toujours évident qu'il faille que ceux-ci soient scolarisés avec assiduité pendant plusieurs années. Beaucoup, et peut-être même la plupart, se contentaient des quelques mois ou années nécessaires pour apprendre à lire et/ou à écrire. 



 Jusqu'à la fin de l'ancien régime l'instruction des paysans (plus de 90~\% de la population) n'était pas jugée nécessaire. Étant donné le niveau des techniques alors en usage l'illettrisme n'avait pas d'incidence sur leur productivité. D'autre part leurs maîtres et seigneurs craignaient qu'une instruction même minime ne les rende « raisonneurs » et « arrogants ». En l'absence d'école les plus brillants pouvaient être distingués par leur curé (qui avait l'obligation de faire à tous le catéchisme) et recevoir de lui les bases nécessaires pour aller au collège.

Quant aux enfants de famille aisée, bourgeois et aristocrates, leur première scolarité se faisait traditionnellement auprès d'un précepteur recruté par leurs parents.  

\subsection[Enseignement « secondaire »]{Enseignement « secondaire »\footnote{Dénomination anachronique, puisqu'on ne l'appelle secondaire que depuis le \siecle{19}, mais dénomination qui nous parle.}}

Les écoles cathédrales et les écoles monastiques ont été créées pour fournir l'Église en clercs, mais elles ont toujours reçu un petit contingent d'élèves promis à la vie civile. Charlemagne leur en a fait une obligation. À partir du \siecle{10} la croissance des villes a provoqué la demande d'une instruction de niveau universitaire (à l'époque le secondaire et le supérieur n'étaient pas encore distingués). À partir du \siecle{12} les universités se sont créées comme des corporations autogérées de professeurs indépendants, avec l'appui des autorités civiles. Elles étaient sous le contrôle de l'évêque du lieu et leur personnel comme leurs étudiants bénéficiaient des avantages et exemptions attachés aux clercs (à cette époque la majorité d'entre eux étaient religieux ou prêtres ou destinés à le devenir). En cas de conflits entre l'évêque et l'Université le pape arbitrait. Dans le cadre des universités ont été créés des collèges caractérisés d'abord par la vie en internat à l'intention des étudiants pauvres, sur le modèle des écoles monastique (p. ex. le collège qui deviendra la Sorbonne). Leur mission était de permettre aux jeunes gens sans fortune d'entrer dans le clergé (puisqu'il y fallait un titre universitaire) mais peu à peu ces collèges d'universités sont devenus des lieux d'enseignement en complément des cours publics, et ils ont de ce fait été recherchés par d'autres candidats. L'enseignement « secondaire » est donc né de l'enseignement « universitaire ».

C'est en se transformant profondément que la formule du collège s'est généralisée à partir du \siecle{16}  avec une pratique de l'externat presque exclusive, une majorité d'élèves promis à la vie laïque, une réduction forte de la dispersion des âges, un classement des élèves par niveaux, etc.. De nombreux collèges ont été créés à la demande des municipalités et/ou des évêques. Les initiatives étaient très décentralisées et les créations partaient le plus souvent des besoins et des demandes locales, et d'abord des demandes des parents d'élèves potentiels. Les enseignants étaient recrutés au sein du clergé diocésain local en fonction des compétences et des titres universitaires. Au fil du temps la gestion de beaucoup de ces nouveaux collèges a été confiée par leurs fondateurs à des ordres religieux spécialisés : surtout les jésuites et les oratoriens, à côté desquels d'autres ordres comme les bénédictins ou les dominicains ont aussi joué un certain rôle. 

Jusqu'à leur expulsion (1765) les jésuites ont eu un rôle prépondérant avec leurs immenses collèges de \nombre{2000} élèves, tous externes.Ceux-ci recevaient une scolarité gratuite valorisant le latin comme une langue vivante et utilisant des méthodes d'enseignement actives. Mais tous les collèges n'étaient pas de plein exercice (avec des classes de tous les niveaux jusqu'à la classe de philosophie comprise). Beaucoup d'entre eux (les "petits collèges") se contentaient des quelques niveaux de base, quand la commune ne se bornait pas à entretenir un seul professeur de latin (une \emph{régence latine}) pour les quelques élèves concernés. Les élèves désireux d'aller jusqu'au bout du cursus secondaire allaient le terminer dans un plus gros établissement,  Beaucoup de parents se contentaient d'une scolarité réduite à quelques années de collège. Au fil des générations la scolarité secondaire a eu tendance à se décentraliser sans que pour autant le nombre d'élèves concernés n'ait augmenté. Ce dernier est resté très stable pendant très longtemps : moins de un pour cent de la population des jeunes garçons d'âge scolaire\footnote{Selon un rapport établi en 1843 par A. F. Villemain  et cité par Antoine LEON (\emph{Histoire de l'enseignement en France}, Que Sais-je ?, PUF, Paris, 1967) il existait à la veille de la Révolution 562 collèges avec \nombre{73000} élèves, dont \nombre{40000} boursiers : 178 collèges congréganistes et 384 collèges dépendant des universités ou gérés par des communes ou des particuliers. En 1812 il y avait 36 lycées et 337 collèges publics avec \nombre{44000} élèves, et \nombre{1000} autres institutions et pensionnats privés pour \nombre{27000} élèves, soit \nombre{71000} élèves au total. En 1880 il y avait environ \nombre{150000} élèves dans les lycées et collèges, et \nombre{500000} en 1940.}. 
 
 Pour les parents des collégiens l'éducation était un investissement familial, même quand ils se destinaient à devenir des clercs, ce qui était le cas d'une proportion significative jusqu'au \siecle{18}. Cela justifiait qu'ils soient improductifs pendant leurs années de scolarité. Ceux qui étaient sans ressources mais dont les dons étaient évidents pouvaient bénéficier de bourses, en particulier ceux qui se destinaient à entrer dans les ordres.




Quant aux jeunes filles de famille aisée, la clôture des couvents leur interdisait toute rencontre avec les jeunes gens de leur âge et protégeait leur « vertu » et leur réputation en attendant que leurs parents les marient. Sauf quand elles se destinaient à être religieuses la durée de leur séjour au couvent était très inférieure à celle de leurs frères dans leurs collèges : un an pour préparer leur communion solennelle, deux ou trois au plus. Les jeunes filles bien dotées étaient mariées bien plus tôt que les autres. Le savoir qui leur était dispensé était nettement moins poussé que celui que recevaient leurs frères, même si les novices bénéficiaient d'un enseignement qui en faisait des lettrées, des sœurs « de chœur », capables au minimum de chanter les offices en comprenant le latin qu'elles chantaient, et d'enseigner aux jeunes pensionnaires.

Même si la gestion d'un internat était lourde et source de tracas beaucoup de pédagogues en avaient une image positive, mais entre l'externat des collèges, gratuit ou presque, et la pension des internats l'écart des coûts était énorme%
%[5]
\footnote{Selon Martine \fsc{SONNET}, la pension d'un seul enfant, garçon ou fille, représentait presque la totalité du salaire d'un ouvrier (« une fille à éduquer », Chapitre 4 de \emph{L'Histoire des femmes en Occident}, III, \siecles{16}{18}, p. 146). C'est pourquoi en 1760 les internats parisiens n'accueillaient que 13~\% de la population scolarisée de la ville, et il semble qu'il en était de même ailleurs.}% 
. Aux familles qui ne pouvaient payer les frais d'une pension, c'est-à-dire la plupart, seul l'externat était accessible, en vivant en ville chez ses parents\footnote{... d'où la pression des municipalités pour créer un collège ou un « petit collège », et au minimum une classe de latin, une « régence latine », pour gagner quelques années de scolarité sans recourir à la pension.} ou chez un parent, ou chez un logeur peu exigeant. Dans les collèges il y avait donc ordinairement beaucoup plus d'externes que d'internes et souvent il n'y avait pas d'internat du tout : cela a été longtemps le cas des collèges jésuites, où l'enseignement était gratuit\footnote{... contrairement à ce qui se passera aux XIXème et \siecle{20}, mais les conditions matérielles et universitaires au sein desquelles s'inscrira la renaissance de l'ordre auront totalement changé entre temps.}. 

 
 

Les collégiens étaient confiés soit à l'un des ordres religieux spécialisés à partir de la Renaissance dans l'enseignement (Jésuites surtout, mais aussi Oratoriens, Dominicains,~etc.) soit à des collèges dépendant des universités, soit à des collèges dépendant des municipalités où enseignaient des clercs (ou des laïcs, mais c'était moins habituel et moins apprécié) recrutés sur place, de niveau universitaire inégal et aux motivations fluctuantes (beaucoup parmi ces derniers gagnaient leur vie en attendant d'obtenir un bénéfice ecclésiastique plus intéressant et plus lucratif que l'enseignement). 

Les collèges étaient ouverts à la ville dans les murs de laquelle ils étaient établis. Ils en formaient souvent l'un des fleurons les plus prestigieux. Ils y entretenaient une vie intellectuelle et mondaine active et d'autant plus valorisée que les autres sources de distraction étaient rares. Ils proposaient aux collégiens de s'investir dans la découverte du savoir, et celui-ci était ressenti par leurs enseignants et leurs parents comme quelque chose qui en valait la peine. Ils entraient dans une aristocratie de l'esprit. À l'époque dans toute l'Europe l'enseignement secondaire et supérieur se faisait en latin. Sans lui on savait peut-être lire, mais on n'en demeurait pas moins un \emph{illettré}%
% [7]
\footnote{C'est en latin qu'Héloïse et Abélard se sont écrit toute leur vie. C'est en latin que la République des Lettres de la Renaissance correspondait d'un bout de l'Europe à l'autre. Dans toute l'Europe les thèses de doctorat seront encore soutenues en latin durant la plus grande partie du \siecle{19}.}% 
. C'était la langue vivante, la langue de communication des communautés intellectuelles du temps. Mais depuis l'\emph{ordonnance de Villers-Cotterêts} (1539) qui imposait le français comme langue administrative du Royaume, il n'était plus possible de tenir un \emph{office} public si on ne le maîtrisait pas suffisamment. La langue française n'était encore que le patois de l'Île-de-France, domaine du roi. Partout ailleurs c'était une langue étrangère qui allait mettre très longtemps à déloger les langues locales des places et des marchés. Même si dans les collèges l'accent était mis sur le latin l'enseignement du français reçu, au moins par la bande, était donc incontournable pour entrer dans les professions libérales, la fonction publique ou le clergé. 


 Le collège était un moyen sûr de promotion individuelle et familiale, même pour les privilégiés de la fortune ou de la naissance. Pour les gens ordinaires c'était la seule voie d'accès aux emplois prestigieux et qualifiés. Le fils de famille confronté par nécessité à l'épreuve de la vie loin de ses parents continuait de dépendre d'eux. Ils payaient sa pension : il continuait de manger leur pain. Il continuait de correspondre avec eux. Il les retrouverait aux prochaines vacances s'ils ne venaient pas le voir avant. Quand la discipline et le travail intellectuel lui pesaient trop il pouvait se dire avec assez de vraisemblance qu'il était en train d'acquérir à ce prix les moyens d'atteindre un statut personnel valorisant, et qu'il s'inscrivait dans le projet de ses parents. En acceptant de se soumettre à cette exigence il pouvait espérer devenir un membre puissant et respecté de sa communauté d'origine : cela présentait l'allure d'une épreuve initiatique.
 
Si pour l'énorme majorité (99\%) des jeunes la durée de l'enseignement se réduisait à quelques années, au mieux, et à l'apprentissage des rudiments de la lecture et de l'écriture, il ne faudrait pas oublier l'apprentissage d'un métier dont bénéficiaient beaucoup de jeunes (la plupart ?) de façon informelle auprès de leur père ou d'un oncle (dont beaucoup de paysans, de pécheurs...), ou de façon contractuelle (et à titre onéreux) auprès d'un maitre d'apprentissage. 

\section{La correction paternelle}

 Tous les jeunes \emph{de famille} n'entraient pas docilement dans les projets parentaux. Certains d'entre eux entraient en conflit ouvert avec leurs parents au-delà des normes reçues (éminemment variables suivant les siècles et les lieux) : vagabonds, fugueurs, jeunes aux fréquentations suspectes, exclus pour indiscipline de collèges successifs, fauteurs de vols domestiques ou d'actes « d'inconduite sexuelle », d'insultes et de voies de faits, « libertins », c'est-à-dire jeunes rétifs à toute mesure éducative,~etc. 

 À la demande de leurs parents, ces jeunes peuvent être traités en tout comme les délinquants con\-dam\-nés. Pour les enfants difficiles des familles aisées il y avait des solutions payantes dans les sections des collèges et internats contemporains affectés à la « correction ». Ceux qui n'en avaient pas les moyens étaient internés avec les délinquants condamnés, dans les sections « de force » des hôpitaux, où s'effectuaient les peines de prison. Leurs parents payaient une pension qui tenait compte de leurs ressources. 

 À partir de la fin du \siecle{17} et de plus en plus souvent au fil du \crmieme{18}, les \emph{enfants de famille}, garçons et filles mineurs \emph{et majeurs}, qui avaient commis de vrais actes de délinquance, mais aussi ceux qui donnaient simplement du mécontentement à leurs parents par leurs fréquentations, leur mauvaise conduite, leur indocilité, leur violence aveugle ou leur absence de sens commun (« insensés »), leurs dépenses inconsidérées, ou leurs dettes de jeu, pouvaient, sur la demande de ces derniers qui exerçaient ainsi leur droit de correction, faire l'objet d'une \emph{lettre de cachet}, c'est-à-dire d'une \emph{décision administrative d'internement} dans un hôpital, une prison, une forteresse, un couvent, un collège, ou même leur déportation aux colonies. Les lettres de cachet, qui ont une origine très ancienne, bien antérieure au \siecle{17}, pouvaient aussi être accordées à l'encontre de conjoints aux comportements répréhensibles (cette mesure a beaucoup plus souvent frappé les épouses que les époux). 

 L'autorité publique n'était pas obligée d'accorder satisfaction aux demandes qui lui était faites, et restait seule juge de l'opportunité de la mesure. Elle était surtout sollicitée à Paris, notamment par les couches populaires, contrairement aux provinces où l'internement administratif était moins facile à obtenir et où les couches populaires n'y avaient guère recours. Même si au fil du temps les lettres de cachet ont fait l'objet de critiques de plus en plus virulentes et si les autorités publiques y répugnaient de plus en plus, les demandes se sont faites de plus en plus nombreuses au fil du \siecle{18}. 

 En effet les familles sollicitaient ces lettres comme une grâce : cela leur évitait la honte causée par la publicité du recours à la justice, le coût d'un procès, et aussi la publicité de la mesure d'enfermement. La réputation du jeune (ou de l'adulte) ainsi placé pouvait s'en relever plus facilement. Cela évitait le contrôle par la justice de la nature exacte des faits et de la proportionnalité des sanctions aux dommages et délits constatés, ce qui permettait à l'occasion à d'authentiques délinquants bien nés d'échapper à peu de frais aux conséquences normales de leurs actes. 

Mais cela permettait aussi aux parents abusifs d'exercer des pressions sur leurs enfants rétifs à leurs projets (ce qui expliquait les critiques de plus en plus virulentes des lettres de cachet au fil du \siecle{18}), à une époque où le consentement des parents était exigé à tout âge et pour tout mariage sous peine d'exhérédation, et où bien des entrées en religion étaient imposées par eux sans tenir compte des désirs du ou de la jeune concerné. 

\section{Enfants « adoptifs »}

 On a vu que dans le but de défendre le mariage monogame et indissoluble, l'Église a tout fait depuis l'Antiquité pour que les enfants illégitimes ne puissent pas devenir des héritiers de plein exercice. C'est pour cette raison que l'adoption était interdite, et pourtant... De l'Antiquité à la fin de l'ancien régime, on peut observer en nombre non négligeable des situations plus ou moins proches d'une adoption, où une personne, souvent un ecclésiastique (cf. \hbox{Villon}, adopté par un chanoine), souvent aussi un couple sans enfants, exerçaient la puissance paternelle sur un enfant qui n'était pas né d'eux et qu'ils élevaient jusqu'à sa majorité. C'était par exemple le cas à Lyon, où les recteurs de l'Hôtel-Dieu « adoptaient » ainsi des orphelins. 

 Ces situations d'\latin{alumnii} (adoptions simples) étaient parfois sanctionnées par des actes juridiques où les nourriciers faisaient un legs à l'enfant devant un procureur fiscal, et où ils s'engageaient à l'élever, instruire et établir matériellement à leurs frais comme leur propre enfant. Pour autant cela ne faisait pas de lui un membre de leur famille ni un héritier. 

 En principe seul un enfant légitime sans parents pouvait bénéficier de ce dispositif. Souvent, probablement le plus souvent, il était orphelin, mais des enfants légitimes pouvaient aussi être abandonnés solennellement par leurs parents, qui reconnaissaient par écrit qu'ils renonçaient à leur puissance paternelle, et à l'héritage de leur enfant s'il décédait. Pour autant ce dernier ne changeait ni de parenté ni de nom. Quand il possédait des biens, l'adoptant, tel un tuteur, les gérait jusqu'à sa majorité et il était responsable sur ses propres biens de sa gestion. 

 Les enfants abandonnés, nés de parents inconnus, ont très longtemps été exclus de ce genre de prise en charge%
% [8]
\footnote{À Lyon jusqu'en 1765. Ensuite ils y ont été traités comme les autres. Ce n'est que dans les dernières années avant la Révolution que les idées ont changé sur ce point : un signe de l'évolution qui s'est faite dans les esprits au fil du \siecle{18} et qui est apparue au grand jour à partir des années 1760-1770.}%
. Pourtant il était courant que des personnes accueillent pour l'élever un enfant abandonné à eux confié par un hôpital ou par une paroisse, qu'elles refusent d'être rémunérées pour l'élever, qu'elles le gardent jusqu'à sa majorité et qu'elles l'établissent dans la vie, ce qui en fait ressemblait beaucoup à la situation des enfants nés légitimes et juridiquement « adoptés ». Si aucun de leurs héritiers légitimes ne s'y opposait, elles faisaient de lui l'un de leurs héritiers. Mais il n'était pas question pour cet enfant d'hériter d'une fonction impliquant l'exercice public du pouvoir. 

 Derrière les mots employés il n'est pas toujours facile de reconnaître les situations réelles : adoption simple ? tutelle ? parrainage ?%
% [9]
\footnote{Cf. Jean-Pierre \fsc{GUTTON}, \emph{Histoire de l'adoption en France}, 1993.} 




\section{Organisation d'une police des pauvres}

 À la fin du Moyen Âge il était courant que les mendiants représentent 10~\% de la population%
% [1]
\footnote{José \fsc{CUBERO}, \emph{Histoire du vagabondage}, 1998, p. 8.}% 
. Les solutions en vigueur depuis la fin de l'antiquité pour traiter l'indigence et les malheurs individuels, pensées pour de petites communautés rurales où tous se connaissent%
%[2]
\footnote{José \fsc{CUBERO}, 1998, p. 42 et suivantes.}% 
, n'étaient plus à l'échelle des problèmes en un temps où les villes débordaient de leurs murailles anciennes, et où les États modernes se constituaient, imposant plus d'ordre, de rigueur et de contrôles, et rognant peu à peu les larges marges jusque là consenties entre les principes et les pratiques réelles. 

 Face à la pauvreté dès 1350 apparaissent les signes avant-coureurs d'un changement des mentalités et des pratiques. On commence à parler de « {bons} » et de « {mauvais pauvres} ». Les vagabonds ne sont plus assimilés aux pèlerins mais sont de plus en plus considérés comme des fauteurs de trouble. Les « {bons pauvres} » ou « {pauvres honteux} » ont le droit moral de mendier parce qu'il leur est impossible de travailler, et parce qu'ils restent rattachés à leur cadre villageois, à leur paroisse d'origine : enfants, infirmes, malades, vieillards... Ils ne se soustraient pas au contrôle de leur communauté. Les \emph{mauvais pauvres} sont ceux qui ont force et santé mais qui fuient le travail par paresse ou par goût de l'errance%
% [3] 
\footnote{... ou par refus de conditions de travail par trop inacceptables (mais cela c'est notre point de vue du \siecle{21}, ce n'était pas celui des décideurs d'alors).} 
loin de tous les cadres sociaux, sans aveu. On soupçonne les vagabonds de vivre dans la débauche et de commettre nombre de délits (notamment des vols). On a peur de leur nombre qui favorise la mendicité agressive et qui intimide les personnes sans défense (enfants, jeunes filles, femmes, vieillards). On les accuse de contrefaire maladies ou infirmités, d'enlever des enfants pour exciter la pitié des passants%
%[4]
\footnote{José \fsc{CUBERO}, 1998, p. 70.}% 
, et même de mutiler ces derniers pour obtenir plus d'aumônes%
%[5]
\footnote{Bronislaw \fsc{GEREMEK} fait état de procès tenus dans la région parisienne en 1449 où ont été condamnés des criminels qui avaient successivement enlevé plusieurs enfants à leurs parents, enfants auxquels ils avaient crevé les yeux et coupé bras ou jambes, pour en tirer profit en mendiant. (in \emph{Les marginaux parisiens aux \crmieme{14} et \crmieme{15} siècles}, Paris, 1976).}% 
. 

 Dès le \siecle{14} les hôpitaux refusent de plus en plus souvent les vagabonds%
% [6]
\footnote{José \fsc{CUBERO}, 1998, p. 68.}% 
, tandis que de nombreuses mesures de police tentent de les contrôler et surtout de les chasser. À l'intention des petits délinquants, des vagabonds et autres chômeurs sans ressources avouables on fait des expériences multiples de travaux « forcés », travaux d'utilité publique, ou même galères du roi%
%[7]
\footnote{En 1456 les États du Languedoc prévoient cette peine pour les vagabonds invétérés. \emph{En 1486, Charles~VIII étend cette mesure à l'ensemble du royaume.} La condamnation aux galères, résurgence de la condamnation antique aux mines, \latin{ad metallas}, se substitue alors dans la plupart des cas à la peine de mort, jusque là appliquée largement en l'absence de peines plus adaptées : \emph{avec la peine des galères... le Moyen Âge renoue avec la notion antique de l'esclavage... Seul le travail rédempteur peut éviter les galères[7] à ces mendiants valides et vagabonds qui menacent la paix}, José \fsc{CUBERO}, p. 78 idem.}% 
. Le fait que ces décisions d'expulsion aient été périodiquement reformulées montre et leur relative inefficacité, et la persistance des représentations qui les sous-tendent.

 Les cités, en expansion, sont dirigées par leurs bourgeois, commerçants, artisans, juristes et autres détenteurs d'offices. Leur expérience personnelle les porte à tenir pour synonymes les vertus familiales et « bourgeoises » : fidélité, économie, sens de l'effort, contrôle de soi et prévoyance. Pour eux un sou est un sou : contrairement aux aristocrates ils ne valorisent ni le panache, ni le faste, ni la prodigalité. Les anathèmes des religieux contre la richesse, traditionnels, ne les impressionnent plus, sauf lorsqu'ils sont à l'article de la mort, et ils sont fiers de leur fortune. Ils ont la tranquille assurance de ceux qui ont réussi. À leurs yeux les autres n'ont qu'à en faire autant, et ils se font forts de le leur enseigner : dans la plus grande partie des sociétés européennes c'est à la suite des réformes protestante et catholique que les écarts seront les plus faibles entre la morale sexuelle et conjugale officielle et les pratiques réelles. Ce sera le moment où tous les laïcs ou presque se marieront et feront des enfants. Ce sera le moment où les taux de naissances illégitimes et de conceptions pré conjugales seront au plus bas de toute l'histoire européenne : de 1650 à 1750, Normandie : 2 à 3~\% d'enfants illégitimes ; bassin parisien : 1~\% ; Languedoc et Bretagne : 1 à 2~\%. Angleterre sous Cromwell : moins de 1~\% ; en 1600 : 3,2~\%. Ces taux impliquent un haut degré de contrôle social, exercé conjointement par les familles, par les autorités civiles et par les autorités religieuses.

 Au moment où les peuples d'Europe sont en train de se cliver entre catholiques et protestants apparaissent simultanément dans tous les grands États européens des mesures très semblables pour contrôler pauvres et vagabonds. Pas un seul instant la marche vers la rationalisation du contrôle des pauvres et l'organisation de leur mise au travail, forcé si nécessaire, n'a été entravée ou modifiée par les guerres de religion, et tous les États concernés connaissent des évolutions à peu de choses près superposables : mêmes représentations, mêmes solutions, mêmes réussites et mêmes impuissances. 

 Le 22 Avril 1532 le Parlement de Paris ordonne une fois de plus que tous ceux qui dans cette ville peuvent travailler et n'ont ni emploi ni revenus avouables seront contraints à entrer dans les ateliers publics qu'il organise pour eux. Ils travailleront enchaînés deux à deux, gardés rigoureusement et employés aux travaux d'utilité publique les plus rudes. On reconnaît là les pratiques des bagnes%
% [8]
\footnote{... décrites par exemple par Philippe \fsc{HENWOOD} dans \emph{Bagnards à Brest} : « l'accouplement » des bagnards enchaînés, deux par deux, p. 40, 41 et 42,~etc.}% 
. \emph{Mais l'Ordonnance royale du 22 avril 1532 est fondamentale en ceci qu'elle ordonne le placement d'office des enfants des vagabonds arrêtés.} L'autorité parentale peut désormais être disqualifiée en l'absence de tout autre délit que le vagabondage. Ce n'était pas la première fois que des essais de ce genre étaient tentés (exemple : Reims, 1454) mais cette fois il s'agit de le faire à Paris, où se trouve la plus grande concentration de vagabonds du royaume (environ un tiers) et l'Ordonnance est signée par le roi. Elle donne aux \emph{bureaux des pauvres}, où siègent des représentants des autorités ecclésiastiques et judiciaires, une part de l'autorité de l'État. Ils exercent une fonction d'autorité sur tous les pauvres, dont ils peuvent et doivent contrôler non seulement l'incapacité de travailler, mais aussi la correction des pratiques conjugales, éducatives et religieuses. Contrôle et assistance sont désormais liés, et les assujettis ont peu de recours judiciaires possibles : ils subissent une justice d'exception. 

 L'Ordonnance royale de 1566 étend l'interdiction de la mendicité à tout le royaume de France et met les pauvres à la charge de leur paroisse d'origine (\emph{domicile de secours} : seul lieu où l'indigent a droit aux secours) ce qui leur interdit de vagabonder. Elle prévoit que les \emph{bureaux des pauvres} et autres \emph{aumônes générales} doivent si nécessaire organiser et financer des ateliers pour donner du travail aux indigents valides. Entre 1550 et 1600, des forces de police spéciales placées sous l'autorité directe des {bureaux des pauvres} (souvent appelées \emph{archers de l'Hôpital}) sont chargées de traquer la mendicité, de poursuivre hors de l'hôpital et d'arrêter les vagabonds, de récupérer les enfants placés par les bureaux des pauvres lorsqu'ils ont fugué de leur lieu de placement, et de faire régner l'ordre dans les hospices et hôpitaux. 

 Au \siecle{17} les expériences réalisées et les réflexions entretenues par les divers acteurs de l'assistance et du contrôle social confluent dans l'idée qu'il convient de regrouper en une seule administration centralisée les hôpitaux et les hospices, et d'y renfermer tous les indigents qui ne peuvent se prendre en charge seuls, en raison de leur immaturité, de leurs infirmités ou maladies, ou bien en raison de leurs comportements%
% [9]
\footnote{Sources principales :
\\Collectif sous la direction de Jean \fsc{IMBERT}, \emph{L'histoire des hôpitaux en France}, 1982.
\\Maurice \fsc{CAPUL}, \emph{Internat et internement sous l'ancien régime, contribution à l'histoire de l'éducation spéciale}, Thèse d'État, 4 tomes, Tomes 1 et 2, \emph{Les enfants placés}, Tome 3 et 4, \emph{La pédagogie des maisons d'assistance}, 1983-1984.
\\Michel \fsc{FOUCAULT}, \emph{Folie et déraison : histoire de la folie à l'âge classique}, 1961.
\\Michel \fsc{FOUCAULT}, \emph{Surveiller et punir, naissance de la prison}, 1975.
\\Bronislaw \fsc{GEREMEK}, \emph{La potence ou la pitié, l'Europe et les pauvres du Moyen Âge à nos jours}, 1987.
\\Jean \fsc{IMBERT}, \emph{Le droit hospitalier de l'ancien régime}, 1993.
\\Jacques \fsc{TENON}, \emph{Mémoires sur les hôpitaux de Paris}, 1788.}% 
. 

 Louis~XIV ordonne en 1656 la création d'un \emph{Hôpital Général} dans toutes les grandes villes du royaume, et le 14 juin 1662 l'établissement d'un hôpital général dans \emph{toutes les villes et gros bourgs}. Les directeurs, nommés à vie, reçoivent des pouvoirs administratifs et de police pour accomplir leurs missions : \emph{tout pouvoir d'autorité, de direction, d'administration, commerce, police, juridiction, corrections et châtiments sur tous les pauvres de Paris, tant en dehors qu'au-dedans de l'hôpital général \emph{[...]} sans que l'appel puisse être reçu des ordonnances qui seront par eux rendues} [...] Les administrateurs de l'hôpital jugent sans appel, à charge pour eux \emph{si lesdits pauvres méritent peine afflictive plus grande que le fouet, de le mettre es mains du juge ordinaire pour à la requête du procureur d'office leur procez estre fait et parfait}. 

 Que le mouvement de création des Hôpitaux généraux se soit poursuivi à la demande des autorités locales, et pas seulement en France, jusqu'à la fin du \siecle{18} montre que cette formule de l'institution fermée et à l'écart du monde correspondait%
% [10] 
\footnote{Maurice \fsc{CAPUL}, idem, T III, p 301.} 
bien aux conceptions de l'époque : partout en Europe on observait à cette période le même mouvement. Les \anglais{Poor Laws} anglaises ordonnaient en 1661 ou 1662 l'enfermement des pauvres dans des \anglais{Workhouses} qui sont l'exact pendant (en plus dur ?) des hôpitaux généraux. Il en était de même à Berlin,~etc.

 Les contemporains essayaient de ne pas avoir personnellement affaire à ces institutions dont le régime n'était pas fait pour être désirable. Par contre ils approuvaient leur utilisation pour mettre à l'écart les indésirables et pour éviter les catastrophes en cas de disette ou de crise de l'emploi. 

 Et pourtant il y avait des listes d'attente pour entrer à l'hôpital et il fallait souvent patienter avant d'y être admis. Une recommandation était ordinairement nécessaire (très souvent celle de son curé). Un certain nombre de personnes, pauvres mais non indigentes, acceptaient même de payer pension pour y entrer, ce qui laisse à penser que même si les conditions de vie y étaient rudes (mais ces personnes-là n'étaient pas astreintes au travail forcé) il y avait encore pire ailleurs. Pour elles l'Hôpital Général fonctionnait comme une maison de retraite (cf. les « petites maisons » dans le cadre de celui de Paris), et assumait une forme de prise en charge qui existait déjà avant sa propre création.

 Quant à ceux des mendiants et vagabonds qui troublaient l'ordre public par leurs débordements, ils ne venaient pas à l'hôpital de leur plein gré et leurs comportements le traduisaient, aussi les employés des hôpitaux généraux ne faisaient-ils aucun effort pour les garder. Au bout d'un siècle d'expériences cela conduira les Intendants du roi à créer à partir de 1768 à l'intention de cette population les \emph{dépôts de mendicité}, dépôts qui seront à l'origine des futures \emph{prisons départementales}%
% [11]
\footnote{Leur histoire est complexe et s'étend sur une bonne part du \siecle{19}. Voir entre autres : \emph{Lieux d'hospitalité : hospices, hôpital, hostellerie}, ouvrage collectif sous la direction d'Alain \fsc{MONTANDON}, P.U. Blaise Pascal, 2001.}% 
.


\section{Les enfants illégitimes}

 Alors que les grossesses légitimes n'avaient pas à être déclarées, à partir de 1556 obligation est faite par Henri~II de déclarer toutes les grossesses illégitimes, sous peine pour les filles non mariées et les femmes veuves depuis plus d'un an qui seraient enceintes d'être accusées d'infanticide si leur enfant décédait avant son baptême%
% [12] 
\footnote{... qui avait valeur officielle de déclaration de naissance puisque les curés avaient reçu peu de temps auparavant l'obligation de tenir les \emph{registres de catholicité}, ou registres de baptême, ancêtres directs des registres d'état civil.} 
(crime en principe puni de mort). Dans la déclaration devait figurer le nom du père allégué par la mère, sauf refus de celle-ci. Cette déclaration renforçait la position de la mère face à l'homme qui l'avait engrossée, et celle de son enfant, et permettait les actions en justice. Cette décision royale a été rappelée par Henri~III en 1585, et renforcée par Louis~XIV. C'est ainsi qu'en 1708 ce dernier ordonnait encore aux curés de la rappeler en chaire tous les trois mois. 

 Si elle l'a été si souvent, c'est qu'elle n'a jamais été observée de manière rigoureuse. Il semble même que la majorité des grossesses illégitimes n'aient jamais été déclarées. En dépit de la sévérité des peines annoncées les mères préféraient oublier de se signaler à l'attention des autorités lorsqu'elles pensaient pouvoir mieux défendre leurs intérêts et leur réputation (et ceux de leur enfant) par un arrangement discret avec le géniteur (ex. : mariage, pension alimentaire, octroi d'une dot,~etc.) ou par un abandon discret. Combien parmi les veuves et filles dont l'enfant est décédé sans baptême ont-elles effectivement subi les peines prévues ? Il ne semble pas que les autorités aient poursuivi ce genre d'infraction avec beaucoup d'énergie : le plus souvent les tribunaux accordaient de larges circonstances atténuantes aux « coupables » déférées devant elles%
% [13]
\footnote{Frédéric \fsc{Chauvaud}, Jacques-Guy \fsc{Petit}, Jean-Jacques \fsc{Yvorel}, \emph{Histoire de la justice de la Révolution à nos jours}, Presses universitaires de Rennes, 2007.}% 
. 

 Tout enfant, même illégitime, avait le droit d'exiger de ses auteurs des « aliments » c'est-à-dire des moyens de vivre. Un vieil adage juridique, toujours cité, disait en effet que \emph{qui fait l'enfant doit le nourrir}. Le représentant naturel de l'enfant né hors mariage est sa mère, et \emph{protéger celle-ci était aussi protéger l'enfant}. Les actions de la mère%
% [14] 
\footnote{Nommée « fille-mère », et n'ayant droit qu'au titre de « mademoiselle » jusqu'au milieu du \siecle{20}. Ce n'est pas un enfant qui pouvait faire d'elle une femme, une « dame », mais un époux en règle.} 
contre le géniteur qui l'avait délaissée étaient encouragées et soutenues, notamment par les hôpitaux, qui en cas d'abandon de l'enfant devaient en assumer seuls la charge. Elle pouvait entreprendre une \latin{actio provisionis} : demande de provisions pour frais de grossesse ou d'accouchement. Si plusieurs hommes avaient partagé à la même période son intimité ils pouvaient être solidairement responsables de l'enfant. Elle pouvait aussi tenter une \latin{actio susceptionnis partus} ou \latin{actio captionis} : action qui demandait de condamner le géniteur à assumer les frais de l'éducation de l'enfant, sur lequel il ne recevait pour autant aucune autorité. 

 \latin{L'actio dotis} prévoyait que le coupable d'un viol épouse la célibataire qu'il avait déflorée, surtout s'il l'avait engrossée. S'il refusait de l'épouser, ce qui était son droit, il devait payer une dot à la mère et financer l'entretien de l'enfant. Il en était de même si le géniteur était déjà engagé ailleurs (mariage, vœux religieux, ordination sacerdotale). Quel que soit son statut (célibataire, marié, clerc, moine, noble, roturier ou serf) il était et demeurait responsable de la vie de l'enfant et devait donc le nourrir. Même si le géniteur n'était pas père légal il restait \latin{nutritor}.

 Par contre les enfants adultérins étaient toujours traités comme des enfants abandonnés, comme s'ils n'avaient ni père, ni mère, ni \latin{nutritor}. Ils n'avaient aucun droit vis-à-vis de leurs deux géniteurs, dans la famille desquels ils n'entraient pas et auxquels ils ne pouvaient pas réclamer des aliments%
% [15]
\footnote{Ceci dit la loi n'interdisait pas à leurs auteurs de prendre librement l'initiative de pourvoir à leur éducation.}% 
. Ils étaient exclus de toute possibilité de légitimation, même par mariage, puisque leurs géniteurs ne pourraient pas se marier, même après la mort de l'époux qui faisait obstacle à leur mariage. 

 Étaient encore plus rigoureusement exclus de toute légitimation les enfants nés d'une relation incestueuse.


\section{Protection des nouveaux-nés abandonnés}

 Les grandes villes ont toujours enregistré des taux de naissances illégitimes plus élevés que les petites et les campagnes : les domestiques, femmes et hommes, y étaient nombreux, presque toujours contraints au célibat par leur emploi et par leur pauvreté, donc condamnés à abandonner les enfants nés de leur activité sexuelle (surtout dans les cas où le géniteur était l'employeur ou un membre de sa famille). C'est dans les villes que se réfugiaient aussi toutes celles qui voulaient accoucher clandestinement, et les filles chassées par leur famille ou par leur patron à cause de leur grossesse.

 Vers 1635 l'attention de Vincent de Paul (1581-1660), aumônier général des galères et spécialiste de l'assistance, possédant l'oreille du roi Louis~XIII, a été attirée par le chapitre de Notre-Dame de Paris et par les \emph{dames de l'Hôtel-Dieu}%
% [16] 
\footnote{... c'est-à-dire les religieuses qui assumaient le fonctionnement de l'hôtel-Dieu, voisin immédiat de Notre Dame et de la Couche.} 
sur la situation « effroyable » des enfants de \emph{La Couche} (maison où vivaient et mouraient les enfants abandonnés dans l'hôpital), ce qui l'a conduit à fonder en 1638 l'\emph{œuvre des enfants trouvés}. Après quelques tâtonnements il a repris les recettes éprouvées, celles qui avaient toujours marché dans le passé, même s'il l'a fait à l'échelle d'une grande capitale et avec beaucoup de détermination. Ce qu'il a apporté de véritablement nouveau, c'est qu'il a affirmé haut et fort que même s'ils étaient (peut-être) de naissance illégitime les enfants trouvés avaient le même droit de vivre que les autres enfants. Il a refusé la situation d'infanticide déguisé qui était celle des nouveaux-nés abandonnés et il a agi pour qu'ils bénéficient \emph{au même titre que les autres} enfants des soins et du lait d'une nourrice. C'est pourquoi il a créé une institution capable de mettre en présence \emph{rapidement} nourrices et nourrissons et mis au point un service de nourrices rurales efficace avec une surveillance effective%
%[17]
\footnote{Voici comment en 1788, un siècle et demi après, \fsc{TENON} raconte dans ses Mémoires l'histoire des enfants abandonnés (p. 89) : \emph{Dès l'an 1180, à l'Hôpital du Saint-Esprit à Montpellier, on avait ouvert des secours pour les enfans exposés. Les Hôpitaux des Enfans-Trouvés à Paris sont plus modernes : ils datent de 1638 : on les doit au zèle éclairé et infatigable de S. Vincent de Paul. Il faut se transporter à cette époque pour juger du mérite de leur institution.}
\emph{ En 1638, une Dame veuve, charitable, se chargeoit officieusement des enfans exposés : elle demeuroit près Saint-Landry ; sa maison fut nommée Maison de la Couche, comme on nomme aujourd'hui celle des Enfans-Trouvés, près Notre-Dame.}
\emph{La tâche qu'elle avait entreprise, excéda ses facultés ; ses servantes, fatiguées des soins qu'elles donnaient aux enfans, en firent un commerce scandaleux : elles les vendoient à des mendiantes, qui s'en servoient, afin d'exciter la charité du public ; des nourrices, dont les enfans étoient morts, en achetoient, s'en faisoient teter ; plusieurs d'entre'elles leur donnoient un lait corrompu ; on en prenoit pour en supposer dans les familles : ils ne coûtoient que vingt sols. Dès que ces désordres furent connus, on cessa de recourir à un hospice si dangereux : les enfans déposés furent transportés près Saint-Victor ; les dons de quelques personnes vertueuses ne suffisoient pas à leur subsistance ; le nombre de ces enfans devenu trop grand, on tira au sort ceux qui seroient élevés : les autres étoient abandonnés.}
\emph{Dans ces circonstances, St. Vincent de Paul, en 1640, convoqua une assemblée de Dames, distinguées par leur naissance, leur piété : il en obtint des secours. Le choix du sort des enfans à élever, fut aboli, la vie conservée à tous. Louis~XIII entra dans ces vues charitables : il accorda le château de Bicêtre pour les retirer ; on se persuada que la vivacité de l'air s'opposait à leur conservation : ils furent ramenés dans le fauxbourg Saint-Lazare, où ils demeurèrent sous les yeux de Mlle de Marillac, veuve Le Gras, jusqu'en 1670, époque de leur translation dans la Maison de la Couche}.}% 
.

 Ceci dit la majorité de ses contemporains n'a été que fort peu ébranlée dans ses certitudes par son exemple et ses arguments : Maître \fsc{Ducros}, cité plus haut, qui écrivait en 1659, soit plus de vingt ans après la fondation de l'œuvre des enfants trouvés, n'avait rien entendu. Jusqu'au milieu du \siecle{18} (au moins) ceux qui mettaient à l'écart les enfants illégitimes ou supposés tels, et qui réservaient le meilleur des ressources de l'assistance aux nouveaux-nés légitimes pensaient faire pour le mieux. 


\section{Les « enfants de l'hôpital »}

 En ce qui concerne les enfants les plus jeunes la croyance en la vertu éducatrice et rééducatrice de l'internat est à cette époque à son apogée. Les décideurs n'ont pas encore compris l'importance des relations interpersonnelle (corps à corps et cœur à cœur) dans la construction d'une personnalité d'enfant. Ils n'ont pas plus compris combien est déterminante, pour l'investissement de quelque enseignement que ce soit, la différence entre le placement en internat scolaire choisi par les parents, et l'internement d'office ordonné contre leur gré par une instance administrative ou judiciaire. Ils n'ont pas compris non plus la différence qui existe entre la prise en charge des enfants sans famille (orphelins ou abandonnés) qui ni les uns ni les autres n'ont plus de parents, et celle des enfants qui connaissent leurs parents mais à qui on prétend interdire de s'identifier à eux. 


\subsection{Enfants trouvés et abandonnés}

Les enfants abandonnés pris en charge par les institutions d'assistance pouvaient avoir été déposés dans un lieu public ou dans le « tour » d'un hôpital, ou confiés par leur père ou leur mère, ou volontairement « perdus » par eux dans un lieu inconnu%
%[18]
\footnote{L'histoire du \emph{Petit Poucet}, racontée par \fsc{Perrault} dans les \emph{Contes de ma mère l'oye} (1697) a parfois correspondu à une réalité, pour des enfants très jeunes incapables de dire de quelle commune ils venaient ni comment s'appelaient leurs parents.}% 
. Beaucoup de nouveaux-nés étaient abandonnés par leurs mères dans les services d'accouchement des hôpitaux, que seules fréquentaient les indigentes qui ne pouvaient accoucher à leur propre domicile ni chez une sage-femme. D'autres tout-petits n'étaient pas abandonnés à proprement parler. Il s'agissait par exemple d'enfants dont les pères ou/et mères étaient incarcérés dans les « \emph{lieux de force} » (dont la prison pour femmes de \emph{La Force} qui faisait partie de l'hôpital de la Salpêtrière) pour vagabondage, prostitution ou autres actes de délinquance, et qui ne pouvaient donc pour un temps s'occuper d'eux. Dès que l'incarcération durait un temps significatif (un an ?) la restauration des droits parentaux devenait impossible. 

 À part ce cas les enfants abandonnés pouvaient être repris par leurs parents. Il fallait évidemment que leur abandon n'ait pas été anonyme pour que ce retour soit possible. En fait ces \emph{retours en famille}étaient rares, les causes de l'abandon, et d'abord la misère, persistant dans la plupart des cas.
 De nombreux enfants entraient à l'Hôpital bien après leur petite enfance : « \emph{... dans la généralité de Lyon, le plus grand nombre d'enfants présentés aux hôpitaux par leurs parents ont une dizaine d'années…} » À cet âge la plupart des enfants « de famille » travaillaient déjà. Ceux qui étaient confiés à l'hôpital étaient donc souvent ceux qui étaient jugés inaptes au travail. Certains d'entre eux se présentaient d'eux-mêmes à l'hôpital. 

 Au-dessous de 4 à 5 ans les enfants de l'Hôpital sont placés en nourrice. Une fois finie la petite enfance, le placement en institution est préféré. Les administrateurs croient que leurs Hôpitaux offrent des possibilités d'éducation nettement supérieures à une famille nourricière, pour des raisons variées, dont la modestie du niveau culturel des nourrices et de leur maris, qui sont le plus souvent paysans ou ouvriers agricoles, et parce que l'hôpital fournit une scolarité qu'on ne trouve pas à la campagne. Ils estiment aussi que les possibilités de trouver un emploi sont plus grandes en ville. Peut-être ne se sentent-ils pas non plus le droit de déraciner pour toujours des jeunes citadins en les laissant vivre à la campagne, surtout s'ils ont de la parenté dans la ville ? Mais il faut aussi tenir compte du fait que le prix de journée de l'hôpital est à l'époque nettement inférieur au salaire d'une nourrice.

 Tous les enfants de 6 ans et plus, non placés chez un maître artisan ou un nourricier, vivent dans les murs de l'hôpital. Même quand ils ont une famille, les enfants placés en sont plus ou moins radicalement coupés, \emph{même quand leurs parents sont placés dans le même établissement}. Les clôtures internes de l'hôpital sont aussi hautes que son mur d'enceinte%
% [19]
\footnote{Il n'est pour en être persuadé que de visiter la chapelle de l'Hôpital de La Salpêtrière.} 
. Pour nombre d'enfants cette coupure est définitive. 

 En dépit d'un souci éducatif certain%
% [20] 
\footnote{... manifesté à Paris par 5 heures 30 d'enseignement par jour, durant six jours par semaine, ce qui n'a rien à envier aux écoles primaires d'aujourd'hui... mais aussi un nombre d'élèves très élevé pour un seul maitre.} 
l'encadrement humain des jeunes placés est extrêmement réduit (d'où la modestie du prix de journée), ce qui contraint les relations entre les jeunes et les adultes à être formelles, distantes et souvent impersonnelles%
%[21]
\footnote{Selon l'expression de Maurice \fsc{CAPUL} : \emph{pour les pauvres, les moyens de la pédagogie étaient pauvres}.}% 
. Contrairement aux jeunes « de famille » inscrits par leurs parents dans les collèges contemporains, il ne s'agit pas d'intégrer ces jeunes à la « grande » culture ni de leur donner les moyens de penser plus ou moins librement : il s'agit seulement, comme dans les petites écoles, de leur donner les rudiments de la lecture et de l'écriture, et d'en faire de bons pauvres.

\subsection{« Correctionnaires »}

Les mineurs « correctionnaires » sont les jeunes qu'il faut « corriger », ceux dont les comportements font problème, c'est-à-dire les délinquants, rebelles et opposants : mineurs condamnés par décision de justice, faux saulniers de moins de 14 ans, vagabonds, mendiants, prostitué(e)s, « enfants de bohême ». Les enfants au dessus de 6 ans sont soumis aux mêmes règles de droit que les adultes. Dès l'âge de 8 ou 10 ans la peine de mort peut leur être appliquée si une « malignité » exceptionnelle justifie de les exclure du bénéfice de l'excuse de minorité. Les jeunes délinquants sont ordinairement condamnés à un temps d'incarcération déterminé : de quelques mois à 20 ans et plus. Mais ils peuvent aussi être enfermés pour une durée indéterminée : aussi longtemps que l'administration estimera qu'ils ne seront pas suffisamment amendés, jusqu'à leurs 25 ans et plus. Les jeunes garçons condamnés aux galères pour des délits commis sans l'excuse de minorité ne peuvent y être envoyés avant leurs 15 ou 16 ans. Ils attendent donc à l'hôpital d'avoir atteint l'âge d'aller au bagne, soumis au régime des autres correctionnaires, mais le temps qu'ils passent à l'hôpital ne compte pas comme temps d'exécution de la peine

\subsection{« Religionnaires »}

À partir de la \emph{Révocation de l'Édit de Nantes} (1685) ce terme désigne les enfants des protestants rebelles à la conversion au catholicisme qu'on exige d'eux%
%[22]
\footnote{L'Angleterre avait précédé la France dans la persécution des dissidents religieux et leur exclusion de toutes les charges et fonctions officielles. C'était l'application stricte du principe \latin{cujus regio, cujus religio}, « {un roi, une foi, une loi} ». Il faudra attendre le \siecle{18} pour que la tolérance apparaisse comme une vertu et non comme une faiblesse.}% 
. La légitimité des mariages des protestants n'est plus reconnue, ce qui fait de leurs enfants des bâtards incapables d'hériter. Ils se voient retirer leurs droits parentaux. Pour cette raison dès l'âge de sept ans leurs enfants leur sont enlevés. 

 À partir de cette date il est demandé aux hôpitaux généraux d'enfermer et rééduquer les membres de la « \emph{religion prétendue réformée} » (RPR) si aucune autre solution n'est possible. Les enfants de ceux qui ne peuvent payer sont placés en hôpital général, avec les correctionnaires. Les autres sont placés aux frais de leurs parents dans une section de correction d'un collège (catholique comme tous les collèges du royaume à partir de la Révocation), avec les enfants indisciplinés ou récalcitrants des mêmes milieux sociaux qu'eux. Ils y sont soumis à une pression morale ouverte ou insidieuse, brutale ou habile, pour les pousser à abjurer la religion de leurs parents et à se convertir au catholicisme. Leur sortie de l'hôpital ou du collège dépend en grande partie de leur « conversion ». 

 Selon Maurice \fsc{CAPUL}, cette politique a été poursuivie activement de 1685 au milieu du \siecle{18}, en dépit du fait qu'elle ne donnait que des résultats insatisfaisants : selon les observateurs du temps elle produisait des adultes peu consistants, qui ne savaient plus à quoi ils croyaient, ou des sceptiques qui ne croyaient plus à rien. D'autre part elle jetait la discorde au sein des familles et la brouille entre les parents et les enfants. Elle va se déliter peu à peu après le milieu du \siecle{18}, mais ce n'est qu'en 1787 que \emph{l'Édit de Versailles} y met un terme en créant un état-civil laïque, qui rend aux enfants protestants leur légitimité. Le roi prend officiellement acte de la tolérance dont le culte protestant avait fini par bénéficier à cette date%
% [23] 
\footnote{Depuis l'affaire Calas (condamné par le parlement de Toulouse à être roué, exécuté en 1762) et l'intervention de Voltaire (qui avait entrainé sa réhabilitation en 1765) la répression du protestantisme s'était adoucie : dans l'opinion publique la légitimité avait changé de camp.} 
dans la réalité quotidienne. Les dispositions de cet édit concernent aussi les français de confession juive.



