%E1 Le tournant Constantinien 
%E2 Constantin et le droit des personnes 
%F1 entrée en scène des barbares 
%F2 Les sociétés du Bas-Empire et du Haut Moyen Âge 
%F3 l'esclavage chez les chrétiens de l'antiquité tardive et du haut moyen-âge 
%F4 clercs et religieux 
%F5 Le "mariage constantinien" 
%F6 Familles de chair 
%F7 Familles spirituelles 
%G1 Les familles de l'Ancien Régime entre autorités civiles et religieuses 
%G2 Les devoirs des pères de l'Ancien Régime 
%G3 Création d'une police des familles (XIVème-XVIIIème siècles)


% Le 03.03.2015 :
% Antiquité
% Moyen Âge
% Au 24.02.2015 :
% ~etc.
% Moyen-Âge
% ~\%


\chapter{Le tournant constantinien}

En 313  Constantin (272-337) rompt avec la politique romaine traditionnelle, qui est aussi celle de ses rivaux dans la course au pouvoir suprême. Il ordonne l'arrêt des persécutions en cours contre les chrétiens dans la partie d'empire qui lui a été confiée \footnote{Sources :
\\Danièle \fsc{ALEXANDRE-BIDON} et Didier \fsc{LETT}, \emph{Les enfants au Moyen Âge, \siecles{5}{15}}, 1997.
\\Didier \fsc{LETT}, \emph{Famille et parenté dans l'Occident médiéval, \siecles{5}{15}}, 2000.
\\Peter \fsc{BROWN}, \emph{Genèse de l'Antiquité tardive}, 1978.
\\Peter \fsc{BROWN}, \emph{Pouvoir et persuasion dans l'Antiquité tardive, vers un empire chrétien}, 1992.
\\Peter \fsc{BROWN}, \emph{Le renoncement à la chair, virginité, célibat et continence dans le christianisme primitif}, 2002.
\\Jean-Michel \fsc{CARRIE}, Aline \fsc{ROUSSELLE}, \emph{L'Empire romain en mutation, des Sévères à Constantin}, 192-337, 1999.
\\Christian \fsc{DELACAMPAGNE}, \emph{Une histoire de l'esclavage, de l'Antiquité à nos jours}, 2002.
\\Jean \fsc{DURLIAT}, \emph{De l'Antiquité au Moyen Âge, l'Occident de 313 à 800}, 2002.
\\Alexandre \fsc{FAIVRE}, \emph{Naissance d'une hiérarchie, les premières étapes du cursus clérical}, 1977.
\\Bertrand \fsc{LANCON}, \emph{Le monde romain tardif, \siecles{3}{7} ap. J.C.}, 1992.
\\Henri-Irénée \fsc{MARROU}, \emph{Décadence romaine ou Antiquité tardive ? \siecles{3}{6}}, 1977.
\\Henri-Irénée \fsc{MARROU}, \emph{L'Église de l'Antiquité tardive}, 303-604, 1963.
\\Paul \fsc{PETIT}, \emph{Histoire générale de l'Empire romain, 3, le Bas-Empire (284-395)}, 1974.
\\Aline \fsc{ROUSSELLE}, \emph{La contamination spirituelle, science, droit et religion dans l'Antiquité}, 1998.},
et il inscrit leur religion dans la liste des cultes reconnus. 

Pourquoi l'empereur de Rome s'allie-t-il avec les chrétiens ? 
C'est sûrement pour étayer son pouvoir \footnote{cf. son rêve de la veille de la bataille décisive du Pont de Milvius, où il aurait entendu que : \latin{in hoc signo, vinces} (c'est à dire "\emph{sous cet étendard tu vaincras}"), à la suite duquel il aurait décidé de faire combattre ses soldats sous les enseignes du Christ.} mais c'est peut-être aussi par conviction personnelle. Les historiens en débattent toujours\footnote{Pour \fsc{VEYNE} c'est d'abord par conviction personnelle : \emph{Quand l'empire romain est devenu chrétien}, 2007.}. Il est de fait qu'il n'a été baptisé que peu de temps avant sa mort, ce qui ne plaide pas \emph{a priori} pour une foi ardente, mais cette façon de faire était fréquente chez les chrétiens de son temps. Elle lui laissait les mains plus libres pour se livrer aux basses oeuvres impliquées parfois par la conquête et l'exercice du pouvoir, ce dont il ne s'est pas privé. D'autre part elle lui permettait d'assumer la fonction de \emph{Grand Pontife} au même titre que les empereurs précédents\footnote{...qui depuis Jules César s'étaient attribué cette fonction. Ce dignitaire était le prêtre le plus important de Rome. Il avait le pas sur tous les desservants de tous les cultes autorisés, et il supervisait le bon fonctionnement de ceux-ci.}. Il a donc été le Grand Pontife jusqu'à sa mort. Ses successeurs ne se dessaisiront totalement de ce titre que vers 380. Est-ce en tant que Grand Pontife ou en tant qu'Empereur que Constantin a placé le culte des chrétiens au rang de religion officielle ? 

A-t-il été convaincu par une supériorité (dont le domaine serait à définir) du christianisme et/ou par le nombre ou l'influence des chrétiens ? C'est l'interprétation \emph{providentielle} proposée par les auteurs chrétiens depuis \emph{Eusèbe de Césarée} (mort en 339) et sa "\emph{Vie de Constantin}" jusqu'à nos jours. Pour Yvon Thébert\footnote{Yvon THEBERT, "A propos du "triomphe du christianisme", in \emph{Dialogues d'histoire ancienne}, vol 14, 1988} il faut se déprendre de cette interprétation qui à ses yeux n'est que propagande et se demander sans inhibitions comment le christianisme a pu être privilégié par un postulant à l'empire qui en matière de \emph{religions à mystère} ou de \emph{religions du salut} n'avait que l'embarras du choix ? Pourquoi s'est-il appuyé sur une secte obscure, lentement et difficilement dégagée de sa matrice juive en dépit de sa prétention à l'universalité, dont le fondateur divin et humain à la fois (ce qui était banal) a été condamné à une mort infamante par un magistrat \emph{romain}, professant un monothéisme exclusif mais dont le dieu est composé de trois entités dont les natures sont un tel tourment pour la raison que depuis l'apparition de la secte elles provoquent chez les adeptes une insécurité dogmatique et des clivages incessants ?  Comment le choix de Constantin a-t-il pu être entériné et même redoublé par tous ses successeurs à l'exception de son neveu Julien dont il avait fait assassiner le père ?  

Il faut se demander si ce n'est pas le choix de l'empereur qui a été l'élément déterminant dans le triomphe du christianisme. Il faut même oser voir que c'est lui qui a organisé cette religion en fonction de ses propres intérêts et cela jusque dans les dogmes (cf. le concile de Nicée, centré sur la nature du christ, convoqué et présidé par Constantin). En conséquence "\emph{ce n'est pas le christianisme qui submerge le pouvoir, c'est le pouvoir qui utilise le christianisme et qui, pour ce faire, va le modeler de façon décisive en fonction de ses besoins}". Et ce qui intéresse Constantin, c'est que "\emph{Fondamentalement, cette église se distingue des autres par l'organisation rigide du pouvoir : tendances constantes au monarchisme épiscopal, embryons déjà vigoureux de la théorie de la suprématie romaine.}" Ce qui l'intéresse c'est la capacité de l'Eglise à encadrer la masse de ses fidèles dans le respect du pouvoir civil et à exténuer les désirs de transformation sociale, et c'est celle de l'eveque de Rome à ramener les déviants idéologiques dans la voie moyenne. Au terme de sa réflexion Thébert conclut (en marxiste orthodoxe) que : "\emph{Le catholicisme n'a pas conquis la société du Bas-Empire, il a été secrété par elle : il en est le produit, tout comme la morale ou l'art de cette époque}"\footnote{En réduisant l'Eglise du Bas-Empire à être seulement le produit conjoncturel d'une politique qui la dépasse cette interprétation est de nature à relativiser les reproches d'intolérance persécutrice que l'on lui fait souvent (Thébert le premier) : en exerçant une police de la pensée, elle n'aurait fait que ce que lui demandait l'empereur, sans le \emph{bras séculier} duquel ses anathèmes seraient restés sans force.}. Reste à comprendre par quel miracle un tel bricolage institutionnel a pu "prendre et coaguler" aussi fermement\footnote{Comme le constate Patrick BOUCHERON  dans "Le Génie de l'athéisme" (in \emph{Afrique et Histoire}, 2005/1, vol. 3) cet article iconoclaste et stimulant ne représente pas la position moyenne des spécialistes du Bas-Empire ni du christianisme ancien, et il a été fort peu commenté.}.

Pour ce qui concerne notre sujet, à savoir l'histoire de la génération, nous entrons à partir de Constantin dans une ère (la "chrétienté", pour faire court) qui ne s'achèvera au plus tot qu'avec les lumières (à bien des aspects et en bien des endroits elle a meme continué sur sa lancée jusqu'au XXème siècle et on vient juste d'en sortir). Ce chapitre sera donc très volumineux. Il se bornera pourtant à tenter de décrire les logiques principales en présence et leurs articulations, en négligeant d'innombrables détails dignes d'intéret.
\section{l'alliance du trône et de l'autel}
Quoi qu'il en soit de ses motivations Constantin a orienté l'histoire à venir de manière irréversible et conclu une alliance du trône et de l'autel qui en Europe comme à Byzance a perduré sans changements significatifs pendant plus d'un millénaire. À partir de sa victoire sur ses rivaux il a traité les évêques chrétiens comme les clergés des autres religions reconnues, c'est-à-dire comme un corps de magistrats religieux associés au pouvoir civil. Ils ont accepté avec gratitude leur nouveau statut comme un développement providentiel de \emph{l'histoire du salut}, comme la reconnaissance de leurs droits légitimes. Pour eux comme pour tous leurs contemporains, il était inimaginable que l'Etat se désintéresse des dieux et les relègue dans la sphère privée, au risque que l'un d'entre eux ne se venge cruellement de cette négligence discourtoise. 
 
 L'Église bénéficiait désormais de tous les droits des cultes reconnus par l'État : droit de recevoir dons et legs, inaliénabilité des biens fonciers, exemptions d'impôts,~etc, et Constantin l'a rapidement favorisée en la dotant de bâtiments et de propriétés terriennes\footnote{Selon les thèses (controversées mais séduisantes) de \fsc{DURLIAT} (2002), il pouvait s'agir de la \emph{propriété éminente} de \latin{villae}, qu'il faut distinguer de la propriété ordinaire, de la propriété \emph{utile} (c'est-à-dire le droit de vendre, et d'acheter la terre, le droit de la mettre en valeur et de jouir des fruits du sol). Selon lui la \latin{villa} était à cette époque \emph{une circonscription fiscale} peuplée notamment de cultivateurs (nommés \emph{colons} en langage administratif) qui pouvaient être propriétaires de leur exploitation agricole, ou simples tenanciers. À cette époque, le terme villa pouvait aussi désigner un château et ses terres, ce qui crée des confusions. Celui qui possédait la propriété \emph{éminente} d'une villa fiscale, nommé le \latin{dominus}, était chargé d'y faire la collecte des impôts : le \emph{cens}, les droits de mutation,~etc. Ceux-ci correspondaient selon les estimations de \fsc{DURLIAT} à environ 20~\% de l'ensemble des revenus des contribuables, fournis en monnaie, en nature, ou en corvées, et dont une partie revenait au \latin{dominus} pour prix de ses services. Il était en quelque sorte le percepteur de cette \latin{villa}. C'était une charge lucrative et honorable tout à la fois. Les grandes fortunes de l'empire romain reposaient sur ces propriétés très particulières, d'ailleurs non exclusives de la propriété utile des mêmes domaines. On pouvait recevoir ces charges de l'empereur, ou les transmettre par héritage, les vendre et les acheter (comme la \emph{ferme des impôts} sous l'Ancien Régime). Le \latin{dominus} assurait le lien entre les colons concernés et les administrations de l'État. Il était en quelque sorte le Seigneur de cet espace. Par certains aspects cela préfigurait les \emph{seigneuries} apparues à partir du milieu du Moyen Âge.}% 
. Pour ne pas les obliger à sacrifier aux dieux civiques, les clercs étaient exemptés des charges curiales, c'est-à-dire de l'obligation pour les plus fortunés de participer à la curie, au conseil municipal de leur cité
\footnote{... ce qui était à la fois un honneur et un impôt, puisqu'ils devaient financer de leurs propres deniers certaines des dépenses de celle-ci. Cf. P. \fsc{VEYNE}, 1976.}. L'exemption des charges curiales (dont bénéficiaient aussi un certain nombre de prêtres des cultes civiques) leur permettait de consacrer leur temps et leur fortune à d'autres formes d'\emph{évergésies} que celles que devaient traditionnellement pratiquer les curiales. L'empereur comptait sur eux pour investir dans l'assistance aux pauvres, la construction d'hôpitaux,~etc. 

Les évêques n'ont jamais eu tout pouvoir sur Constantin ni sur ses successeurs. C'était ordinairement le contraire. L'Église se définissait elle-même comme un partenaire qui n'avait pas vocation à l'exercice du pouvoir temporel. Quand un prêtre, un évêque ou un pape prétendait néanmoins gouverner les affaires courantes, ce qui n'a pas manqué de se produire, les autorités civiles le renvoyaient à l'Évangile : \emph{rendez à César ce qui est à César, et à Dieu ce qui est à Dieu} (Mt 22, 17-21). Mais en contrepartie de ce renoncement les clercs recevaient l'exclusivité sur le culte, sur les dogmes qui servaient de cadre de pensée à tous, sur les mœurs, notamment familiales, et progressivement sur l'enseignement. 

 Il n'existait pas à Rome de pouvoir judiciaire indépendant des autres administrations impériales. l'empereur était la source du droit et tous les jugements étaient faits en son nom. Comme tous les autres magistrats de l'Empire, dont les prêtres des autres religions reconnues, les évêques ont donc reçu délégation pour régler les litiges qu'on leur soumettait et qui ressortaient de leurs compétences. C'était reconnaître officiellement le rôle d'arbitrage qu'ils exerçaient déjà officieusement, notamment en matière familiale et doctrinale. Comme les prêtres des autres cultes reconnus jusque là ils ont reçu le pouvoir d'enregistrer les affranchissements d'esclaves : leurs actes écrits avaient désormais force de preuves. Leurs arbitrages devaient d'autant plus être respectés qu'ils avaient la faveur de l'empereur, comme en atteste le fait qu'ils ont reçu le pouvoir de juger seuls des fautes des clercs chrétiens (hors affaires criminelles).
 
 

D'une certaine façon avec Constantin peu de choses changeaient dans la nature des liens entre l'empereur-grand-pontife et sa religion préférée. Ainsi c'est lui qui en 325 a convoqué dans son palais de Nicée le premier concile "œcuménique"
\footnote{...qui devait rassembler tous les évêques vivants} 
de l'Église, parce qu'il voulait imposer un accord dogmatique entre les ariens et les non-ariens et c'est lui qui l'a présidé. A l'issue de ce concile c'est lui qui a donné force de lois à ses décisions en les contresignant. Lorsque l'un des successeurs de Constantin aura pris fait et cause pour l'hérésie arienne ce seront les catholiques qui subiront la défaveur du prince. Il s'agissait d'un échange de légitimités entre empereur et évêques, d'un étayage réciproque, conforme aux traditions antiques de confusion de la religion et de la cité. Dans la compréhension de l'œuvre législative réalisée sous le règne de Constantin il faut donc tenir compte non seulement de l'influence de l'Église, qui est souvent évidente, mais aussi de sa marque personnelle et de l'évolution générale des attitudes romaines face aux mœurs, au couple et à la famille. Tant que l'alliance des gouvernants et des églises perdurera la question de l'influence réciproque sera posée. Il est parfois très difficile de dire aujourd'hui lequel des deux partenaires était à l'origine d'un fait, d'une décision, d'une règle, d'une institution, en un mot d'un des symptômes de cette alliance. 

Selon son propre mot Constantin se considérait comme « l'évêque du dehors », l'évêque (\emph{episcopos} : le surveillant, l'inspecteur) des non chrétiens. Il estimait de son devoir de conduire l'ensemble de ses sujets, chrétiens ou non, à la vérité (telle qu'il la concevait), et à défaut de les convertir tous, ce qui était du ressort des évêques, il entendait au moins mettre les lois en accord avec les principes chrétiens (avec ceux du moins qu'il approuvait) et créer ainsi un milieu de vie qui favoriserait les conversions. Il ne s'est pas borné pas à favoriser le culte chrétien, il a aussi initié la séparation de l'État romain et des religions païennes. Celle-ci va s'effectuer par étapes entre 325 et 391. En 382 un de ses successeurs supprime les privilèges des vestales et des prêtres païens et interdit aux cités de financer les temples païens (nomination des prêtres, entretien des bâtiments, fourniture des offrandes pour les sacrifices, achat d'encens,~etc.). Progressivement les biens et les terres de ceux-ci sont confisqués au profit du trésor public. Un des petit-fils de Constantin décrète en 391 que seule la religion chrétienne est désormais autorisée. Tous les sujets de l'empire sont à partir de cette date fermement invités à se faire baptiser et à professer publiquement la foi que celui-ci leur désigne : la doctrine définie par l'évêque de Rome \footnote{... ce qui en soi est nouveau : il n'y avait pas de profession de foi consciente et articulée dans les religions antiques, du moins pas avant le triomphe du christianisme.}. Tous les autres cultes sont interdits. Seule la dissidence juive continuera d'être tolérée, comme une espèce de « butte témoin » de l'ancienne alliance, mais il lui sera interdit de faire des prosélytes, surtout parmi les chrétiens\footnote{La conversion des chrétiens ou des païens au judaïsme était interdite, de même que les mariages mixtes. Si un juif faisait circoncire son esclave chrétien cela entraînait \latin{ipso facto} l'affranchissement de ce dernier, s'il le réclamait. Dès 313 Constantin condamnait à mort les juifs qui lapidaient ceux de leurs religionnaires qui se convertissaient au christianisme.}. 

 Si Constantin n'avait pas récusé le titre de \latin{Pontifex Maximus}, ses successeurs n'ont plus porté ce titre, sauf Julien qui a tenté de remettre la religion traditionnelle à l'honneur. Malgré de fortes tentations et une sacralisation de leur fonction ambiguë, aucun empereur, aucun roi "très chrétien" n'osera
\footnote{... jusqu'à Henri~VIII, roi d'Angleterre.} 
se proclamer dignitaire de l'Église. C'est l'évêque de Rome qui héritera du titre de pontife. Cela n'empêchait pas les empereurs et les rois de se considérer comme les partenaires permanents et obligés des évêques, comme les soutiens de leur pouvoir et le bras armé qui défendait leurs enseignements. Soucieux d'ordre et d'unité ils seront étroitement associés à l'Église dans le choix des évêques, les arbitrages théologiques et disciplinaires, la convocation des conciles, la définition de la discipline ecclésiastique,~etc. Jusqu'au milieu du Moyen Âge leurs édits, décrets et codes concerneront le fonctionnement interne des églises et la discipline ecclésiastique au même titre que celui des autres corps de la société, et les décisions des conciles n'auront force exécutoire que pour autant qu'ils les auront approuvées et appuyées de leur autorité (cela est resté vrai en France au moins jusqu'à la révolution). Le sacre (et surtout l'onction) des empereurs et des rois sera l'objet d'une valorisation qui en fera presque un sacrement analogique à l'ordination des clercs. À Constantinople l'Église et l'Empereur vont être inséparables pendant un millénaire, étayés l'un par l'autre et se chargeant à eux deux de régir l'empire. 

Certes, en première analyse, l'évangélisation des populations non catholiques (les "barbares", les juifs et les chrétiens "hérétiques") a été l'œuvre de l'Église. Mais elle a été presque toujours été appuyée dans cette tâche par les pouvoirs publics. Il s'agissait d'abord et avant tout de convertir les rois et les puissants et les rapports de force militaires facilitaient les choses. Le reste suivait presque mécaniquement. En dépit de la doctrine ecclésiale qui voulait que les candidats au baptême agissent librement et de leur propre initiative, ils l'ont parfois fait sous la contrainte, et tous les évêques n'ont pas protesté contre les pressions subies par les « païens », les « hérétiques » (ariens en particulier) et les juifs (notamment en Espagne)
\footnote{Sur ce sujet on peut se référer notamment au livre de Bruno \fsc{DUMEZIL}, \emph{Les racines chrétiennes de l'Europe, Conversion et liberté dans les royaumes barbares \siecles{5}{8}}, Fayard, 2005, Paris.}. 

 Les décrets et rescrits promulgués par Constantin et ses successeurs n'ont pas fait disparaître les pratiques antérieures d'un trait de plume, ni transformé toutes les familles de l'empire romain, plus ou moins bien baptisées, ou pas baptisées du tout, en autant de \emph{saintes familles}\footnote{Sources : Jean-Pierre \fsc{LEGUAY}, \emph{L'Europe des états barbares, \siecles{5}{8}}, Belin, Paris, 2002. Jean-Pierre \fsc{POLY}, \emph{Le chemin des amours barbares, Genèse médiévale de la sexualité européenne}, 2003. Stéphane \fsc{LEBECQ}, \emph{Les origines franques, \siecles{5}{9}, Nouvelle histoire de la France Médiévale}, 1990.}. Les non chrétiens, encore majoritaires en 313, ont certes fait de la résistance face aux décrets de Constantin, mais les baptisés aussi. L'ordre public possède ses propres logiques : la société a toujours toléré ou soutenu bien des choses que les évêques réprouvaient. Inversement l'Église a défendu avec une persévérance millénaire des positions que la société n'a jamais cessé de considérer comme idéalistes, impraticables ou irresponsables. Les sociétés n'ont jamais été une pâte malléable dans les mains du clergé, qui était lui-même divisé sur bien des sujets de morale personnelle et familiale, et sur bien des points d'accord avec ses ouailles. En dépit du poids de l'Église, il n'y a jamais eu de coïncidence rigoureuse entre les lois civiles et les prescriptions religieuses. Face à celles-ci les autorités civiles ont suivant les cas adopté toutes les positions possibles : de la collaboration intéressée et active, et même pressante, quand cela allait dans le sens de leurs intérêts, à l'opposition franche, en passant par l'inertie sceptique.


\section{Constantin et le droit des personnes}


\begin{description}
\item[313] Constantin :
 \begin{enumerate}[leftmargin=*,itemsep=0pt]
%  a)
\item contrairement aux décisions de ses prédécesseurs, reconnaît aux citoyens indigents le droit de vendre leurs enfants, mais seulement à leur naissance ; 
% b)
\item décide que cette vente suspend la puissance paternelle. Le père garde le droit de récupérer son enfant, même contre le gré du possesseur, mais à la condition de fournir en échange un esclave de valeur équivalente, ou la somme correspondante ; 
% c)
\item reconnaît à celui qui recueille un nouveau-né exposé ou qui l'achète à son parent le droit d'en faire son esclave. Cette décision aggravait indiscutablement le sort juridique des enfants puisque le statut d'esclave ne pouvait plus être contesté, ni par les enfants concernés ni par leurs parents : s'ils parvenaient à les racheter, ces enfants étaient des affranchis, non des ingénus, avec toutes les limitations juridiques que cela entraînait. Il paraît plausible que cette décision ait été prise pour favoriser l'accueil de tous les enfants abandonnés sans exception. En effet jusque là les pères qui avaient abandonné, et non vendu, leurs enfants, pouvaient les récupérer à tout moment sans avoir à verser aucune contrepartie financière, puisqu'il était interdit d'asservir un citoyen né libre. Ils pouvaient donc être tentés de réclamer l'enfant qu'ils avaient exposé dès qu'il était capable de leur rapporter de l'argent, quitte à le revendre l'instant d'après : ce n'était pas impossible dans un monde esclavagiste. Cela pouvait suffire à dissuader les bonnes volontés et les spéculateurs de recueillir les enfants abandonnés ? 
\end{enumerate}
 
\item[315] Il décide (ou plutôt il rappelle ce qui était le cas jusque là) que la vente d'un ingénu est illégale dès qu'il ne s'agit plus d'un nouveau-né, et qu'elle ne peut effacer le statut initial d'un ingénu : l'intéressé peut donc en tout temps revendiquer sa liberté devant les tribunaux (à charge de fournir des preuves suffisantes).

% 315 
Il décide que les biens d'une mère défunte sont de droit la propriété de ses enfants. C'était d'ores et déjà l'usage établi, mais il fallait jusque là un testament maternel en bonne et due forme.

% 315 
Le mari devient le \emph{curateur} de sa femme, c'est lui et non plus le père de celle-ci qui gère ses affaires financières tant qu'elle n'a pas eu trois enfants et ne peut donc selon le Droit le faire elle-même. Était-ce un progrès ? Sûrement pour le mari, peut-être moins pour l'épouse qui ne pouvait plus se retourner vers son père ou son tuteur en cas de conflit avec son mari.

\item[316] Constantin revoit l'arsenal des peines prévues contre ceux qui enlèvent les enfants pour les vendre. Il leur promet la condamnation au travail forcé dans les mines%
%[1]
\footnote{Constantin a interdit la crucifixion, la condamnation aux jeux du cirque et celle des femmes aux lupanars, remplacés par la condamnation aux mines. En fait le travail des mines était très dur et particulièrement malsain (ex. emploi du feu en front d'exploitation pour désagréger la roche...) et on y mourait vite, ce n'était donc qu'un adoucissement relatif. L'interdit de Constantin avait l'avantage connexe de tarir l'une des sources des distractions offertes au public dans les jeux du cirque, condamnés depuis toujours par les chrétiens. Constantin interdisait aussi les mutilations du visage, ce qui ne l'empêchait pas de prévoir des peines terribles pour divers délits.}% 
. Cette décision n'était pas nouvelle, et ces rapts continueront aussi longtemps qu'il sera permis de vendre et d'acheter des esclaves.

\item[318] Il décide d'étendre la notion juridique de parricide à tous ceux, parents ou enfants, qui tuent un de leurs parents. Les parents infanticides seront désormais passibles de la peine de mort. 

 Dans le même mouvement \emph{il assimile l'avortement à un infanticide.}

 Par ailleurs à partir de Constantin, la puissance paternelle est retirée non seulement aux pères qui exposent leurs enfants, comme on l'a vu plus haut, mais aussi à ceux qui les prostituent et à ceux qui ont avec eux des relations sexuelles. Les pères déchus de leurs droits continuent de devoir assumer financièrement leurs enfants, mais ces derniers sont confiés à un tuteur.

\item[320] Il abroge les lois d'Auguste contre le célibat : dès l'âge de 25 ans, un homme ou une femme \emph{sans enfants}, célibataire ou non, \latin{sui juris}, peut recevoir tous les héritages venant de personnes extérieures à sa famille. Il n'est plus non plus question de sanctionner financièrement le célibat par un impôt spécial. Le refus du mariage ou du remariage n'est plus pénalisé.

\item[321] Constantin accorde à l'Église le droit de recevoir des legs et des successions, même par une simple déclaration orale. 

\item[324] Il dispense de tutelle les jeunes gens \latin{sui juris} (orphelins de père) qui ne sont pas infâmes%
%[2]
\footnote{... c'est-à-dire qu'il émancipe presque tous les jeunes citoyens : la plupart d'entre eux ne sont en effet pas infâmes, à l'exception des prostitué(e)s.}% 
, les filles dès 18 ans et les garçons dès 20 ans. Jusque là les garçons \latin{sui juris} avaient un tuteur jusqu'à leurs 25 ans, et seules les mères \latin{sui juris} de 3 enfants pouvaient être exemptées de tuteur à partir de leurs 25 ans. Les filles continuent de devoir obtenir l'accord de leurs parents (mère, oncles, grand-pères, frères (?)) pour se marier, \emph{mais elles reçoivent le droit de s'en passer pour entrer en religion}. Ce choix de vie est donc protégé contre les pressions des parents, alors que le choix du conjoint (et de la famille avec laquelle faire alliance) ne l'est pas : \emph{le mariage reste une alliance de deux familles}.

\item[325] Le concile de Nicée, approuvé par l'empereur, ordonne la création auprès de chaque évêque de « maisons de charité », pour les malades, les pauvres, les vieillards, les voyageurs, les pèlerins et les infirmes. Il confie la gestion de ces maisons à un « religieux du désert » c'est-à-dire à un moine, explicitement invité à considérer son travail quotidien au service des indigents et des malades comme une prière. En confirmant les décisions du concile Constantin donnait aux évêques la mission de réaliser en grand, à l'échelle de l'empire, ce qu'ils avaient expérimenté depuis le premier siècle et qu'ils affirmaient être au cœur de leur mission religieuse. Il faisait ainsi de l'assistance un service \emph{public} exercé par l'Église, ce qui donnait aux évêques le droit de réclamer aux autorités civiles des moyens à la mesure de leur mission de protecteurs des pauvres : dotations en argent, en domaines%
%[3] 
\footnote{... sur le statut juridique et fiscal desquels les discussions ne semblent pas terminées (cf. Jean \fsc{DURLIAT}, \emph{De l'Antiquité au Moyen Âge, l'Occident de 313 à 800}, 2002).} 
et en bâtiments%
%[4]
\footnote{En dépit du respect manifesté aux responsables ecclésiastiques, il s'agissait de moyens fournis par les autorités pour une mission et non de dons sans contrepartie, et il n'en sera jamais autrement. Les autorités civiles se sentiront toujours un droit de regard sur les moyens alloués aux évêques, de la même façon que les autorités des cités antiques n'ont jamais hésité à dépouiller les temples civiques de leurs trésors en cas de nécessité. Si Constantin a transféré une grande partie des biens des temples païens aux églises, c'est parce qu'il y voyait l'intérêt de son État. En dépit des protestations des clercs, les autorités civiles ne renonceront jamais longtemps à reprendre les moyens à eux confiés pour les affecter à d'autres fins lorsque cela leur paraitra aller dans le sens de l'intérêt général dont elles sont comptables.}% 
. Cela leur donnait un outil de conquête des esprits%
%[5]
\footnote{Lorsque Julien (empereur de 361 à 363) a tenté de restaurer les religions traditionnelles de l'empire, il a cherché à mettre en place un clergé païen hiérarchisé sous sa direction (il était \latin{pontifex maximus}) et il a voulu l'astreindre à fournir une véritable assistance à partir des temples, afin d'enlever aux chrétiens l'exclusivité d'un outil de séduction dont ils se servaient efficacement depuis leur apparition sur la scène religieuse antique.}% 
. 

\item[326] Le concubinage est interdit aux hommes \emph{mariés}. Cela n'aurait été qu'une décision symbolique sans grande portée s'il n'y avait eu à côté d'elle un ensemble de dispositions qui faisaient que désormais entretenir une concubine alors qu'on est marié présentait beaucoup moins d'intérêt et beaucoup plus d'inconvénients qu'auparavant : le principal de ces inconvénients était que les enfants nés des concubines des hommes mariés ne pouvaient plus être légitimés : ils étaient désormais considérés comme des enfants adultères. Ils ne pouvaient donc ni hériter de leur père ni lui succéder. Cela leur interdisait%
%[6] 
\footnote{… en théorie du moins, mais il y aura assez souvent des passe-droits au fil des siècles, avec des périodes très strictes et des périodes très tolérantes. Ceci dit cette règle va demeurer jusqu'au \siecle{20}.} 
de remplacer en cas de nécessité les héritiers légitimes, espérés en vain, ou décédés. 

% 326 
Les hommes qui n'ont pas d'épouse vivante ni d'enfants légitimes, mais qui ont une concubine ingénue (non esclave et non affranchie), de bonne réputation (non infâme, fidèle, non issue de la prostitution), et qui ont eu des enfants de cette femme, sont invités à l'épouser : s'ils le font les enfants qu'ils ont eu en commun avant le mariage seront reconnus comme légitimes. Dans tous les autres cas la légitimation des enfants illégitimes est interdite. Cette mesure a d'abord été prévue pour une période de transition d'une année, pour apurer le passé : dans un monde idéal il ne devait plus naître à partir de cette date aucun enfant illégitime. Face à la résistance des réalités elle sera renouvelée à plusieurs reprises jusqu'à ce qu'elle devienne permanente moins d'un siècle plus tard. C'est la \emph{légitimation par mariage subséquent}.

% 326 : 
Il n'était pas plus question pour Constantin, en dépit des pressions éventuelles de l'Église, que pour aucun de ses prédécesseurs de sanctionner les maris pour leurs propres infidélités, sauf quand ils avaient des relations avec la femme d'un autre. Par contre il abroge la loi d'Auguste contre l'adultère des femmes. Il décide qu'une femme adultère ne peut plus être dénoncée par son propre père (ce qui décharge ce dernier de l'obligation qui lui était faite de la dénoncer), ni par les étrangers (à qui les dénonciations rapportaient une part significative des fortunes confisquées aux deux coupables par le fisc impérial). Désormais seul le mari, et ses proches (cousin, beau-frère et frère), peuvent dénoncer les amants, mais ils n'y sont pas obligés, et en ce cas le mari règle l'affaire comme il règlerait n'importe quel autre conflit domestique. S'agissait-il pour Constantin de supprimer les dénonciations calomnieuses ? Ou de permettre au mari lésé de pardonner comme le demandait l'Église ? Par contre lorsque le mari choisissait de traîner sa femme en justice la peine maximale n'était plus comme auparavant l'infamie et l'exil, mais la mort des deux complices. En l'absence de données suffisantes on ne sait ni quel était le degré réel de répression des adultères avant Constantin, ni combien de coupables ont effectivement subi la peine qu'il avait prévue en cas de dénonciation par le mari. 

\item[331] La répudiation (rupture unilatérale du mariage, par opposition au divorce par consentement mutuel) devient un délit. L'épouse qui prend l'initiative de divorcer est condamnée à l'exil (assignée à résidence loin de chez elle), ce qui lui interdit le remariage, et elle perd une part substantielle de sa dot. Un homme qui répudie sa femme doit lui rendre l'intégralité de sa dot et ne peut plus se remarier non plus (mais il n'est pas exilé). La répudiation reste néanmoins autorisée, et l'époux(se) innocent(e) peut se remarier, dans les cas suivants :
\begin{enumerate}[leftmargin=*,itemsep=0pt]
% a)
\item si le mari est condamné à une peine infamante, ou s'il devient esclave, ou s'il est condamné comme homicide, violateur de sépulture ou empoisonneur ;
% b)
\item si la femme est convaincue d'être adultère, empoisonneuse ou entremetteuse.
 \end{enumerate}
Ces dispositions concernaient tous les citoyens, chrétiens ou non. Pendant ce temps-là les divorces par consentement mutuel restaient possibles, et dans ce cas les remariages l'étaient aussi. Les lois de l'Église ne concernaient pour le moment que les chrétiens, qui ne constituaient pas encore l'ensemble de la population. Moins d'un siècle plus tard un empereur ordonnera à tous les païens de se faire baptiser. Le remariage après divorce sera désormais \emph{en principe} interdit à tous les citoyens de l'Empire, sauf aux juifs%
%[7]
\footnote{Ceux-ci ne seront jamais interdits de remariage jusqu'à la Restauration (\siecle{19}).}% 
.

\item[334] Constantin interdit de séparer les membres d'une même famille pour les vendre comme esclaves, interdiction qui implique que de telles ventes se faisaient encore. Ce décret reprend à son compte des interdits déjà formulés au \siecle{3}. Son existence montre que les unions des esclaves, même si elles n'étaient pas reconnues comme de vrais mariages, étaient alors perçues de manière suffisamment positive pour créer des droits opposables aux maîtres qui les avaient autorisées. 

\item[336] L'Église avait toujours défendu la légitimité de tous les mariages librement voulus entre deux personnes non parentes, quel que soient leurs statuts légaux (esclaves, affranchis, citoyens, chevaliers, sénateurs...). Cela n'empêche pas Constantin de rappeler l'interdit de reconnaître les enfants nés des unions traditionnellement interdites par la loi : union d'un citoyen (tous les hommes libres depuis 212) avec un infâme, d'un sénateur avec une affranchie ou avec une esclave, d'un citoyen avec une esclave. Jusqu'à cette date les autorités pouvaient malgré tout les déclarer légitimes. Cette décision interdit en principe de le faire. 

\item[342] Les interdits de mariage traditionnels romains sont réaffirmés et le \emph{senatus-consulte} autorisant un oncle paternel à épouser sa nièce est abrogé.

 Si le Droit est l'expression des mœurs, alors il nous faut croire que Constantin a aligné le Droit romain sur les mœurs de son temps. Doit-on en déduire que celles-ci étaient déjà chrétiennes (ou christiano-stoïciennes) bien avant qu'il ne conquière le pouvoir ? Cela signifierait que les limites et interdits que Constantin a opposés à l'expression libre et spontanée des désirs sexuels et la canalisation de ces désirs sur le seul mariage monogame et fidèle faisaient déjà partie de l'idéal moral de son temps, païens et chrétiens confondus%
% [8]
\footnote{Aline \fsc{ROUSSELLE}, \emph{La contamination spirituelle, science, droit et religion dans l'Antiquité}, 1998.} 
 ? Nous avons aujourd'hui du mal à imaginer une telle situation tant nous paraît improbable un mouvement qui irait spontanément d'un niveau élevé de liberté sexuelle et matrimoniale vers un verrouillage du mariage et une limitation drastique des possibilités d'obtenir des enfants légitimes et des héritiers ... à moins que l'impression de grande liberté et d'aisance que suggère ce que l'on croit connaître de la vie des grecs et des romains de l'Antiquité ne corresponde pas à la réalité vécue, au moins par les dépendants (femmes, mineurs, esclaves)%
% [9]
\footnote{Cf. David \fsc{BROWN}, \emph{Le renoncement à la chair}, 2002} 
 ? 

 Mais une autre hypothèse est tout aussi vraisemblable. Les lois peuvent être une déclaration d'intention. Elles peuvent être chargées de désigner le bien, et le Droit peut être un instrument de normalisation des comportements et de remodelage des représentations. En ce sens les lois qu'ont édictées les empereurs chrétiens ont eu pour objectif d'orienter les comportements de leurs sujets dans le sens qui leur convenait, sans attendre qu'ils se soient tous convertis. 

 Au fil des deux siècles suivants les décisions fondatrices de Constantin ont été complétées par ses successeurs : 

\item[374] Valentinien décrète que les parents doivent subvenir aux besoins de tous leurs enfants, légitimes ou non. \emph{L'abandon est interdit}. Ce texte, le premier du genre, ne prévoit en fait aucune sanction. Il se contente d'affirmer un principe, de dénier aux pères le droit à l'abandon de leurs nouveaux-nés \emph{s'ils ont les moyens de l'élever}. Il semble n'avoir jamais été utilisé à l'encontre de parents incapables de nourrir leurs enfants, du moment que la vie de ceux-ci n'était pas mise en danger, et que leur découverte avait été facilitée (enfant placé en évidence, protégé des intempéries et surtout des animaux errants, exposé dans un lieu où passent beaucoup de personnes,~etc.). 

\item[384] $\!$ou \textbf{385}\,{} L'empereur Théodose condamne le mariage entre cousins germains. Vingt années plus tard, l'empereur d'Orient Arcadius lève cet interdit dans les territoires qui relèvent de son autorité. 

\item[390] \emph{L'empereur accorde aux veuves le droit d'exercer la tutelle de leurs propres enfants mineurs}. Par ce biais est pour la première fois reconnue à des femmes une pleine capacité à représenter autrui, même si les conditions de cette reconnaissance sont précises et limitées :
\begin{enumerate}[leftmargin=*,itemsep=0pt]
% a)
\item seuls sont concernés leurs propres enfants mineurs,
% b)
\item il leur faut avoir atteint cinquante ans, âge où elles ne pouvaient plus espérer d'autres grossesses, et
% c)
\item elles doivent promettre de ne pas se remarier : ce faisant elles retomberaient en effet dans la main, sous la coupe d'un homme à qui elles seraient obligées d'obéir, au risque de nuire aux enfants nés de leur union avec le conjoint décédé%
%[11]
\footnote{Il n'est pas sûr que cette mesure ait porté sur de très grands nombres de personnes, compte tenu du fait que les veuves âgées de plus de cinquante ans et ayant encore des enfants mineurs (moins de 25 ans) ne devaient pas être très nombreuses, puisque les femmes commençaient souvent d'avoir leurs enfants très tôt, bien avant leurs 20 ans. C'était néanmoins un pas important vers la reconnaissance du principe de l'égalité juridique des époux.}% 
.
\end{enumerate}

 Théodose II (empereur d'Orient de 408 à 450) ordonne qu'un procès-verbal de découverte soit rédigé pour chaque enfant trouvé. C'est la première fois que ces enfants reçoivent une reconnaissance administrative (et une forme minimale d'existence légale) avant même qu'une personne n'accepte de répondre d'eux et ne leur donne statut de citoyen en les déclarant comme libres. Cela signifie peut-être que désormais le souverain en prend possession même s'il les confie immédiatement à l'Église ? 

\item[438] Le Code Théodosien étend les interdits de mariage aux cousins germains. Cette règle ne sera pas acceptée et ne sera pas reprise par le Code de Justinien, mais elle préfigure la législation des siècles suivants. Le Code Théodosien interdit également les mariages entre beaux-frères et belles-sœurs.

\item[442] Le Concile de Vaison et celui d'Arles (\textbf{452})
%[12] 
décident%
\footnote{Comme il est de règle jusqu'à l'an mil, ces deux conciles ont été convoqués par les autorités civiles, et leurs décisions ont été promulguées par ces mêmes autorités.} 
(ou rappellent%
%[13]
\footnote{Probablement une fois de plus s'agissait-il avec ces décisions de généraliser des mesures déjà largement expérimentées.} 
 ?) que l'enfant exposé sera porté à l'église sur le territoire de laquelle il a été trouvé et qu'il y sera enregistré. \emph{Le dimanche suivant, le prêtre annoncera aux fidèles qu'un nouveau-né a été trouvé, et dix jours seront accordés aux parents pour reconnaître et réclamer leur enfant}. S'ils ne se manifestaient pas dans ce délai l'enfant devait être remis à titre onéreux, et non pas donné, à celui qui se proposait de le prendre en charge. À défaut d'un laïc volontaire pour le prendre (pour l'acheter ?) l'enfant pouvait (devait ?) être mis en nourrice aux frais de la communauté ecclésiale. 

 Le code de Justinien est une compilation du Droit romain réalisée au \siecle{6} sous la direction de cet empereur de Constantinople. Depuis Constantin, l'ancien droit de vie et de mort paternel \latin{(jus vitae necisque)} n'existait plus. Tout père qui tuait volontairement son enfant, même à sa naissance, même non encore né (avortement provoqué) était passible de la peine de mort. Dans le code de Justinien le père conserve le droit de correction paternelle, mais il n'a plus le droit d'infliger de graves châtiments corporels, de blesser ni d'estropier. S'il juge nécessaire de recourir à des châtiments sévères il doit s'adresser au gouverneur de la province ou au préfet de la ville%
% [14]
\footnote{Jean \fsc{IMBERT}, \emph{Le droit antique}, Que sais-je, 1961, p.93.} 
 : c'est la doctrine juridique qui va perdurer jusqu'au \siecle{20}. 

\item[529] Justinien décrète que deux amant adultères n'auront jamais le droit de s'épouser même si venait à décéder l'époux (ou les époux) qui leur faisait obstacle ;

% 529 :
% Il décrète 
Et qu'une femme convaincue d'adultère sera condamnée à vivre dans un couvent de femmes%
% [15] 
\footnote{Jusqu'à la Révolution, et en fait jusqu'au \siecle{20}, les incarcérations de femmes se feront dans des monastères pour femmes ou dans des lieux inspirés de ce modèle.} 
jusqu'à sa mort ... si du moins son époux porte plainte, mais rien n'oblige ce dernier à le faire. Compte tenu du fait que les textes antérieurs (ceux de Constantin) donnaient à l'époux droit de vie et de mort, c'est un adoucissement majeur. Il pourra se séparer de sa femme adultère, mais ne pourra pas se remarier. Cette règle de droit ne concerne que les chrétiens mais à cette date cela fait longtemps que tous les citoyens, à l'exception des juifs, ont reçu l'ordre de se faire chrétiens. Par conséquent s'il veut avoir des enfants légitimes, des héritiers, un époux bafoué n'a plus d'autre choix (en principe) que de se réconcilier avec sa femme et de reprendre la vie commune, ce qui eut été le comble de l'indécence ou de l'infamie deux siècles plus tôt. En dépit de l'infidélité passée de celle-ci il est même expressément invité à lui pardonner au bout d'un certain temps de réclusion dans un couvent pour femmes (2 ans au maximum ?). Autant dire que son intérêt n'est pas forcément de porter le cas de son épouse coupable devant les tribunaux. L'adultère féminin devient de plus en plus une affaire privée, même si le pouvoir civil continue et continuera jusqu'à la fin de l'ancien régime de prêter la main au mari pour soutenir son droit de correction marital. 

\item[533] Les \latin{institutes} de Justinien réaffirment la légitimité du mariage entre cousins, ou celui d'un veuf ou d'une veuve avec le frère ou la sœur de son conjoint décédé. Cela montre la résistance des autorités civiles à l'autorité morale de l'Église. Sur ce point précis l'Orient refusait la position intransigeante de l'église de Rome face à tout ce qui ressemble à l'inceste%
% [10]
\footnote{Cf. Jack \fsc{GOODY}, \emph{L'évolution de la famille et du mariage en Europe}, p. 66.}%
.

%\item[534] Le Code de Justinien décrète que les enfants adultérins et incestueux n'ont aucun droit, au contraire des autres enfants illégitimes : il leur dénie tout droit à des aliments, bien qu'il ne soit pas interdit à leurs
\item[534] Le Code de Justinien dénie tout droit aux enfants adultérins et incestueux, au contraire des autres enfants illégitimes. Ils n'ont pas droit à des aliments, bien qu'il ne soit pas interdit à leurs
géniteurs de leur en donner. Ils ne peuvent en aucun cas être légitimés. Au nom de la sauvegarde de l'institution familiale, cette décision retire à ces enfants les protections que leurs géniteurs pourraient vouloir leur donner. Ils sont réputés \emph{enfants trouvés}, et traités comme ces derniers par les institutions chargées de s'en occuper. 

 Par ailleurs le code de Justinien confirme \emph{l'interdiction de l'adrogation des enfants illégitimes}. Les « enfants du péché » (le péché des adultes contre l'institution familiale) sont désormais à écarter. Cela contraste avec la volonté de protection de tous les enfants qui se traduisait dans les autres décisions de l'époque, mais il est clair que l'objectif de ces lois était d'abord de prévenir la naissance de ces enfants. La suite de l'histoire montrera que cet objectif a été à peu près atteint en dépit d'un « reste » incompressible (seulement ... ou encore, un pour cent de naissances illégitimes dans diverses régions rurales françaises sous Louis~XIV). 

 Le code de Justinien confie les enfants trouvés à l'évêque du lieu de leur découverte. Il ordonne aux magistrats de s'en tenir au principe juridique que les enfants trouvés sont tous nés libres et non esclaves, même lorsque leur mère est une esclave : l'abandon donne donc aux enfants d'esclave une chance de vivre libres. 

 En l'absence d'enfants légitimes le code de Justinien autorise les enfants naturels nés d'un concubinage stable à hériter de leurs parents : il confirme ainsi le concubinage stable dans son statut de mariage à l'usage des pauvres et des humbles. 
\end{description}







\chapter{Constantin et le droit des personnes}


\begin{description}
\item[313] Constantin :
 \begin{enumerate}[leftmargin=*,itemsep=0pt]
%  a)
\item contrairement aux décisions de ses prédécesseurs, reconnaît aux citoyens indigents le droit de vendre leurs enfants, mais seulement à leur naissance ; 
% b)
\item décide que cette vente suspend la puissance paternelle. Le père garde le droit de récupérer son enfant, même contre le gré du possesseur, mais à la condition de fournir en échange un esclave de valeur équivalente, ou la somme correspondante ; 
% c)
\item reconnaît à celui qui recueille un nouveau-né exposé ou qui l'achète à son parent le droit d'en faire son esclave. Cette décision aggravait indiscutablement le sort juridique des enfants puisque le statut d'esclave ne pouvait plus être contesté, ni par les enfants concernés ni par leurs parents : s'ils parvenaient à les racheter, ces enfants étaient des affranchis, non des ingénus, avec toutes les limitations juridiques que cela entraînait. Il paraît plausible que cette décision ait été prise pour favoriser l'accueil de tous les enfants abandonnés sans exception. En effet jusque là les pères qui avaient abandonné, et non vendu, leurs enfants, pouvaient les récupérer à tout moment sans avoir à verser aucune contrepartie financière, puisqu'il était interdit d'asservir un citoyen né libre. Ils pouvaient donc être tentés de réclamer l'enfant qu'ils avaient exposé dès qu'il était capable de leur rapporter de l'argent, quitte à le revendre l'instant d'après : ce n'était pas impossible dans un monde esclavagiste. Cela pouvait suffire à dissuader les bonnes volontés et les spéculateurs de recueillir les enfants abandonnés ? 
\end{enumerate}
 
\item[315] Il décide (ou plutôt il rappelle ce qui était le cas jusque là) que la vente d'un ingénu est illégale dès qu'il ne s'agit plus d'un nouveau-né, et qu'elle ne peut effacer le statut initial d'un ingénu : l'intéressé peut donc en tout temps revendiquer sa liberté devant les tribunaux (à charge de fournir des preuves suffisantes).

% 315 
Il décide que les biens d'une mère défunte sont de droit la propriété de ses enfants. C'était d'ores et déjà l'usage établi, mais il fallait jusque là un testament maternel en bonne et due forme.

% 315 
Le mari devient le \emph{curateur} de sa femme, c'est lui et non plus le père de celle-ci qui gère ses affaires financières tant qu'elle n'a pas eu trois enfants et ne peut donc selon le droit le faire elle-même. Était-ce un progrès ? Sûrement pour le mari, peut-être moins pour l'épouse qui ne pouvait plus se retourner vers son père ou son tuteur en cas de conflit avec son mari.

\item[316] Constantin revoit l'arsenal des peines prévues contre ceux qui enlèvent les enfants pour les vendre. Il leur promet la condamnation au travail forcé dans les mines%
%[1]
\footnote{Constantin a interdit la crucifixion, la condamnation aux jeux du cirque et celle des femmes aux lupanars, remplacés par la condamnation aux mines. En fait le travail des mines était très dur et particulièrement malsain (ex. emploi du feu en front d'exploitation pour désagréger la roche...) et on y mourait vite, ce n'était donc qu'un adoucissement relatif. L'interdit de Constantin avait l'avantage connexe de tarir l'une des sources des distractions offertes au public dans les jeux du cirque, condamnés depuis toujours par les chrétiens. Constantin interdisait aussi les mutilations du visage, ce qui ne l'empêchait pas de prévoir des peines terribles pour divers délits.}% 
. Cette décision n'était pas nouvelle, et ces rapts continueront aussi longtemps qu'il sera permis de vendre et d'acheter des esclaves.

\item[318] Il décide d'étendre la notion juridique de parricide à tous ceux, parents ou enfants, qui tuent un de leurs parents. Les parents infanticides seront désormais passibles de la peine de mort. 

 Dans le même mouvement \emph{il assimile l'avortement à un infanticide.}

 Par ailleurs à partir de Constantin la puissance paternelle est retirée non seulement aux pères qui exposent leurs enfants, comme on l'a vu plus haut, mais aussi à ceux qui les prostituent et à ceux qui ont avec eux des relations sexuelles. Les pères déchus de leurs droits continuent de devoir assumer financièrement leurs enfants mais ces derniers sont confiés à un tuteur.

\item[320] Il abroge les lois d'Auguste contre le célibat : dès l'âge de 25 ans, un homme ou une femme \emph{sans enfants}, célibataire ou non, \emph{sui juris}, peut recevoir tous les héritages venant de personnes extérieures à sa famille. Il n'est plus non plus question de sanctionner financièrement le célibat par un impôt spécial. Le refus du mariage ou du remariage n'est plus pénalisé.

\item[321] Constantin accorde à l'Église le droit de recevoir des legs et des successions, même par une simple déclaration orale. 

\item[324] Il dispense de tutelle les jeunes gens \emph{sui juris} (orphelins de père) qui ne sont pas infâmes%
%[2]
\footnote{... c'est-à-dire qu'il émancipe presque tous les jeunes citoyens : la plupart d'entre eux ne sont en effet pas infâmes, à l'exception des prostitué(e)s.}% 
, les filles dès 18 ans et les garçons dès 20 ans. Jusque là les garçons \emph{sui juris} avaient un tuteur jusqu'à leurs 25 ans, et seules les mères \emph{sui juris} de 3 enfants pouvaient être exemptées de tuteur à partir de leurs 25 ans. Les filles continuent de devoir obtenir l'accord de leurs parents (mère, oncles, grand-pères, frères (?)) pour se marier, \emph{mais elles reçoivent le droit de s'en passer pour entrer en religion}. Ce choix de vie est donc protégé contre les pressions des parents, alors que le choix du conjoint (et de la famille avec laquelle faire alliance) ne l'est pas : \emph{le mariage reste une alliance de deux familles}.

\item[325] Le concile de Nicée, approuvé par l'empereur, ordonne la création auprès de chaque évêque de « maisons de charité », pour les malades, les pauvres, les vieillards, les voyageurs, les pèlerins et les infirmes. Il confie la gestion de ces maisons à un « religieux du désert » c'est-à-dire à un moine, explicitement invité à considérer son travail quotidien au service des indigents et des malades comme une prière. En confirmant les décisions du concile Constantin donnait aux évêques la mission de réaliser en grand, à l'échelle de l'empire, ce qu'ils avaient expérimenté depuis le premier siècle et qu'ils affirmaient être au cœur de leur mission religieuse. Il faisait ainsi de l'assistance un service \emph{public} exercé par l'Église, ce qui donnait aux évêques le droit de réclamer aux autorités civiles des moyens à la mesure de leur mission de protecteurs des pauvres : dotations en argent, en domaines%
%[3] 
\footnote{... sur le statut juridique et fiscal desquels les discussions ne semblent pas terminées (cf. Jean \fsc{DURLIAT}, \emph{De l'antiquité au Moyen-âge, l'Occident de 313 à 800}, 2002).} 
et en bâtiments%
%[4]
\footnote{En dépit du respect manifesté aux responsables ecclésiastiques, il s'agissait de moyens fournis par les autorités pour une mission et non de dons sans contrepartie, et il n'en sera jamais autrement. Les autorités civiles se sentiront toujours un droit de regard sur les moyens alloués aux évêques, de la même façon que les autorités des cités antiques n'ont jamais hésité à dépouiller les temples civiques de leurs trésors en cas de nécessité. Si Constantin a transféré une grande partie des biens des temples païens aux églises, c'est parce qu'il y voyait l'intérêt de son État. En dépit des protestations des clercs, les autorités civiles ne renonceront jamais longtemps à reprendre les moyens à eux confiés pour les affecter à d'autres fins lorsque cela leur paraitra aller dans le sens de l'intérêt général dont elles sont comptables.}% 
. Cela leur donnait un outil de conquête des esprits%
%[5]
\footnote{Lorsque Julien (empereur de 361 à 363) a tenté de restaurer les religions traditionnelles de l'empire, il a cherché à mettre en place un clergé païen hiérarchisé sous sa direction (il était \emph{pontifex maximus}) et il a voulu l'astreindre à fournir une véritable assistance à partir des temples, afin d'enlever aux chrétiens l'exclusivité d'un outil de séduction dont ils se servaient efficacement depuis leur apparition sur la scène religieuse antique.}% 
. 

\item[326] Le concubinage est interdit aux hommes \emph{mariés}. Cela n'aurait été qu'une décision symbolique sans grande portée s'il n'y avait eu à côté d'elle un ensemble de dispositions qui faisaient que désormais entretenir une concubine alors qu'on est marié présentait beaucoup moins d'intérêt et beaucoup plus d'inconvénients qu'auparavant : le principal de ces inconvénients était que les enfants nés des concubines des hommes mariés ne pouvaient plus être légitimés : ils étaient désormais considérés comme des enfants adultères. Ils ne pouvaient donc ni hériter de leur père ni lui succéder. Cela leur interdisait%
%[6] 
\footnote{… en théorie du moins, mais il y aura assez souvent des passe-droits au fil des siècles, avec des périodes très strictes et des périodes très tolérantes. Ceci dit cette règle va demeurer jusqu'au \siecle{20}.} 
de remplacer en cas de nécessité les héritiers légitimes, espérés en vain, ou décédés. 

% 326 
Les hommes qui n'ont pas d'épouse vivante ni d'enfants légitimes, mais qui ont une concubine ingénue (non esclave et non affranchie), de bonne réputation (non infâme, fidèle, non issue de la prostitution), et qui ont eu des enfants de cette femme, sont invités à l'épouser : s'ils le font les enfants qu'ils ont eu en commun avant le mariage seront reconnus comme légitimes. Dans tous les autres cas la légitimation des enfants illégitimes est interdite. Cette mesure a d'abord été prévue pour une période de transition d'une année, pour apurer le passé : dans un monde idéal il ne devait plus naître à partir de cette date aucun enfant illégitime. Face à la résistance des réalités elle sera renouvelée à plusieurs reprises jusqu'à ce qu'elle devienne permanente moins d'un siècle plus tard. C'est la \emph{légitimation par mariage subséquent}.

% 326 : 
Il n'était pas plus question pour Constantin, en dépit des pressions éventuelles de l'Église, que pour aucun de ses prédécesseurs de sanctionner les maris pour leurs propres infidélités, sauf quand ils avaient des relations avec la femme d'un autre. Par contre il abroge la loi d'Auguste contre l'adultère des femmes. Il décide qu'une femme adultère ne peut plus être dénoncée par son propre père (ce qui décharge ce dernier de l'obligation qui lui était faite de la dénoncer), ni par les étrangers (à qui les dénonciations rapportaient une part significative des fortunes confisquées aux deux coupables par le fisc impérial). Désormais seul le mari, et ses proches (cousin, beau-frère et frère), peuvent dénoncer les amants, mais ils n'y sont pas obligés, et en ce cas le mari règle l'affaire comme il règlerait n'importe quel autre conflit domestique. S'agissait-il pour Constantin de supprimer les dénonciations calomnieuses ? Ou de permettre au mari lésé de pardonner comme le demandait l'Église ? Par contre lorsque le mari choisissait de traîner sa femme en justice la peine maximale n'était plus comme auparavant l'infamie et l'exil, mais la mort des deux complices. En l'absence de données suffisantes on ne sait ni quel était le degré réel de répression des adultères avant Constantin, ni combien de coupables ont effectivement subi la peine qu'il avait prévue en cas de dénonciation par le mari. 

\item[331] La répudiation (rupture unilatérale du mariage, par opposition au divorce par consentement mutuel) devient un délit. L'épouse qui prend l'initiative de divorcer est condamnée à l'exil (assignée à résidence loin de chez elle), ce qui lui interdit le remariage, et elle perd une part substantielle de sa dot. Un homme qui répudie sa femme doit lui rendre l'intégralité de sa dot et ne peut plus se remarier non plus (mais il n'est pas exilé). La répudiation reste néanmoins autorisée, et l'époux(se) innocent(e) peut se remarier, dans les cas suivants :
\begin{enumerate}[leftmargin=*,itemsep=0pt]
% a)
\item si le mari est condamné à une peine infamante, ou s'il devient esclave, ou s'il est condamné comme homicide, violateur de sépulture ou empoisonneur ;
% b)
\item si la femme est convaincue d'être adultère, empoisonneuse ou entremetteuse.
 \end{enumerate}
Ces dispositions concernaient tous les citoyens, chrétiens ou non. Pendant ce temps-là les divorces par consentement mutuel restaient possibles, et dans ce cas les remariages l'étaient aussi. Les lois de l'Église ne concernaient pour le moment que les chrétiens, qui ne constituaient pas encore l'ensemble de la population. Moins d'un siècle plus tard un empereur ordonnera à tous les païens de se faire baptiser. Le remariage après divorce sera désormais \emph{en principe} interdit à tous les citoyens de l'Empire, sauf aux juifs%
%[7]
\footnote{Ceux-ci ne seront jamais interdits de remariage jusqu'à la Restauration (\siecle{19}).}% 
.

\item[334] Constantin interdit de séparer les membres d'une même famille pour les vendre comme esclaves, interdiction qui implique que de telles ventes se faisaient encore. Ce décret reprend à son compte des interdits déjà formulés au \siecle{3}. Son existence montre que les unions des esclaves, même si elles n'étaient pas reconnues comme de vrais mariages, étaient alors perçues de manière suffisamment positive pour créer des droits opposables aux maîtres qui les avaient autorisées. 

\item[336] L'Église avait toujours défendu la légitimité de tous les mariages librement voulus entre deux personnes non parentes, quel que soient leurs statuts légaux (esclaves, affranchis, citoyens, chevaliers, sénateurs...). Cela n'empêche pas Constantin de rappeler l'interdit de reconnaître les enfants nés des unions traditionnellement interdites par la loi : union d'un citoyen (tous les hommes libres depuis 212) avec un infâme, d'un sénateur avec une affranchie ou avec une esclave, d'un citoyen avec une esclave. Jusqu'à cette date les autorités pouvaient malgré tout les déclarer légitimes. Cette décision interdit en principe de le faire. 

\item[342] Les interdits de mariage traditionnels romains sont réaffirmés et le \emph{senatus-consulte} autorisant un oncle paternel à épouser sa nièce est abrogé.

 Si le droit est l'expression des mœurs alors il nous faut croire que Constantin a aligné le droit romain sur les mœurs de son temps. Doit-on en déduire que celles-ci étaient déjà chrétiennes (ou christiano-stoïciennes) bien avant qu'il ne conquière le pouvoir ? Cela signifierait que les limites et interdits que Constantin a opposés à l'expression libre et spontanée des désirs sexuels et la canalisation de ces désirs sur le seul mariage monogame et fidèle faisaient déjà partie de l'idéal moral de son temps, païens et chrétiens confondus%
% [8]
\footnote{Aline \fsc{ROUSSELLE}, \emph{La contamination spirituelle, science, droit et religion dans l'antiquité}, 1998.} 
 ? Nous avons aujourd'hui du mal à imaginer une telle situation tant nous paraît improbable un mouvement qui irait spontanément d'un niveau élevé de liberté sexuelle et matrimoniale vers un verrouillage du mariage et une limitation drastique des possibilités d'obtenir des enfants légitimes et des héritiers ... à moins que l'impression de grande liberté et d'aisance que suggère ce que l'on croit connaître de la vie des grecs et des romains de l'antiquité ne corresponde pas à la réalité vécue, au moins par les dépendants (femmes, mineurs, esclaves)%
% [9]
\footnote{Cf. David \fsc{BROWN}, \emph{Le renoncement à la chair}, 2002} 
 ? 

 Mais une autre hypothèse est tout aussi vraisemblable. Les lois peuvent être une déclaration d'intention. Elles peuvent être chargées de désigner le bien, et le droit peut être un instrument de normalisation des comportements et de remodelage des représentations. En ce sens les lois qu'ont édictées les empereurs chrétiens ont eu pour objectif d'orienter les comportements de leurs sujets dans le sens qui leur convenait, sans attendre qu'ils se soient tous convertis. 

 Au fil des deux siècles suivants les décisions fondatrices de Constantin ont été complétées par ses successeurs : 

\item[374] Valentinien décrète que les parents doivent subvenir aux besoins de tous leurs enfants, légitimes ou non. \emph{L'abandon est interdit}. Ce texte, le premier du genre, ne prévoit en fait aucune sanction. Il se contente d'affirmer un principe, de dénier aux pères le droit à l'abandon de leurs nouveaux-nés \emph{s'ils ont les moyens de l'élever}. Il semble n'avoir jamais été utilisé à l'encontre de parents incapables de nourrir leurs enfants, du moment que la vie de ceux-ci n'était pas mise en danger, et que leur découverte avait été facilitée (enfant placé en évidence, protégé des intempéries et surtout des animaux errants, exposé dans un lieu où passent beaucoup de personnes,~etc.). 

\item[384] $\!$ou \textbf{385}\,{} L'empereur Théodose condamne le mariage entre cousins germains. Vingt années plus tard, l'empereur d'Orient Arcadius lève cet interdit dans les territoires qui relèvent de son autorité. 

\item[390] \emph{L'empereur accorde aux veuves le droit d'exercer la tutelle de leurs propres enfants mineurs}. Par ce biais est pour la première fois reconnue à des femmes une pleine capacité à représenter autrui, même si les conditions de cette reconnaissance sont précises et limitées :
\begin{enumerate}[leftmargin=*,itemsep=0pt]
% a)
\item seuls sont concernés leurs propres enfants mineurs,
% b)
\item il leur faut avoir atteint cinquante ans, âge où elles ne pouvaient plus espérer d'autres grossesses, et
% c)
\item elles doivent promettre de ne pas se remarier : ce faisant elles retomberaient en effet dans la main, sous la coupe d'un homme à qui elles seraient obligées d'obéir, au risque de nuire aux enfants nés de leur union avec le conjoint décédé%
%[11]
\footnote{Il n'est pas sûr que cette mesure ait porté sur de très grands nombres de personnes, compte tenu du fait que les veuves âgées de plus de cinquante ans et ayant encore des enfants mineurs (moins de 25 ans) ne devaient pas être très nombreuses, puisque les femmes commençaient souvent d'avoir leurs enfants très tôt, bien avant leurs 20 ans. C'était néanmoins un pas important vers la reconnaissance du principe de l'égalité juridique des époux.}% 
.
\end{enumerate}

 Théodose II (empereur d'Orient de 408 à 450) ordonne qu'un procès-verbal de découverte soit rédigé pour chaque enfant trouvé. C'est la première fois que ces enfants reçoivent une reconnaissance administrative (et une forme minimale d'existence légale) avant même qu'une personne n'accepte de répondre d'eux et ne leur donne statut de citoyen en les déclarant comme libres. Cela signifie peut-être que désormais le souverain en prend possession même s'il les confie immédiatement à l'Église ? 

\item[438] Le Code Théodosien étend les interdits de mariage aux cousins germains. Cette règle ne sera pas acceptée et ne sera pas reprise par le Code de Justinien, mais elle préfigure la législation des siècles suivants. Le Code Théodosien interdit également les mariages entre beaux-frères et belles-sœurs.

\item[442] Le Concile de Vaison et celui d'Arles (\textbf{452})
%[12] 
décident%
\footnote{Comme il est de règle jusqu'à l'an mil, ces deux conciles ont été convoqués par les autorités civiles, et leurs décisions ont été promulguées par ces mêmes autorités.} 
(ou rappellent%
%[13]
\footnote{Probablement une fois de plus s'agissait-il avec ces décisions de généraliser des mesures déjà largement expérimentées.} 
 ?) que l'enfant exposé sera porté à l'église sur le territoire de laquelle il a été trouvé et qu'il y sera enregistré. \emph{Le dimanche suivant, le prêtre annoncera aux fidèles qu'un nouveau-né a été trouvé, et dix jours seront accordés aux parents pour reconnaître et réclamer leur enfant}. S'ils ne se manifestaient pas dans ce délai l'enfant devait être remis à titre onéreux, et non pas donné, à celui qui se proposait de le prendre en charge. À défaut d'un laïc volontaire pour le prendre (pour l'acheter ?) l'enfant pouvait (devait ?) être mis en nourrice aux frais de la communauté ecclésiale. 

 Le code de Justinien est une compilation du droit romain réalisée au \siecle{6} sous la direction de cet empereur de Constantinople. Depuis Constantin, l'ancien droit de vie et de mort paternel \emph{(jus vitae necisque)} n'existait plus. Tout père qui tuait volontairement son enfant, même à sa naissance, même non encore né (avortement provoqué) était passible de la peine de mort. Dans le code de Justinien le père conserve le droit de correction paternelle, mais il n'a plus le droit d'infliger de graves châtiments corporels, de blesser ni d'estropier. S'il juge nécessaire de recourir à des châtiments sévères il doit s'adresser au gouverneur de la province ou au préfet de la ville%
% [14]
\footnote{Jean \fsc{IMBERT}, \emph{Le droit antique}, Que sais-je, 1961, p.93.} 
 : c'est la doctrine juridique qui va perdurer jusqu'au \siecle{20}. 

\item[529] Justinien décrète que deux amant adultères n'auront jamais le droit de s'épouser même si venait à décéder l'époux (ou les époux) qui leur faisait obstacle.

% 529 :
 Il décrète qu'une femme convaincue d'adultère sera condamnée à vivre dans un couvent de femmes%
% [15] 
\footnote{Jusqu'à la Révolution, et en fait jusqu'au \siecle{20}, les incarcérations de femmes se feront dans des monastères pour femmes ou dans des lieux inspirés de ce modèle.} 
jusqu'à sa mort ...si du moins son époux porte plainte, mais rien n'oblige ce dernier à le faire. Compte tenu du fait que les textes antérieurs (ceux de Constantin) donnaient à l'époux droit de vie et de mort mort c'est un adoucissement majeur. Il pourra se séparer de sa femme adultère, mais ne pourra pas se remarier. Cette règle de droit ne concerne que les chrétiens mais à cette date cela fait longtemps que tous les citoyens, à l'exception des juifs, ont reçu l'ordre de se faire chrétiens. Par conséquent s'il veut avoir des enfants légitimes, des héritiers, un époux bafoué n'a plus d'autre choix (en principe) que de se réconcilier avec sa femme et de reprendre la vie commune, ce qui eut été le comble de l'indécence ou de l'infamie deux siècles plus tôt. En dépit de l'infidélité passée de celle-ci il est même expressément invité à lui pardonner au bout d'un certain temps de réclusion dans un couvent pour femmes (2 ans au maximum ?). Autant dire que son intérêt n'est pas forcément de porter le cas de son épouse coupable devant les tribunaux. L'adultère féminin devient de plus en plus une affaire privée, même si le pouvoir civil continue et continuera jusqu'à la fin de l'ancien régime de prêter la main au mari pour soutenir son droit de correction marital. 

\item[533] Les \emph{institutes} de Justinien réaffirment la légitimité du mariage entre cousins, ou celui d'un veuf ou d'une veuve avec le frère ou la sœur de son conjoint décédé. Cela montre la résistance des autorités civiles à l'autorité morale de l'Église. Sur ce point précis l'Orient refusait la position intransigeante de l'église de Rome face à tout ce qui ressemble à l'inceste%
% [10]
\footnote{Cf. Jack \fsc{GOODY}, \emph{L'évolution de la famille et du mariage en Europe}, p. 66.}%
.

\item[534] Le Code de Justinien décrète que les enfants adultérins et incestueux n'ont aucun droit, au contraire des autres enfants illégitimes : il leur dénie tout droit à des aliments, bien qu'il ne soit pas interdit à leurs géniteurs de leur en donner. Ils ne peuvent en aucun cas être légitimés. Au nom de la sauvegarde de l'institution familiale cette décision retire à ces enfants les protections que leurs géniteurs pourraient vouloir leur donner. Ils sont réputés \emph{enfants trouvés}, et traités comme ces derniers par les institutions chargées de s'en occuper. 

 Par ailleurs le code de Justinien confirme \emph{l'interdiction de l'adrogation des enfants illégitimes}. Les « enfants du péché » (le péché des adultes contre l'institution familiale) sont désormais à écarter. Cela contraste avec la volonté de protection de tous les enfants qui se traduisait dans les autres décisions de l'époque, mais il est clair que l'objectif de ces lois était d'abord de prévenir la naissance de ces enfants. La suite de l'histoire montrera que cet objectif a été à peu près atteint en dépit d'un « reste » incompressible (seulement ... ou encore, un pour cent de naissances illégitimes dans diverses régions rurales françaises sous Louis~XIV). 

 Le code de Justinien confie les enfants trouvés à l'évêque du lieu de leur découverte. Il ordonne aux magistrats de s'en tenir au principe juridique que les enfants trouvés sont tous nés libres et non esclaves, même lorsque leur mère est une esclave : l'abandon donne donc aux enfants d'esclave une chance de vivre libres. 

 En l'absence d'enfants légitimes le code de Justinien autorise les enfants naturels nés d'un concubinage stable à hériter de leurs parents : il confirme ainsi le concubinage stable dans son statut de mariage à l'usage des pauvres et des humbles. 
\end{description}




\chapter{Entrée en scène des Barbares}


 L'occident de l'Empire romain était beaucoup plus pauvre et moins peuplé que son orient. Depuis la fin du \siecle{3}, deux empereurs se partageaient de concert la direction de cet ensemble, l'un en occident, dont la capitale se déplaçait en fonction des urgences militaires, l'autre en orient, à Constantinople. 

 Face aux problèmes de recrutement de ses armées, l'empereur d'occident sous-traitait à des tribus germaniques la défense de plusieurs secteurs de ses frontières. Rémunérées par l'octroi de terres, elles étaient intégrées dans le système de défense du \latin{limes} sous le commandement de leurs propres chefs. Malgré l'assistance de ces soldats de métier, la puissance romaine était de moins en moins capable de contenir les barbares vivant de l'autre côté du \latin{limes}. Dès la fin du \siecle{4}, les empereurs d'occident n'exerçaient plus leurs prérogatives régaliennes avec la détermination et la constance nécessaires, et Constantinople ne venait au secours de Rome que de manière ponctuelle et en fonction de ses propres intérêts. Le 31 décembre 406, les frontières de l'Empire d'Occident ont été débordées par de nouvelles tribus germaniques désireuses de profiter elles aussi des avantages de la romanité, et poussées par des peuples des steppes en expansion, les \emph{Huns} et leurs alliés.

 Désormais la faiblesse de l'Empire d'Occident permet aux Barbares avec qui il s'était associé, et à divers autres qu'il n'avait pas invités, de mettre l'empereur en dépendance, de coloniser sa haute fonction publique, et finalement de démembrer ses provinces en se taillant dans la chair de celles-ci des principautés à peu près indépendantes. Le dernier empereur d'occident est déposé en 476 et les insignes de sa fonction sont remis à l'empereur d'orient. 

 Constantinople va encore vivre \nombre{1000} ans sans rupture brutale avec l'histoire et la culture antiques, ce qui donnera à ses dirigeants la conviction de représenter la norme, la voie « orthodoxe », tandis que l'orient va peu à peu devenir prodigieusement exotique aux yeux des occidentaux.

 En occident des Barbares, Germains pour l'essentiel, vivaient désormais à côté des « Romains », avec qui ils ne pouvaient en principe se marier. Ils obéissaient à leurs propres coutumes (notamment en matière matrimoniale%
% [1]
\footnote{Jean-Pierre \fsc{POLY}, \emph{Le chemin des amours barbares, genèse médiévale de la sexualité européenne}, Perrin, 2003.}%
) et religions, et parlaient leurs propres langues. De même que les populations romaines étaient soumises au \emph{code Théodosien} (438), compilation du droit romain de l'antiquité tardive, de même les Barbares ont rédigé leurs propres codes : pour les Wisigoths \emph{Code d'Euric} (476) puis \emph{Loi Gombette} (502) (= loi de Gondebald : \latin{lex gundobada}) ; \emph{Loi Salique} (511) pour les Francs. Ces divers codes représentaient autant de compromis entre le droit romain et les coutumes des groupes concernés. Quant au \emph{Bréviaire d'Alaric} (506) il reprenait le droit romain en le résumant à destination des sujets de droit romain des rois Wisigoths.

 Pendant ce temps la vie continuait. Pendant ce temps les cités et les administrations romaines continuaient de fonctionner sans grands changements. Les nouveaux maîtres ne voulaient pas détruire l'empire mais s'y intégrer et leurs représentations n'étaient pas radicalement différentes de celles de leurs nouveaux sujets. Ils croyaient encore plus fortement qu'eux à un monde de castes et à la légitimité de la force. Ils appliquaient avec la même dureté les mêmes « lois » de la guerre,~etc. Ils étaient tout aussi esclavagistes. Le risque d'être malmené par les armées en guerre ou enlevé par une bande armée et de se retrouver sur un marché aux esclaves était toujours présent, mais était-il plus grand qu'à d'autres périodes difficiles de l'empire ? Avec les « invasions » certains circuits économiques se transformaient, mais si l'on en croit les recherches archéologiques, globalement l'économie ne semblait pas se porter plus mal qu'auparavant : les Barbares, « réfugiés économiques », s'en seraient allés plus loin si cela avait été le cas.

 Pour ce qui concerne notre sujet la chute de l'Empire Romain d'Occident est donc une date beaucoup moins significative que celle de la conversion de Constantin. En effet cette « chute » a été quelque chose de lent, de progressif : un processus très long, émaillé de catastrophes ponctuelles et de réparations successives, et dont le sens n'était pas lisible d'emblée. Les contemporains ne l'ont pas vécue comme une catastrophe sans contreparties : le vieux monde n'avait pas que des bons côtés. Au fur et à mesure que par pans entiers s'effondraient de vénérables institutions, d'autres façons de vivre devenaient imaginables, et ce n'était pas toujours pire qu'auparavant, même si jusqu'à la Révolution Française la \latin{pax romana} sera toujours idéalisée et décrite comme le modèle de gouvernement qu'il convient d'imiter. 

 Les populations d'origine romaine et de langue latine étaient officiellement catholiques depuis que les descendants de Constantin leur avaient ordonné de se détourner de leurs dieux civiques, pas franchi par la majeure partie de leurs élites cultivées et la plupart des citadins. Elles comprenaient aussi une minorité significative de juifs. Quant aux campagnes elles étaient encore en majeure partie païennes. Il faudra attendre le \siecle{6} pour que leur christianisation soit à peu près réalisée. Les élites romaines du pouvoir, de la culture et de la richesse se ralliaient progressivement à l’Église comme à la seule institution capable de faire pièce aux risques que les temps nouveaux faisaient courir à la romanité. Au \siecle{5} l'évêque, seul magistrat subsistant de l'ancienne grandeur romaine était en train de devenir pour longtemps le premier des membres de la Curie de la ville où était située sa cathédrale (la ville la plus grande du diocèse en général). Ceux-ci votaient pour lui, à côté des membres du clergé, en tant que représentants du peuple. Il était le plus puissant des patrons des « pauvres » \latin{(pauperes)}, des populations romaines désarmées et soumises au pouvoir des nouveaux souverains. 

 Presque tous les Barbares étaient païens, et sauf exception ceux qui ne l'étaient pas étaient \emph{ariens}. Mais il ne semble pas que leurs effectifs aient ordinairement représenté beaucoup plus que quelques pour cent de celui des populations romaines sur lesquelles ils avaient établi leur domination. Compte tenu du rapport des forces démographiques en présence et compte tenu de l'importance de la religion comme facteur d'unité politique, sans oublier le prestige de la culture et des institutions romaines, dont les clercs chrétiens étaient porteurs (et de plus en plus les uniques transmetteurs), l'un après l'autre les chefs barbares vont embrasser la religion de leurs nouveaux sujets (cf. Clovis), et entraîner progressivement leurs tribus à les suivre%
% [2]
\footnote{Cf. Bruno \fsc{DUMEZIL}, \emph{Les racines chrétiennes de l'Europe, conversion et liberté dans les royaumes barbares, \siecles{5}{8}}.}
. 



% 28.02.2015 :
% haut Moyen Âge
% _, --> ,
% ~etc.
% Antiquité
% ~\%


\chapter[Les sociétés du Bas-Empire et du haut Moyen Âge]{Les sociétés du Bas-Empire\\et du haut Moyen Âge}


 Nous ne pouvons lire que les textes et inscriptions qui nous sont parvenus, or l'écriture était à la fin de l'Antiquité et au début du Moyen Âge un privilège et une distinction. Compte tenu de la diminution progressive du nombre des laïcs cultivés, les clercs, et surtout les moines, en devenaient peu à peu les spécialistes. C'est par leur truchement, c'est à travers leurs yeux que nous sommes aujourd'hui condamnés à regarder leur monde. Beaucoup d'entre eux étaient eux-mêmes issus des familles de guerriers, de l'aristocratie de la naissance. Leur société était l'héritière de l'empire de Caracalla dans lequel, si tous les hommes libres étaient devenus citoyens romains, seuls les nobles \latin{(clarissimi)} jouissaient de la totalité des droits autrefois garantis par la citoyenneté. Ainsi les serfs et les esclaves ne pouvaient pas être ordonnés, sauf à être affranchis au préalable pour les délier de leur dépendance à leur maître. Que tous les baptisés (hommes et femmes, esclaves ou libres ...) aient la même valeur aux yeux de Dieu, ce qu'ils enseignaient, n'était pas une raison suffisante pour que l'égalité soit recherchée sur cette terre. Au contraire, ici-bas les hiérarchies leur paraissaient naturelles, nécessaires et crées par Dieu en vue du bien commun. Leur point de vue était conforté par les écrits de Paul de Tarse ou ceux d'Augustin.

 Dans la lignée de l'Antiquité grecque et romaine, et donc du mépris des hommes libres pour les tâches serviles, ils pensaient que l'activité intellectuelle avait plus de valeur que le travail manuel, juste bon pour ceux qui ne possédaient ni revenus fonciers ni savoirs, ce qui allait de pair en l'absence d'écoles gratuites. À leurs yeux étaient associés, sauf exception dûment soulignée, le \emph{sang vil}, la lâcheté et l'incapacité à tenir parole, le paganisme (religion des \latin{pagani}, des paysans) et la sorcellerie, la servilité morale et les \emph{tâches serviles}... Ils trouvaient naturel que coïncident le \emph{sang noble} et les \emph{tâches nobles}, telles que l'étude et le \emph{service divin} (prêtres, évêques, moines de chœur chantant les offices en latin), le sang des aristocrates et l'aptitude à prêter serment, à dire le vrai, à tenir sa parole, à s'engager par contrat : si l'on en croit les \emph{Vies de saints} écrites au haut Moyen Âge, rares étaient ceux d'entre ces derniers qui \emph{n'étaient pas} issus de haute noblesse. Les moines qui les rédigeaient étaient \emph{presque} incapables d'imaginer qu'un personnage digne d'être mis sur les autels puisse ne pas être né d'un puissant seigneur et d'une noble et pieuse dame. 

 Chez les Germains comme chez les Celtes, c'est la naissance qui déterminait la valeur. À leurs yeux la société reposait sur le \emph{sang}, c'est-à-dire les ascendants, la lignée, l'hérédité. Il existait quelques lignées nobles, descendantes en partie de l'aristocratie romaine, en partie des aristocraties barbares, et de plus en plus des deux à la fois, distinguées de toutes les autres, celles des multitudes de personnes au sang vil, sans parents dignes de mémoire. La société s'organisait en un système qui serait un jour théorisé (par des clercs) comme l'union de ceux qui prient (et qui prêchent et enseignent), de ceux qui combattent (et qui dirigent), et de ceux qui nourrissent tout le monde (ceux qui transpirent et œuvrent de leurs mains et qui paient taxes et dîmes). Ces derniers étaient d'abord les « vilains », ceux qui habitaient les \latin{villas}, c'est-à-dire les paysans : 95~\% de la population d'alors.

 Les plus humbles n'ont laissé de traces directes que pour les archéologues. On ne peut donc savoir quelles étaient leurs propres représentations. Jusqu'où avaient-ils la possibilité ne pas s'identifier à l'image que les savants de leur époque, tous clercs, avaient d'eux-mêmes ?

 Il faudra attendre le \siecle{12} pour que la croissance des villes, celle des populations et celle des économies permettent une renaissance des civilités, sous des formes inconnues de l'Antiquité, mais aussi brillantes dans certains domaines. 
 
 Les chrétiens apportaient une philosophie de l'histoire, une explication totale du monde et une morale pour tous les instants. Comme l'avait voulu Constantin, ils fournissaient une idéologie unificatrice à l'empire. Mais à la fin de l'Antiquité celle-ci était encore loin d'avoir imprégné la culture et les mœurs. Pour \fsc{DUBY} il faudra attendre le \siecle{12} pour qu'elle soit véritablement intériorisée par l'ensemble des populations. Pourtant dès le \siecle{6}, « chrétien » désignait une identité (une « ethnie » en langage médiatique actuel) au même titre que « Romain », et les deux identités tendaient à se confondre. 

 De son côté l'Empire romain influençait profondément les chrétiens, qui avaient calqué leur organisation territoriale sur lui, avec une hiérarchie religieuse parallèle à la hiérarchie civile. Le christianisme avec ses représentations entrait en résonance avec les conceptions des empereurs, de la même manière que le système impérial lui-même exprimait sans doute le \emph{style de communication} des gens de cette époque%
%[1]
\footnote{Peter \fsc{BROWN}, 1999}%
. Au \siecle{4}, dans les vastes basiliques offertes par Constantin, le nombre des participants, la structure hiérarchique de l'assemblée et le style des homélies, la minutie des rituels, la pompe, le décorum, les luminaires, l'encens, la musique et les chants, tout rappelait les splendeurs des cérémonies des temples romains, grecs ou égyptiens. Au même moment et pour encore un tout petit peu de temps ces derniers continuaient de déployer leurs fastes immémoriaux. 

 Les évêques étaient assez régulièrement issus des familles de sénateurs ou de chevaliers, qui fournissaient ses magistrats à l'Empire, élus par leur clergé et par les membres importants de leur église locale. Certains avaient été eux-mêmes de hauts fonctionnaires avant leur ordination (cf. Ambroise de Milan, ancien préfet). Les évêques parlaient comme les préfets, avec la même conscience de la grandeur de leur mission et de leur légitimité, et la même rhétorique particulière du Bas Empire. Le ton de leurs écrits était en consonance avec celui des mandements et rescrits impériaux. Ils enseignaient et admonestaient leurs ouailles, ils écrivaient et géraient leurs églises avec la même logique intellectuelle, le même esprit juridique, la même conscience professionnelle et le même sens de la grandeur de leur tâche et de leur fonction que les magistrats et fonctionnaires d'alors. Ils présidaient le culte chrétien comme leurs pères avaient présidé les sacrifices des religions civiques dans le cadre de leur \latin{cursus honorum}. 

 Pendant les siècles du haut Moyen Âge, en raison de l'effondrement du système d'enseignement public antique, provoqué en grande partie par la déchéance des villes désertées par l'essentiel de leurs citoyens et passées aux mains des rois barbares, ce sont les clercs qui ont tenté avec plus ou moins de succès de maintenir les traditions littéraires et administratives romaines. De plus en plus souvent ils sont devenus les seuls experts de l'écriture, de la littérature et de l'éloquence, capables d'occuper les emplois de lettrés, et à ce titre ils peuplaient les chancelleries des grands.
 
 % Le 10 mars 2015 :
% Moyen Âge
% ~etc.
% Antiquité
% Romain

% Le 02.03.2015 :
% ~\%
% ~etc.
% Antiquité
% Moyen Âge

% 28.02.2015 :
% haut Moyen Âge
% _, --> ,
% Antiquité
% ~etc.
% ~\%




\section{Le clergé chrétien}


 Les règles de recrutement et de discipline cléricale de l'Église se sont précisées au cours des premiers siècles%
% [1]
\footnote{Sources : Georges \fsc{MINOIS}, \emph{Les religieux en Bretagne sous l'Ancien Régime}, 1989. Léo \fsc{MOULIN}, \emph{La vie quotidienne des religieux au Moyen Âge, \siecles{10}{15}}, 1978. Michel \fsc{PARISSE}, \emph{Les nonnes au Moyen Âge}, 1983.}% 
. Dès le \siecle{4} ces règles reflètent l'état définitif de la doctrine permanente, qu'on retrouvera telle quelle et quasi inchangée dans sa formulation du début du \siecle{20} (droit canon de 1917). Selon les Décrétales du Pape Innocent I (401-417) il est interdit d'admettre au diaconat et à la prêtrise :
\begin{enumerate}
%  1°)
\item ceux qui ont épousé une femme non vierge ;
% 2°)
\item ceux qui ont épousé une veuve ;
% 3°)
\item ceux qui ont été mariés deux fois, quelles que soient les circonstances ;
% 4°)
\item ceux qui se sont fait soldats après leur baptême, qui ont accepté de toucher des armes dont certaines ont versé le sang, et surtout ceux qui ont accepté de verser le sang. Même si l'interdiction faite aux chrétiens d'être militaires a été levée par l'Église à partir du ralliement de Constantin (entre 313 et 315), le sang restait sacré, donc impur, et impur aussi celui qui le versait, même pour la bonne cause (chirurgiens) ;
% 5°)
\item ceux qui, magistrats, ont jugé ou plaidé dans des procès où ils ont requis ou prononcé la peine de mort (même motif que le cas précédent : le juge est condamné à faire couler le sang : infliger la \emph{question}, c'est-a-dire torturer, était alors considéré comme nécessaire pour découvrir la vérité, et donc inévitable; infliger des peines mineures comme le fouet; condamner à mort...). Ce n'était pas le risque de l'erreur judiciaire qui était en jeu, c'était encore une fois le contact avec le sang et le contact avec la mort ;
% 6°)
\item les pécheurs qui ont été condamnés à une pénitence (« pécheurs publics », nouveaux infâmes) ;
% 7°)
\item ceux qui ont donné des jeux publics \latin{(munera)}, où du sang (humain ou animal) a coulé ;
% 8°)
\item ceux qui ont exercé des sacerdoces païens, et qui ont donc eux-mêmes sacrifié aux dieux ;
% 9°)
\item ceux qui se sont mutilés eux-mêmes, ce qui vise surtout l'auto castration. Ces derniers ont à la fois versé leur propre sang et mutilé leur corps à l'instar des \latin{galles} (prêtres de Cybèle).
\end{enumerate}

 On peut comparer trait pour trait ces règles avec celles du Lévitique qui régissaient les lévites et les prêtres du Temple de Jérusalem. C'est la même logique. Dans les discussions sur ces sujets les textes de la Tora ont servi d'arguments décisifs. En effet l'imitation du clergé du Temple s'est faite au fil du temps de plus en plus consciente et volontaire. Et pourtant plus il se voulait identique au clergé du Temple, moins le clergé chrétien lui ressemblait ! L'exigence du célibat lui imprimait en effet une physionomie tout à fait inédite. 

 On a vu que la continence perpétuelle était exigée des diacres et prêtres dès les premiers siècles afin qu'ils soient toujours prêts à toucher les « choses sacrées » (vases et linges sacrés, offrandes, pain consacré,~etc.), non souillés par l'impureté rituelle produite par le coït. Ce qui est remarquable c'est que cet argumentaire a emporté l'adhésion. Pourtant l'organisation du service du Temple de Jérusalem montrait une voie de compromis évidente, le service par roulement. D'autre part la notion même de pureté et de souillure religieuse, qui ne se confond pas avec celle de faute morale \emph{(péché)}, avait été mise en question par le Christ lui-même. On peut en déduire que le refus du service par roulement était motivé par des raisons autrement impérieuses que la difficulté de mettre en place un tour de service. 

 Dès l'élection du remplaçant de l'apôtre Judas et l'institution des diacres, les apôtres avaient estimé que personne ne se donne à soi-même une mission%
% [2]
\footnote{Cf. selon le livre des \emph{Actes des Apôtres} les difficultés de Paul de Tarse pour faire admettre par les apôtres sa mission auto proclamée auprès des gentils et ses prétentions au titre d'apôtre.} 
ni ne la tient de sa naissance%
%[3]
\footnote{... de même que nul ne peut (en stricte doctrine) se dire chrétien par sa naissance : il faut que chaque enfant en passe par le baptême, comme le premier converti venu.}% 
, que c'est l'Église qui appelle, et Dieu à travers elle. C'est pourquoi la succession dans le même poste ecclésiastique du père au fils, de l'oncle au neveu, sans être interdite n'a jamais été reconnue comme un droit, au contraire du droit à hériter d'un « honneur », d'une terre ou d'une entreprise, et encore moins comme un modèle. Passés les premiers siècles elle a au contraire été vue comme une irrégularité grosse de dangers. 

 Si la haute administration de l'Empire romain tardif et des royaumes barbares qui lui ont succédé est devenue vers le \siecle{10} la noblesse héréditaire du Moyen Âge, c'est parce que ceux qui étaient nommés par les autorités civiles à un emploi public ont fini par obtenir le droit de désigner eux-mêmes leur successeur, ce qui signifie que « l'honneur » (responsabilités et biens servant à les rémunérer) qui leur avait été conféré par les souverains est entré dans leur patrimoine, à la faveur de l'affaiblissement de ces mêmes souverains, système d'où est sortie la \emph{féodalité}. Au même moment un clergé marié aurait eu les mêmes chances de devenir héréditaire, et le risque eut été grand de voir se constituer une caste sacerdotale à côté de la caste aristocratique, à la mode indienne ou hébraïque. On connaît d'ailleurs un certain nombre de grandes familles de l'Antiquité et du haut Moyen Âge dont des membres se sont succédé sur le même siège épiscopal pendant plusieurs générations : Sylvère, pape de 536 à 537, était le fils légitime d'Horsmidas, pape de 514 à 523 (né avant son ordination). D'autre part les souverains et autres puissants du \siecle{6} et des siècles suivants ont utilisé leur influence pour conférer l'épiscopat à des serviteurs laïcs afin de les récompenser pour leurs loyaux services, ou bien pour les neutraliser par cet « honneur » particulier, qui leur interdisait (en principe) tout retour aux armes. 

 Si les membres d'une échelle hiérarchique peuvent donner leur poste à un héritier c'est qu'ils en sont devenus propriétaires et c'est toujours au détriment du sommet de la hiérarchie, désormais obligé de composer avec une autre source de légitimité qu'elle-même. Inversement, c'est toujours pour défendre ou renforcer son autorité qu'un souverain refuse que soit limité son pouvoir de nommer et de démettre.

 Au contraire, du point de vue d'une institution, le célibat est idéal :
\begin{enumerate}
% 1°)
\item Un clerc célibataire est plus disponible puisqu'il n'a pas à plaire à sa femme, ni à s'occuper de ses enfants (cf. Paul de Tarse).
% 2°)
\item Un clerc célibataire et sans enfants a moins de besoins matériels qu'un clerc marié et donc il \emph{peut} coûter moins cher. La continence des clercs est d'abord économique en ce qu'il n'y a pas à constituer de dot pour les filles ni à établir les garçons ...
% 3°)
\item ... qui pourraient prétendre avoir des droits sur le poste de leur père.
% 4°)
\item N'ayant pas à craindre pour ses proches, ni à les établir dans la vie, un clerc sans attaches familiales est moins sensible aux pressions et séductions venant de la société civile.
% 5°)
\item Par ailleurs il serait inconvenant que des histoires de famille puissent interférer dans les affaires de l'Église.
% 6°)
\item Enfin une paroisse, un diocèse, un monastère ne sont pas des bâtiments ni des biens fonciers. Ces institutions sont des ensembles de fidèles, et pour l'Église aucun groupe de fidèles, c'est-à-dire d'âmes immortelles, ne peut appartenir à une personne, ou à une famille, à la façon dont la force de travail des serfs (mais non leurs âmes) appartenait à leur seigneur.
\end{enumerate} 

 Voilà pourquoi la doctrine ecclésiale a toujours voulu, malgré toutes les pesanteurs individuelles et collectives qui ont entrainé de nombreux écarts, que les clercs ne soient pas issus de familles de clercs, mais issus du monde des laïcs. Et voilà pour eux autant de raisons très concrètes d'attribuer une valeur spirituelle au célibat et à la continence perpétuelle, tandis que les religieux montraient par leur exemple que cet idéal n'était pas inatteignable. 

 Cela a eu des conséquences très importantes sur la société toute entière. En effet il s'est constitué en son sein une caste non héréditaire recrutée dans les autres castes, cultivant le savoir et la culture, et au sein de laquelle les carrières n'étaient pas déterminées par la naissance, même si ces deux idéaux, toujours poursuivis, n'ont jamais été totalement atteints. La culture cultivée dans les institutions d'Église souffrait de limitations certaines et l'héritage antique n'avait pas été transmis sans pertes. Tous les clercs n'étaient pas savants, et les plus compétents n'obtenaient pas toujours les promotions auxquelles leurs talents les auraient qualifiés. De même tous les princes de l'Église n'étaient pas à la hauteur de leur charge. Mais c'est au sein du corps des moines et des prêtres que se trouvaient les plus savants de leur époque, et pour ceux qui n'avaient pas les privilèges de la naissance c'est au sein de l'Église qu'ils avaient le plus de chances de promotions. 
 
 Joseph \fsc{MORSEL} voit dans cet élitisme ecclésiastique et le modèle qu'elle a fourni, profondément intériorisé, l'une des causes principales du développement ultérieur de l'Europe et de son avance sur les autres civilisations (in \emph{L'Histoire du Moyen Âge est un sport de combat}, texte publié au format pdf sur Internet à l'adresse \url{http://lamop.univ-paris1.fr/IMG/pdf/SportdecombatMac.pdf}).
 
 \section{Les religieux}
 Le mouvement monastique s'est développé depuis les premiers ermites qui ont fui le monde dès le \siecle{3} dans les déserts d'Égypte, et les premières veuves et vierges consacrées qui en ont fait autant à l'ombre des cathédrales, sous la protection des évêques. Il continuera de se développer à un rythme soutenu jusqu'au foisonnement de la fin du Moyen Âge. Il prouvait par sa floraison que la continence \emph{perpétuelle} était possible%
% [2]
\footnote{... même si elle doit parfois s'appuyer sur \emph{l'impuissance de famine}, cf. Aline \fsc{Rousselle}, 1998, p. 203 - 224}% 
, et cela non seulement pour les femmes, de qui depuis toujours on l'exigeait au gré des besoins de leur famille, mais aussi pour les hommes. Saint Augustin, évêque de la fin du \siecle{4} et du début du \siecle{5}, vivait en communauté avec ses collaborateurs immédiats, communauté d'où sortiront un jour les chapitres de chanoines présents dans toutes les cathédrales. Au même moment les hôpitaux s'organisaient dans l'esprit des monastères. Ils étaient construits comme des églises dans lesquelles seraient logés des malades et si l'on en croit le concile de Nicée leur personnel était recruté parmi les religieux.

 S'appuyant sur les lettres de Paul de Tarse et les paroles du Christ, l'Église défendait le droit des jeunes de consacrer volontairement et librement leur vie à Dieu, alors qu'ils étaient encore \emph{dans la main} de leur père. Dans ce cas elle défendait leur droit de recevoir leur part d'héritage sans pour autant suivre la voie prévue par leurs parents, part d'héritage sans laquelle leur liberté de choix serait restée formelle. Cela leur permettait de s'engager dans le monastère ou l'hôpital de leur choix en faisant don à leur communauté de leur part d'héritage%
% [3]
\footnote{C'est ainsi qu'était mis en pratique la proposition de donner tous leurs biens aux pauvres faite par le Christ à ceux qui voulaient choisir la perfection (parabole du « jeune homme riche ») : en effet leur nouvelle famille spirituelle n'était constituée que de membres qui avaient fait vœu de pauvreté.}% 
. On peut supposer que ce n'est pas par hasard qu'en 320 Constantin avait abrogé les lois d'Auguste qui exigeaient d'avoir engendré trois enfants et d'être marié pour recevoir les héritages venant de personnes éloignées, et qu'il avait posé des limites au droit des pères de déshériter un enfant. Contrainte par sa propre logique, et fidèle sur ce point au droit romain, l'Église plaidait pour le consentement mutuel des fiancés et contre l'idée que celui de leurs parents était nécessaire pour que leur mariage soit valide%
%[4]
\footnote{Là aussi elle allait contre l'autorité des pères. Cet enseignement-là restait en travers de la gorge de bien des pères, mais aussi des ecclésiastiques eux-mêmes pour autant qu'ils s'identifiaient aux intérêts temporels de leur famille d'origine, cf. les avanies subies par Abélard, alors qu'il était encore laïc et donc épousable, du fait de l'ecclésiastique qui était oncle et tuteur d'Héloïse.}% 
.

 Les revenus des monastères, des évêchés et des hôpitaux étaient fondés sur des propriétés, terres, domaines,~etc., provenant des dons et des legs. Grâce aux rentes sur la terre%
% [5]
\footnote{Ressentie de l'Antiquité à la fin du Moyen Âge (au moins) comme le seul bien qui ne fait jamais défaut, et dont les fruits permettent de survivre quelle que soit la catastrophe économique qui puisse arriver (Paul \fsc{Veyne}, \emph{La société romaine}, chapitre).} 
et les immeubles (en nature ou en argent) il était possible sans recourir à l'impôt de « fonder » (en principe une fois pour toutes) des emplois \emph{(bénéfices)} de clercs, des écoles, des hôpitaux, des monastères,~etc. Ce mode de financement était hérité de l'Antiquité pré chrétienne. C'était déjà celui des temples païens. S'ajoutaient à ces revenus des contributions régulières notamment les différentes \emph{dîmes} versées par les fidèles, d'abord volontaires, puis obligatoires. Ainsi les institutions ecclésiastiques étaient autonomes et auto-suffisantes, sans courir les risques du marché, ni dépendre étroitement de généreux donateurs ou des pouvoirs locaux. Ce système ne faisait peser aucune charge récurrente sur le budget de la puissance publique et donnait aux institutions un maximum de liberté face aux pressions des pouvoirs publics. 

 Jusqu'à la fin du Moyen Âge une part de presque tous les héritages était donnée aux pauvres (c'est-à-dire à leur protectrice officielle : l'Église) pour \emph{le salut de l'âme} des donateurs. Il existait déjà chez les anciens des fondations identiques auprès des temples païens. Quant aux barbares ils admettaient comme les Égyptiens, les Celtes et les Germains que chaque mort emporte dans son tombeau des biens pour l'au-delà, ce qui du point de vue des chrétiens ou des juifs était un signe de superstition. Cette part des biens du mourant qu'il comptait emporter avec lui (jusqu'à un tiers de sa fortune ?) l'Église lui proposait d'en faire meilleur usage, en l'investissant dans les \emph{œuvres pies} (pieuses). 

 Selon Raymond Goody il y avait un lien entre la défense par l'Église de la liberté de choix de vie des jeunes, celle du droit des jeunes à une part d'héritage même en cas de désaccord paternel, celle des chrétiens à faire des donations (notamment dans leur testament) et le financement des institutions religieuses qui fournissaient les lieux où chercher la perfection. Selon lui la nécessité de trouver des ressources pour faire vivre les paroisses, monastères et hôpitaux a exercé une pression déterminante sur la définition même des règles du droit de la famille. Elle aurait contribué à ce que le droit de l'Église mette des limites au droit des pères à imposer leur volonté à leurs enfants. Elle aurait aussi et surtout contribué à étendre les degrés de parenté interdisant les mariages. Même si cette thèse paraît un peu extrême, comme toute thèse qui attribue à une cause unique un mouvement observable sur plus de dix siècles, elle n'en contient pas moins une part de vérité significative. 

 En dehors du travail de leurs membres, qui exigeait lui-même un minimum d'outils de production et d'abord de terres, le financement des monastères reposait sur les \emph{dots} des postulants, notamment dans les monastères féminins qui ne pouvaient bénéficier comme les monastères d'hommes des honoraires de messes offertes pour le repos de l'âme des défunts. Au décès du religieux sa dot demeurait acquise au monastère (du moins tant qu'elle a consisté en un capital et non en une rente). Celui-ci avait donc des chances de voir grossir peu à peu son capital. Cela permettait (dans les meilleurs des cas) d'accepter les postulants sans le sou et de consacrer le superflu au service des pauvres et des malades.
 

\section{Le « mariage constantinien »}


 J'appelle « mariage constantinien » le mariage romain tel qu'il a été modifié par Constantin et ses successeurs pour l'accommoder aux conceptions chrétienne, même si les laïcs n'ont jamais totalement épousé les points de vue des clercs. Jusqu'à la Réforme Grégorienne (\siecle{11}), l'Église n'avait d'ailleurs pas le monopole du droit familial, et les autorités civiles ne se sentaient pas obligées d'appuyer toutes ses prétentions dans un domaine aussi critique pour la transmission du pouvoir. Les écarts entre le droit religieux (droit \emph{canon}) et les lois civiles n'ont jamais été nuls, pour ne pas parler de \emph{l'à peu près} avec lequel ces lois étaient respectées. Ce que j'appelle le mariage constantinien est donc un modèle qui n'a jamais été pleinement réalisé, et surtout pas sous Constantin. Pourtant il tendra peu à peu à s'incarner dans les pratiques et les représentations, et il ne sera peut-être jamais aussi bien respecté que durant les derniers siècles de notre ancien régime.

 Dans le mariage constantinien, plusieurs fonctions distinctes sont télescopées sur une seule personne : un époux est à la fois le détenteur des droits juridiques de son épouse (son curateur), son amant (point trop empressé si possible), le géniteur de ses enfants, le détenteur des droits de ces enfants mineurs, et le responsable de leur éducation, c'est-à-dire leur père légal. Symétriquement, une épouse est la seule femme capable de donner à son époux des enfants légitimes, des héritiers, quel que soit le nombre de ses concubines. Pour être légitime chaque enfant doit être l'enfant biologique de ses parents légaux (leur enfant « naturel » au sens antique du terme). Et par définition seuls les enfants légitimes ont droit à une part d'héritage et à succéder à leurs parents.

 Le Bas-Empire et le haut Moyen Âge continuent de reconnaître sans discussion la validité des concubinages stables monogames non incestueux, et la légitimité civile et religieuse des enfants qui en naissent : Justinien les autorise à hériter, mais il ne fait ainsi que rappeler une règle de droit ancienne. Dans la pratique du Bas-Empire, le concubinage monogame est une forme de mariage souvent (presque toujours ?) employée par les personnes qui ne possèdent pas de patrimoine significatif et ne voient donc pas la nécessité de s'unir en public et solennellement, ni de passer devant un notaire. Dans le même sens, Augustin d'Hippone enseigne qu'une concubine qui se veut fidèle à son concubin lui est mariée devant Dieu de manière aussi légitime qu'une épouse en titre.

 Pour l'Église, \emph{c'est le mariage qui fonde la famille, et non la naissance des enfants}, même si elle met l'accueil des enfants au premier rang des « fins du mariage ». De son point de vue, le mariage crée en effet \emph{dès sa célébration} une parenté nouvelle, \emph{une seule chair}, entre les époux, \emph{qu'ils soient féconds ou non}. Cette parenté « par alliance » a des effets directs et immédiats sur les membres des parentèles des époux (frères, sœurs,~etc.) : elle étend le cercle des partenaires qui leur sont désormais définitivement interdits, même si l'un des époux décède.

 Selon la doctrine chrétienne, identique sur ce point au droit romain, ce sont les époux qui s'unissent l'un à l'autre : cela implique qu'ils soient capables de discernement (âge suffisant, santé mentale) et libres de leur personne : célibataires ou veufs, non esclaves, non engagés par contrat dans une entreprise qui empêcherait la vie commune, à l'abri de toute pression, libres de tout vœux religieux, sexuellement aptes au mariage. L'Église a toujours soutenu contre les parents que les jeunes gens ont le pouvoir de se marier validement sans leur accord, même si elle admettait qu'en leur désobéissant ces jeunes gens les déliaient de leur devoir de les établir dans la vie. 

 Contrairement au droit romain l'Église en est progressivement venue à ne reconnaître la réalité juridique d'un mariage que lorsqu'il a été consommé, assez probablement parce que chez les barbares (cf. chapitre suivant), même christianisés, les unions se construisaient en plusieurs étapes séparées par de très longs intervalles, les premières étapes (dont les promesses de fiançailles) ayant parfois lieu alors que les futurs époux étaient encore de très jeunes enfants. Pour les besoins des procès en nullité de mariage il a fallu trouver un critère décisif dans cette progression, et c'est la consommation du mariage qui a été retenue. 

 Selon les évêques et théologiens chrétiens, le célibat non consacré est licite, mais chez les jeunes gens sans enfants, en bonne santé et disposant de moyens matériels suffisants, il est suspect d'égoïsme, de libertinage ou de désirs « contraires à la nature » (homosexualité notamment dont la mise en acte a toujours été condamnée moralement, même si elle n'a été semble-t-il que rarement sanctionnée). Quels que soient les préférences individuelles la copulation n'est légitime que dans l'état de mariage monogame, qui est le seul moyen acceptable de répondre à l'ordre divin (\emph{croissez et multipliez} de la Genèse). C'est donc l'état normal de tous ceux qui ne sont pas ordonnés à un ministère ou engagés dans la vie religieuse. Mais comme la fin première du mariage est la procréation d'enfants légitimes les remariages sont déconseillés (quoique autorisés) quand cette fin est à priori inatteignable étant donné l'âge ou l'état de santé des conjoints. 

 À partir du \siecle{4} dans l'empire romain, ce n'est plus d'abord et avant tout par la relation de pouvoir qu'il exerce sur les membres de sa maison que le père est juridiquement défini. En effet, il est soumis au devoir de \emph{piété}%
% [2] 
\footnote{La piété était l'affection réciproque et le respect mutuel entre les divers membres de la famille nucléaire, y compris le devoir d'assistance.} 
à l'égard de ses enfants au même titre qu'ils le sont à son égard, et autant qu'eux. L'accent se déplace sur sa responsabilité envers eux. 

D'autre part chez les Romains (mais pas chez tous les barbares alliés à Rome) en cas de décès du père, c'est la mère qui, à partir de 390, exerce la tutelle de ses enfants mineurs (et d'eux seuls) si elle a cinquante ans et plus, et du moins tant qu'elle ne se remarie pas, ce qui est le cas général, indépendamment même des réserves ecclésiastiques face au remariage des veuves dotées d'enfants. À partir de 390 une femme n'est plus considérée comme incapable par nature de représenter juridiquement une autre personne qu'elle-même. Le fait qu'à partir de ce moment elle puisse exercer, de droit, la tutelle de ses enfants (même si c'est sous le contrôle éventuel et plus ou moins étroit de la famille de son mari), manifeste que les droits et les devoirs dits « paternels » sont en réalité dès ce moment ceux du couple parental, même si tant qu'il vit c'est le mari qui représente le couple face au monde extérieur%
%[3]
\footnote{... et ce sera le cas jusqu'aux années 60 du \siecle{20}.}% 
. Au fil des siècles, la mise en pratique de ce principe a varié de pays en pays en fonction de nombreux facteurs. Il est probable que plus l'héritage était mince et la famille de petite importance, plus le droit de la veuve non remariée à exercer en toute liberté la tutelle de ses enfants mineurs lui était reconnu, et inversement. Ainsi dans les familles riches et puissantes, il pouvait y avoir tellement d'intérêts matériels ou politiques en jeu que la veuve n'avait pas forcément beaucoup d'impact sur l'éducation de celui de ses fils qui devait prendre la succession de son mari dans ses fonctions publiques. 

 Jusqu'à Constantin la fécondité de chaque femme mariée appartenait sans limites à son mari. Désormais elle ne lui appartient plus. Il n'est plus permis de se débarrasser des enfants non voulus par l'avortement ou par l'infanticide. Sauf indigence extrême il n'est pas non plus permis de s'en débarrasser par l'exposition ni la vente. Une femme n'a donc plus autant à craindre qu'auparavant qu'on ne l'oblige contre son gré à avorter ou à abandonner son nouveau-né. Mais sa fécondité ne lui appartient pas non plus. Pas plus que son mari elle n'a droit de vie ou de mort sur l'enfant qu'elle porte. Chacun des époux reconnaît aussi à l'autre un droit sur son propre corps. Les deux époux se doivent réciproquement fidélité : c'est un devoir \emph{moral} pour l'homme autant que pour sa femme, et même si ses propres infidélités ne sont pas sanctionnées par la loi tout est fait pour qu'il n'ait aucun intérêt à entretenir des maîtresses%
% [1]
\footnote{Cela ne l'empêche évidemment pas d'avoir des rapports avec des prostitué(e)s, rapports qui par nature ne s'inscrivent pas dans la durée.}% 
. Chacun des deux époux a l'obligation de satisfaire autant qu'il est en son pouvoir les désirs sexuels de l'autre, ce qui veut dire que l'épouse doit accepter les étreintes de son mari, quoi qu'elle puisse penser des risques de grossesse et de santé à quoi cela l'expose, et quels que soient ses propres désirs. Ceci dit la modération est prêchée aux maris, qui se voient prescrire la continence de nombreux jours par an. Le \emph{devoir conjugal} n'est par ailleurs exempt de faute morale que si aucun obstacle n'est mis à la fécondation (coit interrompu, pessaire, douche intime, sodomie,~etc.). Les seuls moyens de contrôle des naissances autorisés par l'Église sont le retard de l'âge au mariage, le célibat et la continence, périodique ou totale. 

 Il n'est plus possible en principe (mais ce principe a mis de nombreux siècles à s'imposer en dépit de la lutte constante de l'Église) de répudier une épouse présumée stérile (en cas de stérilité dans un couple, c'est celle de la femme qui est toujours suspectée en premier). Ou bien les hommes ont la chance de vivre un mariage fécond et de voir au moins l'un de leurs fils légitimes atteindre l'âge adulte pour leur succéder, ou bien ils doivent renoncer à tout héritier direct tant que vit leur épouse%
% [4]
\footnote{Pour les maris les moins patients il ne restait plus que le « divorce à la carolingienne », c'est-à-dire l'assassinat de l'épouse. Cela ne pouvait se faire que si les institutions policières étaient faibles et les parents de l'épouse moins puissants que ceux du mari. Plus l'État était déliquescent, plus il était possible, comme toujours, de prendre des libertés avec toutes les règles de droit, à la condition de disposer de la force.}% 
. Les couples stériles (dont le nombre n'était pas du tout négligeable jusqu'à l'avènement de la médecine moderne, 20~\% environ) et ceux dont aucun enfant n'a atteint vivant l'âge adulte, sont invités à consacrer aux bonnes œuvres, aux pauvres et à l'Église les ressources qu'ils auraient transmises à leurs héritiers s'ils en avaient eus.

 Tous les enfants nés hors mariage sont pénalisés. En principe il n'est plus possible pour un homme de se faire des héritiers sans se marier, même si la prise en charge d'\latin{alumnii} et leur installation dans l'existence reste une bonne œuvre. La \emph{légitimation par mariage subséquent} est désormais la seule exception de plein droit%
% [5] 
\footnote{... jusqu'au \siecle{20}. Les enfants irréguliers légitimés par les empereurs, les rois ou les papes, ne l'étaient pas de plein droit mais à la faveur d'une grâce, qui pouvait toujours être refusée sans justification, et n'allait pas sans contreparties coûteuses.} 
à la pénalisation des enfants nés hors mariage, et ses conditions sont strictes. Chacun, quelque puissant qu'il soit, doit savoir que s'il a l'imprudence de faire un enfant hors mariage ou dans un mariage contesté par son curé, ou par son évêque, ou par son seigneur, par le roi ou par sa propre parentèle, il ne pourra pas le faire reconnaître comme un de ses héritiers sans combat ou sans procès. Cet enfant ne pourra sans doute pas lui succéder. L'exhérédation totale ou partielle des enfants illégitimes est restée jusqu'à la fin du \siecle{20} le premier frein apporté au désir des hommes de se procurer une descendance ailleurs qu'avec leur épouse légitime, même si d'innombrables exemples montrent que cette règle a mis des siècles à s'imposer.

 Et la perspective de se retrouver avec un enfant à charge, seule, sans aucun espoir d'une légitimation (ni même d'une aide significative venant du père de l'enfant lorsqu'il était déjà marié puisque aucune donation au-delà des frais d'éducation n'était plus autorisée depuis Constantin) a été un obstacle majeur à l'exercice d'une sexualité féminine en dehors du mariage ou avant le mariage. 

 Mais les épouses savent aussi qu'il est devenu, sinon impossible, du moins difficile de les chasser de leur maison ou de leur imposer de cohabiter avec une concubine%
% [6]
\footnote{... même si pour ceux dont la puissance excède de beaucoup celle du commun des mortels, aristocrates, rois, la question peut se présenter différemment, et si les amours ancillaires sont de tous les temps.}% 
. Elles sont à peu près assurées que les infidélités de leur époux n'entraîneront de conséquences graves ni sur elles, ni sur leurs enfants, ni sur le futur héritage de ceux-ci. Tout au plus des « aliments » devront-ils être versés aux enfants nés de leurs maîtresses, mais cela ne portera que sur d'assez petites sommes et seulement jusqu'à ce qu'ils soient mis au travail : 8-10 ans. Il n'est plus question de financer leur établissement dans la vie. 





\section{Familles de chair}


 À partir de Constantin les lois de l'Empire, puis celles des royaumes qui en Occident ont repris sa succession, se sont lentement alignées sur les conceptions chrétiennes du mariage et de la génération
\footnote{Cf. Georges \fsc{DUBY}, \emph{Le chevalier, la femme et le prêtre}, 1981.}% 
. Mais dans le même temps la vie familiale à la romaine était également mise à mal par les « barbares ». Ceux-ci ont introduit des pratiques différentes, principalement germaines
\footnote{Jean-Pierre \fsc{POLY}, \emph{Le chemin des amours barbares, Genèse médiévale de la sexualité européenne}, 2003.}% 
, sur les territoires de l'ancien Empire romain d'Occident. Le haut Moyen Âge est un temps de conflits, de coexistence et de compromis entre les droits et coutumes des royaumes « barbares » et le droit romain%
%[3]
\footnote{Cf. Pierre \fsc{PETOT}, \emph{La famille}, 1992.}% 
. On constate l'effacement progressif des traditions juridiques romaines, compensé dans une large mesure par l'élaboration (ou la résurgence) de pratiques non romaines, dites \emph{coutumières}, propres à chaque lieu et caractérisées d'abord par une très grande variété%
%[4]
\footnote{Mais lorsqu'il s'agira à partir du \siecle{12} de reconstruire un droit unifié et cohérent, à la fois dans le domaine civil \emph{(droit civil)} et dans le domaine religieux \emph{(droit Canon)}, c'est au \emph{Code de Justinien}, publié en 529 et 534, que les juristes savants vont se référer.}% 
. 

 Toutes ces réserves étant faites, et malgré une infinité de particularités tenant aux lieux et aux temps, les lignes de force du système articulant en \emph{chrétienté} les familles, les autorités civiles et les institutions d'Église sont demeurées les mêmes au-delà du Moyen Âge%
% [5]
\footnote{... et même au-delà, jusqu'à la Révolution Française, même si des évolutions très significatives ont eu lieu à partir de la Renaissance et des Réformes protestantes et catholiques. Le paradoxe c'est même que c'est aux \crmieme{17} et \crmieme{18} siècles que les familles se sont le plus étroitement conformées aux principes chrétiens, sous la garde conjointe, vigilante et de plus en plus efficace, des autorités religieuses et civiles.}% 
. Sur la délimitation de cette période de l'histoire et en ce qui concerne mon sujet je ne peux que constater que celle qui convient le mieux est celle du \emph{long Moyen Âge} de Jacques \fsc{LE GOFF} (\emph{Faut-il vraiment découper l'histoire en tranches ?} 2013). À bien des points de vue le Moyen Âge ne s'est achevé qu'avec la Révolution Française, même si la deuxième moitié du \siecle{18} (à partir d'un tournant que l'on peut situer vers 1760/1765) participait déjà du siècle suivant, notamment sur le plan des idées, avec le mouvement européen des \emph{Lumières}, mais aussi sur le plan des institutions, notamment dans le domaine administratif (cf. Pierre Legendre). Pour schématiser on pourrait dire qu'il y a une unité dans la période qui va de Constantin jusqu'à l'Encyclopédie (non comprise). 

\subsection{Contrôle de la sexualité et indissolubilté des unions : Saint-Augustin}

Les chrétiens sont-ils coupables, comme on le croit souvent, d'avoir diabolisé les plaisirs de la chair ? La réponse à cette question n'est pas simple. D'abord il faut dire si l'on en croit Paul Veyne qu'ils \emph{n'ont rien réprimé du tout, c'était déjà fait}. Le monde patriarcal des cités antiques n'était en rien un monde de liberté sexuelle, sauf pour les hommes libres, et encore. Puis les philosophes stoïciens étaient passés par là pour exiger des hommes libres eux-mêmes qu'ils orientent leurs désirs vers la seule procréation. Quant aux médecins ils allaient eux aussi dans le sens d'une grande modération dans l'activité sexuelle. 

 Ceci étant dit il est vrai que les théologiens chrétiens des premiers siècles ont longtemps été tentés par les thèses \emph{encratites}%
% [6]
\footnote{Selon \emph{Encyclopedia universalis} : Encratite est un "...\emph{terme signifiant « les continents » (du grec \emph{enkratès}) et désignant plusieurs sectes hérétiques de l'Église ancienne qui prônaient un rigorisme moral radical (interdiction du mariage, abstention de viande et de vin) fondé sur une condamnation de la matière et du corps considérés comme les œuvres d'un démiurge distinct du Dieu suprême. Tatien, d'abord disciple de Justin, à Rome, est traditionnellement tenu pour le fondateur, vers 170, de cette secte ascétique des encratites, probablement dans la région d'Édesse. L'encratisme fut alors proscrit sous ses diverses formes par de nombreux décrets de Théodose I\ier, à la fin du \siecle{4}, et de Théodose II, en 428.
La sévérité des mesures impériales suffirait à témoigner de l'importance de la secte à cette époque. L'encratisme s'est alors confondu avec le manichéisme et a trouvé des prolongements chez les Messaliens et les Bogomiles (et les Cathares). Le rigorisme que pratiquaient ses adeptes se voulait une négation de l'œuvre du \emph{démiurge} (dieu mauvais opposé au dieu bon). Les fondements doctrinaux de la secte consistaient dans le rejet de certaines parties des Écritures, en particulier de l'Ancien Testament, et dans un recours à des textes de la littérature apocryphe présentant des tendances ascétiques très marquées. Certaines positions doctrinales et liturgiques découlaient généralement de la conception encratiste de la création et de la matière : négation du salut d'Adam (Tatien), négation de la résurrection de la chair, docétisme en christologie, utilisation d'eau à la place du vin pour célébrer l'eucharistie. La ligne de démarcation entre l'encratisme et le gnosticisme est difficile à tracer : ce dernier est dans une large mesure marqué par un courant rigoriste, et l'encratisme semble avoir accueilli des spéculations d'origine gnostique}."
Richard \fsc{GOULET}}%
, proches de celles des manichéens qui soutenaient que la matière est mauvaise par nature, que l'âme préexiste au corps, et qu'avec la conception elle chute dans un monde matériel et charnel, lieu de l'esprit du mal
\footnote{... ce qui paradoxalement pouvait conduire les adeptes de ces doctrines à une licence effrénée puisque rien sous le ciel n'avait plus d'importance : « méprises et fais ce que tu veux ».}% 
. Le plus emblématique des théologiens encratites, Tatien (deuxième siècle après J.-C.), considéré comme hérétique par divers \emph{Pères de l'Église}, rejetait le mariage et condamnait l'usage de la viande et du vin comme de tout autre plaisir de la chair. Préconisant l'eau pour célébrer l'eucharistie à la place du vin, il recommandait de se garder de tout acte sexuel, et de ne pas faire d'enfants pour ne pas prolonger l'existence d'un monde qu'il jugeait mauvais.  Dans le même ordre d'idées, selon Robert Markus%
% [8] 
\footnote{Robert \fsc{MARKUS}, \emph{Au risque du christianisme, l'émergence du modèle chrétien (\siecles{4}{6})}, Cambridge University Press, 1990, réédition en Français, Presses Universitaires de Lyon, 2012.} 
divers auteurs du \siecle{4} (comme Jérôme, Ambroise ou Grégoire de Nysse..) pensaient qu'Adam et Ève \emph{avaient été créés sans sexe, avec une "innocence" que certains comparaient à l'innocence des enfants, et que s'ils n'avaient pas péché, les rapports sexuels n'auraient pas été requis pour reproduire et multiplier la race humaine}.

 Avant sa conversion à 32 ans, en 386, Saint-Augustin avait longtemps adhéré aux thèses des manichéens. Au fil des longues années durant lesquelles il a élaboré son oeuvre théologique (il a été évêque de 395 à 430) il a évolué jusqu'à s'opposer à elles avec autant de fermeté que de subtilité. Son contemporain Pélage (vers 350-420) lui aussi s'opposait au pessimisme manichéen, mais il enseignait que chacun peut atteindre à la sainteté par ses propres forces. Sa doctrine valorisait donc la volonté individuelle et minimisait l'utilité de la grâce divine. C'est dans le cadre de la polémique provoquée par cette thèse (condamnée en 418 par le seizième concile de Carthage) qu'Augustin a mis en forme sa doctrine du \emph{« péché originel »} en se fondant sur les versets de la Genèse qui racontent le \emph{péché d'Adam} et ses conséquences. Il lui fallait rendre compte du fait que la propension au péché, c'est-à-dire à la transgression des lois divines, est présente chez tous les hommes dès leur naissance : tous sont tentés de se donner à eux-même leurs propres règles et de se rebeller contre Dieu. Pour expliquer comment se propage de génération en génération cette inclination pour le mal il en est venu à faire de la reproduction humaine le \emph{lieu} de sa transmission, ce qui implicitement ou explicitement jetait le soupçon sur celle-ci. Pourtant dans ses dernières oeuvres il soutenait que l'union sexuelle et la reproduction ont d'emblée fait partie du plan de Dieu qui a créé les humains hommes et femmes, et non unisexes ou asexués, ce qui le conduisait à écrire que : "\emph{Je ne vois pas pourquoi il ne devrait pas y avoir de mariage honorable au Paradis}." Et il concluait : « \emph{Ce n'est pas la chair corruptible qui a rendu l'âme pécheresse, c'est l'âme pécheresse qui a rendu la chair corruptible. »}\footnote{Selon Robert Markus :  «\emph{ Ce qui devait être expliqué n'était pas l'existence de la sexualité, mais plutôt son mode de fonctionnement et l'impact du péché d'Adam sur la sexualité de ses descendants. » « Les problèmes que posait la sexualité n'étaient ni plus ni moins les mêmes que ceux que posait l'homme. »  « La tension que décrivait l'enseignement manichéen, entre deux natures différentes dans un conflit permanent, était maintenant transposée en terme de conflit interne avec soi-même... Ce qui est répréhensible et honteux dans la sexualité n'est pas son existence même, mais sa tendance à être hors de tout contrôle et à échapper à la raison... un tel constat impliquait une réhabilitation de la chair... Et au bout du compte, cela impliquait aussi une réhabilitation du mariage. »}} 
 
  A ses yeux la concupiscence charnelle était un mal moral, parce qu'elle tendait au péché, mais le mariage chrétien en faisait un bon usage. Dans ses ouvrages \emph{De  bono viduitatis} et \emph{De bono conjugali} Il mettait sans ambigüité la continence et la virginité au-dessus du mariage : « \emph{Nous appuyant donc sur la foi et sur la saine doctrine des Ecritures, nous disons que le mariage n'est point un péché, et cependant qu'il est un état moins parfait, non seulement que celui de la virginité, mais même que celui de la viduité} (l'état des veufs continents). \emph{Nous disons que la nécessité présente pour les époux, sans leur ôter le droit à la vie éternelle, les prive, par le fait même, de cette gloire par excellence réservée à la chasteté perpétuelle. Nous disons que dans cette vie le mariage n'est utile que pour ceux à qui la continence est impossible.} ». Mais il ne donnait pas de valeur à la virginité ou à la continence en elles-mêmes : « \emph{Ce que nous louons dans les vierges, ce n'est pas leur virginité même, c'est leur consécration à Dieu dans les exercices d'une pieuse continence.} » « \emph{...Cette proposition ne fait que confirmer celle qui établit pour les vierges une sainteté plus grande que pour les épouses, sainteté qui obtiendra une récompense proportionnée à son degré de mérite. La raison en est que la virginité permet de tourner vers Dieu toutes ses pensées. En effet, la femme fidèle, tout en observant les lois de la pudeur conjugale, ne peut pas ne penser qu'au Seigneur ; sa perfection est donc moindre, puisqu'elle a aussi les pensées du monde en cherchant à plaire à son mari. C'est d'elle que l'Apôtre a parlé en disant que le mariage lui impose la nécessité de penser aux choses du monde et de chercher à plaire à son époux.} »Et il affirmait la valeur intrinsèque du mariage pour ceux qui ne pouvaient être continents sans \emph{brûler} : « \emph{Toutefois j'affirme que le mariage est bon, non pas précisément parce qu'il produit une postérité, mais parce qu'il la produit dans l'honnêteté, dans le droit, dans la pudeur et pour le bien de la société ; parce qu'il sert à donner aux enfants une éducation commune, salutaire et constante ; parce qu'enfin les époux s'y gardent la fidélité et ne profanent point le sacrement.} »
  
Pour lui l'importance de la sexualité dans le plan de Dieu et les risques qu'elle faisait encourir impliquaient une ascèse à laquelle étaient conviés les époux au même titre que les religieux, ascèse aussi méritoire pour les uns que pour les autres, et au sérieux de laquelle contribuait l'indissolubilité du lien contracté entre les deux époux (ce à quoi il attribue un caractère sacrementel), et le refus de tout remariage quelles que soient les circonstances, hors décès de l'ex-conjoint. 

La doctrine augustinienne a fixé la doctrine de l'Eglise de Rome pour un millénaire au moins, mais elle n'a pas empêché les moines du haut moyen-âge, c'est-à-dire l'essentiel des intellectuels de leur temps, de dénigrer la féminité et l'exercice de la sexualité (de toutes les sexualités), \footnote{Il faudra arriver au \siecle{12} pour que ces points de vue soient en partie remplacés par une exaltation du mariage comme état de vie chrétien. }. Quant aux laïcs ils choisissaient parmi tous ces principes et toutes ces règles celles qui leur paraissaient les plus raisonnables ou les moins intenables. Au sein des sociétés chrétiennes il existait donc une tension permanente
\footnote{Jacques \fsc{ROSSIAUD}, \emph{Sexualités au Moyen Âge}, Éditions Jean-Paul Gisserot, Paris, 2012.}. Le remariage après divorce, du vivant du premier conjoint, ne semble avoir été en droit totalement éradiqué d'Occident qu'après les réformes Grégoriennes du milieu du Moyen Âge
\footnote{Cf. Georges \fsc{DUBY}, \emph{Le chevalier, la femme et le prêtre}, 1981.}% 
. Il a longtemps été reconnu comme valide dans ses effets par les autorités civiles, notamment en ce qui concernait la légitimité des enfants à naître, alors même qu'il était sanctionné comme une faute par l'Église (excommunication, pénitences publiques...) ou par les autorités civiles elles-mêmes (amendes, exil, confiscation de biens...).  

 Si Constantin avait ordonné que l'adultère d'une femme avec un esclave soit puni de la mort des deux complices, il avait autorisé le mari d'une femme adultère, entremetteuse ou empoisonneuse à la répudier tout en conservant sa dot, et à se remarier. À défaut de condamnation plus grave l'épouse adultère était reléguée dans une île. Dans les autres cas une femme répudiée conservait sa dot, et si son ex-époux se remariait elle pouvait « envahir » sa maison et prendre possession de la dot de la nouvelle élue. Une épouse pouvait elle aussi répudier un époux coupable d'homicide, d'empoisonnement, de violation de sépulture, et s'en aller avec sa dot. Elle la perdait dans les autres cas où elle prenait l'initiative de la séparation. L'empereur d'occident Honorius a fixé pour chaque époux trois paliers
\footnote{J.-P.~\fsc{POLY}, \emph{Le chemin des amours barbares}, p. 42.}% 
: si le mari répudie sa femme pour « crime grave » (c'est-à-dire les motifs précisés par Constantin) il garde sa dot et il peut se remarier; s'il la répudie pour « faute contre les mœurs » il reprend la donation qu'il lui a faite en l'épousant, mais doit rendre sa dot, et attendre deux ans pour se remarier; s'il la répudie pour d'autres motifs il perd dot et donation, et il ne peut plus se remarier. L'épouse peut de même quitter son mari pour « cause grave » (toujours les motifs de Constantin) et se remarier après un délai de cinq ans\footnote{Il faudra que les tribunaux de l'Église obtiennent vers le \crm{10}\ieme{} ou \siecle{11} le monopole sur les affaires concernant le mariage pour que l'adultère cesse d'autoriser le divorce et le remariage (même si ce n'était pas si simple ni sans notion de faute), et pour que le vieux mot latin \latin{divortium} prenne le sens de séparation sans droit au remariage du vivant de l'autre conjoint, sens qu'il a gardé jusqu'à la Révolution.}. 
 
  À l'occasion du sac de Rome de 410 par Alaric et ses wisigoths,  et des nombreux viols qui l'ont accompagné, Augustin reprend à son compte l'idée, banale à son époque comme à la nôtre, du moins sur le plan rationnel, que le viol ne « souille » pas la victime, mais seulement son auteur. Il en tire la conclusion qu'il n'est pas question que la victime soit punie pour un acte auquel elle n'a pas consenti : "\emph{La sainteté du corps ne consiste pas à préserver nos membres de toute altération et de tout contact... Ainsi donc, tant que l'âme garde ce ferme propos qui fait la sainteté du corps, la brutalité d'une convoitise étrangère ne saurait ôter au corps le caractère sacré que lui imprime une continence persévérante... Nous soutenons que lorsqu'une femme, décidée à rester chaste, est victime d'un viol sans aucun consentement de sa volonté, il n'y a de coupable que l'oppresseur... }" 
S'opposant aux usages valorisés par l'Antiquité en pareille situation il défend les Romaines victimes de viol qui ont choisi de ne pas se suicider : \emph{"Elles ont voulu vivre, pour ne point venger sur elles le crime d'autrui, pour ne point commettre un crime de plus, pour ne point ajouter l'homicide \emph{[d'elles-même]} à l'adultère; c'est en elles-mêmes qu'elles possèdent l'honneur de la chasteté, dans le témoignage de leur conscience ; devant Dieu, il leur suffit d'être assurées qu'elles ne pouvaient rien faire de plus sans mal faire, résolues avant tout à ne pas s'écarter de la loi de Dieu, au risque même de n'éviter qu'à grand-peine les soupçons blessants de l'humaine malignité"} (\emph{La Cité de Dieu}, livre 1, chapitres 18 et 19). C'est de manière paradoxale qu'Augustin s'y prend pour défendre les victimes de viol : si la honte de l'avoir subi sans pouvoir l'empêcher est de tous les temps, honte qui n'est pas en soi liée à un sentiment de culpabilité, il qualifie néanmoins cette honte de faiblesse, de sentiment compréhensible mais non fondé en raison, et contre lequel il convient de lutter. Ce faisant il reprend les argumentations traditionnelles, mais il leur ajoute l'interdiction du suicide, que celui-ci soit de honte, de protestation ou de désespoir. 
Le suicide a toujours été condamné par l'Église, comme il l'était par la Tora\footnote{ \emph{"...comme un péché grave, sauf chez les « fous » ou les victimes d'un « grand chagrin »"} selon le \emph{premier concile de Braga} qui s'est tenu vers 561. Il s'agissait alors pour l'Église de marquer une différence avec la mentalité héritée de la civilisation romaine qui voyait dans le suicide une mort comme une autre pour le désespéré et une voie honorable, un moyen de rachat pour le criminel. (Wikipédia). }. En déniant que le suicide soit une réaction acceptable à la détresse et à la douleur morale éprouvées par les victimes directes de viol, Augustin posait sur leurs épaules un fardeau qui a pu être insupportable à certaines. Mais en leur faisant un devoir moral de survivre, il déniait aussi aux victimes collatérales du viol (époux, enfants, parentèles, voisins) le droit moral de les tuer ou de les pousser au suicide pour apaiser leur propre honte de n'avoir pas été capables, eux non plus, d'empêcher le forfait.
 


\subsection{Phobie de l'inceste}

 Même si les mariages entre cousins germains, et entre nièce et oncle paternel, avaient fini par être autorisés pendant un temps sous l'empire, l'interdit de l'inceste était ressenti avec force à Rome. L'Église partageait cette horreur de l'inceste : elle avait même choisi de comprendre le mot \latin{Porneia} comme désignant exclusivement les unions incestueuses, considérant que la seule cause acceptable de nullité des mariages était la proximité excessive des époux. L'un de ses objectifs était d'écarter du sein des parentèles toute expression des désirs sexuels (hors couples mariés), avec les rivalités, jalousies et rancœurs qui les accompagnent, pour donner toute la place à la seule fraternité et à la \latin{caritas}%
% [10]
\footnote{\latin{Caritas} = amour désexualisé : souci du bien de l'autre, amour de l'autre pour lui-même (même s'il n'a rien d'aimable). Il est souvent rendu par « charité » (dérivé direct de \latin{caritas}), mot où nous ne percevons plus aujourd'hui beaucoup d'amour.}% 
. L'autre objectif était de renforcer le tissu social. Augustin d'Hippone formule ainsi sa pensée : \emph{L'union du mâle et de la femelle, pour autant qu'elle relève du genre humain, est une sorte de pépinière de charité. \emph{[...]} Une très juste raison de charité%
%[12] 
\footnote{\latin{caritas}}
invita les hommes \emph{[...]} à multiplier leurs liens de parenté ; un seul homme ne devait pas en concentrer trop en lui-même, il fallait les répartir entre des sujets différents ; ainsi leur grand nombre contribuerait à préserver plus efficacement les liens de la vie sociale. Père et beau-père sont, en effet, les noms de deux liens de parenté. Que chacun ait un homme pour père et un autre pour beau-père, la charité s'étend sur un plus grand nombre \emph{[...au lieu qu']} un seul homme eût été, pour ses enfants frères et sœurs mariés entre eux, père, beau-père et oncle \emph{[...]}, autres pour le même homme seront alors la sœur, l'épouse, la cousine ; autres le père, l'oncle, le beau-père ; autres la mère, la tante, la belle-mère. Ainsi, loin de se restreindre à un cercle étroit, le lien social s'étendra plus largement et sur plus de têtes par des alliances multiples%
%[13]
\footnote{Livre XV de \emph{la Cité de Dieu}, d'après la traduction de G.~\fsc{COMBES}.}% 
.}

 Plus l'on étend le périmètre de l'inceste plus il faut aller loin de sa famille de naissance pour trouver un conjoint. Cela diminue le risque que les descendants d'une personne (un homme dans le système patriarcal) ne deviennent si puissants, au moyen d'une endogamie stricte de sa descendance, qu'ils puissent menacer le reste de la société, n'ayant pas à composer entre des allégeances multiples. On peut à contrario évoquer les observations de Germaine \fsc{TILLON} dans \emph{le harem et les cousins}, 1966, et l'opposition qu'elle fait entre la « république des beaux-frères » et la « république des cousins ». L'Église s'opposait ainsi aux pratiques orientales, qui privilégiaient le mariage entre cousins (et même entre frères et sœurs en Égypte), comme aux coutumes germaniques qui favorisaient les unions préférentielles entre familles aux alliances redoublées de génération en génération%
% [11]
\footnote{Jean-Pierre \fsc{POLY}, \emph{Le chemin des amours barbares, Genèse médiévale de la sexualité européenne}, Perrin, 2003.}% 
. On ne peut pas dire qu'elle choisissait pour autant un système de parenté contre les autres, même si elle posait le couple entouré de ses enfants, la {\emph{sainte famille}}, au centre de ses préoccupations. Elle mettait seulement une limite contraignante aux systèmes familiaux qui cherchaient à se fermer sur eux-mêmes.

 Au tournant entre le \crmieme{4} et le \siecle{5} les empereurs Théodose, Arcadius et Honorius, ont tenté d'interdire le mariage entre cousins germains, mais ces interdits ont été levés quelques années plus tard par les empereurs (d'Orient) suivants, même si en accord avec Ambroise de Milan et l'évêque de Rome, Augustin plaidait pour cet interdit : \emph{Qui peut douter qu'il ne soit aujourd'hui plus honnête d'interdire le mariage même entre cousins germains ?} Il arguait de sa proximité excessive avec l'inceste fraternel, et du fait que même si les lois de l'Empire l'avaient effectivement autorisé, la coutume romaine n'y était pas favorable. Mais il n'en était pas de même en Orient, où le mariage entre cousins germains était traditionnellement tenu pour idéal, trop inscrit dans la culture pour que l'argumentation d'Augustin puisse y être entendue. En Occident la position d'Augustin, reprise siècle après siècle par l'église, finira néanmoins par triompher.

 À partir du \siecle{7} et surtout du \crmieme{11} au \crmieme{13} en Occident, l'Église entend la notion d'inceste de manière de plus en plus extensive, jusqu'au septième degré, comme le faisait le droit romain ancien. Par dessus le marché à partir du \siecle{7}, elle s'est ralliée progressivement, et non sans résistances même en son sein, à un mode de calcul de ces degrés qui excluait tous les descendants \emph{des arrière-grands-parents des arrière-grands-parents} du sujet concerné%
% [17]
\footnote{Cf. \emph{Histoire du droit civil}, Jean-Philippe \fsc{LEVY} et André \fsc{CASTALDO}, p. 93-95.}% 
, ce qui multipliait de façon exponentielle le nombre des personnes interdites, du moins dans les familles qui prétendaient connaître leurs ancêtres aussi loin dans le passé, celles dont la légitimité reposait sur leur ascendance. Les humbles n'avaient pas une telle prétention, et on n'attachait pas autant d'importance aux irrégularités formelles de leurs unions. En ce qui les concernait il suffisait que l'interdit porte sur toutes les personnes ressenties par eux comme faisant partie de leur parenté. 

 Non seulement les évêques et théologiens d'Occident y ont ajouté toutes les parentés par alliance, y compris beaux-frères et belles-sœurs, mais ils y ont aussi adjoint la \emph{parenté spirituelle} qui liait les parrains et marraines d'un même enfant, et la parentèle de ceux-ci, sans compter les \emph{parentés} illicites nées des rencontres extra conjugales. La phobie de l'inceste a donc conduit à des extrêmes absurdes qui créaient mécaniquement des situations impossibles dans les communautés étroites où les personnes se déplaçaient fort peu en dehors des familles aristocratiques, et où en l'absence de registres d'état-civil il était difficile ou impossible de faire des généalogies fiables. 

 La déliquescence des États et donc celle des cours de justice a fini par assurer à l'Église l'exclusivité du traitement des litiges touchant aux mariages entre le \crmieme{9} et le \siecle{12}. Au fur et à mesure que son influence sur le droit du mariage grandissait sa définition de l'inceste était bon gré mal gré intégrée par les familles dans leurs stratégies. A-t-elle été pour elle un outil de conquête du pouvoir ? L'extension des limites de l'inceste servait objectivement ses intérêts matériels et politiques en multipliant les risques de nullité et les demandes de \emph{dispense}%
% [18]
\footnote{En ce qui concernait ces interdits il s'agissait d'une question de discipline et non d'une règle de foi. L'Église se reconnaissait donc le droit d'en dispenser les fidèles qui en faisaient la demande, mais cela ne se faisait pas toujours sans frais.}% 
. C'est la thèse de Goody, et elle n'est pas invraisemblable (selon Poly elle est plausible, mais à partir du \siecle{11} seulement),

 ... mais il faut observer qu'au même moment les rois et les autres puissants s'appuyaient sur les mêmes principes pour s'immiscer dans les conflits au sein des familles de leurs dépendants, et pour gérer à leur convenance les transmissions des patrimoines, des héritages, et des fiefs. 

 C'est de la même façon que les autorités civiles se sont opposées à ce que les enfants illégitimes reçoivent le même traitement que les autres, notamment dans les héritages. Ils s'opposaient surtout à ce qu'ils puissent hériter des fonctions fournissant un surcroît d'\emph{honneur}, c'est-à-dire les fonctions de pouvoir. Rois et clercs ont mis des siècles à parvenir à cette fin, mais ils y sont parvenus. C'est qu'ils avaient des intérêts convergents dans l'affaire. Les puissants avaient intérêt à ce que les familles de leurs concurrents et de ceux qui dépendaient d'eux aient des difficultés à trouver un héritier, sachant qu'environ une femme sur cinq ayant l'âge de procréer n'était pas féconde, que suivant les « honneurs » à transmettre, les filles ne convenaient pas aussi bien qu'un garçon ou étaient exclues, suivant les législations ou les coutumes en vigueur (ex. la loi Salique chez les Francs), et que les enfants illégitimes ne pouvaient en hériter. Il était avantageux pour les puissants de plaider l'illégitimité des enfants de leurs ennemis promis à un riche héritage pour les en déposséder, et donner à un autre de leur choix l'honneur qui devait leur échoir. 

 Et l'extension à l'infini de l'inceste rendait paradoxalement plus facile, pour tous ceux qui pouvaient assumer un procès canonique, d'obtenir l'annulation d'un mariage pour inceste si une alliance plus profitable ou une femme plus désirable ou supposée plus féconde se présentait : si tout le monde était parent de tout le monde toutes les unions étaient incestueuses, et donc à la merci d'un procès gagné d'avance (en somme : « si vous ne voulez pas être piégé dans un mariage indissoluble épousez votre petite cousine »). 
 
 \subsection{Disparition de l'adoption}

 La première des donations à visée religieuse des païens était leur héritage. De droit c'est leur héritier qui était l'officiant de leur culte mortuaire. À partir du moment où les cultes païens ont été disqualifiés, puis interdits, il n'y avait plus de motif religieux de se procurer à toute force un héritier, puisqu'il n'y avait plus de culte des morts à assumer.  C'est l'Église, et non plus les familles, qui gérait le culte des défunts en même temps qu'elle veillait sur les corps rassemblés dans les cimetières et le sol des églises, ce pour quoi elle recevait des donations. Elle n'avait pas de raison de se soucier de la pérennité des lignées, au contraire, l'absence d'héritiers n'était de son point de vue qu'un malheur individuel, et seulement pour cette vie. Cette absence n'avait pas d'incidence sur le salut de l'âme des défunts après leur mort. On en revenait donc aux règles juives : pas de filiation « fictive ». L'adoption plénière, celle qui fabriquait des héritiers légitimes avec des étrangers, a presque totalement disparu de la scène, mais cela ne s'est pas fait du jour au lendemain et il y a fallu plusieurs siècles, d'autant plus que les familles détentrices d'un « honneur », d'une charge publique, à commencer par les rois et les \latin{domini}, les seigneurs, avaient impérativement besoin d'héritiers pour ne pas perdre leur position sociale, et n'étaient pas d'accord sur ce point avec les clercs. 


\subsection{Enfants en trop, enfants « irréguliers »}

 Ce n'est pas parce qu'elle était interdite (mais sans que des sanctions soient prévues) que l'exposition des enfants avait disparu. Les pauvres ont toujours eu recours à l'exposition et n'ont jamais été sanctionnés pour ce motif. Quant aux ventes d'enfants, interdites en principe, elles ne tombaient sous le coup de la loi que lorsqu'elles obéissaient à d'autres motifs que le dénuement%
% [19]
\footnote{... mais qui vendait son enfant pour d'autres raisons (sauf un enfant que le père de famille supposait né d'un adultère de son épouse) ?}% 
. Bien au contraire les acheteurs ont en réalité été encouragés par le fait que les parents qui abandonnaient étaient déchus de leurs droits. Il faudra que le servage disparaisse aux \crmieme{12} et \siecle{13} pour que les ventes d'enfants disparaissent aussi... Et c'est à partir de cette époque que le nombre des expositions de nouveaux-nés dans les villes va se mettre à poser de sérieux problèmes d'ordre public.

 En accord avec la Bible, l'Église a toujours interdit à ses fidèles l'avortement et l'infanticide, et Constantin a introduit cette interdiction dans le droit romain. Qu'en était-il en réalité ? Les avortements et les infanticides ont-ils d'un seul coup disparu ? Il est difficile de le croire. Les infanticides n'ont certainement pas disparu. Ainsi Grégoire de Tours (539-594) rapporte le cas d'une femme qui avait mis au monde un enfant monstrueux : \emph{Comme c'était pour beaucoup un sujet de moquerie de l'apercevoir, et qu'on demandait à la mère comment un tel enfant pouvait être né d'elle, elle confessait en pleurant qu'il avait été procréé pendant une nuit de dimanche. Et n'osant le tuer comme les mères ont coutume de le faire, elle l'élevait de même que s'il eût été conforme}%
% [20]
\footnote{... cité par D.~\fsc{ALEXANDRE-BIDON} et D.~\fsc{LETTE}, p. 27.}% 
. On croit en effet à cette époque que les naissances d'enfants mal conformés sont le résultat de relations sexuelles durant les périodes d'abstinence obligatoire, pendant le carême ou l'avent, pendant les règles%
%[21]
\footnote{Il en est de même pour la lèpre. Il est si difficile de ne pas savoir pourquoi le malheur vous frappe qu'on préfère encore s'en proclamer responsable.}% 
,~etc.

 Mais le plus suggestif c'est le « \emph{n'osant le tuer comme les mères ont coutume de le faire} ». Il faut remarquer la simplicité avec laquelle Grégoire de Tours rapporte ce qui est pour lui une évidence contre laquelle il ne s'indigne pas. Entre les règles morales, même celles qui étaient inscrites dans la loi, et les pratiques effectives, il y avait une marge, comme toujours, et l'infanticide est si aisé et si difficile à prouver. Les nouveaux-nés sont si fragiles, et il arrivait si souvent qu'ils soient étouffés par mégarde sans intention maligne lorsqu'ils partageaient le lit de leur mère, pour avoir plus chaud ou lui éviter de se relever la nuit,~etc. 

 Les avortements ont pu se raréfier en l'absence de médecins et de sages-femmes compétents et prêts à louer leurs services (à supposer que ces compétences se soient effectivement perdues chez les femmes d'expérience, ce qui est à prouver), mais les avortements n'ont jamais été ressentis comme des infanticides, et tout au plus comme des fautes lourdes. Les avortements précoces étaient d'autant moins culpabilisés que pour la plupart des théologiens du Moyen Âge comme pour ceux de l'Antiquité, l'animation du fœtus n'avait pas lieu au moment de la fécondation, mais bien plus tard, chacun défendant sa propre théorie (Cf. Maaike \fsc{VAN DER LUGT}, \frquote{L'animation de l'embryon humain et le statut de l'enfant à naître dans la pensée médiévale}, in \emph{L'embryon, formation et animation}, collectif, déc 2004, Paris, Vrin, p. 234-254). 

 Les enfants issus de simples mésalliances (sénateur--affranchie, femme libre--esclave,~etc.) ne posaient pas de problème religieux aux chrétiens, pas plus qu'aux juifs, même s'ils posaient des problèmes familiaux et sociaux, et même si le droit romain pourchassait ces mésalliances. Ceux dont l'Église réprouvait vraiment la naissance étaient ceux qui avaient été conçus dans le cadre d'une transgression de ses propres lois morales, les \emph{fruits du péché}. 

 Dans ce domaine, les règles de l'Église viennent presque intégralement des juifs. L'échelle de gravité des fautes est calquée sur l'échelle des \emph{mamzerim}. Y ont été ajoutés les enfants nés des personnes qui ont fait vœu de célibat, par analogie avec le sort des enfants illégitimes des prêtres du Temple de Jérusalem. 

 Les \emph{irrégularités de conception} étaient classées comme suit de la moins grave à la plus grave :
\begin{enumerate}
% a)
\item ceux qui ont été conçus dans le cadre d'un concubinage stable, monogame et sans interdit de mariage, et qui n'ont pas (encore) été régularisés par un mariage subséquent ;
% b)
\item ceux qui sont nés d'un rapport de hasard (fornication) ou d'un concubinage qui n'a pas duré ;
% c)
\item ceux qui ont été conçus alors que leur mère se prostituait (fornication) ;
% d)
\item ceux qui sont nés d'un adultère avéré (enfants adultérins) ;
% e)
\item ceux qui sont nés des relations coupables, consenties, d'un clerc ou d'une religieuse ayant fait vœu de célibat (sacrilège) ;
% f)
\item ceux qui sont nés du viol d'une femme mariée (sacrilège) ;
% g)
\item ceux qui sont nés du viol d'une vierge consacrée (sacrilège) ;
% h)
\item ceux qui sont nés d'un inceste. 
\end{enumerate}

 Tous ces enfants étaient illégitimes. Quand ils étaient le fruit des œuvres de leur père avec une servante ou une concubine, ils ont souvent été élevés dans la famille de leur père, au moins pendant le haut Moyen Âge : chez les Germains cela allait de soi. Par contre même s'ils étaient invités à pardonner à leurs épouses infidèles, les maris n'avaient pas l'obligation d'assumer les enfants adultérins de celles-ci. En ce cas l'abandon anonyme était un droit reconnu officiellement, même aux maris fortunés. Mais ils pouvaient aussi les assumer, comme en droit romain. Quant aux enfants nés d'un « sacrilège » ou d'un inceste il est vraisemblable qu'ils étaient le plus souvent traités comme des enfants abandonnés.


\subsection{Les éducations}

 Pour la plupart des enfants des villes (qui représentent peu de chose à l'époque) la petite enfance se passe à la campagne chez une nourrice. La mise en nourrice a concerné plus d'enfants que tous les internats éducatifs, collèges ou hôpitaux réunis, et de très loin. C'était en effet une nécessité absolue pour les femmes des villes qui exerçaient un métier : elles ne pouvaient consacrer à l'allaitement le temps nécessaire jusqu'au sevrage de l'enfant (à deux ans), et il n'y a eu jusqu'au \siecle{20} aucun substitut valable au lait féminin. La généralisation du nourrissage mercenaire reposait aussi sur la possibilité de gagner (à compétences égales) beaucoup plus d'argent en ville qu'à la campagne. Cela permettait aux citadines, même de ressources modestes, d'acheter le lait et le temps des paysannes. La croissance des villes a donc entraîné une augmentation massive du recours à la mise en nourrice.

 Cela n'a pu se faire aussi largement que parce qu'était peu ou pas perçue l'influence des premières relations de l'enfant avec sa mère ou un substitut sur la construction de sa personnalité : le bébé était censé n'avoir besoin que de lait. Le nourrissage mercenaire est donc une institution dont il a été fort peu parlé pendant des millénaires. Cette pratique n'était ni pensée, ni pensable. Elle était du côté des corps et de la nature, des réalités féminines, au même titre que la grossesse et l'accouchement, qui se faisaient aussi bien quand les hommes n'en parlaient pas, sinon mieux. Seul présentait de l'intérêt pour ces derniers ce qui commençait avec l'âge de raison (7 ans).

 Dès qu'ils ont l'âge de raison les enfants de ces temps ne sont plus regardés comme fondamentalement différents des adultes. Ils ne reçoivent aucune protection spéciale (protection du corps contre les gestes traumatiques, protection des yeux et des oreilles contre les spectacles traumatiques). L'éducation est rude et les sanctions sévères. Celui qui économiserait les verges et le fouet croirait mal faire. Les orphelins continuent d'être l'objet de toutes les attentions des autorités. Quant aux jeunes délinquants, condamnables dès 7 ans, ils perdent à 12 ans \emph{l'excuse de minorité}, qui de toute façon n'est pas automatique même avant cet âge. Ils sont en tout traités comme des adultes.

 Dans l'immense majorité des cas chacun apprend de son père le métier de son père. Dès 6 ans la plupart des enfants travaillent autant qu'ils le peuvent. A partir de cet âge un enfant de pauvre ne coûte plus guère. Sauf chez les riches et des puissants, dès 12 ans chacun gagne réellement le pain qu'il mange chez ses parents ou chez un maître. Le placement réciproque des adolescents chez des alliés des parents (oncles, surtout maternels, suzerain,~etc..) est un outil éducatif souvent employé (les jeunes nobles servent comme pages, les fils de paysans comme pâtres, les marins comme mousses,~etc.). Le placement en apprentissage chez un artisan (maître ès arts) n'est possible que si les parents paient l'apprentissage : c'est un luxe auquel les pauvres ne peuvent pas prétendre. Pour l'école il en est de même. 

 L'Antiquité grecque ou romaine avait élaboré à l'intention de ceux qui pouvaient se le payer un système complet d'enseignement (primaire, secondaire et supérieur). D'autre part un certain nombre de postes de professeur du secondaire étaient payés par les cités, et l'état romain finançait des chaires d'enseignement supérieur (Augustin d'Hippone en est un représentant illustre : il se décrit successivement élève, étudiant, enseignant et titulaire de chaire). Au \siecle{4} ce système continue de fonctionner, au \siecle{6} il est pratiquement en ruines en Occident, alors qu'il perdurera encore dix siècles à Byzance sans changements de structure. Les universités européennes ne relèveront le flambeau qu'à partir des derniers siècles du Moyen Âge. 

 En attendant seules résistent les écoles cathédrales et monastiques. Les premières écoles \emph{épiscopales} ou \emph{cathédrales} fleurissent au \siecle{4}. Elles ont pour principal objet de former les futurs clercs, mais les élèves peuvent à la fin du cursus refuser d'entrer dans le clergé. Le \emph{deuxième concile de Vaison} (529) prescrit à chaque prêtre chargé de paroisse de mettre en place une \emph{école paroissiale} à l'intention des jeunes les plus vifs d'esprit. Ce sont les premières attestations de \emph{petites écoles}. Elles sont d'abord destinées à alimenter \emph{l'école cathédrale} en sujets d'élite destinés à former le personnel ecclésiastique, et n'ont pas pour but d'apprendre à lire à tous comme c'est le cas chez les juifs : la vie religieuse du chrétien n'exige pas qu'il sache lire, il suffit qu'il sache entendre. Son activité professionnelle ne l'implique pas non plus : l'enseignement lettré (maîtrise du latin, langue de la culture et des savants) est inutile à qui ne sera pas clerc. Si l'on cherche le pouvoir il est alors plus efficace de connaître les armes que la rhétorique ou le droit. 
 
 Les \emph{écoles monastiques} apparaissent aussi au \siecle{4}, mais elles ne prennent en principe que des enfants destinés à devenir moines (« donnés » très jeunes à Dieu par leurs parents) et seuls apprennent le latin, les moines « de chœur », ceux qui chantent dans le chœur, ceux qui pourraient être ordonnés prêtres. Pourtant bien des écoles monastiques acceptent aussi quelques jeunes qui ne sont pas destinés à devenir moines, et Charlemagne leur en fera l'obligation.
 
 
 % Le 12 mars 2015 :
% \latin
% ~etc.
% Moyen Âge
% Antiquité


\section{Familles spirituelles}


 Si les jeunes gens pouvaient se marier validement sans l'accord de leurs parents, en bonne logique ils avaient aussi le droit de ne pas se marier. Devant le désir d'un jeune de devenir religieux l'autorité du père s'arrêtait si le jeune en appelait à son évêque : pour les garçons à partir de 14 ans, pour les filles à partir de 12 ans. Le choix de la vie religieuse émancipait ceux qui le faisaient avant l'âge de leur majorité, et les mettait sous sa protection : cela reposait évidemment sur une reconnaissance par les autorités civiles de la validité des vœux religieux. Cette reconnaissance leur a été accordée par les derniers empereurs romains (chrétiens) et a été reconduite jusqu'à la Révolution française.

 Ceux qui se sentaient attirés par une vie de célibataire consacré pouvaient proposer à une communauté religieuse de les coopter. Ce choix de vie entraînait des incidences légales importantes et définitives, puisque la loi civile l'entérinait. En effet en prononçant ses vœux (pauvreté, chasteté, et surtout obéissance) le moine ou la religieuse se mettaient sous la puissance du responsable de la communauté, comme s'ils s'étaient fait adroger. Ils étaient juridiquement exclus de leur famille de naissance, et de tout héritage à venir. Comme des mineurs ils ne pouvaient plus rien faire de leur propre initiative. S'ils ne pouvaient signer aucun contrat en leur nom propre, ils pouvaient toujours, de la même façon qu'un esclave ou qu'un fils en puissance de \latin{pater familias}, exercer au nom de leur supérieur(e) tout mandat qu'il lui convenait de leur confier. Une fois entrés dans la communauté, c'était en principe pour toujours. Ils ne pouvaient plus sortir de leur état. Ils pouvaient dans une certaine mesure changer de monastère et même d'ordre religieux, mais les \emph{gyrovagues} qui erraient de couvent en couvent étaient mal vus. 

 Aucun religieux ne possédait rien qui lui soit personnel, et pourtant beaucoup d'entre eux avaient reçu de leurs parents une part d'héritage sous forme d'argent, de terre,~etc. À leur entrée dans la communauté ils en avaient fait don (eux ou leurs parents) à la communauté, qui en contrepartie s'était engagée à les prendre en charge jusqu'à leur mort. Chaque communauté vivait de son travail et des revenus des biens qu'elle avait reçus en don : dots des religieux vivants \emph{et décédés}, loyers, récoltes, rentes et autres dons reçus de bienfaiteurs. Tous les biens appartenaient au monastère et celui-ci possédait le droit de posséder et d'exercer des actes juridiques en son nom propre. Si les religieux se succédaient de génération en génération, le monastère en tant qu'entité n'en persistait pas moins dans son être, unissant les morts et les vivants dans le même ensemble intemporel. Le modèle familial ainsi mis en œuvre était accepté en toute connaissance de cause ainsi que le montre l'emploi très précoce du vocabulaire de la famille : « père » (\emph{abba} = père en araméen = abbé), « mère », « frère », « sœur »,~etc. Mais cette famille n'avait pas d'héritiers à pourvoir et ses biens étaient inaliénables et insaisissables, protégés par le statut de la \emph{mainmorte}. 
 
 Ce mot a deux sens :
\begin{enumerate}
% a)
\item c'est le droit du seigneur de prendre les biens de son serf à sa mort. Les biens font \emph{échute}, c'est-à-dire réversion au seigneur qui en hérite. En ce sens les serfs sont \emph{gens de mainmorte}. Ce n'est pas le sens du mot \emph{mainmorte} qui nous concerne ici%
%[1]
\footnote{... même si un bon nombre des derniers serfs (fin \siecle{18}) appartenaient à des communautés religieuses de l'Est de la France, et si à cette époque les religieux ont eux aussi été nommés \emph{gens de mainmorte}, parce qu'incapables de transmettre des biens à des héritiers, non comme serfs d'un seigneur, mais comme religieux.}
 ;
% b)
\item on appelle aussi \emph{biens de mainmorte} ceux qui appartiennent à une personne juridique : ce sont les biens des collectivités qui ont le privilège de pérennité et n'ont pas à transmettre leurs biens à des héritiers. Cela conduisait à l'enrichissement progressif des communautés bien gérées... jusqu'au jour où leurs richesses devenaient trop tentantes pour les puissants du moment et leur étaient (re) prises par l'un d'entre eux : de ce point de vue l'histoire de la plupart des monastères est celle d'une suite de périodes d'accumulation et de moments de spoliation.
\end{enumerate} 

 À la fin de l'Antiquité il était admis que dès leur plus jeune âge (6 ou 7 ans) les parents puissent faire don à un monastère d'un ou plusieurs de leurs enfants, légitimes ou non%
% [2]
\footnote{Sources : Marc \fsc{BLOCH}, \emph{La société féodale}, Paris, 1939, 1994. Georges \fsc{DUBY}, \emph{Féodalité}, Paris, 1996, 1999. Collectif, \emph{L'homme médiéval}, Paris, 1989.}% 
. Ils accompagnaient le « don » de l'enfant d'un cadeau, souvent un bien foncier, qui devait permettre de subvenir à son entretien. Si l'enfant \emph{à Dieu donné} découvrait un jour que ce mode de vie ne lui convenait pas, il lui était extrêmement difficile d'en sortir. Le droit civil et les enseignements de l'Église se liguaient pour lui prêcher la résignation et lui barrer tout retour. Le jeune \emph{donné} à un monastère n'était d'ailleurs pas forcément plus contrarié dans ses choix que les jeunes esclaves, ou que les jeunes gens qui au même moment étaient mariés par leurs familles sans tenir compte de leur avis, ou qui devaient reprendre le métier de leur père. D'autre part, le \emph{don à Dieu} côtoyait des situations contemporaines par rapport auxquelles il représentait un progrès relatif (cf. {Boswell}) : abandon anonyme, infanticide, vente par les parents comme esclave,~etc. 


Les jeunes « donnés » ont pu à certaines périodes représenter une proportion importante de l'effectif des monastères, mais ceux-ci fournissaient aussi à ceux et celles qui n'étaient pas ou plus attirés par le mariage un moyen de l'éviter, alors que le célibat non consacré était mal accepté par la société civile. Cela permettait aux veuves d'échapper à la nécessité de se mettre sous la protection d'un mari. Cela donnait une chance aux femmes les plus douées de jouer un rôle public auquel elles n'auraient jamais pu rêver autrement. C'était le seul moyen pour les filles d'esquiver un mari grossier, mesquin ou brutal, et/ou d'éviter de risquer leur vie dans les grossesses et les accouchements%
% [3]
\footnote{... qui à l'époque faisaient mourir (en hôpital) une femme sur dix environ si l'on en croit les mémoires de \fsc{TENON}, ce qui ne représente pas une naissance sur dix, bien évidemment, mais d'une naissance sur 30 à une naissance sur 120 suivant les temps et les lieux (p. 242 et suivantes). Ce chiffre avait de quoi angoisser les jeunes filles, surtout celles de santé fragile, ou celles qui présentaient une malformation. Jacques Tenon était chirurgien à l'Hôtel-Dieu de Paris avant la Révolution. À la demande des autorités il a rédigé ses \emph{mémoires sur les hôpitaux de Paris} édités en 1788. Nous le citerons souvent.}%
. C'était un refuge pour les jeunes gens mal conformés ou de santé trop fragile. 

 D'un autre point de vue, l'entrée en religion d'un enfant légitime diminuait mécaniquement le nombre des petits-enfants à naître (qu'il faudrait « établir » un jour sur le capital familial). C'était donc une forme indirecte de contrôle des naissances. C'est l'une des raisons, sinon la première, pour lesquelles les seigneurs grands et petits ont créé tant de monastères : ils avaient un intérêt direct à disposer d'institutions où placer l'excédent de leur progéniture dans un cadre conforme à la dignité de leur famille, et sans contrevenir aux lois de l'Église, qu'ils avaient à peu près intériorisées. D'ailleurs si la politique familiale l'exigeait (par exemple si les enfants privilégiés dans un premier temps décédaient) il n'était pas impossible de relever de ses vœux et de faire sortir du cloître une fille ou un fils, à la condition qu'il n'ait pas été ordonné prêtre (mais le plus souvent les moines ne l'étaient pas : pour le droit canon, c'étaient des laïcs, à l'exception de ceux qui étaient ordonnés diacres ou prêtres).

 Par ailleurs même quand pour les personnages publics puissants il était difficile de faire d'un fils illégitime un successeur. D'autre part il était interdit d'ordonner prêtres les garçons illégitimes : pour eux le clergé séculier n'était un débouché envisageable qu'au prix de dispenses coûteuses. Au contraire les monastères ne manifestaient pas de réticences à les accueillir, tout comme ils accueillaient les enfants abandonnés. La vie des religieux est conçue comme une vie de purification. De plus c'est une vie cachée à l'écart du monde. Dans l'esprit du temps cela convenait parfaitement aux pécheurs et pécheresses repentis, aux natures perverties par le péché, et donc aux « impurs de naissance » ou aux clercs séculiers punis pour fautes graves. En outre, on considérait qu'ainsi les enfants illégitimes pouvaient racheter la faute de leurs parents. Il paraissait donc très louable de les vouer à la vie religieuse. Par ailleurs cela les excluait des jeux de pouvoir dont le monde profane est le théâtre. Il n'y avait plus à craindre de les voir parasiter les politiques familiales. C'est ainsi que les parents pouvaient estimer s'en sortir \emph{par le haut} du casse-tête créé par leurs enfants illégitimes ou surnuméraires.
 
 C'est aussi pour cela que tant de filles de rois, légitimes ou non, qui ne pouvaient sans déchoir être données en mariage à des aristocrates trop inférieurs en dignité à leur beau-père, et qui risquaient si on les mariait à de trop puissants seigneurs de donner naissance à des garçons d'ascendance royale susceptibles de menacer les héritiers légitimes du trône, se sont retrouvées abbesses d'abbayes royales, jusqu'au \siecle{18}. 

 La part d'héritage (un bien foncier, une somme acquise définitivement par le couvent dès la profession, la \emph{dot},~etc.) donnée par leurs familles aux futurs religieux était fonction de la fortune familiale et du prestige de la maison religieuse où ils entraient. Chaque ordre et chaque monastère possédaient une « cote » sur le marché des valeurs de prestige, ce qui justifiait un coût (et inversement). En règle générale la part d'héritage de celui ou celle qui entrait dans les ordres était bien plus petite que celle des autres enfants de sa famille. Il convenait en effet qu'une fois tout réglé il reste aux pères un bénéfice financier à faire entrer des enfants en religion. 

 Selon la plupart des règles une communauté vivait non seulement de ses rentes, mais aussi du travail de ses membres. Mais souvent seuls les \emph{convers}, enfants de pauvres reçus gratuitement sans part d'héritage (ou adultes qui se donnaient eux-mêmes ainsi), travaillaient de leurs mains : sauf dons intellectuels ou spirituels éclatants ils n'étaient instruits que superficiellement et ils effectuaient l'essentiel des besognes matérielles, tandis que les religieux mieux dotés par leurs parents étudiaient, apprenaient à lire et à écrire, apprenaient le latin, chantaient au chœur, copiaient les livres, enseignaient,~etc. Il y avait là une évidente sélection par l'argent et par la naissance. Pendant très longtemps il semble que nul n'y ait vu matière à scandale. C'est que depuis l'empire romain (dès Caracalla, sinon avant) les sociétés civiles contemporaines étaient très inégalitaires avec des castes et des hiérarchies civiles justifiées par l'idéologie de l'hérédité (du « sang »), à laquelle les barbares adhéraient autant sinon plus que les Romains. Il est vrai aussi que les exhortations de Paul de Tarse (entre autres) à demeurer à la place où Dieu vous a mis ne favorisaient pas la mise en question de l'ordre établi%
% [4]
\footnote{Il faudra attendre les ordres mendiants à partir du \siecle{12} pour que ces discriminations par l'argent au sein des ordres religieux soient dénoncées, mais non supprimées. François d'Assise, fondateur des franciscains, était fils de bourgeois, non d'aristocrate, ce qui lui avait sans doute donné un autre regard sur le caractère « naturel » des discriminations de caste. Elles ne semblent pas avoir été vécues comme insupportables avant le \crmieme{18} ou \siecle{19}, du moins pour la plupart des religieux qui avaient droit à l'écriture et qui ont laissé des témoignages : mais ce n'étaient pas eux qui étaient ainsi humiliés.}% 
. 

 Si les familles des bienfaiteurs et des fondateurs entretenaient des liens étroits avec « leur » monastère pour conserver la possibilité d'y placer des enfants, elles le faisaient aussi et au moins autant parce qu'elles comptaient sur les prières des religieux, sur leurs messes et sur leurs autres dévotions, offices divers qu'elles « fondaient » contre donation pour garantir le salut éternel des âmes de leurs membres. Chacun de ceux qui le pouvaient affectait une part de ses biens à ces donations comme leurs ancêtres pré chrétiens avaient affecté une part de leurs biens (un tiers selon Goody ?) aux sacrifices à faire après leur mort et aux objets qu'ils emportaient avec eux dans la tombe. C'est pourquoi jusqu'à la fin du Moyen Âge presque tous les testaments contenaient des donations pour le repos de l'âme du défunt, faites à une institution religieuse et/ou d'assistance, ce qui à l'époque était indiscernable : toutes les œuvres d'assistance étaient aussi des « \emph{œuvres pieuses »}. Cela a contribué à produire en quelques siècles un quasi-monopole de l'Église dans le domaine des testaments et des conflits qui y sont liés.
 
 
\section[L'esclavage chez les chrétiens de l'Antiquité tardive et du haut Moyen Âge]{L'esclavage chez les chrétiens\\de l'Antiquité tardive\\et du haut Moyen Âge}


 Aucun auteur antique n'est allé jusqu'à une condamnation de l'esclavage en tant que système. À cette époque, un monde sans esclaves n'était pas pensable. Seules certaines sociétés arriérées et misérables d'alors se passaient réellement d'esclaves, et elles n'avaient rien de désirable pour les autres. 

 Selon Jean \fsc{ANDREAU} et Raymond \fsc{DESCAT} (\emph{Esclave en Grèce et à Rome}, 2006, p. 220) : \emph{celui qui est allé le plus loin dans la condamnation de l'esclavage reste Grégoire de Nysse, au \siecle{4}. Non seulement il estimait que, devant Dieu, les esclaves sont les égaux des hommes libres, mais il regardait la possession d'esclaves comme un péché et un très grave péché. En effet, quoique toutes les créatures soient au service de Dieu et appartiennent à Dieu, le propriétaire d'esclaves s'est approprié certaines de ces créatures, ce qui revient à défier l'ordre divin et à revendiquer un droit qui ne peut être que celui de Dieu} [...] (Homélie IV sur l'Ecclésiaste, 2, 7) [...] \emph{Grégoire a-t-il libéré tous ses esclaves ? Nous n'en savons rien. Mais \emph{[...]} même lui n'a pas milité pour l'abolition de l'esclavage}. Selon les mêmes auteurs, deux groupes dissidents juifs, les Esséniens et les Thérapeutes, étaient opposés à l'esclavage, mais n'ont pas non plus milité en ce sens : \emph{aucun penseur antique ne l'a fait, et il n'y a jamais eu, dans l'Antiquité, de mouvement abolitionniste}... Saint Augustin interprète l'existence de l'esclavage comme une conséquence du péché originel, et pour lui comme pour les stoïciens le fait d'être esclave du péché était bien plus grave que celui d'être esclave d'un maître. 
 
 Même si l'Église a toujours soutenu le caractère méritoire des affranchissements elle ne s'est donc pas attaquée à l'institution de l'esclavage. Elle a reçu sans états d'âme des esclaves en donation, elle en a acheté, elle en a employé. Elle a même fait obligation aux évêques et prêtres qui libéraient un esclave de le racheter sur leur fortune personnelle ou de le remplacer par un autre esclave de valeur équivalente pour que le patrimoine de l’Église dont la gestion leur avait été confiée ne soit pas diminué (ex. conciles espagnols \crm{4}--\siecles{5}{6}). 

 Cela étant dit il y a des choses que l'Église ne supportait pas :
\begin{enumerate}
% 1°)
\item Qu'un esclave ne puisse pas devenir chrétien alors qu'il le désirait, et qu'il soit empêché de satisfaire aux prescriptions d'une vie chrétienne régulière (culte dominical, jeûnes,~etc.). Elle était contrainte par les lois civiles de demander l'autorisation expresse des maîtres avant tout baptême d'esclave, mais elle vivait comme une persécution le fait qu'ils la lui refusent.
% 2°)
\item Qu'un ou une esclave soit contraint à des pratiques contraires au Décalogue, entre autre dans le domaine sexuel. Elle n'acceptait pas qu'un esclave soit condamné à vie à un célibat non choisi, ou à une vie de promiscuité sexuelle, ou à des avortements ou à l'exposition de ses enfants,~etc.
% 3°)
\item Elle défendait le droit au mariage des esclaves. Pour « unir » deux esclaves, il suffisait que le maître permette que soit organisée une cérémonie interne à la \latin{familia} où toutes les personnes présentes, libres et esclaves, étaient les témoins des conjoints et faisaient la fête avec eux, mais cet acte était infra juridique et purement domestique. Cette union (\latin{contubernium}, ou compagnonnage de chambrée) n'avait pas la valeur d'un authentique mariage au-delà des murs du domaine, et le maître n'était pas tenu de la respecter. Elle ne donnait pas aux intéressés de droits parentaux. A contrario l'Église reconnaissait le mariage des esclaves du moment qu'il était monogame, fidèle et inscrit dans la durée, sans distinguer leur union de celle des personnes libres.
% 4°)
\item L'Église ne supportait pas qu'on sépare les couples d'esclaves, qu'on les vende chacun de son côté, ou qu'on leur rende la vie commune impossible. Elle ne supportait pas que les enfants des esclaves soient séparés de leurs parents, confiés à d'autres personnes contre leur gré, et encore moins vendus de leur côté. Le corollaire du droit au mariage, sans lequel ce droit n'a aucun sens, est en effet qu'il soit garanti à ceux qui se marient un minimum de maîtrise sur le temps à venir et de droits sur leur conjoint et sur leurs enfants. 
\end{enumerate}

 L'Église n'acceptait donc l'esclavage que pour autant que le statut des esclaves soit aménagé, de même que les juifs n'acceptaient l'esclavage d'un coreligionnaire que s'il était traité en mercenaire (ou en gagé pour dettes) et non asservi à perpétuité. Si l'on acceptait ces exigences, le statut des esclaves se rapprochait de celui des hilotes grecs et de divers autres statuts de dépendants, dans lesquels la force de travail de ceux-ci appartenait à leurs maîtres de manière héréditaire, mais pas leurs corps ni leurs droits parentaux. Les serfs du Moyen Âge \emph{n'étaient pas des esclaves}, mais le latin ne possède qu'un seul mot pour désigner ces deux statuts \latin{(servi)}, ce qui plaide en faveur d'une évolution progressive de l'un vers l'autre sur plusieurs siècles.

 Les esclaves constituaient une part relativement importante de la population du haut Moyen Âge. Ils étaient toujours l'objet d'achat et de vente, et leurs enfants appartenaient toujours au maître de leur mère. La réduction des chrétiens en esclavage par la force avait été interdite par les derniers empereurs, de même que depuis des siècles il n'était pas permis d'asservir des citoyens grecs ou romains ... mais rien n'empêchait personne de se vendre soi-même. À qui n'avait ni alliés ni capitaux ni culture ni savoir-faire rare, il n'était pas plus facile qu'auparavant de trouver de quoi vivre. Était toujours à bon droit traité comme esclave celui qui se reconnaissait comme tel. L'interdiction d'asservir les chrétiens ne concernait pas les tribunaux, libres de condamner des coupables à l'esclavage. S'il était interdit d'asservir des chrétiens nés libres, rien n'obligeait à libérer les chrétiens qui étaient nés esclaves, même si leur affranchissement était un acte si louable que les évêques s'y impliquaient personnellement. Et les interdits consécutifs au fait d'avoir été un esclave (interdits qui constituent la \emph{marque servile}) frappaient toujours les affranchis, montrant la persistance des anciennes représentations. D'ailleurs les institutions ecclésiastiques possédaient leurs propres esclaves sans y voir aucun mal. Le principal souci des évêques était d'empêcher que les esclaves chrétiens ne soient soumis à des maîtres païens ou juifs susceptibles de les détourner de la foi chrétienne, et qu'ils ne soient déportés dans des contrées non chrétiennes. 

 Si la réduction en esclavage de chrétiens libres (« ingénus ») était un crime pour l'Église et pour les pouvoirs civils, il n'en était pas de même en ce qui concernait les autres (juifs, païens, hérétiques et schismatiques divers...), qui pouvaient être asservis sans problème. A fortiori n'étaient pas affranchis non plus les païens capturés à la guerre ou à la chasse aux esclaves (parmi les attraits de la guerre, celui d'y faire des esclaves demeurait aussi important que par le passé) même si leurs maîtres chrétiens avaient le devoir moral de les faire baptiser, en vertu de leur autorité sur la totalité des membres de leur maison. D'ailleurs ne valait-il pas mieux, comme toujours, asservir les vaincus plutôt que de les passer au fil de l'épée, eux, leurs femmes et leurs enfants, lorsqu'ils n'avaient pas les moyens de payer une rançon ?

 Le commerce des esclaves, notamment non-chrétiens, a donc perduré bien au-delà du Moyen Âge, alimenté par des circuits divers (\anglais{slave} vient de « esclave »), même si à partir du milieu du Moyen Âge, l'esclavage en tant que tel n'a plus joué en Europe un rôle important, sauf exception locale (Espagne, pourtour de la Méditerranée). 

 Le statut des esclaves s'est pourtant insensiblement modifié au fil des siècles : la plupart des esclaves ruraux ont été \emph{chasés}, c'est-à-dire installés dans une \latin{casa}, une maison, avec la pièce de terre plus ou moins étendue que le maître y adjoignait, et une concubine attitrée prise dans sa \latin{familia}, comme les \emph{colons esclaves} de l'Antiquité romaine. Mais dans le même temps le statut de beaucoup des tenanciers libres, \emph{colons libres} d'un propriétaire, ou propriétaires indépendants \emph{(alleutiers)}, sans oublier les affranchis, s'est dégradé au fil du temps pour se rapprocher de celui des esclaves. Pour une part de la population, plus ou moins grande selon les lieux, le résultat de ces deux mouvements a été la généralisation du statut de \emph{serf}, qui attachait chacun de manière héréditaire à une terre ou à un office, et l'assujettissait au seigneur \latin{(dominus)} de cette terre ou de cet office. 

 Ils étaient possédés par leur emploi, ils n'étaient pas totalement libres d'employer leur temps et leurs forces à leur gré. Ils ne pouvaient ni s'en aller ni se soustraire aux ordres reçus. Ils devaient se marier sur le domaine. Une partie de leurs droits personnels appartenaient au seigneur. Par contre et contrairement aux esclaves, ils jouissaient du reste de leurs droits personnels, notamment conjugaux et parentaux ... mais certaines terres, certains offices au service des puissants valaient parfois qu'on s'asservisse pour eux. Certains « postes » de serfs étaient jugés très enviables, au même titre que certains esclaves de personnages puissants pouvaient provoquer des jalousies.

 Le servage était une promotion pour les esclaves, mais une régression pour les personnes libres. En acceptant leur dépendance, celles-ci voyaient s'aliéner une part très significative de leur liberté. En contrepartie, elles faisaient partie d'une communauté villageoise. Les gens des villages, serfs ou libres, n'étaient pas toujours incapables de faire bloc et d'exercer une pression sur leur seigneur, qui avait besoin de leur prospérité matérielle autant qu'ils avaient besoin de sa protection. Ils pouvaient dans une certaine mesure intervenir en tiers entre un serf et lui. En droit comme en fait, il était assez difficile au \latin{dominus} de chasser un serf de sa terre. 
 
 En France l'esclavage a disparu au profit du servage vers le \siecle{8}, et le 3 juillet 1315 Louis Le Hutin a décidé que sont libres tous les esclaves chrétiens qui posent le pied sur le territoire français. 

 Cela n'empêchera pas les Français de recourir à l'esclavage dans leurs colonies avec l'approbation des autorités civiles (cf. \emph{le code noir}) : il suffira d'empêcher les esclaves (et aussi \emph{tous} les noirs, esclaves ou non) de toucher le sol de France. La survie de l'esclavage sur les terres européennes, sans interruption depuis l'Antiquité, dans l'Europe du sud, et d'abord en Espagne, a préparé les esprits de la Renaissance à recourir  à l'esclavage des indiens, puis celui des noirs d'Afrique, pour exploiter des deux Amériques et les autres colonies européennes (Surinam,~etc.). 

 Ce n'est qu'à partir de la fin du \siecle{18} qu'une part significative des intellectuels sont tombés d'accord pour condamner l'esclavage sans lui donner aucune circonstance atténuante (cf. \emph{L'Encyclopédie}).
 

% 28.02.2015 :
% haut Moyen Âge
% _, --> ,
% Antiquité
% ~etc.
% ~\%




% 28.02.2015 :
% haut Moyen Âge
% _, --> ,
% Antiquité
% ~etc.
%~\%


\chapter[L'esclavage chez les chrétiens de l'Antiquité tardive et du haut Moyen Âge]{L'esclavage chez les chrétiens\\de l'Antiquité tardive\\et du haut Moyen Âge}


 Aucun auteur antique n'est allé jusqu'à une condamnation de l'esclavage en tant que système. À cette époque, un monde sans esclaves n'était pas pensable. Seules certaines sociétés arriérées et misérables d'alors se passaient réellement d'esclaves, et elles n'avaient rien de désirable pour les autres. 

 Selon Jean \fsc{ANDREAU} et Raymond \fsc{DESCAT} (\emph{Esclave en Grèce et à Rome}, 2006, p. 220) : \emph{Celui qui est allé le plus loin dans la condamnation de l'esclavage reste Grégoire de Nysse, au \siecle{4}. Non seulement il estimait que, devant Dieu, les esclaves sont les égaux des hommes libres, mais il regardait la possession d'esclaves comme un péché et un très grave péché. En effet, quoique toutes les créatures soient au service de Dieu et appartiennent à Dieu, le propriétaire d'esclaves s'est approprié certaines de ces créatures, ce qui revient à défier l'ordre divin et à revendiquer un droit qui ne peut être que celui de Dieu ...} (Homélie IV sur l'Ecclésiaste, 2, 7) \emph{... Grégoire a-t-il libéré tous ses esclaves ? Nous n'en savons rien. Mais ... même lui n'a pas milité pour l'abolition de l'esclavage}. Selon les mêmes auteurs, deux groupes dissidents juifs, les Esséniens et les Thérapeutes, étaient opposés à l'esclavage, mais n'ont pas non plus milité en ce sens : \emph{aucun penseur antique ne l'a fait, et il n'y a jamais eu, dans l'Antiquité, de mouvement abolitionniste}...

 Saint Augustin interprète l'existence de l'esclavage comme une conséquence du péché originel, et pour lui comme pour les stoïciens le fait d'être esclave du péché était bien plus grave que celui d'être esclave d'un maître. Même si l'Église a toujours soutenu le caractère méritoire des affranchissements elle ne s'est donc pas attaquée à l'institution de l'esclavage. Elle a reçu sans états d'âme des esclaves en donation, elle en a acheté, elle en a employé. Elle a même fait obligation aux évêques et prêtres trop bienveillants de lui racheter les esclaves qu'ils voulaient libérer ou de les remplacer par d'autres esclaves de valeur équivalente au moyen de leur fortune personnelle, pour compenser la perte matérielle qu'un affranchissement pourrait faire subir aux biens (inaliénables) de l’Église dont la gestion leur avait été confiée (ex. conciles espagnols \crm{4}--\siecles{5}{6}). 

 Cela étant dit il y a des choses que l'Église ne supportait pas :
\begin{enumerate}
% 1°)
\item Qu'un esclave ne puisse pas devenir chrétien alors qu'il le désirait, et qu'il soit empêché de satisfaire aux prescriptions d'une vie chrétienne régulière (culte dominical, jeûnes,~etc.). Elle était contrainte par les lois civiles de demander l'autorisation expresse des maîtres avant tout baptême d'esclave, mais elle vivait comme une persécution le fait qu'ils la lui refusent.
% 2°)
\item Qu'un ou une esclave soit contraint à des pratiques contraires au Décalogue, entre autre dans le domaine sexuel. Elle n'acceptait pas qu'un esclave soit condamné à vie à un célibat non choisi, ou à une vie de promiscuité sexuelle, ou à des avortements ou à l'exposition de ses enfants,~etc.
% 3°)
\item Elle défendait le droit au mariage des esclaves. Pour « unir » deux esclaves, il suffisait que le maître permette que soit organisée une cérémonie interne à la \emph{familia} où toutes les personnes présentes, libres et esclaves, étaient les témoins des conjoints et faisaient la fête avec eux, mais cet acte était infra juridique et purement domestique. Cette union (\emph{contubernium}, ou compagnonnage de chambrée) n'avait pas la valeur d'un authentique mariage au-delà des murs du domaine, et le maître n'était pas tenu de la respecter. Elle ne donnait pas aux intéressés de droits parentaux. A contrario l'Église reconnaissait le mariage des esclaves du moment qu'il était monogame, fidèle et inscrit dans la durée, sans distinguer leur union de celle des personnes libres.
% 4°)
\item L'Église ne supportait pas qu'on sépare les couples d'esclaves, qu'on les vende chacun de son côté, ou qu'on leur rende la vie commune impossible. Elle ne supportait pas que les enfants des esclaves soient séparés de leurs parents, confiés à d'autres personnes contre leur gré, et encore moins vendus de leur côté. Le corollaire du droit au mariage, sans lequel ce droit n'a aucun sens, est en effet qu'il soit garanti à ceux qui se marient un minimum de maîtrise sur le temps à venir et de droits sur leur conjoint et sur leurs enfants. 
\end{enumerate}

 L'Église n'acceptait donc l'esclavage que pour autant que le statut des esclaves soit aménagé, de même que les juifs n'acceptaient l'esclavage d'un coreligionnaire que s'il était traité en mercenaire (ou en gagé pour dettes) et non asservi à perpétuité. Si l'on acceptait ces exigences, le statut des esclaves se rapprochait de celui des hilotes grecs et de divers autres statuts de dépendants, dans lesquels la force de travail de ceux-ci appartenait à leurs maîtres de manière héréditaire, mais pas leurs corps ni leurs droits parentaux. Les serfs du Moyen Âge \emph{n'étaient pas des esclaves}, mais le latin ne possède qu'un seul mot pour désigner ces deux statuts \emph{(servi)}, ce qui plaide en faveur d'une évolution progressive de l'un vers l'autre sur plusieurs siècles.

 Les esclaves constituaient une part relativement importante de la population du haut Moyen Âge. Ils étaient toujours l'objet d'achat et de vente, et leurs enfants appartenaient toujours au maître de leur mère. La réduction des chrétiens en esclavage par la force avait été interdite par les derniers empereurs, de même que depuis des siècles il n'était pas permis d'asservir des citoyens grecs ou romains ... mais rien n'empêchait personne de se vendre soi-même. À qui n'avait ni alliés ni capitaux ni culture ni savoir-faire rare, il n'était pas plus facile qu'auparavant de trouver de quoi vivre. Était toujours à bon droit traité comme esclave celui qui se reconnaissait comme tel. L'interdiction d'asservir les chrétiens ne concernait pas les tribunaux, libres de condamner des coupables à l'esclavage. S'il était interdit d'asservir des chrétiens nés libres, rien n'obligeait à libérer les chrétiens qui étaient nés esclaves, même si leur affranchissement était un acte si louable que les évêques s'y impliquaient personnellement. Et les interdits consécutifs au fait d'avoir été un esclave (interdits qui constituent la \emph{marque servile}) frappaient toujours les affranchis, montrant la persistance des anciennes représentations. D'ailleurs les institutions ecclésiastiques possédaient leurs propres esclaves sans y voir aucun mal. Le principal souci des évêques était d'empêcher que les esclaves chrétiens ne soient soumis à des maîtres païens ou juifs susceptibles de les détourner de la foi chrétienne, et qu'ils ne soient déportés dans des contrées non chrétiennes. 

 Si la réduction en esclavage de chrétiens libres (« ingénus ») était un crime pour l'Église et pour les pouvoirs civils, il n'en était pas de même en ce qui concernait les autres (juifs, païens, hérétiques et schismatiques divers...), qui pouvaient être asservis sans problème. A fortiori n'étaient pas affranchis non plus les païens capturés à la guerre ou à la chasse aux esclaves (parmi les attraits de la guerre, celui d'y faire des esclaves demeurait aussi important que par le passé) même si leurs maîtres chrétiens avaient le devoir moral de les faire baptiser, en vertu de leur autorité sur la totalité des membres de leur maison. D'ailleurs ne valait-il pas mieux, comme toujours, asservir les vaincus plutôt que de les passer au fil de l'épée, eux, leurs femmes et leurs enfants, lorsqu'ils n'avaient pas les moyens de payer une rançon ?

 Le commerce des esclaves, notamment non-chrétiens, a donc perduré bien au-delà du Moyen Âge, alimenté par des circuits divers (\emph{slave} vient de « esclave » ), même si à partir du milieu du Moyen Âge, l'esclavage en tant que tel n'a plus joué en Europe un rôle important, sauf exception locale (Espagne, pourtour de la Méditerranée). 

 Le statut des esclaves s'est pourtant insensiblement modifié au fil des siècles : la plupart des esclaves ruraux ont été \emph{chasés}, c'est-à-dire installés dans une \emph{casa}, une maison, avec la pièce de terre plus ou moins étendue que le maître y adjoignait, et une concubine attitrée prise dans sa \emph{familia}, comme les \emph{colons esclaves} de l'Antiquité romaine. Mais dans le même temps le statut de beaucoup des tenanciers libres, \emph{colons libres} d'un propriétaire, ou propriétaires indépendants \emph{(alleutiers)}, sans oublier les affranchis, s'est dégradé au fil du temps pour se rapprocher de celui des esclaves. Pour une part de la population, plus ou moins grande selon les lieux, le résultat de ces deux mouvements a été la généralisation du statut de \emph{serf}, qui attachait chacun de manière héréditaire à une terre ou à un office, et l'assujettissait au seigneur \emph{(dominus)} de cette terre ou de cet office. 

 Ils étaient possédés par leur emploi, ils n'étaient pas totalement libres d'employer leur temps et leurs forces à leur gré. Ils ne pouvaient ni s'en aller ni se soustraire aux ordres reçus. Ils devaient se marier sur le domaine. Une partie de leurs droits personnels appartenaient au seigneur. Par contre et contrairement aux esclaves, ils jouissaient du reste de leurs droits personnels, notamment conjugaux et parentaux ... mais certaines terres, certains offices au service des puissants valaient parfois qu'on s'asservisse pour eux. Certains « postes » de serfs étaient jugés très enviables, au même titre que certains esclaves de personnages puissants pouvaient provoquer des jalousies.

 Le servage était une promotion pour les esclaves, mais une régression pour les personnes libres. En acceptant leur dépendance, celles-ci voyaient s'aliéner une part très significative de leur liberté. En contrepartie, elles faisaient partie d'une communauté villageoise. Les gens des villages, serfs ou libres, n'étaient pas toujours incapables de faire bloc et d'exercer une pression sur leur seigneur, qui avait besoin de leur prospérité matérielle autant qu'ils avaient besoin de sa protection. Ils pouvaient dans une certaine mesure intervenir en tiers entre un serf et lui. En droit comme en fait, il était assez difficile au \emph{dominus} de chasser un serf de sa terre. 
 
 En France l'esclavage a disparu au profit du servage vers le \siecle{8}, et le 3 juillet 1315 Louis Le Hutin a décidé que sont libres tous les esclaves chrétiens qui posent le pied sur le territoire français. 

 Cela n'empêchera pas les Français de recourir à l'esclavage dans leurs colonies avec l'approbation des autorités civiles (cf. \emph{le code noir}) : il suffira d'empêcher les esclaves (et aussi \emph{tous} les noirs) de toucher le sol de France. En effet la survie de l'esclavage sur les terres européennes (surtout dans l'Europe du sud, et d'abord en Espagne) a préparé les esprits a recourir à partir de la Renaissance à l'esclavage des indiens, puis des noirs d'Afrique, pour exploiter des deux Amériques et les autres colonies européennes (Surinam,~etc.). 

 Ce n'est qu'à partir de la fin du \siecle{18} qu'une part significative des intellectuels sont tombés d'accord pour condamner l'esclavage sans aucune circonstance atténuante (cf. \emph{L'Encyclopédie}).


% Le 02.03.2015 :
% ~\%
% ~etc.
% Antiquité
% Moyen Âge


\section{Le clergé chrétien}


 Les règles de recrutement et de discipline cléricale de l'Église se sont précisées au cours des premiers siècles%
% [1]
\footnote{Sources : Georges \fsc{MINOIS}, \emph{Les religieux en Bretagne sous l'Ancien Régime}, 1989. Léo \fsc{MOULIN}, \emph{La vie quotidienne des religieux au Moyen Âge, \siecles{10}{15}}, 1978. Michel \fsc{PARISSE}, \emph{Les nonnes au Moyen Âge}, 1983.}% 
. Dès le \siecle{4} ces règles reflètent l'état définitif de la doctrine permanente, qu'on retrouvera telle quelle et quasi inchangée dans sa formulation du début du \siecle{20} (droit canon de 1917). Selon les Décrétales du Pape Innocent I (401-417) il est interdit d'admettre au diaconat et à la prêtrise :
\begin{enumerate}
%  1°)
\item ceux qui ont épousé une femme non vierge ;
% 2°)
\item ceux qui ont épousé une veuve ;
% 3°)
\item ceux qui ont été mariés deux fois, quelles que soient les circonstances ;
% 4°)
\item ceux qui se sont fait soldats après leur baptême, qui ont accepté de toucher des armes dont certaines ont versé le sang, et surtout ceux qui ont accepté de verser le sang. Même si l'interdiction faite aux chrétiens d'être militaires a été levée par l'Église à partir du ralliement de Constantin (entre 313 et 315), le sang restait sacré, donc impur, et impur aussi celui qui le versait, même pour la bonne cause (chirurgiens) ;
% 5°)
\item ceux qui, magistrats, ont jugé ou plaidé dans des procès où ils ont requis ou prononcé la peine de mort (même motif que le cas précédent : le juge est condamné à faire couler le sang : infliger la \emph{question}, c'est-a-dire torturer, était alors considéré comme nécessaire pour découvrir la vérité, et donc inévitable; infliger des peines mineures comme le fouet; condamner à mort...). Ce n'était pas le risque de l'erreur judiciaire qui était en jeu, c'était encore une fois le contact avec le sang et le contact avec la mort ;
% 6°)
\item les pécheurs qui ont été condamnés à une pénitence (« pécheurs publics », nouveaux infâmes) ;
% 7°)
\item ceux qui ont donné des jeux publics \latin{(munera)}, où du sang (humain ou animal) a coulé ;
% 8°)
\item ceux qui ont exercé des sacerdoces païens, et qui ont donc eux-mêmes sacrifié aux dieux ;
% 9°)
\item ceux qui se sont mutilés eux-mêmes, ce qui vise surtout l'auto castration. Ces derniers ont à la fois versé leur propre sang et mutilé leur corps à l'instar des \latin{galles} (prêtres de Cybèle).
\end{enumerate}

 On peut comparer trait pour trait ces règles avec celles du Lévitique qui régissaient les lévites et les prêtres du Temple de Jérusalem. C'est la même logique. Dans les discussions sur ces sujets les textes de la Tora ont servi d'arguments décisifs. En effet l'imitation du clergé du Temple s'est faite au fil du temps de plus en plus consciente et volontaire. Et pourtant plus il se voulait identique au clergé du Temple, moins le clergé chrétien lui ressemblait ! L'exigence du célibat lui imprimait en effet une physionomie tout à fait inédite. 

 On a vu que la continence perpétuelle était exigée des diacres et prêtres dès les premiers siècles afin qu'ils soient toujours prêts à toucher les « choses sacrées » (vases et linges sacrés, offrandes, pain consacré,~etc.), non souillés par l'impureté rituelle produite par le coït. Ce qui est remarquable c'est que cet argumentaire a emporté l'adhésion. Pourtant l'organisation du service du Temple de Jérusalem montrait une voie de compromis évidente, le service par roulement. D'autre part la notion même de pureté et de souillure religieuse, qui ne se confond pas avec celle de faute morale \emph{(péché)}, avait été mise en question par le Christ lui-même. On peut en déduire que le refus du service par roulement était motivé par des raisons autrement impérieuses que la difficulté de mettre en place un tour de service. 

 Dès l'élection du remplaçant de l'apôtre Judas et l'institution des diacres, les apôtres avaient estimé que personne ne se donne à soi-même une mission%
% [2]
\footnote{Cf. selon le livre des \emph{Actes des Apôtres} les difficultés de Paul de Tarse pour faire admettre par les apôtres sa mission auto proclamée auprès des gentils et ses prétentions au titre d'apôtre.} 
ni ne la tient de sa naissance%
%[3]
\footnote{... de même que nul ne peut (en stricte doctrine) se dire chrétien par sa naissance : il faut que chaque enfant en passe par le baptême, comme le premier converti venu.}% 
, que c'est l'Église qui appelle, et Dieu à travers elle. C'est pourquoi la succession dans le même poste ecclésiastique du père au fils, de l'oncle au neveu, sans être interdite n'a jamais été reconnue comme un droit, au contraire du droit à hériter d'un « honneur », d'une terre ou d'une entreprise, et encore moins comme un modèle. Passés les premiers siècles elle a au contraire été vue comme une irrégularité grosse de dangers. 

 Si la haute administration de l'Empire romain tardif et des royaumes barbares qui lui ont succédé est devenue vers le \siecle{10} la noblesse héréditaire du Moyen Âge, c'est parce que ceux qui étaient nommés par les autorités civiles à un emploi public ont fini par obtenir le droit de désigner eux-mêmes leur successeur, ce qui signifie que « l'honneur » (responsabilités et biens servant à les rémunérer) qui leur avait été conféré par les souverains est entré dans leur patrimoine, à la faveur de l'affaiblissement de ces mêmes souverains, système d'où est sortie la \emph{féodalité}. Au même moment un clergé marié aurait eu les mêmes chances de devenir héréditaire, et le risque eut été grand de voir se constituer une caste sacerdotale à côté de la caste aristocratique, à la mode indienne ou hébraïque. On connaît d'ailleurs un certain nombre de grandes familles de l'Antiquité et du haut Moyen Âge dont des membres se sont succédé sur le même siège épiscopal pendant plusieurs générations : Sylvère, pape de 536 à 537, était le fils légitime d'Horsmidas, pape de 514 à 523 (né avant son ordination). D'autre part les souverains et autres puissants du \siecle{6} et des siècles suivants ont utilisé leur influence pour conférer l'épiscopat à des serviteurs laïcs afin de les récompenser pour leurs loyaux services, ou bien pour les neutraliser par cet « honneur » particulier, qui leur interdisait (en principe) tout retour aux armes. 

 Si les membres d'une échelle hiérarchique peuvent donner leur poste à un héritier c'est qu'ils en sont devenus propriétaires et c'est toujours au détriment du sommet de la hiérarchie, désormais obligé de composer avec une autre source de légitimité qu'elle-même. Inversement, c'est toujours pour défendre ou renforcer son autorité qu'un souverain refuse que soit limité son pouvoir de nommer et de démettre.

 Au contraire, du point de vue d'une institution, le célibat est idéal :
\begin{enumerate}
% 1°)
\item Un clerc célibataire est plus disponible puisqu'il n'a pas à plaire à sa femme, ni à s'occuper de ses enfants (cf. Paul de Tarse).
% 2°)
\item Un clerc célibataire et sans enfants a moins de besoins matériels qu'un clerc marié et donc il \emph{peut} coûter moins cher. La continence des clercs est d'abord économique en ce qu'il n'y a pas à constituer de dot pour les filles ni à établir les garçons ...
% 3°)
\item ... qui pourraient prétendre avoir des droits sur le poste de leur père.
% 4°)
\item N'ayant pas à craindre pour ses proches, ni à les établir dans la vie, un clerc sans attaches familiales est moins sensible aux pressions et séductions venant de la société civile.
% 5°)
\item Par ailleurs il serait inconvenant que des histoires de famille puissent interférer dans les affaires de l'Église.
% 6°)
\item Enfin une paroisse, un diocèse, un monastère ne sont pas des bâtiments ni des biens fonciers. Ces institutions sont des ensembles de fidèles, et pour l'Église aucun groupe de fidèles, c'est-à-dire d'âmes immortelles, ne peut appartenir à une personne, ou à une famille, à la façon dont la force de travail des serfs (mais non leurs âmes) appartenait à leur seigneur.
\end{enumerate} 

 Voilà pourquoi la doctrine ecclésiale a toujours voulu, malgré toutes les pesanteurs individuelles et collectives qui ont entrainé de nombreux écarts, que les clercs ne soient pas issus de familles de clercs, mais issus du monde des laïcs. Et voilà pour eux autant de raisons très concrètes d'attribuer une valeur spirituelle au célibat et à la continence perpétuelle, tandis que les religieux montraient par leur exemple que cet idéal n'était pas inatteignable. 

 Cela a eu des conséquences très importantes sur la société toute entière. En effet il s'est constitué en son sein une caste non héréditaire recrutée dans les autres castes, cultivant le savoir et la culture, et au sein de laquelle les carrières n'étaient pas déterminées par la naissance, même si ces deux idéaux, toujours poursuivis, n'ont jamais été totalement atteints. La culture cultivée dans les institutions d'Église souffrait de limitations certaines et l'héritage antique n'avait pas été transmis sans pertes. Tous les clercs n'étaient pas savants, et les plus compétents n'obtenaient pas toujours les promotions auxquelles leurs talents les auraient qualifiés. De même tous les princes de l'Église n'étaient pas à la hauteur de leur charge. Mais c'est au sein du corps des moines et des prêtres que se trouvaient les plus savants de leur époque, et pour ceux qui n'avaient pas les privilèges de la naissance c'est au sein de l'Église qu'ils avaient le plus de chances de promotions. 
 
 Joseph \fsc{MORSEL} voit dans cet élitisme ecclésiastique et le modèle qu'elle a fourni, profondément intériorisé, l'une des causes principales du développement ultérieur de l'Europe et de son avance sur les autres civilisations (in \emph{L'Histoire du Moyen Âge est un sport de combat}, texte publié au format pdf sur Internet à l'adresse \url{http://lamop.univ-paris1.fr/IMG/pdf/SportdecombatMac.pdf}).
 
 \section{Les religieux}
 Le mouvement monastique s'est développé depuis les premiers ermites qui ont fui le monde dès le \siecle{3} dans les déserts d'Égypte, et les premières veuves et vierges consacrées qui en ont fait autant à l'ombre des cathédrales, sous la protection des évêques. Il continuera de se développer à un rythme soutenu jusqu'au foisonnement de la fin du Moyen Âge. Il prouvait par sa floraison que la continence \emph{perpétuelle} était possible%
% [2]
\footnote{... même si elle doit parfois s'appuyer sur \emph{l'impuissance de famine}, cf. Aline \fsc{Rousselle}, 1998, p. 203 - 224}% 
, et cela non seulement pour les femmes, de qui depuis toujours on l'exigeait au gré des besoins de leur famille, mais aussi pour les hommes. Saint Augustin, évêque de la fin du \siecle{4} et du début du \siecle{5}, vivait en communauté avec ses collaborateurs immédiats, communauté d'où sortiront un jour les chapitres de chanoines présents dans toutes les cathédrales. Au même moment les hôpitaux s'organisaient dans l'esprit des monastères. Ils étaient construits comme des églises dans lesquelles seraient logés des malades et si l'on en croit le concile de Nicée leur personnel était recruté parmi les religieux.

 S'appuyant sur les lettres de Paul de Tarse et les paroles du Christ, l'Église défendait le droit des jeunes de consacrer volontairement et librement leur vie à Dieu, alors qu'ils étaient encore \emph{dans la main} de leur père. Dans ce cas elle défendait leur droit de recevoir leur part d'héritage sans pour autant suivre la voie prévue par leurs parents, part d'héritage sans laquelle leur liberté de choix serait restée formelle. Cela leur permettait de s'engager dans le monastère ou l'hôpital de leur choix en faisant don à leur communauté de leur part d'héritage%
% [3]
\footnote{C'est ainsi qu'était mis en pratique la proposition de donner tous leurs biens aux pauvres faite par le Christ à ceux qui voulaient choisir la perfection (parabole du « jeune homme riche ») : en effet leur nouvelle famille spirituelle n'était constituée que de membres qui avaient fait vœu de pauvreté.}% 
. On peut supposer que ce n'est pas par hasard qu'en 320 Constantin avait abrogé les lois d'Auguste qui exigeaient d'avoir engendré trois enfants et d'être marié pour recevoir les héritages venant de personnes éloignées, et qu'il avait posé des limites au droit des pères de déshériter un enfant. Contrainte par sa propre logique, et fidèle sur ce point au droit romain, l'Église plaidait pour le consentement mutuel des fiancés et contre l'idée que celui de leurs parents était nécessaire pour que leur mariage soit valide%
%[4]
\footnote{Là aussi elle allait contre l'autorité des pères. Cet enseignement-là restait en travers de la gorge de bien des pères, mais aussi des ecclésiastiques eux-mêmes pour autant qu'ils s'identifiaient aux intérêts temporels de leur famille d'origine, cf. les avanies subies par Abélard, alors qu'il était encore laïc et donc épousable, du fait de l'ecclésiastique qui était oncle et tuteur d'Héloïse.}% 
.

 Les revenus des monastères, des évêchés et des hôpitaux étaient fondés sur des propriétés, terres, domaines,~etc., provenant des dons et des legs. Grâce aux rentes sur la terre%
% [5]
\footnote{Ressentie de l'Antiquité à la fin du Moyen Âge (au moins) comme le seul bien qui ne fait jamais défaut, et dont les fruits permettent de survivre quelle que soit la catastrophe économique qui puisse arriver (Paul \fsc{Veyne}, \emph{La société romaine}, chapitre).} 
et les immeubles (en nature ou en argent) il était possible sans recourir à l'impôt de « fonder » (en principe une fois pour toutes) des emplois \emph{(bénéfices)} de clercs, des écoles, des hôpitaux, des monastères,~etc. Ce mode de financement était hérité de l'Antiquité pré chrétienne. C'était déjà celui des temples païens. S'ajoutaient à ces revenus des contributions régulières notamment les différentes \emph{dîmes} versées par les fidèles, d'abord volontaires, puis obligatoires. Ainsi les institutions ecclésiastiques étaient autonomes et auto-suffisantes, sans courir les risques du marché, ni dépendre étroitement de généreux donateurs ou des pouvoirs locaux. Ce système ne faisait peser aucune charge récurrente sur le budget de la puissance publique et donnait aux institutions un maximum de liberté face aux pressions des pouvoirs publics. 

 Jusqu'à la fin du Moyen Âge une part de presque tous les héritages était donnée aux pauvres (c'est-à-dire à leur protectrice officielle : l'Église) pour \emph{le salut de l'âme} des donateurs. Il existait déjà chez les anciens des fondations identiques auprès des temples païens. Quant aux barbares ils admettaient comme les Égyptiens, les Celtes et les Germains que chaque mort emporte dans son tombeau des biens pour l'au-delà, ce qui du point de vue des chrétiens ou des juifs était un signe de superstition. Cette part des biens du mourant qu'il comptait emporter avec lui (jusqu'à un tiers de sa fortune ?) l'Église lui proposait d'en faire meilleur usage, en l'investissant dans les \emph{œuvres pies} (pieuses). 

 Selon Raymond Goody il y avait un lien entre la défense par l'Église de la liberté de choix de vie des jeunes, celle du droit des jeunes à une part d'héritage même en cas de désaccord paternel, celle des chrétiens à faire des donations (notamment dans leur testament) et le financement des institutions religieuses qui fournissaient les lieux où chercher la perfection. Selon lui la nécessité de trouver des ressources pour faire vivre les paroisses, monastères et hôpitaux a exercé une pression déterminante sur la définition même des règles du droit de la famille. Elle aurait contribué à ce que le droit de l'Église mette des limites au droit des pères à imposer leur volonté à leurs enfants. Elle aurait aussi et surtout contribué à étendre les degrés de parenté interdisant les mariages. Même si cette thèse paraît un peu extrême, comme toute thèse qui attribue à une cause unique un mouvement observable sur plus de dix siècles, elle n'en contient pas moins une part de vérité significative. 

 En dehors du travail de leurs membres, qui exigeait lui-même un minimum d'outils de production et d'abord de terres, le financement des monastères reposait sur les \emph{dots} des postulants, notamment dans les monastères féminins qui ne pouvaient bénéficier comme les monastères d'hommes des honoraires de messes offertes pour le repos de l'âme des défunts. Au décès du religieux sa dot demeurait acquise au monastère (du moins tant qu'elle a consisté en un capital et non en une rente). Celui-ci avait donc des chances de voir grossir peu à peu son capital. Cela permettait (dans les meilleurs des cas) d'accepter les postulants sans le sou et de consacrer le superflu au service des pauvres et des malades.
 

\chapter{Les religieux}
% Le 02.03.2015 :
% ~\%
% ~etc.
% Antiquité
% Moyen Âge



 Le mouvement monastique s'est développé depuis les premiers ermites qui ont fui le monde dès le \siecle{3} dans les déserts d'Égypte, et les premières veuves et vierges consacrées qui en ont fait autant à l'ombre des cathédrales, sous la protection des évêques. Il continuera de se développer à un rythme soutenu jusqu'au foisonnement de la fin du Moyen Âge. Il prouvait par sa floraison que la continence \emph{perpétuelle} était possible%
% [2]
\footnote{... même si elle doit parfois s'appuyer sur \emph{l'impuissance de famine}, cf. Aline \fsc{Rousselle}, 1998, p. 203 - 224}% 
, et cela non seulement pour les femmes, de qui depuis toujours on l'exigeait au gré des besoins de leur famille, mais aussi pour les hommes. Saint Augustin, évêque de la fin du \siecle{4} et du début du \siecle{5}, vivait en communauté avec ses collaborateurs immédiats, communauté d'où sortiront un jour les chapitres de chanoines présents dans toutes les cathédrales. Au même moment les hôpitaux s'organisaient dans l'esprit des monastères. Ils étaient construits comme des églises dans lesquelles seraient logés des malades et si l'on en croit le concile de Nicée leur personnel était recruté parmi les religieux.

 S'appuyant sur les lettres de Paul de Tarse et les paroles du Christ, l'Église défendait le droit des jeunes de consacrer volontairement et librement leur vie à Dieu, alors qu'ils étaient encore \emph{dans la main} de leur père. Dans ce cas elle défendait leur droit de recevoir leur part d'héritage sans pour autant suivre la voie prévue par leurs parents, part d'héritage sans laquelle leur liberté de choix serait restée formelle. Cela leur permettait de s'engager dans le monastère ou l'hôpital de leur choix en faisant don à leur communauté de leur part d'héritage%
% [3]
\footnote{C'est ainsi qu'était mis en pratique la proposition de donner tous leurs biens aux pauvres faite par le Christ à ceux qui voulaient choisir la perfection (parabole du « jeune homme riche ») : en effet leur nouvelle famille spirituelle n'était constituée que de membres qui avaient fait vœu de pauvreté.}% 
. On peut supposer que ce n'est pas par hasard qu'en 320 Constantin avait abrogé les lois d'Auguste qui exigeaient d'avoir engendré trois enfants et d'être marié pour recevoir les héritages venant de personnes éloignées, et qu'il avait posé des limites au droit des pères de déshériter un enfant. Contrainte par sa propre logique, et fidèle sur ce point au droit romain, l'Église plaidait pour le consentement mutuel des fiancés et contre l'idée que celui de leurs parents était nécessaire pour que leur mariage soit valide%
%[4]
\footnote{Là aussi elle allait contre l'autorité des pères. Cet enseignement-là restait en travers de la gorge de bien des pères, mais aussi des ecclésiastiques eux-mêmes pour autant qu'ils s'identifiaient aux intérêts temporels de leur famille d'origine, cf. les avanies subies par Abélard, alors qu'il était encore laïc et donc épousable, du fait de l'ecclésiastique qui était oncle et tuteur d'Héloïse.}% 
.

 Les revenus des monastères, des évêchés et des hôpitaux étaient fondés sur des propriétés, terres, domaines,~etc., provenant des dons et des legs. Grâce aux rentes sur la terre%
% [5]
\footnote{Ressentie de l'Antiquité à la fin du Moyen Âge (au moins) comme le seul bien qui ne fait jamais défaut, et dont les fruits permettent de survivre quelle que soit la catastrophe économique qui puisse arriver (Paul \fsc{Veyne}, \emph{La société romaine}, chapitre).} 
et les immeubles (en nature ou en argent) il était possible sans recourir à l'impôt de « fonder » (en principe une fois pour toutes) des emplois \emph{(bénéfices)} de clercs, des écoles, des hôpitaux, des monastères,~etc. Ce mode de financement était hérité de l'Antiquité pré chrétienne. C'était déjà celui des temples païens. S'ajoutaient à ces revenus des contributions régulières notamment les différentes \emph{dîmes} versées par les fidèles, d'abord volontaires, puis obligatoires. Ainsi les institutions ecclésiastiques étaient autonomes et auto-suffisantes, sans courir les risques du marché, ni dépendre étroitement de généreux donateurs ou des pouvoirs locaux. Ce système ne faisait peser aucune charge récurrente sur le budget de la puissance publique et donnait aux institutions un maximum de liberté face aux pressions des pouvoirs publics. 

 Jusqu'à la fin du Moyen Âge une part de presque tous les héritages était donnée aux pauvres (c'est-à-dire à leur protectrice officielle : l'Église) pour \emph{le salut de l'âme} des donateurs. Il existait déjà chez les anciens des fondations identiques auprès des temples païens. Quant aux barbares ils admettaient comme les Égyptiens, les Celtes et les Germains que chaque mort emporte dans son tombeau des biens pour l'au-delà, ce qui du point de vue des chrétiens ou des juifs était un signe de superstition. Cette part des biens du mourant qu'il comptait emporter avec lui (jusqu'à un tiers de sa fortune ?) l'Église lui proposait d'en faire meilleur usage, en l'investissant dans les \emph{œuvres pies} (pieuses). 

 Selon Raymond Goody il y avait un lien entre la défense par l'Église de la liberté de choix de vie des jeunes, celle du droit des jeunes à une part d'héritage même en cas de désaccord paternel, celle des chrétiens à faire des donations (notamment dans leur testament) et le financement des institutions religieuses qui fournissaient les lieux où chercher la perfection. Selon lui la nécessité de trouver des ressources pour faire vivre les paroisses, monastères et hôpitaux a exercé une pression déterminante sur la définition même des règles du droit de la famille. Elle aurait contribué à ce que le droit de l'Église mette des limites au droit des pères à imposer leur volonté à leurs enfants. Elle aurait aussi et surtout contribué à étendre les degrés de parenté interdisant les mariages. Même si cette thèse paraît un peu extrême, comme toute thèse qui attribue à une cause unique un mouvement observable sur plus de dix siècles, elle n'en contient pas moins une part de vérité significative. 

 En dehors du travail de leurs membres, qui exigeait lui-même un minimum d'outils de production et d'abord de terres, le financement des monastères reposait sur les \emph{dots} des postulants, notamment dans les monastères féminins qui ne pouvaient bénéficier comme les monastères d'hommes des honoraires de messes offertes pour le repos de l'âme des défunts. Au décès du religieux sa dot demeurait acquise au monastère (du moins tant qu'elle a consisté en un capital et non en une rente). Celui-ci avait donc des chances de voir grossir peu à peu son capital. Cela permettait (dans les meilleurs des cas) d'accepter les postulants sans le sou et de consacrer le superflu au service des pauvres et des malades.
 
 
 


 J'appelle « mariage constantinien » le mariage romain tel qu'il a été modifié par Constantin et ses successeurs pour l'accommoder aux conceptions chrétiennes, sans pour autant le modeler sur elles. En effet jusqu'à la Réforme Grégorienne (\siecle{11}) l'Église n'avait pas le monopole du droit familial, et les autorités civiles ne se sentaient pas obligées d'appuyer toutes ses prétentions dans un domaine aussi critique pour la transmission du pouvoir. Dans les deux ou trois siècles où l'Église détiendra le monopole du droit familial les rois ne cesseront de le lui contester, avec de plus en plus de succès. Pendant la longue durée de « chrétienté » les laïcs n'ont jamais suivi la discipline ecclésiastique sans une certaine dose d'ambivalence. Les écarts entre le droit religieux (droit \emph{canon}) et les lois civiles n'ont jamais été nuls, pour ne pas parler de l'\emph{à peu près} avec lequel ces lois étaient respectées. Ce que j'appelle le mariage constantinien est donc un modèle qui n'a jamais été pleinement réalisé, et surtout pas sous Constantin. Pourtant ce modèle tendra peu à peu à s'incarner dans les pratiques et les représentations, et il ne sera peut-être jamais aussi respecté que durant les derniers siècles de notre ancien régime.

 Dans le mariage constantinien plusieurs fonctions distinctes sont télescopées sur une seule personne : un époux est à la fois le détenteur des droits juridiques de son épouse (son curateur), son amant, le géniteur de ses enfants, le détenteur des droits de ces enfants mineurs, et le responsable de leur éducation (c'est-à-dire leur père légal). Symétriquement, une épouse est la seule femme capable de donner à son époux des enfants légitimes, des héritiers, quel que soit le nombre de ses concubines. Chaque enfant légitime ne peut être que l'enfant biologique de ses parents légaux (leur enfant « naturel » au sens antique du terme). Seuls les enfants légitimes ont droit à une part d'héritage, par définition.

 Le Bas-Empire et le haut Moyen-Âge continuent de reconnaître sans discussion la validité des concubinages stables monogames non incestueux, et la légitimité civile et religieuse des enfants qui en naissent : Justinien les autorise à hériter, mais il ne fait ainsi que rappeler une règle de droit ancienne. Dans la pratique du Bas-Empire le concubinage monogame est une forme de mariage souvent (presque toujours ?) employée par les personnes qui ne possèdent pas de patrimoine significatif et ne voient donc pas la nécessité de s'unir en public et solennellement ni de passer devant un notaire. Dans le même sens Augustin d'Hippone enseigne qu'une concubine qui se veut fidèle à son concubin lui est mariée devant Dieu de manière aussi légitime qu'une épouse en titre.

 Pour l'Église c'est le mariage qui fonde la famille et non la présence des enfants, même si elle met l'accueil des enfants au premier rang des « fins du mariage ». De son point de vue, le mariage crée en effet \emph{dès sa célébration} une parenté nouvelle \emph{que les époux soient féconds ou non.} Cette parenté « par alliance » a donc des effets directs sur les membres des parentèles des époux (frères, sœurs, etc.) : elle étend le cercle des partenaires qui leur sont désormais définitivement interdits, même si l'un des époux décède.

 Selon la doctrine chrétienne, identique sur ce point au droit romain, ce sont les époux qui s'unissent l'un à l'autre : cela implique qu'ils soient capables de discernement (âge suffisant, santé mentale) et libres de leur personne : célibataires ou veufs, non esclaves, non engagés par contrat dans une entreprise qui empêcherait la vie commune, à l'abri de toute pression, libres de tout vœux religieux, sexuellement aptes au mariage. L'Église a toujours soutenu contre les parents que les jeunes gens ont le pouvoir de se marier validement sans leur accord, même si elle admettait qu'en leur désobéissant ces jeunes gens les déliaient de leur devoir de les établir dans la vie. 

 Contrairement au droit romain l'Église en est progressivement venue à ne reconnaître la réalité juridique d'un mariage que lorsqu'il a été consommé, assez probablement parce que chez les barbares (cf. chapitre suivant), même christianisés, les unions se construisaient en plusieurs étapes séparées par de très longs intervalles, les premières étapes (dont les promesses de fiançailles) ayant parfois lieu alors que les futurs époux étaient encore de très jeunes enfants. Pour les besoins des procès en nullité de mariage il a fallu trouver un critère décisif dans cette progression, et c'est la consommation du mariage qui a été retenue. 

 Selon les évêques et théologiens chrétiens, le célibat non consacré est licite, mais chez les jeunes gens sans enfants, en bonne santé et disposant de moyens matériels suffisants, il est suspect d'égoïsme, de libertinage ou de désirs « contraires à la nature » (homosexualité notamment dont la mise en acte a toujours été condamnée moralement, même si elle n'a été semble-t-il que rarement sanctionnée). Quels que soient les préférences individuelles la copulation n'est légitime que dans l'état de mariage monogame. D'autre part le mariage est le seul moyen légitime de répondre à l'ordre divin (\emph{croissez et multipliez} de la Genèse). C'est donc l'état normal de tous ceux qui ne sont pas ordonnés à un ministère ou engagés dans la vie religieuse. Mais comme la fin première du mariage est la procréation d'enfants légitimes les remariages sont déconseillés (quoique autorisés) quand cette fin est à priori inatteignable étant donné l'âge ou l'état de santé des conjoints. 

 À partir du \siecle{4} dans l'empire romain, ce n'est plus d'abord et avant tout par la relation de pouvoir qu'il exerce sur les membres de sa maison que le père est juridiquement défini. En effet, il est soumis au devoir de \emph{piété}%
% [2] 
\footnote{La piété était l'affection réciproque et le respect mutuel entre les divers membres de la famille nucléaire, y compris le devoir d'assistance.} 
à l'égard de ses enfants au même titre qu'ils le sont à son égard, et autant qu'eux. D'autre part chez les romains (mais pas chez tous les barbares alliés à Rome) en cas de décès du père, c'est la mère qui, à partir de 390, exerce la tutelle de ses enfants mineurs (et d'eux seuls) si elle a cinquante ans et plus, et du moins tant qu'elle ne se remarie pas, ce qui est le cas général, indépendamment même des réserves ecclésiastiques face au remariage des veuves dotées d'enfants. À partir de 390 une femme n'est plus considérée comme incapable par nature de représenter juridiquement une autre personne qu'elle-même. Le fait qu'à partir de ce moment elle puisse exercer, de droit, la tutelle de ses enfants (même si c'est sous le contrôle éventuel et plus ou moins étroit de la famille de son mari), manifeste que les droits et les devoirs dits « paternels » sont en réalité dès ce moment ceux du couple parental, même si tant qu'il vit c'est le mari qui représente le couple face au monde extérieur%
%[3]
\footnote{... et ce sera le cas jusqu'aux années 60 du \siecle{20}.}% 
. Au fil des siècles, la mise en pratique de ce principe a varié de pays en pays en fonction de nombreux facteurs. Il est probable que plus l'héritage était mince et la famille de petite importance, plus le droit de la veuve non remariée à exercer en toute liberté la tutelle de ses enfants mineurs lui était reconnu, et inversement. Ainsi dans les familles riches et puissantes, il pouvait y avoir tellement d'intérêts matériels ou politiques en jeu que la veuve n'avait pas forcément beaucoup d'impact sur l'éducation de celui de ses fils qui devait prendre la succession de son mari dans ses fonctions publiques. 

 Jusqu'à Constantin la fécondité de chaque femme mariée appartenait sans limites à son mari. Désormais elle ne lui appartient plus. Il n'est plus permis de se débarrasser des enfants non voulus par l'avortement ou par l'infanticide. Sauf indigence extrême il n'est pas non plus permis de s'en débarrasser par l'exposition ni la vente. Une femme n'a donc plus autant à craindre qu'auparavant qu'on ne l'oblige contre son gré à avorter ou à abandonner son nouveau-né. Mais sa fécondité ne lui appartient pas non plus. Pas plus que son mari elle n'a droit de vie ou de mort sur l'enfant qu'elle porte. Chacun des époux reconnaît aussi à l'autre un droit sur son propre corps. Les deux époux se doivent réciproquement fidélité : c'est un devoir \emph{moral} pour l'homme autant que pour sa femme, et même si ses propres infidélités ne sont pas sanctionnées par la loi tout est fait pour qu'il n'ait aucun intérêt à entretenir des maîtresses%
% [1]
\footnote{Cela ne l'empêche évidemment pas d'avoir des rapports avec des prostitué(e)s, rapports qui par nature ne s'inscrivent pas dans la durée.}% 
. Chacun des deux époux a l'obligation de satisfaire autant qu'il est en son pouvoir les désirs sexuels de l'autre, ce qui veut dire que l'épouse doit accepter les étreintes de son mari, quoi qu'elle puisse penser des risques de grossesse et de santé à quoi cela l'expose, et quels que soient ses propres désirs. Ceci dit la modération est prêchée aux maris, qui se voient prescrire la continence de nombreux jours par an. Le \emph{devoir conjugal} n'est par ailleurs exempt de faute morale que si aucun obstacle n'est mis à la fécondation. Les seuls moyens de contrôle des naissances autorisés par l'Église sans restriction ni réticences sont le retard de l'âge au mariage, le célibat et la continence. 

 Il n'est plus possible en principe (mais ce principe a mis de nombreux siècles à s'imposer en dépit de la lutte constante de l'Église) de répudier une épouse présumée stérile (en cas de stérilité dans un couple, c'est celle de la femme qui est toujours suspectée en premier). Ou bien les hommes ont la chance de vivre un mariage fécond et de voir au moins l'un de leurs fils légitimes atteindre l'âge adulte pour leur succéder, ou bien ils doivent renoncer à tout héritier direct tant que vit leur épouse%
% [4]
\footnote{Pour les maris les moins patients il ne restait plus que le « divorce à la carolingienne », c'est-à-dire l'assassinat de l'épouse. Cela ne pouvait se faire que si les institutions policières étaient faibles et les parents de l'épouse moins puissants que ceux du mari. Plus l'État était déliquescent, plus il était possible, comme toujours, de prendre des libertés avec toutes les règles de droit, à la condition de disposer de la force.}% 
. Les couples stériles (dont le nombre n'était pas du tout négligeable jusqu'à l'avènement de la médecine moderne, 20~\% environ) et ceux dont aucun enfant n'a atteint vivant l'âge adulte, sont invités à consacrer aux bonnes œuvres, aux pauvres et à l'Église les ressources qu'ils auraient transmises à leurs héritiers s'ils en avaient eus.

 Tous les enfants nés hors mariage sont pénalisés. En principe il n'est plus possible pour un homme de se faire des héritiers sans se marier, même si la prise en charge d'\emph{alumnii} et leur installation dans l'existence reste une bonne œuvre. La \emph{légitimation par mariage subséquent} est désormais la seule exception de plein droit%
% [5] 
\footnote{... jusqu'au \siecle{20}. Les enfants irréguliers légitimés par les empereurs, les rois ou les papes, ne l'étaient pas de plein droit mais à la faveur d'une grâce, qui pouvait toujours être refusée sans justification, et n'allait pas sans contreparties coûteuses.} 
à la pénalisation des enfants nés hors mariage, et ses conditions sont strictes. Chacun, quelque puissant qu'il soit, doit savoir que s'il a l'imprudence de faire un enfant hors mariage ou dans un mariage contesté par son curé, ou par son évêque, ou par son seigneur, par le roi ou par sa propre parentèle, il ne pourra pas le faire reconnaître comme un de ses héritiers sans combat ou sans procès. Cet enfant ne pourra sans doute pas lui succéder. L'exhérédation totale ou partielle des enfants illégitimes est restée jusqu'à la fin du \siecle{20} le premier frein apporté au désir des hommes de se procurer une descendance ailleurs qu'avec leur épouse légitime, même si d'innombrables exemples montrent que cette règle a mis des siècles à s'imposer.

 Et la perspective de se retrouver avec un enfant à charge, seule, sans aucun espoir d'une légitimation (ni même d'une aide significative venant du père de l'enfant lorsqu'il était déjà marié puisque aucune donation au-delà des frais d'éducation n'était plus autorisée depuis Constantin) a été un obstacle majeur à l'exercice d'une sexualité féminine en dehors du mariage ou avant le mariage. 

 Mais les épouses savent aussi qu'il est devenu, sinon impossible, du moins difficile de les chasser de leur maison ou de leur imposer de cohabiter avec une concubine%
% [6]
\footnote{... même si pour ceux dont la puissance excède de beaucoup celle du commun des mortels, aristocrates, rois, la question peut se présenter différemment, et si les amours ancillaires sont de tous les temps.}% 
. Elles sont à peu près assurées que les infidélités de leur époux n'entraîneront de conséquences graves ni sur elles, ni sur leurs enfants, ni sur le futur héritage de ceux-ci. Tout au plus des « aliments » devront-ils être versés aux enfants nés de leurs maîtresses, mais cela ne portera que sur d'assez petites sommes et seulement jusqu'à ce qu'ils soient mis au travail : 8-10 ans. Il n'est plus question de financer leur établissement dans la vie. 

 À l'occasion du sac de Rome et des horreurs qu'il a entraînées, Augustin reprend à son compte l'idée (banale à son époque) que le viol ne « souille » pas la victime, mais seulement son auteur. Il en tire la conclusion qu'il n'est donc pas question que la victime soit punie pour un acte auquel elle n'a pas consenti : \emph{La sainteté du corps ne consiste pas à préserver nos membres de toute altération et de tout contact... Ainsi donc, tant que l'âme garde ce ferme propos qui fait la sainteté du corps, la brutalité d'une convoitise étrangère ne saurait ôter au corps le caractère sacré que lui imprime une continence persévérante... Nous soutenons que lorsqu'une femme, décidée à rester chaste, est victime d'un viol sans aucun consentement de sa volonté, il n'y a de coupable que l'oppresseur... J'admire beaucoup cette parole d'un rhéteur qui déclamait sur Lucrèce : « Chose admirable ! » s'écriait-il ; « ils étaient deux ; et un seul fut adultère ! » \emph{[celui qui viola Lucrèce]} Impossible de dire mieux et plus vrai...}

 \emph{Mais d'où vient que la vengeance est tombée plus terrible sur la tête innocente \emph{[sur Lucrèce qui s'est suicidée]} que sur la tête coupable? Car Sextus \emph{[son violeur]} n'eut à souffrir que l'exil avec son père, et Lucrèce perdit la vie. S'il n'y a pas impudicité à subir la violence, y-a-t-il justice à punir la chasteté ? ... Quant à nous, pour réfuter ces hommes étrangers à toute idée de sainteté qui osent insulter les vierges chrétiennes outragées dans la captivité, qu'il nous suffise de recueillir cet éloge donné à l'illustre Romaine : « Ils étaient deux, un seul fut adultère ». On n'a pas voulu croire, tant la confiance était grande dans la vertu de Lucrèce, qu'elle se fût souillée par la moindre complaisance adultère. Preuve certaine que, si elle s'est tuée pour avoir subi un outrage auquel elle n'avait pas consenti, ce n'est pas l'amour de la chasteté qui a armé son bras, mais bien la faiblesse de la honte. Oui, elle a senti la honte d'un crime commis sur elle, bien que sans elle. Elle a craint, la fière Romaine, dans sa passion pour la gloire, qu'on ne pût dire, en la voyant survivre à son affront, qu'elle y avait consenti. À défaut de l'invisible secret de sa conscience, elle a voulu que sa mort fût un témoignage écrasant de sa pureté, persuadée que la patience serait contre elle un aveu de complicité. Telle n'a point été la conduite des femmes chrétiennes qui ont subi la même violence. Elles ont voulu vivre, pour ne point venger sur elles le crime d'autrui, pour ne point commettre un crime de plus, pour ne point ajouter l'homicide \emph{(d'elles-même)} à l'adultère; c'est en elles-mêmes qu'elles possèdent l'honneur de la chasteté, dans le témoignage de leur conscience ; devant Dieu, il leur suffit d'être assurées qu'elles ne pouvaient rien faire de plus sans mal faire, résolues avant tout à ne pas s'écarter de la loi de Dieu, au risque même de n'éviter qu'à grand-peine les soupçons blessants de l'humaine malignité.} (\emph{La Cité de Dieu} livre 1, chapitres 18 et 19) 

 Augustin s'y prend de manière paradoxale pour défendre les victimes de viol. Il reprend d'abord l'idée, banale à son époque (cf. la parole du rhéteur), que la pénétration de leur corps contre leur gré ne les a pas moralement souillées, et qu'elles n'ont à se sentir coupables de rien. Mais la honte d'avoir subi le viol sans pouvoir l'empêcher est toujours ressentie par les victimes de quelque âge et sexe qu'elles soient. « Faire honte » est d'ailleurs l'un des buts de certains agresseurs, indépendamment du plaisir sexuel qu'ils peuvent prendre dans ces actes. Augustin qualifie cette honte de « faiblesse », de sentiment compréhensible sinon évitable, mais non fondé en raison, et contre lequel il convient de lutter. Ce faisant il reprend les argumentations traditionnelles. À cela il ajoute l'interdiction du suicide, que celui-ci soit de honte, de protestation ou de désespoir. 

 Le suicide a toujours été condamné par l'Église, comme il l'était en général par la Tora : \emph{comme un péché grave, sauf chez les « fous » ou les victimes d'un « grand chagrin » selon le \emph{premier concile de Braga} qui s'est tenu vers 561. Il s'agissait alors pour l'Église de marquer une différence avec la mentalité héritée de la civilisation romaine qui voyait dans le suicide une mort comme une autre pour le désespéré et une voie honorable, un moyen de rachat pour le criminel}. (Wikipédia). 

 En refusant que le suicide soit une solution acceptable face à la détresse et à la douleur morale éprouvées par les victimes de viol, Augustin posait sur leurs épaules un fardeau qui a pu être insupportable à certaines. Pourtant en leur faisant un devoir de survivre, il déniait aussi aux victimes collatérales (époux, enfants, parentèles, voisins) le droit de les tuer ou de les pousser au suicide pour apaiser leur propre honte de n'avoir pas été capables, eux non plus, d'empêcher le forfait.
 
 

% 28.02.2015 :
% haut Moyen Âge
% _, --> ,
% Antiquité
% ~etc.
% ~\%


\chapter{Familles de chair}


 À partir de Constantin les lois de l'Empire, puis celles des royaumes qui en Occident ont repris sa succession, se sont lentement alignées sur les conceptions chrétiennes du mariage et de la génération%
% [1]
\footnote{Cf. Georges \fsc{DUBY}, \emph{Le chevalier, la femme et le prêtre}, 1981.}% 
. Mais dans le même temps la vie familiale à la romaine était également mise à mal par les « barbares ». Ceux-ci ont introduit des pratiques différentes, principalement germaines%
%[2]
\footnote{Jean-Pierre \fsc{POLY}, \emph{Le chemin des amours barbares, Genèse médiévale de la sexualité européenne}, 2003.}% 
, sur les territoires de l'ancien Empire romain d'Occident. Le haut Moyen Âge est un temps de conflits, de coexistence et de compromis entre les droits et coutumes des royaumes « barbares » et le droit romain%
%[3]
\footnote{Cf. Pierre \fsc{PETOT}, \emph{La famille}, 1992.}% 
. On constate l'effacement progressif des traditions juridiques romaines, compensé dans une large mesure par l'élaboration (ou la résurgence) de pratiques non romaines, dites \emph{coutumières}, propres à chaque lieu et caractérisées d'abord par une très grande variété%
%[4]
\footnote{Mais lorsqu'il s'agira à partir du \siecle{12} de reconstruire un droit unifié et cohérent, à la fois dans le domaine civil \emph{(droit civil)} et dans le domaine religieux \emph{(droit Canon)}, c'est au \emph{Code de Justinien}, publié en 529 et 534, que les juristes savants vont se référer.}% 
. 

 Toutes ces réserves étant faites, et malgré une infinité de particularités tenant aux lieux et aux temps, les lignes de force du système articulant en \emph{chrétienté} les familles, les autorités civiles et les institutions d'Église sont demeurées les mêmes au-delà du Moyen Âge%
% [5]
\footnote{... et même au-delà, jusqu'à la Révolution Française, même si des évolutions très significatives ont eu lieu à partir de la Renaissance et des Réformes protestantes et catholiques. Le paradoxe c'est même que c'est aux \crmieme{17} et \crmieme{18} siècles que les familles se sont le plus étroitement conformées aux principes chrétiens, sous la garde conjointe, vigilante et de plus en plus efficace, des autorités religieuses et civiles.}% 
. Sur la délimitation de cette période de l'histoire et en ce qui concerne mon sujet je ne peux que constater que celle qui convient le mieux est celle du \emph{long Moyen Âge} de Jacques \fsc{LE GOFF} (\emph{Faut-il vraiment découper l'histoire en tranches ?} 2013). À bien des points de vue le Moyen Âge ne s'est achevé qu'avec la Révolution Française, même si la deuxième moitié du \siecle{18} (à partir de 1760 à quelques années près) participait déjà du siècle suivant, notamment sur le plan des idées, avec le mouvement européen des \emph{Lumières}. Pour schématiser on pourrait dire qu'il y a une unité dans la période qui va de Constantin à l'Encyclopédie. 

\section{Mépris de la chair ?}

 Les chrétiens sont-ils coupables d'avoir diabolisé les plaisirs de la chair ? La réponse à ces questions n'est pas simple. D'abord il faut dire avec Paul Veyne qu'ils \emph{n'ont rien réprimé du tout, c'était déjà fait}. Le monde patriarcal des cités antiques n'était en rien un monde de liberté sexuelle, sauf pour les hommes libres, et encore. Puis les philosophes stoïciens étaient passés par là pour exiger des hommes libres eux-mêmes qu'ils orientent leurs désirs vers la seule procréation. Quant aux médecins ils allaient eux aussi dans le sens d'une grande modération dans l'activité sexuelle. 

 Ceci étant dit il est vrai que les théologiens chrétiens des premiers siècles ont longtemps été tentés par les thèses \emph{encratites}%
% [6]
\footnote{Selon Encyclopedia universalis : Encratite est un \emph{terme signifiant « les continents » (du grec \emph{enkratès}) et désignant plusieurs sectes hérétiques de l'Église ancienne qui prônaient un rigorisme moral radical (interdiction du mariage, abstention de viande et de vin) fondé sur une condamnation de la matière et du corps considérés comme les œuvres d'un démiurge distinct du Dieu suprême. Tatien, d'abord disciple de Justin, à Rome, est traditionnellement tenu pour le fondateur, vers 170, de cette secte ascétique des encratites, probablement dans la région d'Édesse. L'encratisme fut alors proscrit sous ses diverses formes par de nombreux décrets de Théodose I\ier, à la fin du \siecle{4}, et de Théodose II, en 428.
La sévérité des mesures impériales suffirait à témoigner de l'importance de la secte à cette époque. L'encratisme s'est alors confondu avec le manichéisme et a trouvé des prolongements chez les Messaliens et les Bogomiles (et les Cathares). Le rigorisme que pratiquaient ses adeptes se voulait une négation de l'œuvre du démiurge. Les fondements doctrinaux de la secte consistaient dans le rejet de certaines parties des Écritures, en particulier de l'Ancien Testament, et dans un recours à des textes de la littérature apocryphe présentant des tendances ascétiques très marquées. Certaines positions doctrinales et liturgiques découlaient généralement de la conception encratiste de la création et de la matière : négation du salut d'Adam (Tatien), négation de la résurrection de la chair, docétisme en christologie, utilisation d'eau à la place du vin pour célébrer l'eucharistie. La ligne de démarcation entre l'encratisme et le gnosticisme est difficile à tracer : ce dernier est dans une large mesure marqué par un courant rigoriste, et l'encratisme semble avoir accueilli des spéculations d'origine gnostique}.
Richard \fsc{GOULET}}%
, proches de celles des manichéens qui soutenaient que la matière est mauvaise par nature, que l'âme préexiste au corps, et qu'avec la conception elle chute dans un monde matériel et charnel, lieu de l'esprit du mal%
%[7]
\footnote{... ce qui paradoxalement pouvait conduire les adeptes de ces doctrines à une licence effrénée puisque rien sous le ciel n'avait plus d'importance : « méprises et fais ce que tu veux ».}% 
. Le plus emblématique des théologiens encratites, Tatien (deuxième siècle après J.-C.), considéré comme hérétique par divers \emph{Pères de l'Église}, rejetait le mariage et condamnait l'usage de la viande et du vin comme de tout autre plaisir de la chair. Préconisant l'eau pour célébrer l'eucharistie à la place du vin, il recommandait de se garder de tout acte sexuel, et de ne pas faire d'enfants pour ne pas prolonger l'existence d'un monde qu'il jugeait mauvais. 

 Dans le même ordre d'idées, selon Robert Markus%
% [8] 
\footnote{Robert \fsc{MARKUS}, \emph{Au risque du christianisme, l'émergence du modèle chrétien (\siecles{4}{6})}, Cambridge University Press, 1990, réédition en Français, Presses Universitaires de Lyon, 2012.} 
de nombreux auteurs du \siecle{4} pensaient qu'Adam et Ève \emph{avaient été créés sans sexe, avec une innocence que certains comparaient à l'innocence des enfants. Si Adam et Ève n'avaient pas péché, les rapports sexuels n'auraient pas été requis pour augmenter et multiplier la race humaine}.

 Augustin, 
\tempnote{Vérifier les parties citées et celles qui les commentent, et la présentation de ces trois paragraphes : qu'est-ce qui est à Markus, à Augustin, à nous ?}%
qui avait longtemps été manichéen et qui avait donc partagé cette thèse, finit par la rejeter complètement en disant : \emph{Je ne vois pas pourquoi il ne devrait pas y avoir de mariage honorable au Paradis}. Il soutient \emph{que l'union sexuelle et la reproduction ne dérivent pas de la Chute à partir d'un état pré pubère de l'innocence, mais qu'elles font partie de l'intention originelle de Dieu envers les créatures}. En renonçant à ses convictions antérieures, Augustin rejetait ainsi un large consensus parmi les chrétiens de son époque. Ses nouvelles positions à l'égard de la sexualité sont exceptionnelles à la fin du \siecle{4}. Elles s'opposent aux conceptions d'autres Pères de l'Église comme Jérôme, Ambroise ou Grégoire de Nysse,~etc. 

 « Ce qui devait être expliqué n'était pas l'existence de la sexualité, mais plutôt son mode de fonctionnement et l'impact du péché d'Adam sur la sexualité de ses descendants. » « Les problèmes que posait la sexualité n'étaient ni plus ni moins les mêmes que ceux que posait l'homme. » Selon Augustin à la fin de sa vie : « Ce n'est pas la chair corruptible qui a rendu l'âme pécheresse, c'est l'âme pécheresse qui a rendu la chair corruptible. »

 De son point de vue selon Robert Markus « La tension que décrivait l'enseignement manichéen, entre deux natures différentes dans un conflit permanent, était maintenant transposée en terme de conflit interne avec soi-même... Ce qui est répréhensible et honteux dans la sexualité n'est pas son existence même, mais sa tendance à être hors de tout contrôle et à échapper à la raison... un tel constat impliquait une réhabilitation de la chair... Et au bout du compte, cela impliquait aussi une réhabilitation du mariage. » 

 C'est Augustin qui a théorisé le premier la notion de « péché originel ». Il faisait de la reproduction humaine le lieu de sa transmission. Compte tenu de son immense postérité intellectuelle jusqu'à la fin du Moyen Âge, ses idées ont eu une profonde influence. Elles impliquaient une ascèse à laquelle étaient conviés les époux au même titre que les religieux, ascèse aussi méritoire de son point de vue que celle à laquelle s'astreignaient ces derniers. 

 Les ambivalences d'Augustin n'étaient pas de nature à empêcher les moines, très influents durant tout le haut Moyen Âge où ils fournissaient l'essentiel des intellectuels, de tenir la virginité pour un état de vie plus proche de la perfection que celui des mariés, ni de dénigrer la féminité et l'exercice de la sexualité (de toutes les sexualités), ni d'essayer de lui imposer des limites. Il faudra arriver au \siecle{12} pour que ces points de vue soient en partie remplacés par une exaltation du mariage comme état de vie chrétien. Ce qui n'empêchait pas les laïcs de choisir parmi tous ces principes et toutes ces règles celles qui leur paraissaient les plus raisonnables ou les moins intenables. Au sein des sociétés chrétiennes il existait donc une tension permanente%
% [9]
\footnote{Jacques \fsc{ROSSIAUD}, \emph{Sexualités au Moyen Âge}, Éditions Jean-Paul Gisserot, Paris, 2012.}% 
. 
 

\section{Disparition de l'adoption}

 La première des donations à visée religieuse des païens était leur héritage. De droit c'est leur héritier qui était l'officiant de leur culte mortuaire. À partir du moment où les cultes païens ont été disqualifiés, puis interdits, il n'y avait plus de motif religieux de se procurer à toute force un héritier, puisqu'il n'y avait plus de culte des morts à assumer. L'adoption plénière, celle qui fabriquait des héritiers légitimes avec des étrangers, a donc presque totalement disparu de la scène, et cela pour quinze siècles. C'est l'Église, et non plus les familles, qui gérait le culte des défunts en même temps qu'elle veillait sur les corps rassemblés dans les cimetières et le sol des églises, ce pour quoi elle recevait des donations. Elle n'avait pas de raison de se soucier de la pérennité des lignées, au contraire, l'absence d'héritiers n'était de son point de vue qu'un malheur individuel, et seulement pour cette vie. Cette absence n'avait pas d'incidence sur le salut de l'âme des défunts après leur mort. On en revenait donc aux règles juives : pas de filiation « fictive ». Mais cela ne s'est pas fait du jour au lendemain et il y a fallu plusieurs siècles, d'autant plus que les familles détentrices d'un « honneur », d'une charge publique, à commencer par les rois et les \emph{domini}, les seigneurs, avaient impérativement besoin d'héritiers pour ne pas perdre leur position sociale, et n'étaient pas d'accord sur ce point avec les clercs. 


\section{Divorces et remariages}

 Le remariage après divorce du vivant du premier conjoint a longtemps été reconnu comme valide dans ses effets par les autorités civiles, notamment en ce qui concernait la légitimité des enfants à naître, alors même qu'il était sanctionné comme une faute par l'Église (excommunication, pénitences publiques...) ou par les autorités civiles elles-mêmes (amendes, exil, confiscation de biens...). 

 Si l'adultère d'une femme avec un esclave était depuis Constantin puni de la mort des deux complices, il avait prescrit que le mari d'une femme adultère, entremetteuse ou empoisonneuse pouvait la répudier tout en conservant sa dot, et pouvait se remarier. À défaut de condamnation plus grave elle était reléguée dans une île. Dans les autres cas une femme répudiée conservait sa dot, et si l'époux se remariait elle pouvait « envahir » sa maison et prendre possession de la dot de la nouvelle élue. L'épouse pouvait elle aussi répudier un époux coupable d'homicide, d'empoisonnement, de violation de sépulture, et s'en aller avec sa dot. Elle la perdait dans les autres cas. 

 L'empereur Honorius fixe pour chaque époux trois paliers%
% [15]
\footnote{J.-P.~\fsc{POLY}, \emph{Le chemin des amours barbares}, p. 42.}% 
. Le mari peut répudier sa femme pour « crime grave » (les motifs précisés par Constantin), et il gardera sa dot et pourra se remarier. S'il la répudie pour « faute contre les mœurs » il reprend la donation qu'il lui a faite en l'épousant, mais doit rendre sa dot, et attendre deux ans pour se remarier. S'il la répudie pour d'autres motifs il perd dot et donation, et il ne peut plus se remarier. L'épouse peut de même quitter son mari pour « cause grave » (toujours les motifs de Constantin) et se remarier après un délai de cinq ans.

 Le remariage après divorce, du vivant du premier conjoint, ne semble avoir été en droit totalement éradiqué d'Occident qu'après les réformes Grégoriennes du milieu du Moyen Âge%
% [16]
\footnote{Cf. Georges \fsc{DUBY}, \emph{Le chevalier, la femme et le prêtre}, 1981.}% 
. Il a fallu que les tribunaux de l'Église obtiennent vers le \crm{10}\ieme{} ou \siecle{11} le monopole sur les affaires concernant le mariage pour que le vieux mot latin \emph{divortium} prenne le sens de séparation sans droit au remariage du vivant de l'autre conjoint, sens qu'il a gardé jusqu'à la Révolution. 


\section{Phobie de l'inceste}

 Même si les mariages entre cousins germains, et entre nièce et oncle paternel, avaient fini par être autorisés pendant un temps sous l'empire, l'interdit de l'inceste était ressenti avec force à Rome. L'Église partageait cette horreur de l'inceste : elle avait même choisi de comprendre le mot \emph{Porneia} comme désignant exclusivement les unions incestueuses, considérant que la seule cause acceptable de nullité des mariages était la proximité excessive des époux. L'un de ses objectifs était d'écarter du sein des parentèles toute expression des désirs sexuels (hors couples mariés), avec les rivalités, jalousies et rancœurs qui les accompagnent, pour donner toute la place à la seule fraternité et à la \emph{caritas}%
% [10]
\footnote{\emph{Caritas} = amour désexualisé : souci du bien de l'autre, amour de l'autre pour lui-même (même s'il n'a rien d'aimable). Il est souvent rendu par « charité » (dérivé direct de \emph{caritas}), mot où nous ne percevons plus aujourd'hui beaucoup d'amour.}% 
. L'autre objectif était de renforcer le tissu social. Augustin d'Hippone formule ainsi sa pensée : \emph{L'union du mâle et de la femelle, pour autant qu'elle relève du genre humain, est une sorte de pépinière de charité. \emph{[...]} Une très juste raison de charité%
%[12] 
\footnote{\emph{caritas}}
invita les hommes \emph{[...]} à multiplier leurs liens de parenté ; un seul homme ne devait pas en concentrer trop en lui-même, il fallait les répartir entre des sujets différents ; ainsi leur grand nombre contribuerait à préserver plus efficacement les liens de la vie sociale. Père et beau-père sont, en effet, les noms de deux liens de parenté. Que chacun ait un homme pour père et un autre pour beau-père, la charité s'étend sur un plus grand nombre \emph{[...au lieu qu']} un seul homme eût été, pour ses enfants frères et sœurs mariés entre eux, père, beau-père et oncle \emph{[...]}, autres pour le même homme seront alors la sœur, l'épouse, la cousine ; autres le père, l'oncle, le beau-père ; autres la mère, la tante, la belle-mère. Ainsi, loin de se restreindre à un cercle étroit, le lien social s'étendra plus largement et sur plus de têtes par des alliances multiples%
%[13]
\footnote{Livre XV de \emph{la Cité de Dieu}, d'après la traduction de G.~\fsc{COMBES}.}% 
.}

 Plus l'on étend le périmètre de l'inceste plus il faut aller loin de sa famille de naissance pour trouver un conjoint. Cela diminue le risque que les descendants d'une personne (un homme dans le système patriarcal) ne deviennent si puissants, au moyen d'une endogamie stricte de sa descendance, qu'ils puissent menacer le reste de la société, n'ayant pas à composer entre des allégeances multiples. On peut à contrario évoquer les observations de Germaine \fsc{TILLON} dans \emph{le harem et les cousins}, 1966, et l'opposition qu'elle fait entre la « république des beaux-frères » et la « république des cousins ». L'Église s'opposait ainsi aux pratiques orientales, qui privilégiaient le mariage entre cousins (et même entre frères et sœurs en Égypte), comme aux coutumes germaniques qui favorisaient les unions préférentielles entre familles aux alliances redoublées de génération en génération%
% [11]
\footnote{Jean-Pierre \fsc{POLY}, \emph{Le chemin des amours barbares, Genèse médiévale de la sexualité européenne}, Perrin, 2003.}% 
. On ne peut pas dire qu'elle choisissait pour autant un système de parenté contre les autres, même si elle posait le couple entouré de ses enfants au centre de ses préoccupations. Elle mettait seulement une limite contraignante aux systèmes familiaux qui cherchaient à se fermer sur eux-mêmes.

 Au tournant entre le \crmieme{4} et le \siecle{5} les empereurs Théodose, Arcadius et Honorius, ont tenté d'interdire le mariage entre cousins germains, mais ces interdits ont été levés quelques années plus tard par les empereurs (d'Orient) suivants, même si en accord avec Ambroise de Milan et l'évêque de Rome, Augustin plaidait pour cet interdit : \emph{Qui peut douter qu'il ne soit aujourd'hui plus honnête d'interdire le mariage même entre cousins germains ?}. Il arguait de sa proximité excessive avec l'inceste fraternel, et du fait que même si les lois de l'Empire l'avaient effectivement autorisé, la coutume romaine n'y était pas favorable. Mais il n'en était pas de même en Orient, où le mariage entre cousins germains était traditionnellement tenu pour idéal, trop inscrit dans la culture pour que l'argumentation d'Augustin puisse y être entendue, et où il ne sera plus question de l'interdire aussi rigoureusement par la suite. En Occident la position d'Augustin, reprise siècle après siècle par l'église, finira néanmoins par triompher.

 À partir du \siecle{7} et surtout du \crmieme{11} au \crmieme{13} en Occident, l'Église entend la notion d'inceste de manière de plus en plus extensive, jusqu'au septième degré, comme le faisait le droit romain ancien. Par dessus le marché à partir du \siecle{7}, elle s'est ralliée progressivement, et non sans résistances même en son sein, à un mode de calcul de ces degrés qui excluait tous les descendants \emph{des arrière-grands-parents des arrière-grands-parents} du sujet concerné%
% [17]
\footnote{Cf. \emph{Histoire du droit civil}, Jean-Philippe \fsc{LEVY} et André \fsc{CASTALDO}, p. 93-95.}% 
, ce qui multipliait de façon exponentielle le nombre des personnes interdites, du moins dans les familles qui prétendaient connaître leurs ancêtres aussi loin dans le passé, celles dont la légitimité reposait sur leur ascendance. Les humbles n'avaient pas une telle prétention, et on n'attachait pas autant d'importance aux irrégularités formelles de leurs unions. En ce qui les concernait il suffisait que l'interdit porte sur toutes les personnes ressenties par eux comme faisant partie de leur parenté. 

 Non seulement les évêques et théologiens d'Occident y ont ajouté toutes les parentés par alliance, y compris beaux-frères et belles-sœurs, mais ils y ont aussi adjoint la \emph{parenté spirituelle} qui liait les parrains et marraines d'un même enfant, et la parentèle de ceux-ci, sans compter les \emph{parentés} illicites nées des rencontres extra conjugales. La phobie de l'inceste a donc conduit à des extrêmes absurdes qui créaient mécaniquement des situations impossibles dans les communautés étroites où les personnes se déplaçaient fort peu en dehors des familles aristocratiques, et où en l'absence de registres d'état-civil il était difficile ou impossible de faire des généalogies fiables. 

 La déliquescence des États et donc celle des cours de justice a fini par assurer à l'Église l'exclusivité du traitement des litiges touchant aux mariages entre le \crmieme{9} et le \siecle{12}. Au fur et à mesure que son influence sur le droit du mariage grandissait sa définition de l'inceste était bon gré mal gré intégrée par les familles dans leurs stratégies. A-t-elle été pour elle un outil de conquête du pouvoir ? L'extension des limites de l'inceste servait objectivement ses intérêts matériels et politiques en multipliant les risques de nullité et les demandes de \emph{dispense}%
% [18]
\footnote{En ce qui concernait ces interdits il s'agissait d'une question de discipline et non d'une règle de foi. L'Église se reconnaissait donc le droit d'en dispenser les fidèles qui en faisaient la demande, mais cela ne se faisait pas toujours sans frais.}% 
. C'est la thèse de Goody, et elle n'est pas invraisemblable (selon Poly elle est plausible, mais à partir du \siecle{11} seulement),

 ... mais il faut observer qu'au même moment les rois et les autres puissants s'appuyaient sur les mêmes principes pour s'immiscer dans les conflits au sein des familles de leurs dépendants, et pour gérer à leur convenance les transmissions des patrimoines, des héritages, et des fiefs. 

 C'est de la même façon que les autorités civiles se sont opposées à ce que les enfants illégitimes reçoivent le même traitement que les autres, notamment dans les héritages. Ils s'opposaient surtout à ce qu'ils puissent hériter des fonctions fournissant un surcroît d'\emph{honneur}, c'est-à-dire les fonctions de pouvoir. Rois et clercs ont mis des siècles à parvenir à cette fin, mais ils y sont parvenus. C'est qu'ils avaient des intérêts convergents dans l'affaire. Les puissants avaient intérêt à ce que les familles de leurs concurrents et de ceux qui dépendaient d'eux aient des difficultés à trouver un héritier, sachant qu'environ une femme sur cinq ayant l'âge de procréer n'était pas féconde, que suivant les « honneurs » à transmettre, les filles ne convenaient pas aussi bien qu'un garçon ou étaient exclues, suivant les législations ou les coutumes en vigueur (ex. la loi Salique chez les Francs), et que les enfants illégitimes ne pouvaient en hériter. Il était avantageux pour les puissants de plaider l'illégitimité des enfants de leurs ennemis promis à un riche héritage pour les en déposséder, et donner à un autre de leur choix l'honneur qui devait leur échoir. 

 Et l'extension à l'infini de l'inceste rendait paradoxalement plus facile, pour tous ceux qui pouvaient assumer un procès canonique, d'obtenir l'annulation d'un mariage pour inceste si une alliance plus profitable ou une femme plus désirable ou supposée plus féconde se présentait : si tout le monde était parent de tout le monde toutes les unions étaient incestueuses, et donc à la merci d'un procès gagné d'avance (en somme : « si vous ne voulez pas être piégé dans un mariage indissoluble épousez votre petite cousine »). 


\section{Enfants en trop, enfants « irréguliers »}

 Ce n'est pas parce qu'elle était interdite que l'exposition des enfants avait disparu. Les pauvres ont toujours eu recours à l'exposition et n'ont jamais été sanctionnés pour ce motif. Quant aux ventes d'enfants, interdites en principe, elles ne tombaient sous le coup de la loi que lorsqu'elles obéissaient à d'autres motifs que le dénuement%
% [19]
\footnote{... mais qui vendait son enfant pour d'autres raisons (sauf un enfant que le père de famille supposait né d'un adultère de son épouse) ?}% 
. Bien au contraire les acheteurs ont en réalité été encouragés par le fait que les parents qui abandonnaient étaient déchus de leurs droits. Il faudra que le servage disparaisse aux \crmieme{12} et \siecle{13} pour que les ventes d'enfants disparaissent aussi... Et c'est à partir de cette époque que le nombre des expositions de nouveaux-nés dans les villes va se mettre à poser de sérieux problèmes d'ordre public.

 En accord avec la Bible, l'Église a toujours interdit à ses fidèles l'avortement et l'infanticide, et Constantin a introduit cette interdiction dans le droit romain. Qu'en était-il en réalité ? Les avortements et les infanticides ont-ils d'un seul coup disparu ? Il est difficile de le croire. Les infanticides n'ont certainement pas disparu. Ainsi Grégoire de Tours (539-594) rapporte le cas d'une femme qui avait mis au monde un enfant monstrueux : \emph{Comme c'était pour beaucoup un sujet de moquerie de l'apercevoir, et qu'on demandait à la mère comment un tel enfant pouvait être né d'elle, elle confessait en pleurant qu'il avait été procréé pendant une nuit de dimanche. Et n'osant le tuer comme les mères ont coutume de le faire, elle l'élevait de même que s'il eût été conforme}%
% [20]
\footnote{... cité par D.~\fsc{ALEXANDRE-BIDON} et D.~\fsc{LETTE}, p. 27.}% 
. On croit en effet à cette époque que les naissances d'enfants mal conformés sont le résultat de relations sexuelles durant les périodes d'abstinence obligatoire, pendant le carême ou l'avent, pendant les règles%
%[21]
\footnote{Il en est de même pour la lèpre. Il est si difficile de ne pas savoir pourquoi le malheur vous frappe qu'on préfère encore s'en proclamer responsable.}% 
,~etc.

 Mais le plus suggestif c'est le « \emph{n'osant le tuer comme les mères ont coutume de le faire} ». Il faut remarquer la simplicité avec laquelle Grégoire de Tours rapporte ce qui est pour lui une évidence contre laquelle il ne s'indigne pas. Entre les règles morales, même celles qui étaient inscrites dans la loi, et les pratiques effectives, il y avait une marge, comme toujours, et l'infanticide est si aisé et si difficile à prouver. Les nouveaux-nés sont si fragiles, et il arrivait si souvent qu'ils soient étouffés par mégarde sans intention maligne lorsqu'ils partageaient le lit de leur mère, pour avoir plus chaud ou lui éviter de se relever la nuit,~etc. 

 Les avortements ont pu se raréfier en l'absence de médecins et de sages-femmes compétents et prêts à louer leurs services (à supposer que ces compétences se soient effectivement perdues chez les femmes d'expérience, ce qui est à prouver), mais les avortements n'ont jamais été ressentis comme des infanticides, et tout au plus comme des fautes lourdes. Les avortements précoces étaient d'autant moins culpabilisés que pour la plupart des théologiens du Moyen Âge comme pour ceux de l'Antiquité, l'animation du fœtus n'avait pas lieu au moment de la fécondation, mais bien plus tard, chacun défendant sa propre théorie (Cf. Maaike \fsc{VAN DER LUGT}, \frquote{L'animation de l'embryon humain et le statut de l'enfant à naître dans la pensée médiévale}, in \emph{L'embryon, formation et animation}, collectif, déc 2004, Paris, Vrin, p. 234-254). 

 Les enfants issus de simples mésalliances (sénateur--affranchie, femme libre--esclave,~etc.) ne posaient pas de problème religieux aux chrétiens, pas plus qu'aux juifs, même s'ils posaient des problèmes familiaux et sociaux, et même si le droit romain pourchassait ces mésalliances. Ceux dont l'Église réprouvait vraiment la naissance étaient ceux qui avaient été conçus dans le cadre d'une transgression de ses propres lois morales, les \emph{fruits du péché}. 

 Dans ce domaine, les règles de l'Église viennent presque intégralement des juifs. L'échelle de gravité des fautes est calquée sur l'échelle des \emph{mamzerim}. Y ont été ajoutés les enfants nés des personnes qui ont fait vœu de célibat, par analogie avec le sort des enfants illégitimes des prêtres du Temple de Jérusalem. 

 Les \emph{irrégularités de conception} étaient classées comme suit de la moins grave à la plus grave :
\begin{enumerate}
% a)
\item ceux qui ont été conçus dans le cadre d'un concubinage stable, monogame et sans interdit de mariage, et qui n'ont pas (encore) été régularisés par un mariage subséquent ;
% b)
\item ceux qui sont nés d'un rapport de hasard (fornication) ou d'un concubinage qui n'a pas duré ;
% c)
\item ceux qui ont été conçus alors que leur mère se prostituait (fornication) ;
% d)
\item ceux qui sont nés d'un adultère avéré (enfants adultérins) ;
% e)
\item ceux qui sont nés des relations coupables, consenties, d'un clerc ou d'une religieuse ayant fait vœu de célibat (sacrilège) ;
% f)
\item ceux qui sont nés du viol d'une femme mariée (sacrilège) ;
% g)
\item ceux qui sont nés du viol d'une vierge consacrée (sacrilège) ;
% h)
\item ceux qui sont nés d'un inceste. 
\end{enumerate}

 Tous ces enfants étaient illégitimes. Quand ils étaient le fruit des œuvres de leur père avec une servante ou une concubine, ils ont souvent été élevés dans la famille de leur père, au moins pendant le haut Moyen Âge : chez les Germains cela allait de soi. Par contre même s'ils étaient invités à pardonner à leurs épouses infidèles, les maris n'avaient pas l'obligation d'assumer les enfants adultérins de celles-ci. En ce cas l'abandon anonyme était un droit reconnu officiellement, même aux maris fortunés. Mais ils pouvaient aussi les assumer, comme en droit romain. Quant aux enfants nés d'un « sacrilège » ou d'un inceste il est vraisemblable qu'ils étaient le plus souvent traités comme des enfants abandonnés.


\section{Les éducations}

 Pour la plupart des enfants des villes (qui représentent peu de chose à l'époque) la petite enfance se passe à la campagne chez une nourrice. La mise en nourrice a concerné plus d'enfants que tous les internats éducatifs, collèges ou hôpitaux réunis, et de très loin. C'était en effet une nécessité absolue pour les femmes des villes qui exerçaient un métier : elles ne pouvaient consacrer à l'allaitement le temps nécessaire jusqu'au sevrage de l'enfant (à deux ans), et il n'y a eu jusqu'au \siecle{20} aucun substitut valable au lait féminin. La généralisation du nourrissage mercenaire reposait aussi sur la possibilité de gagner (à compétences égales) beaucoup plus d'argent en ville qu'à la campagne. Cela permettait aux citadines, même de ressources modestes, d'acheter le lait et le temps des paysannes. La croissance des villes a donc entraîné une augmentation massive du recours à la mise en nourrice.

 Cela n'a pu se faire aussi largement que parce qu'était peu ou pas perçue l'influence des premières relations de l'enfant avec sa mère ou un substitut sur la construction de sa personnalité : le bébé était censé n'avoir besoin que de lait. Le nourrissage mercenaire est donc une institution dont il a été fort peu parlé pendant des millénaires. Cette pratique n'était ni pensée, ni pensable. Elle était du côté des corps et de la nature, des réalités féminines, au même titre que la grossesse et l'accouchement, qui se faisaient aussi bien quand les hommes n'en parlaient pas, sinon mieux. Seul présentait de l'intérêt pour ces derniers ce qui commençait avec l'âge de raison (7 ans).

 Dès qu'ils ont l'âge de raison les enfants de ces temps ne sont plus regardés comme fondamentalement différents des adultes. Ils ne reçoivent aucune protection spéciale (protection du corps contre les gestes traumatiques, protection des yeux et des oreilles contre les spectacles traumatiques). L'éducation est rude et les sanctions sévères. Celui qui économiserait les verges et le fouet croirait mal faire. Les orphelins continuent d'être l'objet de toutes les attentions des autorités. Quant aux jeunes délinquants, condamnables dès 7 ans, ils perdent à 12 ans \emph{l'excuse de minorité}, qui de toute façon n'est pas automatique même avant cet âge. Ils sont en tout traités comme des adultes.

 Dans l'immense majorité des cas chacun apprend de son père le métier de son père. Dès 6 ans la plupart des enfants travaillent autant qu'ils le peuvent. A partir de cet âge un enfant de pauvre ne coûte plus guère. Sauf chez les riches et des puissants, dès 12 ans chacun gagne réellement le pain qu'il mange chez ses parents ou chez un maître. Le placement réciproque des adolescents chez des alliés des parents (oncles, surtout maternels, suzerain,~etc..) est un outil éducatif souvent employé (les jeunes nobles servent comme pages, les fils de paysans comme pâtres, les marins comme mousses,~etc.). Le placement en apprentissage chez un artisan (maître ès arts) n'est possible que si les parents paient l'apprentissage : c'est un luxe auquel les pauvres ne peuvent pas prétendre. Pour l'école il en est de même. 

 L'Antiquité grecque ou romaine avait élaboré à l'intention de ceux qui pouvaient se le payer un système complet d'enseignement (primaire, secondaire et supérieur). D'autre part un certain nombre de postes de professeur du secondaire étaient financés par les cités. L'état romain finançait des chaires d'enseignement supérieur (Augustin d'Hippone en est un représentant illustre : ses écrits le décrivent successivement élève, étudiant, enseignant et titulaire de chaire). Au \siecle{4} ce système continue de fonctionner, au \siecle{6} il est pratiquement en ruines en Occident, alors qu'il perdurera encore dix siècles à Byzance sans changements de structure. Les universités européennes ne relèveront le flambeau qu'à partir des derniers siècles du Moyen Âge. 

 En attendant seules résistent les écoles cathédrales et monastiques. Les premières écoles \emph{épiscopales} ou \emph{cathédrales} fleurissent au \siecle{4}. Elles ont pour principal objet de former les futurs clercs, mais les élèves peuvent à la fin du cursus refuser d'entrer dans le clergé. Le \emph{deuxième concile de Vaison} (529) prescrit à chaque prêtre chargé de paroisse de mettre en place une \emph{école paroissiale} à l'intention des jeunes les plus vifs d'esprit. Ce sont les premières \emph{petites écoles}. Elles sont d'abord destinées à alimenter \emph{l'école cathédrale} en sujets d'élite destinés à former le personnel ecclésiastique, et n'ont pas pour but d'apprendre à lire à tous comme c'est le cas chez les juifs : la vie religieuse du chrétien n'exige pas qu'il sache lire, il suffit qu'il sache entendre. Son activité professionnelle ne l'implique pas non plus : l'enseignement lettré (maîtrise du latin, langue de la culture et des savants) est inutile à qui ne sera pas clerc. Si l'on cherche le pouvoir il est alors plus efficace de connaître les armes que la rhétorique ou le droit. Les \emph{écoles monastiques} apparaissent au \siecle{4}, mais elles ne prennent en principe que des enfants destinés à devenir moines (« donnés » très jeunes à Dieu par leurs parents) et seuls apprennent le latin, les moines « de chœur », ceux qui chantent dans le chœur, ceux qui pourraient être ordonnés prêtres. Pourtant bien des écoles monastiques acceptent aussi quelques jeunes qui ne sont pas destinés à devenir moines, et Charlemagne leur en fera l'obligation.
 
 
 


\chapter{Familles spirituelles}


 Si les jeunes gens pouvaient se marier validement sans l'accord de leurs parents, en bonne logique ils avaient aussi le droit de ne pas se marier. Devant le désir d'un jeune de devenir religieux l'autorité du père s'arrêtait si le jeune en appelait à son éveque : pour les garçons à partir de 14 ans, pour les filles à partir de 12 ans. Le choix de la vie religieuse émancipait ceux qui le faisaient avant l'âge de leur majorité, et les mettait sous sa protection : cela reposait évidemment sur une reconnaissance par les autorités civiles de la validité des vœux religieux. Cette reconnaissance leur a été accordée par les derniers empereurs romains (chrétiens) et a été reconduite jusqu'à la Révolution française.

 Ceux qui se sentaient attirés par une vie de célibataire consacré pouvaient proposer à une communauté religieuse de les coopter. Ce choix de vie entraînait des incidences légales importantes et définitives, puisque la loi civile l'entérinait. En effet en prononçant ses vœux (pauvreté, chasteté, et surtout obéissance) le moine ou la religieuse se mettaient sous la puissance du responsable de la communauté, comme s'ils s'étaient fait adroger. Ils étaient juridiquement exclus de leur famille de naissance, et de tout héritage à venir. Comme des mineurs ils ne pouvaient plus rien faire de leur propre initiative. S'ils ne pouvaient signer aucun contrat en leur nom propre, ils pouvaient toujours, de la même façon qu'un esclave ou qu'un fils en puissance de \emph{pater familias}, exercer au nom de leur supérieur(e) tout mandat qu'il lui convenait de leur confier. Une fois entrés dans la communauté, c'était en principe pour toujours. Ils ne pouvaient plus sortir de leur état. Ils pouvaient dans une certaine mesure changer de monastère et même d'ordre religieux, mais les \emph{gyrovagues} qui erraient de couvent en couvent étaient mal vus. 

 Aucun religieux ne possédait rien qui lui soit personnel, et pourtant beaucoup d'entre eux avaient reçu de leurs parents une part d'héritage sous forme d'argent, de terre, etc. À leur entrée dans la communauté ils en avaient fait don (eux ou leurs parents) à la communauté, qui en contrepartie s'était engagée à les prendre en charge jusqu'à leur mort. Chaque communauté vivait de son travail et des revenus des biens qu'elle avait reçus en don : dots des religieux vivants \emph{et décédés}, loyers, récoltes, rentes et autres dons reçus de bienfaiteurs. Tous les biens appartenaient au monastère et celui-ci possédait le droit de posséder et d'exercer des actes juridiques en son nom propre. Si les religieux se succédaient de génération en génération, le monastère en tant qu'entité n'en persistait pas moins dans son être, unissant les morts et les vivants dans le même ensemble intemporel. Le modèle familial ainsi mis en œuvre était accepté en toute connaissance de cause ainsi que le montre l'emploi très précoce du vocabulaire de la famille : « père » (\emph{abba} = père en araméen = abbé), « mère », « frère », « sœur », etc.  ... Mais cette famille n'avait pas d'héritiers à pourvoir et ses biens étaient inaliénables et insaisissables, protégés par le statut de la \emph{mainmorte}. 
 
 Ce mot a deux sens :
\begin{enumerate}
% a)
\item c'est le droit du seigneur de prendre les biens de son serf à sa mort. Les biens font \emph{échute}, c'est-à-dire réversion au seigneur qui en hérite. En ce sens les serfs sont \emph{gens de mainmorte}. Ce n'est pas le sens du mot \emph{mainmorte} qui nous concerne ici%
%[1]
\footnote{... même si un bon nombre des derniers serfs (fin \siecle{18}) appartenaient à des communautés religieuses de l'Est de la France, et si à cette époque les religieux ont eux aussi été nommés \emph{gens de mainmorte}, parce qu'incapables de transmettre des biens à des héritiers, non comme serfs d'un seigneur, mais comme religieux.}
 ;
% b)
\item on appelle aussi \emph{biens de mainmorte} ceux qui appartiennent à une personne juridique : ce sont les biens des collectivités qui ont le privilège de pérennité et n'ont pas à transmettre leurs biens à des héritiers. Cela conduisait à l'enrichissement progressif des communautés bien gérées... jusqu'au jour où leurs richesses devenaient trop tentantes pour les puissants du moment et leur étaient (re) prises par l'un d'entre eux : de ce point de vue l'histoire de la plupart des monastères est celle d'une suite de périodes d'accumulation et de moments de spoliation.
\end{enumerate} 

 À la fin de l'antiquité il était admis que dès leur plus jeune âge (6 ou 7 ans...) les parents puissent faire don à un monastère d'un ou plusieurs de leurs enfants, légitimes ou non%
% [2]
\footnote{Sources : Marc \fsc{BLOCH}, \emph{La société féodale}, Paris, 1939, 1994. Georges \fsc{DUBY}, \emph{Féodalité}, Paris, 1996, 1999. Collectif, \emph{L'homme médiéval}, Paris, 1989.}% 
. Ils accompagnaient le « don » de l'enfant d'un cadeau, souvent un bien foncier, qui devait permettre de subvenir à son entretien. Si l'enfant \emph{à Dieu donné} découvrait un jour que ce mode de vie ne lui convenait pas, il lui était extrêmement difficile d'en sortir. Le droit civil et les enseignements de l'Église se liguaient pour lui prêcher la résignation et lui barrer tout retour. Le jeune \emph{donné} à un monastère n'était d'ailleurs pas forcément plus contrarié dans ses choix que les jeunes esclaves, ou que les jeunes gens qui au même moment étaient mariés par leurs familles sans tenir compte de leur avis, ou qui devaient reprendre le métier de leur père. D'autre part, le \emph{don à Dieu} côtoyait des situations contemporaines par rapport auxquelles il représentait un progrès relatif (cf. {Boswell}) : abandon anonyme, infanticide, vente par les parents comme esclave, etc. 


Les jeunes « donnés » ont pu à certaines périodes représenter une proportion importante de l'effectif des monastères, mais ceux-ci fournissaient aussi à ceux et celles qui n'étaient pas ou plus attirés par le mariage un moyen de l'éviter, alors que le célibat non consacré était mal accepté par la société civile. Cela permettait aux veuves d'échapper à la nécessité de se mettre sous la protection d'un mari. Cela donnait une chance aux femmes les plus douées de jouer un rôle public auquel elles n'auraient jamais pu rêver autrement. C'était le seul moyen pour les filles d'esquiver un mari grossier, mesquin ou brutal, et/ou d'éviter de risquer leur vie dans les grossesses et les accouchements%
% [3]
\footnote{... qui à l'époque faisaient mourir (en hôpital) une femme sur dix environ si l'on en croit les mémoires de \fsc{TENON}, ce qui ne représente pas une naissance sur dix, bien évidemment, mais d'une naissance sur 30 à une naissance sur 120 suivant les temps et les lieux (p. 242 et suivantes). Ce chiffre avait de quoi angoisser les jeunes filles, surtout celles de santé fragile, ou celles qui présentaient une malformation. Jacques Tenon était chirurgien à l'Hôtel-Dieu de Paris avant la Révolution. À la demande des autorités il a rédigé ses \emph{mémoires sur les hôpitaux de Paris} édités en 1788. Nous le citerons souvent.}%
. C'était un refuge pour les jeunes gens mal conformés ou de santé trop fragile. 

 D'un autre point de vue l'entrée en religion d'un enfant légitime diminuait mécaniquement le nombre des petits-enfants à naître (qu'il faudrait « établir » un jour sur le capital familial). C'était donc une forme indirecte de contrôle des naissances. C'est l'une des raisons, sinon la première, pour lesquelles les seigneurs grands et petits ont créé tant de monastères : ils avaient un intérêt direct à disposer d'institutions où placer l'excédent de leur progéniture dans un cadre conforme à la dignité de leur famille, et sans contrevenir aux lois de l'Église, qu'ils avaient à peu près intériorisées. D'ailleurs si la politique familiale l'exigeait (par exemple si les enfants privilégiés dans un premier temps décédaient) il n'était pas impossible de relever de ses vœux et de faire sortir du cloître une fille ou un fils, à la condition qu'il n'ait pas été ordonné prêtre (mais le plus souvent les moines ne l'étaient pas : pour le droit canon, c'étaient des laïcs, à l'exception de ceux qui étaient ordonnés diacres ou pretres).

 Par ailleurs même quand pour les personnages publics puissants il était difficile de faire d'un fils illégitime un successeur. D'autre part il était interdit d'ordonner prêtres les garçons illégitimes : pour eux le clergé séculier n'était un débouché envisageable qu'au prix de dispenses coûteuses. Au contraire les monastères ne manifestaient pas de réticences à les accueillir, tout comme ils accueillaient les enfants abandonnés. La vie des religieux est conçue comme une vie de purification. De plus c'est une vie cachée à l'écart du monde. Dans l'esprit du temps cela convenait parfaitement aux pécheurs et pécheresses repentis, aux natures perverties par le péché, et donc aux « impurs de naissance » ou aux clercs séculiers punis pour fautes graves. En outre, on considérait qu'ainsi les enfants illégitimes pouvaient racheter la faute de leurs parents. Il paraissait donc très louable de les vouer à la vie religieuse. Par ailleurs cela les excluait des jeux de pouvoir dont le monde profane est le théâtre. Il n'y avait plus à craindre de les voir parasiter les politiques familiales. C'est ainsi que les parents pouvaient estimer s'en sortir \emph{par le haut} du casse-tête créé par leurs enfants illégitimes ou surnuméraires.
 
 C'est aussi pour cela que tant de filles de rois, légitimes ou non, qui ne pouvaient sans déchoir être données en mariage à des aristocrates trop inférieurs en dignité à leur beau-père, et qui risquaient si on les mariait à de trop puissants seigneurs de donner naissance à des garçons d'ascendance royale susceptibles de menacer les héritiers légitimes du trône, se sont retrouvées abbesses d'abbayes royales, jusqu'au \siecle{18}. 

 La part d'héritage (un bien foncier, une somme acquise définitivement par le couvent dès la profession, la \emph{dot}, etc.) donnée par leurs familles aux futurs religieux était fonction de la fortune familiale et du prestige de la maison religieuse où ils entraient. Chaque ordre et chaque monastère possédaient une « cote » sur le marché des valeurs de prestige, ce qui justifiait un coût (et inversement). En règle générale la part d'héritage de celui ou celle qui entrait dans les ordres était bien plus petite que celle des autres enfants de sa famille. Il convenait en effet qu'une fois tout réglé il reste aux pères un bénéfice financier à faire entrer des enfants en religion. 

 Selon la plupart des règles une communauté vivait non seulement de ses rentes, mais aussi du travail de ses membres. Mais souvent seuls les \emph{convers}, enfants de pauvres reçus gratuitement sans part d'héritage (ou adultes qui se donnaient eux-mêmes ainsi), travaillaient de leurs mains : sauf dons intellectuels ou spirituels éclatants ils n'étaient instruits que superficiellement et ils effectuaient l'essentiel des besognes matérielles, tandis que les religieux mieux dotés par leurs parents étudiaient, apprenaient à lire et à écrire, apprenaient le latin, chantaient au chœur, copiaient les livres, enseignaient, etc. Il y avait là une évidente sélection par l'argent et par la naissance. Pendant très longtemps il semble que nul n'y ait vu matière à scandale. C'est que depuis l'empire romain (dès Caracalla, sinon avant) les sociétés civiles contemporaines étaient très inégalitaires avec des castes et des hiérarchies civiles justifiées par l'idéologie de l'hérédité (du « sang »), à laquelle les barbares adhéraient autant sinon plus que les Romains. Il est vrai aussi que les exhortations de Paul de Tarse (entre autres) à demeurer à la place où Dieu vous a mis ne favorisaient pas la mise en question de l'ordre établi%
% [4]
\footnote{Il faudra attendre les ordres mendiants à partir du \siecle{12} pour que ces discriminations par l'argent au sein des ordres religieux soient dénoncées, mais non supprimées. François d'Assise, fondateur des franciscains, était fils de bourgeois, non d'aristocrate, ce qui lui avait sans doute donné un autre regard sur le caractère « naturel » des discriminations de caste. Elles ne semblent pas avoir été vécues comme insupportables avant le \crmieme{18} ou \siecle{19}, du moins pour la plupart des religieux qui avaient droit à l'écriture et qui ont laissé des témoignages : mais ce n'étaient pas eux qui étaient ainsi humiliés.}% 
. 

 Si les familles des bienfaiteurs et des fondateurs entretenaient des liens étroits avec « leur » monastère pour conserver la possibilité d'y placer des enfants, elles le faisaient aussi et au moins autant parce qu'elles comptaient sur les prières des religieux, sur leurs messes et sur leurs autres dévotions, offices divers qu'elles « fondaient » contre donation pour garantir le salut éternel des âmes de leurs membres. Chacun de ceux qui le pouvaient affectait une part de ses biens à ces donations comme leurs ancêtres pré chrétiens avaient affecté une part de leurs biens (un tiers selon Goody ?) aux sacrifices à faire après leur mort et aux objets qu'ils emportaient avec eux dans la tombe. C'est pourquoi jusqu'à la fin du moyen-âge presque tous les testaments contenaient des donations pour le repos de l'âme du défunt, faites à une institution religieuse et/ou d'assistance, ce qui à l'époque était indiscernable : toutes les œuvres d'assistance étaient aussi des « \emph{œuvres pieuses »}. Cela a contribué à produire en quelques siècles un quasi-monopole de l'Église dans le domaine des testaments et des conflits qui y sont liés.
 
 
 

% 28.02.2015 :
% haut Moyen Âge
% _, --> ,
% ~etc.
% Antiquité
% ~\%


\chapter[Les familles de l'Ancien Régime entre autorités civiles et religieuses]{Les familles de l'Ancien Régime\\entre autorités civiles et religieuses}


\section{La réforme grégorienne}

 Durant le haut Moyen Âge les institutions religieuses étaient si intriquées dans le fonctionnement des sociétés civiles que les fonctions ecclésiastiques étaient souvent pourvues selon les mêmes principes que les « honneurs » civils et sur désignation des autorités temporelles : parfois elles s'achetaient et se vendaient \emph{(simonie)}, ou bien elles étaient dévolues comme des fiefs. Les abbés et évêques étaient engrenés dans le système féodal, prêtant et recevant des hommages. De nombreux biens ecclésiastiques, et les revenus afférents (dîmes,~etc.), étaient inclus dans les patrimoines de petits et grands seigneurs. De nombreuses seigneuries ont été fondées sur les terres des abbayes ou des évêchés, en principe pour les défendre. Les comportements de nombreux prêtres, évêques ou religieux s'écartaient largement des règles internes de l'Église : hommage rendu aux seigneurs, participation active à des combats, communauté de table et de lit avec une concubine ou une épouse, achat de fonctions ecclésiastiques \emph{(Nicolaïsme)}, établissement d'héritiers sur les biens ecclésiastiques. Cela n'encourageait pas les laïcs à suivre les enseignements de l'Église dans toute leurs rigueurs. 

 C'est pour enlever aux laïcs toute autorité sur les institutions religieuses (paroisses, évêchés et monastères), ce qui impliquait notamment de rendre impossible le recours au principe héréditaire dans la dévolution des charges, que la \emph{réforme Grégorienne}%
% [1] 
\footnote{\emph{Nom donné au mouvement animé et dirigé dans la seconde moitié du \siecle{11} par la papauté, particulièrement à l'initiative du pape Grégoire VII. L'objectif proclamé de la réforme grégorienne fut de rétablir la discipline et de corriger les mœurs des clercs afin de mieux encadrer la société laïque et de faire davantage pénétrer dans les esprits et dans les âmes les obligations de vie découlant du dogme chrétien. Pratiquement, il s'agissait de mettre en place un meilleur épiscopat, grâce auquel le recrutement des prêtres et le contrôle de leur activité seraient améliorés, ce qui, finalement, devait être profitable à la santé morale de tous les fidèles. L'entreprise était ainsi comparable, bien que plus vaste, à celle que, depuis un siècle, de nombreux prélats et des abbés clunisiens conduisaient à l'intérieur du monde seigneurial. Elle en différait fondamentalement par une extrême méfiance à l'égard des pouvoirs laïcs, regardés comme responsables des vices des évêques parce qu'ils intervenaient directement dans l'élection de ceux-ci, et surtout par l'édification dans l'Église, lors de la réalisation, d'un système de gouvernement monarchique entre les mains du pape, au point que certains historiens estiment que la réforme servit d'alibi à une grandiose opération visant à transformer les structures ecclésiales et à faire du pontife romain, avec ses cardinaux et ses légats, grâce au pouvoir de dispense et à l'exemption, la seule autorité souveraine.}

\emph{L'accomplissement du programme grégorien donna lieu aussitôt à un grave conflit avec l'Empire, la querelle des Investitures, du fait que Grégoire VII (1073-1085) interdit aux laïcs de choisir et d'investir les évêques et prit des sanctions à l'encontre de l'empereur Henri IV, qui récusait le bien-fondé de ces mesures. Poursuivie sous une autre forme par Urbain II, Gélase II et Calixte II, la Querelle s'acheva par un compromis lors du concordat de Worms (1122). Cependant, pendant ces cinquante années, de grands efforts avaient été faits afin de sensibiliser davantage les hommes au fait religieux et d'inviter les meilleurs à réfléchir et à agir selon les normes chrétiennes. Du coup, et sans doute par l'intermédiaire d'un épiscopat plus digne, un esprit nouveau anima l'Église jusqu'au milieu du \siecle{12} : l'esprit grégorien, façonné par la conviction que tout acte, public ou privé, s'intègre dans un contexte religieux et doit avoir une signification chrétienne.}

\emph{C'est à ses résultats que l'on juge le mieux la réforme et ses objectifs. L'œuvre grégorienne aboutit effectivement à faire disparaître presque totalement la simonie et le nicolaïsme, à mettre en place des évêques plus responsables, qui tentèrent de mieux contrôler le bas clergé et y parvinrent, notamment grâce à de meilleurs moyens juridiques, mais ne recrutèrent pas toujours des prêtres moralement valables ; le progrès était toutefois très net par rapport au siècle précédent. Plus encore, la réforme conduisit au plein pouvoir du pape dans l'Église et à l'exaltation de son autorité dans le monde, au point que l'ambition gagna les divers échelons de la hiérarchie et que papes et prélats se mirent à œuvrer d'abord pour conserver et renforcer leur puissance en maintenant et agrandissant leurs biens et droits temporels, ce qui devait conduire à une grave crise dès la fin du \siecle{12}. Ce goût du pouvoir chez les chefs de l'Église n'empêcha pas l'esprit de la réforme de pénétrer dans les mœurs : la religion cessa peu à peu de n'être qu'une pratique, qu'un culte s'accompagnant de quelques interdits pour devenir davantage une règle de comportement ; la piété se fit plus sensible ; la culture intellectuelle plus profonde. Enfin, de grandes entreprises politiques, telles les croisades, furent en partie suscitées par ce dynamisme, sans lequel on n'expliquerait sans doute pas leur succès.}
Marcel \fsc{PACAUT}, Article \emph{réforme grégorienne}, Encyclopédia Universalis, édition 2011.} 
a été mise en œuvre du \crmieme{11} au \siecle{12}, et d'abord au niveau de l'élection papale. C'était l'aboutissement d'une montée de l'influence du Pape et d'une forte centralisation de l'Église d'Occident. 

 Grâce à un effort continu de plusieurs siècles, et en dépit de fortes résistances, les papes, les évêques et les monastères ont progressivement réussi à accroître assez leur indépendance pour que les rois, l'empereur, et les seigneurs soient contraints de trouver d'autres moyens de récompenser un fidèle serviteur chargé de famille que d'en faire un évêque ou un abbé, ce qu'ils avaient fait à tour de bras depuis la chute de l'Empire romain. Ils ont été contraints d'abandonner leur autorité sur les églises paroissiales qu'ils avaient fondées, de ne plus en nommer les desservants et de ne plus les inclure dans leurs patrimoines. Les autorités ecclésiastiques ont obtenu le contrôle des \emph{investitures}, même si elles ont toujours dû tenir compte de l'avis des autorités civiles. 

 En pratique beaucoup de couvents et d'abbayes resteront encore sous l'influence de la famille du fondateur ou de bienfaiteurs puissants. Du moins les religieux, élus et seuls électeurs, seront-ils canoniquement en règle. Jusqu'à la fin du Moyen Âge ces formes seront dans l'ensemble respectées et les scandales les plus criants évités\footnote{Ce n'est qu'à partir de la renaissance et jusqu'à la Révolution que les revenus ecclésiastiques attachés aux monastères pourront être (en France) soumis au système de la \emph{commende}, ce qui conduira à de nouvelles situations d'empiètements.}. 
 
 

\section{Monopole de l'Église sur le droit familial}

 L'effort de normalisation des mœurs ne s'est pas borné aux seuls clercs. Au terme d'une longue évolution commencée par Constantin, à partir du \siecle{11} et de la réforme grégorienne, il n'y a plus de droit civil (laïc) du mariage dans une Europe totalement christianisée où les seuls dissidents religieux tolérés (plus ou moins suivant les lieux et les époques) sont les juifs
\footnote{\fsc{DUBY}, \emph{Le chevalier, la femme et le prêtre}, 1981.}% 
. Même s'ils se moulent de plus en plus dans les habits du droit romain le mariage et la famille sont du ressort exclusif du droit de l'Église (droit canon), qui a suffisamment de puissance pour imposer une bonne part de ses principes. 

Le mariage est désormais défini comme un sacrement, ce qui en un sens ne change pas grand chose, puisque aucun des traits qu'on lui attribue n'est nouveau (le mariage ne change pas, c'est la doctrine des sacrements qui se précise) mais cela rend les compromis plus difficiles à trouver. Cela interdit encore plus qu'avant d'y voir seulement un contrat entre deux parties et renforce le caractère indiscutable de son indissolubilité, toujours présent mais toujours contesté depuis l'origine du christianisme. 

 En 1215, le quatrième concile de Latran a décidé de restreindre les interdits au quatrième degré canonique, soit le huitième degré romain (ou moderne), incluant l'interdit des cousins germains, de restreindre les cas de parenté par alliance, et aussi de faciliter largement l'octroi de \emph{dispenses}.

 Comme les tribunaux civils ne s'occupent plus des affaires de mariage, il n'y a plus de divorce%
% [3]
\footnote{... pour les chrétiens mais pas pour les juifs qui avaient toujours droit de recourir aux tribunaux rabbiniques. Avec la Réforme le divorce redeviendra possible chez les protestants, même si en pratique ils feront tout pour le décourager : la stabilité des couples était ressentie comme essentielle pour des raisons multiples qui débordaient largement les seules représentations religieuses.}. 
Seules restent possibles les actions en nullité, ou les actions en séparation. Le concubinage n'est plus traité comme une union de second ordre, propre aux gens qui n'ont pas d'héritage à transmettre, mais comme un état de fornication durable.



 La paternité est exaltée en liaison avec la paternité divine, mais la valorisation des parentalités spirituelles à travers le culte de Saint Joseph (père adoptif de Jésus selon les évangiles) affirme la prééminence de la relation éducative sur la reproduction biologique. Au même moment la maternité est très fortement idéalisée à travers le culte de la Vierge Marie. Au total c'est la famille nucléaire, le couple et ses enfants légitimes, qui est sacralisée. Cela s'exprime entre autres dans le culte de la « sainte famille » qui prend son essor au \siecle{17} : c'est en 1665 qu'est fondée la \emph{confrérie de la Sainte Famille} (la fête religieuse de la sainte famille n'a été instaurée qu'en 1893). 
 
 La morale familiale et sexuelle enseignée par l'Église est la morale commune. Si du \crmieme{11} au \siecle{16} des mouvements de contestation religieuse se succèdent, qui culmineront avec la réforme protestante, rares sont ceux qui à cette période mettent vraiment en question la morale familiale et sexuelle enseignée par l'Église. Au contraire les opposants s'appuient sur elle, même les cathares dont les doctrines sont par ailleurs très loin du christianisme, pour critiquer les écarts des clercs avec leurs propres principes. Cette morale a été formulée dès les premiers temps de l'Église, mais les règles de droit qui en découlent ont mis un millénaire à s'imposer comme le droit commun. Ceci dit jamais elles n'ont réussi à le faire partout ni parfaitement. Certaines régions s'y sont conformées avec beaucoup de rigueur tandis que d'autres ont été beaucoup plus tolérantes avec les écarts. Chaque société repose sur un ensemble de logiques sociales, économiques et politiques qui n'ont pas forcément à voir avec une religion quelconque. Le christianisme a certes contribué à façonner les sociétés d'ancien régime, mais il en a été lui-même très fortement influencé. Il lui a été demandé de bénir (et même de sacraliser) leurs mécanismes et leurs logiques, et de les conforter dans leur fonctionnement, et c'est ce qu'il a souvent fait. 
 
 \section{Écarts avec les règles canoniques}

 Les contraintes et limites imposées à la reproduction par le roi et par l'Église n'ont jamais été acceptées totalement ni par tous. C'était notamment le cas dans la noblesse. Depuis l'Antiquité tardive et durant tout le Moyen Âge elle était tenue, par elle-même et par les autres ordres de la société, pour une \emph{race} supérieure qui transmettait ses vertus par son sang. Cette très antique croyance n'accordait pas d'importance au statut juridique ou religieux de l'union des parents. Elle coexistait sereinement dans les têtes avec le modèle canonique judéo-chrétien. 

 Les membres les plus puissants de la noblesse, et d'abord les rois eux-mêmes, n'ont jamais cessé de pratiquer une polygamie de fait qui leur donnait de nombreux enfants, de second rang certes, mais parfois bien utiles à défaut ou en complément d'enfants légitimes et valeureux. Jusqu'au \siecle{11} les différences faites dans les familles puissantes entre enfants légitimes et bâtards nés du chef de famille étaient faibles (capacité d'hériter, de succéder,~etc.). En effet le sang du père ennoblissait. Cette conception était un héritage des mœurs d'inspiration germanique du haut Moyen Âge. Le nombre des bâtards nobles semble même avoir crû au  \siecle{15}. Par comparaison les bourgeois reconnaissaient beaucoup moins d'enfants illégitimes. De 1400 à 1649 les rois de France ont reconnu 24~\% d'enfants naturels tandis que les grands officiers, employés roturiers de la maison du roi, n'en avouaient que 10,3~\%. 

 Alors que le mariage des grands seigneurs ne répondait habituellement qu'à des critères politiques, les enfants qu'ils avaient conçus avec leurs maîtresses, \emph{enfants de l'amour}, étaient fréquemment \emph{réussis}. S'ils étaient légitimés, ces enfants pouvaient leur permettre des alliances profitables. Or les rois d'Europe, héritiers en cela aussi de l'empereur de Rome, pouvaient légitimer les « bâtards » par \emph{lettres royaux}. Les bâtards des familles aristocratiques ont donc souvent été légitimés par le roi ou par mariage subséquent. Même s'ils ne l'étaient pas, cela n'a pas fait problème pendant longtemps. Souvent ils n'ont été légitimés qu'après la mort de leur père, \latin{ad honores}, c'est-à-dire pour accéder aux honneurs qu'ils détenaient, c'est-à-dire pour leur succéder dans les emplois d'intérêt public, les charges qu'ils exerçaient. Par contre aucune reine, princesse du sang ou femme d'officier n'a pu légitimer d'enfant naturel autrement que par un mariage subséquent : c'est l'homme qui ennoblissait, c'est également lui qui légitimait. 

 Les cas d'illégitimité susceptibles de bénéficier d'une légitimation par mariage subséquent avaient été étendus par les rois au-delà des critères de Constantin, de façon à inclure les enfants nés d'une relation passagère, et ceux conçus dans le cadre d'un enlèvement suivi d'un viol (enfants dont le géniteur n'était pas forcément le futur mari). Par ce biais la fiction retrouvait une place dans la filiation : l'adoption par l'époux de la mère redevenait possible. 

 Et pourtant les déclarations officielles de l'Église stigmatisaient tous les enfants illégitimes. Le Con\-ci\-le de Bourges (1031) confirmait les jugements des conciles antérieurs (« semence maudite »). Et l'Église continuait d'écarter de la prêtrise les fils de prêtres, sauf dispense (à vrai dire facilement accordée). 

\section{« Bâtards »}

 Ceux que les gens du Moyen Âge nommaient « bâtards » étaient d'abord les enfants abandonnés dont on ne connaissait pas les parents. À partir du \siecle{11}, le mot « bâtard » devient un terme de mépris, une injure. Dès le \siecle{12}, avec la renaissance du droit romain, le bâtard n'appartient plus à aucune famille même dans les pays de droit coutumier%
% [6] 
\footnote{En gros les pays situés au nord de la Loire, opposés aux pays de droit (romain) écrit, situés au sud de la Loire.} 
: pas même celle de sa mère. Être un « bâtard » était une tare, et semble avoir été de plus en plus pénalisant du Moyen Âge au \siecle{18} : est-ce vraiment pour des raisons religieuses ? ou parce que la société reposait sur l'alliance des familles, alliance que protégeait la stigmatisation des naissances irrégulière ?

 Dans une société où l'on n'était fils ou fille de quelqu'un que si l'on était né de deux parents légitimement mariés, un jeune de naissance illégitime ou né de parents inconnus portait une tare indélébile. Il était considéré comme fils de personne, hors parenté, même s'il vivait au foyer de l'un de ses deux parents (la mère en général). Un enfant non légitime ne pouvait ni succéder à un membre de sa parentèle dans un office, ni en hériter, sauf si aucun autre héritier légitime n'y trouvait ombrage. Personne ne se portait caution pour lui. Sans père il ne pouvait pas apporter sa contrepartie dans un système d'alliance. Il était condamné à une position marginale, du moins par rapport à celle de ses éventuels demi-frères et sœurs. En contrepartie de ces exclusions il n'était pas non plus contraint de se porter caution pour un parent. Il n'avait aucune autorisation parentale à demander pour convoler : ses géniteurs comme ses tuteurs ne pouvaient pas le lui interdire.

 L'Hôpital du Saint Esprit de Paris était initialement un hospice destiné à toutes les personnes démunies. À la fin du Moyen Âge il était devenu l'orphelinat de Paris. Vers le milieu du \siecle{15} le roi lui a demandé de prendre en charge les enfants abandonnés de Paris. Tout roi qu'il fût ses demandes réitérées ont été récusées par les maîtres de l'un des hôpitaux de sa propre ville : \emph{en faisant prévaloir les statuts de fondation et la bonne réputation dont jouissent les enfants qu'il \emph{[l'hôpital]} entretient et éduque déjà. Si tous les orphelins d'origine inconnue lui étaient conduits, les gens de métier qui viennent chercher un apprenti, ou les jeunes compagnons qui y prennent femme ne seraient plus assurés de la légitimité, et partant de la moralité de l'adolescent}%
% [7]
\footnote{\fsc{SAUNIER}, \emph{Le « pauvre malade » dans le cadre hospitalier médiéval, France du Nord}, vers 1300-1500, 1993, p. 53.}% 
.

 Réellement convaincu par ces arguments, ou bien de guerre lasse, le roi a confirmé les maîtres de l'Hôpital du Saint-Esprit dans l'idée que leurs statuts et eux-mêmes se faisaient de leur mission : en 1445 il a donc accepté qu'on n'y admet que les \emph{orphelins et orphelines nés en loyal mariage} et à qui on ne peut reprocher \emph{la tache de bâtardise}, car, selon un autre argument fourni par les maîtres de l'hôpital, \enquote{[...] \emph{ſy on y admettoit des baſtards, il ſeroit à craindre que la division ne ſe miſt bientôt dans cette maiſon par les reproches continuels que les enfants légitimes feroient aux baſtards}%
% [8]
\footnote{Lettres patentes de Charles~VII du 7 août 1445, A.A.P. Saint-Esprit, L, II, p. 32 ; cité par \fsc{SAUNIER}, id. p. 212.}% 
}.

 À cette époque les maîtres du Saint-Esprit faisaient remettre tous les enfants exposés qu'on leur présentait aux paroisses sur le territoire desquelles on les avait trouvés%
% [9]
\footnote{\fsc{SAUNIER}, id. p. 212. On pourrait se demander où et comment les paroisses pouvaient prendre en charge ces enfants ? Elles n'avaient ni hôpital ni hospice pour cela. Cela n'est jamais précisé parce que c'était probablement évident : elles faisaient ce que faisait n'importe quelle artisane encombrée d'un enfant. Elles payaient une nourrice pour s'en occuper, en attendant qu'il soit assez grand pour être mis \emph{en condition}, au travail, au même âge que les plus pauvres de ses contemporains légitimes, c'est-à-dire dès l'âge de raison, bien avant ses dix ans.}%
, alors qu'ils acceptaient de prendre en charge les adolescents légitimes (orphelins pauvres) qui sortaient convalescents de l'Hôtel-Dieu. 

 Même les léproseries excluaient les bâtards \enquote{\emph{parce que les gardes des maladreries diſaient que les bâtards n'avaient pas de lignage, ni n'étaient à hériter de nul droit par quoy ils ne ſe pouvaient aider de leur maiſon, pas plus qu'un étranger qui ſerait venu d'Eſpagne}%
% [10] 
\footnote{\fsc{SAUNIER}, id. p. 213.}%
}. Mais ces mêmes léproseries admettaient sans réserves les lépreux sans ressources s'ils étaient de naissance légitime : l'indigence leur faisait encore moins peur que l'illégitimité.

 Jusqu'à la fin de l'ancien régime les « bâtards » étaient exclus de nombreux métiers. En règle générale les corporations les refusaient, de la même façon et pour les mêmes raisons que la prêtrise leur était interdite. Maître Jacques \fsc{Ducros}, avocat au Parlement de Bordeaux, et premier Consul d'Agen en 1659, écrit dans ses \href{http://www.babordnum.fr/files/original/859d36685f2d7b2f871c648ea08bd103.pdf}{\emph{Réflexions singulières sur l'ancienne coutume d'Agen}}  :
%
\begin{displayquote}

{[...] \emph{les batards n'ont pas le bonheur de paſſer pour des domeſtiques%
% [11] 
\footnote{Ici « domestiques » signifie « appartenant à une maison », pas forcément comme salarié au service du maître. Cela inclut aussi tous les enfants et parents vivant dans la maison.} 
ny d'auoir rien en propre dans les maiſons. Ils ſont les productions du vice \& les enfans d'iniquité. Les peres les forment dans les tenebres \emph{[et]} les meres en cachent la conception. A meſme qu'ils sont nez , ces infames producteurs les deſauoüent. Les enfans legitimes cherchent le iour \& la lumiere, les illegitimes la nuit \& l'obſcurité. A proprement parler ce ſont des excremens, deſquels à meſme que la nature les chaſſe \& les pouſſe dehors, on couure d'ordure \& de ſaleté : ils n'ont ny nom ny race ny famille , c'eſt pourquoy ils ne peuuent eſtre admis au nombre des proches de ſang de conſanguinité ny d'allience}}%
%[12]
\footnote{\fsc{CAPUL}, Thèse, tome II, p. 111.%
\label{notecapul111}}%
.

\end{displayquote}


 \fsc{CAPUL} rapporte que lors des États généraux de 1614, le Tiers-état d'Agenais demande au roi : \enquote{\emph{que toutes lettres de légitimation ſeront deſnyees a tous enfens nez d'inceſte, d'adultère ou filz de prebſtres, et qu'on n'y aura aucun eſgard, ſoit pour ſucceſion, dignites, offices, bennefices (eccléſiaſtiques) et tous autres droitz}}%
% [13]
\footnote{%\fsc{CAPUL}, Thèse, tome II, p. 111.}%
Voir note \ref{notecapul111}.}%
. Les places désirables étaient trop peu nombreuses pour se montrer généreux. Ceci dit leur démarche conforte l'idée que les interdits qui pesaient sur les « fruits du péché » pouvaient assez aisément être tournés avec de l'argent et de l'entregent, mais cela ne concernait que les rares enfants illégitimes qui étaient investis par des parents puissants ou fortunés. Ainsi Erasme, (1469-1536), « prince des humanistes », âme de la « république des lettres » de son temps, était-il fils de prêtre. Fils d'un père cultivé il reçut une instruction soignée dans les écoles monastiques de son temps et entra en 1688 chez les chanoines de Saint-Augustin, où il fut ordonné prêtre en 1492. S'il ne fit pas une brillante carrière dans les allées du pouvoir temporel, comme évêque ou cardinal, c'est parce qu'il refusa les offres qu'on lui en fit au profit de la recherche intellectuelle et théologique, où il réussit il est vrai de manière exceptionnelle. Sa bâtardise et le statut ecclésiastique de son père ne semblent avoir fait problème à personne.

 Les bourgeois prospères qui représentaient leurs concitoyens de l'Agenais, et qui exprimaient probablement l'opinion publique de leur époque, ne s'identifiaient en aucune façon aux enfants nés des unions sexuelles illicites, pourtant innocents des actes de leurs pères et mères. Ces députés tenaient fermement à ce qu'aucun passe-droit ne puisse désavantager leurs propres fils dans la course aux honneurs. C'était la défense la plus intransigeante de la morale conjugale qui servait leurs intérêts, puisqu'elle leur permettait d'écarter une partie des concurrents nobles ou bourgeois de leurs propres enfants. Ils ne toléraient pas que les « bâtards » soient confondus avec les enfants légitimes, et surtout pas avec les orphelins. Presque tous se félicitaient quand les distinctions étaient rigoureusement défendues.
 
 

\section{Conflits de juridictions}

 Le droit romain n'a jamais totalement disparu dans les pays de droit écrit, au Sud de la Loire ou en Italie, mais à partir de sa redécouverte au \siecle{12} il a connu une nouvelle faveur en tant que modèle et outil de réflexion. La Renaissance a vu le triomphe du droit tel que les empereurs chrétiens l'ont mis en forme%
% [4]
\footnote{Justinien~I\ier{} (483 - 565) ou Justinien le Grand, empereur de Byzance de 527 à 565, essaya de restaurer l'unité de l'empire romain. Il a ordonné et dirigé une compilation du droit romain, le \latin{Corpus iuris civilis}, qui est l'une des bases du droit civil de divers pays européens.}% 
. Si bien qu'au bout de plus d'un millénaire, c'étaient encore les choix de Constantin et de ses successeurs immédiats qui modelaient en profondeur les mœurs familiales européennes : celles-ci n'ont jamais été aussi conformes à ses décrets qu'aux \crmieme{16} et \siecle{17}. À partir de la Renaissance et jusqu'au \siecle{20} les femmes \emph{mariées} ont été pratiquement réduites à un statut de mineures. Par rapport au Moyen Âge le retour en faveur du droit romain a appesanti l'autorité des pères sur les enfants, et contribué à enlever aux femmes, et surtout aux épouses, une part significative des libertés économiques et juridiques que le Moyen Âge leur avait reconnues.

 Du \crmieme{13} au \siecle{18} les autorités civiles reprennent peu à peu une grande partie du terrain qu'elles avaient concédé aux autorités religieuses au fil du haut Moyen Âge. Le \emph{Concordat de Bologne} (1516) accorde au roi de France le droit de nommer les titulaires des principaux bénéfices (évêques et abbés et abbesses), en dépit la règle qui depuis l'Antiquité voulait qu'ils soient désignés (élus) par leur communauté. 

 À partir du \siecle{14} un petit nombre de curés ont commencé de tenir des \emph{registres de catholicité} où ils enregistraient les baptêmes (autant dire les naissances dans un monde où tous sont baptisés) et parfois aussi les décès et les mariages. L'intérêt de ces registres était de faire foi dans les procès éventuels, même si manquaient les témoins capables de répondre à des questions concernant par exemple l'âge des personnes, ou leurs liens de parenté,~etc. En raison de cet intérêt quelques évêques ont ordonné à tous leurs curés d'en faire autant. \emph{L'Ordonnance de Villers-Cotterêts} (1539) a généralisé à tous les curés du royaume de France l'obligation d'enregistrer par écrit tous les baptêmes. \emph{L'Ordonnance de Blois} (1579) la complète en ordonnant que soient également notés sur ces registres tous les mariages et tous les décès, ce qui permettait de lutter contre les bigames. À cela s'ajoute l'obligation légale d'une publication des bans préalable au mariage, préconisée depuis longtemps par les conciles, mais appliquée de manière irrégulière, afin que ceux qui connaîtraient un empêchement au mariage projeté puissent le déclarer en temps utile. Par ces diverses initiatives et par d'autres les autorités civiles ont repris pied dans le domaine matrimonial. Le contrôle des unions importait en effet au moins autant aux rois et aux parents qu'à l'Église, et les mariages avaient un effet déterminant sur le bon fonctionnement de la société civile, sur la paix des familles et sur l'organisation économique. Les juges royaux ont cherché et trouvé, ou reçu du roi, des moyens de contester certaines des décisions des juges ecclésiastiques, si bien qu'on a pu observer un retour progressif du contentieux des mariages devant les tribunaux civils. 

 Le conflit le plus rude entre les autorités civiles et les autorités religieuses a porté sur la place à donner à l'autorité des parents sur les unions. Selon l'Église catholique les conjoints s'unissaient irrévocablement l'un à l'autre par leur « oui », devant le prêtre qui n'était qu'un témoin représentant l'Église, un témoin privilégié à partir du moment où il tenait les registres d'état civil (le curé de la paroisse de l'un ou l'autre des époux sauf dispense). La position traditionnelle de l'Église était que l'autorisation parentale n'était pas nécessaire pour la validité du mariage, même si elle déconseillait aux jeunes gens de s'en passer et si elle ne contestait pas aux parents le droit de déshériter les contrevenants. Les autorités civiles et les familles pensaient au contraire qu'un mariage, alliance entre deux familles et contrat civil, ne pouvait pas être valide, quel que soit l'âge des conjoints, sans l'accord formel de leurs parents. Du point de vue de ces derniers (et des clercs eux-mêmes lorsqu'ils ne parlaient pas en tant que représentants de l'Église, mais en tant que membres d'une famille particulière) l'absence de cet accord était une preuve du manque de bon sens, de l'immaturité des deux jeunes concernés, ou de la perversité de l'un d'eux (cf. la réaction du chanoine Fulbert, oncle d'Héloïse, face au mariage secret, et néanmoins canoniquement valide, de sa nièce avec Abélard). 

 Malgré la pression du roi de France, et le besoin qu'ils avaient de son appui et de ses armées, les évêques rassemblés en concile à Trente ont refusé de modifier la doctrine traditionnelle. Le roi a donc promulgué \emph{l'édit de 1556} qui ne contestait pas la validité religieuse des mariages célébrés sans l'accord des parents \emph{(mariages clandestins)} mais qui les déclarait civilement illégaux. Il confirmait le droit traditionnellement accordé aux parents de déshériter les enfants qui se rendaient coupables de tels mariages. Il décidait surtout que l'instigateur ou l'instigatrice d'un tel mariage (c'est-à-dire celui qui avait à y gagner, en principe le plus pauvre) pouvait être condamné à la peine de mort pour \emph{rapt}, ce qui réglait radicalement la question de l'indissolubilité du mariage. Cet édit a été en vigueur jusqu'à la Révolution, et semble avoir réglé le problème à la satisfaction des pères et des mères de familles. 
 
 \section{Les protestant, le mariage et le sexe}

 La position des protestants était très proche de celle du roi de France. Pour eux le mariage n'était pas un sacrement mais seulement un contrat entre deux personnes, par nature révocable, et du ressort des seules autorités civiles. Selon Luther (Traubüchlein, 1529) : « \emph{il faut laisser à chaque ville et à chaque pays ses us et coutumes tels qu'ils sont pratiqués \emph{[... tous ces usages]} c'est aux princes et aux magistrats qu'il appartient de les établir et de les régler} ». Le roi et les pères et mères étaient d'accord sur l'idée que le choix d'un conjoint était trop important pour être laissé à la discrétion des futurs époux.
 
 
 Mais le fait de dénier au mariage le poids d'un sacrement et de n'y voir qu'un contrat ne lui enlevait pas une certaine forme d'indissolubilité. Même Henri~VIII n'avait pas rompu le lien entre l'Église d'Angleterre et Rome pour divorcer, mais pour faire reconnaître par les évêques de son royaume la nullité de son premier mariage. À partir du moment où le mariage était invalide si les parents des futurs conjoints ne lui apportaient pas leur approbation, comme pour tout autre contrat. Il était du devoir des jeunes gens matériellement dépendants de leurs parents de se soumettre à la volonté de ces derniers dans ce domaine-là comme dans tous les autres. Une fois accordés par leurs parents au terme de négociations plus ou moins âpres entre ces derniers et après que le père de la mariée ait remis celle-ci en mains propres à son futur gendre, les époux se devaient de respecter la volonté de leurs auteurs et de rester ensemble en dépit des difficultés éventuelles. C'est pour cela que tout en reconnaissant aux époux le droit de divorcer, les protestants leur imposaient tant de conditions (en Angleterre il y fallait entre autres un acte du parlement) que leurs divorces étaient en réalité difficiles à obtenir et coûteux (800 livres au \siecle{19} en Angleterre) et donc rares : en Angleterre 184 divorces entre 1715 et 1852, pour 9 millions d'habitants environ ; au Massachusetts, état américain bien plus libéral, 143 divorces entre 1692 et 1786 pour \nombre{300000} habitants environ (un et demi par an !) 
 \footnote{Indépendamment des doctrines il serait extrêmement intéressant de comparer la réalité des pratiques chez les catholiques et chez les protestants : pays par pays et dénomination par dénomination : tâche immense !}. 


 

% Le 10 mars 2015 :
% Antiquité
% ~etc.
% Moyen Âge
% Droit

\chapter{L'évolution des pratiques éducatives}


 \section{Paternité et absolutisme}


Vécu comme un père par ses sujets, le roi s'identifiait à son tour à tous les pères de famille. Eux et lui étaient autant de représentants de Dieu « le Père » et ils se confortaient les uns les autres. Il en avait plus ou moins été ainsi depuis toujours (est-ce d'ailleurs si différent avec les dirigeants politiques de l'heure actuelle ?) mais à la fin du Moyen Âge le fonctionnement des familles semble avoir eu tendance à se rigidifier dans un patriarcat de plus en plus rigoureux, en même temps que les états montaient en puissance et que les doctrines absolutistes gagnaient de l'audience. 

Qu'ils soient protestants ou catholiques les philosophes, les théoriciens du droit et les chantres de l'absolutisme (Jean Bodin, Omer Talon, Bossuet, Thomas Hobbes,~etc.) soutenaient la nécessité d'un pouvoir fort, incarné par un souverain aussi absolu que possible : absolu, c'est-à-dire sans contre-pouvoirs significatifs. On peut supposer que cela découlait pour une part de l'expérience des guerres, civiles ou entre états, dans lesquelles les européens se sont laissés entraîner par les divergences entre options religieuses coexistantes. Cette expérience a révélé la violence mortelle que peuvent provoquer les identités religieuses. Elle a aussi révélé les limites de la capacité du pape et des évêques à réguler pacifiquement les conflits d'interprétation des discours de référence (écritures, théologiens...), ce qui les a délogés de leur millénaire position d'autorités et de partenaires autonomes et incontournables des autorités civiles. Bon gré mal gré les européens s'en sonr donc remis à "\emph{César}", quitte à s'accomoder de son despotisme ("\emph{cujus regio ejus religio}") et des injustices de la raison d'état. Tout valait mieux à leurs yeux que les désordres d'un monde où chacun serait un loup pour l'autre. 

Les ecclésiastiques eux-mêmes se sont ralliés à cette position : les protestants bien entendu, qui déniaient toute autorité particulière à l'éveque de Rome, mais aussi les catholiques. Ainsi Pierre de Bérulle (fondateur de l'Ordre de l'Oratoire) écrivait en 1623, dans un discours\footnote{\emph{Discours de l'État et des grandeurs de Jésus}.} au Roi (Louis~XIII)  :
    \enquote{\emph{un monarque est un Dieu selon le langage de l'écriture : un Dieu non par essence mais par puissance ; un Dieu non par nature mais par grâce ; un Dieu non pour toujours mais pour un temps. Un Dieu non pour le Ciel mais pour la Terre. Un Dieu non subsistant, mais dépendant de celui qui est le subsistant par soi-même ; qui étant le Dieu des Dieux, fait les rois Dieux en ressemblance, en puissance et en qualité, Dieux visibles, images du Dieu invisible}}. Jusqu'au milieu du \siecle{18} l'image de l'autorité était globalement positive, et l'exercice que les rois et les pères (et avec eux les « pères spirituels » de tous ordres) faisaient de leurs pouvoirs était regardé comme légitime et bénéfique. Dans ce cadre de pensée s'opposer au souverain comme aux pères c'était faire preuve de présomption et peut-être s'opposer à Dieu lui-même. 
    
    
    Dans quelle mesure cette vision du pouvoir et de la paternité a-t-elle rejailli sur l'image que les gens d'alors se faisaient de Dieu ? Ils prêtaient en effet à celui-ci une très grande dureté et meme une cruauté impitoyable : arbitraire de la grâce, arbitraire de la prédestination, dureté des exigences morales, poids de la culpabilisation, fréquence de la damnation, rareté des élus, menace omniprésente de l'enfer,~etc. Mais on peut aussi bien se demander si ce ne sont pas les thèses des théologiens de la fin du Moyen Age qui ont renforcé l'absolutisme des pères et des rois. Ils valorisaient en effet sans limites la toute-puissance divine. Selon Jean-Claude Monod : "\emph{L'importance du nominalisme à la fin du Moyen Âge tient à ce que ce courant de pensée a mis en crise le système scolastique en voulant pousser l'homme à une capitulation sans condition dans l'acte de foi, et a retiré à la théologie toute tâche de médiation entre la connaissance et la foi. Ainsi en est-il de la souveraineté absolue de Dieu : volonté insaisissable et opaque "potentia absoluta", le Dieu du nominalisme et ses "décrets" se situent au-delà de toute tentative de compréhension par l'esprit humain. Tout ce qui est fait peut être défait, toute loi peut être suspendue, nulle garantie ne doit être attendue de Dieu, dont l'entendement est incommensurable au nôtre et dont dépend pourtant entièrement notre salut."} Comment penser la liberté et la responsabilité des individus si Dieu connaît à l'avance tout leur avenir ? Comment imaginer Dieu comme bon s'il n'est lié par aucune exigence de justice ? etc. Ces difficultés ont contribué à l'éclatement de la chrétienté médiévale\footnote{ in \emph{La querelle de la sécularisation : théologie politique et philosophie de l'histoire de hegel à Blumenberg}, Paris, Vrin, 2002.}.       
    
    
    Puisqu'il n'était plus possible de faire confiance à Dieu ni à ses ministres pour contenir les désordres il fallait conforter les autres autorités. Les pères se voyaient donc confirmés dans leur devoir de maintenir leur maison en bon ordre, dans le respect des lois civiles et religieuses. On attendait d'eux qu'ils le fassent sans faiblir, quelles que soient les difficultés. Pour y parvenir il leur était reconnu une grande part de la puissance paternelle des romains. Dans les pays de Droit écrit, comme le sud de la France, revenus avant la fin du Moyen Âge à une application stricte du Droit romain, leur puissance ne cessait qu'avec leur mort. Partout leur mission éducative impliquait le \emph{droit de correction}. On considérait que c'était pour eux un devoir moral et social que de corriger les enfants \emph{et les épouses} indisciplinés. Jusqu'au \siecle{18} (au moins) il était admis qu'une tendresse excessive était plus dommageable, et donc plus coupable, qu'une sévérité excessive : « {qui aime bien châtie bien} ». Montaigne nous dit qu'il fut placé de sa naissance à l'âge de quatre ans chez des bûcherons, puis mis en pension en collège à partir de six ans. Il dit s'être trouvé mieux de cette enfance loin de sa famille... parce qu'il lui semblait que son père était « trop tendre%
% [1] 
\footnote{En justifiant la décision de son père par son « excès de tendresse » Montaigne nous fournit un bel exemple de ce qu'on désigne aujourd'hui sous le nom de « fidélité » ou de « loyauté » des enfants, et des trésors de compréhension dont ils sont prêts à faire preuve face à toutes les décisions, quelles qu'elles soient, que leurs parents ont pu prendre.} 
» !

 Le roi soutenait l'autorité des époux sur leur épouse et leurs enfants, et il leur prêtait main-forte s'ils le demandaient, entre autres moyens par les \emph{lettres de cachet} ordonnant sans jugement\footnote{...ancêtre des placements administratifs actuels, dont il faut reconnaître qu'ils sont mieux contrôlés qu'alors par les autorités judiciaires. Il est infiniment plus aisé aujourd'hui de mettre en question leur pertinence parce que nous n'idéalisons plus la parole des pères ni des autres autorités. Au contraire nous les tenons en suspicion.} l'incarcération de l'enfant récalcitrant, mineur ou majeur, ou de l'épouse indigne, volage ou frivole ou de mauvais caractère,~etc. S'il le jugeait nécessaire, il pouvait se substituer de sa propre initiative%
% [2] 
\footnote{De même que lorsqu'il s'agit de ses enfants un père n'attend pas d'être saisi : par définition il parle en leur nom et à leur place (et en Droit romain même quand ils sont adultes).} 
aux pères défaillants dans leur fonction de faire régner l'ordre dans leurs familles. 
 Mais il se devait aussi de contrôler qu'ils n'abusaient pas de leurs pouvoirs : leur droit de correction n'était pas un droit de vie ou de mort. Jamais les parents n'ont été autorisés à estropier leurs enfants, et l'appui donné par la force publique à leurs décisions n'était pas automatique.

 

\section{Les enseignements}

    
   Il n'est pas question ici de faire une histoire de l'enseignement, mais seulement d'en esquisser les traits qui ont directement rapport à notre sujet\footnote{\\\fsc{FURET} et \fsc{OZOUF}, \emph{Lire et écrire, l'alphabétisation des français de Calvin à Jules Ferry}, 1977.
\\Maurice \fsc{CAPUL}, \emph{Internat et internement sous l'ancien régime, contribution à l'histoire de l'éducation spéciale}, Thèse d'état, CTNERHI-PUF, Paris, 1983-1984.
\\Martine \fsc{SONNET}, {« Une fille à éduquer », in \emph{Histoire des femmes en Occident}, III, \siecles{16}{18}}, Collectif, sous la direction de Georges \fsc{DUBY} et Michelle \fsc{PERROT}, 2002, Chapitre 4, p. 131 à 168.
\\Sous la direction de Marie-Madeleine \fsc{COMPERE} et Philippe\fsc{SAVOIE}, \emph{L’établissement scolaire. Des collèges d'humanités à l'enseignement secondaire, XVIe-XXe siècles}, numéro spécial 90 de la revue \emph{Histoire de l’éducation}, mai 2001
\\ Marie-Madeleine \fsc{COMPERE}, \emph{Du collège au lycée. Généalogie de l'enseignement secondaire français (1500-1850)}
Collection Archives (n° 96), Gallimard,1985 
\\sous la dir. de Marie-Madeleine \fsc{COMPERE} et d'André \fsc{CHERVEL}, \emph{Les Humanités classiques}, Paris : Institut national de la recherche pédagogique, 1997
\\Marie-Madeleine \fsc{COMPERE},	\emph{L'histoire de l'éducation en Europe : essai comparatif sur la façon dont elle s'écrit} Paris : Institut national de recherche pédagogique ; Bern : P. Lang, c1995. }.

 
Jusqu'à la fin de l'ancien Régime il n'existait rien qui ressemblât à un enseignement public, et l'ensemble du domaine scolaire était sous le contrôle et à la charge des évêques. Ceci dit des décisions royales répétées au fil des siècles (à commencer par Charlemagne) confirmaient ceux-ci dans leur droit et dans leurs obligations : ils remplissaient une mission de service public en plein accord avec les autorités civiles (ce qui n'excluait pas la possibilité de frictions). L'enseignement était assuré en majeure partie par des prêtres et des religieux(ses), même si des laïcs y collaboraient parfois. En principe il était gratuit. Le financement venait de subventions, de dons, ou de taxes affectées ou du produit de fondations qui faisaient partie des biens ecclésiastiques.
 
Le réseau des écoles autres que monastiques s'est développé avec des hauts et des bas dès le début du Moyen Âge (cf. troisième Concile de Vaison, en 529) à partir des \emph{écoles cathédrales}, d'une part vers l'enseignement élémentaire avec les \emph{petites écoles} (écoles primaires), d'autre part vers l'enseignement supérieur (et secondaire) avec les \emph{universités} et leurs \emph{collèges}.  Elles étaient placées sous le contrôle du Chapitre de la Cathédrale. L'un des chanoines exerçait cette responsabilité l'\emph{écolâtre},  le \emph{chantre} ou le \emph{chancelier}... C'est lui qui jusqu'à la fin de l'ancien régime agréera tous les candidats à l'enseignement, agrément sans lequel nul n'avait le droit d'enseigner sur le territoire sous sa juridiction.

Enseignement primaire

À partir des derniers siècles du Moyen Âge des \emph{petites écoles} paroissiales existaient dans toutes les villes importantes, fondées par les curés, ou par les municipalités, et ordinairement par les deux à la fois. A côté des connaissances profanes (d'abord la lecture, le calcul, souvent l'écriture, mais pas toujours) on enseignait aussi la religion, les disciplines du corps et de l'esprit, les bonnes manières de se conduire. Leur mission était en effet d'éduquer autant que d'enseigner. L'instruction, une fois entendu qu'elle se devait d'inclure la religion, était considérée comme la meilleure défense contre l'envie de mal faire. Curés ou pasteurs protestants, parents et autorités locales étaient d'accord sur ce point. D'autre part les citadins voyaient aussi en elle la meilleure arme pour trouver et pour garder un travail, ce qui avait à la fois un intérêt économique et un intérêt social. À partir de la Réforme et du Concile de Trente cette foi en l'enseignement s'est exprimée en un véritable apostolat. C'est pourquoi divers ordres enseignants ont été créés au fil des siècles :
%\begin{description}
%\item[
\siecle{12} : religieuses bernardines,
\siecle{15} : Frères de la vie commune, Minimes,
\siecle{16} : Jésuites, Ursulines, Bénédictins de Saint-Maur, Prêtres de la Doctrine chrétienne,
\siecle{17} : Oratoriens, Dames de Saint-Maur, Piaristes, Religieuses de Notre-Dame, Dames de la Providence, Frères des écoles chrétiennes,~etc. (pour ne citer que les principaux).
%\end{description}

 Les petites écoles s'adressaient aux « enfants des pauvres », c'est-à-dire, dans le langage d'alors, de tous ceux dont les ressources étaient précaires, ceux qui n'avaient pas de rentes, de quelque nature qu'elles soient, et qui devaient gagner leur vie en travaillant. Il s'agissait donc de l'essentiel de la population des villes. Mais les petites écoles ne pouvaient pas toujours être complètement gratuites (pas plus que les universités). Elles étaient donc à la portée des bourgeois aisés, des commerçants et artisans, mais pas toujours à celle des autres. Lorsque les paroisses ne pouvaient pas exempter ces derniers des frais de scolarité, ce qui était le cas lorsque l'ensemble de leurs paroissiens étaient réellement pauvres, seuls de généreux donateurs et surtout des ordres religieux pouvaient les prendre en charge (cf.  les « écoles de charité »). Les religieux bénéficiaient en effet d'une sécurité financière, d'une surface sociale et d'un entregent que ne pouvaient avoir des particuliers ou des communes pauvres. Certains ordres avaient d'ailleurs explicitement pour vocation d'assurer gratuitement l'enseignement des indigents.

Beaucoup d'enfants n'étaient pourtant pas scolarisés, même dans les villes où l'enseignement était gratuit : leurs parents avaient trop besoin du produit de leur travail, ou bien ils ne voyaient aucune utilité à un apprentissage scolaire. Même aux yeux de ceux qui envoyaient leurs enfants à l'école il n'était pas toujours évident qu'il faille que ceux-ci soient scolarisés avec assiduité pendant plusieurs années. Beaucoup, et peut-être meme la plupart, se contentaient des quelques mois ou années nécessaires pour apprendre à lire et/ou à écrire. Les enfants de famille aisée étaient traditionnellement confiés à un précepteur.  

Quant à l'instruction des paysans (plus de 90 \% de la population) jusqu'à la fin de l'ancien régime elle n'était pas jugée nécessaire. Etant donné le niveau des techniques alors en usage l'illettrisme n'avait pas d'incidence sur leur productivité . D'autre part les maîtres et seigneurs craignaient qu'une instruction même minime ne les rende « raisonneurs » et « arrogants »\footnote{Voltaire, représentatif de sa classe et de son temps, sera lui aussi de cet avis.}.

Enseignement secondaire 

Les écoles cathédrales et les écoles monastiques ont été créées dès la fin de l'Antiquité pour fournir l'Église en clercs, mais elles ont toujours reçu un petit contingent d'élèves promis à la vie civile. À partir du \siecle{10} la croissance des villes a provoqué la demande d'une instruction de niveau universitaire (à l'époque le secondaire et le supérieur n'étaient pas distincts). À partir du \siecle{12} les universités se sont créées comme des corporations autogérées de professeurs indépendants, sur l'impulsion de ces derniers et avec l'appui des autorités civiles. Elles étaient le contrôle de l'évêque du lieu (le pape arbitrant les conflits avec lui), et leur personnel comme leurs étudiants bénéficiaient des avantages et exemptions attachés aux clercs (la majorité d'entre eux étaient religieux ou prêtres ou destinés à le devenir). En lien avec les universités ont été créés des collèges avec internat pour les boursiers pauvres, sur le modèle des écoles monastique (ex. : le collège qui deviendra la Sorbonne). Leur mission était de permettre aux jeunes gens sans fortune d'entrer dans le clergé (il fallait un titre universitaire pour entrer dans le clergé) mais peu à peu ces collèges d'universités sont devenus des lieux d'enseignement recherchés par d'autres candidats. 

La formule du collège s'est généralisée à partir du XVIème siècle tout en se transformant profondément : externat presque excusif et non internat, une majorité d'élèves promis à la vie laïque, réduction forte de la dispersion des âges, classement des élèves par niveaux, etc.. A partir du XVIème siècle de nombreux établissements ont été créés à la demande des municipalités et/ou des évêques. Les initiatives étaient très décentralisées et les créations partaient le plus souvent des besoins et des demandes locales, et d'abord des demandes des parents d'élèves potentiels. Au fil du temps la gestion de beaucoup de ces nouveaux collèges ont été confiés à des ordres religieux spécialisés comme les jésuites ou les oratoriens (mais d'autres aussi, comme les bénédictins...). Les collèges n'étaient pas toujours de "plein exercice", avec des classes de tous niveaux jusqu'à la classe de philosophie comprise. En fait beaucoup d'entre eux (les "petits collèges") se contentaient de quelques niveaux, les élèves désireux d'aller jusqu'au bout du cursus secondaire  devant le terminer dans un autre établissement\footnote{..quand la commune ne se bornait pas à entretenir un seul professeur de latin (une \emph{régence latine}) pour les quelques élèves concernés.}. 

Au fil du temps la scolarité secondaire a eu tendance à se décentraliser, sous la pression de municipalités sans que pour autant le nombre d'élèves du secondaire n'ait augmenté. Il est resté très stable pendant très longtemps : moins de un pour cent de la population des jeunes garçons d'âge scolaire\footnote{Selon un rapport établi en 1843 par A. F. Villemain  et cité par Antoine LEON (\emph{Histoire de l'enseignement en France}, Que Sais-je ?, PUF, Paris, 1967) il existait à la veille de la Révolution 562 collèges avec 73000 élèves, dont 40000 boursiers : 178 collèges congréganistes et 384 collèges dépendant des universités ou gérés par des communes ou des particuliers. En 1812 il y avait 36 lycées et 337 collèges publics avec 44000 élèves, et 1000 autres institutions et pensionnats privés pour 27000 élèves, soit 71000 élèves au total. En 1880 il y avait environ 150000 élèves dans les lycées et collèges, et 500000 en 1940.}. Il faut aussi souligner que beaucoup d'élèves se contentaient d'une scolarité secondaire réduite à quelques années. 
 
 Pour les parents des collégiens l'éducation était un investissement familial, même quand ils se destinaient à devenir des clercs (ce qui a été le cas d'une minorité très significative jusqu'au \siecle{18}). Cela justifiait qu'ils soient improductifs pendant leurs années de scolarité. Ceux qui étaient sans ressources mais doués pour les études, et en particulier ceux qui se destinaient à entrer dans les ordres, pouvaient bénéficier de bourses.




Quant aux jeunes filles de bonne famille, la clôture des couvents leur interdisait toute rencontre avec les jeunes gens de leur âge. Elle protégeait leur « vertu » et leur réputation en attendant que leurs parents les marient (les jeunes filles des familles aisées, bien dotées, étaient mariées bien plus tot que les autres). Il semble que la durée de leurs séjours dans les couvents ait été très inférieure à celle de leurs frères dans les collèges. L'enseignement qui leur était dispensé était habituellement nettement moins poussé (exclusion du latin,~etc.). Par contre celles qui étaient destinées à la vie religieuses bénéficiaient d'un enseignement qui en faisait des lettrées, des soeurs "de choeur"au minimum capables de chanter les offices en comprenant le latin qu'elles chantaient, et d'enseigner aux éventuelles jeunes filles pensionnaires.

Beaucoup de pédagogues avaient une image positive de l'internat mais entre l'externat des collèges, très souvent gratuit (ex. les collèges jésuites) ou presque, et la pension des internats l'écart des coûts était énorme%
%[5]
\footnote{Selon Martine \fsc{SONNET}, la pension d'un seul enfant, garçon ou fille, représentait presque la totalité du salaire d'un ouvrier (« une fille à éduquer », Chapitre 4 de \emph{l'Histoire des femmes en Occident}, III, \siecles{16}{18}, p. 146). C'est pourquoi en 1760 les internats parisiens n'accueillaient que 13~\% de la population scolarisée de la ville, et il semble qu'il en était de même ailleurs.}% 
. Aux familles qui ne pouvaient payer les frais d'une pension, c'est-à-dire la plupart, seul l'externat était accessible, en vivant en ville chez ses parents (d'où la pression des municipalités pour créer un collège, un petit collège, ou au minimum une classe de latin une"régence latine", pour gagner quelques années de scolarité sans recourir à la pension) ou chez un parent, ou chez un logeur peu exigeant. Dans les collèges il y avait donc ordinairement beaucoup plus d'externes que d'internes et souvent il n'y avait pas d'internat du tout (c'était le cas de la grande majorité des collèges jésuites). 

 
 

Les collégiens étaient confiés soit à l'un des ordres religieux spécialisés à partir de la Renaissance dans l'enseignement (Jésuites surtout, mais aussi Oratoriens, Dominicains,~etc.) soit à des collèges dépendant des universités, soit à des collèges dépendant des municipalitésoù enseignaient des clercs recrutés sur place, de niveau universitaire inégal et aux motivations fluctuantes (beaucoup parmi ces derniers gagnaient leur vie en enseignant en attendant d'obtenir un bénéfice ecclésiastique plus intéressant et plus lucratif). 

Les collèges étaient ouverts à la ville dans les murs de laquelle ils étaient établis. Ils en formaient souvent l'un des fleurons les plus prestigieux. Ils y entretenaient une vie intellectuelle et mondaine active et d'autant plus valorisée que les autres sources de distraction étaient rares. Ils proposaient aux collégiens de s'investir dans la découverte du savoir, et celui-ci était ressenti par leurs enseignants et leurs parents comme quelque chose qui en valait la peine. Ils entraient dans une aristocratie de l'esprit. À l'époque dans toute l'Europe l'enseignement secondaire et supérieur se faisait en latin. Sans lui on savait peut-être lire, mais on n'en demeurait pas moins un \emph{illettré}%
% [7]
\footnote{C'est en latin qu'Héloïse et Abélard se sont écrit toute leur vie. C'est en latin que la République des Lettres de la Renaissance correspondait d'un bout de l'Europe à l'autre. Dans toute l'Europe les thèses de doctorat seront encore soutenues en latin durant la plus grande partie du \siecle{19} (pourquoi avoir cherché à créer un espéranto  au XIX ème siècle ? il était déjà là, bien visible de tous).}% 
. C'était la langue vivante, la langue de communication des communautés intellectuelles du temps. Mais depuis l'\emph{ordonnance de Villers-Cotterêts} (1539) qui imposait le français comme langue administrative du Royaume, il n'était plus possible de tenir un \emph{office} public si on ne le maîtrisait pas suffisamment. La langue française n'était encore que le parler de l'Île-de-France, domaine du roi. Partout ailleurs c'était une langue étrangère qui allait mettre très longtemps à déloger les langues locales des places et des marchés. L'enseignement du français reçu dans les collèges, meme si l'accent était mis sur le latin (surtout chez les jésuites) était donc incontournable pour entrer dans les professions libérales, la fonction publique ou le clergé. 

Conclusion :

 Le collège était pour une élite choisie un moyen sûr de promotion individuelle et familiale. Pour les gens ordinaires c'était la seule voie d'accès aux emplois prestigieux et qualifiés. Le fils de famille confronté à l'épreuve de la vie loin de ses parents continuait de dépendre d'eux. Ils payaient sa pension : il continuait de manger leur pain. Il continuait de correspondre avec eux. Il les retrouverait aux prochaines vacances s'ils ne venaient pas le voir avant. Quand la discipline et le travail intellectuel lui pesaient trop il pouvait se dire avec assez de vraisemblance qu'il était en train d'acquérir à ce prix les moyens d'atteindre un statut personnel valorisant, et qu'il s'inscrivait dans le projet de ses parents. En acceptant de se soumettre à cette exigence il pouvait espérer devenir un membre puissant et respecté de sa communauté d'origine : cela présentait l'allure d'une épreuve initiatique.
 
Par contre pour l'énorme majorité des jeunes l'enseignement se réduisait à quelques années, au mieux, et à l'apprentissage des rudiments de la lecture et de l'écriture. Par contre il ne faudrait pas oublier l'apprentissage d'un métier dont bénéficiaient beaucoup de jeunes (la plupart ?) de façon informelle auprès de leur père ou d'un oncle (dont beaucoup de paysans, de pécheurs...), ou de façon contractuelle (et à titre onéreux) auprès d'un maitre d'apprentissage. 
 
 Dans le même esprit au fil du XVIIIème siècle un enseignement de niveau secondaire \emph{sans latin} a été mis en place à l'initiative en particulier des \emph{frères des écoles chrétiennes} à l'intention des jeunes qui se destinaient à des métiers techniques. De la meme façon ont été créées des écoles militaires, avec internat obligatoire, à destination des jeunes nobles, qui n'accordaient au latin qu'une place mesurée, au bénéfice de l'histoire ou d'autres matières plus adaptées. De meme encore ont été créées des écoles techniques comme l'école des Ponts et Chaussées. Mais ces nouveautés sont déjà le signe du changement d'époque de la deuxième moitié du XVIIIème siècle.

\section{La correction paternelle}

 Tous les jeunes \emph{de famille} n'entraient pas docilement dans les projets parentaux. Certains d'entre eux entraient en conflit ouvert avec leurs parents au-delà des normes reçues (éminemment variables suivant les siècles et les lieux) : vagabonds, fugueurs, jeunes aux fréquentations suspectes, exclus pour indiscipline de collèges successifs, fauteurs de vols domestiques ou d'actes « d'inconduite sexuelle », d'insultes et de voies de faits, « libertins », c'est-à-dire jeunes rétifs à toute mesure éducative,~etc. 

 À la demande de leurs parents, ces jeunes peuvent être traités en tout comme les délinquants con\-dam\-nés. Pour les enfants difficiles des familles aisées il y avait des solutions payantes dans les sections des collèges et internats contemporains affectés à la « correction ». Ceux qui n'en avaient pas les moyens étaient internés avec les délinquants condamnés, dans les sections « de force » des hôpitaux, où s'effectuaient les peines de prison. Leurs parents payaient une pension qui tenait compte de leurs ressources. 

 À partir de la fin du \siecle{17} et de plus en plus souvent au fil du \crmieme{18}, les \emph{enfants de famille}, garçons et filles mineurs \emph{et majeurs}, qui avaient commis de vrais actes de délinquance, mais aussi ceux qui donnaient simplement du mécontentement à leurs parents par leurs fréquentations, leur mauvaise conduite, leur indocilité, leur violence aveugle ou leur absence de sens commun (« insensés »), leurs dépenses inconsidérées, ou leurs dettes de jeu, pouvaient, sur la demande de ces derniers qui exerçaient ainsi leur droit de correction, faire l'objet d'une \emph{lettre de cachet}, c'est-à-dire d'une \emph{décision administrative d'internement} dans un hôpital, une prison, une forteresse, un couvent, un collège, ou même leur déportation aux colonies. Les lettres de cachet, qui ont une origine très ancienne, bien antérieure au \siecle{17}, pouvaient aussi être accordées à l'encontre de conjoints aux comportements répréhensibles (cette mesure a beaucoup plus souvent frappé les épouses que les époux). 

 L'autorité publique n'était pas obligée d'accorder satisfaction aux demandes qui lui était faites, et restait seule juge de l'opportunité de la mesure. Elle était surtout sollicitée à Paris, notamment par les couches populaires, contrairement aux provinces où l'internement administratif était moins facile à obtenir et où les couches populaires n'y avaient guère recours. Même si au fil du temps les lettres de cachet ont fait l'objet de critiques de plus en plus virulentes et si les autorités publiques y répugnaient de plus en plus, les demandes se sont faites de plus en plus nombreuses au fil du \siecle{18}. 

 En effet les familles sollicitaient ces lettres comme une grâce : cela leur évitait la honte causée par la publicité du recours à la justice, le coût d'un procès, et aussi la publicité de la mesure d'enfermement. La réputation du jeune (ou de l'adulte) ainsi placé pouvait s'en relever plus facilement. Cela évitait le contrôle par la justice de la nature exacte des faits et de la proportionnalité des sanctions aux dommages et délits constatés, ce qui permettait à l'occasion à d'authentiques délinquants bien nés d'échapper à peu de frais aux conséquences normales de leurs actes. 

Mais cela permettait aussi aux parents abusifs d'exercer des pressions sur leurs enfants rétifs à leurs projets (ce qui expliquait les critiques de plus en plus virulentes des lettres de cachet au fil du \siecle{18}), à une époque où le consentement des parents était exigé à tout âge et pour tout mariage sous peine d'exhérédation, et où bien des entrées en religion étaient imposées par eux sans tenir compte des désirs du ou de la jeune concerné. 

\section{Enfants « adoptifs »}

 On a vu que dans le but de défendre le mariage monogame et indissoluble, l'Église a tout fait depuis l'Antiquité pour que les enfants illégitimes ne puissent pas devenir des héritiers de plein exercice. C'est pour cette raison que l'adoption était interdite, et pourtant... De l'Antiquité à la fin de l'ancien régime, on peut observer en nombre non négligeable des situations plus ou moins proches d'une adoption, où une personne, souvent un ecclésiastique (cf. \hbox{Villon}, adopté par un chanoine), souvent aussi un couple sans enfants, exerçaient la puissance paternelle sur un enfant qui n'était pas né d'eux et qu'ils élevaient jusqu'à sa majorité. C'était par exemple le cas à Lyon, où les recteurs de l'Hôtel-Dieu « adoptaient » ainsi des orphelins. 

 Ces situations d'\latin{alumnii} (adoptions simples) étaient parfois sanctionnées par des actes juridiques où les nourriciers faisaient un legs à l'enfant devant un procureur fiscal, et où ils s'engageaient à l'élever, instruire et établir matériellement à leurs frais comme leur propre enfant. Pour autant cela ne faisait pas de lui un membre de leur famille ni un héritier. 

 En principe seul un enfant légitime sans parents pouvait bénéficier de ce dispositif. Souvent, probablement le plus souvent, il était orphelin, mais des enfants légitimes pouvaient aussi être abandonnés solennellement par leurs parents, qui reconnaissaient par écrit qu'ils renonçaient à leur puissance paternelle, et à l'héritage de leur enfant s'il décédait. Pour autant ce dernier ne changeait ni de parenté ni de nom. Quand il possédait des biens, l'adoptant, tel un tuteur, les gérait jusqu'à sa majorité et il était responsable sur ses propres biens de sa gestion. 

 Les enfants abandonnés, nés de parents inconnus, ont très longtemps été exclus de ce genre de prise en charge%
% [8]
\footnote{À Lyon jusqu'en 1765. Ensuite ils y ont été traités comme les autres. Ce n'est que dans les dernières années avant la Révolution que les idées ont changé sur ce point : un signe de l'évolution qui s'est faite dans les esprits au fil du \siecle{18} et qui est apparue au grand jour à partir des années 1760-1770.}%
. Pourtant il était courant que des personnes accueillent pour l'élever un enfant abandonné à eux confié par un hôpital ou par une paroisse, qu'elles refusent d'être rémunérées pour l'élever, qu'elles le gardent jusqu'à sa majorité et qu'elles l'établissent dans la vie, ce qui en fait ressemblait beaucoup à la situation des enfants nés légitimes et juridiquement « adoptés ». Si aucun de leurs héritiers légitimes ne s'y opposait, elles faisaient de lui l'un de leurs héritiers. Mais il n'était pas question pour cet enfant d'hériter d'une fonction impliquant l'exercice public du pouvoir. 

 Derrière les mots employés il n'est pas toujours facile de reconnaître les situations réelles : adoption simple ? tutelle ? parrainage ?%
% [9]
\footnote{Cf. Jean-Pierre \fsc{GUTTON}, \emph{Histoire de l'adoption en France}, 1993.} 




% 28.02.2015 :
% haut Moyen Âge
% _, --> ,
% Antiquité
% Le 24.02.2015 :
% ~etc.
% Moyen-Âge
%~\%


\chapter[Création d'une police des familles --- \siecles{14}{18}]{Création d'une police des familles\\\siecles{14}{18}}


\section{Organisation d'une police des pauvres}

 À la fin du Moyen Âge il était courant que les mendiants représentent 10~\% de la population%
% [1]
\footnote{José \fsc{CUBERO}, \emph{Histoire du vagabondage}, 1998, p. 8.}% 
. Les solutions en vigueur depuis la fin de l'antiquité pour traiter l'indigence et les malheurs individuels, pensées pour de petites communautés rurales où tous se connaissent%
%[2]
\footnote{José \fsc{CUBERO}, 1998, p. 42 et suivantes.}% 
, n'étaient plus à l'échelle des problèmes en un temps où les villes débordaient de leurs murailles anciennes, et où les États modernes se constituaient, imposant plus d'ordre, de rigueur et de contrôles, et rognant peu à peu les larges marges jusque là consenties entre les principes et les pratiques réelles. 

 Face à la pauvreté dès 1350 apparaissent les signes avant-coureurs d'un changement des mentalités et des pratiques. On commence à parler de « {bons} » et de « {mauvais pauvres} ». Les vagabonds ne sont plus assimilés aux pèlerins mais sont de plus en plus considérés comme des fauteurs de trouble. Les « {bons pauvres} » ou « {pauvres honteux} » ont le droit moral de mendier parce qu'il leur est impossible de travailler, et parce qu'ils restent rattachés à leur cadre villageois, à leur paroisse d'origine : enfants, infirmes, malades, vieillards... Ils ne se soustraient pas au contrôle de leur communauté. Les \emph{mauvais pauvres} sont ceux qui ont force et santé mais qui fuient le travail par paresse ou par goût de l'errance%
% [3] 
\footnote{... ou par refus de conditions de travail par trop inacceptables (mais cela c'est notre point de vue du \siecle{21}, ce n'était pas celui des décideurs d'alors).} 
loin de tous les cadres sociaux, sans aveu. On soupçonne les vagabonds de vivre dans la débauche et de commettre nombre de délits (notamment des vols). On a peur de leur nombre qui favorise la mendicité agressive et qui intimide les personnes sans défense (enfants, jeunes filles, femmes, vieillards). On les accuse de contrefaire maladies ou infirmités, d'enlever des enfants pour exciter la pitié des passants%
%[4]
\footnote{José \fsc{CUBERO}, 1998, p. 70.}% 
, et même de mutiler ces derniers pour obtenir plus d'aumônes%
%[5]
\footnote{Bronislaw \fsc{GEREMEK} fait état de procès tenus dans la région parisienne en 1449 où ont été condamnés des criminels qui avaient successivement enlevé plusieurs enfants à leurs parents, enfants auxquels ils avaient crevé les yeux et coupé bras ou jambes, pour en tirer profit en mendiant. (in \emph{Les marginaux parisiens aux \crmieme{14} et \crmieme{15} siècles}, Paris, 1976).}% 
. 

 Dès le \siecle{14} les hôpitaux refusent de plus en plus souvent les vagabonds%
% [6]
\footnote{José \fsc{CUBERO}, 1998, p. 68.}% 
, tandis que de nombreuses mesures de police tentent de les contrôler et surtout de les chasser. À l'intention des petits délinquants, des vagabonds et autres chômeurs sans ressources avouables on fait des expériences multiples de travaux « forcés », travaux d'utilité publique, ou même galères du roi%
%[7]
\footnote{En 1456 les États du Languedoc prévoient cette peine pour les vagabonds invétérés. \emph{En 1486, Charles~VIII étend cette mesure à l'ensemble du royaume.} La condamnation aux galères, résurgence de la condamnation antique aux mines, \emph{ad metallas}, se substitue alors dans la plupart des cas à la peine de mort, jusque là appliquée largement en l'absence de peines plus adaptées : \emph{Avec la peine des galères... le Moyen Âge renoue avec la notion antique de l'esclavage... Seul le travail rédempteur peut éviter les galères[7] à ces mendiants valides et vagabonds qui menacent la paix}, José \fsc{CUBERO}, p. 78 idem.}% 
. Le fait que ces décisions d'expulsion aient été périodiquement reformulées montre et leur relative inefficacité, et la persistance des représentations qui les sous-tendent.

 Les cités, en expansion, sont dirigées par leurs bourgeois, commerçants, artisans, juristes et autres détenteurs d'offices. Leur expérience personnelle les porte à tenir pour synonymes les vertus familiales et « bourgeoises » : fidélité, économie, sens de l'effort, contrôle de soi et prévoyance. Pour eux un sou est un sou : contrairement aux aristocrates ils ne valorisent ni le panache, ni le faste, ni la prodigalité. Les anathèmes des religieux contre la richesse, traditionnels, ne les impressionnent plus, sauf lorsqu'ils sont à l'article de la mort, et ils sont fiers de leur fortune. Ils ont la tranquille assurance de ceux qui ont réussi. À leurs yeux les autres n'ont qu'à en faire autant, et ils se font forts de le leur enseigner : dans la plus grande partie des sociétés européennes c'est à la suite des réformes protestante et catholique que les écarts seront les plus faibles entre la morale sexuelle et conjugale officielle et les pratiques réelles. Ce sera le moment où tous les laïcs ou presque se marieront et feront des enfants. Ce sera le moment où les taux de naissances illégitimes et de conceptions pré conjugales seront au plus bas de toute l'histoire européenne : de 1650 à 1750, Normandie : 2 à 3~\% d'enfants illégitimes ; bassin parisien : 1~\% ; Languedoc et Bretagne : 1 à 2~\%. Angleterre sous Cromwell : moins de 1~\% ; en 1600 : 3,2~\%. Ces taux impliquent un haut degré de contrôle social, exercé conjointement par les familles, par les autorités civiles et par les autorités religieuses.

 Au moment où les peuples d'Europe sont en train de se cliver entre catholiques et protestants apparaissent simultanément dans tous les grands États européens des mesures très semblables pour contrôler pauvres et vagabonds. Pas un seul instant la marche vers la rationalisation du contrôle des pauvres et l'organisation de leur mise au travail, forcé si nécessaire, n'a été entravée ou modifiée par les guerres de religion, et tous les États concernés connaissent des évolutions à peu de choses près superposables : mêmes représentations, mêmes solutions, mêmes réussites et mêmes impuissances. 

 Le 22 Avril 1532 le Parlement de Paris ordonne une fois de plus que tous ceux qui dans cette ville peuvent travailler et n'ont ni emploi ni revenus avouables seront contraints à entrer dans les ateliers publics qu'il organise pour eux. Ils travailleront enchaînés deux à deux, gardés rigoureusement et employés aux travaux d'utilité publique les plus rudes. On reconnaît là les pratiques des bagnes%
% [8]
\footnote{... décrites par exemple par Philippe \fsc{HENWOOD} dans \emph{Bagnards à Brest} : « l'accouplement » des bagnards enchaînés, deux par deux, p. 40, 41 et 42,~etc.}% 
. \emph{Mais l'Ordonnance royale du 22 avril 1532 est fondamentale en ceci qu'elle ordonne le placement d'office des enfants des vagabonds arrêtés.} L'autorité parentale peut désormais être disqualifiée en l'absence de tout autre délit que le vagabondage. Ce n'était pas la première fois que des essais de ce genre étaient tentés (exemple : Reims, 1454) mais cette fois il s'agit de le faire à Paris, où se trouve la plus grande concentration de vagabonds du royaume (environ un tiers) et l'Ordonnance est signée par le roi. Elle donne aux \emph{bureaux des pauvres}, où siègent des représentants des autorités ecclésiastiques et judiciaires, une part de l'autorité de l'État. Ils exercent une fonction d'autorité sur tous les pauvres, dont ils peuvent et doivent contrôler non seulement l'incapacité de travailler, mais aussi la correction des pratiques conjugales, éducatives et religieuses. Contrôle et assistance sont désormais liés, et les assujettis ont peu de recours judiciaires possibles : ils subissent une justice d'exception. 

 L'Ordonnance royale de 1566 étend l'interdiction de la mendicité à tout le royaume de France et met les pauvres à la charge de leur paroisse d'origine (\emph{domicile de secours} : seul lieu où l'indigent a droit aux secours) ce qui leur interdit de vagabonder. Elle prévoit que les \emph{bureaux des pauvres} et autres \emph{aumônes générales} doivent si nécessaire organiser et financer des ateliers pour donner du travail aux indigents valides. Entre 1550 et 1600, des forces de police spéciales placées sous l'autorité directe des {bureaux des pauvres} (souvent appelées \emph{archers de l'Hôpital}) sont chargées de traquer la mendicité, de poursuivre hors de l'hôpital et d'arrêter les vagabonds, de récupérer les enfants placés par les bureaux des pauvres lorsqu'ils ont fugué de leur lieu de placement, et de faire régner l'ordre dans les hospices et hôpitaux. 

 Au \siecle{17} les expériences réalisées et les réflexions entretenues par les divers acteurs de l'assistance et du contrôle social confluent dans l'idée qu'il convient de regrouper en une seule administration centralisée les hôpitaux et les hospices, et d'y renfermer tous les indigents qui ne peuvent se prendre en charge seuls, en raison de leur immaturité, de leurs infirmités ou maladies, ou bien en raison de leurs comportements%
% [9]
\footnote{Sources principales :
\\Collectif sous la direction de Jean \fsc{IMBERT}, \emph{L'histoire des hôpitaux en France}, 1982.
\\Maurice \fsc{CAPUL}, \emph{Internat et internement sous l'ancien régime, contribution à l'histoire de l'éducation spéciale}, Thèse d'État, 4 tomes, Tomes 1 et 2, \emph{Les enfants placés}, Tome 3 et 4, \emph{La pédagogie des maisons d'assistance}, 1983-1984.
\\Michel \fsc{FOUCAULT}, \emph{Folie et déraison : histoire de la folie à l'âge classique}, 1961.
\\Michel \fsc{FOUCAULT}, \emph{Surveiller et punir, naissance de la prison}, 1975.
\\Bronislaw \fsc{GEREMEK}, \emph{La potence ou la pitié, l'Europe et les pauvres du Moyen Âge à nos jours}, 1987.
\\Jean \fsc{IMBERT}, \emph{Le droit hospitalier de l'ancien régime}, 1993.
\\Jacques \fsc{TENON}, \emph{Mémoires sur les hôpitaux de Paris}, 1788.}% 
. 

 Louis~XIV ordonne en 1656 la création d'un \emph{Hôpital Général} dans toutes les grandes villes du royaume, et le 14 juin 1662 l'établissement d'un hôpital général dans \emph{toutes les villes et gros bourgs}. Les directeurs, nommés à vie, reçoivent des pouvoirs administratifs et de police pour accomplir leurs missions : \emph{tout pouvoir d'autorité, de direction, d'administration, commerce, police, juridiction, corrections et châtiments sur tous les pauvres de Paris, tant en dehors qu'au-dedans de l'hôpital général \emph{[...]} sans que l'appel puisse être reçu des ordonnances qui seront par eux rendues} [...] Les administrateurs de l'hôpital jugent sans appel, à charge pour eux \emph{si lesdits pauvres méritent peine afflictive plus grande que le fouet, de le mettre es mains du juge ordinaire pour à la requête du procureur d'office leur procez estre fait et parfait}. 

 Que le mouvement de création des Hôpitaux généraux se soit poursuivi à la demande des autorités locales, et pas seulement en France, jusqu'à la fin du \siecle{18} montre que cette formule de l'institution fermée et à l'écart du monde correspondait%
% [10] 
\footnote{Maurice \fsc{CAPUL}, idem, T III, p 301.} 
bien aux conceptions de l'époque : partout en Europe on observait à cette période le même mouvement. Les \emph{Poor Laws} anglaises ordonnaient en 1661 ou 1662 l'enfermement des pauvres dans des \emph{Workhouses} qui sont l'exact pendant (en plus dur ?) des hôpitaux généraux. Il en était de même à Berlin,~etc.\tempnote{Commande \texttt{\string\anglais{}\{\}} à créer !}

 Les contemporains essayaient de ne pas avoir personnellement affaire à ces institutions dont le régime n'était pas fait pour être désirable. Par contre ils approuvaient leur utilisation pour mettre à l'écart les indésirables et pour éviter les catastrophes en cas de disette ou de crise de l'emploi. 

 Et pourtant il y avait des listes d'attente pour entrer à l'hôpital et il fallait souvent patienter avant d'y être admis. Une recommandation était ordinairement nécessaire (très souvent celle de son curé). Un certain nombre de personnes, pauvres mais non indigentes, acceptaient même de payer pension pour y entrer, ce qui laisse à penser que même si les conditions de vie y étaient rudes (mais ces personnes-là n'étaient pas astreintes au travail forcé) il y avait encore pire ailleurs. Pour elles l'Hôpital Général fonctionnait comme une maison de retraite (cf. les « petites maisons » dans le cadre de celui de Paris), et assumait une forme de prise en charge qui existait déjà avant sa propre création.

 Quant à ceux des mendiants et vagabonds qui troublaient l'ordre public par leurs débordements, ils ne venaient pas à l'hôpital de leur plein gré et leurs comportements le traduisaient, aussi les employés des hôpitaux généraux ne faisaient-ils aucun effort pour les garder. Au bout d'un siècle d'expériences cela conduira les Intendants du roi à créer à partir de 1768 à l'intention de cette population les \emph{dépôts de mendicité}, dépôts qui seront à l'origine des futures \emph{prisons départementales}%
% [11]
\footnote{Leur histoire est complexe et s'étend sur une bonne part du \siecle{19}. Voir entre autres : \emph{Lieux d'hospitalité : hospices, hôpital, hostellerie}, ouvrage collectif sous la direction d'Alain \fsc{MONTANDON}, P.U. Blaise Pascal, 2001.}% 
.


\section{Les enfants illégitimes}

 Alors que les grossesses légitimes n'avaient pas à être déclarées, à partir de 1556 obligation est faite par Henri~II de déclarer toutes les grossesses illégitimes, sous peine pour les filles non mariées et les femmes veuves depuis plus d'un an qui seraient enceintes d'être accusées d'infanticide si leur enfant décédait avant son baptême%
% [12] 
\footnote{... qui avait valeur officielle de déclaration de naissance puisque les curés avaient reçu peu de temps auparavant l'obligation de tenir les \emph{registres de catholicité}, ou registres de baptême, ancêtres directs des registres d'état civil.} 
(crime en principe puni de mort). Dans la déclaration devait figurer le nom du père allégué par la mère, sauf refus de celle-ci. Cette déclaration renforçait la position de la mère face à l'homme qui l'avait engrossée, et celle de son enfant, et permettait les actions en justice. Cette décision royale a été rappelée par Henri~III en 1585, et renforcée par Louis~XIV. C'est ainsi qu'en 1708 ce dernier ordonnait encore aux curés de la rappeler en chaire tous les trois mois. 

 Si elle l'a été si souvent, c'est qu'elle n'a jamais été observée de manière rigoureuse. Il semble même que la majorité des grossesses illégitimes n'aient jamais été déclarées. En dépit de la sévérité des peines annoncées les mères préféraient oublier de se signaler à l'attention des autorités lorsqu'elles pensaient pouvoir mieux défendre leurs intérêts et leur réputation (et ceux de leur enfant) par un arrangement discret avec le géniteur (ex. : mariage, pension alimentaire, octroi d'une dot,~etc.) ou par un abandon discret. Combien parmi les veuves et filles dont l'enfant est décédé sans baptême ont-elles effectivement subi les peines prévues ? Il ne semble pas que les autorités aient poursuivi ce genre d'infraction avec beaucoup d'énergie : le plus souvent les tribunaux accordaient de larges circonstances atténuantes aux « coupables » déférées devant elles%
% [13]
\footnote{Frédéric \fsc{Chauvaud}, Jacques-Guy \fsc{Petit}, Jean-Jacques \fsc{Yvorel}, \emph{Histoire de la justice de la Révolution à nos jours}, Presses universitaires de Rennes, 2007.}% 
. 

 Tout enfant, même illégitime, avait le droit d'exiger de ses auteurs des « aliments » c'est-à-dire des moyens de vivre. Un vieil adage juridique, toujours cité, disait en effet que \emph{qui fait l'enfant doit le nourrir}. Le représentant naturel de l'enfant né hors mariage est sa mère, et \emph{protéger celle-ci était aussi protéger l'enfant}. Les actions de la mère%
% [14] 
\footnote{Nommée « fille-mère », et n'ayant droit qu'au titre de « mademoiselle » jusqu'au milieu du \siecle{20}. Ce n'est pas un enfant qui pouvait faire d'elle une femme, une « dame », mais un époux en règle.} 
contre le géniteur qui l'avait délaissée étaient encouragées et soutenues, notamment par les hôpitaux, qui en cas d'abandon de l'enfant devaient en assumer seuls la charge. Elle pouvait entreprendre une \emph{actio provisionis} : demande de provisions pour frais de grossesse ou d'accouchement. Si plusieurs hommes avaient partagé à la même période son intimité ils pouvaient être solidairement responsables de l'enfant. Elle pouvait aussi tenter une \emph{actio susceptionnis partus} ou \emph{actio captionis} : action qui demandait de condamner le géniteur à assumer les frais de l'éducation de l'enfant, sur lequel il ne recevait pour autant aucune autorité. 

 L'\emph{actio dotis} prévoyait que le coupable d'un viol épouse la célibataire qu'il avait déflorée, surtout s'il l'avait engrossée. S'il refusait de l'épouser, ce qui était son droit, il devait payer une dot à la mère et financer l'entretien de l'enfant. Il en était de même si le géniteur était déjà engagé ailleurs (mariage, vœux religieux, ordination sacerdotale). Quel que soit son statut (célibataire, marié, clerc, moine, noble, roturier ou serf) il était et demeurait responsable de la vie de l'enfant et devait donc le nourrir. Même si le géniteur n'était pas père légal il restait \emph{nutritor}.

 Par contre les enfants adultérins étaient toujours traités comme des enfants abandonnés, qui n'avaient ni père ni mère, ni \emph{nutritor}. Ils n'avaient aucun droit vis-à-vis de leurs deux géniteurs, dans la famille desquels ils n'entraient pas et auxquels ils ne pouvaient pas réclamer des aliments%
% [15]
\footnote{Ceci dit la loi n'interdisait pas à leurs auteurs de prendre librement l'initiative de pourvoir à leur éducation.}% 
. Ils étaient exclus de toute possibilité de légitimation, même par mariage, puisque leurs géniteurs ne pourraient pas se marier, même après la mort de l'époux qui faisait obstacle à leur mariage. 

 Étaient encore plus rigoureusement exclus de toute légitimation les enfants nés d'une relation incestueuse.


\section{Protection des nouveaux-nés abandonnés}

 Les grandes villes ont toujours enregistré des taux de naissances illégitimes plus élevés que les petites et les campagnes : les domestiques, femmes et hommes, y étaient nombreux, presque toujours contraints au célibat par leur emploi et par leur pauvreté, donc condamnés à abandonner les enfants nés de leur activité sexuelle (surtout dans les cas où le géniteur était l'employeur ou un membre de sa famille). C'est dans les villes que se réfugiaient aussi toutes celles qui voulaient accoucher clandestinement, et les filles chassées par leur famille ou par leur patron à cause de leur grossesse.

 Vers 1635 l'attention de Vincent de Paul (1581-1660), aumônier général des galères et spécialiste de l'assistance, possédant l'oreille du roi Louis~XIII, a été attirée par le chapitre de Notre-Dame de Paris et par les \emph{dames de l'Hôtel-Dieu}%
% [16] 
\footnote{... c'est-à-dire les religieuses qui assumaient le fonctionnement de l'hôtel-Dieu, voisin immédiat de Notre Dame et de la Couche.} 
sur la situation « effroyable » des enfants de \emph{La Couche} (maison où vivaient et mouraient les enfants abandonnés dans l'hôpital), ce qui l'a conduit à fonder en 1638 l'\emph{œuvre des enfants trouvés}. Après quelques tâtonnements il a repris les recettes éprouvées, celles qui avaient toujours marché dans le passé, même s'il l'a fait à l'échelle d'une grande capitale et avec beaucoup de détermination. Ce qu'il a apporté de véritablement nouveau, c'est qu'il a affirmé haut et fort que même s'ils étaient (peut-être) de naissance illégitime les enfants trouvés avaient le même droit de vivre que les autres enfants. Il a refusé la situation d'infanticide déguisé qui était celle des nouveaux-nés abandonnés et il a agi pour qu'ils bénéficient \emph{au même titre que les autres} enfants des soins et du lait d'une nourrice. C'est pourquoi il a créé une institution capable de mettre en présence \emph{rapidement} nourrices et nourrissons et mis au point un service de nourrices rurales efficace avec une surveillance effective%
%[17]
\footnote{Voici comment en 1788, un siècle et demi après, \fsc{TENON} raconte dans ses Mémoires l'histoire des enfants abandonnés (p. 89) : \emph{Dès l'an 1180, à l'Hôpital du Saint-Esprit à Montpellier, on avait ouvert des secours pour les enfans exposés. Les Hôpitaux des Enfans-Trouvés à Paris sont plus modernes : ils datent de 1638 : on les doit au zèle éclairé et infatigable de S. Vincent de Paul. Il faut se transporter à cette époque pour juger du mérite de leur institution.}
\emph{ En 1638, une Dame veuve, charitable, se chargeoit officieusement des enfans exposés : elle demeuroit près Saint-Landry ; sa maison fut nommée Maison de la Couche, comme on nomme aujourd'hui celle des Enfans-Trouvés, près Notre-Dame.}
\emph{La tâche qu'elle avait entreprise, excéda ses facultés ; ses servantes, fatiguées des soins qu'elles donnaient aux enfans, en firent un commerce scandaleux : elles les vendoient à des mendiantes, qui s'en servoient, afin d'exciter la charité du public ; des nourrices, dont les enfans étoient morts, en achetoient, s'en faisoient teter ; plusieurs d'entre'elles leur donnoient un lait corrompu ; on en prenoit pour en supposer dans les familles : ils ne coûtoient que vingt sols. Dès que ces désordres furent connus, on cessa de recourir à un hospice si dangereux : les enfans déposés furent transportés près Saint-Victor ; les dons de quelques personnes vertueuses ne suffisoient pas à leur subsistance ; le nombre de ces enfans devenu trop grand, on tira au sort ceux qui seroient élevés : les autres étoient abandonnés.}
\emph{Dans ces circonstances, St. Vincent de Paul, en 1640, convoqua une assemblée de Dames, distinguées par leur naissance, leur piété : il en obtint des secours. Le choix du sort des enfans à élever, fut aboli, la vie conservée à tous. Louis~XIII entra dans ces vues charitables : il accorda le château de Bicêtre pour les retirer ; on se persuada que la vivacité de l'air s'opposait à leur conservation : ils furent ramenés dans le fauxbourg Saint-Lazare, où ils demeurèrent sous les yeux de Mlle de Marillac, veuve Le Gras, jusqu'en 1670, époque de leur translation dans la Maison de la Couche}.}% 
.

 Ceci dit la majorité de ses contemporains n'a été que fort peu ébranlée dans ses certitudes par son exemple et ses arguments : Maître \fsc{Ducros}, cité plus haut, qui écrivait en 1659, soit plus de vingt ans après la fondation de l'œuvre des enfants trouvés, n'avait rien entendu. Jusqu'au milieu du \siecle{18} (au moins) ceux qui mettaient à l'écart les enfants illégitimes ou supposés tels, et qui réservaient le meilleur des ressources de l'assistance aux nouveaux-nés légitimes pensaient faire pour le mieux. 


\section{Les « enfants de l'hôpital »}

 En ce qui concerne les enfants les plus jeunes la croyance en la vertu éducatrice et rééducatrice de l'internat est à cette époque à son apogée. Les décideurs n'ont pas encore compris l'importance des relations interpersonnelle (corps à corps et cœur à cœur) dans la construction d'une personnalité d'enfant. Ils n'ont pas plus compris combien est déterminante, pour l'investissement de quelque enseignement que ce soit, la différence entre le placement en internat scolaire choisi par les parents, et l'internement d'office ordonné contre leur gré par une instance administrative ou judiciaire. Ils n'ont pas compris non plus la différence qui existe entre la prise en charge des enfants sans famille (orphelins ou abandonnés) qui ni les uns ni les autres n'ont plus de parents, et celle des enfants qui connaissent leurs parents mais à qui on prétend interdire de s'identifier à eux. 


\subsection{Enfants trouvés et abandonnés}

Les enfants abandonnés pris en charge par les institutions d'assistance pouvaient avoir été déposés dans un lieu public ou dans le « tour » d'un hôpital, ou confiés par leur père ou leur mère, ou volontairement « perdus » par eux dans un lieu inconnu%
%[18]
\footnote{L'histoire du \emph{Petit Poucet}, racontée par \fsc{Perrault} dans les \emph{Contes de ma mère l'oye} (1697) a parfois correspondu à une réalité, pour des enfants très jeunes incapables de dire de quelle commune ils venaient ni comment s'appelaient leurs parents.}% 
. Beaucoup de nouveaux-nés étaient abandonnés par leurs mères dans les services d'accouchement des hôpitaux, que seules fréquentaient les indigentes qui ne pouvaient accoucher à leur propre domicile ni chez une sage-femme. D'autres tout-petits n'étaient pas abandonnés à proprement parler. Il s'agissait par exemple d'enfants dont les pères ou/et mères étaient incarcérés dans les « \emph{lieux de force} » (dont la prison pour femmes de \emph{La Force} qui faisait partie de l'hôpital de la Salpêtrière) pour vagabondage, prostitution ou autres actes de délinquance, et qui ne pouvaient donc pour un temps s'occuper d'eux. Dès que l'incarcération durait un temps significatif (un an ?) la restauration des droits parentaux devenait impossible. 

 À part ce cas les enfants abandonnés pouvaient être repris par leurs parents. Il fallait évidemment que leur abandon n'ait pas été anonyme pour que ce retour soit possible. En fait ces \emph{retours en famille}étaient rares, les causes de l'abandon, et d'abord la misère, persistant dans la plupart des cas.
 De nombreux enfants entraient à l'Hôpital bien après leur petite enfance : « \emph{... dans la généralité de Lyon, le plus grand nombre d'enfants présentés aux hôpitaux par leurs parents ont une dizaine d'années…} » À cet âge la plupart des enfants « de famille » travaillaient déjà. Ceux qui étaient confiés à l'hôpital étaient donc souvent ceux qui étaient jugés inaptes au travail. Certains d'entre eux se présentaient d'eux-mêmes à l'hôpital. 

 Au-dessous de 4 à 5 ans les enfants de l'Hôpital sont placés en nourrice. Une fois finie la petite enfance, le placement en institution est préféré. Les administrateurs croient que leurs Hôpitaux offrent des possibilités d'éducation nettement supérieures à une famille nourricière, pour des raisons variées, dont la modestie du niveau culturel des nourrices et de leur maris, qui sont le plus souvent paysans ou ouvriers agricoles, et parce que l'hôpital fournit une scolarité qu'on ne trouve pas à la campagne. Ils estiment aussi que les possibilités de trouver un emploi sont plus grandes en ville. Peut-être ne se sentent-ils pas non plus le droit de déraciner pour toujours des jeunes citadins en les laissant vivre à la campagne, surtout s'ils ont de la parenté dans la ville ? Mais il faut aussi tenir compte du fait que le prix de journée de l'hôpital est à l'époque nettement inférieur au salaire d'une nourrice.

 Tous les enfants de 6 ans et plus, non placés chez un maître artisan ou un nourricier, vivent dans les murs de l'hôpital. Même quand ils ont une famille, les enfants placés en sont plus ou moins radicalement coupés, \emph{même quand leurs parents sont placés dans le même établissement}. Les clôtures internes de l'hôpital sont aussi hautes que son mur d'enceinte%
% [19]
\footnote{Il n'est pour en être persuadé que de visiter la chapelle de l'Hôpital de La Salpêtrière.} 
. Pour nombre d'enfants cette coupure est définitive. 

 En dépit d'un souci éducatif certain%
% [20] 
\footnote{... manifesté à Paris par 5 heures 30 d'enseignement par jour, durant six jours par semaine, ce qui n'a rien à envier aux écoles primaires d'aujourd'hui... mais aussi un nombre d'élèves très élevé pour un seul maitre.} 
l'encadrement humain des jeunes placés est extrêmement réduit (d'où la modestie du prix de journée), ce qui contraint les relations entre les jeunes et les adultes à être formelles, distantes et souvent impersonnelles%
%[21]
\footnote{Selon l'expression de Maurice \fsc{CAPUL} : \emph{Pour les pauvres, les moyens de la pédagogie étaient pauvres}.}% 
. Contrairement aux jeunes « de famille » inscrits par leurs parents dans les collèges contemporains, il ne s'agit pas d'intégrer ces jeunes à la « grande » culture ni de leur donner les moyens de penser plus ou moins librement : il s'agit seulement, comme dans les petites écoles, de leur donner les rudiments de la lecture et de l'écriture, et d'en faire de bons pauvres.

\subsection{« Correctionnaires »}

Les mineurs « correctionnaires » sont les jeunes qu'il faut « corriger », ceux dont les comportements font problème, c'est-à-dire les délinquants, rebelles et opposants : mineurs condamnés par décision de justice, faux saulniers de moins de 14 ans, vagabonds, mendiants, prostitué(e)s, « enfants de bohême ». Les enfants au dessus de 6 ans sont soumis aux mêmes règles de droit que les adultes. Dès l'âge de 8 ou 10 ans la peine de mort peut leur être appliquée si une « malignité » exceptionnelle justifie de les exclure du bénéfice de l'excuse de minorité. Les jeunes délinquants sont ordinairement condamnés à un temps d'incarcération déterminé : de quelques mois à 20 ans et plus. Mais ils peuvent aussi être enfermés pour une durée indéterminée : aussi longtemps que l'administration estimera qu'ils ne seront pas suffisamment amendés, jusqu'à leurs 25 ans et plus. Les jeunes garçons condamnés aux galères pour des délits commis sans l'excuse de minorité ne peuvent y être envoyés avant leurs 15 ou 16 ans. Ils attendent donc à l'hôpital d'avoir atteint l'âge d'aller au bagne, soumis au régime des autres correctionnaires, mais le temps qu'ils passent à l'hôpital ne compte pas comme temps d'exécution de la peine

\subsection{« Religionnaires »}

À partir de la \emph{Révocation de l'Édit de Nantes} (1685) ce terme désigne les enfants des protestants rebelles à la conversion au catholicisme qu'on exige d'eux%
%[22]
\footnote{L'Angleterre avait précédé la France dans la persécution des dissidents religieux et leur exclusion de toutes les charges et fonctions officielles. C'était l'application stricte du principe \emph{cujus regio, cujus religio}, « {un roi, une foi, une loi} ». Il faudra attendre le \siecle{18} pour que la tolérance apparaisse comme une vertu et non comme une faiblesse.}% 
. La légitimité des mariages des protestants n'est plus reconnue, ce qui fait de leurs enfants des bâtards incapables d'hériter. Ils se voient retirer leurs droits parentaux. Pour cette raison dès l'âge de sept ans leurs enfants leur sont enlevés. 

 À partir de cette date il est demandé aux hôpitaux généraux d'enfermer et rééduquer les membres de la « \emph{religion prétendue réformée} » (RPR) si aucune autre solution n'est possible. Les enfants de ceux qui ne peuvent payer sont placés en hôpital général, avec les correctionnaires. Les autres sont placés aux frais de leurs parents dans une section de correction d'un collège (catholique comme tous les collèges du royaume à partir de la Révocation), avec les enfants indisciplinés ou récalcitrants des mêmes milieux sociaux qu'eux. Ils y sont soumis à une pression morale ouverte ou insidieuse, brutale ou habile, pour les pousser à abjurer la religion de leurs parents et à se convertir au catholicisme. Leur sortie de l'hôpital ou du collège dépend en grande partie de leur « conversion ». 

 Selon Maurice \fsc{CAPUL}, cette politique a été poursuivie activement de 1685 au milieu du \siecle{18}, en dépit du fait qu'elle ne donnait que des résultats insatisfaisants : selon les observateurs du temps elle produisait des adultes peu consistants, qui ne savaient plus à quoi ils croyaient, ou des sceptiques qui ne croyaient plus à rien. D'autre part elle jetait la discorde au sein des familles et la brouille entre les parents et les enfants. Elle va se déliter peu à peu après le milieu du \siecle{18}, mais ce n'est qu'en 1787 que \emph{l'Édit de Versailles} y met un terme en créant un état-civil laïque, qui rend aux enfants protestants leur légitimité, et en prenant officiellement acte de la tolérance dont le culte protestant avait fini par bénéficier à cette date%
% [23] 
\footnote{Depuis l'affaire Calas (condamné par le parlement de Toulouse à être roué, exécuté en 1762) et l'intervention de Voltaire (qui avait entrainé sa réhabilitation en 1765) la répression du protestantisme s'était adoucie : dans l'opinion publique la légitimité avait changé de camp.} 
dans la réalité quotidienne. Les dispositions de cet édit concernent aussi les français de confession juive.



