% 28.02.2015 :
% haut Moyen Âge
% _, --> ,
% ~etc.
% Antiquité
% ~\%


\chapter[Les familles de l'Ancien Régime entre autorités civiles et religieuses]{Les familles de l'Ancien Régime\\entre autorités civiles et religieuses}


\section{La réforme grégorienne}

 Durant le haut Moyen Âge les institutions religieuses étaient si intriquées dans le fonctionnement des sociétés civiles que les fonctions ecclésiastiques étaient souvent pourvues selon les mêmes principes que les « honneurs » civils et sur désignation des autorités temporelles : parfois elles s'achetaient et se vendaient \emph{(simonie)}, ou bien elles étaient dévolues comme des fiefs. Les abbés et évêques étaient engrenés dans le système féodal, prêtant et recevant des hommages. De nombreux biens ecclésiastiques, et les revenus afférents (dîmes,~etc.), étaient inclus dans les patrimoines de petits et grands seigneurs. De nombreuses seigneuries ont été fondées sur les terres des abbayes ou des évêchés, en principe pour les défendre. Les comportements de nombreux prêtres, évêques ou religieux s'écartaient largement des règles internes de l'Église : hommage rendu aux seigneurs, participation active à des combats, communauté de table et de lit avec une concubine ou une épouse, achat de fonctions ecclésiastiques \emph{(Nicolaïsme)}, établissement d'héritiers sur les biens ecclésiastiques. Cela n'encourageait pas les laïcs à suivre les enseignements de l'Église dans toute leurs rigueurs. 

 C'est pour enlever aux laïcs toute autorité sur les institutions religieuses (paroisses, évêchés et monastères), ce qui impliquait notamment de rendre impossible le recours au principe héréditaire dans la dévolution des charges, que la \emph{réforme Grégorienne}%
% [1] 
\footnote{\emph{Nom donné au mouvement animé et dirigé dans la seconde moitié du \siecle{11} par la papauté, particulièrement à l'initiative du pape Grégoire VII. L'objectif proclamé de la réforme grégorienne fut de rétablir la discipline et de corriger les mœurs des clercs afin de mieux encadrer la société laïque et de faire davantage pénétrer dans les esprits et dans les âmes les obligations de vie découlant du dogme chrétien. Pratiquement, il s'agissait de mettre en place un meilleur épiscopat, grâce auquel le recrutement des prêtres et le contrôle de leur activité seraient améliorés, ce qui, finalement, devait être profitable à la santé morale de tous les fidèles. L'entreprise était ainsi comparable, bien que plus vaste, à celle que, depuis un siècle, de nombreux prélats et des abbés clunisiens conduisaient à l'intérieur du monde seigneurial. Elle en différait fondamentalement par une extrême méfiance à l'égard des pouvoirs laïcs, regardés comme responsables des vices des évêques parce qu'ils intervenaient directement dans l'élection de ceux-ci, et surtout par l'édification dans l'Église, lors de la réalisation, d'un système de gouvernement monarchique entre les mains du pape, au point que certains historiens estiment que la réforme servit d'alibi à une grandiose opération visant à transformer les structures ecclésiales et à faire du pontife romain, avec ses cardinaux et ses légats, grâce au pouvoir de dispense et à l'exemption, la seule autorité souveraine.}

\emph{L'accomplissement du programme grégorien donna lieu aussitôt à un grave conflit avec l'Empire, la querelle des Investitures, du fait que Grégoire VII (1073-1085) interdit aux laïcs de choisir et d'investir les évêques et prit des sanctions à l'encontre de l'empereur Henri IV, qui récusait le bien-fondé de ces mesures. Poursuivie sous une autre forme par Urbain II, Gélase II et Calixte II, la Querelle s'acheva par un compromis lors du concordat de Worms (1122). Cependant, pendant ces cinquante années, de grands efforts avaient été faits afin de sensibiliser davantage les hommes au fait religieux et d'inviter les meilleurs à réfléchir et à agir selon les normes chrétiennes. Du coup, et sans doute par l'intermédiaire d'un épiscopat plus digne, un esprit nouveau anima l'Église jusqu'au milieu du \siecle{12} : l'esprit grégorien, façonné par la conviction que tout acte, public ou privé, s'intègre dans un contexte religieux et doit avoir une signification chrétienne.}

\emph{C'est à ses résultats que l'on juge le mieux la réforme et ses objectifs. L'œuvre grégorienne aboutit effectivement à faire disparaître presque totalement la simonie et le nicolaïsme, à mettre en place des évêques plus responsables, qui tentèrent de mieux contrôler le bas clergé et y parvinrent, notamment grâce à de meilleurs moyens juridiques, mais ne recrutèrent pas toujours des prêtres moralement valables ; le progrès était toutefois très net par rapport au siècle précédent. Plus encore, la réforme conduisit au plein pouvoir du pape dans l'Église et à l'exaltation de son autorité dans le monde, au point que l'ambition gagna les divers échelons de la hiérarchie et que papes et prélats se mirent à œuvrer d'abord pour conserver et renforcer leur puissance en maintenant et agrandissant leurs biens et droits temporels, ce qui devait conduire à une grave crise dès la fin du \siecle{12}. Ce goût du pouvoir chez les chefs de l'Église n'empêcha pas l'esprit de la réforme de pénétrer dans les mœurs : la religion cessa peu à peu de n'être qu'une pratique, qu'un culte s'accompagnant de quelques interdits pour devenir davantage une règle de comportement ; la piété se fit plus sensible ; la culture intellectuelle plus profonde. Enfin, de grandes entreprises politiques, telles les croisades, furent en partie suscitées par ce dynamisme, sans lequel on n'expliquerait sans doute pas leur succès.}
Marcel \fsc{PACAUT}, Article \emph{réforme grégorienne}, Encyclopédia Universalis, édition 2011.} 
a été mise en œuvre du \crmieme{11} au \siecle{12}, et d'abord au niveau de l'élection papale. C'était l'aboutissement d'une montée de l'influence du Pape et d'une forte centralisation de l'Église d'Occident. 

 Grâce à un effort continu de plusieurs siècles, et en dépit de fortes résistances, les papes, les évêques et les monastères ont progressivement réussi à accroître assez leur indépendance pour que les rois, l'empereur, et les seigneurs soient contraints de trouver d'autres moyens de récompenser un fidèle serviteur chargé de famille que d'en faire un évêque ou un abbé, ce qu'ils avaient fait à tour de bras depuis la chute de l'Empire romain. Ils ont été contraints d'abandonner leur autorité sur les églises paroissiales qu'ils avaient fondées, de ne plus en nommer les desservants et de ne plus les inclure dans leurs patrimoines. Les autorités ecclésiastiques ont obtenu le contrôle des \emph{investitures}, même si elles ont toujours dû tenir compte de l'avis des autorités civiles. 

 En pratique beaucoup de couvents et d'abbayes resteront encore sous l'influence de la famille du fondateur ou de bienfaiteurs puissants. Du moins les religieux, élus et seuls électeurs, seront-ils canoniquement en règle. Jusqu'à la fin du Moyen Âge ces formes seront dans l'ensemble respectées et les scandales les plus criants évités\footnote{Ce n'est qu'à partir de la renaissance et jusqu'à la Révolution que les revenus ecclésiastiques attachés aux monastères pourront être (en France) soumis au système de la \emph{commende}, ce qui conduira à de nouvelles situations d'empiètements.}. 
 
 

\section{Monopole de l'Église sur le droit familial}

 L'effort de normalisation des mœurs ne s'est pas borné aux seuls clercs. Au terme d'une longue évolution commencée par Constantin, à partir du \siecle{11} et de la réforme grégorienne, il n'y a plus de droit civil (laïc) du mariage dans une Europe totalement christianisée où les seuls dissidents religieux tolérés (plus ou moins suivant les lieux et les époques) sont les juifs
\footnote{\fsc{DUBY}, \emph{Le chevalier, la femme et le prêtre}, 1981.}% 
. Même s'ils se moulent de plus en plus dans les habits du droit romain le mariage et la famille sont du ressort exclusif du droit de l'Église (droit canon), qui a suffisamment de puissance pour imposer une bonne part de ses principes. 

Le mariage est désormais défini comme un sacrement, ce qui en un sens ne change pas grand chose, puisque aucun des traits qu'on lui attribue n'est nouveau (le mariage ne change pas, c'est la doctrine des sacrements qui se précise) mais cela rend les compromis plus difficiles à trouver. Cela interdit encore plus qu'avant d'y voir seulement un contrat entre deux parties et renforce le caractère indiscutable de son indissolubilité, toujours présent mais toujours contesté depuis l'origine du christianisme. 

 En 1215, le quatrième concile de Latran a décidé de restreindre les interdits au quatrième degré canonique, soit le huitième degré romain (ou moderne), incluant l'interdit des cousins germains, de restreindre les cas de parenté par alliance, et aussi de faciliter largement l'octroi de \emph{dispenses}.

 Comme les tribunaux civils ne s'occupent plus des affaires de mariage, il n'y a plus de divorce%
% [3]
\footnote{... pour les chrétiens mais pas pour les juifs qui avaient toujours droit de recourir aux tribunaux rabbiniques. Avec la Réforme le divorce redeviendra possible chez les protestants, même si en pratique ils feront tout pour le décourager : la stabilité des couples était ressentie comme essentielle pour des raisons multiples qui débordaient largement les seules représentations religieuses.}% 
. Seules restent possibles les actions en nullité, ou les actions en séparation. Le concubinage n'est plus traité comme une union de second ordre, propre aux gens qui n'ont pas d'héritage à transmettre, mais comme un état de fornication durable.

 La paternité est exaltée en liaison avec la paternité divine. Au même moment la maternité est elle aussi très fortement idéalisée, notamment à travers le culte de la Vierge Marie. Au total c'est la famille nucléaire, le couple et ses enfants légitimes, qui est sacralisée à travers le culte de la « sainte famille ». Mais dans le même temps la valorisation des parentalités spirituelles à travers le culte de Saint Joseph affirme la prééminence de la relation éducative sur la reproduction biologique.

 La morale familiale et sexuelle enseignée par l'Église est la morale commune. Si du \crmieme{11} au \siecle{16} des mouvements de contestation religieuse se succèdent, qui culmineront avec la réforme protestante, rares sont ceux qui à cette période mettent vraiment en question la morale familiale et sexuelle enseignée par l'Église. Au contraire les opposants s'appuient sur elle, même les cathares dont les doctrines sont par ailleurs très loin du christianisme, pour critiquer les écarts des clercs avec leurs propres principes. Cette morale a été formulée dès les premiers temps de l'Église, mais les règles de droit qui en découlent ont mis un millénaire à s'imposer comme le droit commun. Ceci dit jamais elles n'ont réussi à le faire partout ni parfaitement. Certaines régions s'y sont conformées avec beaucoup de rigueur tandis que d'autres ont été beaucoup plus tolérantes avec les écarts. Chaque société repose sur un ensemble de logiques sociales, économiques et politiques qui n'ont pas forcément à voir avec une religion quelconque. Le christianisme a certes contribué à façonner les sociétés d'ancien régime, mais il en a été lui-même très fortement influencé. Il lui a été demandé de bénir (et même de sacraliser) leurs mécanismes et leurs logiques, et de les conforter dans leur fonctionnement, et c'est ce qu'il a souvent fait. 
 
 \section{Ecarts avec les règles canoniques}

 Les contraintes et limites imposées à la reproduction par le roi et par l'Église n'ont jamais été acceptées totalement ni par tous. C'était notamment le cas dans la noblesse. Depuis l'Antiquité tardive et durant tout le Moyen Âge elle était tenue, par elle-même et par les autres ordres de la société, pour une \emph{race} supérieure qui transmettait ses vertus par son sang. Cette très antique croyance n'accordait pas d'importance au statut juridique ou religieux de l'union des parents. Elle coexistait sereinement dans les têtes avec le modèle canonique judéo-chrétien. 

 Les membres les plus puissants de la noblesse, et d'abord les rois eux-mêmes, n'ont jamais cessé de pratiquer une polygamie de fait qui leur donnait de nombreux enfants, de second rang certes, mais parfois bien utiles à défaut ou en complément d'enfants légitimes et valeureux. Jusqu'au \siecle{11} les différences faites dans les familles puissantes entre enfants légitimes et bâtards nés du chef de famille étaient faibles (capacité d'hériter, de succéder,~etc.). En effet le sang du père ennoblissait. Cette conception était un héritage des mœurs d'inspiration germanique du haut Moyen Âge. Le nombre des bâtards nobles semble même avoir crû au  \siecle{15}. Par comparaison les bourgeois reconnaissaient beaucoup moins d'enfants illégitimes. De 1400 à 1649 les rois de France ont reconnu 24~\% d'enfants naturels tandis que les grands officiers, employés roturiers de la maison du roi, n'en avouaient que 10,3~\%. 

 Alors que le mariage des grands seigneurs ne répondait habituellement qu'à des critères politiques, les enfants qu'ils avaient conçus avec leurs maîtresses, \emph{enfants de l'amour}, étaient fréquemment \emph{réussis}. S'ils étaient légitimés, ces enfants pouvaient leur permettre des alliances profitables. Or les rois d'Europe, héritiers en cela aussi de l'empereur de Rome, pouvaient légitimer les « bâtards » par \emph{lettres royaux}. Les bâtards des familles aristocratiques ont donc souvent été légitimés par le roi ou par mariage subséquent. Même s'ils ne l'étaient pas, cela n'a pas fait problème pendant longtemps. Souvent ils n'ont été légitimés qu'après la mort de leur père, \latin{ad honores}, c'est-à-dire pour accéder aux honneurs qu'ils détenaient, c'est-à-dire pour leur succéder dans les emplois d'intérêt public, les charges qu'ils exerçaient. Par contre aucune reine, princesse du sang ou femme d'officier n'a pu légitimer d'enfant naturel autrement que par un mariage subséquent : c'est l'homme qui ennoblissait, c'est également lui qui légitimait. 

 Les cas d'illégitimité susceptibles de bénéficier d'une légitimation par mariage subséquent avaient été étendus par les rois au-delà des critères de Constantin, de façon à inclure les enfants nés d'une relation passagère, et ceux conçus dans le cadre d'un enlèvement suivi d'un viol (enfants dont le géniteur n'était pas forcément le futur mari). Par ce biais la fiction retrouvait une place dans la filiation : l'adoption par l'époux de la mère redevenait possible. 

 Et pourtant les déclarations officielles de l'Église stigmatisaient tous les enfants illégitimes. Le Con\-ci\-le de Bourges (1031) confirmait les jugements des conciles antérieurs (« semence maudite »). Et l'Église continuait d'écarter de la prêtrise les fils de prêtres, sauf dispense (à vrai dire facilement accordée). 

\section{« Bâtards »}

 Ceux que les gens du Moyen Âge nommaient « bâtards » étaient d'abord les enfants abandonnés dont on ne connaissait pas les parents. À partir du \siecle{11}, le mot « bâtard » devient un terme de mépris, une injure. Dès le \siecle{12}, avec la renaissance du droit romain, le bâtard n'appartient plus à aucune famille même dans les pays de droit coutumier%
% [6] 
\footnote{En gros les pays situés au nord de la Loire, opposés aux pays de droit (romain) écrit, situés au sud de la Loire.} 
: pas même celle de sa mère. Être un « bâtard » était une tare, et semble avoir été de plus en plus pénalisant du Moyen Âge au \siecle{18} : est-ce vraiment pour des raisons religieuses ? ou parce que la société reposait sur l'alliance des familles, alliance que protégeait la stigmatisation des naissances irrégulière ?

 Dans une société où l'on n'était fils ou fille de quelqu'un que si l'on était né de deux parents légitimement mariés, un jeune de naissance illégitime ou né de parents inconnus portait une tare indélébile. Il était considéré comme fils de personne, hors parenté, même s'il vivait au foyer de l'un de ses deux parents (la mère en général). Un enfant non légitime ne pouvait ni succéder à un membre de sa parentèle dans un office, ni en hériter, sauf si aucun autre héritier légitime n'y trouvait ombrage. Personne ne se portait caution pour lui. Sans père il ne pouvait pas apporter sa contrepartie dans un système d'alliance. Il était condamné à une position marginale, du moins par rapport à celle de ses éventuels demi-frères et sœurs. En contrepartie de ces exclusions il n'était pas non plus contraint de se porter caution pour un parent. Il n'avait aucune autorisation parentale à demander pour convoler : ses géniteurs comme ses tuteurs ne pouvaient pas le lui interdire.

 L'Hôpital du Saint Esprit de Paris était initialement un hospice destiné à toutes les personnes démunies. À la fin du Moyen Âge il était devenu l'orphelinat de Paris. Vers le milieu du \siecle{15} le roi lui a demandé de prendre en charge les enfants abandonnés de Paris. Tout roi qu'il fût ses demandes réitérées ont été récusées par les maîtres de l'un des hôpitaux de sa propre ville : \emph{en faisant prévaloir les statuts de fondation et la bonne réputation dont jouissent les enfants qu'il \emph{[l'hôpital]} entretient et éduque déjà. Si tous les orphelins d'origine inconnue lui étaient conduits, les gens de métier qui viennent chercher un apprenti, ou les jeunes compagnons qui y prennent femme ne seraient plus assurés de la légitimité, et partant de la moralité de l'adolescent}%
% [7]
\footnote{\fsc{SAUNIER}, \emph{Le « pauvre malade » dans le cadre hospitalier médiéval, France du Nord}, vers 1300-1500, 1993, p. 53.}% 
.

 Réellement convaincu par ces arguments, ou bien de guerre lasse, le roi a confirmé les maîtres de l'Hôpital du Saint-Esprit dans l'idée que leurs statuts et eux-mêmes se faisaient de leur mission : en 1445 il a donc accepté qu'on n'y admet que les \emph{orphelins et orphelines nés en loyal mariage} et à qui on ne peut reprocher \emph{la tache de bâtardise}, car, selon un autre argument fourni par les maîtres de l'hôpital, \enquote{[...] \emph{ſy on y admettoit des baſtards, il ſeroit à craindre que la division ne ſe miſt bientôt dans cette maiſon par les reproches continuels que les enfants légitimes feroient aux baſtards}%
% [8]
\footnote{Lettres patentes de Charles VII du 7 août 1445, A.A.P. Saint-Esprit, L, II, p. 32 ; cité par \fsc{SAUNIER}, id. p. 212.}% 
}.

 À cette époque les maîtres du Saint-Esprit faisaient remettre tous les enfants exposés qu'on leur présentait aux paroisses sur le territoire desquelles on les avait trouvés%
% [9]
\footnote{\fsc{SAUNIER}, id. p. 212. On pourrait se demander où et comment les paroisses pouvaient prendre en charge ces enfants ? Elles n'avaient ni hôpital ni hospice pour cela. Cela n'est jamais précisé parce que c'était probablement évident : elles faisaient ce que faisait n'importe quelle artisane encombrée d'un enfant. Elles payaient une nourrice pour s'en occuper, en attendant qu'il soit assez grand pour être mis \emph{en condition}, au travail, au même âge que les plus pauvres de ses contemporains légitimes, c'est-à-dire dès l'âge de raison, bien avant ses dix ans.}%
, alors qu'ils acceptaient de prendre en charge les adolescents légitimes (orphelins pauvres) qui sortaient convalescents de l'Hôtel-Dieu. 

 Même les léproseries excluaient les bâtards \enquote{\emph{parce que les gardes des maladreries diſaient que les bâtards n'avaient pas de lignage, ni n'étaient à hériter de nul droit par quoy ils ne ſe pouvaient aider de leur maiſon, pas plus qu'un étranger qui ſerait venu d'Eſpagne}%
% [10] 
\footnote{\fsc{SAUNIER}, id. p. 213.}%
}. Mais ces mêmes léproseries admettaient sans réserves les lépreux sans ressources s'ils étaient de naissance légitime : l'indigence leur faisait encore moins peur que l'illégitimité.

 Jusqu'à la fin de l'ancien régime les « bâtards » étaient exclus de nombreux métiers. En règle générale les corporations les refusaient, de la même façon et pour les mêmes raisons que la prêtrise leur était interdite. Maître Jacques \fsc{Ducros}, avocat au Parlement de Bordeaux, et premier Consul d'Agen en 1659, écrit dans ses \href{http://www.babordnum.fr/files/original/859d36685f2d7b2f871c648ea08bd103.pdf}{\emph{Réflexions singulières sur l'ancienne coutume d'Agen}}  :
%
\begin{displayquote}

{[...] \emph{les batards n'ont pas le bonheur de paſſer pour des domeſtiques%
% [11] 
\footnote{Ici « domestiques » signifie « appartenant à une maison », pas forcément comme salarié au service du maître. Cela inclut aussi tous les enfants et parents vivant dans la maison.} 
ny d'auoir rien en propre dans les maiſons. Ils ſont les productions du vice \& les enfans d'iniquité. Les peres les forment dans les tenebres \emph{[et]} les meres en cachent la conception. A meſme qu'ils sont nez , ces infames producteurs les deſauoüent. Les enfans legitimes cherchent le iour \& la lumiere, les illegitimes la nuit \& l'obſcurité. A proprement parler ce ſont des excremens, deſquels à meſme que la nature les chaſſe \& les pouſſe dehors, on couure d'ordure \& de ſaleté : ils n'ont ny nom ny race ny famille , c'eſt pourquoy ils ne peuuent eſtre admis au nombre des proches de ſang de conſanguinité ny d'allience}}%
%[12]
\footnote{\fsc{CAPUL}, Thèse, tome II, p. 111.%
\label{notecapul111}}%
.

\end{displayquote}


 \fsc{CAPUL} rapporte que lors des États généraux de 1614, le Tiers-état d'Agenais demande au roi : \enquote{\emph{que toutes lettres de légitimation ſeront deſnyees a tous enfens nez d'inceſte, d'adultère ou filz de prebſtres, et qu'on n'y aura aucun eſgard, ſoit pour ſucceſion, dignites, offices, bennefices (eccléſiaſtiques) et tous autres droitz}}%
% [13]
\footnote{%\fsc{CAPUL}, Thèse, tome II, p. 111.}%
Voir note \ref{notecapul111}.}%
. Les places désirables étaient trop peu nombreuses pour se montrer généreux. Ceci dit leur démarche conforte l'idée que les interdits qui pesaient sur les « fruits du péché » pouvaient assez aisément être tournés avec de l'argent et de l'entregent, mais cela ne concernait que les rares enfants illégitimes qui étaient investis par des parents puissants ou fortunés. Ainsi Erasme, (1469-1536), « prince des humanistes », âme de la « république des lettres » de son temps, était-il fils de prêtre. Fils d'un père cultivé il reçut une instruction soignée dans les écoles monastiques de son temps et entra en 1688 chez les chanoines de Saint-Augustin, où il fut ordonné prêtre en 1492. S'il ne fit pas une brillante carrière dans les allées du pouvoir temporel, comme évêque ou cardinal, c'est parce qu'il refusa les offres qu'on lui en fit au profit de la recherche intellectuelle et théologique, où il réussit il est vrai de manière exceptionnelle. Sa bâtardise et le statut ecclésiastique de son père ne semblent avoir fait problème à personne.

 Les bourgeois prospères qui représentaient leurs concitoyens de l'Agenais, et qui exprimaient probablement l'opinion publique de leur époque, ne s'identifiaient en aucune façon aux enfants nés des unions sexuelles illicites, pourtant innocents des actes de leurs pères et mères. Ces députés tenaient fermement à ce qu'aucun passe-droit ne puisse désavantager leurs propres fils dans la course aux honneurs. C'était la défense la plus intransigeante de la morale conjugale qui servait leurs intérêts, puisqu'elle leur permettait d'écarter une partie des concurrents nobles ou bourgeois de leurs propres enfants. Ils ne toléraient pas que les « bâtards » soient confondus avec les enfants légitimes, et surtout pas avec les orphelins. Presque tous se félicitaient quand les distinctions étaient rigoureusement défendues.
 
 

\section{Conflits de juridictions}

 Le droit romain n'a jamais totalement disparu dans les pays de droit écrit, au Sud de la Loire ou en Italie, mais à partir de sa redécouverte au \siecle{12} il a connu une nouvelle faveur en tant que modèle et outil de réflexion. La Renaissance a vu le triomphe du droit tel que les empereurs chrétiens l'ont mis en forme%
% [4]
\footnote{Justinien~I\ier{} (483 - 565) ou Justinien le Grand, empereur de Byzance de 527 à 565, essaya de restaurer l'unité de l'empire romain. Il a ordonné et dirigé une compilation du droit romain, le \latin{Corpus iuris civilis}, qui est l'une des bases du droit civil de divers pays européens.}% 
. Si bien qu'au bout de plus d'un millénaire, c'étaient encore les choix de Constantin et de ses successeurs immédiats qui modelaient en profondeur les mœurs familiales européennes : celles-ci n'ont jamais été aussi conformes à ses décrets qu'aux \crmieme{16} et \siecle{17}. À partir de la Renaissance et jusqu'au \siecle{20} les femmes \emph{mariées} ont été pratiquement réduites à un statut de mineures. Par rapport au Moyen-Age le retour en faveur du droit romain a appesanti l'autorité des pères sur les enfants, et contribué à enlever aux femmes, et surtout aux épouses, une part significative des libertés économiques et juridiques que le Moyen Âge leur avait reconnues.

 Du \crmieme{13} au \siecle{18} les autorités civiles reprennent peu à peu une grande partie du terrain qu'elles avaient concédé aux autorités religieuses au fil du haut Moyen Âge. Le \emph{Concordat de Bologne} (1516) accorde au roi de France le droit de nommer les titulaires des principaux bénéfices (évêques et abbés et abbesses), en dépit la règle qui depuis l'Antiquité voulait qu'ils soient désignés (élus) par leur communauté. 

 À partir du \siecle{14} un petit nombre de curés ont commencé de tenir des \emph{registres de catholicité} où ils enregistraient les baptêmes (autant dire les naissances dans un monde où tous sont baptisés) et parfois aussi les décès et les mariages. L'intérêt de ces registres était de faire foi dans les procès éventuels, même si manquaient les témoins capables de répondre à des questions concernant par exemple l'âge des personnes, ou leurs liens de parenté,~etc. En raison de cet intérêt quelques évêques ont ordonné à tous leurs curés d'en faire autant. \emph{L'Ordonnance de Villers-Cotterêts} (1539) a généralisé à tous les curés du royaume de France l'obligation d'enregistrer par écrit tous les baptêmes. \emph{L'Ordonnance de Blois} (1579) la complète en ordonnant que soient également notés sur ces registres tous les mariages et tous les décès, ce qui permettait de lutter contre les bigames. À cela s'ajoute l'obligation légale d'une publication des bans préalable au mariage, préconisée depuis longtemps par les conciles, mais appliquée de manière irrégulière, afin que ceux qui connaîtraient un empêchement au mariage projeté puissent le déclarer en temps utile. Par ces diverses initiatives et par d'autres les autorités civiles ont repris pied dans le domaine matrimonial. Le contrôle des unions importait en effet au moins autant aux rois et aux parents qu'à l'Église, et les mariages avaient un effet déterminant sur le bon fonctionnement de la société civile, sur la paix des familles et sur l'organisation économique. Les juges royaux ont cherché et trouvé, ou reçu du roi, des moyens de contester certaines des décisions des juges ecclésiastiques, si bien qu'on a pu observer un retour progressif du contentieux des mariages devant les tribunaux civils. 

 Le conflit le plus rude entre les autorités civiles et les autorités religieuses a porté sur la place à donner à l'autorité des parents sur les unions. Selon l'Église catholique les conjoints s'unissaient irrévocablement l'un à l'autre par leur « oui », devant le prêtre qui n'était qu'un témoin représentant l'Eglise, un témoin privilégié à partir du moment où il tenait les registres d'état civil (le curé de la paroisse de l'un ou l'autre des époux sauf dispense). La position traditionnelle de l'Église était que l'autorisation parentale n'était pas nécessaire pour la validité du mariage, même si elle déconseillait aux jeunes gens de s'en passer et si elle ne contestait pas aux parents le droit de déshériter les contrevenants. Les autorités civiles et les familles pensaient au contraire qu'un mariage, alliance entre deux familles et contrat civil, ne pouvait pas être valide, quel que soit l'âge des conjoints, sans l'accord formel de leurs parents. Du point de vue de ces derniers (et des clercs eux-mêmes lorsqu'ils ne parlaient pas en tant que représentants de l'Église, mais en tant que membres d'une famille particulière) l'absence de cet accord était une preuve du manque de bon sens, de l'immaturité des deux jeunes concernés, ou de la perversité de l'un d'eux (cf. la réaction du chanoine Fulbert, oncle d'Héloïse, face au mariage secret, et néanmoins valide, de sa nièce avec Abélard). 

 Malgré la pression du roi de France les évêques rassemblés en \emph{concile à Trente} ont refusé de modifier leur doctrine traditionnelle. Il a donc promulgué \emph{l'édit de 1556} qui ne contestait pas la validité religieuse des mariages célébrés sans l'accord des parents \emph{(mariages clandestins)} mais qui les déclarait civilement illégaux. Il confirmait le droit traditionnellement accordé aux parents de déshériter les enfants qui se rendaient coupables de tels mariages. Il décidait surtout que l'instigateur ou l'instigatrice d'un tel mariage (c'est-à-dire celui qui avait à y gagner, en principe le plus pauvre) pouvait être condamné à la peine de mort pour \emph{rapt}, ce qui réglait radicalement la question de l'indissolubilité du mariage. Cet édit a été en vigueur jusqu'à la Révolution, et semble avoir réglé le problème à la satisfaction des pères et des mères de familles. 
 
 \section{Les protestant, le mariage et le sexe}

 La position des protestants était très proche de celle du roi de France. Pour eux le mariage n'était pas un sacrement mais seulement un contrat entre deux personnes, par nature révocable, et du ressort des seules autorités civiles. Selon Luther (Traubüchlein, 1529) : "\emph{il faut laisser à chaque ville et à chaque pays ses us et coutumes tels qu'ils sont pratiqués}",Tous ces usages "(...)\emph{ c'est aux princes et aux magistrats qu'il appartient de les établir et de les régler}". Le roi et les pères et mères étaient d'accord sur l'idée que le choix d'un conjoint était trop important pour être laissé à la discrétion des futurs époux.
 
 
 Mais le fait de dénier au mariage le poids d'un sacrement et de n'y voir qu'un contrat ne lui enlevait pas une certaine forme d'indissolubilité. Même Henri VIII n'avait pas rompu le lien entre l'Eglise d'Angleterre et Rome pour divorcer, mais pour faire reconnaître par les évêques de son royaume la nullité de son premier mariage. A partir du moment où le mariage était invalide si les parents des futurs conjoints ne lui apportaient pas leur approbation, comme pour tout autre contrat. Il était du devoir des jeunes gens matériellement dépendants de leurs parents de se soumettre à la volonté de ces derniers dans ce domaine-là comme dans tous les autres. Une fois accordés par leurs parents au terme de négociations plus ou moins âpres entre ces derniers et après que le père de la mariée ait remis celle-ci en mains propres à son futur gendre, les époux se devaient de respecter la volonté de leurs auteurs et de rester ensemble en dépit des difficultés éventuelles. C'est pour cela que tout en reconnaissant aux époux le droit de divorcer, les protestants leur imposaient tant de conditions (en Angleterre il y fallait entre autres un acte du parlement) que leurs divorces étaient en réalité difficiles à obtenir et coûteux (800 livres au \siecle{19} en Angleterre) et donc rares : en Angleterre 184 divorces entre 1715 et 1852, pour 9 millions d'habitants environ ; au Massachusetts, état américain bien plus libéral, 143 divorces entre 1692 et 1786 pour \nombre{300000} habitants environ 
 \footnote{Indépendamment des doctrines il serait extrêmement intéressant de comparer la réalité des pratiques chez les catholiques et chez les protestants :pays par pays et dénomination par dénomination : tâche immense !}. 


 