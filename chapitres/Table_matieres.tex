TABLE DES MATIERES
 
 GENERATIONS ET HISTOIRE	1
AO PROLOGUE	5
A1 CITES ANTIQUES INEGALITAIRES ET PATRIARCALES	8
 \emph{Le pater familias romain	14}
 \emph{Le patriarcat	16}
A2 CITES ESCLAVAGISTES	19
 \emph{Un esclave pour quoi faire ?	20}
 Une main-d'œuvre à bon marché	20
 Un corps sans défenses	21
 \emph{La fabrique des esclaves	24}
 La violence	24
 La sanction pénale	25
 Le surendettement	25
 La naissance	26
 L'abandon	27
 La vente par un parent	27
 La vente par soi-même	28
 \emph{Qui peut-on asservir légitimement ?	29}
A3 LES REPRESENTATIONS ANTIQUES	31
 \emph{Des hommes et des dieux	31}
 \emph{La vie bonne	33}
 \emph{Piété et solidarité familiale	34}
 \emph{Morale d'esclaves	36}
 \emph{Vertu virile	37}
 \emph{Pudeur féminine	40}
B1 L'EXCEPTION JUIVE	43
 \emph{Un dieu à part	43}
 \emph{Le problème du mal	45}
 \emph{Un culte spirituel	47}
B2 MŒURS JUIVES	49
 \emph{Le pur et l'impur	49}
 \emph{Morale et société	52}
 \emph{Exaltation de la sexualité conjugale	53}
 \emph{Naissances impures	56}
 \emph{Clergé et familles	58}
B3 LES JUIFS, LE TRAVAIL ET LES ESCLAVES	60
C1 LES LOIS D'AUGUSTE	64
 \emph{Promotion des naissances ingénues	66}
 \emph{Répression de l'adultère féminin	67}
 \emph{Promotion du mariage	69}
 \emph{limitation des affranchissements	70
}
 
 
C2 EVOLUTION DES DROITS PERSONNELS SOUS L'EMPIRE	72
 \emph{Des droits pour les femmes	72}
 \emph{Des droits pour les enfants	72}
 \emph{Des droits pour les esclaves	73}
 \emph{Tous romains, tous égaux ?	75}
D1 LA GENERATION CHEZ LES CHRETIENS	77
 \emph{Indissolubilité du mariage	77}
 \emph{Valorisation du célibat et de la continence	81}
 \emph{Désacralisation de la fécondité, valorisation des enfants	83}
D2 UNE CONTRE-SOCIETE CHRETIENNE	86
 \emph{Laïcs et laïques consacrés	86}
 \emph{Clergé et continence	88}
 \emph{Il n'y a plus ni esclave ni homme libre ?	90}
 \emph{Le service des pauvres	91}
 Veuves	93
 Orphelins	94
 Malades	95
 Énergumènes	95
 Voyageurs, vagabonds, mendiants	97
 Captifs	97
 Morts sans sépulture	98
 Enfants trouvés	98
E1 CONSTANTIN ET LE CHRISTIANISME	102
E2 CONSTANTIN ET LE DROIT DES PERSONNES	107
F1 ENTREE EN SCENE DES BARBARES	115
F2 BAS-EMPIRE ET HAUT-MOYEN-AGE	118
F3 L'ESCLAVAGE DU BAS-EMPIRE ET DU HAUT-MOYEN-AGE	121
F4 A LE CLERGE CHRETIEN	125
F4 B LES RELIGIEUX	129
F5 LE MARIAGE CONSTANTINIEN	132
F6 FAMILLES DE CHAIR	138
 \emph{Mépris de la chair ?	138}
 \emph{Disparition de l'adoption	140}
 \emph{Divorces et remariages	140}
 \emph{Phobie de l'inceste	141}
 \emph{Enfants en trop, enfants « irréguliers »	144}
 \emph{Les éducations	146}
F7 FAMILLES SPIRITUELLES	149
G1 LES FAMILLES DE L'ANCIEN REGIME ENTRE AUTORITES CIVILES ET RELIGIEUSES	154
 \emph{La Réforme Grégorienne	154}
 \emph{Monopole de l'Eglise sur le droit familial	155}
 \emph{Conflits de juridictions	156}
 \emph{Résistances aux règles canoniques du mariage	158}
 \emph{« Bâtards »	159
}
 
G2 LES DEVOIRS DES PERES DE L'ANCIEN REGIME	163
 \emph{Montée en puissance de l'enseignement	164}
 \emph{La Correction paternelle	168}
 \emph{Enfants « adoptifs »	169}
G3 LA POLICE DES FAMILLES	171
 \emph{Organisation d'une police des pauvres	171}
 \emph{Les enfants illégitimes	174}
 \emph{Protection des nouveaux-nés abandonnés	175}
 \emph{Les « enfants de l'hôpital »	176}
 Enfants trouvés et abandonnés	176
 « Correctionnaires »	178
 « Religionnaires »	178
H1 LES LUMIERES CONTRE LES FAMILLES TRADITIONNELLES	181
 \emph{Contestation de l'autonomie de l'Eglise	181}
 \emph{Contrôle des autorités civiles sur les congrégations religieuses	182}
 \emph{Contestation de la puissance des pères	185}
 \emph{Banalisation des abandons	186}
 \emph{Valorisation de l'éducation familiale et maternelle	187}
H2 LA REVOLUTION FRANÇAISE	191
 \emph{Limitation de la puissance paternelle	191}
 \emph{Privatisation des vœux perpétuels et droit au divorce	192}
 \emph{Libération des enfants majeurs	192}
 \emph{« Nul ne peut être parent contre son gré »	192}
 \emph{Démembrement de l'hôpital général	194}
I1 LA FAMILLE DU CODE NAPOLEON	195
 \emph{Suppression du divorce	195}
 \emph{Restauration de l'autorité des pères	196}
 \emph{Interdiction des recherches en paternité	198}
I2 LA POLICE DES FAMILLES DU XIX EME SIECLE	200
 \emph{Les enfants trouvés et abandonnés	200}
 \emph{La prévention des abandons	202}
 \emph{Des clivages idéologiques durables autour des familles	203}
J1 LEGISLATION REPUBLICAINE (1880-1946)	205
J2 SEPARATION DES EGLISES ET DE L'ETAT	209
J3 CONTESTATION DU CODE NAPOLEON	212
 \emph{Critiques théoriques	212}
 \emph{Evolutions du droit.	213}
 \emph{Erosion du droit de correction	214}
J4 L'ETAT, PROVIDENCE DES FAMILLES ?	219
K1 DEMANTELEMENT DEPUIS 1965 DE LA FAMILLE TRADITIONNELLE	222
 \emph{Lois principales	222}
 \emph{Le sens des évolutions	224}
 \emph{Séparation des familles et de l'État ?	227
}
 
K2 VICTOIRE DE LA LIBERTE INDIVIDUELLE ET DU MARIAGE D'AMOUR	229
K3 « LE CORPS DES FEMMES EST A ELLES »	233
K4 INCESTES ET PARADOXES	236
K5 PERPLEXITES EDUCATIVES	238
K6 DESARROIS MASCULINS	241
K7 INERTIE DES PRATIQUES	246
L1 UN ENFANT POUR QUOI ? POUR QUI ?	249
L2 QUI POUR ACCUEILLIR L'ENFANT ?	252
L3 DROIT A L'ENFANT ?	256
L4 PROGRES OU REGRESSIONS ?	259
L5 CONCLUSION	265
M, BIBLIOGRAPHIE	269
TABLE DES MATIERES	273