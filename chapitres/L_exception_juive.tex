% Le 9 mars 2015 :
% ~etc.
% ~\%
% Antiquité
% Moyen Âge
% _, --> ,
% Grec, Romain


\chapter{L'exception juive}

 Les Romains nommaient \emph{juifs} (originaires de \emph{Judée}) ceux que les Grecs nommaient \emph{hébreux}%
%[1]
\footnote{Ils seront présentés ici tels qu'ils étaient après leur intégration à l'Empire romain, mais alors que le Temple de Jérusalem était encore en fonction (avant l'an 70 de notre ère).
Sources : André \fsc{CHOURAQUI}, \emph{La vie quotidienne des hommes de la Bible}, 1978. A.~\fsc{COHEN}, \emph{Le Talmud, exposé synthétique du Talmud et de l'enseignement des rabbins sur l'éthique, la religion les coutumes et la jurisprudence}, 1980. Alain \fsc{STEINSALTZ}, \emph{Introduction au Talmud}, 1976, Paris 2002. Collectif, \emph{Aux origines du christianisme}, 2000. André-Marie \fsc{GERARD}, \emph{Dictionnaire de la Bible}, 1989. Alan \fsc{UNTERMAN}, \emph{Dictionnaire du judaïsme, Histoires, mythes, traditions}, 1997.}%
.

 Aux alentours du début de notre ère, et jusqu'aux deux « guerres juives » (\hbox{66-70} et \hbox{132-135} après J.-C.), il y avait un fort noyau de juifs en Judée et en Galilée : ils représentaient semble-t-il la majorité des habitants (2 millions ?) de ces territoires. Mais la plupart d'entre eux étaient dispersés sur toute la terre alors connue : c'était la \latin{diaspora} (la « dispersion »). Toutes les grandes villes antiques abritaient des communautés juives, parfois plus nombreuses que la population de Jérusalem (Alexandrie, Antioche...)%
% [2]
\footnote{Il semble qu'au moins un habitant sur dix de l'Empire romain était juif : de 6 à 8 millions pour un nombre total compris entre 50 et 60 millions ? Dans la partie orientale de la méditerranée c'était un habitant sur cinq ?}%
. Et il existait aussi des communautés juives en dehors du monde romain (Mésopotamie, Arabie, Éthiopie,~etc.).

 C'est souvent comme esclaves que les juifs avaient voyagé vers la terre de leur exil (ou de leur déportation). Cela avait commencé dès 722 avant J.-C., avec la fin du Royaume du nord, celui des 10 tribus, écrasé par les Assyriens. Cela avait continué en 586 avant notre ère, avec la destruction du Royaume de Juda et du premier Temple de Jérusalem, et la déportation des élites vers Babylone. Cela s'était poursuivi avec la guerre des Macchabées (deuxième siècle avant notre ère). Ces tribulations culmineront avec la grande révolte de l'an 70 de notre ère, avec la destruction du Temple d'Hérode et la déportation des survivants de Jérusalem. En l'an 135 la diaspora deviendra l'unique lieu de vie des juifs, qu'ils soient descendants d'immigrés ou de déportés, nés en Judée ou en Galilée, ou d'autochtones convertis à leur religion. 

%UN DIEU A PART
\section{Un dieu unique}

 Les juifs étaient un cas particulier dans le paysage religieux de l'empire romain. Ils se désignaient comme un « peuple saint », le peuple de leur Dieu : YHWH. Par respect pour sa transcendance ils évitaient de prononcer son nom, et le désignaient par un autre mot (le Nom, le Seigneur, le Saint...). La particularité des juifs n'était pas qu'ils affirmaient avoir un dieu bien à eux, ni qu'ils prétendaient que leur dieu était le plus puissant ou le meilleur de tous. Cela, c'est ce que pensaient \emph{toutes} les cités de l'Antiquité. Ce que croyaient les juifs de la fin de la République romaine c'est que leur dieu était \emph{le seul} dieu existant, et donc \emph{le dieu de tous} les hommes. Depuis le retour de leur exil à Babylone les autres dieux n'étaient plus pour eux que des images, des simulacres, des \emph{idoles} impuissantes, derrière lesquelles se cachaient des démons malfaisants, qui cherchaient à tromper les hommes. C'est pour ce motif que les juifs étaient qualifiés par leurs contemporains de « {sans dieux} », d'athées. 
 
 
 Ils avaient réussi à faire admettre aux romains qu'en termes de religion civique ils se borneraient à faire des sacrifices et des prières à leur propre dieu, pour les cités, pour l'État. Les autorités romaines leur avaient accordé ce droit quand ils avaient fait appel à elles pendant les guerres des Macchabées, parce que la Judée et de la Galilée valaient bien une entorse à leurs propres croyances et coutumes. En contrepartie la diaspora versait un impôt spécial au Temple de Jérusalem pour payer les sacrifices offerts pour l'État romain. C'est pourquoi l'administration romaine protégeait comme une œuvre d'intérêt public la collecte et le transport de cet impôt, le \latin{fiscus judaïcus}. Le fait que les Romains acceptaient le refus des juifs de sacrifier aux dieux de Rome ne veut pas dire qu'ils appréciaient leur religion et leurs manières de vivre. Au contraire ils les trouvaient détestables et ne se privaient pas de le dire. Si les citoyens romains qui étaient juifs étaient exemptés du service militaire (c'étaient les seuls), cette exemption n'était pas un privilège mais une exclusion. Leur mode de vie était trop contraire à celui des militaires romains et interdisait leur intégration dans des légions à la vie rythmée par les cérémonies de la religion civique, et leurs règles de pureté interdisaient leur intégration dans une chambrée de \emph{gentils} (de membres des \latin{gens}, des peuples autres que le peuple juif). 

 

 Ils se définissaient comme le peuple de la \emph{Tora}. Au sens large celle-ci est une collection de 24 écrits (\emph{ta biblia} en Grec : les livres) de plus ou moins grande ancienneté, et au sens strict ce sont les cinq livres du \emph{Pentateuque}, attribués à Moïse\footnote{Au sens large la \emph{Tora} recouvre pour l'essentiel ce que les chrétiens appellent \emph{Ancien Testament} (« ancien témoignage »). La plupart de ces textes sont très largement antérieurs à notre ère, mais la date de mise par écrit de chacun d'eux fait encore l'objet de débats entre spécialistes : du huitième siècle avant notre ère au dernier ? La \emph{Septante} est la version grecque de la Tora, traduite pour les juifs de la diaspora grecque aux troisième et second siècles avant notre ère.}%
. La déportation de l'élite du peuple à Babylone à partir de 586 avant notre ère, et l'interruption forcée des sacrifices à Jérusalem, avaient provoqué une révolution dans leur pensée et dans leurs pratiques religieuses. S'ils s'étaient crus autorisés à se construire des temples à Babylone, comme le faisaient les autres déportés qu'ils y côtoyaient, peut-être auraient-ils oublié qu'ils étaient des exilés, peut-être auraient-ils fini par s'y sentir chez eux, satisfaits d'être l'une des minorités influentes d'un riche et puissant Empire ? Mais à leurs yeux les sacrifices à leur dieu ne pouvaient être valides qu'à Jérusalem. Ils avaient refusé l'interprétation de leur situation qui pour tous leurs contemporains aurait été considérée comme évidente. Ils avaient refusé de voir dans la ruine du Temple de Jérusalem et leur propre déportation la défaite et l'assujettissement de leur dieu national par les dieux de Babylone. Contre le sens commun antique ils avaient choisi de comprendre ces tribulations comme une punition à eux infligée en raison de leurs infidélités : le Seigneur s'était servi des étrangers pour punir Israël, ce qui prouvait une fois de plus qu'il était plus puissant que tous leurs dieux. Les exilés avaient choisi de croire que leur retour vers leur Dieu se traduirait par la fin de leur déportation : le temps de l'exil avait été un temps de ferveur et d'approfondissement de leur foi. 

%LE PROBLEME DU MAL 
\section{Une approche originale du problème du mal}

 Si le dieu des juifs était le seul vrai dieu, il n'était que trop visible qu'eux-mêmes étaient loin d'être puissants : jamais riches, rarement souverains, et plusieurs fois écrasés par leurs puissants voisins et déportés. Ceux qui suivaient les prescriptions de la loi de Moïse (les \emph{dix commandements}) n'étaient pas plus épargnés que les autres par les soucis et les malheurs. Des méchants prospéraient avec insolence, tandis que des justes étaient accablés. Comment l'idée d'un dieu unique, créateur, tout-puissant et bon pouvait-elle tenir face à tout ce mal ? Si vraiment il savait et pouvait tout, pourquoi permettait-il l'injustice ? Pouvait-il exiger des faibles hommes une justice impeccable et ne pas la pratiquer Lui-même ? Quel était l'intérêt de passer alliance avec Lui ? 

 Pour répondre à cette question, peu à peu l'hypothèse s'est faite jour chez les prophètes que si le juste (ou Israël) est accablé de maux et de souffrances, c'est peut-être que le Seigneur le met à l'épreuve, pour tester sa foi et sa détermination. 

 D'autre part (avec le \emph{Second Isaïe} surtout) est apparue l'idée que le juste peut souffrir non seulement du fait des pécheurs, mais \emph{à leur place} :

\begin{displayquote}
{\emph{Le rôle du serviteur de YHWH dans le procès du salut est souligné ici pour la première fois : on sait quel destin eut cette idée dans la littérature religieuse des Juifs et des Chrétiens. La souffrance a une valeur expiatrice, rédemptrice et salvatrice. Telle est la nouvelle réponse que le prophète apporte au problème du mal. Israël, serviteur de YHWH, souffre non pas à cause de ses fautes mais pour expier celles des peuples qui le haïssent, le persécutent et le pillent. Sa souffrance est féconde puisqu'elle fera prendre conscience aux nations des crimes qui les souillent. La grandeur du serviteur se situe ainsi dans son rejet, sa déchéance et sa souffrance ; il accepte de les subir afin d'amener la rédemption du monde.}%
% [4]
\footnote{\fsc{CHOURAQUI}, 1978, p. 281.}%
}
\end{displayquote}

Le Juste souffrirait donc pour racheter la faute des autres, pour payer pour eux la dîme%
%[5]
\footnote{Rédimer, d'où vient le mot \emph{rédemption}.}%
. C'était une interprétation nouvelle des très anciens rites hébraïques du « \emph{bouc émissaire} » et de la « \emph{victime propitiatoire} » : c'était justement parce que la victime, bouc émissaire y compris, était innocente, que sa souffrance était en mesure de racheter les fautes des méchants. 

 L'idée que ceux qui souffraient payaient pour les autres faisait appel à une économie de la douleur et des mérites humainement incompréhensible sinon scandaleuse : comment cela pouvait-il se faire ? Selon quelle comptabilité sinistre ou obscène ? Se pouvait-il que le Seigneur jouisse (sadiquement) du spectacle de la souffrance humaine ? Ces questions sont au cœur du \emph{Livre de Job}. La réponse de l'auteur de celui-ci est que Dieu est si transcendant, si au-dessus de toute comparaison avec l'homme, qu'il n'est pas possible à celui-ci de comprendre ses intentions. Il est donc vain de lui demander des comptes et de lui faire des reproches. Malgré l'épreuve du mal injustement, absurdement subi, il convient au contraire de lui faire confiance et de continuer de croire en sa bonté. 

 On ne pouvait commencer à trouver humainement cohérente une telle doctrine, \emph{ce qui n'est pas la même chose que la comprendre}, que si l'on croyait que le Seigneur répondrait par une rétribution \latin{post mortem} à l'injustice subie et à toutes les peines endurées par l'innocent. La plupart des courants du judaïsme des derniers siècles avant J.-C. en étaient venus à croire en la survie des morts, avec un jugement individuel portant sur la totalité des actes de chaque individu, et en conséquence un paradis éternel de jouissance pour les justes, notamment ceux qui mouraient \emph{à cause de} leur fidélité au Seigneur (cf. les \emph{martyrs} du \emph{Deuxième livre des Macchabées}) et une éternité de tourments pour les méchants%
% [6].
\footnote{La notion d'une survie éternelle des morts ou de leur principe vital (leur « âme »), avec un jugement rétrospectif de la vie du mort, avait été élaborée par les égyptiens bien avant que les juifs n'y adhèrent. Durant les derniers siècles avant notre ère c'était une notion presque universelle dans le monde antique, mais la qualité de vie promise par la plupart des « enfers », ou lieux de survie des « ombres » des morts, laissait encore beaucoup à désirer. Pour Kant la morale implique logiquement un jugement et une sanction \latin{post mortem}. Les hommes sont ainsi faits et ne peuvent penser autrement. À ses yeux cette exigence de la pensée n'est d'ailleurs ni la preuve qu'il existe une divinité bonne et juste ni la preuve qu'elle n'existe pas.}%
.

 Si la solution biblique au problème du mal n'avait rien de raisonnable elle entrainait néanmoins des conséquences positives pour ceux qui souffraient :
\begin{enumerate}
 % a)
\item ni une santé prospère, ni une vie amoureuse et conjugale réussie, ni des affaires florissantes, ni une nombreuse progéniture, ni le pouvoir conquis, ni les victoires sur l'ennemi ne pouvaient plus être considérés comme des preuves de vertu ;
% b) 
\item si ni la maladie, le malheur, la souffrance physique, les deuils, les persécutions, l'exil ni la mort n'étaient la sanction des fautes du sujet qui les subissait ou de ses ascendants, il n'était plus nécessaire de croire qu'il les avait mérités ;
% c) 
\item il n'était pas nécessaire de dénier l'existence de ces maux, pour protéger la perfection ou la bonté du Seigneur, ni de minimiser leur poids : le mal restait un mal.
\end{enumerate}

 De cela il découlait que :
\begin{enumerate}
%  a)
\item celui qui souffrait n'avait pas à croire que le Seigneur lui en voulait ni qu'il le punissait ;
% b)
\item rien n'autorisait les autres à le mépriser ;
% c) 
\item il était même possible qu'il soit en train de payer à leur place leurs dettes morales ;
% d)
\item leur devoir le plus élémentaire était donc de prendre leur part de son fardeau en lui apportant aide et assistance;
% e)
\item sinon c'est eux qui seraient un jour dans le malheur après leur mort, tandis qu'il serait glorifié comme Job, s'il endurait ses maux sans perdre confiance dans la justice du Seigneur%
% [7]
\footnote{Cf. dans l'évangile de Luc la parabole de Lazare et du « mauvais riche » : Luc 16, 19-31.}%
.
\end{enumerate}

 Dans cette perspective le premier devoir de chacun c'était de tout mettre en œuvre pour soulager celui qui était dans la peine. La preuve de la sincérité de l'attachement au Seigneur, c'était le service des pauvres, des malades, des malheureux de toutes sortes :

\begin{verse}
 « \emph{Quel est le jeûne que je veux ? \\
 C'est briser les chaînes du crime, \\
 Délier le harnais et le joug, \\
 Renvoyer libre l'opprimé \\
 Et déposer le joug. \\
 Partager ton pain avec l'affamé, \\
 Ramener chez toi le pauvre des rues, \\
 Couvrir celui que tu vois nu, \\
 C'est ta propre chair que tu ne fuis plus.} » \\
 (Isaïe, chap. 58, 6--7)
\end{verse}



 La justice était inséparable de la religion. À côté des fonctions cultuelles ou de direction spirituelle la fonction de prêtre ou de sage (\emph{rabbi}) impliquait aussi de dire le droit et d'arbitrer les conflits soumis par les fidèles. Même en diaspora les tribunaux des synagogues jugeaient les affaires que leurs membres voulaient bien leur soumettre. Cela posait problème si l'un des plaignants refusait le jugement. Les communautés de la diaspora n'avaient en effet aucun intérêt à ce que les autorités civiles se mêlent de leurs affaires. L'impossibilité où elles se trouvaient d'exercer une contrainte physique sur leurs membres devait donc être compensée par l'autorité morale de leurs juges. Il fallait que leur équité soit indiscutable. Lorsqu'il s'agissait de mesurer la gravité d'un acte l'intention du sujet était déterminante. Chacun ne répondait que pour lui-même. Il n'y avait pas de responsabilité héréditaire ou collective. Les châtiments s'appliquaient à la seule personne des criminels et à leurs biens. Parmi les délits on trouvait l'inceste, le meurtre, le vol, mais aussi la profanation du sabbat, les jurons contre le Seigneur, l'idolâtrie, la sorcellerie, l'adultère... Le meurtre était la faute la plus grave. Les sacrifices humains étaient prohibés depuis Abraham. Tout ce qui s'en rapprochait de près ou de loin l'était aussi. Les spectacles de gladiateurs étaient interdits à un double titre : d'une part comme meurtres, d'autre part comme sacrifices idolâtres, puisque leur origine se situait dans le cadre du culte des ancêtres. Même la chasse et les spectacles sanglants où étaient abattus des animaux étaient prohibés : en effet ils étaient impurs puisqu'on y versait le sang. 

 Maltraiter autrui était une \emph{faute contre YHWH}, et cette faute était aussi grave que de ne pas rendre un culte à celui-ci ou de se prosterner devant les idoles. La Tora revenait sans cesse sur le devoir d'aider les pauvres, et d'abord les veuves et les orphelins. Il était pieux et méritoire d'entretenir les orphelins (en Grec \emph{orphanos} pouvait englober les enfants abandonnés sans parents. En était-il de même quand c'étaient des juifs qui parlaient ?) de les recueillir et de les élever, de doter les orphelines et de les marier. C'était le prototype de l'œuvre vraiment bonne (\fsc{COHEN}, 1980, p. 225). Ce faisant il ne s'agissait pas de les adopter, mais de les élever comme des \latin{alumnii}, comme des enfants choisis, comme des enfants nourriciers ou spirituels. 

 Les pauvres avaient droit à ce dont ils avaient besoin pour vivre. L'aumône (\emph{tzedaka} = justice) rétablissait l'équité, puisque selon la Tora les richesses des riches ne leur avaient été confiées par Dieu qu'en gérance et non en pleine propriété (la punition de celui qui s'en dispensait était laissée à la discrétion du Seigneur puisque ce n'était qu'une obligation morale). Certaines dîmes spéciales étaient affectées aux pauvres, ce qui impliquait une administration collective de l'assistance (caisses de secours,~etc.). Il était interdit de saisir ce qui était nécessaire aux débiteurs pour vivre. 

 Quand aux étrangers, résidents ou de passage, la Tora prescrivait de les traiter comme avec équité et sans discrimination : les ancêtres des juifs étaient eux aussi des étrangers quand ils vivaient en Égypte. 

 Comme chez les autres peuples de l'Antiquité l'hospitalité était un devoir pour tous et à l'égard de tous, à charge de revanche. Pas plus qu'ailleurs elle n'était illimitée. Il s'agissait d'un droit moral à un hébergement ponctuel (trois nuits, sauf maladie ou blessure). Au-delà le voyageur était invité à pourvoir lui-même à ses besoins en travaillant. 

 En diaspora la judéité se superposait à la citoyenneté locale. C'était une citoyenneté comme une autre, puisqu'elle avait des effets au regard de la loi civile, même romaine, et il est évident qu'il y avait une forte ressemblance entre la notion de mamzer (cf. plus loin) et celle d'infâme, avec son caractère transmissible par contact relationnel et par hérédité,~etc. Mais celui qui le désirait pouvait entrer dans le peuple juif, par la circoncision, ou par le mariage avec un juif, de même qu'il était possible d'en sortir, à Rome, en sacrifiant aux dieux, ce qui ne donnait pas pour autant la citoyenneté romaine, mais si on la possédait déjà on pouvait rejoindre les autres citoyens non juifs sans les contraintes des règles de pureté, alors qu'il était presque impossible de devenir citoyen d'une cité grecque autrement que par la naissance.
 
\section{Un Dieu jaloux}

 À Babylone les exilés avaient procédé à une relecture de leurs traditions, en vue d'adapter leurs prescriptions à de nouvelles conditions de vie au milieu de peuples étrangers. Le point clé de leur révolution cultuelle et culturelle, c'est qu'ils avaient mis la pratique des bonnes œuvres à égalité avec les sacrifices du Temple. D'autre part, pour remplacer ces derniers et toutes les fêtes grâce auxquels les autres peuplent renouvelaient leur communion, ils avaient institué l'obligation de l'étude personnelle et collective des textes sacrés. C'est à ce moment-là qu'aurait commencé le processus de mise par écrit des livres de la \emph{Tora}. Cette bibliothèque devait en quelque sorte remplacer le temple et le pays perdus. À l'encontre des religions contemporaines, la pratique religieuse des simples fidèles incluait désormais l'étude et la réflexion. 

 Pour un homme l'étude de la Tora était à la fois un devoir et une prière. Les parents avaient pour premier devoir d'initier leurs fils à la Tora et de tenir fils et filles à l'écart des séductions du monde païen pour en faire des adultes fidèles d'Israël. L'enseignement devait veiller à ne laisser à l'écart ni les orphelins ni les indigents. Dès l'âge de six ou sept ans on apprenait à lire et écrire dans le texte de la Tora%
% [8]
\footnote{Chez les Grecs et les Romains non plus personne n'aurait imaginé à cette époque-là un enseignement « primaire » non imbibé de religion : les textes des premiers exercices scolaires grecs et romains étaient les vies des divers dieux.}%
. L'enseignement de la langue grecque était accepté mais la littérature et surtout la philosophie grecques étaient récusées. 

 Les synagogues étaient les instruments les plus visibles du nouveau culte. Chacune d'elles était à la fois école primaire, maison d'étude, maison de prière, centre communautaire, lieu d'assemblée, restaurant et lieu de réception%
% [9]
\footnote{De la même façon les temples grecs et romains louaient des salles fermées ou des salles à manger en plein air pour ceux qui voulaient recevoir plus de personnes qu'ils ne pouvaient loger dignement chez eux.}%
, hôtellerie pour les coreligionnaires de passage, et tribunal pour les mariages, répudiations, et autres conflits de tous ordres entre coreligionnaires. Aucune autorité centrale ne les créait ni ne les contrôlait. Tout juif adulte (mâle) pouvait en diriger le culte.

 








 Les catégories du pur et de l'impur étaient investies par tous les peuples de l'Antiquité, mais tout particulièrement par les hébreux. Diverses choses rendaient impur, « souillaient », et exigeaient donc un rite et/ou un délai de purification : ne pas respecter le Shabbat ; manger sans avoir fait préalablement les ablutions rituelles ; manger des aliments pour lesquels la dîme (un impôt du dixième de chaque production) n'avait pas été payée ; approcher les non circoncis (impurs par excellence) et surtout manger avec eux, comme eux,~etc. Rendaient impur l'exercice de certains métiers sales ou malodorants (tanneur, vidangeur,~etc.) ; le fait de s'approcher d'un cadavre d'homme ou d'animal ; de faire couler le sang ; de consommer du sang, de consommer des animaux impurs (porcs, poissons sans écailles,~etc.), des animaux abattus de façon irrégulière, ou morts d'accident (non abattus rituellement), ou abattus depuis trop longtemps, ou cuisinés de manière irrégulière,~etc. Étaient impurs tous les écoulement issus des organes génitaux (règles, écoulement séminal spontané ou non, maladies vénériennes,~etc.), et toutes les maladies de peau (« lèpre »,~etc.). L'impureté c'était aussi « connaître » charnellement une femme ou un homme, même son conjoint légitime. Toucher l'autel du Temple sans être dans l'état de pureté convenable souillait l'offrande et invalidait le sacrifice. Toucher à mains nues les objets consacrés au culte du Temple « souillait les mains », toucher à mains nues les rouleaux de la Thora « souillait les mains »,~etc.

 

 Les règles de pureté contenaient de nombreuses contradictions : c'est que la notion de pureté est complexe : à un premier niveau, le plus archaïque sans doute, la pureté n'était comme partout qu'un problème de frontières. Le divin et le démoniaque, le surnaturel faste et le surnaturel néfaste étaient également opposés au monde sans danger, banal et familier des hommes ordinaires. En ce sens-là le contact à mains nues des rouleaux de la Tora était impur, puisqu'ils étaient sacrés. Dans le coït l'homme et la femme participaient de manière directe à l'œuvre du Créateur. C'est pourquoi le coït le plus conforme au {\emph{croissez et multipliez}} de la Genèse, le plus légitime, le plus innocent, le plus sain, pour ne pas dire le plus saint, rendait \emph{impur jusqu'au soir}. Dans ce cadre on pouvait résumer ainsi : pur = inoffensif, profane, normal, quotidien, vivant, animal conforme à la norme de sa classe d'animaux (ex. : poisson avec écailles, ruminant au sabot fendu,~etc.). Impur = sacré, divin, démoniaque, prodige, sang, sperme, coït, naissance, mort, animal anormal, réunissant les caractéristiques anatomiques de plusieurs classes d'animaux, par exemple le porc au sabot fendu mais non ruminant, les poissons sans écailles,~etc. 

 Selon une deuxième perspective la pureté et l'impureté étaient des propriétés des choses et des corps. Dans ce cadre de pensée on pouvait faire les oppositions suivantes : pur = vivant, propre, net, limpide, clair, beau, harmonieux, gracieux, habile, droit, sain, intègre, jeune, neuf, vierge, intact ; impur = mort, sale, trouble, louche, sombre, laid, difforme, gauche, maladroit, malade, lépreux, infirme, mutilé, vieux, usagé, usé, abîmé, cassé, défloré, marqué d'un sceau, d'une marque de propriété.

 Ces deux acceptions du sacré et du profane étaient également présentes dans toutes les religions de l'Antiquité. Même si chacune d'entre celles-ci variait dans le détail de ses classifications elles définissaient toutes de la même façon les impuretés essentielles : le coït, le sang, le sperme, la naissance, la mort,~etc. Mais à un troisième niveau les hébreux enrichissaient la notion de souillure de connotations morales d'une manière plus affirmée que ne le faisaient les autres peuples de l'Antiquité : Pur = bon, bien, vrai, juste, droit, sincère, bienveillant, honnête, intègre, innocent, juste, équitable, désintéressé. Impur = mauvais, mal, faute, faux, pervers, menteur, malveillant, méchant, injuste, inéquitable, intéressé, coupable.



 Dans cet approfondissement les prophètes d'Israël avaient joué un rôle déterminant. En effet ils avaient interprété l'histoire des relations de leur peuple avec le Seigneur comme celle d'une relation amoureuse entre deux personnes. Dans leur bouche leur dieu parlait comme un époux épris, blessé par l'infidélité de son épouse au moins autant que comme un père ou un maître tout-puissant et jaloux. 
Les prophètes suivaient presque toujours le canevas suivant :
\begin{enumerate}
% A)
\item Il n'est ni facile ni drôle d'être le peuple élu,
 \begin{enumerate}
 % 1)
 \item parce que les règles de pureté auxquelles il s'est engagé le séparent des autres, et le rendent infréquentable, alors qu'il voudrait qu'on l'aime, comme tout le monde,
 % 2)
 \item parce que ces règles sont pour lui-même un joug et un carcan, et qu'il lui est impossible de les observer toutes,
 % 3)
 \item parce qu'elles lui interdisent les jouissances faciles et sans dangers des autres peuples,
 % 4°)
 \item parce que ses prétentions à l'exclusivité de l'élection divine le rendent odieux au reste du monde. 
 \end{enumerate}
% B)
\item Voilà pourquoi Israël finit toujours par regarder ailleurs. Il oublie constamment les promesses faites par ses pères et se montre infidèle. 
% C)
\item Mais Le Seigneur est un Dieu jaloux. Il laisse sa colère s'abattre sur son peuple. Il le châtie, c'est-à-dire qu'Il permet à ses ennemis de l'accabler. 
% D)
\item Face à ces maux Israël se repent et revient vers Lui, parce qu'il n'y a pas d'autre dieu.
% E)
\item Il est fidèle et tient ses promesses. Il ne sait pas résister à la prière de son peuple. Il le délivre et disperse ceux qui le tourmentaient.
\end{enumerate}

 À partir du moment où les rapports du Seigneur et de son peuple devenaient une histoire d'amour, les problèmes de pureté et d'impureté ne pouvaient plus être traités de manière rituelle. Ce qui comptait désormais c'était la réponse à la question : \frquote{\emph{est-ce que tu m'aimes ?}}. Cet amour poussait des personnes des deux sexes à faire le vœu de devenir \emph{Nazir}, « consacré ». Ils (elles) rasaient alors leur chevelure et durant un temps plus ou moins long. Ils s'abstenaient de certaines nourritures et boissons comme de toute relation sexuelle. Il a existé pendant au moins un siècle et demi des communautés dont le style de vie était quasi monastique, les Esséniens, et qui fuyaient toute impureté. 

 La véritable impureté c'était désormais l'infidélité et le refus de reconnaître la faute commise. Pour les prophètes la véritable offrande de réparation, la seule qui ne serait jamais refusée, ce n'était plus la bête de choix, c'était un « cœur brisé », c'est-à-dire un repentir sincère. Le paradoxe est que ce nouveau point de vue n'abolissait aucune des exigences rituelles. La propension des amoureux est au contraire d'en rajouter, d'en faire plus que ce que l'usage ne prescrit. Les exigences rituelles devenaient une façon de parler, une façon de prier. 
 
 Afin d'achever de se différencier définitivement des autres peuples auxquels ils étaient mêlés les « sages » avaient voulu que pour chaque activité humaine il y ait une manière juive de procéder. Selon le Talmud%
% [10]
\footnote{\fsc{Cohen}, 1980, p. 227. Le Talmud est la deuxième grande œuvre des juifs, après la Tora. Il a fixé par écrit à partir du deuxième siècle de notre ère la « Loi orale » transmise de sage en sage, de rabbin en rabbin, à côté de la Tora. En ce qui concerne la question de l'existence d'écoles \emph{pour tous} au tout début de notre ère, l'information donnée par le Talmud est pourtant d'autant plus vraisemblable qu'à l'époque où nous nous situons les cités grecques et romaines finançaient elles aussi des institutions scolaires pour leurs jeunes citoyens.}
ils avaient planté autour du peuple la \emph{haie} des prescriptions de la Tora, dressée comme un rempart qui le séparait des autres peuples et le gardait \emph{pur} de toute contamination. La réaction de rejet des observateurs de l'Antiquité devant la « superstition » des juifs s'expliquait en grande partie par la rigidité du cadre dans lequel ces derniers s'étaient corsetés. Ils trouvaient que les juifs étaient infréquentables, et « ennemis du genre humain » : de toutes leurs bizarreries les plus discourtoises étaient leur refus d'assister à tout sacrifice aux dieux des autres et de manger avec aucun incirconcis, en un temps où il n'y avait pas de vraie cérémonie publique sans sacrifice aux dieux et pas de sacrifice sans repas en commun. 

 Et pourtant, malgré la coupure avec le monde ordinaire qu'impliquait le mode d'existence juif, les communautés de la diaspora exerçaient une attraction certaine sur leur environnement, et le nombre des « prosélytes » était relativement important. Beaucoup se satisfaisaient d'être des \emph{craignant Dieu}%
% [11]
\footnote{La \emph{Crainte de Dieu} désignait l'attitude d'adoration respectueuse du Seigneur et la volonté de respecter ses commandements : moins une attitude de peur (encore qu'elle n'en soit pas exempte) que de révérence.}%
, non circoncis%
% [12]
\footnote{La circoncision était douloureuse et non sans risques, et surtout très mal vue chez les Grecs et les Romains, quand elle n'était pas interdite par ces derniers à tous les hommes libres comme toute autre mutilation.}%
. Les plus courageux ou les plus convaincus se faisaient circoncire, ce qui en faisait de nouveaux juifs. Si la judéité découlait normalement de la naissance, elle pouvait aussi être le fruit d'un choix délibéré par amour pour le Seigneur (un amour qui pouvait aller jusqu'à la mort si les autorités civiles exigeaient quelque chose de contraire à la Tora). Comme le fait remarquer Paul \fsc{VEYNE}, c'était radicalement différent des cités contemporaines qu'on ne choisissait jamais par amour pour leurs dieux.

 
 
 
\section{Règles de mariage}

 L'appartenance à Israël était certes inscrite dans la chair des hommes par la circoncision, mais elle l'était d'abord par la filiation. Le \emph{Deutéronome} interdisait d'épouser un étranger ou une étrangère (non juif), ou un eunuque (incapable d'engendrer). Au retour de Babylone (cinquième siècle avant J.-C.), Esdras aurait chassé du peuple toutes les femmes étrangères, tous leurs enfants, tous les hommes qui ne voulaient pas se séparer d'elles, et tous ceux dont les origines familiales étaient discutables ou qui ne pouvaient apporter la preuve du contraire (\emph{Esdras}, 10, 17). Que ce récit soit fondé sur un fait historique ou non, ce dont il parle c'est de la volonté de construire une nation pure, à une période où les cités grecques renforçaient elles aussi le lien entre citoyenneté et descendance légitime. 

La polygynie était permise : à côté des épouses légitimes, celles qui avaient reçu une dot de leur père et dont le mariage avait fait l'objet d'un contrat écrit, on admettait aussi des concubines : femmes libres sans dot ou bien esclaves acquises à prix d'argent ou reçues comme butin de guerre%
% [5]
\footnote{Selon le Talmud l'homme qui voulait avoir plusieurs épouses ou concubines devait obtenir l'accord de sa première épouse, pouvoir entretenir matériellement de façon convenable deux ou plusieurs foyers, et être en mesure de remplir son devoir conjugal comme il convenait avec chacune de ses femmes. Sinon les épouses délaissées étaient en droit de se plaindre à la justice, et d'obtenir le divorce à leur avantage.}%
. Une femme qui n'avait pas donné d'enfant à son mari pouvait lui suggérer de prendre une deuxième épouse ou une concubine : ainsi avait fait Sara, stérile, qui avait offert sa propre servante (esclave) Agar comme concubine à son époux Abraham, afin qu'ils aient tous deux une descendance. Par ailleurs si deux frères demeuraient ensemble et que l'un d'eux  mourait sans laisser d'enfant mâle, le frère survivant avait le devoir d'épouser sa belle-soeur, veuve du défunt (\emph{lévirat}, Dt 25, 5-10), toute autre alliance étant en ce cas interdite à celle-ci. En cas de refus de ce beau-frère, elle pouvait se remarier avec un autre homme une fois constaté publiquement son refus de l'épouser et de "relever la maison de son frère".



Selon les premières pages de la Genèse \emph{Dieu créa l'homme à son image, à l'image de Dieu il le créa, homme et femme il le créa. Dieu les bénit et leur dit : Soyez féconds, multipliez, emplissez la terre et soumettez-la.} (Gn 1, 27-28). \emph{C'est pourquoi l'homme quitte son père et sa mère et s'attache à sa femme, et ils deviennent une seule chair.} (Gn 2, 24). Selon ce texte le mariage créait une parenté aussi proche que celle entre parents et enfants (une seule chair). Les interdits de mariage de la Bible découlaient de cette parenté nouvelle, qui ne privilégiait ni le père ni la mère : ils portaient donc de manière presque symétrique du côté paternel et du côté maternel, par le sang ou par l'alliance. Etaient exclus : père, mère, femme du père, frère, soeur, demi-frère, demi-soeur, époux du frère ou de la soeur, du demi-frère ou de la demi-soeur, fils et filles et leurs conjoints, petits-enfants et leurs conjoints, oncles et tantes paternels et maternels, et leurs conjoints. Les soeurs de l'épouse étaient interdites du vivant de l'épouse.  Les mariages entre cousins étaient autorisés (et parfois préférés), tout comme les mariages entre oncle et nièce. Par contre les tantes étaient interdites aux neveux et il était interdit d'avoir des relations avec une mère et sa fille ou sa petite-fille en même temps\footnote{Le Lévitique, chapitre 18 : \emph{1 Et l'Éternel parla à Moïse en disant :
2 Parle aux fils d'Israël et dis-leur : Je suis l'Éternel, votre Dieu.
3 Vous ne ferez pas comme on fait au pays d'Egypte où vous avez habité, et vous ne ferez pas comme on fait au pays de Canaan où je vous conduis ; vous ne marcherez pas selon leurs statuts ;
4 vous écouterez mes ordonnances et vous observerez mes statuts pour y marcher. Je suis l'Éternel, votre Dieu.
5 Vous observerez mes statuts et mes ordonnances l'homme qui les pratiquera vivra par elles : je suis l'Éternel.
6 Nul de vous ne s'approchera de sa proche parente pour découvrir sa nudité : je suis l'Éternel.
7 Tu ne découvriras point la nudité de ton père et la nudité de ta mère ; c'est ta mère, tu ne découvriras pas sa nudité.
8 Tu ne découvriras point la nudité de la femme de ton père ; c'est la nudité de ton père.
9 Tu ne découvriras point la nudité de ta sœur, fille de ton père ou fille de ta mère ; qu'elle soit née dans la maison ou qu'elle soit née au dehors, tu ne découvriras point leur nudité.
10 Tu ne découvriras point la nudité de la fille de ton fils ou de la fille de ta fille, car c'est ta nudité.
11 Tu ne découvriras pas la nudité de la fille de la femme de ton père, née de ton père ; c'est ta sœur.
12 Tu ne découvriras pas la nudité de la sœur de ton père ; elle est du sang de ton père.
13 Tu ne découvriras pas la nudité de la sœur de ta mère ; elle est du sang de ta mère.
14 Tu ne découvriras pas la nudité du frère de ton père, tu ne t'approcheras point de sa femme ; c'est ta tante.
15 Tu ne découvriras pas la nudité de ta belle-fille ; c'est la femme de ton fils, tu ne découvriras point sa nudité.
16 Tu ne découvriras pas la nudité de la femme de ton frère ; c'est la nudité de ton frère.
17 Tu ne découvriras pas la nudité d'une femme et de sa fille ; tu ne prendras pas la fille de son fils, ni la fille de sa fille pour découvrir leur nudité ; elles sont proches parentes, c'est un crime.
18 Tu ne prendras pas la sœur de ta femme de manière à créer une rivalité, en découvrant la nudité de l'une avec celle de l'autre de son vivant.
19 Tu ne t'approcheras point d'une femme pendant son impureté périodique pour découvrir sa nudité.
20 Tu n'auras point commerce avec la femme de ton prochain pour te souiller avec elle.
21 Tu ne donneras point de tes enfants pour les sacrifier à Moloch et tu ne profaneras pas le nom de ton Dieu. Je suis l'Éternel.
22 Tu ne coucheras point avec un homme comme on couche avec une femme ; c'est une abomination.
23 Tu ne coucheras point avec aucune bête pour te souiller avec elle. La femme ne s'approchera point d'une bête pour se prostituer à elle ; c'est une chose monstrueuse.
24 Ne vous souillez par aucune de ces choses ; car c'est par toutes ces, choses que se sont souillées les nations que je chasse devant vous.
25 Le pays en a été souillé, j'ai puni son iniquité et la terre a vomi ses habitants.
26 Mais vous, vous garderez mes statuts et mes ordonnances, et vous ne commettrez aucune de ces abominations, ni l'indigène, ni l'étranger qui séjourne au milieu de vous.
27 Car toutes ces abominations, les hommes du pays, qui y ont été avant vous, les ont commises, et la terre en a été souillée.
28 Et la terre ne vous vomira pas pour l'avoir souillée, comme elle a vomi la nation qui y a été avant vous.
29 Car tous ceux qui auront commis quelqu'une de ces abominations, ceux qui auront fait cela seront retranchés du milieu de leur peuple.
30 Vous garderez mes observances afin de ne pratiquer aucune des coutumes abominables qui ont été pratiquées avant vous ; vous ne vous souillerez point par elles. Je suis l'Éternel, votre Dieu. }}. 



 
 
 La pureté de l'ascendance était un brevet de valeur religieuse et sociale. Connaître ses ancêtres sur de nombreuses générations était un signe d'excellence. Comme les quartiers de noblesse de l'ancien régime français, ou la « pureté du sang » des siècles classiques d'Espagne et du Portugal, cette pureté-là se capitalisait au fil des générations. Elle permettait de s'allier avec d'autres familles à la pureté aussi préservée, aux alliances aussi bien choisies que les siennes. Elle s'accroissait par la gestion intelligente des alliances, et se perdait si on se laissait aller aux rencontres de hasard. 

 Les mariages mixtes étaient en principe interdits, mais un \emph{gentil} (non-juif) pouvait se convertir au judaïsme, et la Bible en fournissait de nombreux exemples. Comme c'est par la mère que se transmettait l'appartenance au peuple hébreu, celui ou celle qui se convertissait ne pouvait en faire pleinement partie, puisque sa mère n'était pas juive. Il (elle) n'était au sens strict qu'un allié du peuple saint. La conversion ne pouvait complètement annuler le fait qu'une femme soit née non juive, donc « impure ». Seules les femmes juives nées de mère juive pouvaient donner le jour à des enfants à la légitimité incontestable (légitimité religieuse et non pas civile). Ce n'est qu'au niveau de ses enfants que la famille d'un(e) converti(e) serait juive de plein droit. Cela ressemblait à l'intégration d'un affranchi dans le peuple Romain, qui lui non plus n'était jamais pleinement assimilé aux citoyens, au contraire de ses enfants. Cela décourageait les hommes de chercher leurs épouses et leurs concubines à l'extérieur de leur communauté, sauf quand ils ne pouvaient faire autrement.

 Il était interdit d'entretenir un culte familial : cela aurait été faire preuve de paganisme. Le Seigneur seul connaissait le destin des ancêtres \emph{(dans le sein d'Abraham)}, et il n'y avait pas à s'en occuper. Ce n'était donc pas l'existence d'un culte familial qui définissait la famille. Pour cette raison l'adoption d'un étranger à la famille pour en faire un héritier était sans objet et donc interdite. 

 Le \emph{Deutéronome} (Deut. 23, 3-4 ; 24, 4) interdisait enfin d'épouser un \emph{mamzer} (pluriel : \emph{mamzerim}). On appelait \emph{mamzerim} les \emph{impurs de naissance}  c'est-à-dire :%
% 
\footnote{in \emph{histoire de la famille, I}, p. 377-379. Cela ne concerne en principe que les enfants nés de deux parents juifs.}%
 :
\begin{enumerate}
% A) 
\item tous les enfants issus d'unions interdites :
 \begin{enumerate}
 % 1)
 \item les unions incestueuses, qui étaient l'objet d'une réprobation sévère. Les interdits semblent avoir été les mêmes que les interdits romains contemporains ;
 % 2)
 \item les unions adultérines (adultère féminin) ;
 % 3)
 \item le mariage ou le concubinage avec un non juif, légalement nul pour les juifs, mais légitime pour la loi romaine : les enfants étaient juifs si leur mère était juive, non juifs dans le cas contraire ;
 % 4)
 \item l'union d'un juif non mamzer avec un juif mamzer ;
 % 5)
 \item le remariage d'un homme avec une femme dont il avait d'abord divorcé, et qui s'était remariée entre temps avec un autre homme.
\end{enumerate}
% B)
\item les enfants nés de père inconnu et ceux nés de père et mère inconnus (c'est-à-dire tous les enfants trouvés). Les enfants nés d'unions interdites avaient évidemment plus de risques que les autres d'être abandonnés. Par précaution les enfants exposés étaient donc classés dans les mamzerim ;
% C) 
\item les enfants nés d'une femme adultère ou prostituée, dont le père ne pouvait pas être désigné avec certitude ;
% D)
\item les enfants des mamzerim : ce statut se transmettait en effet en principe à toute leur descendance. 
\end{enumerate}

 L'enfant d'une juive non mariée et d'un juif dont on connaissait l'identité, qu'il soit marié ou non, n'était un mamzer que si ses deux géniteurs étaient interdits de mariage. Il n'était donc ni nécessaire ni suffisant de naître en légitime mariage pour échapper au statut de mamzer, mais la reconnaissance par un homme non interdit de mariage avec la mère était toujours indispensable. Les enfants trouvés étaient présumés mamzerim, mais il n'était pas possible de les rejeter sans autre forme de procès puisqu'ils étaient présumés descendants d'Abraham comme les autres, et puisque toute vie humaine était sacrée. Si nécessaire, la communauté prenait donc en charge les enfants exposés, mais elle les mettait au dernier rang.

 Les mamzerim étaient \emph{exclus de l'assemblée du peuple jusqu'à la dixième génération}, ce qui veut entre autre dire qu'ils ne pouvaient épouser ceux qui n'étaient \emph{pas} nés impurs, Aucune famille bien née ne pouvait s'allier aux mamzerim. \emph{C'était même un devoir religieux que de ne pas leur donner un de ses enfants comme époux.} Quand il y avait des failles, des lacunes ou des faux-pas dans la chaîne des générations, la sanction c'était donc de ne plus pouvoir se marier qu'entre familles d'impureté équivalente. Un mamzer ne pouvait se marier qu'avec d'autres mamzerim, ou des païens, impurs par définition, ou des convertis, nés impurs. 

 On note les ressemblances entre le statut des mamzerim juifs et celui des Romains atteints par \emph{l'infamie}, ou des Grecs frappés par \emph{l'atimie}. Il s'agissait là de l'expression d'une vision du monde commune à toutes les sociétés antiques. Même si chacune délimitait la \emph{mauvaise réputation} à sa façon, le noyau des hontes était commun. 




\section{Sacralisation de la sexualité conjugale}

 

 Selon \fsc{CHOURAQUI}, aux yeux des hommes de la Bible%
% [1]
\footnote{A.~\fsc{CHOURAQUI}, \emph{La vie quotidienne des hommes de la bible}, 1978, p.153-155.}%
 :

\begin{displayquote}
{%
\emph{L'activité sexuelle normale et licite est un bien. Elle constitue même l'objet du premier commandement que Dieu donne à l'homme dans la Genèse au terme de la création du monde : « fructifiez, multipliez, remplissez la terre... ». La vie sexuelle n'est d'ailleurs pas dissociée du couple et aucun mot n'existe en hébreu pour la désigner comme telle...}

 \emph{Celle-ci est étalée au grand jour et fait l'objet d'une législation très stricte, très abondante et très détaillée qui prouve moins la vertu du peuple de la Bible que l'importance pour lui de ces problèmes. Les documents, faits divers ou lois, que la Bible nous lègue sur ce thème n'ont sans doute aucun parallèle dans aucune civilisation de l'Antiquité...}

 \emph{La femme mariée est consacrée, sanctifiée, mise à part pour son époux. De ce fait, la copulation provoque une impureté comme tout contact avec le sacré. Après le coït, le couple doit faire ses ablutions et se purifier : il restera impur jusqu'au soir. La loi est la même pour l'homme après toute copulation ou toute perte séminale. L'activité sexuelle, de quelle nature qu'elle soit, introduit l'homme dans l'univers du sacré. Il doit être purifié pour retrouver la plénitude des fonctions profanes. Aussi les mœurs tendent-elles à une ségrégation des sexes.}

 \emph{Toute activité sexuelle est prohibée avec une femme qui a ses règles, et tout contact direct ou indirect avec elle est également interdit. La perte du sang provoque l'impureté de la femme \emph{[...]} Car le sang, c'est la vie, et la perte du sang menstruel, comme les suites d'un accouchement, placent la femme dans la zone redoutable et mystérieuse qui se situe entre la vie et la mort, entre les pôles du pur et de l'immonde, qui définissent les termes majeurs de la dialectique biblique. En fait l'acte sexuel n'est ainsi permis qu'aux époques de fécondité de la femme et interdit quand elle est stérile.}

 \emph{Les interdits sexuels pleuvent dans la législation et les peines sont d'une redoutable sévérité : la mort par lapidation ou par « tranchement du peuple ». À l'opposé de la licence qui règne dans ce domaine dans toute l'Antiquité \emph{[...]} on constate dans la Bible un effort quasi désespéré qui tend à discipliner et à orienter l'activité sexuelle du couple.}

 \emph{L'homosexualité \emph{[...]} est qualifiée « d'abomination » \emph{[...]} Les prophètes et les législateurs hébreux sont sans doute les premiers à la prohiber avec une implacable sévérité.}

 \emph{La bestialité \emph{[...]} est, elle aussi, punie de mort \emph{[...]} la prostitution est condamnée par la loi, mais elle subsiste en fait \emph{[...]}}

 \emph{Les précautions prises pour définir l'activité sexuelle illicite mettent en relief le caractère profondément original de l'amour et de la vie du sexe selon les hébreux. Leur souci majeur a été de provoquer une démythisation, une démystification, une libération et une sacralisation de l'activité sexuelle du couple.}
}%
\end{displayquote}

 La Tora glorifiait l'amour entre l'homme et la femme (cf. \emph{le Cantique des cantiques}) et l'amour des parents pour les enfants. Le célibataire n'était pas considéré comme un homme complet et le célibat définitif n'était accepté qu'en cas d'incapacité complète de procréer. En effet chacun se devait d'avoir une descendance, et la stérilité était un malheur. Les épouses de ceux qui étaient décédés sans enfants devaient leur en donner dans le cadre du \emph{Lévirat}, en se mariant avec un de leurs frères. La répudiation d'une épouse avait souvent ce motif. Si la répudiation de l'épouse était autorisée, elle n'était pourtant pas bien vue ; \emph{Je hais la répudiation, dit le Dieu d'Israël} (Malachie, 2, 16).

 Les pratiques sexuelles qui ne peuvent aboutir à une conception étaient interdites. Les relations homosexuelles masculines étaient considérées comme abominables et en théorie punies de mort quel que soit le statut ou l'âge des partenaires. 

 L'exposition des nouveaux-nés était interdite, sauf en situation d'absolue détresse. De même il était interdit de mettre à mort un enfant quel qu'il soit. C'était la conséquence directe du \frquote{\emph{tu ne tueras point}} du Décalogue. Les avortements étaient considérés jusqu'à un certain point comme des assassinats, sauf risque pour la vie de la femme, préférée en cas de danger mortel à celle du fœtus. Ceci étant dit les juifs adhéraient comme l'ensemble des gens de l'Antiquité à l'idée que le fœtus ne devenait humain qu'au bout d'une certaine durée, avant laquelle il n'était pas animé, ce qui autorisait l'avortement. Sur cette durée les opinions variaient : quarante jour pour les garçons ? Quatre vingt ou quatre vingt dix pour les filles ? 

 Ces prescriptions favorisaient les naissances et l'équilibre du ratio garçons -- filles. Les hébreux en ont-ils eu conscience ? Bien sûr, même si ce n'était pas leur objectif premier. Leurs contemporains étaient parfaitement conscients des effets à long terme de ces comportements natalistes et nous en ont laissé des témoignages%
% [2]
\footnote{Cf. les commentaires de Tacite, dans Aline \fsc{ROUSSELLE}, 2001.}%
.

 

Comme à Rome, les enfants des épouses et des concubines étaient légitimes s'ils étaient reconnus par leur père et si la fidélité de la mère n'était pas mise en doute. La monogamie n'en était pas moins le modèle comme chez les Grecs et les Romains, ainsi que le signifiait l'obligation faite au Grand Prêtre de n'avoir qu'une seule épouse, et aucune concubine%
%[6]
\footnote{Les œuvres les plus tardives de la littérature hébraïque ne mettent en scène que des couples monogames. Aucun rabbin du passé, même dans l'Antiquité, n'est connu pour avoir eu plus d'une femme : à vrai dire les sociétés où la polygamie est autorisée ne peuvent jamais compter beaucoup de foyers polygames, ne serait-ce que parce qu'il est fort coûteux d'entretenir plus d'un foyer, surtout si on élève tous les enfants qui y naissent. À Rome, il était usuel que les grossesses des concubines soient interrompues par un avortement, contrairement aux grossesses des épouses légitimes, du moins tant que celles-ci n'avaient pas donné le jour au nombre d'enfants légitimes souhaitable : l'usage le plus fréquent était donc de n'entretenir qu'un seul foyer.}%
 : la polygamie impliquait donc une moindre perfection ou une moindre pureté.

 La famille juive était aussi patriarcale que les autres familles de l'Antiquité méditerranéenne, et ses règles de fonctionnement ne différaient guère. Le père avait tout pouvoir sur les siens : femme, enfants%
% [3]
\footnote{Selon la Tora et le Talmud la rébellion d'un enfant contre son père serait punie de mort (comme à Rome) \emph{si ce dernier le demandait}, mais il faudrait qu'un jugement en bonne et due forme approuve sa demande.}%
, esclaves. Pour l'héritage les fils aînés étaient privilégiés. La virginité des femmes avant le mariage était attendue. L'épouse surprise en flagrant délit d'adultère était lapidée avec son complice sauf si son mari préférait la répudier. Tout le monde s'attendait à ce qu'il le fasse, comme à Rome, et c'était une cause de répudiation sans appel. Une épouse ne pouvait pas prendre l'initiative de divorcer. Il fallait que son mari lui accorde le droit de le faire, comme à Rome au même moment%
%[7]
\footnote{Selon le Talmud en cas de conflit conjugal insoluble un homme pouvait néanmoins être conduit à divorcer par la pression des membres influents de sa communauté : cela était-il une pratique courante avant notre ère ?}%
. La démence de l'épouse interdisait à l'époux de la répudier (et réciproquement), mais non de prendre une concubine. 

 Aux hommes les femmes de tous les autres hommes étaient interdites, par contre toutes les femmes non mariées leur étaient permises, comme dans les autres sociétés antiques. La prostitution était interdite par la Tora, mais coucher avec une prostituée n'était qu'une faute morale sans gravité et non une infraction légale. Seules étaient strictement interdits les prostituées et prostitués sacrés des temples païens, avec qui coïter équivalait à sacrifier aux idoles%
% [4]
\footnote{La fidélité masculine est néanmoins présentée par la Tora (ex. : Proverbes 5, 15 et 20) ou le Talmud comme le modèle à atteindre[4] : l'infidélité de l'époux favorise et entraîne celle de l'épouse : \emph{lui parmi les fruits mûrs, elle parmi les plantes croissantes}.}%
. 

 Le viol d'une femme libre était puni de mort. Le fautif n'échappait au châtiment que s'il épousait sa victime (pour cela il devait être accepté comme gendre par le père de celle-ci), mais il perdait en ce cas le droit de la répudier. Une femme répudiée puis épousée par un autre homme ne pouvait plus être épousée à nouveau par son premier mari. 





\section{Les prêtres du Temple}

 Il n'y avait qu'un seul temple pour tous les juifs, celui de Jérusalem. Les charges et dignités religieuses étaient héréditaires. Les prêtres et les lévites étaient choisis exclusivement dans la \emph{tribu de Lévi}. Les prêtres descendaient en ligne directe d'Aaron, frère de Moïse. La tribu de Lévi vivait du produit de la Dîme versée par les 11 autres tribus. Les communautés de la Diaspora contribuaient elles aussi par l'impôt spécial \latin{(fiscus judaïticus)} à l'entretien du Temple et de ses desservants. 

 Les prêtres se succédaient au Temple de semaine en semaine selon un tour de service (une semaine toutes les 24 semaines ?). Le reste du temps ils vivaient chez eux, pas toujours à Jérusalem. Durant leur semaine de service ils présidaient aux cérémonies. Sacrificateurs ils abattaient rituellement, ils « immolaient » les bêtes offertes au nom du peuple ou des particuliers. Ils les découpaient et les préparaient comme le rituel le prescrivait. Les lévites étaient chargés des tâches autres que les sacrifices. Ils n'approchaient pas de l'autel et ne le touchaient pas : c'est qu'ils ne descendaient pas d'Aaron en ligne directe, ou bien que leur généalogie présentait des irrégularités, des impuretés.

 De la conception à la mort les prêtres et lévites vivaient dans l'obsession de la pureté rituelle. Pour officier ils devaient être parfaitement sains de corps, sans aucune maladie, sans difformité physique et sans infirmité, et dans la force de l'âge (25 à 50 ans). Ils ne devaient pas avoir la voix embarrassée. Il ne devait pas leur manquer une seule phalange : une mutilation même minime (oreille ou phalange coupée, entorse mal réparée...) leur faisait d'autant plus sûrement perdre leur emploi qu'il y avait plus de candidats que de postes à pourvoir. 

 Les généalogies des membres de la tribu de Lévi, et surtout celles des prêtres, étaient d'autant plus impeccables que s'ils voulaient officier au Temple (et vivre des offrandes) ils ne pouvaient épouser ni une juive divorcée, ni la fille de deux parents convertis, ni une convertie, ni la fille d'un homme non juif et d'une mère juive, ni une veuve refusée par son beau-frère \emph{(halizah)} dans le cadre du \emph{lévirat},~etc. 

 Il leur était interdit de vivre dans l'adultère, eux, et tous ceux et celles qui vivaient sous leur toit. Ils devaient épouser une femme vierge, et surtout qui n'ait pas été prostituée. Si leur épouse était infidèle, elle était lapidée jusqu'à la mort. Leurs filles devaient rester vierges tant qu'elles vivaient auprès d'eux et qu'elles mangeaient donc les \emph{choses sacrées} provenant du Temple, les parts réservées au clergé sur les offrandes. 

 Le grand prêtre était au sommet de la hiérarchie cléricale. Sa personne était sacrée car il avait été consacré \emph{(oint)}. Il était enserré dans le réseau de règles le plus contraignant. Au contraire de ses concitoyens il n'avait droit ni à la polygamie, ni à une concubine : il n'avait droit qu'à une seule épouse, de pure origine juive, et épousée vierge. Ces contraintes qu'il subissait plus durement qu'aucun autre dans sa vie privée soulignent le cœur des conceptions juives en matière de morale matrimoniale.
 
 Les prêtres et les lévites respectaient les mêmes interdits que les autres hébreux mais en outre ils n'avaient pas le droit de porter des armes, puisque celles-ci versaient le sang et étaient impures par nature. Ils avaient encore moins le droit de s'en servir. Verser le sang humain leur interdisait définitivement de remplir leurs fonctions cultuelles. Dans le même sens ils devaient rester éloignés de tout blessé susceptible de les souiller par son sang, et se tenir à l'écart de tout cadavre humain et animal, sauf celui de leurs parents les plus proches. C'est d'abord durant leur semaine de service au Temple que les prêtres devaient éviter toute impureté. Les actes sexuels leur étaient interdits durant les jours de purification préalable et durant tous leurs jours de présence à l'autel. Mais puisque seuls les fils de la Tribu de Lévi pouvaient servir le Temple, ils devaient \emph{se procurer une descendance} en dehors de leur temps de service. Plusieurs dizaines de siècles auparavant il en était de même en Égypte (Serge \fsc{SAUNERON}, \emph{les prêtres de l'ancienne Égypte}, Editions du Seuil, 1998 (première édition 1957); cf. page 47 et suivantes) : obligation de la circoncision, du rasage intégral de la tête et du corps, d'ablutions répétées à heures fixes, du respect de divers interdits alimentaires, de jeûnes, abstinence précédant de plusieurs jours et accompagnant les périodes de service au temple, interdiction de la polygamie, interdiction de porter certains tissus,~etc. 
 
 À partir de la destruction du Temple de Jérusalem (en l'an 70 de notre ère), les rabbins vont plus ou moins soumettre l'ensemble du peuple aux règles de pureté qui jusque là n'étaient imposées qu'à la caste sacerdotale.
 
 \section{Travailleurs et esclaves juifs}

 La Genèse fait du travail un devoir qui, à égalité avec la génération, construit l'homme : {\emph{Soyez féconds, multipliez, emplissez la terre et soumettez-la}} (Gn 1, 28), même si ce devoir est pénible à cause de la faute d'Adam : {\emph{à la sueur de ton visage tu mangeras ton pain}} (Genèse, 3, 19). Contrairement aux Grecs et aux Romains, les juifs ne ressentaient pas d'équivalence entre le loisir, {\latin{l'otium}}, et la vertu, pas plus qu'entre le labeur ({\latin{nec-otium}}, d'où vient le négoce) et l'incapacité à la vertu. Ils pensaient que celui qui vit dans l'oisiveté court le risque de se livrer à l'immoralité. Il n'y avait donc pas de métiers réellement et inconditionnellement impurs, en dehors (comme partout à l'époque) de ceux qui touchaient de près ou de loin à la prostitution et aussi, selon le Talmud, de ceux d'organisateurs de spectacles, acteurs, gladiateurs et cochers de cirque. Comme le souligne Aline \fsc{ROUSSELLE} (\emph{La contamination spirituelle}, Science, 1998), ce sont exactement les mêmes métiers que ceux que les Romains jugeaient infâmes. L'impureté attachée à certains autres métiers n'était pas une tare morale mais découlait (mécaniquement) du contact avec les ordures ou avec les sources de la vie et de la mort. Elle compliquait la vie du travailleur à cause des purifications incessantes qu'elle entraînait, mais si le métier exercé était lucratif il n'était pas méprisé. Cette impureté fonctionnelle n'avait en tous cas rien d'une tare indélébile, transmissible à la descendance, contrairement à celle du mamzer.
 
  Entre juifs seul le prêt sans intérêt était autorisé. Le métier d'usurier était l'un des plus vigoureusement blâmé : {\emph{les usuriers sont comparables à ceux qui répandent le sang}} (\fsc{Cohen}, p. 250) : les intérêts exigés par les prêteurs antiques étaient en effet exorbitants. 

 Comme les peuples contemporains le peuple hébreu comprenait des travailleurs libres et des esclaves. Face au travailleur libre, au mercenaire, face à celui qui n'avait que ses mains pour vivre, et dont aucun patron ne garantissait la subsistance, la première exigence était la justice : justice dans le contrat de travail, dans les horaires, dans le salaire, sans retenir indûment ce salaire, en le payant au contraire tous les jours.

 Comme partout à cette époque un hébreu pouvait se vendre lui-même (et en avait le droit moral) s'il manquait de ressources ou s'il était écrasé de dettes. S'il ne pouvait plus nourrir ses enfants il en était pour lui comme pour les autres pères de l'Antiquité placés dans la même situation. Il se devait si possible de les vendre à un coreligionnaire. De même des nouveaux-nés ou des petits enfants étaient exposés à la porte des synagogues afin d'être pris en charge par un juif et non par un païen. 

 Les juifs pouvaient posséder des esclaves juifs ou gentils. Ils ne déniaient à aucun d'eux son appartenance à l'humanité commune. Leurs propres prières leur rappelaient qu'eux aussi avaient été esclaves en Égypte. Ils n'acceptaient pas l'idée que les esclaves soient d'une autre nature que les hommes libres%
%[1]
\footnote{Pourtant selon le Talmud (A.~\fsc{COHEN}, 1980, p. 259) les juifs n'avaient pas une grande opinion des esclaves. Ils accusaient l'esclavage d'être une source de démoralisation pour toute la maison: par le vol (esclave mâles), ou par la lubricité (esclaves femelles). Ils disaient que l'esclave est paresseux et ne gagne pas sa nourriture, qu'il est indolent, infidèle et libertin. Ils estimaient que le travailleur libre produisait deux fois plus que l'esclave : il n'y avait pas besoin de le nourrir quand il ne travaillait pas, d'autre part son ardeur au travail était stimulée par le fait qu'il ne pouvait compter sur un autre que lui-même pour le vivre et le couvert. En fait on retrouvait là tous les jugements classiques des Romains et des Grecs de l'époque sur les esclaves.}%
 : {\emph{As-tu un serviteur ? Considère-le comme un frère et ne jalouse pas ton propre sang}} (\emph{L'ecclésiaste}, XXXIII, 32). {\emph{Si j'ai privé de droits l'esclave, la servante qui se rebelle, que faire à l'heure où Dieu se dresse, Que répondre à son examen ? Car dans un ventre il les a faits comme il m'a fait, dans le sein il nous a unis}} (Job, XXXI, 13-15). 

 Cela étant dit il n'en reste pas moins qu'un esclave juif ne pouvait pas parler pour lui-même, mais seulement sous le contrôle de son maître, comme dans les autres sociétés antiques. Religieusement et civilement il était dépendant. Il était soumis à peu près aux mêmes obligations cultuelles que les femmes. À la synagogue, l'esclave circoncis mâle adulte ne comptait pas dans le quorum nécessaire pour certaines prières ou célébrations, sauf s'il n'y avait que des esclaves présents.

 Quant à l'esclave mâle incirconcis, il était aussi impur que tous les autres incirconcis. Cela rendait impossible sa cohabitation avec une famille juive. Il devait donc absolument être circoncis%
%[2]
\footnote{Selon le Talmud, il avait droit à un délai d'un an pour accepter l'opération (ce qui peut évoquer un temps d'initiation religieuse) : s'il refusait l'opération, il était revendu à des non juifs. C'était la législation de l'Empire romain, hostile à tout ce qui évoquait une mutilation. Elle ne tolérait de circoncire que les seuls esclaves ayant consenti de manière expresse et par écrit à l'opération. Dans le cas contraire l'esclave circoncis contre son gré était libéré de droit, comme tous ceux que leur maître avait mutilés ou blessés gravement.}%
. Une fois circoncis il devait participer au culte dans tous les actes qui avaient lieu à la maison du maître, observer le sabbat, célébrer la pâque : on ne lui demandait pas d'acte de foi on lui demandait seulement le respect formel des rites. S'il servait un membre du clergé, il pouvait désormais sans sacrilège manger les nourritures consacrées, la part des offrandes faites au temple qui revenait aux prêtres pour leur subsistance. Il était compté au nombre des juifs potentiels : ainsi il pouvait désormais épouser la fille de son maître, ou hériter de ce dernier. Dans ces deux dernières éventualités il était affranchi de droit, sans autre forme de procès, comme partout ailleurs.

 Selon la Tora, tout esclave devait être traité comme un hôte. Selon le Talmud, la loi punissait sévèrement le maître qui mettait à mort son esclave (c'était la même règle qu'à Rome \emph{sous l'Empire}). Celui qui avait été blessé ou mutilé, ou maltraité sévèrement, devait être libéré sans attendre : s'il avait des obligations ou des dettes, elles étaient annulées. S'il était mécontent l'esclave pouvait s'enfuir, il n'était pas poursuivi et il était interdit de le livrer à son maître. Des arrangements devaient être trouvés pour dédommager ce dernier (revente). C'était la pratique traditionnelle dans l'aire grecque, et elle s'était à cette époque répandue chez les Romains \emph{de l'Empire} où les esclaves pouvaient trouver refuge dans certains temples ou au pied des statues de l'empereur sans être poursuivis par la force publique comme esclaves fugitifs.

 Un juif ne pouvait pas être retenu comme esclave par un autre juif plus de six années : il devait alors être libéré, avec sa femme et ses enfants s'ils étaient mariés avant qu'il ne devienne esclave. Il ne devait même pas partir les mains vides : plutôt qu'un statut d'esclave, cela suggère le statut d'un gagé pour dettes que par convention (juridique), un forfait de six années de travail servile libérait de toutes ses obligations ?

 Mais il pouvait aussi refuser sa libération. Il pouvait préférer la sécurité de son emploi à la vie hasardeuse d'un pauvre, d'un journalier. S'il ne pouvait pas subvenir seul à ses besoins, le libérer n'était d'ailleurs pas lui rendre service. Il pouvait également refuser de s'en aller pour ne pas quitter la femme (esclave) dont son maître lui avait donné l'exclusivité, et les enfants qui leur étaient nés. Il choisissait alors de demeurer à vie dans la maison de son maître, dépendance consacrée par le percement de son oreille contre la porte (ou le pilier central) de la maison de ce dernier. Rivé à cette maison --- à cette famille, il en faisait désormais partie définitivement.

 Le Décalogue donnait aux esclaves un jour de repos par semaine (le \emph{Shabbat}). Selon le Talmud un esclave ne devait pas travailler plus longtemps qu'un travailleur libre, ni la nuit, ni à des tâches humiliantes. Il ne devait pas être soumis au travail forcé. Il (elle) devait dans tous les cas être traité avec les égards dus à un mercenaire libre qui habiterait dans la maison. Il ne devait pas être mis à la disposition du public (prostitué, acteur,~etc.) \emph{sauf si c'était son métier auparavant}. Il ou elle ne pouvait pas être prostitué de force, ni contraint à d'autres tâches d'esclave : \enquote{[...] \emph{à laver les pieds de son maître, à lui mettre ses sandales, à porter des vases pour lui dans la maison des bains, à lui prêter appui pour monter un escalier, ou à le transporter dans une litière, un fauteuil ou une chaise à porteurs, toutes choses que les esclaves font pour leur maître.}} (\fsc{COHEN}, p. 254.) Par contre il pouvait choisir de poser les mêmes actes de son plein gré : s'il se reconnaissait esclave, il n'y avait pas de faute à lui demander des actes d'esclave.

 Mais même dans ce cas, aucun rapport sexuel avec un ou une esclave n'était considéré comme insignifiant, comme une affaire entre soi (le maître) et soi (l'esclave). Si un juif voulait prendre pour concubine une captive, une prisonnière de guerre (et même une esclave achetée au marché ?) il devait lui laisser un mois de répit pour s'habituer à sa situation et apaiser sa douleur d'être loin des siens et sans moyens de résistance. Dès que le maître d'une esclave, ou l'un de ses fils, avait usé d'elle charnellement, il ne pouvait plus la revendre, quelle que soit son origine ou sa religion. Elle ne pouvait plus être libérée contre son gré, ce qui aurait signifié la jeter à la rue, puisque par définition elle était sans dot et sans famille. Le maître devait la garder et l'entretenir tant qu'il continuait d'avoir des relations sexuelles avec elle, c'est-à-dire qu'il devait la traiter comme une concubine non esclave. Elle devait être laissée libre de s'en aller à sa guise et partir en femme libre.

 Du fait de ces multiples contraintes, un esclave juif n'avait qu'une valeur médiocre pour un de ses coreligionnaires, d'où le dicton du Talmud : {\emph{quiconque acquiert un esclave hébreu se donne un maître à lui-même}}. Il était plus avantageux de le vendre à des non juifs qu'aucune règle n'obligeait à le libérer au bout de six ans, ni à ménager son corps. Mais en ce cas le devoir des siens était de le racheter parce qu'il serait condamné à vivre dans l'impureté qu'impliquait le commerce continuel avec les incirconcis, et ne pourrait plus suivre la Loi. Par contre, il n'y avait pas à libérer un esclave non hébreu au bout de 6 ans. L'obligation de libérer tous les esclaves quels qu'ils soient ne revenait que les années jubilaires, tous les 50 ans.

 
