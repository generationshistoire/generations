% Le 02.03.2015 :
% Antiquité
% Moyen Âge
% ~etc.
% ~\%

\part*{Conclusion}

\chapter[Conclusion]{}


 Celui qui embrasse d'un seul regard le passé des familles et de la reproduction humaine ne peut pas ne pas observer que le mouvement de l'histoire semble par moments « repasser les plats » et reproduire des configurations déjà observées durant des périodes antérieures. Pourtant à bien y regarder c'est à chaque fois sous une forme originale et imprévue.  
 
 Ainsi, nous avons pu remarquer : 
\begin{itemize}

\item qu'une grande partie du Moyen Âge européen a accordé plus d'autonomie aux femmes et aux enfants que ne l'avait jamais fait une Antiquité grecque et romaine au patriarcat exemplaire ;

\item qu'au rebours de cette tendance, les \crmieme{17} et \siecle{18}s ont atteint un sommet dans le renforcement du pouvoir des pères sur toute leur famille, épouse comprise, sur le modèle du \latin{pater familias} de l'Antiquité tardive, renforcement initié par les professeurs de droit civil et religieux du onzième et du douzième siècle. Sous la pression de la compétition des deux Réformes ennemies et jumelles pour diriger les consciences et les comportements, et avec l'appui des États modernes (à moins que ce ne soit le contraire ?), les sociétés européennes ont atteint alors dans le domaine de la reproduction un niveau de conformité avec le droit canon inégalé jusque là ;

\item que la suite de cet apogée des maîtrises patriarcales au sein des familles a été la réaction anti pères, anti familles et anticléricale des Lumières et de la Révolution française : \frquote{\emph{divorce facile, autorité paternelle partagée avec la mère et contrôlée par la justice, égalité de l'enfant naturel avec l'enfant légitime, plénitude de l'adoption, administration commune des époux, diminution des incapacités dues à l'âge, les exemples sont nombreux, dans le droit de la famille, à témoigner du surprenant modernisme du législateur révolutionnaire%
% [2]
\footnote{Marcel \fsc{GARAUD}, \emph{La révolution française et la famille}, 1978, p. 191.}%
}} ;

\item qu'à cette réaction, le code civil de Napoléon a réagi en restaurant l'essentiel du droit familial d'inspiration romaine de l'ancien régime ;
 
\item qu'en fait c'est la troisième République qui a inscrit durablement dans le droit français de la famille les changements que voulaient faire les révolutionnaires ;
 
\item que c'est pourtant dans le cadre de l'État providence initié par cette même troisième république, et mis en œuvre par la quatrième, que se sont le mieux épanouies les familles traditionnelles : \enquote{\emph{Les années 1945-1965, qu'on pourrait appeler les « vingt glorieuses » de la famille (ce moment où s'impose un modèle familial, quasiment unique, où presque tout le monde se marie, où la famille est relativement féconde, où elle est stable avec un taux de divorces de moins d'un mariage sur dix, et où elle est organisée selon un principe assez strict de partage des rôles sexués, masculin et féminin) : Cette période de 1945-1965 qui nous sert souvent de référence pour penser la famille « traditionnelle » et lui opposer la famille actuelle, a été à bien des égards un moment historique exceptionnel%
%[1]
\footnote{Irène \fsc{THERY}, « Peut-on parler d'une crise de la famille ? Un point de vue sociologique », \emph{Neuropsychiatrie de l'enfance et de l'adolescence}, 2001, 49, p. 492-501.}%
}} \mbox{-- }moment où la réalité vécue par les familles a été plus proche du modèle traditionnel qu'à toute autre période ;

\item que ce sont justement ceux qui sont nés à ce moment-là, les enfants du baby-boom, qui ont joyeusement et férocement poussé la critique des familles si exemplairement traditionnelles dont ils sortaient. Ils ont plébiscité sans réserves les conceptions des révolutionnaires français dans le domaine familial.
 
\end{itemize}
 
 Nous avons vu en quoi la famille constantinienne était une synthèse originale des pratiques matrimoniales des grecs, des juifs, des romains et des chrétiens. Cet amalgame avait pris en une seule génération, au milieu du \siecle{4}. Il avait « pris » comme une mayonnaise ou un mortier peuvent se coaguler en une forme stable et résistante aux déformations.  Si l'on fait abstraction des inflexions apportées par le Code Napoléon, de la création en 1884 du divorce pour faute et des adoucissements apportés à la belle époque au sort des enfants illégitimes, il a fonctionné jusqu'aux années cinquante du \siecle{20}, soit près de \nombre{1600} ans. 
 
 Cette antiquité vénérable donnait en quelque sorte à la famille constantinienne l'air d'être « naturelle », mais cela n'a pas empêché toutes les règles de droit sur lesquelles elle était fondée d'être abrogées entre 1965 et 1985, c'est-à-dire en un temps encore plus bref que celui qui avait été nécessaire pour les mettre en place. 
 
 La famille constantinienne n'était que l'une des modalités des familles patriarcales, et pas la plus typique. Mais elle est en train d'entraîner toutes les autres dans sa chute, même celles des sociétés qui ne l'ont pas pratiquée. Elles sont toutes travaillées par une même lame de fond partie de l'occident et qui déferle sur la planète. Le domaine de la reproduction humaine est aujourd'hui en pleine révolution.
  
 Lorsque ce remue-ménages sera fini tout redeviendra-t-il comme avant, à la façon dont le Code civil n'a gardé du droit révolutionnaire que ce que Napoléon et ses juristes ont choisi d'en préserver, tandis que pour l'essentiel ils restauraient le droit (fondé sur le droit romain) de l'ancien régime, parfois en le durcissant ? 
 
 Ou bien n'en sommes-nous qu'au début d'un processus dont l'issue est à proprement parler inimaginable pour ceux qui ont grandi dans le monde révolu des familles traditionnelles ?

 Sommes-nous parvenus au terminus indépassable de l'histoire des mœurs et de la reproduction humaine ? ... ou seulement à l'étape actuelle d'une course indéfinie, d'une histoire encore à écrire ? 
 
 La seule chose qui soit certaine c'est qu'un retour pur et simple aux pratiques du passé n'est pas possible. C'est le moment de se souvenir que selon Michel \fsc{FOUCAULT}, l'histoire avance par crises entre lesquelles règnent des périodes de stabilité, définies par un cadre de pensée (\emph{épistémè}, ou paradigmes) à chaque fois différent (cf. \emph{Les mots et les choses}). Il faut certes que les règles ainsi adoptées soient tenables, qu'elles ne provoquent pas d'effets trop pervers ni de dysfonctionnements trop insupportables ... mais les humains sont rusés et inventifs et capables d'interpréter de façon créative n'importe quel système.

 En même temps que pan à pan s'effondre dans la loi, dans les têtes et dans les comportements ce qui reste de la synthèse constantinienne, les traits de ce qui va la remplacer commencent sans aucun doute à se dessiner sous nos yeux, même lorsque nous ne savons pas les reconnaître.

On peut penser que bien des historiens, des sociologues et des moralistes du passé auraient donné cher pour être à notre place devant l'expérience d'écologie humaine en grandeur réelle dans laquelle le mouvement de l'histoire nous entraîne inexorablement ? Il sera en tout cas passionnant d'observer comment les enfants et les petits enfants des « baby-boomers » reprendront à leur propre compte toutes les questions actuellement en crise.


(bibliographie en cours de refonte)

