% Le 02.03.2015 :
% Antiquité
% Moyen Âge
% ~etc.
% ~\%

\part{Conclusion}




  \section{De quel droit condamner nos aïeux ?} 
 
 Dans \emph{L'avenir d'une illusion} (1927), FREUD se demande jusqu'où une société humaine peut se permettre d'être souple et tolérante étant donnée la violence (naturelle) des pulsions, désirs et angoisses qu'elle a pour tâche d'humaniser. Sceptique et pessimiste il soutient qu'une grande dose de répression est inévitable, et que c'est même une des conditions de l'élaboration de sociétés vivables. Dans une période donnée peuvent être invisibles, inconscients, ou plutôt innommables et innommés (déniés) les traits de dureté qu'elle n'a pas suffisamment élaborés, les blocs de sauvagerie qu'elle n'a pas su penser. Ce sont des points aveugles dans la représentation que cette période se donne d'elle-même. Ils sont involontaires et bien évidemment personne ne les a voulus en connaissance de cause... par contre ils sauteront aux yeux des générations suivantes qui ne comprendront pas comment il a été possible de ne pas les voir. 
 
 Le plus bel exemple est que depuis des milliers d'années on a admis comme un fait établi et ne souffrant pas la discussion que les femmes étaient inférieures aux hommes, faites pour leur obéir et les servir, qu'il était donc indispensable qu'une part plus ou moins grande de leurs droits soient détenus et exercés par des membres masculins de leur entourage, et qu’il allait de soi qu’elles étaient  exclues de l’enseignement des apprentissages les plus prestigieux et du maniement des outils les plus efficaces.
 
L'histoire de la prise en charge des enfants est elle aussi marquée par plusieurs de ces points aveugles, à commencer par la dureté du sort fait partout, depuis toujours et jusqu'aujourd'hui en toute bonne conscience aux enfants de naissance illégitime, quelles qu'aient été les manières successives de définir en quoi leur naissance était illégitime, c'est-à-dire inopportune. Quoi de plus barbare que la croyance en une impureté ou une infamie de naissance ? Quoi de plus arbitraire et déraisonnable que l'idée qu'être né d'un ou d'une esclave ou encore de ne pas être reconnu par un homme, interdisait irrévocablement de prétendre à des postes à responsabilité ? Quoi de plus étrange pour nous que la valeur religieuse du sang, ou la « pureté » d'une généalogie ? Quoi de plus absurde que de disqualifier moralement les « enfants du péché » tout en absolvant ceux qui auraient commis le « péché » dont ils étaient nés ? 

 Tout se passe comme si les conceptions archaïques du pur et de l'impur avaient continué d'être tenues pour vraies jusqu'à nos jours alors que le caractère moralement inadéquat de ces représentations avait été dénoncé il y a deux mille ans par les stoïciens aussi bien que par les évangiles, dont les thèses ont pourtant été méditées sans interruption depuis lors. Jusqu'au début du 20e siècle chacune de ces propositions a été tenue pratiquement pour vraie par tous ou presque tous, ou par chacun presque tout le temps, alors qu'elles étaient théoriquement insoutenables du point de vue même de ceux qui s'y conformaient. 
 
 Jusqu'à Vincent de Paul on n'appelait pas négligence le sort mortifère qui était fait aux nouveaux-nés abandonnés, parce que les exclure du monde des familles légitimes paraissait être la façon correcte de les traiter et qu'on n'en imaginait pas d'autre. Lui a su le premier ou l'un des premiers, voir en eux autre chose que des êtres impurs qu'il était moralement indifférent de laisser mourir du moment qu'ils étaient baptisés. C'est sur les représentations de ses contemporains qu'il a travaillé et non sur l'art d'accommoder les bébés séparés de leur mère (cet art ne posait pas plus de problèmes aux femmes de son époque qu'à celles d'aujourd'hui). 
 
 Il y a moins d'un siècle les mineurs vagabonds étaient encore considérés et traités comme des délinquants : la criminalisation de leurs errances avait commencé à la fin du Moyen Âge : auparavant on les assimilait aux pèlerins et on se recommandait à leurs prières. 
 
 De même il n'y a guère plus d'un demi-siècle qu'on regardait encore avec méfiance les rencontres entre les enfants placés en institution et leurs parents. 
 
 Et il n'y a guère plus de trente ans qu'on a vraiment pris la mesure de la gravité des dégâts psychologiques produits par les "abus" sexuels perpétrés par les adultes sur les enfants, surtout quand ils ont autorité sur eux, et qu'on a accepté de voir que les "abuseurs" font le plus souvent partie de l'entourage immédiat de leurs victimes, et d’abord de leur famille et de leurs éducateurs. 
 
 Ces points aveugles étaient visibles par tous, mais ils n'étaient pas vus, mais ces violences et ces cruautés faisaient d'autant moins problème qu'elles paraissaient aussi inexorables que le jour et la nuit, aussi naturelles et nécessaires que le soleil et la pluie. 
 
 Il n'est donc pas impossible, il est même probable qu'aujourd'hui aussi s'étalent sous nos yeux des malheurs et des souffrances que nous ne voyons pas, des maltraitances que nous produisons en toute bonne (in)conscience. Si c'est réellement le cas, alors dans un siècle, ou dans dix, on nous reprochera de les avoir méconnus, sans comprendre que nous ne pouvions pas les voir, aveuglés que nous sommes par nos théories, nos croyances, nos désirs ou nos intérêts inconscients, de la même façon que nous sommes scandalisés par la brutalité, l'insensibilité et les aberrations logiques de nos prédécesseurs. 
 En l'absence d'observateurs venus d'un autre monde seules des recherches scientifiques rigoureuses sont en mesure d'apporter des éléments de réponse à de telles cécités, mais elles le font toujours trop lentement.
 
 Est-ce que les lois et les pratiques qui encadreront à l'avenir la conception des enfants et l'art de les accommoder produiront moins de souffrances et de troubles que celles du passé ? 
 
 Le recours à la prévention des naissances, à la pilule anticonceptionnelle, à la pilule « du lendemain » et à l'avortement permet en principe qu'il ne naisse plus d'enfants non désirés. Mais suffit-il que ceux qui naissent aient été originellement désirés par leurs géniteurs ou par leurs parents adoptifs pour que disparaissent les problèmes qu'ils posent ou ceux qu'ils rencontrent ? Les enfants ne sont pas sans influence, pour le meilleur et pour le pire, sur la relation que leurs parents construisent avec eux. Volontairement ou sans pouvoir s'en empêcher ils peuvent déplaire à leurs parents sur des points auxquels ces derniers sont viscéralement attachés. Nul ne peut donc garantir qu'à l'avenir il y aura moins d'enfants mal assumés que par le passé. 
 
On ne voit pas bien non plus en quoi les enfances organisées par les manipulations de la biologie et des relations inter-personnelles évoquées dans les chapitres précédents seraient un progrès du point de vue des enfants concernés. Il est vrai que ce n'est pas leur objectif. 
 
 
 Si la pauvreté matérielle n'est plus depuis longtemps un motif suffisant à lui seul pour séparer les enfants de leurs parents, est-on assuré pour autant qu'il n'existe et n'existera plus jamais d'enfants privés de l'un ou de l'autre de leurs parents alors que ceux-ci sont disponibles, volontaires pour les élever et suffisamment compétents ? L'absence de l'un des deux parents pour d'autres raisons que la maladie ou la mort devient au contraire quelque chose de plus en plus fréquent. 
 
   \section{L'histoire de la reproduction humaine va-t-elle vers une fin ?}
 
 Celui qui embrasse d'un seul regard le passé des familles et de la reproduction humaine peut observer à quel point le mouvement de l'histoire semble par moments « repasser les plats » et reproduire des configurations déjà observées durant des périodes antérieures. Pourtant à bien y regarder c'est à chaque fois sous une forme originale et imprévue.  
 
 Ainsi, nous avons pu remarquer : 
\begin{itemize}

\item qu'une grande partie du Moyen Âge européen a accordé plus d'autonomie aux femmes et aux enfants que ne l'avait jamais fait une Antiquité grecque et romaine au patriarcat sans compromis ;

\item qu'au rebours de cette tendance, les \crmieme{17} et \siecle{18}s ont atteint un sommet dans le renforcement du pouvoir des pères sur toute leur famille, épouse comprise, sur le modèle du \latin{pater familias} de l'Antiquité tardive, renforcement initié par les professeurs de droit civil et religieux du onzième et du douzième siècle. Sous la pression de la compétition entre les deux Réformes ennemies et jumelles pour diriger les consciences et les comportements, et avec l'appui des États modernes (à moins que ce ne soit le contraire ?), les sociétés européennes ont atteint alors dans le domaine de la reproduction un niveau de conformité avec le droit canon inégalé jusque là ;

\item que la suite de cet apogée des maîtrises patriarcales au sein des familles a été la réaction anti pères, anti familles et anticléricale des Lumières et de la Révolution française, réaction qui préfigurait de manière frappante notre dernier demi-siècle : \emph{"divorce facile, autorité paternelle partagée avec la mère et contrôlée par la justice, égalité de l'enfant naturel avec l'enfant légitime, plénitude de l'adoption, administration commune des époux, diminution des incapacités dues à l'âge, les exemples sont nombreux, dans le droit de la famille, à témoigner du surprenant modernisme du législateur révolutionnaire}
\footnote{Marcel \fsc{GARAUD}, \emph{La révolution française et la famille}, 1978, p. 191.}" ;

\item qu'à cette réaction, le code civil de Napoléon a réagi en restaurant l'essentiel du droit familial de l'ancien régime, et de la puissance paternelle sur les épouses et les enfants ;
 
\item que c'est la troisième République qui a réussi à inscrire durablement dans le droit français de la famille certains des changements que voulaient faire les révolutionnaires ;
 
\item que c'est pourtant dans le cadre de l'État providence initié par cette même troisième république, et mis en œuvre par la quatrième, que se sont le mieux épanouies les familles traditionnelles : \enquote{\emph{Les années 1945-1965, qu'on pourrait appeler les « vingt glorieuses » de la famille (ce moment où s'impose un modèle familial, quasiment unique, où presque tout le monde se marie, où la famille est relativement féconde, où elle est stable avec un taux de divorces de moins d'un mariage sur dix, et où elle est organisée selon un principe assez strict de partage des rôles sexués, masculin et féminin) : Cette période de 1945-1965 qui nous sert souvent de référence pour penser la famille « traditionnelle » et lui opposer la famille actuelle, a été à bien des égards un moment historique exceptionnel%
%[1]
\footnote{Irène \fsc{THERY}, « Peut-on parler d'une crise de la famille ? Un point de vue sociologique », \emph{Neuropsychiatrie de l'enfance et de l'adolescence}, 2001, 49, p. 492-501.}%
}} \mbox{-- }moment où la réalité vécue par les familles a été plus proche du modèle traditionnel qu'à toute autre période ;

\item que ce sont justement ceux qui sont nés à ce moment-là, les enfants du baby-boom, qui ont si joyeusement et si férocement poussé la critique des familles si exemplairement traditionnelles dont ils sortaient. Ils ont plébiscité sans réserves les conceptions des révolutionnaires français dans le domaine familial.
 
\end{itemize}
 
 La famille constantinienne était une synthèse originale des pratiques matrimoniales des grecs, des juifs, des romains et des chrétiens, et en une seule génération cet amalgame avait été circonscrit par le droit. Au milieu du \siecle{4} il avait « pris » comme prend un béton en une forme stable et résistante aux déformations.  Si l'on fait abstraction des inflexions apportées par le Code Napoléon, de la création en 1884 du divorce pour faute et des adoucissements apportés à la belle époque au sort des enfants illégitimes, il a fonctionné jusqu'aux années cinquante du \siecle{20} au moins, soit près de \nombre{1600} ans. 
 
 Cette antiquité vénérable donnait en quelque sorte à la famille constantinienne l'air d'être « naturelle », mais cela n'a pas empêché toutes les règles de droit sur lesquelles elle était fondée d'être abrogées entre 1965 et 1985, c'est-à-dire en un temps encore plus bref que celui qui avait été nécessaire pour les mettre en place. 
 
 La famille constantinienne n'était que l'une des modalités des familles patriarcales, et pas la plus typique. Mais elle est en train d'entraîner toutes les autres dans sa chute, même celles des sociétés qui ne l'ont jamais connue. Elles sont toutes travaillées par une même lame de fond partie de l'occident et qui déferle sur la planète. Le domaine de la reproduction humaine est aujourd'hui en pleine révolution.
  
 Lorsque ce remue-ménages sera fini tout redeviendra-t-il comme avant, à la façon dont le Code civil n'a gardé du droit révolutionnaire que ce que Napoléon et ses juristes ont choisi d'en préserver, tandis que pour l'essentiel ils restauraient le droit (fondé sur le droit romain) de l'ancien régime, parfois même en le durcissant ? Ou bien n'en sommes-nous qu'au début d'un processus dont l'issue est à proprement parler inimaginable pour ceux qui ont grandi dans le monde révolu des familles traditionnelles ? Sommes-nous parvenus au terminus indépassable de l'histoire des mœurs et de la reproduction humaine ? ... ou seulement à l'étape actuelle d'une course indéfinie, d'une histoire encore à écrire ? 
 
 

 En même temps que pan à pan s'effondre dans la loi, dans les têtes et dans les comportements ce qui reste de la synthèse constantinienne, les traits de ce qui va la remplacer commencent sans aucun doute à se dessiner sous nos yeux, même lorsque nous ne savons pas les reconnaître. De toutes les expériences qui dans la génération et la reproduction ont été entreprises depuis les années 70 du vingtième siècle, lesquelles se révèleront sans intérêt, lesquelles produiront à la longue des résultats catastrophiques, lesquelles convaincront les plus sceptiques de leur pertinence ?  La seule chose qui soit certaine c'est qu'un retour pur et simple aux pratiques du passé n'aura jamais lieu.  Selon Michel \fsc{FOUCAULT} l'histoire avance de crise en crise entre lesquelles règnent des périodes de stabilité, définies par un cadre de pensée (\emph{épistémè}, ou paradigme) à chaque fois différent. Etant donné que rien (sauf une foi) ne garantit que chaque crise nous rapproche d'un avenir toujours plus radieux on peut seulement espérer que les paradigmes à venir ne provoqueront pas d'effets trop pervers ni de souffrances trop insupportables. Avec un peu de chances les humains sauront tirer des leçons de leurs expériences, et montrer qu'ils sont rusés, inventifs et capables d'interpréter de façon créative n'importe quel système ...ou presque. 
 
 Convenons que bien des historiens, des sociologues et des moralistes du passé auraient donné cher pour être à notre place devant les expériences d'écologie familiale et reproductive dans lesquelles le mouvement de l'histoire nous entraîne inexorablement. Il sera passionnant d'observer comment les enfants et les petits enfants des « baby-boomers » reprendront à leur propre compte toutes les questions actuellement en crise et ce que les nouveaux couples parentaux nous apprendront sur les relations interpersonnelles et sur les relations inter et intra-genres.
 
 


