\chapter{À propos du changement des mœurs}
 
L'histoire des familles et de la génération depuis l'ancien régime conduit à se demander   lesquelles sont premières des lois ou des moeurs ? Les mœurs sont-elles modelées par les  lois ? Ou bien celles-ci ne font-elles que suivre les mœurs ? Le législateur court-il après la société civile pour bénir les libertés qu'elle conquiert ou mène-t-il le jeu en lui ouvrant ou lui fermant à son gré les posibilités qui lui semblent bonnes ? On trouverait dans ce qui s'est passé depuis trois siècles des exemples propres à soutenir les deux thèses\footnote{\emph{Droit de la famille}, 5e édition, Hugues Fulchiron, Philippe Malaurie, Editeur : L.G.D.J, 2016}.

Mais la véritable source des évolutions est ailleurs.
Tout a commencé par ceux qui se désignaient du nom de "philosophes". En revendiquant ce titre ils ne voulaient pas dire qu'ils étaient des professionnels de la philosophie, mais ils affirmaient leur volonté d'autonomie par rapport aux théologiens. Aix interprétations de ceux-ci, qui avaient dominé la sphère intellectuelle depuis la fin de l'antiquité, et qui s'étaient imposées aux pouvoirs civils eux-mêmes, ils opposaient leur droit d'en proposer d'autres avec une légitimité égale. Ce faisant ils se situaient dans le prolongement du mouvement initié par la Réforme protestante.  

Une nouvelle anthropologie (ou plusieurs)
 
