% Le 09.03.2015 :
% Grec, Romain
% ~etc.
% Antiquité


\chapter{Travailleurs et esclaves juifs}

 La Genèse fait du travail un devoir qui, à égalité avec la génération, construit l'homme : {\emph{Soyez féconds, multipliez, emplissez la terre et soumettez-la}} (Gn 1, 28), même si ce devoir est pénible à cause de la faute d'Adam : {\emph{à la sueur de ton visage tu mangeras ton pain}} (Genèse, 3, 19). Contrairement aux Grecs et aux Romains, les juifs ne ressentaient pas d'équivalence entre le loisir, {\latin{l'otium}}, et la vertu, pas plus qu'entre le labeur ({\latin{nec-otium}}, d'où vient le négoce) et l'incapacité à la vertu. Ils pensaient que celui qui vit dans l'oisiveté court le risque de se livrer à l'immoralité. Il n'y avait donc pas de métiers réellement et inconditionnellement impurs, en dehors (comme partout à l'époque) de ceux qui touchaient de près ou de loin à la prostitution et aussi, selon le Talmud, de ceux d'organisateurs de spectacles, acteurs, gladiateurs et cochers de cirque. Comme le souligne Aline \fsc{ROUSSELLE} (\emph{La contamination spirituelle}, Science, 1998), ce sont exactement les mêmes métiers que ceux que les Romains jugeaient infâmes. L'impureté attachée à certains autres métiers n'était pas une tare morale mais découlait (mécaniquement) du contact avec les ordures ou avec les sources de la vie et de la mort. Elle compliquait la vie du travailleur à cause des purifications incessantes qu'elle entraînait, mais si le métier exercé était lucratif il n'était pas méprisé. Cette impureté fonctionnelle n'avait en tous cas rien d'une tare indélébile, transmissible à la descendance, contrairement à celle du mamzer. Entre juifs seul le prêt sans intérêt est autorisé. Le métier d'usurier est l'un des plus vigoureusement blâmé : {\emph{les usuriers sont comparables à ceux qui répandent le sang}} (\fsc{Cohen}, p. 250) : les intérêts exigés par les prêteurs antiques étaient en effet exorbitants. 

 Comme les peuples contemporains le peuple hébreu comprenait des travailleurs libres et des esclaves. Face au travailleur libre, au mercenaire, face à celui qui n'avait que ses mains pour vivre, et dont aucun patron ne garantissait la subsistance, la première exigence était la justice : justice dans le contrat de travail, dans les horaires, dans le salaire, sans retenir indûment ce salaire, en le payant au contraire tous les jours.

 Comme partout à cette époque un hébreu pouvait se vendre lui-même (et en avait le droit moral) s'il manquait de ressources ou s'il était écrasé de dettes. S'il ne pouvait plus nourrir ses enfants il en était pour lui comme pour les autres pères de l'Antiquité placés dans la même situation. Il se devait si possible de les vendre à un coreligionnaire. De même des nouveaux-nés ou des petits enfants étaient exposés à la porte des synagogues afin d'être pris en charge par un juif et non par un païen. 

 Les juifs pouvaient posséder des esclaves juifs ou gentils. Ils ne déniaient à aucun d'eux son appartenance à l'humanité commune. Leurs propres prières leur rappelaient qu'eux aussi avaient été esclaves en Égypte. Ils n'acceptaient pas l'idée que les esclaves soient d'une autre nature que les hommes libres%
%[1]
\footnote{Pourtant selon le Talmud (A.~\fsc{COHEN}, 1980, p. 259) les juifs n'avaient pas une grande opinion des esclaves. Ils accusaient l'esclavage d'être une source de démoralisation pour toute la maison: par le vol (esclave mâles), ou par la lubricité (esclaves femelles). Ils disaient que l'esclave est paresseux et ne gagne pas sa nourriture, qu'il est indolent, infidèle et libertin. Ils estimaient que le travailleur libre produisait deux fois plus que l'esclave : il n'y avait pas besoin de le nourrir quand il ne travaillait pas, d'autre part son ardeur au travail était stimulée par le fait qu'il ne pouvait compter sur un autre que lui-même pour le vivre et le couvert. En fait on retrouvait là tous les jugements classiques des Romains et des Grecs de l'époque sur les esclaves.}%
 : {\emph{As-tu un serviteur ? Considère-le comme un frère et ne jalouse pas ton propre sang}} (\emph{L'ecclésiaste}, XXXIII, 32). {\emph{Si j'ai privé de droits l'esclave, la servante qui se rebelle, que faire à l'heure où Dieu se dresse, Que répondre à son examen ? Car dans un ventre il les a faits comme il m'a fait, dans le sein il nous a unis}} (Job, XXXI, 13-15). 

 Cela étant dit il n'en reste pas moins qu'un esclave juif ne pouvait pas parler pour lui-même, mais seulement sous le contrôle de son maître, comme dans les autres sociétés antiques. Religieusement et civilement il était dépendant. Il était soumis à peu près aux mêmes obligations cultuelles que les femmes. À la synagogue, l'esclave circoncis mâle adulte ne comptait pas dans le quorum nécessaire pour certaines prières ou célébrations, sauf s'il n'y avait que des esclaves présents.

 Quant à l'esclave mâle incirconcis, il était aussi impur que tous les autres incirconcis. Cela rendait impossible sa cohabitation avec une famille juive. Il devait donc absolument être circoncis%
%[2]
\footnote{Selon le Talmud, il avait droit à un délai d'un an pour accepter l'opération (ce qui peut évoquer un temps d'initiation religieuse) : s'il refusait l'opération, il était revendu à des non juifs. C'était la législation de l'Empire romain, hostile à tout ce qui évoquait une mutilation. Elle ne tolérait de circoncire que les seuls esclaves ayant consenti de manière expresse et par écrit à l'opération. Dans le cas contraire l'esclave circoncis contre son gré était libéré de droit, comme tous ceux que leur maître avait mutilés ou blessés gravement.}%
. Une fois circoncis il devait participer au culte dans tous les actes qui avaient lieu à la maison du maître, observer le sabbat, célébrer la pâque : on ne lui demandait pas d'acte de foi on lui demandait seulement le respect formel des rites. S'il servait un membre du clergé, il pouvait désormais sans sacrilège manger les nourritures consacrées, la part des offrandes faites au temple qui revenait aux prêtres pour leur subsistance. Il était compté au nombre des juifs potentiels : ainsi il pouvait désormais épouser la fille de son maître, ou hériter de ce dernier. Dans ces deux dernières éventualités il était affranchi de droit, sans autre forme de procès, comme partout ailleurs.

 Selon la Tora, tout esclave devait être traité comme un hôte. Selon le Talmud, la loi punissait sévèrement le maître qui mettait à mort son esclave (c'était la même règle qu'à Rome \emph{sous l'Empire}). Celui qui avait été blessé ou mutilé, ou maltraité sévèrement, devait être libéré sans attendre : s'il avait des obligations ou des dettes, elles étaient annulées. S'il était mécontent l'esclave pouvait s'enfuir, il n'était pas poursuivi et il était interdit de le livrer à son maître. Des arrangements devaient être trouvés pour dédommager ce dernier (revente). C'était la pratique traditionnelle dans l'aire grecque, et elle s'était à cette époque répandue chez les Romains \emph{de l'Empire} où les esclaves pouvaient trouver refuge dans certains temples ou au pied des statues de l'empereur sans être poursuivis par la force publique comme esclaves fugitifs.

 Un juif ne pouvait pas être retenu comme esclave par un autre juif plus de six années : il devait alors être libéré, avec sa femme et ses enfants s'ils étaient mariés avant qu'il ne devienne esclave. Il ne devait même pas partir les mains vides : plutôt qu'un statut d'esclave, cela suggère le statut d'un gagé pour dettes que par convention (juridique), un forfait de six années de travail servile libérait de toutes ses obligations ?

 Mais il pouvait aussi refuser sa libération. Il pouvait préférer la sécurité de son emploi à la vie hasardeuse d'un pauvre, d'un journalier. S'il ne pouvait pas subvenir seul à ses besoins, le libérer n'était d'ailleurs pas lui rendre service. Il pouvait également refuser de s'en aller pour ne pas quitter la femme (esclave) dont son maître lui avait donné l'exclusivité, et les enfants qui leur étaient nés. Il choisissait alors de demeurer à vie dans la maison de son maître, dépendance consacrée par le percement de son oreille contre la porte (ou le pilier central) de la maison de ce dernier. Rivé à cette maison --- à cette famille, il en faisait désormais partie définitivement.

 Le Décalogue donnait aux esclaves un jour de repos par semaine (le \emph{Shabbat}). Selon le Talmud un esclave ne devait pas travailler plus longtemps qu'un travailleur libre, ni la nuit, ni à des tâches humiliantes. Il ne devait pas être soumis au travail forcé. Il (elle) devait dans tous les cas être traité avec les égards dus à un mercenaire libre qui habiterait dans la maison. Il ne devait pas être mis à la disposition du public (prostitué, acteur,~etc.) \emph{sauf si c'était son métier auparavant}. Il ou elle ne pouvait pas être prostitué de force, ni contraint à d'autres tâches d'esclave : \enquote{[...] \emph{à laver les pieds de son maître, à lui mettre ses sandales, à porter des vases pour lui dans la maison des bains, à lui prêter appui pour monter un escalier, ou à le transporter dans une litière, un fauteuil ou une chaise à porteurs, toutes choses que les esclaves font pour leur maître.}} (\fsc{COHEN}, p. 254.) Par contre il pouvait choisir de poser les mêmes actes de son plein gré : s'il se reconnaissait esclave, il n'y avait pas de faute à lui demander des actes d'esclave.

 Mais même dans ce cas, aucun rapport sexuel avec un ou une esclave n'était considéré comme insignifiant, comme une affaire entre soi (le maître) et soi (l'esclave). Si un juif voulait prendre pour concubine une captive, une prisonnière de guerre (et même une esclave achetée au marché ?) il devait lui laisser un mois de répit pour s'habituer à sa situation et apaiser sa douleur d'être loin des siens et sans moyens de résistance. Dès que le maître d'une esclave, ou l'un de ses fils, avait usé d'elle charnellement, il ne pouvait plus la revendre, quelle que soit son origine ou sa religion. Elle ne pouvait plus être libérée contre son gré, ce qui aurait signifié la jeter à la rue, puisque par définition elle était sans dot et sans famille. Le maître devait la garder et l'entretenir tant qu'il continuait d'avoir des relations sexuelles avec elle, c'est-à-dire qu'il devait la traiter comme une concubine non esclave. Elle devait être laissée libre de s'en aller à sa guise et partir en femme libre.

 Du fait de ces multiples contraintes, un esclave juif n'avait qu'une valeur médiocre pour un de ses coreligionnaires, d'où le dicton du Talmud : {\emph{quiconque acquiert un esclave hébreu se donne un maître à lui-même}}. Il était plus avantageux de le vendre à des non juifs qu'aucune règle n'obligeait à le libérer au bout de six ans, ni à ménager son corps. Mais en ce cas le devoir des siens était de le racheter parce qu'il serait condamné à vivre dans l'impureté qu'impliquait le commerce continuel avec les incirconcis, et ne pourrait plus suivre la Loi. Par contre, il n'y avait pas à libérer un esclave non hébreu au bout de 6 ans. L'obligation de libérer tous les esclaves quels qu'ils soient ne revenait que les années jubilaires, tous les 50 ans.

