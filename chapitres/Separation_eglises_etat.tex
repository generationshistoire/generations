% Le 18 mars 2015 :
% Antiquité
% Moyen Âge
% ~etc.
% Droit


\chapter{Séparation des églises et de l'État}


 La France n'était jusque là jamais sortie du cadre intellectuel et moral du catholicisme dans lequel elle s'était constituée, sauf durant quelques années pendant la Révolution française. Les mouvements de laïcisation de la fin du Moyen Âge et de la Renaissance, comme ceux du \siecle{18}, avaient travaillé sur les limites entre ce qui revenait aux pouvoirs civils et ce qui revenait au personnel ecclésiastique, mais ils n'avaient pas fondamentalement mis en question la place de la religion catholique comme source du Droit. Le Concordat de \hbox{Napoléon} avait remanié cette situation sans la modifier radicalement. Les autres confessions et les « sans religion » ne représentaient en 1804 qu'un très faible pourcentage de la population. Même si le degré d'identification des français à l'Église fluctuait beaucoup suivant les régions et les milieux sociaux, la population française était très majoritairement catholique. A la fin du \siecle{19} les religieux étaient bien plus nombreux qu'à la fin de l'ancien régime : {\emph{\nombre{81000} religieux en 1789, \nombre{13000} en 1808, \nombre{160000} en 1878}%
% [1]
\footnote{Christian \fsc{SORREL}, \emph{La République contre les congrégations – Histoire d'une passion française 1899-1904}, éd. du Cerf 2003, p. 12. Dans \emph{L'ancien régime, institutions et sociétés} (Le livre de poche, 1993, p.68) François \fsc{BLUCHE} donne des chiffres différents, mais du même ordre de grandeur pour les religieux : \emph{le monde ecclésiastique comprenait, à l'extrême fin de l'ancien régime, un peu moins de \nombre{140000} membres. Le clergé régulier (religieux et religieuses, moines et moniales) regroupait quelques \nombre{59000} âmes (dont \nombre{28000} femmes)... Le clergé séculier représentait quelque \nombre{80000} hommes d'Église (\nombre{139} prélats, environ \nombre{10000} chanoines et les \nombre{70000} prêtres assurant le culte des \nombre{40000} paroisses).}}%
}... La Révolution avait supprimé les monastères et les couvents, et confisqué tous leurs biens, et le Concordat n'avait prévu aucun cadre juridique pour les congrégations religieuses. Et pourtant d'innombrables congrégations nouvelles avaient été créées durant tout le siècle, tandis que beaucoup parmi les anciennes s'étaient relevées de leur état de langueur du \siecle{18}. Les « congréganistes » s'investissaient d'abord et avant tout dans l'enseignement, alors en plein essor, notamment dans le primaire, et aussi et comme toujours dans les services hospitaliers, eux aussi en expansion : en 1847 il y avait en France plus de sept mille religieuses hospitalières, à la fin du Second Empire plus de dix mille, en 1905 plus de douze mille.

 C'est justement là que le bât blessait : le programme des républicains qui avaient conquis le pouvoir en 1879 faisait de la solidarité et de l'enseignement des outils essentiels de gouvernement%
% [2]
\footnote{Cf. \emph{L'invention du social, essai sur le déclin des passions politiques}, Jacques \fsc{DONZELOT}, 1994.}% 
, et il n'était pas question pour eux de les laisser aux mains des employés permanents de l'Église. Ils voulaient retirer à celle-ci les points d'appui institutionnels sur lesquels elle avait assis son influence depuis Constantin. À partir du moment où la gauche radicale l'a emporté dans les urnes, l'histoire des familles comme celle du traitement de la pauvreté a changé de direction. La nouvelle majorité s'est donnée pour mission des tâches traditionnellement dévolues à la Providence%
%[3]
\footnote{Le terme « Providence » est l'un des noms de Dieu.}%
. Au lieu de déplorer les malheurs et les injustices de la \emph{vallée de larmes} où vivraient les hommes, tout en comptant sur un \emph{au-delà} paradisiaque ou infernal pour régler à chacun son compte, ses membres ont estimé du devoir de l'État de s'attaquer lui-même aux sources des malheurs individuels, et d'abord aux injustices sociales, sans se reposer sur les initiatives privées, expressions de la Providence, et de procurer aux citoyens sinon le bonheur du moins un droit effectif à une aide efficace, afin de prévenir le malheur quand c'est possible et de soulager les souffrances quand cela ne l'est pas. 
 Ils s'attaquaient résolument et en pleine connaissance de cause à son autorité sur les esprits.

 À côté de mesures de portée limitée ou relativement symbolique%
% [4]
\footnote{Exemples : suppression de l'obligation du repos dominical (rétabli dès 1906 sous la pression des associations ouvrières) ; sécularisation des cimetières ; suppression des prières publiques constitutionnelles et de tous les signes religieux présents dans les lieux publics ; imposition du service militaire aux religieux et aux séminaristes ; exclusion des membres du clergé des commissions d'enseignement des hôpitaux en tant que membres de droit : curés chargés d'une paroisse, évêques,~etc.}% 
, les républicains ont d'emblée exclu les facultés de théologie des universités publiques, et le personnel religieux du corps enseignant universitaire. L'institution de l'obligation scolaire jusqu'à 12 ans était devenue inéluctable à cette époque%
%[5]
\footnote{Selon \fsc{FURET} et \fsc{OZOUF}, dès le milieu du \siecle{19} près des trois quarts des enfants français sont scolarisés. Chaque commune était depuis Guizot astreinte à l'obligation de fournir une école primaire publique à ses habitants, mais pas à en garantir la laïcité, d'ailleurs souvent refusée par la majorité de la population, comme la suite de l'histoire l'a montré. Le nombre d'enfants scolarisés en 1850 dans les écoles primaires (héritières des petites écoles des siècles précédents) représentait 73~\% du nombre des enfants de la tranche d'âge des 6-13 ans. Il en représentait même 105~\% en 1876-1877 : plus de 100~\%, ce qui s'explique par les enfants scolarisés avant 6 ans et après 13 ans (\fsc{FURET} et \fsc{OZOUF}, 1977, p. 173). Par conséquent en 1880 les enfants d'âge scolaire non scolarisés ne représentaient plus qu'une petite minorité. Mais ces chiffres moyens couvraient des disparités extrêmement grandes :
%\begin{itemize}
\begin{enumerate}[label=\alph*.,itemsep=0pt]
%A)
\item entre régions (le nord et l'est étaient très scolarisés depuis des siècles, au contraire du sud et de l'ouest, très peu scolarisés),
% B)
\item entre villes et campagnes,
% C)
\item parmi les régions rurales elles-mêmes, entre celles de civilisation exclusivement orale comme la Bretagne (valorisant la parole « vivante », et se défiant de la parole « morte », c'est-à-dire écrite), le Pays Basque, la Catalogne,~etc. et celles (de langue française) largement pénétrées par l'écrit,
% D)
\item et au moins autant entre classes sociales.
\end{enumerate}
%\end{itemize}

 Rien ne permettait de penser que ces petites minorités réfractaires à l'école d'alors étaient prêtes à rejoindre spontanément et rapidement le mouvement général, ce qui justifiait d'obliger par la loi les parents à scolariser leurs enfants.}% 
, mais il n'en était pas de même de la laïcité de l'enseignement. Celle-ci était évidemment une arme contre l'Église et son influence dans le domaine scolaire. Alors qu'à cette époque les femmes étaient les plus fidèles soutiens de l'Église, les républicains ont créé pour elles un enseignement secondaire public et laïque similaire en (presque) tout point à celui des garçons. Il s'agissait à la fois de lutter contre l'influence des congrégations en leur interdisant tout enseignement avant de les expulser, et contre la vision traditionnelle d'une femme soumise au contrôle masculin pour l'accès au savoir et à la culture. 

 En légalisant le divorce en 1884, les républicains affranchissaient le mariage civil des règles du Droit Canon. En 1904 l'adultère cesse d'être une faute contre la société dans son ensemble et n'est plus qu'une affaire privée : une fois libérés de leurs unions antérieures, les amants adultères ont le droit de s'unir légalement, ce qui leur était interdit à vie depuis l'empereur Justinien --- interdit qui les empêchait de légitimer leurs enfants déjà nés (adultérins) et leurs enfants encore à naître. 

 Le titre III de la loi 1901 sur les associations a refusé aux congrégations religieuses la liberté d'association. Par conséquent à partir de sa promulgation toutes les congrégations existantes ont été dans l'obligation d'obtenir une autorisation législative, ce dont elles s'étaient le plus souvent passées depuis le début du \siecle{19}. Cette autorisation a été refusée à la grande majorité d'entre elles, \emph{ipso facto} dissoutes. Les congrégations enseignantes, dont les effectifs étaient de beaucoup supérieurs à celui des religieuses hospitalières, ont toutes été interdites%
% [6]
\footnote{Sur un nombre de plus de \nombre{1300} congrégations, \nombre{140} congrégations masculines et \nombre{888} congrégations féminines ont été dissoutes. Cela a concerné plus de cent cinquante mille personnes dont 80~\% de femmes...}% 
. Seules ont été épargnées les congrégations hospitalières ... et toutes les congrégations implantées aux colonies. 

 La laïcisation des hôpitaux a commencé dès 1879, mais elle ne pouvait se faire qu'au rythme de la formation du personnel laïc d'encadrement et infirmier, ce qui demandait d'abord de créer les écoles d'infirmières et de surveillantes nécessaires, puisque les noviciats des congrégations hospitalières en avaient jusque là tenu le rôle. On observe quelques créations vers 1880, puis en 1899 est prise la décision de créer une école d'infirmière dans toutes les villes de faculté. Ceci étant dit la laïcisation de chaque institution dépendait d'abord et surtout de la couleur politique du conseil municipal dont elle dépendait.

 En 1900 les religieuses formaient la plus grande part du personnel soignant, par contre en 1975 l'ensemble du personnel soignant (sans compter les autres employés des hôpitaux) se comptait à près de \nombre{300000} personnes. Les religieuses ne représentaient plus à cette date qu'une toute petite minorité vieillissante. C'est qu'il ne s'agissait plus des anciens hôpitaux et hospices voués essentiellement aux indigents. Désormais il s'agissait d'établissements industriels, la réalisation concrète des « machines à guérir » dont les penseurs de la fin du \siecle{18} avaient rêvé. Les clients avaient complètement changé, et l'échelle aussi : en 1975 on trouvait dans les hôpitaux publics plus de \nombre{10000} (dix mille) médecins des hôpitaux à plein temps et \nombre{26900} internes, beaucoup plus qu'il n'y avait de personnels soignants, tous statuts confondus, dans tous les hôpitaux du \siecle{19}%
% [8]
\footnote{Cf. Jean \fsc{IMBERT}, \emph{Histoire des hôpitaux en France}, 1982.}% 
.


