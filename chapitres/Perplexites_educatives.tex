
\chapter{Perplexités éducatives}


 Les décideurs du passé n'avaient guère de difficultés à se mettre d'accord sur l'éducation des enfants et adolescents : jusqu'au milieu du \siecle{20} a régné dans le domaine éducatif un assez grand consensus autour de règles communes et de limites peu ou pas discutées. Cela se traduisait par exemple par le fait qu'au même moment les établissements éducatifs du \crmieme{19} ou du \siecle{20} présentaient partout à des nuances près les mêmes modes de fonctionnement, les mêmes limites, la même séparation des sexes, les mêmes styles de communication, qu'ils se réfèrent à un corps de doctrine religieux ou à une laïcité stricte. Les décideurs ne doutaient pas de leur droit à imposer leurs analyses aux parents qu'ils jugeaient négligents, délinquants, ou mal pensants. Ils valorisaient l'éducation au sein de la famille, mais au nom même de celle-ci ils plaçaient sans hésiter les enfants loin de leurs parents lorsque les conditions de leur éducation leur paraissaient compromises. C'est pourquoi l'époque actuelle est atypique en ce que depuis un bon demi-siècle elle hésite dans l'idée qu'elle se fait de l'intérêt de l'enfant, dans le choix de ce qu'elle veut lui transmettre, et dans les modalités de la transmission. 

 Il n'existe pas de savoir scientifique sur ce qu'est l'intérêt de l'enfant : il existe certes des connaissances scientifiques de plus en plus affinées sur les liens entre telle mesure éducative et tel résultat, telle performance, tel taux de morbidité,~etc. mais l'intérêt de l'enfant dépend de bien autre chose. Il est lié aux fins que les hommes se donnent, qui ne sont pas scientifiques, mais politiques, philosophiques, religieuses, éthiques... 

 En l'absence de croyance partagée sur ce qu'est l'intérêt de l'enfant, l'accord minimal se fait sur l'idée qu'il faut avant tout ne pas lui nuire. Le reproche majeur qu'encourt un éducateur ce n'est plus de le « gâter » par sa complaisance et par son manque de fermeté. Le principal des risques actuels de son métier, c'est d'être accusé de le maltraiter par des exigences excessives. Parfois les enfants semblent assimilés à une population à libérer (par le droit) de l'arbitraire oppressif que les parents, les enseignants et les institutions d'assistance et de rééducation exerceraient sur eux%
% [1]
\footnote{Références : 
\\Gérard \fsc{MENDEL}, \emph{Pour décoloniser l'enfant, socio psychanalyse de l'autorité}, 1971.
\\Alain \fsc{RENAUT}, \emph{La libération des enfants, contribution philosophique à une histoire de l'enfance}, 2002.}% 
.

 Les « \emph{événements de mai 1968} » ont rendu visible et accéléré une remise en question des institutions, des hiérarchies et de l'argument d'autorité, mise en question qui avait commencé bien avant : avec les « maîtres du soupçon » ? Dès 1889 et la possibilité de déchoir les pères indignes ? Dès la Révolution française et l'exécution de Louis~XVI ? Avec les Lumières et Rousseau ? Avec la Renaissance et Rabelais,~etc. ? C'est aussi que bien des personnes et des institutions revêtues d'autorité avaient montré leurs propres limites, au nom de l'ordre et de l'obéissance, au cours des guerres et sous les régimes totalitaires dont le \siecle{20} a connu quelques beaux exemples, tandis que la psychanalyse soulignait la place centrale du désir du sujet dans son propre développement. D'autre part différentes recherches et expériences scientifiques avaient montré que les méthodes autoritaires d'éducation pouvaient être nocives, et que les méthodes non autoritaires de direction des groupes (Moren, Rogers) comme les pédagogies basées sur la découverte et l'initiative (Freinet, Montessori,~etc.) pouvaient être plus efficaces que les autres.

 Lorsqu'un anonyme inspiré a écrit sur un mur : « \emph{il est interdit d'interdire} », cette proposition jaillie d'on ne sait où a donc été reprise comme une évidence. Pour quelle raison ce paradoxe a-t-il à ce point fait vibrer la génération née au sortir de l'occupation ? ... sans doute parce qu'il était le corollaire d'un autre slogan aussi fameux de la même période : « \emph{jouissons sans entraves} ». 

 Les enfants se sont vus reconnaître le droit à la parole sur tout ce qui les concerne, notamment leurs orientations, mais aussi le droit à une vie sexuelle active dès l'âge de quinze ans (y compris récemment le droit au secret médical, et y compris pour les filles le droit de mener à bien une grossesse ou de choisir une IVG,~etc.) tandis que leurs parents ont été sommés par la loi de les conduire démocratiquement vers une indépendance aussi précoce que possible. 

 Depuis 1882, la durée de l'obligation scolaire s'est allongée et l'âge minimum de la mise au travail est passé de douze à seize ans afin de donner aux jeunes le maximum de chances d'insertion, de les protéger de toute exploitation au travail, et d'écrêter les différences entre milieux scolaires et sociaux différents. Depuis la création du Collège unique (\emph{Réforme \fsc{Haby}}, 1975), tous les enfants bénéficient de ce qui était un privilège jusqu'aux années soixante du \siecle{20}. Les études longues sont plus que jamais la voie royale vers la réussite personnelle. Grâce à leur quasi gratuité, tous ceux qui en ont les moyens intellectuels et le désir (et aussi des parents suffisamment aisés pour subvenir à leurs besoins matériels jusqu'à la fin de leurs études) ont des chances sérieuses de pouvoir en faire. Quant à savoir s'ils sont contents de l'extension de l'âge de 12 ans à 14 ans, puis à 16 ans, de leurs 5 ou 7 heures journalières de fréquentation des enseignants, il est assez évident que nombre d'entre eux, et notamment de garçons, ne la vivent pas bien et le font bruyamment savoir.

 Dans un contexte de concurrence scolaire généralisée, les richesses financières et culturelles des parents ne peuvent plus suppléer aussi massivement qu'autrefois à l'incompétence d'un jeune ou à son absence d'implication personnelle. C'est pourquoi même s'ils sont toujours soucieux de l'avenir de leurs enfants, la pression qu'ils exercent a changé de lieu d'application : du contrôle rigoureux de leur sexualité pré conjugale, autrefois impératif pour leur futur établissement, et désormais sans importance, à l'exigence de performances scolaires aussi brillantes que possible, désormais sans alternative. C'est que rien n'a changé, bien au contraire, dans les règles du jeu qui permettent d'accéder aux meilleures sections des grands lycées et aux plus réputées des grandes écoles françaises et par là aux emplois les mieux payés, les plus attrayants ou les plus influents. Les jeunes n'ont donc plus guère à réprimer leurs désirs sexuels ni à supporter la culpabilité qui s'y attachait, devant un dieu ou devant leurs parents. Par contre il leur faut satisfaire à des normes exigeantes d'autonomie, de productivité et de compétitivité. Ceux qui n'y parviennent pas vivent une « honte » qui peut être au moins aussi insupportable que les anciennes culpabilités. Il n'y a sans doute pas moins de pression parentale aujourd'hui qu'autrefois, et il n'est peut-être pas plus agréable d'y être soumis, ni plus facile d'y satisfaire... 

 Sans parler de la responsabilité qui repose sur les épaules des enfants sur qui l'on compte, à défaut d'autre liens, pour donner sens à la vie de leurs parents :

\begin{displayquote}
\emph{Encore plus importante, naturellement, cette question : qu'est-ce qu'un enfant ? Le paradoxe est ici encore plus important car on n'a jamais autant prêté attention à l'enfant, on ne s'est jamais autant soucié de lui et on n'a jamais autant désenfantisé l'enfant.}
 
\emph{Désenfantiser l'enfant, comme s'il n'était possible de le concevoir comme notre égal qu'en le concevant comme notre semblable}[...]

\emph{L'enfant soutien de famille : ceci évoque un renversement tout à fait fondamental. \emph{[...]} la parentification des enfants dans les familles recomposées, c'est-à-dire un mouvement nouveau où, de façon tout à fait inattendue, la prise en compte de l'enfance aboutit à un déni d'enfance et où l'infantilisation du monde des adultes aboutit à une parentification du monde des enfants.}

 [... cette question encore paradoxale :] \emph{est-ce à l'enfant de dire qui appartient ou qui n'appartient pas à sa propre famille ?}%
% [2]
\footnote{Irène \fsc{THERY}, « Peut-on parler d'une crise de la famille ? un point de vue sociologique », \emph{Neuropsychiatrie de l'enfance et de l'adolescence}, 2001, 49, 492-501.} 
\end{displayquote}

