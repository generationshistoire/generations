
\section{Entrée en scène des Barbares}


 L'occident de l'Empire romain était beaucoup plus pauvre et moins peuplé que son orient. Depuis la fin du \siecle{3}, deux empereurs se partageaient de concert la direction de cet ensemble, l'un en occident, dont la capitale se déplaçait en fonction des urgences militaires, l'autre en orient, à Constantinople. 

 Face aux problèmes de recrutement de ses armées, l'empereur d'occident sous-traitait à des tribus germaniques la défense de plusieurs secteurs de ses frontières. Rémunérées par l'octroi de terres, elles étaient intégrées dans le système de défense du \latin{limes} sous le commandement de leurs propres chefs. Malgré l'assistance de ces soldats de métier, la puissance romaine était de moins en moins capable de contenir les barbares vivant de l'autre côté du \latin{limes}. Dès la fin du \siecle{4}, les empereurs d'occident n'exerçaient plus leurs prérogatives régaliennes avec la détermination et la constance nécessaires, et Constantinople ne venait au secours de Rome que de manière ponctuelle et en fonction de ses propres intérêts. Le 31 décembre 406, les frontières de l'Empire d'Occident ont été débordées par de nouvelles tribus germaniques désireuses de profiter elles aussi des avantages de la romanité, et poussées par des peuples des steppes en expansion, les \emph{Huns} et leurs alliés.

 Désormais la faiblesse de l'Empire d'Occident permet aux Barbares avec qui il s'était associé, et à divers autres qu'il n'avait pas invités, de mettre l'empereur en dépendance, de coloniser sa haute fonction publique, et finalement de démembrer ses provinces en se taillant dans la chair de celles-ci des principautés à peu près indépendantes. Le dernier empereur d'occident est déposé en 476 et les insignes de sa fonction sont remis à l'empereur d'orient. 

 Constantinople va encore vivre \nombre{1000} ans sans rupture brutale avec l'histoire et la culture antiques, ce qui donnera à ses dirigeants la conviction de représenter la norme, la voie « orthodoxe », tandis que l'orient va peu à peu devenir prodigieusement exotique aux yeux des occidentaux.

 En occident des Barbares, Germains pour l'essentiel, vivaient désormais à côté des « Romains », avec qui ils ne pouvaient en principe se marier. Ils obéissaient à leurs propres coutumes (notamment en matière matrimoniale%
% [1]
\footnote{Jean-Pierre \fsc{POLY}, \emph{Le chemin des amours barbares, genèse médiévale de la sexualité européenne}, Perrin, 2003.}%
) et religions, et parlaient leurs propres langues. De même que les populations romaines étaient soumises au \emph{code Théodosien} (438), compilation du droit romain de l'antiquité tardive, de même les Barbares ont rédigé leurs propres codes : pour les Wisigoths \emph{Code d'Euric} (476) puis \emph{Loi Gombette} (502) (= loi de Gondebald : \latin{lex gundobada}) ; \emph{Loi Salique} (511) pour les Francs. Ces divers codes représentaient autant de compromis entre le droit romain et les coutumes des groupes concernés. Quant au \emph{Bréviaire d'Alaric} (506) il reprenait le droit romain en le résumant à destination des sujets de droit romain des rois Wisigoths.

 Pendant ce temps la vie continuait. Pendant ce temps les cités et les administrations romaines continuaient de fonctionner sans grands changements. Les nouveaux maîtres ne voulaient pas détruire l'empire mais s'y intégrer et leurs représentations n'étaient pas radicalement différentes de celles de leurs nouveaux sujets. Ils croyaient encore plus fortement qu'eux à un monde de castes et à la légitimité de la force. Ils appliquaient avec la même dureté les mêmes « lois » de la guerre,~etc. Ils étaient tout aussi esclavagistes. Le risque d'être malmené par les armées en guerre ou enlevé par une bande armée et de se retrouver sur un marché aux esclaves était toujours présent, mais était-il plus grand qu'à d'autres périodes difficiles de l'empire ? Avec les « invasions » certains circuits économiques se transformaient, mais si l'on en croit les recherches archéologiques, globalement l'économie ne semblait pas se porter plus mal qu'auparavant : les Barbares, « réfugiés économiques », s'en seraient allés plus loin si cela avait été le cas.

 Pour ce qui concerne notre sujet la chute de l'Empire Romain d'Occident est donc une date beaucoup moins significative que celle de la conversion de Constantin. En effet cette « chute » a été quelque chose de lent, de progressif : un processus très long, émaillé de catastrophes ponctuelles et de réparations successives, et dont le sens n'était pas lisible d'emblée. Les contemporains ne l'ont pas vécue comme une catastrophe sans contreparties : le vieux monde n'avait pas que des bons côtés. Au fur et à mesure que par pans entiers s'effondraient de vénérables institutions, d'autres façons de vivre devenaient imaginables, et ce n'était pas toujours pire qu'auparavant, même si jusqu'à la Révolution Française la \latin{pax romana} sera toujours idéalisée et décrite comme le modèle de gouvernement qu'il convient d'imiter. 

 Les populations d'origine romaine et de langue latine étaient officiellement catholiques depuis que les descendants de Constantin leur avaient ordonné de se détourner de leurs dieux civiques, pas franchi par la majeure partie de leurs élites cultivées et la plupart des citadins. Elles comprenaient aussi une minorité significative de juifs. Quant aux campagnes elles étaient encore en majeure partie païennes. Il faudra attendre le \siecle{6} pour que leur christianisation soit à peu près réalisée. Les élites romaines du pouvoir, de la culture et de la richesse se ralliaient progressivement à l’Église comme à la seule institution capable de faire pièce aux risques que les temps nouveaux faisaient courir à la romanité. Au \siecle{5} l'évêque, seul magistrat subsistant de l'ancienne grandeur romaine était en train de devenir pour longtemps le premier des membres de la Curie de la ville où était située sa cathédrale (la ville la plus grande du diocèse en général). Ceux-ci votaient pour lui, à côté des membres du clergé, en tant que représentants du peuple. Il était le plus puissant des patrons des « pauvres » \latin{(pauperes)}, des populations romaines désarmées et soumises au pouvoir des nouveaux souverains. 

 Presque tous les Barbares étaient païens, et sauf exception ceux qui ne l'étaient pas étaient \emph{ariens}. Mais il ne semble pas que leurs effectifs aient ordinairement représenté beaucoup plus que quelques pour cent de celui des populations romaines sur lesquelles ils avaient établi leur domination. Compte tenu du rapport des forces démographiques en présence et compte tenu de l'importance de la religion comme facteur d'unité politique, sans oublier le prestige de la culture et des institutions romaines, dont les clercs chrétiens étaient porteurs (et de plus en plus les uniques transmetteurs), l'un après l'autre les chefs barbares vont embrasser la religion de leurs nouveaux sujets (cf. Clovis), et entraîner progressivement leurs tribus à les suivre%
% [2]
\footnote{Cf. Bruno \fsc{DUMEZIL}, \emph{Les racines chrétiennes de l'Europe, conversion et liberté dans les royaumes barbares, \siecles{5}{8}}.}
. 

