

 
 \chapter{L'enfance des baby-boomers}
 
  On a vu que les cadres juridiques du mariage traditionnel (« constantinien ») avaient été modifiés de manière significative à partir de l'avènement de la III\ieme{} République mais pour bien des raisons les changements dans les mœurs sont restés peu visibles jusqu'aux années 60--70 du \siecle{20}. Entre 1945 et 1960 la famille traditionnelle définie par le mariage hétérosexuel monogame et (presque) indissoluble, élevant sous l'autorité du mari les seuls enfants nés de ses oeuvres, a même semblé triompher  de toutes les formes de famille concurrentes (concubinage hétéro ou homosexuel, femmes seules avec enfants...). Les années du \anglais{baby-boom}, de l'explosion de la natalité qui a suivi la fin de la II\ieme{} guerre mondiale ont pu faire croire, alors, que les idéologies natalistes avaient définitivement triomphé. Et pourtant depuis 1960 les évolutions du droit, des mœurs et des représentions ont été fulgurantes. Il est souvent arrivé que le moment de la plus grande perfection, de la plus large acceptation, le moment où plus personne ne conteste une pratique, une idée, une croyance, soit immédiatement suivi par leur désaffection et par l'irruption d'une nouveauté qui les rend obsolètes. On peut citer entre autres la marine à voile, la machine à vapeur, la machine à écrire. On peut citer le marxisme brutalement « ringardisé » à partir de 1975--80 au moment où il triomphait dans l'Université française, ou encore l'effondrement de la pratique religieuse des catholiques juste après leur ferveur collective au moment du Concile « Vatican II ». Le contraste entre  le monde dans lequel ont grandi les enfants du \anglais{baby-boom} et celui où vivent leurs petits-enfants est si grand qu'il mérite un « arrêt sur image ». 
  
Jusqu'aux années 60 du \siecle{20} le statut légal des femmes françaises était plus proche de celui des femmes de l'Antiquité tardive que de celui de leurs petites-filles de l'an 2010, et même si elle était questionnée la prééminence du masculin était la règle. Le mâle \latin{(vir)} restait le modèle accompli du genre humain \latin{(homo)}, la femme était toujours son exception. Certes pleine de charme et de mystère et belle à troubler les plus chastes, elle n'en était pas moins caractérisée par le manque. Avec plus ou moins de rigueur mais d'un mouvement unanime toutes les sociétés faisaient d'elle un être de second rang, presque toujours limité dans ses droits et dans son autonomie\footnote{Même si dans la plupart des couples populaires c'étaient en réalité les épouses qui tenaient les cordons de la bourse,  de nombreux prolétaires chefs de famille (si l'on en croit les observateurs du \siecle{19}), dilapidaient les gains des uns et des autres (épouse, enfants d'âge à travailler) dont ils étaient jusque là les bénéficiaires et les gestionnaires. C'est au nom de ces abus, pour les protéger elles et leurs enfants, et non pour favoriser la parité homme-femme, que les femmes mariées ont reçu au début du \siecle{20} le droit de percevoir et de gérer elles-mêmes leurs gains et leurs économies personnelles.}, comme une mineure, et toujours exclue des postes de pouvoir. Les hommes remerciaient leurs dieux de n'être pas nés femmes, et celles-ci exerçaient souvent une pression efficace sur leurs filles pour qu'elles ne s'écartent pas du rôle attendu d'elles. La publication en 1949 du \emph{Deuxième sexe} de Simone de Beauvoir a fait l'effet d'un « coup de tonnerre dans un ciel bleu », et c'est très lentement que ses thèses ont diffusé.
 
 
 
La première des fonctions du mariage a toujours été de donner des enfants aux hommes (cf. Irène Théry). À part dans l'aire de civilisation chrétienne les hommes pouvaient répudier les épouses qui ne leur avaient pas donné les héritiers mâles qu'ils voulaient ou bien leur adjoindre des concubines. Même en chrétienté il a fallu bien des siècles pour l'empêcher. Tout semblait fait, inconsciemment bien évidemment, pour décourager les femmes de concevoir des enfants sans en passer par un homme publiquement désigné. Pour défendre l'institution familiale et la société comprise comme un réseau de familles alliées (un filet de relations noué par les mariages) il fallait décourager les femmes de faire naître des enfants sans pères, et interdire aux hommes de se procurer des héritiers en dehors d'une alliance avec une autre famille. 

L'infériorité des salaires féminins à travail égal favorisait leur dépendance. Les brimades, interdits et humiliations subis par les « mal nés » du passé, par les enfants nés d'unions « irrégulières »,  avaient pour but ultime de prévenir la conception d'enfants en dehors du cadre du mariage légitime, en dehors de l'alliance de la mère avec un homme nommément désigné. Depuis la fin du Moyen âge, la coutume en France était de nommer les enfants du nom de leur père. Dans ce contexte porter le nom de sa mère signifiait que l'on était né hors mariage et qu'on n'avait pas de père légitime. Les hommes tenaient les femmes en leur dépendance « à cause des enfants » dont le statut et l'installation dans l'existence dépendaient plus d'eux que d'elles. 

Compte tenu de la faiblesse du taux de divorces le mariage permettait à presque tous ceux qui le désiraient d'avoir des enfants bien à eux qui ne leur seraient contestés par personne et en même temps de s'attacher une femme et les services de tous ordres que seule celle-ci pouvait fournir : la pertinence de la répartition des tâches selon le sexe n'a guère été contestée jusqu'à la seconde moitié du \siecle{20}. Mais la réciproque était vraie aussi : le mariage permettait aux femmes d'avoir des enfants sans être obligées de les élever seules, dans la pauvreté et l'humiliation. Quant à celles qui y attachaient du prix, il leur permettait de s'attacher solidement tel ou tel homme, sa personne, son statut social et ses ressources... ce que symbolisaient (depuis l'Antiquité romaine) les anneaux que s'échangeaient les conjoints, et ce qu'exprimaient sur le mode burlesque des expressions comme \frquote{\emph{se laisser mettre le grappin dessus}}, ou \frquote{\emph{se passer la corde au cou}}. Il est symptomatique que c'étaient les hommes qui employaient ces expressions : dans les représentations d'alors ce sont les femmes qui cherchaient le plus activement et le plus anxieusement à se marier.


 La généralisation du salariat au cours de la première moitié du \siecle{20} supprimait l'obligation d'avoir un capital pour « s'établir » et permettait au mariage d'amour de devenir le modèle du bon mariage, même si les éléments raisonnables du choix du conjoint n'avaient pas disparu. Cela posait en axiome que l'amour mutuel des époux était à l'origine de leur union et non la conséquence de leur vie commune. C'est ce même amour qui donnait sens au renoncement des femmes à une carrière indépendante de leur conjoint. Cette représentation était d'autant mieux  partagée que les salaires masculins étaient attractifs, que les emplois salariés étaient relativement sûrs, et que les aides à la famille étaient substantielles.
 
 Sauf crime caractérisé les membres dépendants d'une famille antique (épouse, enfants, serviteurs libres ou esclaves) ne pouvaient faire appel d'aucune des décision du père de famille devant un magistrat. Chez les Grecs et les Romains l'unité de commandement était considérée comme au moins aussi nécessaire à la famille qu'à toute autre institution. Sous l'Ancien Régime chaque père était de la même façon responsable des membres de sa famille et avait autorité sur eux pour éviter tout désordre, avec l'appui des institutions : ce qui se passait au sein de la famille ne regardait que lui. Du haut Moyen âge aux Lumières, la continuité entre le pouvoir du roi et celui des pères allait sans dire et n'avait pas à être démontrée. L'ensemble des chefs de famille tenait la société en main. En cas de désaccord entre les deux époux, c'est le mari qui tranchait comme dans toutes les sociétés patriarcales. Au début du \siecle{19} le Code Civil reprenait encore à son compte cette conception monarchique du rôle parental. Il n'imaginait pas un instant un fonctionnement familial démocratique mettant les époux à égalité et associant les enfants aux décisions qui les concernaient. Jusqu'à leur majorité  leurs parents avaient pleine autorité sur eux, parlaient pour eux, et si nécessaire le père, \emph{chef de la famille}, tranchait en dernier ressort. Ce n'est qu'au \siecle{20} que leur droit de contrôle sur le choix de leurs conjoints a cessé de s'exercer au-delà de leur majorité.
 
 
 

  Jusqu'aux années soixante du \siecle{20}, tout mari était le chef de sa famille et avait à ce titre autorité, sur ses enfants mineurs certes, mais aussi sur sa femme, puisque celle-ci avait abdiqué une bonne part de ses droits et capacités juridiques en se mariant. C'est lui qui détenait ces droits, et même s'il l'autorisait à les exercer, c'était sous sa propre responsabilité. Il donnait son nom à leurs enfants communs. Sauf contrat de mariage particulier c'est lui qui gérait tous leurs biens. Il signait seul la déclaration de revenus, et pouvait légalement laisser son épouse dans l'ignorance sur le montant des ressources du couple. Il était censé être le principal pourvoyeur financier même quand par son travail ou par sa dot son épouse contribuait autant ou plus que lui aux dépenses du ménage. Au nom de l'unité de commandement nécessaire à toute institution il pouvait lui interdire d'exercer un emploi salarié, d'ouvrir un commerce, de prendre une gérance en son nom propre, et même de posséder un compte en banque personnel. Lorsqu'elle travaillait avec lui elle était censée lui être subordonnée, et son travail était rarement reconnu et individualisé. 
Lorsque les deux époux s'entendaient suffisamment bien sur ce qui comptait pour eux, la réalité de leurs rapports et l'influence de chacun sur les décisions communes pouvaient être fort différentes du modèle que la loi définissait (sans compter les innombrables situations où l'épouse avait plus de caractère, de volonté ou d'esprit que son mari) et il en a toujours été ainsi, mais \emph{en cas de conflit entre eux} le mari avait la préséance et c'est à ce modèle hiérarchique et inégalitaire que se référaient les juges.

Tout cela étant dit il faut aussi rappeler que depuis Constantin (au plus tard) les femmes n'étaient plus considérées comme incapables ou déficientes par nature. Une femme célibataire majeure était \latin{sui juris} Si l'on en croit Aristote c'est comme un couple royal que les deux époux régnaient sur leur maison, et qu'ils exerçaient sur leurs enfants et leurs esclaves ce qu'il appelle la \emph{justice domestique}. Quand Thomas d'Aquin a réintroduit Aristote dans les universités du Moyen âge, il a repris cette doctrine du pouvoir royal des (deux) parents sur leur maison. Quand leur « seigneur et maître » venait à mourir les veuves retrouvaient le plein exercice de leurs droits personnels, ceux qu'elles lui avaient remis en se donnant à lui par le mariage. Une veuve non remariée remplaçait donc de plein droit son époux dans tous les actes juridiques ou commerciaux, comme dans l'exercice de l'autorité éducative sur leurs enfants communs.  

Jusqu'à ce qu'existent des méthodes fiables de contrôle des naissances, les grossesses extra conjugales étaient des catastrophes aux conséquences dévastatrices pour l'avenir des jeunes filles et pour les stratégies d'alliance de leurs familles. Celles-ci s'en protégeaient en contrôlant étroitement leurs corps. Il leur fallait entretenir un lourd appareil répressif, d'où les grilles, parloirs et clôtures, les portiers, eunuques, duègnes et chaperons, les ouvroirs et les gynécées, les institutions d'éducation fermées,~etc. À cela s'ajoutait la nécessité de la culture de « l'innocence », autrement dit de l'ignorance et du refoulement du désir. Cet appareil matériel et idéologique était source de violences, explicites et intériorisées, et de frustrations sans nombre. Il mettait à son service, outre les couvents et autres internats, toutes les représentations religieuses disponibles, de la sacralisation de la virginité à celle des souffrances et des frustrations subies. Il s'agissait que les filles à marier se maintiennent de manière ostensible et vérifiable dans une continence absolue, afin que personne ne puisse en douter. Leur hymen servait de sceau et prouvait leur « pureté », leur « vertu », leur « honneur ». Un viol les « déshonorait », les dépréciait irrémédiablement. Il fallait donc que les débordements masculins ne puissent pas s'exercer à leurs dépens et qu'elles prouvent leur capacité à s'identifier aux objectifs de leurs pères et mères, à ne pas se mettre en danger d'être « séduites », à contrôler et contenir elles-mêmes leurs pulsions sexuelles, et donc à montrer sur ce point leur capacité à gérer leur vie au lieu de la subir. 
 
 Même si les garçons étaient bien plus libres que leurs sœurs, il n'était pas question non plus qu'ils soient acculés à un mariage non voulu par un passage à l'acte inconsidéré dans les bras d'une jeune fille de leur monde (cf. le « coup du canapé » dont étaient menacés ceux des jeunes promis à une belle carrière ou à un gros héritage qui se montraient trop naïfs). Chez eux non plus un certain degré d'inhibition concourrait à une meilleure maîtrise de sa vie et un internat non mixte présentait bien des avantages, d'où la faveur dont jouissait cette formule éducative.
 
  
 
 Les travailleurs sociaux d'autrefois ont toujours su que même dans les « meilleures » familles il pouvait se passer des choses « pas très catholiques ». Durant tout le \siecle{19} les visiteurs des pauvres et les médecins n'ont pas arrêté de dénoncer la promiscuité des logements des pauvres et de plaider pour qu'à défaut d'une chambre par enfant il y ait au moins une chambre pour les garçons et une chambre pour les filles, avec un lit par enfant, et d'abord et avant tout une chambre pour le couple parental. Cela dépassait évidemment le seul souci d'hygiène. S'ils n'en disaient pas plus, tous comprenaient ce que ces euphémismes voilaient pudiquement, à savoir que l'excès de proximité, contraint ou choisi, conduisait souvent à la promiscuité sexuelle, et pas seulement dans les familles mal logées. Jusqu'à la seconde moitié du \siecle{20} la justice ne voulait rien entendre non plus des violences conjugales tant qu'il n'y avait pas de lésions physiques sérieuses
\footnote{En vertu de l'adage : \frquote{\emph{entre l'arbre et l'écorce il ne faut pas mettre le doigt.}}}. C'est par définition que la notion de \emph{viol conjugal} paraissait absurde.
 
Les paniers de linge sale et les cadavres des placards familiaux étaient protégés par la rigueur du secret absolu auquel étaient tenus médecins, ecclésiastiques, infirmières visiteuses,~etc. On croyait généralement que les victimes se remettaient aisément des sévices subis et qu'elles les « oubliaient ». On avait tendance à mettre systématiquement en doute leur parole, à leur supposer une complaisance coupable envers les actes subis (interprétant comme une acceptation la sidération qui s'observe si souvent chez les victimes), à minimiser les violences subies et leur retentissement sur leur état psychique ultérieur. Il aura fallu arriver aux années 80 du \siecle{20} pour que commencent de changer ces représentations.
 

 

 

 

 
