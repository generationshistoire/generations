% Le 20 mars 2015 :
% Grec, Romain
% Le 3 mars 2015 :
% Antiquité
% Moyen Âge
% ~etc.
% ~\%


\chapter{Les représentations des hommes de l'Antiquité}

\section{Des hommes et des dieux}

À part quelques esprits forts, les Grecs et les Romains sont convaincus de l'existence d'un monde surnaturel%
%[1]
\footnote{Paul \fsc{VEYNE}, \emph{L'empire gréco-romain}, 2005, notamment le chapitre 8, pages 419 à 544.}%
. Ils l'imaginent sous la forme d'un monde invisible où des dieux innombrables vivent à côté des hommes. À leurs yeux la nature est enchantée, pleine d'entités spirituelles de toutes espèces et imprégnée de sacré, et la communication avec les dieux est chose aisée et banale. Il suffit d'observer avec un minimum d'attention les signes qu'ils fournissent en abondance et de les interpréter avec compétence. Des spécialistes, (augures et haruspices à Rome, devins ailleurs) sont à la disposition des cités et des particuliers pour poser aux dieux des questions et pour interpréter leurs réponses.

 Les relations des dieux de l'Antiquité avec les hommes sont des relations de voisinage et non de parenté. Même si les hommes se doivent de les honorer, les dieux s'intéressent d'abord à leurs propres aventures et à leur propre monde. Ils aiment d'ailleurs les mêmes choses que les hommes. On les séduit donc comme on séduit ces derniers : par l'offrande de fêtes, de spectacles et de banquets. D'où les décorations des temples, les beaux vêtement (le blanc est la couleur liturgique principale) et les processions, les chants et les danses des jeunes filles, les concours de force et d'adresse des jeunes gens, l'offrande de bêtes sans défauts pour les sacrifices. 

 Les principaux cultes sont organisés et financés par les cités, les temples sont construits et entretenus par elles, même si le rôle de l'initiative privée est important (avec les omniprésents \emph{évergètes}%
 %[2]
\footnote{Cf. Paul \fsc{VEYNE}, \emph{Le pain et le cirque}, 1976.}%
 ). L'action politique implique le service des dieux. Même si une pureté rituelle est ordinairement requise pour exercer les fonctions cultuelles (prêtrise, haruspice,~etc.), leurs exigences sont ordinairement faciles à satisfaire. Sauf dans les cultes exotiques importés d'Égypte ou d'Asie (mineure) il n'est pas question de mettre à part des officiants pour en faire un clergé professionnel permanent. Au cours de sa carrière politique (de son \latin{cursus honorum}) chaque notable est appelé à tenir un jour ou l'autre la place du célébrant d'un des cultes autorisés, mais seulement pour un temps, comme pour toutes les autres fonctions civiques. Au même moment chaque père de famille est le prêtre et le sacrificateur de son culte familial. 

 Les festins sont indissociables de la religion civique%
 %[3]
\footnote{... en Grèce du moins. Les viandes non brûlées et non consommées par les prêtres étaient vendues comme viandes de boucherie. Il semble que les anciens, dont le niveau de vie moyen était celui des habitants des pays sous-développés d'aujourd'hui, aient consommé très peu de viande, si bien qu'il se peut que les banquets religieux civiques aient été pour les plus humbles des Grecs les seules occasions où ils en mangeaient.}%
. Le plus souvent les non citoyens en sont exclus, et encore plus strictement les esclaves. Lors des fêtes civiques les citoyens célèbrent leur unité en communiant à la chair des victimes pendant que la part des dieux (la graisse) monte vers ces derniers avec la fumée. C'est aussi par des sacrifices, sanglants ou non, et des conduites de repentance qu'on cherche à obtenir des dieux le pardon des fautes commises à leur égard. 

 Mais il ne faut s'abuser ni sur leur puissance, ni sur leur sollicitude, ni sur la constance de leurs intérêts pour les humains. Les dieux ont à peu près autant de respect et de sympathie que les hommes pour les conduites vertueuses, c'est-à-dire qu'il leur arrive de penser à autre chose, ou d'être à l'occasion injustes, menteurs, ou infidèles. On ne peut donc pas compter sur leurs interventions en faveur du droit et de la justice, même s'il n'est pas interdit de garder espoir. Il faut s'attacher leur bienveillance comme on cultive celle d'un voisin puissant. Il faut éviter de les avoir contre soi. Le vrai problème c'est donc de les connaître, de les hiérarchiser et de vivre en bonne entente à côté d'eux sans en vexer aucun, d'où les autels « \emph{à l'ensemble des dieux} », ou bien « \emph{au dieu inconnu} ». 

 Si bien des dieux sont attachés à un lieu, comme la déesse Athéna à Athènes, chaque cité a la liberté de rendre un culte à tous les dieux de son choix, et chacun peut rendre un culte à ses dieux préférés. On pense d'ailleurs souvent que sous des noms différents il s'agit des mêmes dieux. L'idée de liberté religieuse est pourtant incompréhensible puisque chacun croit à tous les dieux ... même à ceux de ses ennemis, et que le monde des dieux ne peut pas être dissocié du monde civique. Chacun sait que la négligence de ses devoirs religieux fait courir des risques à la cité, et qu'il n'est donc pas question de l'accepter. En ce sens-là on pourrait dire qu'il n'existe qu'une seule religion, à l'intérieur de laquelle toutes les dévotions individuelles sont possibles et compatibles. Si les dieux des vainqueurs sont les plus forts, ce que leur victoire démontre, il n'en est pas moins absurde et imprudent de prétendre que ceux des vaincus n'ont pas ou plus d'importance. C'est pourquoi les dieux d'une cité vaincue sont respectueusement invités à déménager chez les vainqueurs. 

 D'où viennent les dieux ? Ils n'ont pas créé le monde puisqu'ils ont eux-mêmes été créés. Ils en font partie au même titre que les humains, même s'ils sont plus puissants, et immortels. Quant à l'origine du monde le champ des spéculations est ouvert aux philosophes. Beaucoup comme Platon ou Aristote postulent qu'il existe un être créateur de tout ce qui existe \emph{(premier moteur)}, ce qui ne dit rien de ses autres attributs. D'autres pensent que la matière est éternelle et incréée. Mais en dehors du cercle des philosophes ces idées ont peu d'influence. 

 Les anciens Grecs et Romains ne doutent pas de l'existence d'une âme individuelle (au moins une par personne, souvent plusieurs). Ils croient en une survie, au moins partielle, de cette âme. Ils croient en l'existence d'un autre monde pour les morts, ordinairement localisé sous terre : les Enfers, l'Hadès,~etc. Le destin des âmes des morts est misérable quelle qu'ait été leur valeur humaine : ce ne sont plus que des ombres qui se traînent mélancoliquement dans les Enfers, leur séjour lugubre où n'entre pas le soleil, et où ils peuvent finir par se dissoudre. S'il existe aussi une forte tendance populaire à vouloir que les méchants subissent après leur mort des punitions appropriées à leurs forfaits, cette tendance n'emporte pas la conviction : c'est peut-être prendre ses désirs pour la réalité.

 Les anciens croient que les défunts peuvent au même titre que les dieux exercer une certaine influence sur les vivants et leur destin. Ils croient aussi que la qualité de la survie des âmes des morts dépend des soins que prennent leurs descendants à les honorer, d'où l'importance du culte qui leur est rendu. Quand ils leur font des libations et leur offrent de la nourriture ou des sacrifices cela les ranime un peu, et diminue d'autant leur envie de nuire à ceux qui jouissent encore de la vie. Toute famille pieuse garde donc la mémoire de ses ancêtres sur plusieurs générations. Pour éviter leur courroux et une éventuelle punition individuelle qui retentirait sur la collectivité il convient que chacun prenne soin de ses morts et des dieux de sa maison (dieux \latin{lares} attachés au foyer, et plus précisément à l'âtre où se faisait la cuisine). La cité exerce donc un droit de regard sur l'exercice du culte domestique et prend elle-même en charge celui des familles disparues sans héritier. La négligence de ces devoirs familiaux et religieux (qui constitue \emph{l'impiété}) est sanctionnée par la loi%
%[4]
\footnote{À côté de la religion traditionnelle, civique et donc obligatoire, existaient des \emph{cultes à mystère} dans lesquels l'investissement individuel et l'implication affective étaient au premier plan. Concernant l'au-delà de la mort ils promouvaient des perspectives plus exaltantes que les enfers lugubres de la religion traditionnelle. Ils n'étaient pas vécus comme incompatibles avec cette dernière et n'exigeaient pas de profession de foi exclusive.}%
. 


\section{La vie bonne (le bonheur)}

 À l'exception des stoïciens et des juifs, les hommes de l'Antiquité sont unanimes dans leur mépris pour le travail d'exécution, et d'abord pour le travail des mains, celui qui « sent la sueur ». L'opinion des plus grands philosophes grecs, au premier rang desquels Platon et Aristote, est que l'on ne peut pas être un homme de bien, c'est-à-dire cultiver la vertu, si l'on travaille de ses propres mains. Ce travail est digne des seuls esclaves, et disqualifie ceux qui le font. Même les artistes les plus admirés ne sont que des artisans, des gens sans dignité, à peine plus estimables que les tâcherons sans qualification. N'importe quel propriétaire est plus noble, du fait qu'il dirige le travail des autres.

 Les sociétés antiques apprécient la richesse sans mauvaise conscience. La fortune matérielle est la preuve qu'on est béni des dieux. De là à penser que la richesse prouve la valeur personnelle il n'y a qu'un pas. L'idéal de l'homme antique est de vivre de ses rentes. Cet idéal n'interdit d'ailleurs aucunement de faire des affaires, de combiner des coups financiers, de prêter de l'argent ou de spéculer. Mais seule la possession de la terre donne droit à l'exercice plénier des droits du citoyen. Elle est le signe du lien charnel à la patrie, à la terre des pères, à la terre mère de la patrie, d'où l'interdiction faite aux étrangers et aux non citoyens de l'acheter ou d'en hériter. Celui qui possède un domaine peut à volonté se couper des circuits de l'échange et vivre en autarcie des seuls produits de sa terre. Il est le plus libre des hommes et donc le plus digne de confiance. Seuls les riches accèdent aux honneurs civiques, en fonction du niveau de leur fortune : cela ne suffit pas, mais c'est une condition nécessaire%
% [5]
\footnote{Cf. \fsc{VEYNE}, \emph{Le pain et le cirque}.}%
.

 Tout citoyen se doit de soutenir son rang noblement, en particulier s'il veut jouer un rôle politique. L'exercice d'une autre profession que celle de propriétaire terrien fait déroger, sauf le métier d'avocat, de médecin (dont la promotion a été lente et progressive, à Rome les premiers d'entre eux étaient des esclaves grecs), de professeur (enseignement secondaire et supérieur), d'artiste (ceux qui ont réussi : peintre, sculpteur, architecte, mais pas acteur ou chanteur), et les activités de banquier, armateur ou commerçant (à la condition qu'ils soient riches). Celui qui exerce une profession \emph{libérale}, une profession qui ne fait pas déroger, par opposition aux métiers serviles, est censé fournir gratuitement ses services à ses amis et relations, qui sont censés lui témoigner leur reconnaissance par des cadeaux « spontanés » qu'on appelle \emph{honoraires}. Cela dit s'il atteint à la reconnaissance sociale c'est moins pour ses compétences techniques que pour les richesses qu'elles lui procurent. 

 Le statut de travailleur libre, \emph{mercenaire}, c'est-à-dire salarié, existe depuis la plus haute Antiquité. Bien que non infâme, ce statut n'est pas plus honorable ni plus désirable que celui d'un manœuvre d'aujourd'hui. Celui à qui son manque de fortune interdit de vivre du travail des autres et qui ne dispose pas de compétences particulières, doit se louer pour gagner son pain. Son activité concrète n'est pas différente de celle des esclaves. C'est donc un métier \emph{servile} qui le dévalorise. Le fait qu'il choisisse son maître et ne lui soit pas lié au-delà de son contrat est la preuve de sa liberté, mais il n'en reste pas moins un pauvre parmi les pauvres. 


\section{Piété et solidarité familiale}

 Dans un monde sans systèmes d'assurance, sans assistance mutuelle et sans retraites, l'individu est solidaire du groupe auquel il appartient. Seule la famille est tenue de fournir de l'aide à ses membres : tous les membres d'une famille se doivent mutuellement secours et assistance. Cela fait partie de la \emph{piété filiale}, dont nul ne peut s'estimer quitte sauf les enfants émancipés, abandonnés ou vendus comme esclaves. Cela explique la soumission des enfants à leurs parents : ils ne peuvent compter que sur ces derniers pour défendre leurs droits personnels, pour subvenir à leurs besoins et pour les établir dans la vie. Chez les Romains les gains de tout enfant, quel que soit son âge, reviennent à son \latin{pater familias} jusqu'à la mort de ce dernier. Les plus pauvres ne possèdent aucun capital et ne vivent que du travail de leurs propres mains. Leurs enfants (\latin{proles} : d'où leur nom de \emph{prolétaires}) sont leur seule richesse. Ils ont le droit d'exiger (si nécessaire par voie de justice) que les enfants ou petits-enfants \emph{qu'ils ont élevés et établis} les prennent matériellement en charge dès qu'ils sont assez grands pour travailler, et surtout quand ils sont eux-mêmes trop âgés et trop faibles. Ce n'est qu'à cette condition qu'ils peuvent vieillir dignement. 

 Dans le monde des républiques gréco-romaines on ne peut pas parler d'assistance au sens où nous l'entendons aujourd'hui. L'hospitalité est un devoir impérieux, mais il va sans dire que c'est à charge de revanche. Quant à ceux qui ne peuvent \emph{rendre} aucun service (malades, infirmes, vieillards,~etc.) ils ne peuvent compter que sur la pitié du public, dans un monde où celle-ci n'est pas plus valorisée que dans le nôtre.

 Personne ne soutient sérieusement qu'un pauvre sans ressources et sans travail est fondé à revendiquer des secours au nom du seul fait qu'il est un homme. S'il est citoyen, et seulement dans ce cas, on admet qu'il a un droit moral à demander des secours, mais à sa seule cité. Celle-ci a en effet intérêt à conserver ses citoyens pauvres, eux et leurs enfants légitimes qui sont mobilisables comme soldats, plutôt qu'à les voir soustraits à sa juridiction s'ils sont asservis. De ce fait les cités se sentent concernées par les difficultés de leurs membres et en cas de disette tentent de fournir des aliments à leurs concitoyens affamés. Les pauvres peuvent aussi rechercher la protection et le soutien des riches, que ceux-ci soient motivés par leur bon cœur ou par la crainte du scandale que provoquerait leur indifférence. Ce soutien est institutionnalisé dans la pratique du \emph{patronage}. Par ailleurs \emph{l'évergétisme} désamorce l'envie des pauvres en obligeant les citoyens riches à financer les budgets des cités et à offrir de somptueux cadeaux à leurs concitoyens moins fortunés pour obtenir leurs votes.

 Par contre les étrangers domiciliés n'ont aucun recours de ce genre à espérer. Au mieux ils peuvent compter sur ceux de leurs propres concitoyens qui vivent dans la même cité, d'où l'importance des associations d'étrangers. S'ils ne payent pas une mensualité de leur taxe de séjour les métèques d'Athènes sont vendus comme esclaves, même s'ils l'ont acquittée scrupuleusement pendant des années.

 En bien des cités, dont Rome, des médecins et des enseignants sont recrutés et payés par la cité et soignent ou enseignent : mais si l'état de besoin donne droit à ces services, il est réservé aux citoyens. 

 Quand les légionnaires romains ont achevé leur engagement ils reçoivent de l'argent ou des terres. Il s'agit là de rémunérer les services rendus par un capital retraite. Si les enfants d'un citoyen mort au combat sont élevés aux frais de la collectivité, c'est parce que l'ensemble des citoyens survivants a contracté une dette envers leur père, dette que la cité honore ainsi, ne serait-ce que pour que les futurs pères de famille continuent d'accepter de mourir pour elle. Dans un état organisé qui se respecte les militaires en retraite et les orphelins de guerre ne peuvent pas être confondus avec les indigents, même s'ils vivent dans une certaine austérité : ils personnalisent en effet la vertu, le courage et le dévouement à la cité. Ils sont donc particulièrement honorables, au contraire des métèques, des indigents et des vagabonds, et il ne s'agit pas d'assistance.

 Il serait pourtant injuste d'oublier les institutions vouées aux malades, établies %constituées 
autour d'un temple aux dieux de la médecine (\emph{aesculapium} en Grec ou \latin{valetudinarium} en latin) qui sont accessibles à tous, esclaves compris, mais qui tiennent plus du lieu de pélerinage que de l'hôpital. 

 Quant aux infirmeries et hôpitaux fonctionnant au sein des unités militaires ou des entreprises utilisant de nombreux esclaves (jusqu'à \nombre{500} lits), ce ne sont pas des institutions d'assistance, mais des outils au service d'une bonne gestion d'un capital humain précieux. Ils n'en fournissent pas moins à leurs médecins (qui ont très longtemps été des esclaves formés sur le tas, comme les autres artisans) un cadre adapté à la réalisation d'observations systématiques et à l'acquisition de compétences professionnelles.


\section{Morale d'esclave}

 Du point de vue des citoyens des cités antiques l'esclavage est regrettable pour ceux qui y sont assujettis, pour ceux que les dieux n'ont pas protégé d'un destin si funeste, mais ils voient de leurs propres yeux les bénéfices très concrets qu'ils en tirent. Le miracle économique et culturel grec reposait d'abord sur l'efficacité avec laquelle les libres citoyens des cités grecques avaient su durablement asservir leurs voisins et les mettre au travail forcé dans les industries lourdes de l'époque. Cela leur avait permis sans effort financier exorbitant de s'équiper des matériels militaires les plus coûteux et les plus performants, et leurs guerres leur avaient fourni de nouveaux esclaves : le travail des esclaves payait l'achat de nouveaux esclaves. 

 Un monde sans esclaves n'est alors pas pensable : si l'on veut éviter de travailler soi-même, ce qui selon les philosophes est tout de même la moindre des choses, il faut bien se résoudre à asservir d'autres hommes. Pour Aristote il est nécessaire, il est inévitable, et donc conforme à la nature, qu'il y ait des personnes faites pour commander et d'autres faites pour obéir. Il convient seulement de savoir qui peut légitimement être soumis par la violence, et qui ne doit pas l'être. Il s'agit de désigner correctement ceux à qui une valeur égale à soi-même peut à bon droit être déniée. 

 Quels sont ceux qui ont vocation à être esclaves ? Les anciens pensent qu'il n'est finalement pas injuste, qu'il est même légitime de traiter comme des esclaves ceux qui ont assez de mépris pour eux-mêmes pour se reconnaître comme tels. Ils considèrent que l'asservissement ne peut être regardé comme moins grave que la mort que par ceux qui tiennent d'abord à la vie, c'est-à-dire à peu près tout le monde sauf les aristocrates. Les autres, les aristocrates et ceux qui méritent de l'être, ne se laissent pas réduire en esclavage : ils meurent au combat, ils s'évadent, ils se suicident, ils se laissent mourir de faim. Ceux qui acceptent de survivre à leur asservissement signent par là leur renoncement au statut d'homme libre, au droit à la parole. 

 L'existence de l'esclavage n'est pas perçue comme un scandale. D'ailleurs les esclaves peuvent acheter des esclaves, même si légalement ces derniers appartiennent à leur propre maître, comme tout ce dont celui-ci leur laisse la jouissance. Même les esclaves révoltés n'espèrent rien de mieux que de posséder des esclaves à leur tour. C'est qu'il y a pire : vaincus égorgés sur le champ de bataille ; cités entièrement passées au fil de l'épée, femmes et enfants compris ; chômeurs qui meurent réellement de faim ; infirmes et malades (esclaves ou libres) abandonnés dans les temples aux bons soins des dieux, c'est-à-dire à la pitié du public ; enfants chétifs ou mal formés abandonnés sans aucune chance d'être accueillis par quiconque ; enfants non désirés tués à la naissance,~etc.

 En se soumettant à leur maître les esclaves montrent qu'ils tiennent plus à la vie qu'à la liberté et à l'honneur, ce qui implique qu'ils n'ont pas de volonté propre ni de courage. C'est pourquoi leur parole compte pour rien devant la justice. La morale qu'on leur accorde se résume en peu de mots : puisque leur maître est tout-puissant, leur devoir se borne à faire tout ce qu'il veut : \emph{il n'y a pas de honte à faire ce que le maître commande}. C'est une dénégation puisqu'ils sont justement affectés aux prestations qu'il est honteux à tout autre de fournir. Ils doivent se laisser utiliser par autrui comme un moyen, et surtout ne jamais montrer de fierté : chez eux c'est de l'insolence, une arrogance insupportable, et ils reçoivent des coups s'ils s'y risquent. C'est de la docilité qu'on attend d'eux, ou plus exactement de la \emph{servilité}. C'est pour elle qu'on les a achetés ou élevés. 

 Le paradoxe c'est évidemment qu'on les méprise justement à cause de cette servilité. On exige d'elles et d'eux une soumission qui est infamante en elle-même%
% [9]
\footnote{\emph{Les Bas-Fonds de l'Antiquité}, Catherine \fsc{SALLES}, 1982.\\
\emph{La société romaine}, Paul \fsc{VEYNE}, 1991.\\
\emph{La contamination spirituelle, science, droit et religion dans l'Antiquité}, Aline \fsc{ROUSSELLE}, 1998.}% 
. Ils sont donc dans une position morale impossible : ils ne peuvent survivre que s'ils acceptent de faire ce qui est une honte pour tous leurs contemporains. Ce n'est que s'ils mouraient héroïquement qu'on leur reconnaîtrait avec étonnement, mais un peu tard, le courage et la dignité d'hommes (de femmes) véritables : \emph{un homme libre dans un corps d'esclave}. 


\section{Vertu virile}

 \latin{Virtus} vient de \latin{vir} qui désigne l'homme par opposition à la femme, le mâle. Ce mot signifie « vertu, courage, force ». La vertu antique s'identifie à la force, au courage physique et à la maîtrise de soi. C'est le contraire de la faiblesse, de la passivité, de la mollesse et de la lascivité, tous traits de caractère prêtés aux femmes, et par extension aux homosexuels passifs ou aux esclaves. 

 Le maître d'une \latin{familia}, qui vit constamment au milieu d'esclaves plus ou moins dociles, et parfois très nombreux, ne doit pas craindre de s'imposer, de faire peur et de faire mal quand il le juge nécessaire, ni de faire couler le sang. En un mot il doit être ce que l'argot des entreprises d'aujourd'hui appelle un « tueur ». Dans ce monde c'est la force des forts et leur mépris de la mort (celle des autres et la leur) qui légitiment l'asservissement des moins forts, d'où l'importance « pédagogique » des jeux du cirque, de la mise à mort de l'autre comme spectacle%
% [6]
\footnote{\emph{À Pompéi, sur une fresque illustrant l'accomplissement de tâches domestiques et artisanales, l'artiste a substitué aux vilains esclaves qui les remplissent habituellement de doux et souriants angelots. Les Pompéiens vivaient un mythe où tout leur paraissait venir du ciel en gratification naturelle et méritée de leur raffinement obtus. Comme toutes les exploitations, l'esclavage ne conduit pas qu'à l'aliénation des exploités, mais aussi à celle des exploiteurs. Il conduit à la négation de l'humanité des hommes et des femmes, à leur mépris et à la haine. Il incite au racisme, à l'arbitraire, aux sévices et aux meurtres purificateurs, armes caractéristiques des guerres de classe les plus cruelles. Si tant est que l'esclavage ait contribué à un quelconque progrès matériel, il nous a aussi légué comme maîtres à penser des philosophes et des politiques dont la conscience était le produit de cet aveuglement et de ces préjugés. N'est-ce pas parce qu'elle s'est communiquée jusqu'à nous, portée par une culture indiscutée et ininterrompue d'exploiteurs, que leur aliénation nous demeure toujours imperceptible et nous donne encore pour humanistes des sociétés construites sur le saccage de l'homme.} \fsc{MEILLASSOUX}, \emph{Anthropologie de l'esclavage} (1986, 1998, p. 321).}%
.

 La dureté et l'insensibilité sont les qualités les plus nécessaires aux chasseurs d'esclaves. Il ne faut pas qu'ils lésinent sur les forces à mettre en œuvre s'ils veulent capturer leur « gibier » sans l'abîmer ni le déprécier, sans qu'il ait le temps et la possibilité de se blesser lui-même pendant la capture, ni celle de se suicider dans les moments de désespoir qui suivent : il leur faut livrer leurs proies aux « consommateurs » en suffisamment bon état pour les vendre avec profit. Et il faut « assouplir » leurs capacités de résistance psychologique, ce qui ne se fait pas au moyen de caresses. Afin que la victime consente à s'identifier \emph{servilement} aux désirs d'un maître il convient d'induire en elle un niveau suffisant de dépersonnalisation et de sidération des défenses, à coup d'humiliations et de sévices terrorisants. 

 La chasse aux esclaves est un sport qui demande de ne pas avoir froid aux yeux, c'est un « travail d'homme » qui permet à ceux qui s'y livrent de faire la preuve de leur virilité, telle qu'on la conçoit alors. Elle leur permet de faire la preuve de leur fécondité. En effet ceux qui savent comment fabriquer des esclaves augmentent le nombre des individus qui dépendent d'eux, et donc leur propre poids social, sans avoir à en passer pour ce faire par la volonté d'un beau-père et le sexe d'une épouse. 

 En Grèce et à Rome face aux femmes, aux enfants et aux esclaves des deux sexes, définis par leur soumission et leur passivité, l'homme se définit par son activité. Dans ce système de représentations la fidélité du mari à son épouse n'a pas de sens. Elle ne regarde que lui. Il est en droit de compter sur l'obéissance sans limite de ses esclaves et sur la complaisance de ses affranchis et de leurs épouses. Rien ne lui interdit sexuellement les impubères, du moment qu'ils lui appartiennent. Seuls lui sont interdits ses propres enfants et les membres trop proches de sa lignée, ainsi que toutes les personnes qui dépendent d'un autre citoyen : leurs enfants, leurs épouses ou leurs esclaves.

 Ce que l'esprit du temps trouve le plus répugnant, le plus avilissant, c'est qu'un homme libre de tout lien de dépendance -- ni esclave, ni affranchi -- se mette au service du plaisir sexuel de quelqu'un d'autre, homme ou femme, soit qu'il pratique le cunnilingus, soit qu'il subisse de son plein gré « \emph{ce que l'on fait aux femmes} » : des rapports homosexuels passifs%
% [7]
\footnote{\fsc{VEYNE}, 1991; \fsc{ROUSSELLE}, 1998. En Grèce tout homme fait qui avait été victime d'un viol, par exemple pendant qu'il était prisonnier de guerre, perdait à tout jamais ses droits civiques. La passivité sexuelle, même subie et non assumée, n'était chez eux admise que jusqu'à ce que la barbe ait poussé. À Rome par contre une fois libéré celui qui avait subi un viol pendant sa captivité retrouvait l'intégralité de ses droits civiques.}%
. Le soupçon qu'un citoyen ait du goût pour ces pratiques est gravissime. Si la preuve en est faite il perd son droit à la parole publique et devient infâme avec toutes les conséquences légales liées à ce statut. Les armées romaines prohibent vigoureusement ces comportements chez leurs membres. Une telle accusation peut pousser au suicide. Par contre les pratiques homosexuelles masculines \emph{actives} ne soulèvent aucune objection morale, pas plus d'ailleurs que les pratiques homosexuelles des femmes. En ce qui concerne ces dernières c'est parce que ce qu'elles peuvent faire n'a aucune importance sociale tant qu'aucun homme n'y est impliqué. Il s'agit donc d'une morale \emph{machiste} sans nuances ni aménagements.

 L'infamie est une peine accessoire entraînée par certaines condamnations judiciaires \emph{(peines infamantes)}. C'est aussi la sanction automatique de certains actes et de certaines activités : le fait de coucher avec la femme d'autrui, de choisir volontairement de subir « ce que l'on fait aux femmes », de travailler dans le monde du spectacle, des jeux du cirque, ou dans le monde de la prostitution. Le service d'autrui en tant que tel fait problème. Se mettre au service du plaisir d'autrui est honteux, particulièrement pour les hommes. Sans même parler des proxénètes et des organisateurs de spectacles \latin{(leno)}, regardés avec un mépris sans nuances, les acteurs, même célèbres et adulés, sont aussi infâmes que les prostitué(e)s libres et les gladiateurs libres.

 L'infamie retire à l'intéressé(e) ses droits civiques et parentaux, invalide (rompt) son mariage, lui interdit de passer contrat et de se marier à l'avenir. Aux hommes elle interdit de prendre la parole dans une assemblée politique, de porter plainte comme de plaider pour un autre qu'eux-mêmes (comme avocat), d'être magistrats, d'exercer les prérogatives d'un \latin{pater familias}, d'un patron, ou la tutelle d'un mineur. De ce fait les esclaves affranchi(e)s par un infâme sont libres des obligations légales des clients, auxquelles tous les autres affranchis sont assujettis.

 Les affranchies qui ont été prostituées par leur maître pendant leur servitude mais qui ne se prostituent plus depuis leur affranchissement échappent à l'infamie, et elles n'ont pas d'obligation face à leur ancien proxénète. Ce n'est pas le cas des hommes affranchis qui ont été gladiateurs ou prostitués (ou au service d'un proxénète comme tenancier ou employé de maison close) avant leur affranchissement : ceux-là demeurent personnellement infâmes quoi qu'ils fassent, même retirés des jeux sous les applaudissements du public (pourquoi sont-ils marqués à vie au contraire de celles-là ?). 

 L'infamie est héréditaire et se transmet aux enfants de l'intéressé(e) nés après l'acte qui l'a entrainée, ou après la condamnation à une peine infamante.

 Le suicide n'est pas l'objet d'un jugement moral de condamnation. Si la vie devient sans intérêt, si le déshonneur est imminent et inévitable, si l'on se retrouve dans une impasse existentielle, quel qu'en soit le motif, alors le suicide est une porte de sortie honorable. Dans ce cas les médecins se font un devoir de fournir une aide compétente. 


\section{Pudeur féminine}

 Pour les philosophes et les médecins gréco-romains les femmes sont des hommes incomplets, des hommes ratés%
% [8]
\footnote{Ce point de vue a été universel et indiscuté jusqu'au \siecle{18}. Georges \fsc{DUBY}, Michelle \fsc{PERROT}, \emph{Histoire des femmes en occident, I, l'Antiquité}, 2002, chapitre 2 : \emph{philosophies du genre}, Giulia \fsc{SISSA}, p. 83-124.}%
 : elles sont définies par le manque. Elles sont \emph{celles qui n'ont pas} (d'où leur désir pour les hommes et ce dont ils sont pourvus).

 Les veuves sans famille, sans fortune et chargées de petits enfants (orphelins de père), les épouses répudiées, les concubines abandonnées sont l'archétype de la faiblesse, de l'impuissance et de la pauvreté quand elles sont âgées et sans enfants. 

 Infâmes sont les femmes \emph{de mauvaise vie} : les prostituées libres, les comédiennes, les danseuses, chanteuses, musiciennes,~etc. Les femmes convaincues d'adultère sont elles aussi condamnées à l'infamie. À Rome aucune de ces femmes n'a droit à la protection légale de leur corps qu'offre le \emph{manteau des matrones}. Elles portent une toge comme les hommes, ou tout autre vêtement de leur choix, et elles sont punies si elles osent porter le manteau des matrones. Qu'elles le veuillent ou non leur vêtement affiche la disponibilité de leur corps. Les femmes esclaves non plus n'ont pas droit au manteau des matrones, ni au vêtement des enfants et adolescents \emph{de famille}.

 Même citoyennes, les femmes n'ont pas de liberté de choix : que les hommes dont elles dépendent les marient ou les démarient, que leurs époux les répudient ou leur refusent le divorce, de toute manière elles font ce qu'on leur dit de faire. Qu'elles aient une vie sexuelle ou que cela leur soit refusé ne dépend pas d'elles. Il en est de même pour le fait de conserver leurs enfants ou de les abandonner, d'avorter ou de ne pas avorter. Leur corps n'est pas à elle. Elles se doivent d'être dociles, soumises et dévouées aux objectifs de leur maître du moment. La notion de viol marital n'a aucun sens : elles ne peuvent être violées par leurs maris puisqu'ils sont maîtres de leurs corps.

 Si elles subissent un viol on s'attend à ce qu'elles se suicident. Ce n'est pas une obligation légale, mais c'est une sorte d'obligation morale. L'histoire (ou la légende) de Lucrèce (vers 500 avant J.-C.) donne l'exemple d'une conduite digne d'être admirée chez une femme mariée qui a été violée :

\begin{displayquote}[\fsc{Tite-Live}, \emph{Histoire romaine}, Livre~I, trad. Annette \fsc{Flobert}.]
\emph{Ils trouvèrent Lucrèce assise dans sa chambre, accablée de chagrin. Elle se mit à pleurer en voyant arriver les siens. Son mari lui demanda si elle était souffrante : « Oui, répondit-elle ; comment une femme qui a perdu son honneur pourrait-elle bien se porter ? Un homme \emph{[...]} a souillé ta couche ; on m'a fait violence, mais mon cœur est resté pur : ma mort en fournira la preuve. Prenez ma main et jurez de punir mon déshonneur. Sextus Tarquin m'a fait violence ; il est venu la nuit dernière avec une arme, non comme un hôte mais comme un ennemi et il est reparti après avoir pris un plaisir dont je meurs et dont il mourra aussi si vous êtes des hommes ». Ils promirent tous, l'un après l'autre. Ils cherchèrent à apaiser son tourment, affirmant que le coupable n'était pas la victime mais l'auteur de l'attentat ; c'était l'intention et non l'acte qui constituait la faute. « Fixez vous-mêmes le prix qu'il doit payer ; pour moi, bien qu'innocente, je ne m'estime pas quitte de la mort. Jamais une femme ne s'autorisera de l'exemple de Lucrèce pour survivre à son déshonneur. » Elle plongea dans son cœur un couteau qu'elle tenait caché sous son vêtement et tomba sous le coup, mourante.}
\end{displayquote}

 Ce récit définit l'honneur comme le fait de n'avoir subi dans son corps aucune effraction non autorisée par une autorité légitime (son \latin{dominus} pour les femmes). Les victimes féminines d'un viol qui, au contraire de Lucrèce, choisissent de ne pas se suicider (les plus nombreuses ?) courent le risque d'être considérées comme adultères (si elles n'ont pas crié et ne se sont pas débattues avec assez de force) ou comme accessibles à tous les hommes comme le sont les filles publiques. Au contraire celles qui se suicident restaurent activement leur honneur perdu, au prix d'un sacrifice dont elles sont le sacrificateur et l'offrande. En protestant par leur acte de leur refus sans concession de l'offense subie {\emph{(plutôt mourir que de vivre dans un corps ainsi souillé)}} elles se mettent définitivement à l'abri des récriminations et autres représailles futures de leurs proches, et les mettent en demeure de les venger, alors que les proxénètes font tout pour que leurs protégées ne se plaignent pas des sévices subis dans l'exercice de leur métier, et acceptent de continuer de se prostituer. 
 
En ce qui concerne les hommes adultes victimes de viol le suicide ne restaurerait pas l'image perdue. Seule la vengeance à l'encontre du violeur, autrement dit sa mort, serait suffisante. 


