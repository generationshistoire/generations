
\chapter{Victoire du mariage d'amour}


 Jusqu'au \siecle{19}, le premier objectif des jeunes gens raisonnables n'était pas tant de vivre mieux que leurs parents et de s'enrichir, que de réussir au moins à reproduire le même mode de vie qu'eux, et de ne pas tomber dans l'indigence. Pour cela, ils n'avaient le choix qu'entre un mariage arrangé par leurs parents, ou pas de mariage du tout, et si le plus souvent (pas toujours) il valait mieux à tous points de vue se marier que ne pas le faire, il leur fallait aussi éviter de compromettre, par enthousiasme naïf, par imprudence ou par sottise, les bases économiques de leur futur couple et le statut social de leurs enfants à venir. 

 Selon les moralistes d'alors le choix du mariage d'inclination, fondé sur l'amour passion et non sur la raison (c'est-à-dire l'intérêt) était la marque des imprévoyant(e)s. Entre mariage d'inclination et concubinage les liens paraissaient évidents. C'est ainsi que s'unissaient ceux qui ne possédaient que leurs bras, les ouvriers, les manœuvres, les valets, les ouvrières et les servantes, etc. Ceux qui se mettaient en ménage avant d'avoir « assis » leur « situation » se condamnaient à « tirer le diable par la queue ». Selon les mêmes moralistes, avec lesquels faisaient chorus tous les parents angoissés (et dans beaucoup de moralistes, même célibataires, il y a un parent angoissé), la soumission des jeunes imprévoyants à leurs appétits charnels leur faisait courir le risque de gâcher leur vie, de connaître la misère et de perdre un jour la main sur leurs propres enfants, ainsi qu'il en avait toujours été depuis le début du monde. 

 Ils risquaient en effet de ne pas pouvoir les élever et de devoir les abandonner aux institutions d'assistance. Ils ne pourraient les « établir », ni en leur donnant un capital matériel, ni en finançant leur apprentissage professionnel auprès d'un maître qualifié, ni en les mettant à l'école, même gratuite, puisqu'ils seraient contraints de les placer chez un maître dès que leur âge le permettrait. En cas de chômage et de disette, ils les enverraient mendier. Ils ne pourraient pas compter sur ces enfants, condamnés à être pauvres à leur tour, pour soutenir leur propre vieillesse. Ils risquaient de finir leurs jours dans la solitude et la misère, affective et matérielle, des hospices.

 Au contraire les parents prévoyants établissaient leurs enfants dans un mariage profitable grâce à leurs économies, à leurs relations et à des stratégies complexes : échanges simultanés et réciproques d'enfants, de terres, de droits d'exploitation, d'entreprises, de gérances, d'offices (ministériels), etc. sans compter jusqu'au \siecle{18} l'entrée en religion plus ou moins volontaire de ceux qu'ils ne pouvaient ou ne voulaient pas marier de manière conforme à leur milieu social. 

 Ces stratégies complexes ne pouvaient pas toujours tenir compte des préférences sexuelles ou amoureuses de chacun, et on n'en faisait pas grief aux parents. Les femmes s'en consoleraient avec leurs enfants ou la religion, les hommes avec le travail, le pouvoir, les prostituées ou les maîtresses (le recours à celles-là et aux "maisons closes" étant toujours préférable, du point de vue des épouses, au choix de celles-ci). Les patrimoines étaient verrouillés contre les effets des infidélités des uns et des autres. Une épouse ne pouvait introduire d'enfant adultérin dans sa famille que si son mari le voulait bien, mais en ce cas la paternité de celui-ci devenait absolument inattaquable : le géniteur n'avait aucun recours. Quant aux enfants illégitimes du mari, ils ne pouvaient en aucun cas être légitimés ni menacer l'héritage des enfants légitimes. Les épouses pouvaient dormir tranquilles (sur ce point en tout cas) même quand leurs maris découchaient. 

 La pérennité des couples raisonnables était favorisée par la synergie des ressources que leurs familles respectives avaient sagement et laborieusement conjointes. Leurs parents étaient les premiers à tenir fermement à ce qu'ils, et elles plus encore, ne mettent pas ces arrangements en danger par des comportements imprudents ou des passions irréfléchies, d'où leur accord profond avec les autorités morales et religieuses de l'époque. L'intérêt matériel des époux était le plus souvent de rester ensemble, quitte à accepter des renoncements ou des compromis sur les vrais désirs de chacun, et à promouvoir comme un des fondements du savoir-vivre une dose convenable d'hypocrisie : d'ailleurs, dès l'antiquité païenne, il était très inconvenant d'afficher publiquement une affection trop vive entre conjoints. 

 Certes, l'impossibilité de placer les préférences individuelles avant tout autre critère pouvait faire souffrir, et l'amour passion comme la liberté de choix du conjoint faisaient rêver. Les œuvres littéraires du passé reflètent la prégnance de ces représentations. Ainsi, pour ne prendre qu'un seul exemple, la plupart des intrigues de Molière reposent sur le refus d'un mariage arrangé. les romans de Jane Austen surnagent comme des modèles parmi des milliers d'autres fondés sur les "problèmes de coeur" de jeunes gens et surtout de jeunes filles, apparemment libres de leurs choix, plus libres en Angleterre qu'en France au moins au premier regard, et en réalité excessivement contraints. Les gens raisonnables savaient que le choix du conjoint n'était habituellement pas libre. Ce n'était que du rêve. Les contraintes économiques étaient indépassables, en dépit des souffrances et des renoncements qu'elles entraînaient. Cela n'empêchait pas la société de continuer siècle après siècle à fonctionner sur le même mode. 

 Aujourd'hui, à la condition de posséder une vraie qualification professionnelle (capital intellectuel), il n'est plus besoin de capitaux pour s'établir. Rien ne vaut un « bon » métier : un métier qui implique beaucoup de savoirs et de savoir-faire, dans un secteur d'activité porteur. Il n'est plus honteux de « servir ». Au fil du \siecle{20}, le salariat s'est souvent révélé plus sûr que la possession de capitaux ou d'outils de production, surtout au service de l'État. D'autre part la scolarité est désormais gratuite ou presque jusqu'aux niveaux les plus élevés (même si ce n'est souvent pas vrai aux niveaux les plus élevés). Les parents ont intégré cette logique : depuis la Libération, le taux de scolarisation n'a cessé de s'accroître bien au-delà de la fin de l'obligation scolaire, qui elle-même est passée de 14 (1936) à 16 ans (1959). Le nombre des diplômés de l'enseignement supérieur a explosé. Le nombre de bacheliers se situe actuellement entre 60~\% et 80~\% d'une classe d'âge, contre 5~\% à la Libération et 8~\% en 1960. Même si ce diplôme s'est largement dévalué et ne peut plus depuis longtemps procurer un emploi à lui seul, le niveau de culture moyen a indiscutablement progressé. 

 Au \siecle{19}, un homme dépensait plus s'il était célibataire que s'il était marié, sauf à employer une « bonne à tout faire ». Il était plus rentable d'entretenir une « ménagère » à domicile que de manger tous les jours au restaurant, de faire blanchir son linge,~etc. En dehors de sa dot (très mince ou inexistante dans les milieux populaires), une épouse fournissait gratuitement une somme de prestations qu'il eût été coûteux de se procurer sur le marché. Aujourd'hui la rentabilité du travail domestique a été fortement réduite par les innovations techniques, commerciales et sociales du \siecle{20} : infrastructures collectives (électricité, tout à l'égout, eau courante) ; machines qui économisent le temps de travail (chauffage central, machine à laver le linge, la vaisselle, cuisines équipées électriques ou au gaz, aspirateur,~etc.) ; grande distribution qui rend non-compétitive l'auto production en couture, en jardinage vivrier, en préparation des aliments,~etc. Sans oublier les écoles maternelles et les garderies d'enfants. 

 Si les ménagères ont été libérées d'une grande partie du poids des tâches domestiques, elles ont en même temps été réduites au chômage technique : c'est au même moment qu'on observe la fin des « bonnes ». Pour contribuer significativement aux ressources de leur ménage elles doivent désormais travailler au dehors de leur foyer. Cela leur a ouvert la possibilité de -- mais on peut tout aussi bien dire que cela les a contraintes à -- se trouver d'autres emplois que d'être la « maîtresse de maison » titulaire d'un homme. 

 Elles y ont d'autant plus été poussées, que depuis la libéralisation du divorce elles ne peuvent plus être assurées, comme l'étaient leurs grand-mères, de leur position d'épouse en titre d'un homme nommément désigné. Les filles de la bourgeoisie ont compris que leur avenir serait mieux assuré par un « bon » métier que par une « belle » dot, un « beau » parti, et par le « grand » mariage qui était jusqu'alors le point de focalisation de tous les désirs familiaux, le signe et le sommet de la réussite féminine. Toutes ont compris que grâce aux savoirs et aux diplômes elles seraient libres : indépendantes des désirs d'un homme et de sa bonne volonté, à l'abri des effets matériels des répudiations, en mesure de prendre l'initiative et de sortir des situations affectives ou familiales dans lesquelles elles ne trouveraient pas leur compte. Dans la course au diplôme les filles se sont (donc ?) montrées significativement plus déterminées que les garçons (quant aux décrochages scolaires de ces derniers, ils ont bien entendu d'autres causes). 

 S'il n'est plus nécessaire pour « s'établir » d'avoir l'appui financier de ses parents ni de leurs relations, alors le mariage ne scelle plus l'alliance (économique et éventuellement politique) de deux familles : alors rien n'exige plus que les jeunes gens subissent un mariage arrangé, un mariage d'argent et d'entregent. Ils peuvent sans risque \emph{matériel} excessif s'offrir le luxe de n'être pas raisonnables et de baser leur couple sur la seule passion amoureuse.

 C'est en tout cas tellement devenu notre logique que cela révolutionne notre compréhension du mariage lui-même. Si celui-ci se définit d'abord comme l'union de deux personnes qui s'aiment, alors la question de la durée perd de son sens. L'authenticité des désirs inscrits dans les actes posés ici et maintenant a plus d'importance que la fidélité à une promesse ancienne. L'infidélité conjugale n'est plus une offense à un ordre public qui ne se donne plus pour but de sanctuariser les familles. Ce n'est plus qu'une offense privée, le signe d'un désaccord entre deux associés. La séduction devient une obligation permanente. L'accord du conjoint à une relation charnelle ne peut plus être tenu pour acquis d'avance, par contrat. La notion de \emph{devoir conjugal} s'est vidée de son sens, et la loi ne le reconnaît plus. La notion de viol entre époux prend du sens, et comme tout viol c'est un délit punissable par la loi. 

 Si c'est l'amour mutuel qui fonde le couple, alors sa fécondité potentielle perd de son importance. Que le couple soit constitué d'un homme et d'une femme ne va plus sans le dire. La reconnaissance publique d'un couple de deux hommes ou de deux femmes n'est plus impossible à penser. 

 Mais si c'est l'enfant qui fait la famille, à quoi bon se marier ?
