% Le 18.03.2015 :
% Moyen Âge
% Antiquité
% droit-Droit
% Le 24.02.2015 :
% ~etc.
% Moyen-Âge
%~\%



\chapter{La famille du Code Napoléon}

\section{Suppression du divorce}
 En 1802 Napoléon signait avec le Pape un Concordat qui reconnaissait la religion catholique comme la \emph{religion de la majorité des Français}. À ce titre, il reconnaissait à cette religion une vocation à être l'une des sources du droit et à l'État le droit de nommer les évêques. Il prenait acte de l'expropriation par les révolutionnaires des propriétés de l'Église. En contrepartie l'État s'engageait à salarier et loger les ministres du culte (avant la révolution c'était l'une des revendications du bas clergé). Au nom de l'égalité, l'État reconnaissait aussi les églises protestantes et le judaïsme, et salariait également les pasteurs protestants et les rabbins.
 
 Napoléon a fait réorganiser le droit civil par les professeurs de Droit les plus réputés en faisant une synthèse de la législation révolutionnaire et du droit coutumier de l'ancien régime.  Le \emph{Code Civil} (ou \emph{Code Napoléon}) paraît en 1804. Que ce soit en souci de conformité avec le droit Canon, dans le cadre du Concordat, ou en réaction aux innovations révolutionnaires, qui n'avaient eu guère de succès en dehors de la population des villes, très minoritaire, il restaure presque intégralement la \emph{famille constantinienne}, fondée sur l'union monogame et (quasi) indissoluble d'un homme et d'une femme, et qui exclut de l'héritage tous les enfants adultérins. Il la défend contre tous les courants centrifuges qui pourraient menacer son unité et donc la fragiliser. 
 
 La société, ou du moins une grande partie de celle-ci, se sent attaquée lorsqu'un mariage est menacé, comme du temps d'Auguste, et elle réprouve le divorce, même quand elle le permet. En 1804 le Code Civil supprime le divorce pour \emph{incompatibilité d'humeur} et pose tant de conditions au divorce par \emph{consentement mutuel} qu'il devient très rare, environ cinquante par an, alors qu'en l'an~VII de la Révolution le nombre des divorces dans les villes était le tiers de celui des mariages. Il comprend le divorce comme la sanction d'une faute : adultère du partenaire, condamnation à une peine infamante, excès, sévices ou injures graves... Il restaure la \emph{séparation de corps}, qui interdit le remariage. D'autre part les époux divorcés n'ont plus le droit de se remarier l'un avec l'autre (comme dans le droit juif). Enfin l'époux condamné pour adultère se voit interdire à vie d'épouser son ou sa complice. On reconnaît là une règle de droit instituée par Constantin et ses successeurs immédiats, et jamais abrogée ensuite jusqu'à la Révolution. 



 La Restauration poursuit dans le même sens et supprime le droit au remariage après divorce dès 1816
\footnote{Sous l'ancien régime chacun était soumis dans le domaine familial au droit de sa propre religion, ce qui permettait aux protestants français (à partir du moment où leur religion était tolérée), aux protestants étrangers résidents permanents (en tout temps), et aux juifs, de rompre une union selon leurs propres règles et d'en contracter légalement une nouvelle : les "sans-religion" n'avaient pas de place dans ce modèle. En raison du principe de l'universalité de la loi institué par la Révolution, et donc de l'impossibilité de reconnaître des droits particuliers à certains citoyens, le divorce a été interdit à tous par le Code Napoléon quelle que soit leur religion ou leur absence de religion. }
. Aux époux mal mariés il ne restait plus que la séparation, comme avant la Révolution. Pour l'obtenir, le demandeur\footnote{qui était le plus souvent une demanderesse : ce n'est pas d'aujourd'hui que les femmes demandent le divorce plus souvent que les hommes.} devait invoquer la faute de son conjoint. L'accord des deux partenaires ne suffisait pas. Les femmes accusaient ordinairement leurs maris de les maltraiter, physiquement ou moralement. Les hommes invoquaient le plus souvent l'adultère de leur épouse. Aux yeux de la loi les infidélités masculines n'étaient des injures graves que s'ils introduisaient leurs maîtresses sous le toit conjugal. 

 Selon une tradition française ancienne, la garde des enfants était ordinairement remise quel que soit leur âge à celui des parents qui était jugé non coupable : pour ce motif elle était le plus souvent confiée aux mères (en Angleterre au contraire les enfants ont été assez systématiquement remis à leur père jusqu'au milieu du \siecle{19}, comme sous l'empire romain). Dans tous les cas de figure, c'est le père qui devait subvenir aux besoins des enfants. Comme toujours depuis l'Antiquité la condamnation d'un conjoint à une peine infamante permettait au conjoint innocent d'obtenir la séparation et la garde des enfants.

\section{Restauration de l'autorité des pères}

 Les points de friction les plus irritants de l'Ancien Régime avaient disparu : les jeunes gens étaient libres de leurs choix professionnels et (dans une certaine mesure) conjugaux à partir de 21 ans, et il n'était guère possible de les déshériter... mais beaucoup d'entre eux travaillaient toujours sous l'autorité de leur père dans son entreprise, son atelier et surtout dans son exploitation agricole. Ils devaient attendre son décès ou son retrait volontaire, ce qui les maintenait dans sa dépendance jusque dans le choix de leur conjoint, choix d'autant plus contrôlé qu'il restait souvent l'une des clés de leur établissement professionnel. 

 Dans le principe, les droits parentaux (ce qu'on appelait la \emph{puissance paternelle}) étaient reconnus à chacun des deux parents mais au nom de l'unité du commandement jugée nécessaire à toute institution la cellule familiale était confiée à la direction du mari, et les femmes mariées étaient mises sous la tutelle de leurs maris. Seuls les hommes participaient à la vie publique et pouvaient exercer le pouvoir politique. Tant qu'ils étaient vivants et non déchus de leurs droits pour condamnation infamante ou pour maltraitance grave de leurs enfants, ou pour démence, ou pour absence, c'étaient eux qui exerçaient la puissance paternelle. Ce n'est qu'en cas d'absence, de séparation à leurs torts, de condamnation à une peine infamante, ou de décès, que les mères pouvaient les remplacer, et encore devaient-elles dans certaines circonstances être assistées dans l'exercice de ce droit par un ou plusieurs membres mâles de la famille de leur époux. Les familles du Code Napoléon étaient donc aussi patriarcales que celles de l'ancien régime. Elles tenaient leurs enfants en main fermement. Ils avaient besoin de l'accord de leurs parents pour se marier, quel que soit leur âge, et ne pouvaient passer outre à leur refus qu'à certaines conditions. 

 Le Code Civil de 1804 donnait au père le droit de faire appel au juge s'il estimait que son autorité n'était pas respectée par son enfant mineur%
%[3]
\footnote{Cf. Pascale \fsc{QUINCY-LEFEBVRE}, « Une autorité sous tutelle. La justice et le droit de correction des pères sous la Troisième République », in \emph{Lien social et politiques, Politiques du père,} RIAC, 37, Printemps 1997, p. 99-109.}%
. Il pouvait faire enfermer un de ses enfants de moins de 16 ans pendant un mois (renouvelable s'il le jugeait nécessaire). Le mineur « de famille » interné pour ce motif était traité comme les délinquants du même âge. S'il avait 16 ans et plus (majorité pénale), ou s'il possédait des biens, ou si son père était remarié, il bénéficiait de plus de garanties : le magistrat pouvait accepter, réduire ou refuser la demande d'incarcération. Mais à partir de seize ans celle-ci pouvait durer six mois renouvelables. Même si la loi mettait des limites au droit de correction, le juge n'avait qu'une assez faible liberté d'appréciation : \emph{il se devait} d'apporter son aide au père qui la sollicitait. En cas de décès du père et si la mère ne s'était pas remariée, c'est elle qui exerçait le droit de correction paternelle \emph{avec l'accord des deux plus proches parents du défunt}. 

 Les lettres de cachet avaient certes disparu, mais la Justice restait \emph{tenue} de fournir son aide aux parents qui la lui demandaient pour contenir et corriger les mineurs dont la conduite préoccupait ces derniers. Durant la majeure partie du \siecle{19} elle l'a fait sans trop se poser de questions. Rapportées au nombre de jeunes français, le nombre des mesures administratives de \emph{correction paternelle} était d'ailleurs limité : quelques milliers par an tout au plus. Et il y avait de grandes disparités dans le nombre des recours au juge suivant les régions et suivant les milieux sociaux. Ils étaient beaucoup plus fréquents dans les familles populaires de Paris que partout ailleurs : plus de la moitié des mesures%
%[4]
\footnote{Cf. Pascale \fsc{QUINCY-LEFEBVRE}, article cité, p. 99.}%
. Ailleurs on se débrouillait autrement avec les jeunes « récalcitrants », fugueurs, « paresseux », « libertins », ou « vicieux » (c'était le langage de l'époque). Il était peut-être plus facile d'élever un adolescent à la campagne ou dans des villes beaucoup plus petites, plus paisibles et moins bouillonnantes de sollicitations que Paris ? Et surtout nulle part ailleurs qu'à Paris n'existait la même tradition de proximité, et même de familiarité, avec la personne du souverain, ce qui facilitait les recours. Quant aux bourgeois, de Paris ou d'ailleurs, ils disposaient toujours de toute une gamme d'internats pour mettre un peu de distance entre eux-mêmes et leurs adolescents trop difficiles à élever, et pour offrir à ceux-ci une rencontre avec des éducateurs professionnels en principe plus sereins et moins impliqués.

 Jusqu'à 1882 l'école n'était pas obligatoire et les enfants des classes populaires qui n'y allaient pas commençaient à travailler très tôt. Comme toujours, s'ils ne les employaient pas eux-mêmes leurs pères les plaçaient chez un patron et touchaient leurs gains jusqu'à leur majorité. Les enfants qui avaient été scolarisés les rejoignaient dès dix à douze ans. C'est dans ce cadre que doivent souvent être interprétés les reproches formulés par les pères contre leurs enfants, tout comme les fugues et le vagabondage de ces derniers, qui fuyaient peut-être moins la maison paternelle ou l'école que l'atelier, la boutique, l'usine ou la maison bourgeoise où ils (elles) avaient été placés.

 De même qu'en 1801 on avait séparé les aliénés des délinquants, sous la Restauration on a séparé autant que faire se pouvait les mineurs, délinquants et vagabonds, des majeurs, pour éviter qu'ils ne soient maltraités ou « pervertis » par eux. On a donc créé des établissements de correction (ou de redressement) spécialisés dans la prise en charge et la rééducation des délinquants et vagabonds mineurs : prisons spéciales vers 1820 (quartiers spécialisés au sein des prisons, la Petite Roquette,~etc.) puis en 1830 pénitenciers pour mineurs, puis à partir de 1840 les \emph{Colonies agricoles et pénitentiaires} privées. Comme aux siècles précédents, depuis le début du \crmieme{19} les fugueurs, les vagabonds, les prostitués et les mendiants de moins de seize ans (mineurs pénaux) étaient arrêtés par la force publique (du moins s'ils causaient du trouble à l'ordre public). À Paris ils étaient conduits à la Préfecture de police. Ceux qui étaient condamnés allaient en prison. Ceux qui étaient acquittés mais que leurs parents ne réclamaient pas étaient déférés à l'autorité judiciaire. Ils allaient en Colonie Pénitentiaire.

 Les jeunes de la correction paternelle étaient placés à la Petite Roquette pour les garçons, au couvent des dames de Saint-Michel pour les filles. En province ils étaient placés en maison d'arrêt avec les détenus de tous les âges (d'où le moindre recours des parents à cette mesure ?). 

 Face à leurs jeunes « indisciplinés », les familles plus aisées recouraient à des internats scolaires comme aux siècles précédents, sans faire appel à la Justice. Ainsi à partir de 1850 à côté de la Colonie agricole et pénitentiaire de Mettray existait une \emph{Maison Paternelle} réputée, créée par le même fondateur que la Colonie Pénitentiaire, et qui fonctionnait toujours vers 1910, jusqu'à ce que le suicide d'un pensionnaire la fasse fermer. Elle était vouée à la correction des fils des familles suffisamment aisées pour en payer la pension.

\section{Interdiction des recherches en paternité}

 Le Code Napoléon (1804) ramenait les « bâtards » non reconnus par mariage subséquent à leur situation antérieure à la Révolution. Il interdisait la reconnaissance des enfants adultérins et incestueux par leurs géniteurs. Même lorsqu'ils avaient été reconnus par ceux-ci il les excluait de leur succession, donc de leur famille, et ne leur reconnaissait que leur droit traditionnel à des legs « alimentaires ». 

 L'adoption était autorisée par le Code Napoléon, mais il s'agissait uniquement de \emph{l'adoption d'adultes majeurs} par des personnes de 50 ans et plus, et non d'enfants mineurs ni de nouveaux-nés. Contrairement aux vœux des révolutionnaires, il ne s'agissait pas en principe de donner une famille à un enfant sans parents, mais de répondre au besoin d'enfant d'une famille en mal d'héritier. Ces adoptions seront rares durant tout le \siecle{19} et jusqu'en 1923 : 114,4 par an en moyenne de 1840 à 1886 pour toute la France, dont 49,4 enfants naturels, reconnus ou non, et 17,76 neveux, nièces et autres alliés
\footnote{« Statistiques des adoptions au \siecle{19} d'après les comptes généraux de l'administration de la justice civile », tableau cité dans \emph{l'avis présenté au nom de la Commission des affaires sociales sur la proposition de loi, adoptée par l'Assemblée Nationale, relative à l'adoption}, \no~298, session ordinaire du Sénat de 1995-1996, annexe au procès-verbal de la séance du 28 mars 1996.}
. L'objectif de ces adoptions était d'abord de transmettre un patrimoine
\footnote{Plus de la moitié des adoptants étaient des rentiers, terme qui désignait notamment les personnes qui s'étaient retirées des affaires après avoir vendu leur entreprise ou leur commerce et qui vivaient des rentes produites par leur capital : c'étaient en somme des retraités.}
. 

 Le nouveau Code durcissait encore l'interdiction révolutionnaire des recherches en paternité naturelle. Pourtant les recherches en maternité naturelle restaient autorisées. Il n'acceptait les recherches en paternité qu'en cas d'enlèvement, mais il les excluait en cas de viol sans enlèvement,~etc. Comme preuves de la paternité il n'acceptait que les aveux formels écrits par le père, ou bien la cohabitation prolongée du père avec la mère, ou encore la \emph{possession d'état}%
%[7]
\footnote{Situation où le mineur est élevé comme son enfant par le père supposé, même s'il ne l'a pas formellement reconnu.}%
,~etc. Dans ces conditions aucun homme ou presque ne pouvait être contraint contre son gré à reconnaître un enfant naturel, ni condamné à verser une pension alimentaire, même quand tout le monde savait parfaitement à quoi s'en tenir sur ses responsabilités. Il ne restait aux enfants naturels qu'à espérer que leur géniteur veuille bien prendre librement l'initiative de les reconnaître. 

 Cela ne pouvait que pousser les mères célibataires à ne pas garder leur enfant et à l'abandonner anonymement, et cette conséquence était acceptée sans état d'âme. Le placement à la campagne des enfants abandonnés était jugé satisfaisant par tout le monde, et le "bâtard" était un obstacle presque insurmontable à la « rédemption » de sa mère par le mariage, sauf s'il était légitimé par le mariage de celle-ci avec son géniteur (à la rigueur avec un autre homme), ce qui restait la solution préférée. 


\section{les contraintes du choix du conjoint}
 
 Depuis toujours le premier objectif des jeunes gens raisonnables n'était pas tant de vivre mieux que leurs parents et de s'enrichir, que de réussir à obtenir le même niveau de fortune qu'eux, et de ne pas tomber dans l'indigence. Au \siecle{19} (et sans doute en était-il de même durant les siècles antérieurs) un homme dépensait plus s'il était célibataire que s'il était marié, sauf à employer une « bonne à tout faire ». Il était plus rentable d'entretenir une « ménagère » à domicile que de manger tous les jours au restaurant, de faire blanchir son linge (et, accessoirement, de fréquenter les prostituées) ~etc. En dehors de sa dot (très mince ou inexistante dans les milieux populaires), une épouse fournissait une somme de prestations qu'il était coûteux de se procurer sur le marché. Il était donc mutuellement avantageux pour les jeunes gens et les jeunes filles sans fortune de se mettre en ménage. Mais s'il était souvent préférable pour eux de se marier que de ne pas le faire, il leur fallait aussi éviter de compromettre, par enthousiasme naïf, par imprudence ou par sottise, les bases économiques de leur futur couple et le statut social de leurs enfants à venir. La survenue de ces derniers pouvait être un problème et une cause de déstabilisation de leur situation économique : le recul de l'âge au mariage permettait aux filles pauvres de contrôler leur nombre tout en de se constituer une dot par leur travail, et on observe que leur âge moyen au mariage était bien plus élevé que celui des héritières. A partir d'un certain âge (peu à peu reculé par la loi) les enfants contribuaient à leur tour aux revenus du ménage et ils étaient une garantie pour l'avenir (un "bâton de vieillesse"). 

 Le choix du mariage d'inclination, fondé sur l'amour passion et non sur la raison (l'intérêt) la marque des imprévoyant(e)s. Entre mariage d'inclination et concubinage les liens paraissaient évidents. C'est ainsi que s'unissaient ceux qui ne possédaient que leurs bras, les ouvriers, les manœuvres, les valets, les ouvrières et les servantes, etc. Ceux qui se mettaient en ménage avant d'avoir « assis » leur « situation » se condamnaient à « tirer le diable par la queue ». Selon les moralistes, avec lesquels faisaient chorus tous les parents angoissés, la soumission des jeunes imprévoyants à leurs appétits charnels et à leurs affects leur faisait courir le risque de gâcher leur vie, de connaître la misère et de perdre un jour la main sur leurs propres enfants, ainsi qu'il en avait toujours été depuis le début du monde. Ils risquaient en effet de ne pas pouvoir les élever et de devoir les abandonner aux institutions d'assistance. Ils ne pourraient les « établir », ni en leur donnant un capital matériel, ni en finançant leur apprentissage professionnel auprès d'un maître qualifié, ni en les mettant à l'école, même gratuite, puisqu'ils seraient contraints de les placer chez un maître dès que leur âge le permettrait. En cas de chômage et de disette, ils seraient contraints de les envoyer mendier. Ils ne pourraient pas compter sur ces enfants, condamnés à être pauvres à leur tour, pour soutenir leur propre vieillesse. Ils risquaient de finir leurs jours dans la solitude et la misère, affective et matérielle, des hospices.

 Au contraire les parents prévoyants établissaient leurs enfants dans un mariage profitable grâce à leurs économies, à leurs relations et à des stratégies complexes : échanges simultanés et réciproques d'enfants, de terres, de droits d'exploitation, d'entreprises, de gérances, d'offices (ministériels), etc. sans compter jusqu'à la Révolution l'entrée en religion de ceux qu'ils ne pouvaient ou ne voulaient pas marier de manière conforme à leur milieu social. Voilà pourquoi depuis la Renaissance l'accord des parents était demandé pour tout mariage : selon le Code civil de 1804 l'âge à partir duquel le mariage était autorisé était de 15 ans pour les filles et 18 ans pour les garçons, mais l'âge où l'accord des parents cessait d'être exigé était bien plus tardif : 21 ans pour les filles et 25 ans pour les garçons.
 
 Des stratégies familiales si complexes ne pouvaient pas toujours tenir compte des préférences sexuelles ou amoureuses de chacun, et on n'en faisait pas grief aux parents : les femmes s'en consoleraient avec leurs enfants ou la religion, les hommes avec le travail, le pouvoir, les prostituées ou les maîtresses (le recours à celles-là étant toujours préférable, du point de vue des épouses, au choix de celles-ci). Les patrimoines étaient verrouillés contre les effets des infidélités des uns et des autres. Jusqu'au début du XXème siècle une épouse ne pouvait introduire d'enfant adultérin dans sa famille que si son mari le voulait bien, mais en ce cas la paternité de celui-ci devenait absolument inattaquable : le géniteur n'avait aucun recours. Quant aux enfants illégitimes du mari, ils ne pouvaient pas être légitimés et menacer l'héritage des enfants de l'épouse. 

 Sauf emploi salarié stable et suffisamment rémunérateur (au service de l'état si possible) la pérennité des couples raisonnables était favorisée par la synergie des ressources que leurs familles respectives avaient sagement et laborieusement conjointes (même les militaires de carrière étaient fermement invités à épouser des filles bien dotées). Leurs parents étaient les premiers à tenir fermement à ce qu'ils, et elles plus encore, ne mettent pas ces arrangements en danger par des comportements imprudents ou des passions irréfléchies, d'où leur accord profond sur ce point avec les autorités morales et religieuses de l'époque. L'intérêt matériel des époux était le plus souvent de rester ensemble, quitte à accepter des renoncements ou des compromis sur les vrais désirs de chacun, et à cultiver comme un des fondements du savoir-vivre une dose convenable d'hypocrisie : d'ailleurs, dès l'antiquité païenne, il était très inconvenant d'afficher publiquement une affection trop vive entre les conjoints. 

 Certes, l'impossibilité de placer les préférences individuelles avant tout autre critère pouvait faire souffrir, et l'amour passion comme la liberté de choix du conjoint faisaient rêver. Les œuvres littéraires du passé reflètent la prégnance de ces représentations. Ainsi, pour ne prendre qu'un seul exemple, la plupart des intrigues de Molière reposent sur le refus d'un mariage arrangé. Les romans de Jane Austen sont des archétypes parmi les milliers d'autres fondés sur les "problèmes de coeur" de jeunes gens et de jeunes filles, apparemment libres de leurs choix et en réalité exctrêmement contraints. Les contraintes économiques étaient indépassables, en dépit des souffrances et des renoncements qu'elles entraînaient. Cela n'empêchait pas la société de continuer siècle après siècle à fonctionner sur le même mode. 
 