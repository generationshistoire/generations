
%I1 LA FAMILLE DU CODE NAPOLEON

 En 1802 Napoléon signe avec le Pape un Concordat qui reconnaît la religion catholique comme la « \emph{religion de la majorité des français} ». À ce titre il reconnaît à cette religion une vocation à être l'une des sources du droit[1]. À la manière des empereurs de l'antiquité tardive et du début du moyen-âge Napoléon fait réorganiser le droit civil par les professeurs de droit les plus réputés du moment en synthétisant la législation révolutionnaire et le droit coutumier de l'ancien régime. 
 Le \emph{Code Civil} (ou \emph{Code Napoléon}) paraît en 1804. Que ce soit en souci de conformité avec le droit Canon, dans le cadre du Concordat, ou en réaction aux innovations révolutionnaires dans le droit des familles, qui n'avaient eu à peu près aucun succès en dehors de la population des villes, très minoritaire, il restaure presque intégralement la \emph{famille constantinienne}, fondée par et sur l'union monogame et indissoluble d'un homme et d'une femme, et qui exclut tous les enfants adultérins, et il la replace à la base de l'état et de la société. Il la défend contre tous les courants centrifuges qui pourraient menacer son unité et donc la fragiliser. Il restaure la \emph{séparation de corps}, qui interdit le remariage, et il limite le droit au divorce, compris exclusivement comme sanction d'une faute : adultère du partenaire, condamnation à une peine infamante, excès, sévices ou injures graves. 
 Comme du temps d'Auguste la société se sent attaquée lorsqu'un mariage est menacé, et réprouve le divorce, même quand elle le permet. En 1804 le Code civil supprime le divorce pour \emph{incompatibilité d'humeur} et pose tant de conditions au divorce par \emph{consentement mutuel} qu'il devient très rare, environ cinquante par an, alors qu'en l'an VII de la Révolution le nombre des divorces dans les villes était le tiers de celui des mariages. D'autre part les époux divorcés n'ont plus le droit de se remarier l'un avec l'autre (comme dans le droit juif). Enfin l'époux condamné pour adultère se voit à nouveau interdire à vie d'épouser son ou sa complice. On reconnaît là une règle de droit instituée par Constantin et ses successeurs immédiats, et jamais abrogée ensuite avant la Révolution.
SUPPRESSION DU DIVORCE
 La Restauration poursuit dans le même sens et supprime le droit au remariage après divorce dès 1816[2]. Aux époux mal mariés il ne restait plus que la séparation, comme avant la Révolution. Pour l'obtenir le demandeur (qui était le plus souvent une demanderesse : ce n'est pas d'aujourd'hui que les femmes demandent le divorce plus souvent que les hommes) devait invoquer la faute de son conjoint. L'accord des deux partenaires ne suffisait pas. Les femmes accusaient ordinairement leurs maris de les maltraiter, physiquement ou moralement. Les hommes invoquaient le plus souvent l'adultère de leur épouse. Aux yeux de la loi leurs propres infidélités n'étaient des injures graves que s'ils introduisaient leurs maîtresses sous le toit conjugal. 
 Selon une tradition française ancienne la garde des enfants était ordinairement remise quel que soit leur âge à celui des parents qui était jugé non coupable : pour ce motif elle était le plus souvent confiée aux mères (en Angleterre au contraire les enfants ont été assez systématiquement remis à leur père jusqu'au milieu du XIXème siècle, comme sous l'empire romain). Dans tous les cas de figure c'est le père qui devait subvenir aux besoins des enfants. Comme toujours depuis l'antiquité la condamnation d'un conjoint à une peine infamante permettait au conjoint innocent d'obtenir la séparation et la garde des enfants.
RESTAURATION DE L'AUTORITE DES PERES
 Les points de friction les plus irritants de l'Ancien Régime avaient disparu : les jeunes gens étaient libres de leurs choix professionnels et (dans une certaine mesure) conjugaux à partir de 21 ans, et il n'était guère possible de les déshériter... mais beaucoup d'entre eux travaillaient toujours sous l'autorité de leur père dans son entreprise, son atelier et surtout dans son exploitation agricole. Ils devaient attendre son décès ou son retrait volontaire, ce qui les maintenait dans sa dépendance jusque dans le choix de leur conjoint, choix d'autant plus contrôlé qu'il restait souvent l'une des clés de leur établissement professionnel. 
 Dans le principe les droits parentaux (ce qu'on appelait la \emph{puissance paternelle}) étaient reconnus à chacun des deux parents mais au nom de l'unité du commandement jugée nécessaire à toute institution la cellule familiale était confiée à la direction du mari, et les femmes mariées étaient mises sous la tutelle de leurs maris. Seuls les hommes participaient à la vie publique et pouvaient exercer le pouvoir politique. Tant qu'ils étaient vivants et non déchus de leurs droits pour condamnation infamante ou pour maltraitance grave de leurs enfants, ou pour démence, ou pour absence, c'étaient eux qui exerçaient la puissance paternelle. Ce n'est qu'en cas d'absence, de séparation à leurs tort, de condamnation à une peine infamante, ou de décès, que les mères pouvaient les remplacer, et encore devaient-elles dans certaines circonstances être assistées dans l'exercice de ce droit par un ou plusieurs membres mâles de la famille de leur époux. Les familles du Code Napoléon étaient donc aussi patriarcales que celles de l'ancien régime. Elles tenaient leurs enfants en main fermement. Ils avaient besoin de l'accord de leurs parents pour se marier, quel que soit leur âge, et ne pouvaient passer outre à leur refus qu'à certaines conditions. 
 Le Code civil de 1804 donnait au père le droit de faire appel au juge s'il estimait que son autorité n'était pas respectée par son enfant mineur[3]. Il pouvait faire enfermer un de ses enfants de moins de 16 ans pendant un mois (renouvelable s'il le jugeait nécessaire). Le mineur « de famille » interné pour ce motif était traité comme les délinquants du même âge. S'il avait 16 ans et plus (majorité pénale), ou s'il possédait des biens, ou si son père était remarié, il bénéficiait de plus de garanties : le magistrat pouvait accepter, réduire ou refuser la demande d'incarcération. Mais à partir de seize ans celle-ci pouvait durer six mois renouvelables. Même si la loi mettait des limites au droit de correction, le juge n'avait qu'une assez faible liberté d'appréciation : \emph{il se devait}d'apporter son aide au père qui la sollicitait. En cas de décès du père et si la mère ne s'était pas remariée c'est elle qui exerçait le droit de correction paternelle \emph{avec l'accord des deux plus proches parents du défunt}. 
 Les lettres de cachet avaient certes disparu, mais la Justice restait \emph{tenue} de fournir son aide aux parents qui la lui demandaient pour contenir et corriger les mineurs dont la conduite préoccupait ces derniers. Durant la majeure partie du XIXème siècle elle l'a fait sans trop se poser de questions. Rapportées au nombre de jeunes français le nombre des mesures administratives de « \emph{correction paternelle} » était d'ailleurs limité : quelques milliers par an tout au plus. Et il y avait de grandes disparités dans le nombre des recours au juge suivant les régions et suivant les milieux sociaux. Ils étaient beaucoup plus fréquents dans les familles populaires de Paris que partout ailleurs : plus de la moitié des mesures[4]. Ailleurs on se débrouillait autrement avec les jeunes « récalcitrants », fugueurs, « paresseux », « libertins », ou « vicieux » (c'était le langage de l'époque). Il était peut-être plus facile d'élever un adolescent à la campagne ou dans des villes beaucoup plus petites, plus paisibles et moins bouillonnantes de sollicitations que Paris ? Et surtout nulle part ailleurs qu'à Paris n'existait la même tradition de proximité, et même de familiarité, avec la personne du souverain, ce qui facilitait les recours. Quant aux bourgeois, de Paris ou d'ailleurs, ils disposaient toujours de toute une gamme d'internats pour mettre un peu de distance entre eux-mêmes et leurs adolescents trop difficiles à élever, et pour offrir à ceux-ci une rencontre avec des éducateurs professionnels en principe plus sereins et moins impliqués.
 Jusqu'à 1882 l'école n'était pas obligatoire et les enfants des classes populaires qui n'y allaient pas commençaient à travailler très tôt. Comme toujours s'ils ne les employaient pas eux-mêmes leurs pères les plaçaient chez un patron et touchaient leurs gains jusqu'à leur majorité. Les enfants qui avaient été scolarisés les rejoignaient dès dix à douze ans. C'est dans ce cadre que doivent souvent être interprétés les reproches formulés par les pères contre leurs enfants, tout comme les fugues et le vagabondage de ces derniers, qui fuyaient peut-être moins la maison paternelle ou l'école que l'atelier, la boutique, l'usine ou la maison bourgeoise où ils (elles) avaient été placés.
 De même qu'en 1801 on avait séparé les aliénés des délinquants, sous la Restauration on a séparé autant que faire se pouvait les mineurs, délinquants et vagabonds, des majeurs, pour éviter qu'ils ne soient maltraités et/ou « pervertis » par eux. On a donc créé des établissements de correction (ou de redressement) spécialisés dans la prise en charge et la rééducation des délinquants et vagabonds mineurs : prisons spéciales vers 1820 (quartiers spécialisés au sein des prisons, la Petite Roquette, etc.), puis en 1830 pénitenciers pour mineurs, puis à partir de 1840 les \emph{Colonies agricoles et pénitentiaires}privées. Comme aux siècles précédents depuis le début du XIXème les fugueurs, les vagabonds, les prostitués et les mendiants de moins de seize ans (mineurs pénaux) étaient arrêtés par la force publique (du moins s'ils causaient du trouble à l'ordre public). À Paris ils étaient conduits à la Préfecture de police. Ceux qui étaient condamnés allaient en prison. Ceux qui étaient acquittés mais que leurs parents ne réclamaient pas étaient déférés à l'autorité judiciaire. Ils allaient en Colonie Pénitentiaire.
 Les jeunes de la correction paternelle étaient placés à la Petite Roquette pour les garçons, au couvent des dames de Saint-Michel pour les filles. En province ils étaient placés en maison d'arrêt avec les détenus de tous les âges (d'où le moindre recours des parents à cette mesure ?). 
 Face à leurs jeunes « indisciplinés » les familles plus aisées recouraient à des internats scolaires comme aux siècles précédents, sans faire appel à la Justice. Ainsi à partir de 1850 à côté de la Colonie agricole et pénitentiaire de Mettray existait une \emph{Maison Paternelle} réputée, créée par le même fondateur que la Colonie Pénitentiaire, et qui fonctionnait toujours vers 1910, jusqu'à ce que le suicide d'un pensionnaire la fasse fermer. Elle était vouée à la correction des fils des familles suffisamment aisées pour en payer la pension.
INTERDICTION DES RECHERCHES EN PATERNITE
 Le Code Napoléon (1804) ramenait les « bâtards » non reconnus par mariage subséquent à leur situation antérieure à la Révolution. Il interdisait la reconnaissance des enfants adultérins et incestueux par leurs géniteurs. Même lorsqu'ils avaient été reconnus par ceux-ci il les excluait de leur famille, donc des successions, et ne leur reconnaissait que leur droit traditionnel à des legs « alimentaires ». 
 L'adoption était autorisée par le Code Napoléon, mais il s'agissait uniquement de l\emph{'adoption d'adultes majeurs} par des personnes de 50 ans et plus, et non d'enfants mineurs ni de nouveaux-nés. Contrairement aux vœux des révolutionnaires, il ne s'agissait pas de donner une famille à un enfant sans parents, mais de répondre au besoin d'enfant d'une famille en mal d'héritier. Ces adoptions seront rares durant tout le XIXème siècle et jusqu'en 1923 : 114,4 par an en moyenne de 1840 à 1886 pour toute la France, dont 49,4 enfants naturels, reconnus ou non, et 17,76 neveux, nièces et autres alliés[5]. L'objectif de ces adoptions était d'abord de transmettre un patrimoine[6]. 
 Le nouveau code durcissait encore l'interdiction révolutionnaire des recherches en paternité naturelle. Pourtant les recherches en maternité naturelle restaient autorisées. Il n'acceptait les recherches en paternité qu'en cas d'enlèvement, mais il les excluait en cas de viol sans enlèvement, etc. Comme preuves de la paternité il n'acceptait que les aveux formels écrits par le père, ou bien la cohabitation prolongée du père avec la mère, ou encore la \emph{possession d'état}[7], etc. Dans ces conditions aucun homme ou presque ne pouvait plus être contraint à reconnaître un enfant naturel, ni condamné à verser contre son gré une pension alimentaire, même quand tout le monde savait parfaitement à quoi s'en tenir sur ses responsabilités. Il ne restait plus aux enfants naturels qu'à espérer que leur géniteur veuille bien prendre librement l'initiative de les reconnaître. 
 Cela ne pouvait que pousser les mères célibataires à ne pas garder leur enfant et à l'abandonner anonymement, et cette conséquence était acceptée sans état d'âme. L'enfant sans père était en effet un obstacle à la « rédemption » de sa mère par un mariage légitime, sauf dans le cas de la légitimation du « bâtard » par le mariage de la mère avec son géniteur (à la rigueur avec un autre homme), ce qui restait la solution préférée, et le placement à la campagne des enfants abandonnés fonctionnait d'une manière jugée satisfaisante par tout le monde. 
 
[1] Le Concordat reconnaissait à l'état le droit de contrôler la nomination des évêques. Il entérinait la vente des propriétés de l'Église comme biens nationaux. En contrepartie l'état s'engageait à salarier et loger les ministres du culte : avant la révolution c'était l'une des revendications du bas clergé. Au nom de l'égalité l'état reconnaissait désormais les églises protestantes et le judaïsme, et salariait de la même façon pasteurs et rabbins.
[2] En raison du principe de l'universalité de la loi institué par la Révolution, et donc de l'impossibilité de reconnaître des droits particuliers à certains citoyens, le divorce a été interdit à tous quelle que soit leur religion ou leur absence de religion. Sous l'ancien régime chacun était soumis dans le domaine familial au droit de sa propre religion, ce qui permettait aux protestants français (à partir du moment où leur religion était tolérée), aux protestants étrangers résidents permanents (en tout temps), et aux juifs, de rompre une union selon leurs propres règles et d'en contracter légalement une nouvelle : les athées n'avaient pas de place dans ce modèle.
[3] Cf. Pascale QUINCY-LEFEBVRE, « Une autorité sous tutelle. La justice et le droit de correction des pères sous la Troisième République », in \emph{Lien social et politiques, Politiques du père,} RIAC, 37, Printemps 1997, p. 99-109.
[4] cf. Pascale QUINCY-LEFEBVRE, article cité, p. 99.
[5] « \emph{Statistiques des adoptions au XIXème siècle d'après les comptes généraux de l'administration de la justice civile} », tableau cité dans « \emph{l'Avis présenté au nom de la commission des Affaires sociales sur la proposition de loi, ADOPTÉE PAR L'ASSEMBLÉE NATIONALE, relative à l'adoption} », n° 298, session ordinaire du Sénat de 1995-1996, Annexe au procès-verbal de la séance du 28 mars 1996. 
[6] Plus de la moitié des adoptants étaient des rentiers, terme qui désignait notamment les personnes qui s'étaient retirées des affaires après avoir vendu leur entreprise ou leur commerce et qui vivaient des rentes produites par leur capital : c'étaient en somme des retraités.
[7] Situation où le mineur est élevé comme son enfant par le père supposé, même s'il ne l'a pas formellement reconnu.
