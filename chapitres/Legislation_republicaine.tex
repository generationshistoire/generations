% Le 18 mars 2015 :
% Antiquité
% Moyen Âge
% ~etc.


\chapter{III\ieme{} et IV\ieme{} républiques}


 Depuis la fin du \siecle{19} il s'est produit beaucoup d'évènements qui ont eu un impact décisif, directement ou indirectement, sur les familles et pour commencer voici les décisions essentielles :
\footnote{Sources :
\\Ouvrage collectif de l'Administration générale de l'Assistance Publique, \emph{L'Assistance Publique en 1900}, écrit à l'occasion de l'Exposition Universelle de Paris de 1900, composé et imprimé par les pupilles de la Seine de l'école d'Alembert à Montévrain, Paris, 1900. Consultable au Musée Social, Paris, VIIème.
\\Collectif, sous la direction de Michel \fsc{CHAUVIERE}, Pierre \fsc{LENOËL}, Eric \fsc{PIERRE}, \emph{Protéger l'enfant, Raison juridique et pratiques socio-judiciaires (\crmieme{19} et \siecle{20}{}s)}, Presses Universitaires de Rennes, 1996.
\\Collectif, sous la direction de Jean \fsc{DELUMEAU} et Daniel \fsc{ROCHE}, \emph{Histoire des pères et de la paternité}, Larousse, 1990, édition 2000.
\\Collectif, sous la direction de Jean \fsc{IMBERT}, \emph{Histoire des hôpitaux en France}, Privat, 1982, 559 p.
\\Collectif, sous la direction d'Alain \fsc{BURGUIERE}, Christine \fsc{KLAPISH-ZUBER}, Martine \fsc{SEGALEN}, Françoise \fsc{ZONABEND}, \emph{Histoire de la famille, 3, Le choc des modernités}, Armand Colin Éditeur, Paris, 1986.
\\\fsc{BROUSOLLE} Paul, \emph{Délinquance et déviance, brève histoire de leurs approches psychiatriques}, Privat, Toulouse, 1978.
\\\fsc{CUBERO} José, \emph{Histoire du vagabondage du Moyen Âge à nos jours}, Imago, Paris, 1998.
\\\fsc{DONZELOT} Jacques, \emph{L'invention du social, essai sur le déclin des passions politiques}, Seuil, Paris, 1994.
\\\fsc{DUPOUX} Albert, \emph{Sur les pas de Monsieur Vincent, 300 ans d'histoire parisienne de l'enfance abandonnée}, Édité par la Revue de l'Assistance Publique, Paris, 1958.
\\Patrice \fsc{PINELL}, Markos \fsc{ZAFIROPOULOS}, \emph{Un siècle d'échecs scolaires (1882-1982}), Les éditions ouvrières, Paris, 1983.}% 
 :

\begin{description}
\item[1880] Création d'un enseignement secondaire public pour les filles, calqué sur le modèle de celui des garçons. 

\item[1881] Obligation pour chaque commune de mettre à la disposition de ses administrés une école gratuite et laïque.

%1881 : 
Création du \emph{Service des enfants moralement abandonnés}. Il a pour objet de recevoir les jeunes de 12 à 16 ans sans support familial, pénalement mineurs, non secourus puisque le service ne recevait pas de nouveaux entrants après l'âge de douze ans, et n'ayant que la mendicité et le vol pour subsister : mineurs arrêtés pour {\emph{vagabondage et autres menus délits, et aussi ceux que leurs parents se montraient incapables de diriger}}.

\item[1882] Obligation scolaire pour les garçons et filles de 6 à 12 ans. Avant leurs 12 ans (11 ans s'ils ont le certificat d'études) il est interdit aux pères de placer leurs enfants chez un employeur, ou de les employer eux-mêmes à plein-temps.

\item[1884] Réouverture du droit au divorce (uniquement pour faute, comme en 1804).

\item[1886] Laïcisation du personnel enseignant des écoles publiques. 

\item[1889] La loi du 24 juillet {\emph{sur la protection judiciaire des enfants maltraités et moralement abandonnés}} précise les conditions de la déchéance de la puissance paternelle. Cette déchéance totale est prononcée par un Juge :
\begin{enumerate}[leftmargin=*,itemsep=0pt]
% 1°)
\item facultativement pour inconduite des parents,
% 2°)
\item facultativement en cas de mauvais traitements ou de délaissement de l'enfant,
% 3°)
\item et de plein droit dans le cas de certaines condamnations infamantes (ce qui était le cas depuis l'Antiquité).
\end{enumerate}

% 1889 : 
Cette loi du 24 juillet 1889 %{\emph{sur la protection judiciaire des enfants maltraités et moralement abandonnés}} 
confie à l'administration (c'est-à-dire à l'Assistance Publique) la tutelle des enfants maltraités, victimes de crimes, ou de délits, ou délaissés. Le service les prend en charge même s'ils sont âgés de plus de 12 ans à leur entrée. Ces enfants sont retirés autoritairement à leurs parents et deviennent des pupilles comme les autres. Ils sont traités à l'instar des autres enfants du service. Quel que soit leur âge, autant que faire se peut ils seront placés en nourrice, pour de longues durées, et dans tous les cas ils seront coupés de leurs parents déchus.

\item[1893] Les femmes séparées de corps ont la pleine capacité civile : elles récupèrent les droits qu'elles avaient quand elles étaient célibataires.

%\item[$\!\!\!$] À partir de \textbf{1896} 
\item[1896] {\emph{Les familles indigentes mises devant la nécessité d'abandonner \emph{[sont]} autorisées après enquête à correspondre directement avec enfants et nourriciers}}. D'autre part les {\emph{enfants de parents internés \emph{[sont désormais]} considérés comme n'ayant pas été abandonnés volontairement}}, et les correspondances directes entre parents et enfants sont autorisées.

\item[1897] Les femmes mariées peuvent être témoins dans les actes civils et notariés.

\item[1901] Loi sur les associations à but non lucratif. Leur fondation est libre, basée sur la notion de contrat entre personnes. Elles ne peuvent recevoir ni dons ni legs.

\item[1904] Dénonciation unilatérale du Concordat de 1802.

% 1904 : 
Autorisation donnée aux amants condamnés pour adultère de s'épouser après leur(s) divorce(s) ou le décès du conjoint trompé.

% 1904 : 
 Loi du 12 avril : majorité pénale à 18 ans au lieu de 16, élargissement de l'excuse de minorité, affirmation de la nécessité de faire passer l'éducatif avant le répressif pour les mineurs pénaux.

\item[1907] La loi du 13 juillet permet aux femmes mariées de toucher et de gérer elles-mêmes leur propre salaire, au lieu qu'il soit remis à leur mari comme c'était la règle durant tout le \siecle{19}. 

\item[1912] Autorisation des recherches en paternité naturelle. Les enfants naturels peuvent demander des aliments à chacun de leurs géniteurs : ce texte vise essentiellement les pères, et les mères peuvent agir au nom de leurs enfants. 

% 1912 : 
La loi du 22 juillet créée des tribunaux spéciaux pour enfants et adolescents. Elle pose les premiers jalons de la liberté surveillée

\item[1913] Mesures d'assistance en faveur des femmes en couche nécessiteuses, et des familles nombreuses nécessiteuses.

\item[1917] Une femme peut être nommée tutrice et siéger au conseil de famille

\item[1920] Une femme mariée peut adhérer à un syndicat sans l'autorisation de son mari.

% 1920 : 
Toute forme de propagande anticonceptionnelle ou de publicité pour des instruments de lutte anticonceptionnelle est interdite (préservatifs, pessaires, diaphragmes,~etc. qui restent néanmoins disponibles en pharmacie).

\item[1921] La loi ouvre la possibilité de prononcer une déchéance partielle de l'autorité paternelle. 

\item[1923] L'adoption des enfants abandonnés (sans limite d'âge inférieure) est ouverte aux couples mariés. Nommée légitimation adoptive, elle n'annule pas le passé de l'enfant.

\item[1924] Identité complète des programmes d'études dans le secondaire féminin et masculin.

\item[1925] L'A.P. commence à placer en nourrice les jeunes enfants (âgés de moins de quatre ans, dans un premier temps) placés \emph{en dépôt} par leurs parents et elle les y laisse grandir. 

\item[1932] \emph{Allocations familiales} (pour tous les enfants).

\item[1931] Les femmes peuvent être nommées (élues ?) juges.

\item[1935] Le décret-loi du 30 octobre sur {\emph{la correction paternelle et l'assistance éducative}} institue l'assistance éducative à domicile. 

% 1935 : 
Le \emph{vagabondage} des mineurs cesse d'être un délit, (contrairement à la mendicité et au racolage qui demeurent des délits). 

\item[1938] La femme mariée acquiert certains des droits des femmes célibataires : droit à une carte d'identité, à un passeport, à ouvrir un compte en banque sans l'autorisation de son époux.

\item[1941] Allocation de salaire unique \emph{(parmi les textes promulgués sous l'occupation, ne comptent que ceux qui ont été confirmés à la Libération : plusieurs d'entre ces derniers avaient été préparés bien avant la guerre)}. 

% 1941 : 
Ouverture des hôpitaux à tous, quels que soient leurs revenus : depuis longtemps déjà les hôpitaux recevaient des malades qui payaient leur séjour et les soins qui leur étaient dispensés. Ils payaient eux-mêmes ou c'est un tiers qui le faisait : militaires (dès l'ancien régime), accidentés du travail (1897), assurés sociaux (1928),~etc. En 1900 cela contribuait pour 20~\% aux recettes des hôpitaux. En 1940 40~\% des hospitalisés donnaient lieu à un remboursement. Le 21 décembre 1941 il est décidé d'étendre cette possibilité à tout le monde, sans maintenir d'exclusive. Comme bien des décisions de cette époque, ce n'était que la mise en œuvre de décisions préparées dés 1938, c'est pourquoi cette orientation n'a pas été remise en question à la libération.

\item[1943] La loi du 15 avril 1943 donne un droit aux secours aux enfants \emph{qui ont un père, même quand celui-ci est valide}. Le droit des parents au « dépôt » volontaire de leurs enfants à l'Assistance Publique est élargi.

\item[1944] Octroi du \emph{droit de vote} aux femmes. 

\item[1945] L'ordonnance du 2 février crée le corps des juges pour enfants, pour les jeunes de moins de 18 ans. Elle crée l'éducation surveillée à l'intention des mineurs délinquants.

\item[1946] Création des \emph{allocations prénatales}.

% 1946 : 
La constitution déclare \emph{égaux} les droits des hommes et des femmes.
\end{description}

\section{Séparation de l'église catholique et de l'État}


 La France n'était jusque là jamais sortie du cadre intellectuel et moral du catholicisme dans lequel elle s'était constituée, sauf durant quelques années pendant la Révolution française. Les mouvements de laïcisation de la fin du Moyen Âge et de la Renaissance, comme ceux du \siecle{18}, avaient travaillé sur les limites entre ce qui revenait aux pouvoirs civils et ce qui revenait au personnel ecclésiastique, mais ils n'avaient pas fondamentalement mis en question la place de la religion catholique comme source du Droit. Le Concordat de \hbox{Napoléon} avait remanié cette situation sans la modifier radicalement. Les autres confessions et les « sans religion » ne représentaient en 1804 qu'un très faible pourcentage de la population. Même si le degré d'identification des français à l'Église fluctuait beaucoup suivant les régions et les milieux sociaux, la population française était très majoritairement catholique. A la fin du \siecle{19} les religieux étaient bien plus nombreux qu'à la fin de l'ancien régime : {\emph{\nombre{81000} religieux en 1789, \nombre{13000} en 1808, \nombre{160000} en 1878}%
% [1]
\footnote{Christian \fsc{SORREL}, \emph{La République contre les congrégations – Histoire d'une passion française 1899-1904}, éd. du Cerf 2003, p. 12. Dans \emph{L'ancien régime, institutions et sociétés} (Le livre de poche, 1993, p.68) François \fsc{BLUCHE} donne des chiffres différents, mais du même ordre de grandeur pour les religieux : \emph{le monde ecclésiastique comprenait, à l'extrême fin de l'ancien régime, un peu moins de \nombre{140000} membres. Le clergé régulier (religieux et religieuses, moines et moniales) regroupait quelques \nombre{59000} âmes (dont \nombre{28000} femmes)... Le clergé séculier représentait quelque \nombre{80000} hommes d'Église (\nombre{139} prélats, environ \nombre{10000} chanoines et les \nombre{70000} prêtres assurant le culte des \nombre{40000} paroisses).}}%
}... La Révolution avait supprimé les monastères et les couvents, et confisqué tous leurs biens, et le Concordat n'avait prévu aucun cadre juridique pour les congrégations religieuses. Et pourtant d'innombrables congrégations nouvelles avaient été créées durant tout le siècle, tandis que beaucoup parmi les anciennes s'étaient relevées de leur état de langueur du \siecle{18}. Les « congréganistes » s'investissaient d'abord et avant tout dans l'enseignement, alors en plein essor, notamment dans le primaire, et aussi et comme toujours dans les services hospitaliers, eux aussi en expansion : en 1847 il y avait en France plus de sept mille religieuses hospitalières, à la fin du Second Empire plus de dix mille, en 1905 plus de douze mille.

 C'est justement là que le bât blessait : le programme des républicains qui avaient conquis le pouvoir en 1879 faisait de la solidarité et de l'enseignement des outils essentiels de gouvernement%
% [2]
\footnote{Cf. \emph{L'invention du social, essai sur le déclin des passions politiques}, Jacques \fsc{DONZELOT}, 1994.}% 
, et il n'était pas question pour eux de les laisser aux mains des employés permanents de l'Église. Ils voulaient retirer à celle-ci les points d'appui institutionnels sur lesquels elle avait assis son influence depuis Constantin. À partir du moment où la gauche radicale l'a emporté dans les urnes, l'histoire des familles comme celle du traitement de la pauvreté a changé de direction. La nouvelle majorité s'est donnée pour mission des tâches traditionnellement dévolues à la Providence%
%[3]
\footnote{Le terme « Providence » est l'un des noms de Dieu.}%
. Au lieu de déplorer les malheurs et les injustices de la \emph{vallée de larmes} où vivraient les hommes, tout en comptant sur un \emph{au-delà} paradisiaque ou infernal pour régler à chacun son compte, ses membres ont estimé du devoir de l'État de s'attaquer lui-même aux sources des malheurs individuels, et d'abord aux injustices sociales, sans se reposer sur les initiatives privées, expressions de la Providence, et de procurer aux citoyens sinon le bonheur du moins un droit effectif à une aide efficace, afin de prévenir le malheur quand c'est possible et de soulager les souffrances quand cela ne l'est pas. 
 Ils s'attaquaient résolument et en pleine connaissance de cause à son autorité sur les esprits.

 À côté de mesures de portée limitée ou relativement symbolique%
% [4]
\footnote{Exemples : suppression de l'obligation du repos dominical (rétabli dès 1906 sous la pression des associations ouvrières) ; sécularisation des cimetières ; suppression des prières publiques constitutionnelles et de tous les signes religieux présents dans les lieux publics ; imposition du service militaire aux religieux et aux séminaristes ; exclusion des membres du clergé des commissions d'enseignement des hôpitaux en tant que membres de droit : curés chargés d'une paroisse, évêques,~etc.}% 
, les républicains ont d'emblée exclu les facultés de théologie des universités publiques, et le personnel religieux du corps enseignant universitaire. L'institution de l'obligation scolaire jusqu'à 12 ans était devenue inéluctable à cette époque%
%[5]
\footnote{Selon \fsc{FURET} et \fsc{OZOUF}, dès le milieu du \siecle{19} près des trois quarts des enfants français sont scolarisés. Chaque commune était depuis Guizot astreinte à l'obligation de fournir une école primaire publique à ses habitants, mais pas à en garantir la laïcité, d'ailleurs souvent refusée par la majorité de la population, comme la suite de l'histoire l'a montré. Le nombre d'enfants scolarisés en 1850 dans les écoles primaires (héritières des petites écoles des siècles précédents) représentait 73~\% du nombre des enfants de la tranche d'âge des 6-13 ans. Il en représentait même 105~\% en 1876-1877 : plus de 100~\%, ce qui s'explique par les enfants scolarisés avant 6 ans et après 13 ans (\fsc{FURET} et \fsc{OZOUF}, 1977, p. 173). Par conséquent en 1880 les enfants d'âge scolaire non scolarisés ne représentaient plus qu'une petite minorité. Mais ces chiffres moyens couvraient des disparités extrêmement grandes :
%\begin{itemize}
\begin{enumerate}[label=\alph*.,itemsep=0pt]
%A)
\item entre régions (le nord et l'est étaient très scolarisés depuis des siècles, au contraire du sud et de l'ouest, très peu scolarisés),
% B)
\item entre villes et campagnes,
% C)
\item parmi les régions rurales elles-mêmes, entre celles de civilisation exclusivement orale comme la Bretagne (valorisant la parole « vivante », et se défiant de la parole « morte », c'est-à-dire écrite), le Pays Basque, la Catalogne,~etc. et celles (de langue française) largement pénétrées par l'écrit,
% D)
\item et au moins autant entre classes sociales.
\end{enumerate}
%\end{itemize}

 Rien ne permettait de penser que ces petites minorités réfractaires à l'école d'alors étaient prêtes à rejoindre spontanément et rapidement le mouvement général, ce qui justifiait d'obliger par la loi les parents à scolariser leurs enfants.}% 
, mais il n'en était pas de même de la laïcité de l'enseignement. Celle-ci était évidemment une arme contre l'Église et son influence dans le domaine scolaire. Alors qu'à cette époque les femmes étaient les plus fidèles soutiens de l'Église, les républicains ont créé pour elles un enseignement secondaire public et laïque similaire en (presque) tout point à celui des garçons. Il s'agissait à la fois de lutter contre l'influence des congrégations en leur interdisant tout enseignement avant de les expulser, et contre la vision traditionnelle d'une femme soumise au contrôle masculin pour l'accès au savoir et à la culture. 

 En légalisant le divorce en 1884, les républicains affranchissaient le mariage civil des règles du Droit Canon. En 1904 l'adultère cesse d'être une faute contre la société dans son ensemble et n'est plus qu'une affaire privée : une fois libérés de leurs unions antérieures, les amants adultères ont le droit de s'unir légalement, ce qui leur était interdit à vie depuis l'empereur Justinien --- interdit qui les empêchait de légitimer leurs enfants déjà nés (adultérins) et leurs enfants encore à naître. 

 Le titre III de la loi 1901 sur les associations a refusé aux congrégations religieuses la liberté d'association. Par conséquent à partir de sa promulgation toutes les congrégations existantes ont été dans l'obligation d'obtenir une autorisation législative, ce dont elles s'étaient le plus souvent passées depuis le début du \siecle{19}. Cette autorisation a été refusée à la grande majorité d'entre elles, \emph{ipso facto} dissoutes. Les congrégations enseignantes, dont les effectifs étaient de beaucoup supérieurs à celui des religieuses hospitalières, ont toutes été interdites%
% [6]
\footnote{Sur un nombre de plus de \nombre{1300} congrégations, \nombre{140} congrégations masculines et \nombre{888} congrégations féminines ont été dissoutes. Cela a concerné plus de cent cinquante mille personnes dont 80~\% de femmes...}% 
. Seules ont été épargnées les congrégations hospitalières ... et toutes les congrégations implantées aux colonies. 

 La laïcisation des hôpitaux a commencé dès 1879, mais elle ne pouvait se faire qu'au rythme de la formation du personnel laïc d'encadrement et infirmier, ce qui demandait d'abord de créer les écoles d'infirmières et de surveillantes nécessaires, puisque les noviciats des congrégations hospitalières en avaient jusque là tenu le rôle. On observe quelques créations vers 1880, puis en 1899 est prise la décision de créer une école d'infirmière dans toutes les villes de faculté. Ceci étant dit la laïcisation de chaque institution dépendait d'abord et surtout de la couleur politique du conseil municipal dont elle dépendait.

 En 1900 les religieuses formaient la plus grande part du personnel soignant, par contre en 1975 l'ensemble du personnel soignant (sans compter les autres employés des hôpitaux) se comptait à près de \nombre{300000} personnes. Les religieuses ne représentaient plus à cette date qu'une toute petite minorité vieillissante. C'est qu'il ne s'agissait plus des anciens hôpitaux et hospices voués essentiellement aux indigents. Désormais il s'agissait d'établissements industriels, la réalisation concrète des « machines à guérir » dont les penseurs de la fin du \siecle{18} avaient rêvé. Les clients avaient complètement changé, et l'échelle aussi : en 1975 on trouvait dans les hôpitaux publics plus de \nombre{10000} (dix mille) médecins des hôpitaux à plein temps et \nombre{26900} internes, beaucoup plus qu'il n'y avait de personnels soignants, tous statuts confondus, dans tous les hôpitaux du \siecle{19}%
% [8]
\footnote{Cf. Jean \fsc{IMBERT}, \emph{Histoire des hôpitaux en France}, 1982.}% 
.
\section{Critiques de gauche et anarchistes de la famille}

 

Selon Jacques \fsc{Donzelot} depuis la Belle Époque des « militants », qu'il classe parmi les anarchistes ou à côté d'eux, ont mis {\emph{"...en place les petites machines de guerre contre la famille \emph{[... que sont]} la célébration de l'union libre, \emph{[...]} la distribution des produits anticonceptionnels et \emph{[...]} la propagande pour la grève des ventres}\footnote{Idem p. 163.}.}\footnote{Jacques \fsc{DONZELOT}, \emph{La police des familles}, 1977, 220 pages. Chapitre 5, « La régulation des images », p. 154 à 211.}" 
Parmi eux on trouvait, à côté des militants de base, des médecins comme Adolphe \fsc{Pinard}, des écrivains comme Octave \fsc{Mirbeau}, des hommes politiques de gauche comme Léon \fsc{Blum}, des savants comme Paul \fsc{Langevin}, soucieux {\emph{"...d'incorporer l'hygiène et donc le contrôle des naissances dans le fonctionnement des institutions."}} On trouvait également en première ligne la \emph{Ligue des droits de l'homme} et la \emph{Société de prophylaxie sanitaire et morale}, dirigées toutes deux par le docteur \fsc{Sicard~de Plauzolles}. Ils s'exprimaient dans divers ouvrages tels que \emph{La fonction sexuelle} (1908) du même docteur, ou \emph{Du mariage} de Léon \fsc{BLUM} (1908). 

Leur discours \emph{"...est à peu près celui-ci : puisque la famille est détruite par les nécessités économiques de l'ordre social actuel, il faut que la collectivité remplace le père pour assurer la subsistance de la mère et des enfants. Au père se substituera ainsi la mère comme chef de la famille ; puisqu'elle en est le centre fixe, la matrice et le cœur, elle en sera la tête. Les enfants seront sous sa tutelle, centralisée par l'autorité publique. Tous porteront le nom de leur mère ; ainsi les enfants nés d'une même femme mais de pères différents auront le même nom ; aucune différence n'existera plus entre légitimes et bâtards. L'influence de l'homme sur la femme et sur les enfants sera en rapport avec l'amour et l'estime qu'il inspirera ; il n'aura d'autorité que par sa valeur morale : il n'aura de place au foyer que celle qu'il méritera..\footnote{Idem, p. 164.}.} 

 Pour les plus radicaux de ces théoriciens, tels l'avocat Ernest \fsc{TARBOURIECH} (in \emph{La cité future, essai d'une utopie scientifique}, 1902), socialiste marxiste et collectiviste : \emph{"La puissance paternelle aura disparu... Le père et la mère n'auront sur leur progéniture aucun droit d'aucune sorte, mais seulement des devoirs qui peuvent ainsi se formuler : aider l'état dans la tâche qui lui incombe vis à vis des jeunes générations. L'éducation et l'instruction, affaires d'état, seront réglées souverainement par l'état au mieux. Les médecins représentant la communauté confieront chaque enfant à la personne qui donnera les soins les plus tendres et les plus éclairés. La loi présumera que cette personne est la mère mais cette présomption si naturelle... ne sera pas... de Jure... mais... susceptible de preuve contraire.
 ...l'autorité médico-judiciaire pourra intervenir}..." à tout instant jusqu'à la majorité du mineur (p. 309). 
 
 En résumé  c'est l'État, détenteur des moyens de production et pourvoyant aux besoins de la totalité de la population, active comme inactive, qui gère les effectifs de ses employés. Il dirige donc la reproduction et l'éducation. Il ordonne l'euthanasie des nouveaux-nés jugés par une commission scientifique (le "juge médical") mal conformés, "vicieux", "tarés", ou voués au crime ou à l'impuissance économique \emph{"...pendant cette période où ils ne sont pas encore une personnalité."} (p. 397). Le même "juge médical" accorde ou refuse aux individus le droit à une vie sexuelle. L'état prescrit de déclarer toutes les grossesses et il les surveille. C'est lui qui décide si la génitrice est apte à collaborer avec l'état dans la mission d'élever le futur citoyen. Il peut à tout moment la remplacer au profit d'un éleveur ou d'un éducateur offrant plus de garanties qu'elle (éventuellemet le géniteur de l'enfant). Il s'agit pour \fsc{tarbouriech} d'étendre à toute la société le régime de la tutelle, et à toutes les mères l'attribution des secours éducatifs et du contrôle sanitaire, afin qu'elles soient payées comme nourrices de leurs propres enfants et qu'elles les élèvent non pour elles mais pour l'État et sous son controle. Il étend à tous les enfants le régime des "enfants assistés" (pupilles)  de l'Assistance Publique : \emph{"Bref,} selon \fsc{Donzelot}  \emph{...une gestion médicale de la sexualité libérera la femme et les enfants de la tutelle patriarcale, cassera le jeu familial des alliances et des filiations au profit d'une emprise plus grande de la collectivité sur la reproduction et d'une prééminence de la mère. Soit un féminisme d'état\footnote{Une telle utopie serait-elle vraiment un "féminisme (d'Etat)" ? Il faut n'avoir aucune idée de la dépendance des "nourrices" de l'Assistance face à l'administration pour le croire. La responsabilité de l'éducation reviendrait entièrement à l'Etat, et la puissance enlevée aux pères ne serait pas donnée aux mères, qui ne détiendraient qu'une délégation d'autorité parentale, révocable à tout instant. Face à leurs enfants elles auraient moins de garanties juridiques qu'avec le Code Napoléon, tout patriarcal que soit celui-ci.}."\footnote{Idem, p. 164.}}
 



 De la Belle Epoque à la fin du baby-boom les néo-malthusiens se sont opposés aux « populationnistes ». Ceux-ci se recrutaient dans la bourgeoisie traditionnelle, attachée pour de multiples raisons à la transmission de son patrimoine, mais aussi parmi {"[...] \emph{les ligues de pères de famille, la Ligue des mères de familles nombreuses, l'Association des parents d'élèves des lycées et collèges, l'École des parents, l'Union des assistantes sociales, les organisation scoutes, les ligues d'hygiène morale, d'assainissement des kiosques de journaux, des abords des lycées,~etc}\footnote{Idem, p. 162.}".} Les membres de ces groupes de pression défendaient la répartition traditionnelle des rôles sexués et des pouvoirs au sein de la famille. Ils pensaient en effet que plus la structure familiale était forte, plus elle avait de chances d'être prolifique, et de bien réaliser sa mission éducative. Ils luttaient {\emph{"...contre tout ce qui peut fragiliser la famille : le divorce, les pratiques anticonceptionnelles, l'avortement."}\footnote{Idem.}}




 Bien des mesures décidées à cette époque par la Gauche au pouvoir allaient dans le sens des néo-malthusiens et contre les populationnistes. Le divorce%
% [8] 
\footnote{... pour faute seulement, parce que l'opinion d'alors n'acceptait pas d'autre motif, mais divorce tout de même : d'où jusqu'aux années 1970 tout un folklore de manœuvres vaudevillesque pour fabriquer en commun une « faute » légalisable (lettres d'injures...) même quand les conjoints étaient d'accord sur l'objectif.} 
permettait aux épouses maltraitées, délaissées ou bafouées, de sortir de la prison où le mariage les retenait jusque là. 

 Le premier objectif de l'obligation scolaire était certes de répandre le savoir, la culture commune, et de ne laisser personne à l'écart de cette richesse, mais un effet pleinement assumé, et même désiré, de cette obligation était aussi de contraindre toutes les familles à accepter l'entrée en leur sein de points de vue extérieurs. Elles ne pouvaient plus élever leur enfant à l'écart du monde. 

 La création d'un enseignement secondaire public pour filles calqué sur celui des garçons promouvait l'égalité complète des filles et des garçons, même si en 1880 on en était loin. C'était un choix historique, une rupture dans la répartition sexuée traditionnelle des tâches et compétences. 

 L'obligation scolaire interdisait aux parents de placer leurs enfants chez un employeur avant leurs 12 ans (11 ans s'ils avaient obtenu le certificat d'études) ou de les employer eux-mêmes à plein-temps%
% [9]
\footnote{Après le rapport de \fsc{VILLERMÉ} sur le travail des enfants, la loi du 22 mars 1841 avait fixé pour la première fois une limite d'âge au-dessous de laquelle il était interdit aux employeurs, et donc (indirectement) aux parents, de mettre les enfants au travail. La première borne avait été posée à l'âge de 8 ans. Elle avait été plus ou moins respectée mais ce n'en était pas moins le début d'une lente progression. La loi sur la scolarité obligatoire s'inscrivait comme une nouvelle étape dans cette progression, et l'exploitation du travail de l'enfant par ses parents commençait d'apparaître comme une forme de maltraitance.}% 
. 

 Quant au droit des femmes mariées à gérer leur propre salaire, c'était une part de souveraineté symbolique en moins pour les maris. En fait dans bien des ménages populaires c'étaient les femmes qui tenaient les cordons de la bourse, d'un commun accord entre conjoints (en a-t-il toujours été ainsi ? Les épouses semblent être presque toujours chargées de gérer les réserves, les resserres et les greniers, ce que symbolise le fait qu'on leur confiait les clés).

 À partir de 1912 les enfants sans père reçoivent le droit de demander des aliments à leur géniteur \emph{(recherche en paternité naturelle)}. Une mère célibataire n'est plus sans recours devant celui qui l'a laissée seule avec son enfant, qu'elle représente devant la justice. Cela entraîne pour corollaire qu'une femme mariée n'est plus aussi à l'abri qu'avant des conséquences matérielles et sociales des frasques pré ou extra conjugales de son conjoint. Pour autant un enfant illégitime ne peut toujours pas hériter de son père. 


\section{Mise en question du droit de correction}

 À aucun moment de l'histoire les parents n'ont été autorisés à faire subir à leurs enfants \emph{tout} ce qu'ils pouvaient imaginer. La tolérance à leurs abus de pouvoir a pu varier au fil des siècles, mais ils n'ont jamais eu le droit de les estropier, pas plus qu'ils n'avaient le droit d'estropier les enfants des autres, ni d'en faire leurs partenaires sexuels. Mais la Justice a toujours beaucoup de difficultés à les poursuivre lorsque les traces des sévices ne se voient pas, ou dans les cas de négligence simple, d'abandon moral. En faisant du délaissement et de la maltraitance des délits, la loi {\emph{sur la protection judiciaire des enfants maltraités et moralement abandonnés}} a permis de prononcer la déchéance des droits parentaux pour ces seuls motifs. 

 Le fait de ne pas juger les mineurs et les majeurs selon les mêmes critères est sans doute aussi vieux que la justice elle-même. Refuser de traiter les fautes des mineurs autrement que celles des adultes serait manquer de bon sens. Ce qui fait la différence, c'est l'âge de la coupure entre l'irresponsabilité complète, l'atténuation de la responsabilité \emph{(l'excuse de minorité)} et la responsabilité pleine et entière. Ce sont aussi les peines encourues : nature des peines, durée... La minorité pénale était fixée à 16 ans depuis l'ancien régime. La loi du 12 avril 1904 la repousse de 16 à 18 ans, et elle affirme la prééminence de l'éducatif sur le répressif.

 La loi du 28 juin 1904 s'inscrit dans le courant d'idées qui a confié les enfants \emph{moralement abandonnés} à l'Assistance Publique. Elle ordonne que les pupilles \emph{difficiles} soient confiés non plus à des prisons, mais à des écoles professionnelles publiques ou privées. Ce texte confirme à l'administration du service des enfants assistés, détentrice de la puissance paternelle sur les pupilles, le droit de désigner ceux qu'elle garderait et ceux qu'elle refuserait d'assumer et pour lesquels elle solliciterait l'aide de la Justice. Au même moment c'étaient encore en principe les pères qui définissaient ce qui sous leur toit était indiscipline et insoumission à leur autorité.

 La loi du 24 juillet 1889 {\emph{sur la protection judiciaire des enfants maltraités et moralement abandonnés}} donne aux juges la possibilité de prononcer la déchéance totale de la puissance paternelle pour inconduite des parents, en cas de mauvais traitements ou de délaissement de l'enfant (et de plein droit dans le cas de certaines condamnations infamantes). La même loi confie à l'administration (c'est-à-dire à l'Assistance Publique) la tutelle des enfants maltraités, victimes de crimes ou de délits ou délaissés. Le service les prend en charge même s'ils sont âgés de plus de 12 ans à leur entrée. Ces enfants deviennent des pupilles comme les autres. Ils sont traités à l'instar des autres enfants du service. Quel que soit leur âge, autant que faire se peut ils seront placés en nourrice, pour de longues durées, et dans tous les cas ils seront totalement coupés de leurs parents déchus.

 Une nouveauté majeure est introduite en 1912, avec la création des tribunaux pour enfants (sans magistrats spécialisés) et la création de la liberté surveillée%
% [10]
\footnote{Cf. l'ouvrage collectif \emph{Protéger l'enfant} (1996), qui aborde les problèmes de la jeunesse sous l'angle de la \emph{protection judiciaire}. Il présente un résumé de l'histoire de celle-ci, et des débats d'idée et des conflits de pouvoir qui ont présidé à sa naissance et qui la traversent encore...}%
. Cette nouveauté avait été précédée depuis les années 80 par tout un mouvement d'idées, notamment chez les magistrats chargés de l'application du droit de correction paternelle. Il y a en effet un lien direct entre la dénonciation de l'indignité des pères (cf. la loi de 1889 sur la déchéance paternelle), et la mise en cause du droit de correction%
%[11]
\footnote{Pascale \fsc{QUINCY-LEFEBVRE}, « Une autorité sous tutelle. La justice et le droit de correction des pères sous la troisième république », in \emph{Lien social et politiques-RIAC}, 37, Printemps 1997, p. 99 à 109.}% 
. Ceux qui s'intéressaient à ce problème ne contestaient en aucune façon l'existence d'enfants \emph{insoumis}, difficiles à élever et qui provoquaient le \emph{légitime} mécontentement de leurs parents. Ils estimaient par contre que c'était un problème qui débordait le cadre familial, parce qu'on pensait qu'en règle générale ceux qui étaient insoumis à leurs parents ne faisaient pas de bons citoyens, et risquaient de devenir délinquants, c'est pourquoi l'état ne pouvait s'en désintéresser. Ils estimaient surtout qu'il n'était pas possible de s'en tenir à la parole du parent, et qu'il fallait s'assurer par une enquête approfondie de la réalité et de la nature des problèmes. 

 D'autre part ils estimaient que la prison n'était pas un outil de correction efficace, et qu'il fallait fournir aux jeunes insoumis une prestation éducative de durée suffisante pour obtenir d'eux un amendement réel. Ils pensaient que cette prestation devait être fournie par un internat sous le contrôle de la Justice et non sous celui des pères. Ils accusaient en effet ceux-ci (ceux du moins qui réclamaient à la justice son aide, c'est-à-dire ceux des pères, tous de milieu populaire, qui ne pouvaient supporter les frais d'une pension dans l'un des internats privés dont c'était la spécialité) d'être trop prompts à retirer leurs enfants (comme ils en avaient le droit) dès que ceux-ci semblaient suffisamment \emph{intimidés} par l'incarcération. Ils les suspectaient de n'avoir qu'un seul but, celui de mettre le plus vite possible leurs enfants au travail pour toucher leur salaire. Aux yeux des réformateurs, les droits des pères (éducatifs ou financiers) importaient moins que l'intérêt des enfants, qui était de recevoir une bonne éducation durant le temps nécessaire et avec la sévérité qui convenait, et que l'intérêt de la société, qui était de voir conduire à son terme la \emph{correction des insoumis}. 

 C'est donc du fait des juges et non à la demande de la société que la correction paternelle est peu à peu tombée en désuétude. Ils ont pris l'habitude dès les années 1890 de demander systématiquement une enquête pour vérifier si le parent demandeur avait vraiment des \emph{sujets de mécontentement très graves}, et s'il n'était pas plutôt un parent \emph{indigne}. Ils ont ainsi retiré aux parents leur droit de qualifier eux-mêmes de fautifs les comportements de leurs enfants. Puis la loi de 1889 leur a donné la possibilité non seulement de refuser aux parents indignes une demande de correction paternelle, mais encore de leur retirer la garde de l'enfant. Enfin la loi de 1904 les a explicitement autorisés à mettre les pupilles indisciplinés en maison de correction pendant plusieurs années (c'était déjà le sort des pupilles indisciplinés ou récalcitrants du \siecle{19}). Ces pupilles pouvaient être les enfants de parents déclarés \emph{indignes} une fois que leur déchéance était prononcée. Les parents étaient souvent qualifiés d'indignes parce qu'ils laissaient la bride sur le cou de leur enfant en ne le contrôlant pas d'assez près, ou parce qu'ils entravaient les efforts des éducateurs qui tentaient de les amender. Il semble qu'à cette époque les juges et les premiers travailleurs sociaux déploraient plus leur laxisme que leur autoritarisme. 

 En 1921 une loi ouvre la possibilité de prononcer une \emph{déchéance partielle} de l'autorité paternelle. Une déchéance totale des droits parentaux était une mesure aux effets quasi irréversibles. À partir de 1921 les magistrats n'ont plus été réduits au tout ou rien d'une telle mesure face aux parents qu'ils jugeaient incompétents, délinquants ou négligents. Au contraire ils pouvaient prononcer une déchéance partielle et provisoire, non seulement là où la déchéance totale aurait été injustifiée, mais même là où ils y auraient recouru par nécessité en l'absence d'une mesure plus souple. Le nombre de ces décisions a donc crû rapidement. Cela ne s'est pas traduit par un accroissement important du nombre de jeunes placés, mais par un changement du statut de beaucoup d'enfants placés : le nombre des pupilles a décru au fur et à mesure qu'augmentait celui des enfants en garde, sans que l'effectif total ne se modifie sensiblement. Pendant ce temps le nombre des abandons ne cessait de diminuer.

 On a vu que dès la fin du \siecle{19} des juges avaient commencé d'ordonner des enquêtes pour évaluer la pertinence des demandes de correction paternelle. En 1923 le succès de cette pratique a conduit à la création à Paris, où étaient traitées les deux tiers des demandes de correction paternelle faites en France, d'un service social réalisant pour le tribunal des enquêtes débordant largement la matérialité des faits reprochés par les parents à leur enfant. Désormais la demande d'intervention des parents était entendue comme l'expression d'un dysfonctionnement dans la famille, qui dépassait largement le mineur concerné. Cela entraînait une enquête sociale, c'est-à-dire l'introduction au sein de la famille, d'un observateur extérieur mandaté par les juges. À partir de cette base ces derniers se sont donné le droit de conseiller les parents face aux problèmes que leur posaient leurs enfants : dans la plupart des cas cela les conduisait à mettre en œuvre une action non judiciaire, confiée sous leur contrôle à des institutions privées. Il s'agissait très souvent d'une \emph{action éducative en milieu ouvert}, mais ils pouvaient aussi prendre l'initiative de placer en établissement de correction les mineurs qui leur semblaient en avoir besoin, entre autres au titre des lois de 1889 et de 1921 sur la déchéance paternelle, et de 1904 sur les pupilles difficiles ou vicieux. Tous ces placements écartaient le contrôle paternel.

 Les magistrats n'accédaient plus à la demande de correction paternelle que dans un nombre de cas de plus en plus petit : environ un cas sur quatre ou cinq en 1917, un sur dix dans les années trente. Ils ont ainsi vidé de sa substance le droit de correction paternelle. Le nombre des mineurs placés à ce titre n'a donc cessé de baisser jusqu'à devenir marginal, comme le montre la table \vref{ord-corr-pat}.
 
\begin{table}[h]
\centering
\caption{Ordonnances de correction paternelle}
\label{ord-corr-pat}
\begin{tabular}{ccc}
 & France & part \\
Année & entière & Seine \\
\hline
1881 & 1192 & 63,7~\% \\
1891 & 737 & 59,6~\% \\
1901 & 731 & 50,0~\% \\
1911 & 644 & 66,9~\% \\
1921 & 270 & 83,3~\% \\
1931 & 60 & 66,7~\%
\end{tabular}
\end{table}
 
 Le décret-loi du 30 octobre 1935 sur {\emph{la correction paternelle et l'assistance éducative}} institue l'assistance éducative à domicile. Il entérine les changements qui travaillaient depuis deux générations l'exercice du droit de correction paternelle. Il a également dépénalisé le vagabondage des mineurs (les fugues simples, sans délits caractérisés) ce qui suggère que ces deux ordres de faits se recouvraient. Désormais les jeunes vagabonds ne ressortissaient plus de la Justice, mais mais d'une assistance placée sous le contrôle des juges. 

\section{Vers une assistance non punitive ?}

Les mesures d'assistance en faveur des femmes en couche et des familles nécessiteuses, mais aussi les allocations ouvertes à toutes les familles (allocations familiales, salaire unique, allocations prénatales, etc.), les consultations de nourrissons, les crèches, et les améliorations progressives des conditions de travail, tout cela a facilité la vie des familles. Depuis le début du siècle le nombre des abandons a décru régulièrement et massivement.


Depuis le milieu du \siecle{19}, l'administration pouvait verser aux mères seules une aide afin qu'elles placent elles-mêmes leur enfant chez une nourrice de leur choix. Dans le même esprit, le placement chez une nourrice directement salariée par l'assistance publique a de plus en plus souvent été perçu comme une forme de secours à la famille, et non plus comme le remplacement d'une famille par une autre. On aidait la mère en lui fournissant une nourrice, là où les familles citadines non indigentes de l'époque se la procuraient elles-mêmes. À partir de 1924, l'assistance publique de Paris a commencé de placer en nourrice des enfants non abandonnés, dont les parents n'étaient pas déchus de leurs droits, des enfants qui n'étaient pas des pupilles. 

 Pour cela, il avait fallu franchir une barrière psychologique et oser placer en nourrice pour une durée indéterminée, l'enfant qu'une femme pourrait reprendre un jour. Cela allait contre les pratiques antérieures de l'assistance publique, mais (et ce n'est sans doute pas un hasard) c'est également à partir de l'année 1924 que la loi a permis d'adopter les enfants mineurs. C'est à partir de cette date que le service a la possibilité de procéder à des adoptions d'enfants abandonnés. Face à la réalité d'adoptions authentiques (quel que soit leur nombre réel, quelques centaines par ans semble-t-il), l'illusion que le placement dans une famille nourricière salariée était une espèce d'adoption ne pouvait plus tenir. Il ne pouvait plus être question pour un « nourricier » de prendre la place d'un parent dans le cœur de l'enfant, mais seulement de fournir à ce dernier une assistance pendant un temps plus ou moins long. 

 Au motif que les liens avec leurs parents n'étaient pas coupés, l'administration pouvait laisser les petits enfants concernés en collectivité (on a vu que c'était sa position traditionnelle face aux enfants qui avaient des parents), mais :
\begin{enumerate}
%a)
\item elle priverait alors autant de nourrices de leur emploi alors que les régions pauvres où elles vivaient avaient besoin de ces emplois et que le nombre des enfants abandonnés avait déjà beaucoup baissé depuis le début du siècle,
% b)
\item les nourrices coûtaient moins cher que les internats,
% c)
\item d'autre part, et surtout, on savait qu'en collectivité l'état de santé des petits enfants (0~à 4 ans) se dégrade inexorablement et rapidement au fil du temps, comme l'expérience l'avait régulièrement démontré depuis plusieurs siècles. Dès que le placement courait le risque d'être durable (plus de quelques semaines), il fallait donc autant que possible éviter aux plus jeunes les « dépôts » des enfants de l'Assistance Publique et les « orphelinats ».
\end{enumerate}
 Les jeunes concernés ont donc assez systématiquement été placés en famille d'accueil. On a rapidement observé, comme il était prévisible au vu de l'histoire antérieure de l'assistance, que le placement en famille d'accueil des tout-petits donnait de très bons résultats en ce qui concerne la santé physique et psychologique (indissociables à cet âge). Cette observation a assuré le succès de cette formule. Peu à peu les âges d'admission ont été assouplis. Les enfants qui ont des parents et qui gardent un lien avec eux ont pu entrer en famille d'accueil à un âge de plus en plus avancé, et tous ont fini par bénéficier de cette formule. 

 Le droit des parents au « dépôt » de leurs enfants à l'A.P. a été élargi par la loi du 15 avril 1943. Cette loi ouvrait un droit aux secours (dont le placement est l'une des formes possibles) aux enfants \emph{qui ont un père}, même quand celui-ci est valide et donc capable de travailler. Elle impliquait qu'un homme qui ne peut subvenir financièrement aux besoins de sa famille n'était pas pour autant disqualifié comme époux et comme père. Il n'avait pas pour autant à être sanctionné comme un débiteur insolvable à écarter de sa partenaire et de ses enfants. Il convenait plutôt de l'assister.

 On peut supposer que les séparations familiales et les privations de cette période de rationnement avaient facilité cette décision. L'un de ses objectifs a pu être de fournir une aide aux femmes et aux enfants des prisonniers de guerre retenus en Allemagne (valides certes, mais enfermés au loin et travaillant sans rémunération dans la tradition plurimillénaire de l'esclavage des vaincus). Mais il s'agissait aussi de reconnaître l'évolution des pratiques réelles des services. Preuve en est que cette réglementation n'a pas été abrogée après la guerre. 

 C'était aussi un corollaire du fait qu'on se libérait un peu de la représentation patriarcale du monde qui dominait les siècles précédents : si les hommes n'étaient plus les patriarches tout-puissants qu'on avait pensé qu'ils étaient (ou voulu qu'ils soient), leur impuissance financière ne justifiait plus leur éviction.
 
 \section{Construction de l'État-providence}
 
 À côté des drames qui en ont fait aussi une période tragique (deux guerres mondiales, diverses guerres locales, les guerres de décolonisation et au moins une grande crise économique) le demi-siècle qui va de 1910 à 1960 a vu la fin silencieuse d'une civilisation rurale millénaire (ce qui a représenté un drame d'une autre nature pour bien des gens) et révolutionné la vie quotidienne : création des média de masse, construction de banlieues concentrationnaires, progrès fulgurants de la lutte contre les maladies infectieuses, début de la croissance explosive qui a caractérisé les « trente glorieuses »,~etc. Il a vu l'essor du salariat, celui de multiples caisses d'assurances sociales et de retraite (déjà initié à petite échelle dès la fin du \siecle{19}) puis leur extension à l'ensemble de la population. Il a vu la création des allocations familiales. Il a vu le début de la démocratisation des enseignements secondaire et supérieur, qui étaient à l'époque des outils indiscutables d'ascension sociale. 

 Ces années étaient marquées par la conviction qu'il était possible d'aller vers un monde meilleur, lorsque les forces du mal seraient vaincues (guerres mondiales, guerres coloniales, capitalisme, fascisme, communisme, etc.) et ce monde semblait alors à portée de main. Apparue après la seconde guerre mondiale, l'expression \emph{État-providence} (en Angleterre le « \anglais{Welfare State} », état de bien-être par opposition à l'état de guerre) exprimait un aspect de ce projet : à défaut de faire descendre {\emph{ici-bas}} le paradis, procurer à tous au moins un solide filet de sécurité contre l'indigence et l'abandon social. 

 Cet effort de longue haleine, commencé par endroits dès le \siecle{19}, a obtenu des résultats très significatifs, grâce auxquels à partir de 1945 l'ensemble de la population européenne (entre autres), et en particulier les travailleurs les plus pauvres, a bénéficié d'assurances sociales et de retraites par répartition qui mutualisaient les risques. Ces systèmes imposés par les États rendaient en principe inutile le recours à l'assistance et à la bienfaisance comme la prise en charge des indigents par leur propre parentèle. 

 Le montant des aides financières accordées à tous les parents pour la prise en charge de leurs enfants a connu son apogée entre 1945 et 1965. Elles protégeaient de l'indigence les enfants des pauvres mieux qu'on ne l'avait jamais fait jusqu'alors.  
 
 Jusqu'aux années soixante du vingtième siècle, la législation, les mœurs, les discours dominants et le niveau élevé des aides matérielles à la famille et à la procréation, étaient conformes aux vœux des « populationnistes ». Les allocations familiales n'ont jamais été aussi fortes qu'alors par rapport aux salaires de base, ce qui a contribué à permettre à beaucoup de femmes de rester chez elles élever plus d'enfants qu'elles n'en auraient eu sans cela. Le marché de l'emploi aurait probablement permis à beaucoup d'entre elles de travailler au dehors de leur famille. Si les allocations étaient si substantielles, c'est bien parce que l'atmosphère nataliste d'alors était favorable à cette représentation des familles. Il s'agissait pour l'État de promouvoir les naissances, ce qui justifiait d'aider les familles prolifiques et de ne pas heurter de front les idées de ceux qui les représentaient. C'est dans cette atmosphère idéologique très favorable aux couples conjugaux, aux familles et aux associations qui les représentaient que les enfants du « \anglais{baby-boom} » ont été conçus et élevés. 
 
