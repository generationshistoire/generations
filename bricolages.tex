% \fsc comme avec efrench
\makeatletter
%
\def\@fscespace[#1 #2]%
{%
 \MakeUppercase{\@car#1\@nil}%
 \MakeLowercase{\@cdr#1\@nil}%
 \if\relax\detokenize{#2}\relax%
 \else%
  ~\@fscespace[#2]%
 \fi%
}
%
\newcommand\fsc[1]{\@fsctiret[#1-]}
\def\@fsctiret[#1-#2]%
{%
 \bsc%
 {%
  \@fscespace[#1 ]%
  \if\relax\detokenize{#2}\relax%
  \else%
   --\@fsctiret[#2]%
  \fi%
 }%
}
%
\makeatother


% D'après l'Imprimerie Nationale, les siècles s'écrivent avec des chiffres romains
% en petites capitales... Comme il n'y a pas de petites capitales italiques, on essaie
% d'utiliser dans ce cas des petites capitales penchées : c'est pas très différent
% Malheureusement, ça ne marche qu'avec la police Computer Modern (en tous cas pas avec Garamond)
%%%\makeatletter
%%%\newcommand*{\my@test@it}{it}
%%%\newcommand*{\crm}[1]%
%%%{%
%%%\ifx\f@shape\my@test@it%
%%% %[italique détecté]
%%%  \textsc{\itshape\romannumeral #1}\relax
%%%\else%
%%% %[italique non-détecté]
%%%  \textsc{\romannumeral #1}\relax
%%%\fi%
%%%}
%%%\makeatother

