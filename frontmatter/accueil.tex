

”\chapter{Présentation}







C'est au cours d'une recherche dans un autre champ que celui de l'histoire, au fil d'une thèse portant sur les jeunes placés dans le cadre de l'Aide Sociale à l'Enfance\footnote{\emph{Jeunes cas sociaux et difficultés éducatives}, Thèse de doctorat en psychologie, directeur de thèse Hervé Beauchesne, Université de Bretagne Occidentale, 1999.} que m'est apparue l'importance de la révolution actuelle des lois et des moeurs, et la nécessité d'un retour sur le passé des familles et de la reproduction pour comprendre notre présent. 

En effet après une très longue période de relative stabilité en Europe, depuis les années soixante du \siecle{20} la reproduction des humains
est  soumise à de tels bouleversements   
qu'il ne s'agit plus d'une évolution, mais bien d'une révolution, et pour le moment il
ne semble pas que soient parvenus à leurs termes les changements en
cours dans le droit et dans les mœurs. Le présent est donc instable, déroutant et difficile à penser. C'est dans ce genre de situation qu'un point de vue extérieur est utile pour se décentrer et comprendre un peu mieux où l'on en est. Un tel point de vue peut être fourni par les observations des ethnologues, sociologues ou meme démographes\footnote{Cf. les \emph{Métamorphoses de la parenté} de l'ethnologue Maurice Godelier (2004), ou bien \emph{L'origine des systèmes familiaux, T. I} (2011) ou encore \emph{Où en sommes-nous ? Une esquisse de l'histoire humaine} (2017) d'Emmanuel Todd, qui parcourt à la fois l'espace et le temps.}, mais il peut également être trouvé dans un retour sur l'histoire. La situation présente de la reproduction ne prend en effet tout son sens que par ses écarts avec les pratiques des siècles antérieurs.








Selon Quentin Meillassoux\footnote{Pour rédiger cette présentation je me suis particulièrement appuyé sur Quentin MEILLASSOUX : Anthropologie de l'esclavage, 1986, notamment sur son introduction.} Les membres des sociétés primitives se sentaient (et se sentent encore pour les quelques représentants de ces sociétés que la civilisation n'a pas encore arrachés à leur monde de représentations) liés par une continuité organique avec leur territoire et avec l'univers matériel dans son ensemble, avec les esprits dans (ou de) la nature, avec leurs ancêtres, avec le monde du ou des dieux. Souvent ils se pensent comme les seuls humains dignes de ce nom : chacun dans sa langue, ils se désignent eux-mêmes comme « les humains par excellence », ce qui implique que les autres, ceux qui leur sont étrangers, ne sont pas humains, ou pas vraiment humains, ou pas au même degré qu’eux. Pour eux la bonne vie, la seule vie vivable, n'est possible que sur le territoire dont ils ont hérité (meme s'ils pratiquent le nomadisme c'est dans des bornes relativement limitées). Partout ailleurs c'est l'inconnu, l'étrange, l'étranger, le non humain, l'inhumain.


Le plus souvent ces sociétés ne connaissent aucune forme d'écriture. Elles se caractérisent d'abord par l'absence d'échanges marchands et de moyens de paiement, comme par la faiblesse ou l'absence de leurs structures étatiques. Elles peuvent se choisir plutot démocratiquement un chef, mais son autorité est limitée. Elles vivent d'une économie de subsistance tournée vers l'auto consommation. L'accumulation des biens n'est pas pensée par elles comme la constitution d'un capital susceptible d'être réinvesti dans des opérations économiques nouvelles. Il s'agit plutôt d'acquérir des objets à haute valeur symbolique (religieuse, esthétique, magique, etc.), ou de constituer des réserves destinées à être consommées de manière festive ou/et ostentatoire. La nature leur donne ses fruits (maternellement) et c'est la fécondité de leur territoire qui limite la récolte et non le nombre de bras ou celui des heures de travail disponibles.


Dans ces sociétés la famille est le cadre essentiel, et parfois le cadre unique des rapports entre individus. Le chef de famille, presque toujours un homme, a pour première tâche de veiller à la pérennité de son groupe familial. La vie de chacun appartient au groupe, et il n'est pas question d'opposer à celui-ci les droits d'un individu particulier ni de mettre l'ensemble du groupe en danger pour un seul de ses membres. Si trop de naissances mettent en danger l'intérêt collectif le don des nouveaux nés excédentaires, leur abandon ou leur infanticide sont des pratiques ordinaires. En cas de disette il arrive que des vieillards se laissent mourir pour que les jeunes survivent, ou qu'on les y pousse.

La parenté assigne à chacun une fonction précise : des obligations mais aussi des droits sur les ressources du groupe. Aucun membre de la famille n'est exclu des redistributions, mais les parts peuvent être très inégalement distribuées, sans tenir compte de la contribution de chaque membre du groupe au volume des biens à répartir, ni de ses besoins réels, mais plutôt de son rang et de sa place symbolique. Il est fréquent qu’à ce compte les femmes soient mal loties, mais ce n’est pas systématique. La règle de base est que les adultes travaillent pour nourrir les plus jeunes et les plus vieux. Les plus jeunes reçoivent plus qu'ils ne donnent, jusqu'à ce qu'ils soient à leur tour capables de nourrir tous ceux qui les ont nourris. Les plus vieux sont directement ou indirectement nourris par ceux qu'ils ont élevés : chacun investit dans une descendance pour préparer ses vieux jours. Quand tout se passe harmonieusement c'est au fil d'une vie entière que les tâches et les droits s'équilibrent pour chaque individu.

Dans un tel système aucun garçon ne possède rien en propre : ni terres, ni troupeaux, etc. Si son clan refuse de lui procurer une femme, ou de lui donner les moyens d'en acquérir une, il reste bloqué dans un statut de dépendance (juvénile). Condamné à travailler toute sa vie pour les enfants des autres, il n'accèdera jamais au statut avantageux et respecté de ceux qui ont de grands enfants productifs.
Ce devenir concerne moins les filles. Étant donné le taux de mortalité qui frappe les femmes enceintes, les parturientes et les enfants, les sociétés primitives ont le plus souvent besoin que chaque femme ait tous les enfants qu'elle peut porter. Les familles ont trop besoin d'enfants pour ne pas marier leurs filles dès lors qu'elles sont nubiles, éventuellement à des hommes bien plus âgés qu'elles et même déjà dotés d'autres femmes.

Dans ces sociétés il n'y a pas de sens à faire une place à un étranger : à quel titre, au nom de quoi ? Et quel rôle lui donner ? Comment l'accueillir sans déséquilibrer le réseau compliqué et tendu des échanges et des obligations réciproques ? Du point de vue d'un guerrier l'étranger qu’il a capturé vivant est une preuve de sa valeur, mais il ne peut être un moyen d'entretenir et d'accroître sa puissance, un moyen de production de richesses (un esclave). Il n'est bon qu'à être rapidement consommé d'une façon ostentatoire : il n'est bon qu'à être sacrifié. Accepter de ses proches une rançon serait déjà entrer dans le monde marchand où une vie humaine a un coût et peut donc s'acheter ou s'échnger contre des biens réels, ce qui ne fait pas partie de leurs représentations.
Par contre s'il y a une place vacante dans une famille, celle-ci peut adopter un étranger ou une étrangère pour occuper cette place, afin que la vie continue, afin que les prestations masculines et féminines continuent d'être procurées, afin que les enfants continuent de naître, que les vieillards ne soient pas à l'abandon, que les ancêtres continuent d'être honorés, et que le monde continue sa course, etc. S'il n'y a pas assez d'épouses pour tous les garçons, on peut enlever des filles dans un autre groupe, ou leur en acheter. Un ennemi prisonnier peut d'autant plus facilement remplacer un mari ou un fils mort, que c'est ordinairement à ses voisins, à ceux que l'on pourrait épouser, qu'on fait la guerre.

En cas de conflit, de délit ou de crime, la mise au ban du groupe est d'autant plus fréquemment choisie qu'elle présente sur la mise à mort l'avantage d'éviter la souillure du territoire familial par un meurtre, ainsi que le ressentiment des ancêtres ou des dieux contre le ou les exécuteurs éventuels. Celui qui est condamné à l'exil est comme mort pour son groupe d'origine. S'il tombe aux mains d'un autre groupe, s'il est asservi (et a fortiori s'il leur était vendu) sa famille ne cherchera ni à le racheter ni à le délivrer. Ainsi le livre de la Genèse raconte comment Joseph, benjamin de Jacob, a été vendu par ses frères parce qu'ils étaient jaloux de voir qu'il était le préféré de son père. Leur première intention était de le tuer, mais comme une caravane de marchands passait par là cela leur a évité d'avoir à assumer la culpabilité de sa mort, et par dessus le marché la vente leur a rapporté de l'argent : l'Asie Mineure où cela se passait était en partie entrée dans le monde des marchands au moment où ce récit a été écrit. 
Si un individu banni est tué ses parents ne chercheront pas à le venger. Le jour où il mourra, les rites et sacrifices funéraires nécessaires au repos de son esprit ne seront pas exécutés. Il ne pourra pas rejoindre le monde de ses ancêtres et il ne sera pas rituellement nourri par les vivants. Son souvenir ne sera pas honoré. Cela l'exclura de son clan une deuxième fois. Aux yeux des intéressés l'errance et l'exil valent-ils mieux qu'une mort immédiate, mais au milieu des siens, dans son pays ?



A cette description Emmanuel Todd ajoute que selon lui l'organisation des familles primitives était et est encore la meme d'une extrémité de la terre à l'autre. Elle se caractérise par la vie des parents en couples stables, élevant eux-memes les enfants qu'ils ont conçus et respectant l'interdit de l'inceste. Le conjoint est choisi au sein du groupe de vie (au sens large) mais en dehors de la famille nucléaire. Les relations entre familles apparentées (frères et soeurs, beaux-frères et belles-soeurs) ont de l'importance étant donné le soutien mutuel qu'elles peuvent se fournir. Elles sont donc entretenues. Les familles du père et de la mère ont autant d'importance l'une que l'autre. Les règles de succession sont souples et il n'y a pas de souci d'égalité stricte ni de principe de primogéniture. Il n'existe pas de société réellement matriarcale mais le statut des femmes n'est pas dévalorisé, meme s'il peut y avoir à l'occasion de la polygamie. Selon Todd c'est ce type de famille qui prévalait en Europe de l'Ouest (Germanie, Gaule, Iles britanniques, Scandinavie...) avant l'entrée en scène des romains et les bouleversements de tous ordres qu'ils ont apportés. 

Les sociétés primitives ne faisaient pas le poids militairement face aux sociétés plus développées, telles que les cités antiques (grecques et romaines par exemple) et elles leur ont servi de réserve d'hommes, de femmes et d'enfants à capturer à intervalles plus ou moins réguliers (dès qu'elles avaient surmonté la crise démographique créée par le précédent prélèvement) pour en faire leurs esclaves. Ce système a duré très longtemps puisqu'à la fin du XIXème siècle on pouvait encore observer de telles chasses à l'homme.


A la différence des familles primitives la famille traditionnelle occidentale (pour fixer les idées : celle du milieu du XXème siècle) est une création. Elle est
née d'une synthèse entre les pratiques de l'Empire de Rome, celles des
juifs et celles des chrétiens de l'Antiquité. Ces pratiques et les représentations
qui les sous-tendaient étaient elles-mêmes le point d'aboutissement
d'autant d'évolutions particulières.
Les bases juridiques de la famille « traditionnelle » européenne ont été promulguées sous le règne
de l'empereur Constantin et celui de ses successeurs directs et  mais elles ont mis de nombreux
siècles à s'imposer, non sans résistances ni déformations multiples par
rapport aux desseins initiaux. La trajectoire de cette forme de famille n'a atteint son apogée qu'aux
siècles « classiques » de l'Ancien Régime, par contre dans les pratiques elle s'y est grosso-modo maintenue  (non sans quelques replâtrages) jusqu'au
\emph{baby-boom}.  Si ses bases juridiques ont été déconstruites à partir des années soixante du \siecle{20}, elle ne s'efface pourtant pas sans résistances, et elle n'est encore tout à fait morte ni dans les têtes ni dans les mœurs, même si dans le même temps de nouvelles formes d'union et de parentalité ont fait leur apparition et si de très anciennes problématiques que l'on croyait définitivement résolues reviennent au devant des préoccupations. 

C'est le panorama de cette très longue histoire que cet essai se propose de déployer, en soulignant les articulations et les ruptures, les
conflits, les crises et leurs enjeux. Ce champ est si vaste et la matière à y traiter  si démesurée que s'il s'agissait d'y trouver du nouveau une nombreuse équipe de chercheurs serait à peine à sa mesure. Il n'est donc pas question pour moi de prétendre concurrencer les historiens professionnels sur leur terrain. 
Cet essai est fondé sur les écrits de ceux qui ont abordé ce champ. Il s'agit pour moi de faire oeuvre de vulgarisation dans un domaine dont l'expérience m'a montré l'intérêt, et de fournir aux personnes intéressées les moyens de s'y repérer et d'aller plus loin si elles le désirent. Si ce texte parvient à exposer clairement la situation où en est aujourd'hui la reproduction humaine et à la problématiser, alors il aura atteint son but\footnote{L'histoire de la reproduction humaine interfère largement avec celle
de la prise en charge des personnes faibles, malades, âgées, infirmes ou
démunies, la famille, quelle que soit sa composition et son organisation, ayant été jusqu'au \siecle{20} la première institution
d'assistance, quand elle n'était pas la seule. Cet aspect de l'histoire sera donc évoqué
au passage succinctement. Pour aller plus loin sur cet aspect précis on
pourra se reporter entre autres à : Hervé Tigréat, Pascale Planche et Jean-Luc Goascoz, \emph{L'aide sociale à l'enfance de l'antiquité à nos jours}, Tikinagan, 2010.}.

Cet essai est à la disposition de tous pour un usage privé ou dans le cadre d'un enseignement. 

Usage commercial non autorisé. 

Tous droits de représentation et de reproduction réservés.

Copyright : libre de droits, mentionner l'auteur