

”\chapter{Présentation}

\emph{Ceux qui ont tout oublié n'ont pas d'avenir.}


Régis DEBRAY, 

France Culture, le 4.1.2013 à 8 h 35.






C'est au cours d'une recherche dans un autre champ que celui de l'histoire, alors que je rédigeais une thèse de doctorat portant sur les jeunes placés dans le cadre de l'Aide Sociale à l'Enfance\footnote{\emph{Jeunes cas sociaux et difficultés éducatives}, Thèse de doctorat en psychologie, directeur de thèse Hervé Beauchesne, Université de Bretagne Occidentale, 1999.} que m'est apparue l'importance de la révolution actuelle des lois et des moeurs, et l'utilité d'un retour sur le passé des familles et de la reproduction pour comprendre notre présent. 

En effet après une très longue période de relative stabilité en Europe, la reproduction des humains
est  soumise à de tels bouleversements depuis les années soixante du \siecle{20}  
qu'il ne s'agit plus d'une évolution, mais bien d'une révolution, et pour le moment il
ne semble pas que soient parvenus à leurs termes les changements en
cours dans le droit et dans les mœurs. Le présent est donc instable, déroutant et difficile à penser. C'est dans ce genre de situation qu'un point de vue extérieur est utile pour se décentrer et comprendre un peu mieux où l'on en est : un tel point de vue peut être fourni par les observations des ethnologues\footnote{Cf. les \emph{Métamorphoses de la parenté} de l'ethnologue Maurice Godelier (2004), ou bien \emph{L'origine des systèmes familiaux, T. I} d'Emmanuel Todd (2011), qui parcourt à la fois l'espace et le temps.}, mais il peut également être trouvé dans un retour sur notre passé. La situation présente de la reproduction ne prend en effet tout son sens que par ses écarts avec les pratiques des siècles antérieurs.

La famille traditionnelle occidentale est
née d'une \emph{synthèse entre les pratiques de l'Empire de Rome, celles des
juifs et celles des chrétiens de l'Antiquité}. Ces pratiques et les représentations
qui les sous-tendaient étaient elles-mêmes le point d'aboutissement
d'autant d'évolutions particulières.
Les bases juridiques de la famille « traditionnelle » européenne ont été promulguées sous le règne
de l'empereur Constantin mais elles ont mis de nombreux
siècles à s'imposer, non sans résistances ni déformations multiples par
rapport aux desseins initiaux. La trajectoire de cette forme de famille n'a atteint son apogée qu'aux
siècles « classiques » de l'Ancien Régime, par contre elle s'y est grosso-modo maintenue dans les pratiques (non sans quelques replâtrages) jusqu'au
\emph{baby-boom}.  Si ses bases juridiques ont été déconstruites à partir des années soixante du \siecle{20}, elle ne s'efface pourtant pas sans résistances, et elle n'est encore tout à fait morte ni dans les têtes ni dans les mœurs, même si dans le même temps de nouvelles formes d'union et de parentalité ont fait leur apparition et si de très anciennes problématiques que l'on croyait définitivement résolues reviennent au devant des préoccupations. 

C'est le panorama de cette très longue histoire que je me propose de déployer, en soulignant les articulations et les ruptures, les
conflits, les crises et leurs enjeux. Ce champ est si vasre et la matière à y traiter  si démesurée que s'il s'agissait d'y trouver du nouveau une nombreuse équipe serait à peine à sa mesure, et je n'ai pas la prétention de concurrencer les historiens professionnels sur leur terrain. 

Cet essai n'est donc fondé que sur les écrits des historiens, juristes, théologiens, sociologues et autres experts qui ont abordé ce champ. Mon objectif est de faire oeuvre de vulgarisation dans un domaine dont l'expérience m'a montré l'intérêt, et de fournir aux personnes intéressées les moyens de s'y repérer (et d'aller plus loin si elles le désirent). Si ce texte parvient à exposer clairement la situation où en est aujourd'hui la reproduction humaine et à la problématiser, alors j'aurai atteint mon but\footnote{L'histoire de la reproduction humaine recouvre largement celle
de la prise en charge des personnes faibles, malades, âgées, infirmes ou
démunies, la famille, quelle que soit sa composition et son organisation, ayant été jusqu'au \siecle{20} la première institution
d'assistance, quand elle n'était pas la seule. Cet aspect de l'histoire sera donc évoqué
au passage succinctement. Pour aller plus loin sur cet aspect précis on
pourra se reporter à : Hervé Tigréat, Pascale Planche et Jean-Luc Goascoz, \emph{L'aide sociale à l'enfance de l'antiquité à nos jours}, Tikinagan, 2010.}.

Cet essai est à la disposition de tous pour un usage privé ou dans le cadre d'un enseignement. 

Usage commercial non autorisé. 

Tous droits de représentation et de reproduction réservés.

Copyright : libre de droits, mentionner l'auteur


