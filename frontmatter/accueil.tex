

”\chapter{Présentation}









Depuis les années soixante du vingtième siècle la reproduction des humains est soumise à de tels bouleversements qu'il ne s'agit plus d'une évolution, mais bien d'une révolution. Pour le moment les changements en cours dans le droit et dans les mœurs ne sont pas parvenus à leurs termes et le présent est instable, déroutant et difficile à penser. Jusqu'à la génération actuelle la famille occidentale paraissait si naturelle que des points de vue extérieurs sont utiles pour se décentrer des habitudes de pensée et pour comprendre un peu mieux où l'on en est. 
De tels points de vue peuvent être fourni par les observations des ethnologues et sociologues\footnote{Cf. \emph{les Métamorphoses de la parenté} de l'ethnologue Maurice GODELIER (2004), ou bien \emph{L'origine des systèmes familiaux, T. I} (2011) ou encore \emph{Où en sommes-nous ? Une esquisse de l'histoire humaine} (2017) d'Emmanuel TODD, qui parcourt à la fois l'espace et le temps.} mais ils peuvent aussi être trouvés dans l'histoire. La situation présente de la reproduction ne prend en effet tout son sens que par ses écarts avec les pratiques des siècles antérieurs.
 
 Pour introduire mon sujet je vais commencer par me décentrer et me tourner vers les sociétés primitives. 
 Dans son \emph{Anthropologie de l’esclavage}, (1986) Quentin MEILLASSOUX montre que les membres des sociétés primitives se sentent liés par une continuité organique avec leur territoire et avec l'univers matériel dans son ensemble, avec les esprits dans (ou de) la nature, avec leurs ancêtres, avec le monde du ou des dieux. Souvent ils se pensent comme les seuls humains dignes de ce nom : chacun dans sa langue, ils se désignent eux-mêmes comme « les humains par excellence », ce qui implique que les autres, ceux qui leur sont étrangers, ne sont pas humains, ou pas vraiment humains, ou pas au même degré qu’eux. Pour eux la bonne vie, la seule vie vivable, n'est possible que sur le territoire dont ils ont hérité (même quand ils pratiquent le nomadisme c'est dans des bornes relativement limitées). Partout ailleurs c'est l'inconnu, l'étrange, l'étranger, le non humain, l'inhumain.

Le plus souvent ces sociétés ne connaissent aucune forme d'écriture. Elles se caractérisent d'abord par l'absence d'échanges marchands et de moyens de paiement, comme par la faiblesse ou l'absence de leurs structures étatiques. Elles peuvent se choisir plutôt démocratiquement un chef, mais son autorité est limitée. 

Dans ces sociétés la famille est le cadre essentiel, et parfois le cadre unique des rapports entre individus. Le chef de famille, presque toujours un homme, a pour première tâche de veiller à la pérennité de son groupe familial. La vie de chacun appartient au groupe, et il n'est pas question d'opposer à celui-ci les droits d'un individu particulier ni de mettre l'ensemble en danger pour un seul de ses membres. Si trop de naissances mettent en danger l'intérêt collectif le don des nouveaux nés excédentaires, leur abandon ou leur infanticide sont des pratiques ordinaires. En cas de disette il arrive que des vieillards se laissent mourir pour que les jeunes survivent, ou qu'on les y pousse.

La parenté assigne à chacun une fonction précise : des obligations mais aussi des droits sur les ressources du groupe. Celui-ci vit d'une économie de subsistance tournée vers l'auto-consommation. L'accumulation des biens n'est pas pensée comme la constitution d'un capital susceptible d'être réinvesti dans des opérations économiques nouvelles. Il s'agit plutôt d'acquérir des objets à haute valeur symbolique (religieuse, esthétique, magique, etc.), ou de constituer des réserves destinées à être consommées de manière festive ou/et ostentatoire. La nature lui \emph{donne} maternellement ses fruits et c'est la fécondité de son territoire qui limite la récolte et non le nombre de bras ou celui des heures de travail disponibles. Aucun membre de la famille n'est exclu des redistributions, mais les parts peuvent être très inégalement distribuées, sans tenir compte de la contribution de chaque membre du groupe au volume des biens à répartir, ni de ses besoins réels, mais plutôt de son rang et de sa place symbolique. Il est fréquent qu’à ce compte les femmes soient mal loties, mais ce n’est pas systématique. La règle de base est que les adultes travaillent pour nourrir les plus jeunes et les plus vieux. Les plus jeunes reçoivent plus qu'ils ne donnent, jusqu'à ce qu'ils soient à leur tour capables de nourrir tous ceux qui les ont nourris. Les plus vieux sont directement ou indirectement nourris par ceux qu'ils ont élevés : chacun investit dans une descendance pour préparer ses vieux jours. Quand tout se passe harmonieusement c'est au fil d'une vie entière que les tâches et les droits s'équilibrent pour chaque individu. 

Dans un tel système aucun garçon ne possède rien en propre : ni terres, ni troupeaux, etc. Si son clan refuse de lui procurer une femme, ou de lui donner les moyens d'en acquérir une, il reste bloqué dans un statut de dépendance juvénile. Condamné à travailler toute sa vie pour les enfants des autres, il n'accèdera jamais au statut avantageux et respecté de ceux qui ont de grands enfants productifs.

Ce devenir concerne moins les filles. Étant donné le taux de mortalité qui frappe les femmes enceintes, les parturientes et les enfants, les sociétés primitives ont le plus souvent besoin que chaque femme ait tous les enfants qu'elle peut porter. Les familles ont trop besoin d'enfants pour ne pas marier leurs filles dès lors qu'elles sont nubiles, éventuellement à des hommes bien plus âgés qu'elles et même déjà dotés d'autres femmes. Cette valeur qu'on leur accorde ne conduit pas à leur donner le pouvoir. Il est bien entendu qu'elles ne font pas des enfants pour elles seules, mais pour les partager ou les donner. 

Dans ces sociétés il n'y a pas de sens à faire une place à un étranger : à quel titre, au nom de quoi ? Et quel rôle lui donner ? Comment l'accueillir sans déséquilibrer le réseau compliqué et tendu des échanges et des obligations réciproques ? Du point de vue d'un guerrier l'étranger qu’il a capturé vivant est une preuve de sa valeur, mais il ne peut être un moyen d'entretenir et d'accroître sa puissance, un moyen de production de richesses (un esclave). Il n'est bon qu'à être rapidement consommé d'une façon ostentatoire : il n'est bon qu'à être sacrifié. Accepter de ses proches une rançon serait déjà entrer dans le monde marchand où une vie humaine a un coût et peut donc s'acheter ou s'échanger contre des biens réels, ce qui ne fait pas partie de leurs représentations.

Par contre s'il y a une place vacante dans une famille, celle-ci peut adopter un étranger ou une étrangère pour occuper cette place, afin que la vie continue, afin que les prestations masculines et féminines continuent d'être procurées, afin que les enfants continuent de naître, que les vieillards ne soient pas à l'abandon, que les ancêtres continuent d'être honorés, et que le monde continue sa course, etc. S'il n'y a pas assez d'épouses pour tous les garçons, on peut enlever des filles dans un autre groupe, ou leur en acheter. Un ennemi prisonnier peut d'autant plus facilement remplacer un mari ou un fils mort, que c'est ordinairement à ses voisins, à ceux que l'on pourrait épouser, qu'on fait la guerre.

En cas de conflit, de délit ou de crime, la mise au ban du groupe est d'autant plus fréquemment choisie qu'elle présente sur la mise à mort l'avantage d'éviter la souillure du territoire familial par un meurtre, ainsi que le ressentiment des ancêtres ou des dieux contre le ou les exécuteurs éventuels. Celui qui est condamné à l'exil est comme mort pour son groupe d'origine. S'il tombe aux mains d'un autre groupe, s'il est asservi (et a fortiori s'il leur était vendu) sa famille ne cherchera ni à le racheter ni à le délivrer\footnote{Ainsi le livre de la Genèse raconte comment Joseph, benjamin de Jacob, a été vendu par ses frères parce qu'ils étaient jaloux de voir qu'il était le préféré de son père. Leur première intention était de le tuer, mais comme une caravane de marchands passait par là cela leur a évité d'avoir à assumer la culpabilité de sa mort, et par dessus le marché la vente leur a rapporté de l'argent : l'Asie Mineure où se passe cette histoire était en partie entrée dans le monde marchand au moment où ce récit a été écrit.}. Si un individu qui a été banni est tué ses parents ne chercheront pas à le venger. Le jour où il mourra, les rites et sacrifices funéraires nécessaires au repos de son esprit ne seront pas exécutés. Il ne pourra pas rejoindre le monde de ses ancêtres et il ne sera pas rituellement nourri par les vivants. Son souvenir ne sera pas honoré. Cela l'exclura de son clan une deuxième fois. A ses yeux l'errance et l'exil dans un monde hostile \emph{cf. le sortde Cain après l'assassinat d'Abel} valent-ils mieux qu'une mort immédiate au milieu des siens, dans son pays ?

 Emmanuel TODD complète cette description par l'idée que l'organisation des familles primitives (l'organisation primitive des familles) est la même d'une extrémité de la terre à l'autre et se caractérise par des couples stables (même si les divorces sont possibles), de parents élevant eux-mêmes les enfants qu'ils ont conçus (même si les avortements, les infanticides et les abandons sont pratiqués à l'occasion) et respectant l'interdit de l'inceste. Selon lui le conjoint est choisi au sein du groupe de vie (au sens large) mais en dehors de la famille nucléaire. Les relations entre familles apparentées (frères et soeurs, beaux-frères et belles-soeurs) ont de l'importance étant donné le soutien mutuel qu'elles peuvent se fournir. Elles sont donc entretenues et les familles du père et de la mère ont autant d'importance l'une que l'autre. Les règles de succession sont souples et il n'y a pas de souci d'égalité stricte ni de principe de primogéniture. Il n'existe chez les primitifs aucune société réellement matriarcale mais le statut des femmes n'y est pas dévalorisé, même s'il peut y avoir à l'occasion de la polygamie. Selon TODD c'est ce type de famille qui prévalait en Europe de l'Ouest (Germanie, Gaule, Iles britanniques, Scandinavie...) avant l'entrée en scène des romains et les bouleversements de tous ordres qu'ils ont apportés, dont notamment une vision assez radicale du patriarcat. 



La famille traditionnelle occidentale a fonctionné en Europe du haut moyen-âge au milieu du XXème siècle. Elle est née d'une synthèse entre les pratiques des romains de l'Empire, celles des juifs et celles des chrétiens de l'Antiquité (ces pratiques et les représentations qui les sous-tendaient étaient elles-mêmes le point d'aboutissement d'autant d'évolutions particulières). Les bases juridiques de la famille européenne ont été promulguées sous le règne de l'empereur Constantin et celui de ses successeurs directs mais elles ont mis de nombreux siècles à s'imposer, non sans résistances ni déformations multiples par rapport aux desseins initiaux. La trajectoire de cette forme de famille n'a atteint son apogée qu'aux derniers siècles de l'Ancien Régime. Depuis lors elle a fonctionné de manière presque hégémonique, demeurant le modèle de référence jusqu'au baby-boom, en dépit de quelques modifications significatives. Si à partir des années soixante du vingtième siècle ses fondations juridiques ont été presque totalement dynamitées, elle ne s'efface pourtant pas sans résistances, et pour l'instant elle n'est tout à fait morte ni dans les têtes ni dans les comportements. Mais dans le même temps de nouvelles formes d'union et de parentalité ont fait leur apparition et de très anciennes problématiques que l'on croyait définitivement obsolètes reviennent au devant des préoccupations. 

Il n'est pas question pour moi de prétendre concurrencer les historiens professionnels sur leur terrain et cet essai est fondé sur leurs écrits.  Si ce texte parvient à exposer clairement la situation où en est aujourd'hui la reproduction humaine et à la problématiser, alors il aura atteint son but.  
 

Des sociétés primitives à celles d'aujourd’hui l'histoire de la reproduction humaine est indissociable de celle de la prise en charge des personnes faibles, malades, âgées, infirmes ou démunies. En effet la famille, quelle que soit sa composition et son organisation, a toujours été la première institution d'assistance, quand elle n'était pas la seule\footnote{Pour aller plus loin on pourra se reporter à l'\emph{Histoire des enfants, des familles et des institutions d'assistance, La protection de l'enfance de l'antiquité à nos jours}, Hervé Tigréat, Pascale Planche et Jean-Luc Goascoz, préface de Pascal David, L'Harmattan, 2018.}.





Cet essai est à la disposition de tous pour un usage privé ou dans le cadre d'un enseignement. 

Usage commercial non autorisé. 

Tous droits de représentation et de reproduction réservés.

Copyright : libre de droits, mentionner l'auteur







