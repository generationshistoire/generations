

\chapter{Présentation}

\emph{Ceux qui ont tout oublié n'ont pas d'avenir.}


\qauthor{Régis~\fsc{Debray}, France Culture le 4.1.2013 à 8 h 35



Après plus de quinze siècles de relative stabilité, en Europe du moins, la reproduction humaine
est soumise depuis les années soixante du \siecle{20} à de tels bouleversements
qu'il ne s'agit plus d'une évolution, mais d'une révolution. Il
ne semble pas que soient  parvenus à leurs termes les changements en
cours dans le droit et dans les mœurs. C'est pour cela que ne sont pas
encore pleinement reconnaissables les procédés que les individus emploieront
à l'avenir pour mettre en œuvre leurs désirs à travers le filtre
de ces changements et des institutions qu'ils généreront ou infléchiront.
Face à un présent instable, déroutant et difficile à penser, pour commencer à
comprendre où nous en sommes
 un retour sur le passé s'impose.

La situation actuelle de la reproduction ne prend en
effet tout son sens que par ses écarts avec les pratiques des siècles antérieurs.
La famille traditionnelle occidentale, celle que le droit a évincée un peu partout entre 1960 et
1980 même si elle n'est pas encore tout à fait morte dans les têtes ni dans les mœurs, est
née d'une synthèse entre les pratiques de l'Empire de Rome et celles des
juifs et des chrétiens de l'Antiquité. Ces pratiques et les représentations
qui les sous-tendaient étaient elles-mêmes le point d'aboutissement
d'autant d'évolutions particulières.

Les bases juridiques de la famille « traditionnelle » ont été promulguées sous le règne
de l'empereur Constantin mais elle a mis de nombreux
siècles à s'imposer, non sans résistances ni déformations multiples par
rapport aux desseins initiaux. Sa trajectoire a fini par atteindre son apogée aux
siècles « classiques » de l'Ancien Régime et elle s'y est maintenue jusqu'au
\emph{baby-boom}. Ses bases juridiques ont été sapées à partir des années soixante du \siecle{20}, mais elle ne s'efface pas sans résistances, en même temps que de nouvelles formes d'union et de parentalité font leur apparition.

C'est le panorama de cette longue histoire que ce travail se propose de déployer, en soulignant les articulations et les ruptures, les
conflits, les crises et leurs enjeux. Son fil directeur sera d'essayer d'éclairer la situation française actuelle.

L'histoire de la reproduction humaine recouvre largement celle
de la prise en charge des personnes faibles, malades, âgées, infirmes ou
démunies, la famille, quelle que soit sa composition et son organisation, ayant été jusqu'au \siecle{20} la principale institution
d'assistance, sinon la seule. Cet aspect de l'histoire sera donc évoqué
au passage succinctement. Pour aller plus loin on
pourra se reporter à \emph{L'aide sociale à l'enfance de l'antiquité à
nos jours}, Hervé \fsc{Tigréat}, Pascale \fsc{Planche}, Jean-Luc \fsc{Goascoz},
préface de Pascal \fsc{David}, Tikinagan, 2010.

Cet ouvrage est à la disposition de tous pour un usage privé ou dans le cadre d'un enseignement. Usage commercial non autorisé. Tous droits de représentation et de reproduction réservés.

Copyright : libre de droits, mentionner l'auteur


